
\documentclass[12pt]{ctexbook}
\usepackage{unicode-math}
\usepackage[top=10mm,left=2mm,bottom=0mm,right=2mm,headsep=2mm,paperwidth=128mm,paperheight=170mm]{geometry}
\setlength{\headheight}{14pt}
\usepackage{fancyhdr}
\pagestyle{fancy}
\fancyhf{}
\renewcommand{\headrulewidth}{0pt}
\renewcommand{\chaptermark}[1]{\markboth{心體與性體 \quad 第一冊}{}}
\renewcommand{\sectionmark}[1]{\markboth{心體與性體 \quad 第一冊}{}}
\renewcommand{\subsectionmark}[1]{\markboth{心體與性體 \quad 第一冊}{}}
\ctexset{
  contentsname = 目次,
  appendixname = 附錄,
  appendix/number = {},
  part = {
    name = {第,部},
    number = \chinese{part},
    aftername = {\quad},
    format = {\Huge\kaishu\centering},
    pagestyle = plain
  },
  chapter = {
    name = {第,章},
    number = \chinese{chapter},
    aftername = {\quad},
    format = {\LARGE\kaishu},
    pagestyle = fancy,
    beforeskip = 40pt,
    afterskip = 30pt
  },
  section = {
    name = {第,節},
    number = \chinese{section},
    aftername = {\quad},
    format = {\large\kaishu},
  },
  subsection = {
    name = {第,段},
    number = \chinese{subsection},
    aftername = {\quad},
    format = {\kaishu},
  }
}
\renewenvironment{quotation} {\list{}{
    \listparindent 0mm
    \itemindent \listparindent
    \rightmargin 0mm
    \parsep 1mm
  } \item\relax} {\endlist}
\usepackage{graphicx}

\begin{document}

\frontmatter

\begin{titlepage}
  \begin{flushright}
    \vspace*{\fill}
    {\heiti 牟宗三先生全集\textcircled{5}}

    \medskip

    {\Huge\songti 心體與性體}

    \smallskip

    {\kaishu (第一冊)}

    \smallskip

    {\songti 牟宗三 \quad 著}
    \vspace*{\fill}
  \end{flushright}
\end{titlepage}

\pagenumbering{arabic}
\fancyhead[LE]{\small (\thepage )\ $\odot$\ \kaishu\leftmark}
\fancyhead[RO]{\small \kaishu\rightmark\ $\odot$\ (\thepage )}


\newpage\markright{}

\chapter{《心體與性體》全集本編校說明}
\begin{flushright}蔡仁厚、林月惠\end{flushright}

《心體與性體》共分三冊,起草於1961年先生任教香港大學之時,至1968年5月,由台北正中書局出版第一冊,10月出版第二冊,次年6月出版第三冊。十年後由台灣學生書局出版的〈從陸象山到劉蕺山》可視為此書之第四冊。

牟先生在正式撰寫《心體與性體》之前,已有《宋明儒學綜述〉之講辭。該講辭係他於1963年為香港大學校外課程部講授「宋明儒學」課程時所講,最初計畫分為十二講。前六講由王煜記錄、整理,經牟先生校訂後,在香港《人生》雜誌連載,由第25卷第12期至第26卷第4期(1963年5月1日至7月1日)分五期刊出。據文前的(小序〉可知,牟先生本擬以此講辭作為《心體與性體〉之引論。

在發表了前六講之後,牟先生又親自撰成〈寂感真幾之生化之理與「道德的形上學」之完成〉一文,刊載於《民主評論》第14卷第16期(1963年8月20日)。據牟先生在文前之說明,此文應為《宋明儒學綜述》之第七、八講,其內容原定為六節,篇目如下:

第一節鑑對於前講之回顧

\newpage\thispagestyle{empty}\addtocounter{page}{-1}\vspace*{-12mm}\begin{center}\noindent
\includegraphics[clip, trim=149pt 128pt 153pt 266pt, height=162mm]{ocr-input/image-0008.png}\end{center}

\newpage

第二節鑑實現之理與形成之理之區別

第三節鑑實現之理與科學歸納所得之理之區別

第四節鑑道德性的實理天理與實然自然之契合

第五節康德所以只有「道德的神學」而無「道德的形上學」
之故

第六節 「道德的形上學」之完成

但牟先生此文只寫到第三節,其中第二、三節之思路相當於《心體與性體》第一部〈綜論〉第二章第三、四節之思路。至於此文之後三節,則分別獨立成文,亦刊於《民主評論》。其篇目及出處如下:

〈道德性的實理、天理與實然自然之契合——兼論道德理性三
義〉第14卷第17期1963年9月5日

〈康德所以只有「道德的神學」而無「道德的形上學」之故〉

第14卷第18/19期1963年9月20日/10月5日

〈「道德的形上學」之完成〉第14卷第20期1963年10月20日

這三篇論文之標題與內容與《心體與性體》第一冊第一部〈綜論〉第三章之三節幾乎完全相同,惟第三節之標題改為〈論道德理性三義〉。此書第一、二、三冊有關濂溪橫渠、明道、伊川、胡五峰與朱子的思想,在其1988年所發表的《宋明理學演講錄》中也有扼要的講解。

本書之編校工作以最新版(第一冊1991年,第二冊1996年,第三冊1995年)為準,凡有訛誤之處,均直接修改,不另註明。但初版不誤而後版有誤者,則以初版為準,共有二處:

第一冊第470頁第9-10行:「知太虛即氣,則無,何無」當作

\newpage\thispagestyle{empty}\addtocounter{page}{-1}\vspace*{-12mm}\begin{center}\noindent
\includegraphics[clip, trim=190pt 148pt 119pt 247pt, height=162mm]{ocr-input/image-0012.png}\end{center}

\newpage\markright{〈心體與性體〉全集本編校說明}

\noindent 「知太虛即氣,則無無。」

第一冊第514頁第6行:「『之』字皆代萬表道。」當作「『之』字皆代表道。」

本書之編校工作,第一冊由蔡仁厚負責,第二、三冊由林月惠負責。

\newpage\thispagestyle{empty}\addtocounter{page}{-1}\vspace*{-12mm}\begin{center}\noindent
\includegraphics[clip, trim=173pt 433pt 134pt 88pt, height=162mm]{ocr-input/image-0016.png}\end{center}

\newpage\thispagestyle{empty}

\newpage\markright{}

\chapter{序}

王龍溪有言:悟道有解悟,有證悟,有澈悟。今且未及言悟道,姑就宋、明六百年中彼體道諸大儒所留之語言文字視作一期學術先客觀了解之,亦是欲窺此學者之一助。

了解有感性之了解,有知性之了解,有理性之了解。彷彿一二,望文生義,曰感性之了解。意義釐清而確定之,曰知性之了解。會而通之,得其系統之原委,曰理性之了解。

荀子曰:「倫類不通,仁義不一,不足謂善學。學也者固學一之也。」又曰:「全之盡之,然後學者也。君子知夫不全不粹之不足以為美也,故誦數以貫之,思索以通之,為其人以處之。」「全之盡之」即通過知性之了解而至理性之了解也。

予以頑鈍之資,恍惚搖蕩困惑於此學之中者有年矣。五十以前,未專力於此,猶可說也;五十而後,漸為諸生講說此學,而困惑滋甚,寢食難安。自念若未能了然於心,誠無以對諸生,無以對先賢,亦無以對此期之學術也。乃發憤誦數,撰成此書,亦八年來之心血也。或於語意之釐清與系統之確定稍盡力焉,然究能至「全之盡之」否,亦未敢必也。

前賢對於人物之品題輒有高致,而對於義理系統之確解與評

\newpage\thispagestyle{empty}\addtocounter{page}{-1}\vspace*{-12mm}\begin{center}\noindent
\includegraphics[clip, trim=161pt 128pt 135pt 261pt, height=162mm]{ocr-input/image-0020.png}\end{center}

\newpage

\noindent 鑑,則稍感不足。此固非前賢之所重視,然處於今日,則將為初學之要務,未可忽也。

理性之了解亦非只客觀了解而已,要能融納於生命中方為真實,且亦須有相應之生命為其基點,否則未有能通解古人之語意而得其原委者也。

莊生有云:「聖人懷之,衆人辯之以相示也。」吾所作者亦只辯示而已。過此以往,則期乎各人之默成。吾未敢云有若何自得處,願與天下之善士共勉之,此非筆舌所可宜也。

凡吾所欲言者俱見於(綜論部〉,茲略贅數語以為序。

\newpage\thispagestyle{empty}\addtocounter{page}{-1}\vspace*{-12mm}\begin{center}\noindent
\includegraphics[clip, trim=165pt 406pt 132pt 103pt, height=162mm]{ocr-input/image-0024.png}\end{center}

\newpage\markright{}

\tableofcontents

\mainmatter
\fancyhead[LE]{\small \thepage\ $\odot$\ \kaishu\leftmark}
\fancyhead[RO]{\small \kaishu\rightmark\ $\odot$\ \thepage}

\newpage

\markright{}

\part{綜論}

\newpage\markright{}

\chapter{宋明儒學之課題}

\section{正名:宋明儒學之定位}

宋明六百年之儒學通常亦名「宋明理學」。「理學」之「理」字固有實指,但人可就表面只想其通泛之意義。平常有詞章、義理、考據之分,就《易經》言,有象數、義理之別。若如此使用「義理」,則「義理」一詞便甚通泛,其意當是普通所說之「道理」或「理論」,或如今日所說之廣義之「哲學」。若依此意義之「義理」想宋明理學之「理」字,則太通泛,不能標明其特質,亦不能使人知其與先秦儒家之關係。是以若用「理」字去想宋明儒學之所講,則須有簡別。

先秦典籍未有依「理」之不同劃分學問者,開始作此區分者是漢末魏初之劉劭。其《人物志·材理》篇第四云:

\begin{quotation}\kaishu 夫理有四部,明有四家。[……】若夫天地氣化,盈虛損
益,道之理也。法制正事,事之理也。禮教宜適,義之理
也。人情樞機,情之理也。四理不同,其於才也,須明而\end{quotation}

\newpage\thispagestyle{empty}\addtocounter{page}{-1}\vspace*{-12mm}\begin{center}\noindent
\includegraphics[clip, trim=166pt 150pt 127pt 241pt, height=162mm]{ocr-input/image-0055.png}\end{center}

\newpage

\begin{quotation}\kaishu 章。明待質而行。是故質與理合,合而有明。明足見理,理
足成家。是故質性平淡,思心玄微,能通自然,道理之家
也。質性警徹,權略機捷,能理煩速,事理之家也。質性和
平,能論禮教,辨其得失,義理之家也。質性機解,推情原
意,能適其變,情理之家也。\end{quotation}

\noindent 據此,則理分四部,即道理、事理、義理、情理。其所謂「道理」即天道之理,其所謂「道理之家」,其心目中大體是指「道家者流」而言,實則儒家亦講「天道」。其所謂「事理」是就政治制度與政治措施兩面而言;其所謂「事理之家」是就政治家以及有處事之才之人而言。若就依理成學而言,此當屬於政治哲學(包括人法而言),此是依橫的與靜的觀點說「事理」;若復依縱的與動的觀點看事理,則當屬於「歷史哲學」。其所謂「義理」是指禮樂教化而言,此是屬於道德的,非通泛之義理;其所謂「義理之家」大抵是指「儒家者流」而言。其所謂「情理」與「情理之家」大體可包括於「事理」與「事理之家」中。「事理」是政治性歷史性的,而「情理」則比較偏於社會性的。明「情理」者雖不必能進而為「事理之家」,然「事理之家」必通「情理」。

依以上四理之分,宋、明儒所講者當是兼攝「道理」與「義理」兩者而一之之學。「道理」是儒家所講的天道、天命之理。.「義理」是自覺地作道德實踐時所見的內在的當然之理,亦不只是如劉劭所說之「禮教宜適」之只為外部的。

但此四理之分當然不能盡此「理」字之全部意義。友人唐君毅先生依中國思想史之發展,分理為六義:

\newpage\thispagestyle{empty}\addtocounter{page}{-1}\vspace*{-12mm}\begin{center}\noindent
\includegraphics[clip, trim=144pt 143pt 145pt 254pt, height=162mm]{ocr-input/image-0059.png}\end{center}

\newpage\markright{第一部 \quad 第一章 \quad 宋明儒學之課題}

\begin{quotation}\kaishu 一是文理之理,此大體是先秦思想家所重之理。二是名理之
理,此亦可指魏晉玄學中所重之玄理。三是空理之理,此可
指隋、唐佛學家所重之理。四是性理之理,此是宋、明理學
家所重之理。五是事理之理,此是王船山以至清代一般儒者
所重之理。六是物理之理,此為現代中國人受西方思想影響
後特重之理。此六種理,同可在先秦經籍中所謂理之涵義中
得其淵源。如以今語言之,文理之理乃人倫人文之理,即人
與人相互活動或相互表現其精神而合成之社會或客觀精神中
之理。名理玄理之理是由思想名言所顯之意理,而或通於哲
學之本體論上之理者。空理之理是一種由思想言說以超思想
言說所顯之理。性理之理是人生行為之內在的當然之理而有
形而上之意義並通於天理者。事理之理是歷史事件之理。物
理之理是作為客觀對象看的存在事物之理。((中國哲學原
論》第一章(原理〉上,一、導言)\end{quotation}

\noindent 唐先生所說之「理之六義」實可駭攝中國思想史中「理」字之全部意義。但若以學門觀之,先秦所說之「文理」之理很難劃在一門學問内,其意蓋甚通泛。故若從「理」字之意義上想,有此一義,但若從學門觀之,則不知其當何所屬。是以理之諸義,若以學門範域之,吾意當重列如下:

1.名理——此屬於邏輯,廣之,亦可該括數學。

2.物理——此屬於經驗科學,自然的或社會的。

3.玄理——此屬于道家。

4.空理——此屬于佛家。

\newpage\thispagestyle{empty}\addtocounter{page}{-1}\vspace*{-12mm}\begin{center}\noindent
\includegraphics[clip, trim=159pt 138pt 134pt 246pt, height=162mm]{ocr-input/image-0063.png}\end{center}

\newpage

5.性理——此屬於儒家。

6.事理(亦攝情理)——此屬於政治哲學與歷史哲學。

依是,3、4、5三者當屬於道德宗教者。宋、明儒所講者即「性理之學」也。此亦道德亦宗教,即道德即宗教,道德宗教通而一之者也。

此「性理之學」亦可直曰「心性之學」。蓋宋、明儒講學之中點與重點唯是落在道德的本心與道德創造之性能(道德實踐所以可能之先天根據)上。「性理」一詞並非性底理,乃是即性即理。若只說「性理之學」,人可只以伊川、朱子所說之「性即理也」之「性理」義去想,此則便不周遍,不能概括「本心即性」之「性理」義。當吾人說「性理之學」時,此中「性理」一詞,其義蘊並不專限於伊川、朱子所說之「性即理」之義,故亦不等於其所說之「性即理」之「性理」義,乃亦包括「本心即性」之「性理」義。依此之故,直曰「心性之學」,或許更較恰當。說心性,人易想到「空談心性」。實則欲自覺地作道德實踐,心性不能不談。念茲在茲而講習之,不能說是空談。空談者自是空談,不能因此而影響此學之本質。

此「心性之學」亦曰「內聖之學」。「內聖」者,內而在於個人自己,則自覺地作聖賢工夫(作道德實踐)以發展完成其德性人格之謂也。「內聖外王」一語雖出於《莊子·天下》篇,然以之表象儒家之心願實最為恰當。「外王」者,外而達於天下,則行王者之道也。王者之道,言非霸道,此一面足見儒家之政治思想。宋、明儒所講習者特重在「內聖」一面,「內聖」一面在先秦儒家本已彰顯而成定型,因而亦早已得其永恆之意義。此本屬於孟子所謂:

\newpage\thispagestyle{empty}\addtocounter{page}{-1}\vspace*{-12mm}\begin{center}\noindent
\includegraphics[clip, trim=171pt 128pt 134pt 260pt, height=162mm]{ocr-input/image-0067.png}\end{center}

\newpage\markright{第一部 \quad 第一章 \quad 宋明儒學之課題}

\noindent 「求則得之,舍則失之,是求有益於得也,是求之在我者也」。此「求之在我者」實是儒家之最內在的本質。經過宋、明儒六百年之弘揚與講習,益達完整而充其極之境。本來即此一面亦可使儒家自足獨立。與政治劃開,如普通宗教然,亦未嘗不可。此或更可使儒家不與政治糾纏於一起,不隨時代為浮沈,而只以個人之成德為人類開光明之門,以保持其永恆獨立之意義。然儒家究與一般宗教不同。其道德的心願究不能與政治劃開,完全退縮於以個人之成德為滿足。就個人言,「外王」一面雖屬孟子所謂「求之有道,得之有命,是求無益於得也,是求之在外者也」,然政治意識之方向究亦為儒家本質之一面,此與個人之能得不能得、能作不能作,並無關。不因我不能得、不能作,即可不過問,而認為與我不相干。但此一面在先秦儒家即未達定型之境,只有一大體之傾向,只順現實歷史稱贊堯、舜三代。但堯、舜三代究不如內聖面之完整與清晰。內聖面可即得其完整而永恆之意義,而外王面之堯、舜三代卻並不能即代表政治型態之完整而永恆之意義·是以儒家之政治思想尚只在朦朧之發展中,宋、明儒對此亦貢獻甚少,只以堯、舜三代寄託其外王之理想。以堯、舜三代為外王之定型,此即其政治思想不如內聖面之完整與清晰之故也。對於內聖面有積極之講習與浸潤,而對於外王面則並無積極之討論。彼等或以為只正心誠意即可直接達之治國平天下,實則政治問題不如此之簡單,只一「家天下」便非只是道德的正心誠意所能解決。不滿意於宋、明儒之只講內聖之學而要求事功者,皆是屬於「外王」面之問題。然要求事功者皆只成第二義寡頭的事功,而不知其第一義之只為政治型態問題也。黃梨洲、王船山已知之矣,而猶不得其解決之道,此可見儒家外王一面

\newpage\thispagestyle{empty}\addtocounter{page}{-1}\vspace*{-12mm}\begin{center}\noindent
\includegraphics[clip, trim=146pt 140pt 143pt 247pt, height=162mm]{ocr-input/image-0071.png}\end{center}

\newpage

\noindent 之困難以及其思想之尚在發展中。凡此吾已論之於《政道與治道》,下第五章論葉水心處亦將有詳言。然無論如何,宋、明儒之弘揚內聖一面並無過。衡之「內聖外王之道」之全體,謂其不足可,詬詆而反對之則不可。實則要求事功者皆未得其門而入,其成就遠不如弘揚「內聖之學」者成就之大。

此「內聖之學」亦曰「成德之教」。「成德」之最高目標是聖、是仁者、是大人,而其真實意義則在於個人有限之生命中取得一無限而圓滿之意義。此則即道德即宗教,而為人類建立一「道德的宗教」也。此則與佛教之以捨離為中心的滅度宗教不同,亦與基督教之以神為中心的救贖宗教不同。在儒家,道德不是停在有限的範圍內,不是如西方者然以道德與宗教為對立之兩階段。道德即通無限,道德行為有限,而道德行為所依據之實體以成其為道德行為者則無限。人而隨時隨處體現此實體以成其道德行為之「純亦不已」,則其個人生命雖有限,其道德行為亦有限,然而有限即無限,此即其宗教境界。體現實體以成德(所謂盡心或盡性),此成德之過程是無窮無盡的。要說不圓滿,永遠不圓滿,無人敢以聖自居;然而要說圓滿,則當體即圓滿,聖亦隨時可至。要說解脫,此即是解脫;要說得救,此即是得救。要說信仰,此即是信仰,此是內信內仰,而非外信外仰以假祈禱以賴救恩者也。聖不聖且無所謂,要者是在自覺地作道德實踐,本其本心性體澈底清澈其生命。此將是一無窮無盡之工作,一切道德宗教性之奧義盡在其中,一切關於內聖之學之義理盡由此展開。

此「成德之教」本非是宋、明儒無中生有之誇大,乃是先秦儒者已有之弘規。孔子即教人作「仁者」,而亦不輕易以「仁」許

\newpage\thispagestyle{empty}\addtocounter{page}{-1}\vspace*{-12mm}\begin{center}\noindent
\includegraphics[clip, trim=183pt 132pt 121pt 255pt, height=162mm]{ocr-input/image-0075.png}\end{center}

\newpage\markright{第一部 \quad 第一章 \quad 宋明儒學之課題}

\noindent 人,其本人亦說:「若聖與仁,則吾豈敢?」然而其「教不倦、學不厭」即是「仁且智」。是以其践仁以知天即是「成德之教」之弘規。《中庸》說:「肫肫其仁,淵淵其淵,浩浩其天」,即是就此弘規而說,亦是對於聖人生命之「上達天德」之最恰當的體會。

曾子守約慎獨是真能自覺地作道德實踐者,「士不可以不弘毅,任重而道遠。仁以為己任,不亦重乎?死而後已,不亦遠乎?」此亦是真切于「成德之教」之精神者。

孟子說「士尚志」。又說「得其大者為大人,得其小者為小人」。又說「君子所性,雖大行不加焉,雖窮居不損焉,分定故也。」又說「君子所性,仁義禮智根於心,其生色也,醉然見於面,盎於背,施於四體,四體不言而喻」。其「盡心知性知天」、「存心養性事天」「夭壽不貳,修身以俟之,所以立命」即是成德之教之全部展開。象山說:「夫子以仁發明斯道,其言渾無罅縫。孟子十字打開,更無隱遁。」所謂「十字打開」者即是將此「成德之教」之弘規全部展開也。

《荀子·勸學篇》亦曰「學惡乎始?惡乎終?曰:其數,則始乎誦經,終乎讀禮。其義,則始乎為士,終乎為聖人。」不管其言心性非儒者之正宗,然而「成德之教」則仍自若。

《易·乾文言》則曰「夫大人者,與天地合其德,與日月合其明,與四時合其序,與鬼神合其吉凶。先天而天弗違,後天而奉天時。天且弗違,而況于人乎?況于鬼神乎?」此即「成德之教」之極致。〈坤文言〉亦說「直其正也,方其義也。君子敬以直內,義以方外。敬義立而德不孤。直方大,不習无不利,則不疑其所行也。

\newpage\thispagestyle{empty}\addtocounter{page}{-1}\vspace*{-12mm}\begin{center}\noindent
\includegraphics[clip, trim=160pt 146pt 145pt 248pt, height=162mm]{ocr-input/image-0079.png}\end{center}

\newpage

宋、明儒所弘揚者無能越此「成德之教」之弘規。

此「成德之教」,就其為學說,以今語言之,亦可說即是一「道德哲學」(moral philosophy)。進一步,此道德哲學亦函一「道德的形上學」(moral metaphysics)。道德哲學意即討論道德的哲學,或道德之哲學的討論,故亦可轉語為「道德底哲學」(philosophy of morals)。人對於哲學的態度不一,哲學的思考活動(釐清活動)亦可到處應用,故「道德底哲學」其系統亦多端,其所處理之問題亦可有多方面。但自宋、明儒觀之,就道德論道德,其中心問題首在討論道德實踐所以可能之先驗根據(或超越的根據),此即心性問題是也。由此進而復討論實踐之下手問題,此即工夫入路問題是也。前者是道德實踐所以可能之客觀根據,後者是道德實踐所以可能之主觀根據,宋、明儒心性之學之全部即是此兩問題。以宋、明儒詞語說,前者是本體問題,後者是工夫問題。就前者說,此一「道德底哲學」相當於康德所講的「道德底形上學」,即其《道德底形上學之基本原則》(FundamentalPrinciples of the Metaphysics of Morals)一書是也·康德此書並未涉及工夫問題。此蓋由於西哲對此學常只視為一純哲學之問題,而不知其復亦是實踐問題也。然而宋、明儒之講此學則是由「成德之教」而來,故如當作「道德底哲學」而言之,亦當本體與工夫兩面兼顧始完備。而且他們首先所註意者勿寧是工夫問題,至於本體問題則是由自覺地作道德實踐而反省澈至者,澈至之以成全其道德實踐者。

由「成德之教」而來的「道德底哲學」必含本體與工夫之兩面,而且在實踐中有限即通無限,故其在本體一面所反省澈至之本

\newpage\thispagestyle{empty}\addtocounter{page}{-1}\vspace*{-12mm}\begin{center}\noindent
\includegraphics[clip, trim=172pt 130pt 120pt 249pt, height=162mm]{ocr-input/image-0083.png}\end{center}

\newpage\markright{第一部 \quad 第一章 \quad 宋明儒學之課題}

\noindent 體,即本心性體,必須是絕對的普遍者,是所謂「體物而不可遺」之無外者,頓時即須普而為「妙萬物而為言」者,不但只是吾人道德實踐之本體(根據),且亦須是宇宙生化之本體,一切存在之本體(根據)。此是由仁心之無外而說者,因而亦是「仁心無外」所必然函其是如此者。不但只是「仁心無外」之理上如此,而且由「肫肫其仁,淵淵其淵,浩浩其天」之聖證之示範亦可驗其如此。由此一步澈至與驗證,此一「道德底哲學」即函一「道德的形上學」。此與「道德之(底)形上學」並不相同:此後者重點在道德,即重在說明道德之先驗本性;而前者重點則在形上學,乃涉及一切存在而為言者,故應含有一些「本體論的陳述」與「宇宙論的陳述」,或綜曰「本體宇宙論的陳述」(onto-cosmologicalstatements),此是由道德實踐中之澈至與聖證而成者,非如西方希腊傳統所傳的空頭的或純知解的形上學之純為外在者然,故此曰「道德的形上學」,意即由道德的進路來接近形上學,或形上學之由道德的進路而證成者,此是相應「道德的宗教」而成者。

康德建立起一個「道德的神學」(moral theology),但並無「道德的形上學」一詞;但雖無此詞,卻並非無此學之實。他由意志之自由自律來接近「物自身」(thing in itself),並由美學判斷來溝通道德界與自然界(存在界)。吾人以為此一套規劃即是一「道德的形上學」之內容。但他只成立一個「道德的神學」,卻並未成立一個「道德的形上學」。當然名之有無不算重要,如果真有此學之實而真能作得出,則即實有一「道德的形上學」;但吾人以為他所規劃的屬於「道德的形上學」之一套卻並未能充分作得成。意志之自由自律是道德實踐所以可能之先天根據(本體),此不

\newpage\thispagestyle{empty}\addtocounter{page}{-1}\vspace*{-12mm}\begin{center}\noindent
\includegraphics[clip, trim=151pt 130pt 149pt 260pt, height=162mm]{ocr-input/image-0087.png}\end{center}

\newpage

\noindent 錯;但此本體能達其「無外」之絕對普遍性否,此則康德並無明確之態度。「物自身」一概念是就一切存在而言,並不專限於人類或一切有理性的存在;但自由自律之意志能普遍地相應此概念否,此則康德亦無明確之態度。又,以美學判斷來溝通道德界與自然界,此並非一康莊之大道,此只是一旁蹊曲徑,作為一輔助的指點可,作為一擔綱則不可。康德走上旁蹊曲徑,故兩界合一問題實未能得到充分之解決,此本是由依據道德實踐中所證的絕對普遍之實體而來的稱體起用之問題。康德不從此處著眼,卻由輔助的指點處著眼,此其所以不能充分解決之故。此處走上旁蹊曲徑之途,則其對於前兩點無明確之態度蓋亦甚顯。此三點綜起來即表示康德所規劃的屬於「道德的形上學」之一套並未能充分作得成,此亦是其所以不能積極地意識到一個「道德的形上學」之故。但只順其宗教傳統而意識到一個「道德的神學」,但卻又有此一套屬於「道德的形上學」之規劃!

在此,立即顯出一個問題,即此兩套規劃能免於床上架床之重疊否?能終於維持其為兩套否?如果「道德的形上學」能充分作得成,「道德的神學」還有必要否?還有其獨立的意義否?我看只有一套,並無兩套。如真維持其「道德的神學」,則「道德的形上學」即應取消。如果「道德的形上學」真能充分作得成,則此形上學即是神學,兩者合一,仍只是一套,並無兩套。康德後的發展即向此而趨;而宋、明儒者卻正是將此「道德的形上學」充分地作得出者。故在宋、明儒,此「道德的形上學」即是其「成德之教」下相應其「道德的宗教」之「道德的神學」。此兩者是一,除此「道德的形上學」外,並無另一套「道德的神學」之可言。在此,宋、

\newpage\thispagestyle{empty}\addtocounter{page}{-1}\vspace*{-12mm}\begin{center}\noindent
\includegraphics[clip, trim=177pt 118pt 123pt 263pt, height=162mm]{ocr-input/image-0091.png}\end{center}

\newpage\markright{第一部 \quad 第一章 \quad 宋明儒學之課題}

\noindent 明儒者依據先秦儒家「成德之教」之弘規所弘揚之「心性之學」實超過康德而比康德為圓熟。但吾人亦同樣可依據康德之意志自由、物自身、以及道德界與自然界之合一,而規定出一個「道德的形上學」,而說宋、明儒之「心性之學」若用今語言之,其為「道德哲學」正函一「道德的形上學」之充分完成,使宋、明儒六百年所講者有一今語學術上更為清楚而確定之定位。

\section{所謂「新儒學」:新之所以為新之意義}

宋、明心性之學,西方學者一般亦稱之為「新儒學」(neo-Confucianism)。中國以前並無此名,儒學即儒學耳,何「新」之有?宋、明儒者亦不以為其所講者是「新儒學」,彼等以為其所講者皆是聖人原有之義,(彼等以聖人代表其所講習之儒家經典之全部),皆是聖教本有之舊;民國以來,中國人之習慣亦不用此名,惟最近順西人之習慣亦常沿用之。此名亦有其新鮮恰當處,且可避免就内容起名之麻煩,只是一儒家之思想加一「新」字而已;且可表示思想之發展,免得像以前之渾淪而為一。

但是新之所以為新究何在,則頗難說,亦未見有人能作確定之規定。若只因時代之不同而為新,則無意義;若謂其因雜有佛老而爲新,則是虛妄;若謂其與先秦儒家總有相當之距離,即此即可說為新,不管其距離如何講,此則太空洞。依是,新之所以為新實有待於詳細確定也。本節試依以下之線索作一說明。

1.《韓非子·顯學》篇云:「自孔子之死也,有子張之儒,有子思之儒,有顏氏之儒,有孟氏之儒,有漆雕氏之儒,有仲良氏之

\newpage\thispagestyle{empty}\addtocounter{page}{-1}\vspace*{-12mm}\begin{center}\noindent
\includegraphics[clip, trim=152pt 143pt 151pt 248pt, height=162mm]{ocr-input/image-0095.png}\end{center}

\newpage

\noindent 儒,有孫氏之儒,有樂正氏之儒。」是則自孔子沒,「儒分為八」,見仁見智,各有所得。此一龐大之集團究誰能代表儒家之真?韓非所舉,在今日有許多已無文獻可徵,如顏氏、漆雕氏、仲良氏(仲梁子)樂正氏便是。自今日觀之,孔子後有二百年之發展,有孟子,有荀子,亦有不能確知作者之名之作品,如《中庸》,如《易傳》,如《大學》。時時在新中,究誰能代表正宗之儒家?究誰是儒家之本質?孟子固赫然之大家,然荀子又非之。在先秦,大家齊頭並列,吾人只知其皆宗孔氏,然並無一確定傳法之統系。吾人如不能單以孔子個人為儒家,亦不能孤懸孔子於隔絕之境,復亦不便如西方哲學史然只以分別地論各個人之思想為已足,則孔子之生命與智慧必有其前後相呼應,足以決定一基本之方向,以代表儒家之本質。此點可得而確定否?如能確定,則於了解儒家之本質,孔子生命智慧之基本方向,必大有助益;如不能確定,則必只是一團混雜,難有清晰之眉目。

2.兩漢以傳經為儒。對於孔子之真生命以及其所立之教之本質亦未能有所確定。司馬談云:「夫儒者以六藝為法。六藝經傳以千萬數,累世不能通其學,當年不能究其禮。故曰博而寡要,勞而少功。若夫列君臣父子之禮,序夫婦長幼之別,雖百家弗能易也。」(〈論六家要旨〉)。此觀點大體支配兩漢四百年,亦是一般所易接近之觀點,此可曰通俗之觀點。魏、晉以後,則是以王弼之「聖人體無」、向、郭之跡本論為主流,此則對於儒聖之體會已超過兩漢之經生,然此是當時會通孔、老之說,或不能盡儒聖之實,故一般習儒業者仍是以傳經為儒也。夫「以六藝為法」非必無是處,孔子即習六藝,亦傳經。然六藝是孔子以前之經典,(《春秋》稍不

\newpage\thispagestyle{empty}\addtocounter{page}{-1}\vspace*{-12mm}\begin{center}\noindent
\includegraphics[clip, trim=182pt 124pt 119pt 258pt, height=162mm]{ocr-input/image-0099.png}\end{center}

\newpage\markright{第一部 \quad 第一章 \quad 宋明儒學之課題}

\noindent 同),傳經以教是一事,孔子之獨特生命又是一事。只習六藝不必真能了解孔子之獨特生命也。以習六藝傳經為儒,是從孔子繞出去,以古經典為標準,不以孔子生命智慧之基本方向為標準,孔子亦只是一媒介人物而已。故累世當年窮究六藝,而對於孔子之所以為孔子反不了解。此真荀子所謂「雜而無統」者也。傳經亦非無價值,然就儒家論儒家,則不能盡儒家之本質。又,「列君臣父子之禮,序夫婦長幼之別」,此固「雖百家弗能易」,亦當為任何儒者所共許。「儒分為八」,雖見仁見智各有不同,然禮樂人倫或仁義教化則當為儒者所共執。人或可謂此即是儒家之本質。然此義太松泛,既是「雖百家弗能易」,則知徒如此說,不足以盡孔子獨特生命智慧之實。是以傳經只可為了解孔子之助緣,了解其生命智慧之歷史文化的背景,而禮樂人倫、仁義教化,則只是孔子抒發其生命智慧之底據,固不足以盡其生命智慧之本質也。孔子所說之「仁」決不只是普通所說仁義教化之仁也。言仁義者多矣,豈真能皆合孔子之實意乎?是以儒之所以為儒必須有進一步之規定,決不能認為止於禮樂人倫、仁義教化為已足。必須由外部通俗的觀點進而至於內在本質的觀點方能見儒家生命智慧之方向。

如果宋、明儒所講者可稱為新儒學,則其新之所以為新首先即是對上述兩點而為新。1.對先秦之龐雜集團、齊頭並列,並無一確定之傳法統系,而確定出一個統系,藉以決定儒家生命智慧之基本方向,因而為新。他們對於孔子生命智慧前後相呼應之傳承有一確定之認識,並確定出傳承之正宗,決定出儒家之本質。他們以曾子、子思、孟子及《中庸》、《易傳》與《大學》為足以代表儒家傳承之正宗,為儒家教義發展之本質,而荀子不與焉,子夏傳經亦

\newpage\thispagestyle{empty}\addtocounter{page}{-1}\vspace*{-12mm}\begin{center}\noindent
\includegraphics[clip, trim=164pt 137pt 142pt 255pt, height=162mm]{ocr-input/image-0103.png}\end{center}

\newpage

\noindent 不與焉。2.對漢人以傳經為儒而為新,此則直接以孔子為標準,直就孔子之生命智慧之方向而言成德之教以為儒學,或直相應孔孟之生命智慧而以自覺地作道德實踐以清澈自己之生命,以發展其德性人格,為儒學。宋以前是周孔並稱,宋以後是孔、孟並稱。周、孔並稱,孔子只是堯、舜、禹、湯、文、武、周公之驥尾,對後來言,只是傳經之媒介,此只是外部看孔子,孔子並未得其應得之地位,其獨特之生命智慧並未凸現出。但孔、孟並稱,則是以孔子為教主,孔子之所以為孔子始正式被認識。故二程品題聖賢氣象唯是以孔、顏、孟為主。王充《論衡·超奇》篇云:「孔子得史記以作《春秋》。及其立義創意,褒貶賞誅,不復因史記者,眇思自出於胸中也。」。「立義創意」,「眇思自出於胸中」,即是孔子之獨特的生命與智慧。若徒習《魯史》,則墨子能讀百國春秋,亦傳經也。然而墨子無此生命與智慧,至少未表現出如此方向之生命與智慧。又如冠禮為成人之禮,《禮記·冠義》云:「成人之者,將責成人禮焉也。責成人禮焉者,將責為人子、為人弟、為人臣、為人少者之禮行焉。將責四者之行於人,其禮可不重與?」冠禮是由兒童進至於「成人」之標誌。「成人之者」是把一個兒童的人看成一個「成人」也。視之為一「成人」即在此可以責成其盡成人之禮焉。責成其盡成人之禮即責成其對父母盡「為人子」之禮、對兄盡「為人弟」之禮、對君盡「為人臣」之禮、對長者盡「爲人少者」之禮也。能盡此等等禮即為一獨立之人格(成人)。但冠禮所規定之成人只是一形式的成人,其規定亦只是形式地規定之,此即是王者禮樂中之成人,王者禮樂中之人倫。但由形式地成進至自覺地實踐地成則是聖者成德之教中的成人,成德之教中的人倫。故荀子

\newpage\thispagestyle{empty}\addtocounter{page}{-1}\vspace*{-12mm}\begin{center}\noindent
\includegraphics[clip, trim=178pt 127pt 134pt 262pt, height=162mm]{ocr-input/image-0107.png}\end{center}

\newpage\markright{第一部 \quad 第一章 \quad 宋明儒學之課題}

\noindent 云:「王者盡制者也,聖者盡倫者也。」成德之教中的成人即是孔子的仁教之所開啓,此代表孔子生命智慧之方向。又如昏禮「成婦禮,明婦順,又申之以著代」(《禮記·昏義》),此是夫婦之道之規定,此規定亦是形式地規定,但《中庸》說:「君子之道造端乎夫婦,及其至也,察乎天地」,則是成德之教中的夫婦之道。君子自覺地實踐人倫以成其德即從這裡開始,「及其至也」,無窮無盡,故云「察乎天地」。開始時,雖夫婦之愚不肖可以與知能行,及其至也,雖聖人亦有所不知,亦有所不能。此種成德之教是孔子之所開啟,與王者盡制中之禮樂人倫不同也。是故劉蕺山〈人譜續篇〉二,〈證人要旨〉:「四曰敦大倫以凝道」,解之云:「人生七尺,墮地後,便為五大倫關切之身,而所性之理與之一齊俱到。〔……】然必待其人而後行。故學者工夫,自慎獨以來,根心生色,暢於四肢,自當發於事業;而其大者,先授之五倫。於此尤加致力,外之何以極其規模之大,內之何以究其節目之詳,總期踐履敦篤,慥慥君子以無忝此率性之道而已。昔人之言曰:「五倫間有多少不盡分處。』夫惟常懷不盡之心,而黽黽以從事焉,庶幾其逭於責乎?」(《劉子全書》卷一,〈人譜〉)。此方真是「成德之敎」之實,此是孔子生命智慧之所開啟,王者盡制並未言及此也。是故徒以傳經為儒,徒以「列君臣父子之禮,序夫婦長幼之別」爲儒,則是從孔子繞出去而從王者,是並未真能了解儒家之本質。故儒之為儒必須從王者盡制之外部的禮樂人倫處規定者進而至於由聖者盡倫之「成德之教」來規定,方能得其本質,盡其生命智慧方向之實。此則必須以孔子為標準,而不能以堯、舜、禹、湯、文、武爲標準也。此中之差別亦恰似基督教與猶太教之差別。而為宋儒所

\newpage\thispagestyle{empty}\addtocounter{page}{-1}\vspace*{-12mm}\begin{center}\noindent
\includegraphics[clip, trim=164pt 134pt 141pt 257pt, height=162mm]{ocr-input/image-0111.png}\end{center}

\newpage

\noindent 認識,此其所以為新也。故以傳經為儒,固是以王者之禮樂人倫為中心,此不足以了解孔子,不足以盡儒家之本質,即後來之言經制事功者,亦仍是以王者為中心,進而更輕忽於孔子,至于曾子、子思、孟子尤不在話下,此則以葉水心為代表,詳論見下第五章。

惟以上所述之兩點「新」猶是外部的。如果誠有此事實,如果孔子真不只是堯、舜、禹、湯、文、武之驥尾,傳經之媒介,而有其獨特處,則宋、明儒亦不過將此事實予以圈出而已,於客觀事實無增減。如果孔子誠有一傳統,其師弟間誠有一生命智慧上之相呼應,而孟子、(中庸》、《易傳》與《大學》亦事實俱在,誠能代表此呼應,不容抹殺,則宋、明儒亦只是圈出此傳統,於客觀事實亦無增減。如果此生命智慧相呼應所成之傳統確足以代表儒家之本質,確足以表示孔子生命智慧之方向,則宋、明儒就此規定儒亦只是圈出一事實,於客觀事實亦無增減。是以上述兩點新只是外部的認識,尚不是客觀內容之新。此種認識上圈定上的「新」,人易見也,雖不必能知其實義。然則宋、明儒於此兩點新以外,是否尚有客觀內容上的新?如有之,則真成其為新,如無之,則終不足成其為新。「新」有二義:一是順本有者引申發展而為本有者之所函,此種「函」是調適上逐地函;二是於基本處有相當之轉向,(不是徹底轉向),歧出而另開出一套以為輔助,而此輔助亦可為本有者之所允許,此種允許是迂曲歧出間接地允許,不是其本質之所直接地允許者。前者之新於本質無影響,亦即是說恰合原義;後者之新於本質有影響,亦即是說於原義有不合處。依通常使用「新」字之義說,於本質無影響者實不得為「新」,只是同一本質之不同表示法而已。而於本質有影響者始有「新」的意義。然則宋、明儒所講

\newpage\thispagestyle{empty}\addtocounter{page}{-1}\vspace*{-12mm}\begin{center}\noindent
\includegraphics[clip, trim=170pt 116pt 136pt 272pt, height=162mm]{ocr-input/image-0115.png}\end{center}

\newpage\markright{第一部 \quad 第一章 \quad 宋明儒學之課題}

\noindent 者之客觀內容底新,如其有之,究是前者之新?抑是後者之新?抑是兩者兼而有之?此則未易言也。此須對於孔子傳統真有生命上之感應,對於宋、明儒所圈定之代表此傳統之儒家經典真有生命上之相契,而對於宋、明儒諸大家真有確實之經歷與檢定,方足以決定之,此非浮氾、摇荡、淺嘗者所能知也。吾兹先言其大略如下:

1.孔子踐仁知天,未說仁與天合一或爲一,但依宋、明儒,其共同傾向則認為仁之內容的意義(intensional meaning)與天之內容的意義到最後完全合一,或即是一。(在此,伊川、朱子稍有不同)。

2.孟子言盡心知性知天,心性是一,但未顯明地表示心性與天是一。宋、明儒的共同傾向則認為心性天是一。(在此,伊川、朱子亦有不同)。

3.《中庸》說「天命之謂性」,但未顯明地表示天所命於吾人之性其內容的意義完全同於那「天命不已」之實體,或「天命不已」之實體內在於個體即是個體之性。宋、明儒則顯明地如此表示。此所謂天道性命通而為一也。在此,伊川朱子亦無異辭,惟對於天命實體與性體理解有不同。

4.《易傳》說「乾道變化,各正性命」(〈乾彖〉),此字面的意思只表示在乾道(天道)變化底過程中各個體皆得正定其性命,未顯明地表示此所正之「性」即是乾道實體或「為物不貳,生物不測」之天道實體內在於各個體而為其性,所正之「命」亦即是此實體所定之命。但宋、明儒則顯明地如此表示,在此處與在《中庸》處同。

5.《大學》言「明明德」,未表示「明德」即是吾人之心性

\newpage\thispagestyle{empty}\addtocounter{page}{-1}\vspace*{-12mm}\begin{center}\noindent
\includegraphics[clip, trim=149pt 132pt 140pt 252pt, height=162mm]{ocr-input/image-0119.png}\end{center}

\newpage

\noindent (就本有之心性說明德),甚至根本不表示此意,乃只是「光明的德行」之意。但宋、明儒一起皆認為「明德」是就因地之心性說,不就果地之「德行」說。又《大學》言「致知在格物」亦不必如伊川、朱子所理解,「致知」為致吾心氣之靈之知,「格物」為即物而窮其存在之理(窮究實然者之所以然之理)。至於陽明解為「致良知之天理以正物」,則只是孟子學之《大學〉,非必《大學〉之本義·劉蕺山之誠意教則亦只是《中庸〉、孟子學之《大學》,亦非《大學〉之本義。大學之「明德」只是就「德行」說,知是「知止」知「至善」知「本末先後」之「知」,物是「心、意、身、家、國、天下」之物,至善之道(止處)是就應物之「事」上說,至於至善之道究往何處落,則不能定。陽明、蕺山是往心性處落,伊川、朱子是往存在之理處落,皆非《大學〉原有之義。是則《大學〉只列舉出一個實踐底綱領,只說一個當然,而未說出其所以然,在內聖之學之義理方向上為不確定者,究往那裡走,其自身不能決定,故人得以填彩而有三套之講法。

以上前四點是就《論》《孟》《中庸》、《易傳》而推進一步,自然表示一種「新」的意義,但此「新」吾人可斷定是調適上遂的新,雖是引申發展,但卻為原有者之所函。第五點就《大學》所表示的新,陽明、蕺山的講法雖不合大學章句原義,然如將《大學》納於《論》《孟》《中庸》《易傳》之成德之教中而提挈規範之,則該兩種講法於先秦儒家之本質不生影響。但伊川、朱子之講法,再加上其對於《論》《孟》《中庸》、《易傳〉之仁體、心體、性體,乃至道體理解有差,結果將重點落在《大學》,以其所理解之《大學》為定本,則於先秦儒家原有之義

\newpage\thispagestyle{empty}\addtocounter{page}{-1}\vspace*{-12mm}\begin{center}\noindent
\includegraphics[clip, trim=176pt 120pt 125pt 263pt, height=162mm]{ocr-input/image-0123.png}\end{center}

\newpage\markright{第一部 \quad 第一章 \quad 宋明儒學之課題}

\noindent 有基本上之轉向,此則轉成另一系統,此種新於本質有影響,此為岐出之「新」。此一系統雖在工夫方面有輔助之作用,可為原有者之所允許,然亦是迂曲岐出間接地助緣地允許,不是其本質之所直接地允許者,即不是其本質的工夫之所在。至於在本體方面,則根本上有偏差,有轉向,此則根本上非先秦儒家原有之義之所允許。如果前一種新,以《論》、《孟》、《中庸》、《易傳》爲主者,實不算得是新,則宋、明儒學中有新的意義而可稱為「新儒學」者實只在伊川朱子之系統。大體以《論》《孟》《中庸》、《易傳》為主者是宋、明儒之大宗,而亦較合先秦儒家之本質;伊川、朱子之以《大學》為主則是宋、明儒之旁枝,對先秦儒家之本質言則為歧出。然而自朱子權威樹立後,一般皆以朱子為正宗,儷侗稱之曰程、朱,實則只是伊川與朱子,明道不在內。朱子固偉大,能開一新傳統,其取得正宗之地位,實只是別子為宗也。人忘其舊,遂以為其紹孔、孟之大宗矣。

何以能如此判斷?此則須有進一步之說明。

\section{宋、明儒之課題}

如上節所述,宋明儒是把《論》《孟》、《中庸》、《易傳》與《大學》劃為孔子傳統中內聖之學之代表。此五部經典,就分量方面說,亦並不甚多。但此中當有辨。據吾看,《論》、《孟》、《中庸》《易傳》是孔子成德之教(仁教)中其獨特的生命智慧方向之一根而發,此中實見出其師弟相承之生命智慧之存在地相呼應。至於《大學》,則是開端別起,只列出一個綜括性

\newpage\thispagestyle{empty}\addtocounter{page}{-1}\vspace*{-12mm}\begin{center}\noindent
\includegraphics[clip, trim=155pt 132pt 140pt 251pt, height=162mm]{ocr-input/image-0127.png}\end{center}

\newpage

\noindent 的,外部的(形式的)主客觀實踐之綱領,所謂只說出其當然,而未說出其所以然。宋、明儒之大宗實以《論》、《孟》、《中庸〉、《易傳》為中心,只伊川、朱子以《大學》為中心。分別言之,濂溪開始,只注意《中庸》《易傳》,對於《論》《孟》所知甚少,且無一語道及《大學》。橫渠漸能注意《論》、《孟》,亦未言及《大學》。至明道,通《論》、《孟》、《中庸〉、《易傳》而一之,以言其「一本」義,亦少談大學。胡五峰亦不論《大學》。象山純是《孟子》學,以《孟子》攝《論語》。就關涉於《中庸〉《易傳》之理境言,則只是一心之申展,是亦兼攝《中庸》、《易傳》也。然而亦很少論《大學》。偶有言及,亦只是假借《大學〉之詞語以寄意耳。自朱子權威成立後,陽明亦著力於《大學》,著落於《大學》以展示其系統,實則仍是《孟子》學,假《大學》以寄意耳。劉蕺山就《大學》言誠意,其背景仍是《中庸》、《易傳》與《孟子》也。伊川、朱子所講之《大學》雖亦不必合《大學》之原義,然一因伊川、朱子對於《論》、《孟》《中庸》《易傳》所言之仁體、心體性體、道體不能有相應之契悟,(心性為二、性道只是理、心理為二),二因《大學》之「明德」不必是因地之心性,「至善之則」不能確定往何處落,故伊川、朱子以其實在論的、順取的態度將其所理解之性體、道體、仁體(都只是理)著落於致知格物以言之,以成其能所之二,認知關係之靜攝,將致知格物解為常情所易見之認知義,將「至善之則」著落在所格之物之「存在之理」上,此雖不合《大學》之原義,然因在《大學》,至善之則不能確定往何處落,則如此解《大學》亦甚順適,此即成主智論,以智決定意,此是直接從

\newpage\thispagestyle{empty}\addtocounter{page}{-1}\vspace*{-12mm}\begin{center}\noindent
\includegraphics[clip, trim=172pt 117pt 125pt 264pt, height=162mm]{ocr-input/image-0131.png}\end{center}

\newpage\markright{第一部 \quad 第一章 \quad 宋明儒學之課題}

\noindent 《大學》上順著講而即可講出者。此是以《大學》為主而決定《論》《孟》、《中庸》、《易傳》也。是故《大學》在伊川、朱子之系統中,其比重比以《論》、《孟》、《中庸》、《易傳》為主者為重,對於其系統有本質上之作用,而在其他則只是假託以寄意耳。其實意是將《大學》上提於《論》《孟》、《中庸》、《易傳》,而以《論》、《孟》、《中庸》、《易傳》決定或規範《大學》也。此是宋明儒之事實。故吾人實可將《大學》與《論》、《孟》、《中庸》《易傳》分開看,而以《大學》為待決定者,由此以識宋、明儒之大宗。若以《大學》為決定者,則即形成伊川、朱子之系統。

識宋明儒之大宗即是恢復《論》、《孟》《中庸》《易傳》之主導的地位。在此,吾人首先須知:依宋、明儒大宗之看法,《論》、《孟》、《中庸》、《易傳》是通而為一而無隔者,故成德之教是道德的同時即宗教的,就學問言,道德哲學即函一道德的形上學。在此,吾人可問:此通而為一的看法是否可允許?先秦儒家的發展是否能啟發出此看法而可以使吾人認為此看法為合法?茲仍順上節所開之大略申明之如下:

1.關於仁與天。孔子所說的「天」、「天命」,或「天道」當然是承《詩》、《書》中的帝、天、天命而來。此是中國歷史文化中的超越意義,是一老傳統。以孔子聖者之襟懷以及其歷史文化意識(文統意識)之強,自不能無此超越意識,故無理由不繼承下來。但孔子不以三代王者政權得失意識中的帝、天、天命為已足,其對於人類之絕大的貢獻是暫時撇開客觀面的帝、天、天命而不言(但不是否定),而自主觀面開啟道德價值之源、德性生命之門以

\newpage\thispagestyle{empty}\addtocounter{page}{-1}\vspace*{-12mm}\begin{center}\noindent
\includegraphics[clip, trim=170pt 135pt 143pt 263pt, height=162mm]{ocr-input/image-0135.png}\end{center}

\newpage

\noindent 言「仁」。孔子是由践仁以知天,在踐仁中或「肫肫其仁」中知之、默識之、契接之或崇敬之。故其暫時撇開客觀面的帝、天、天命而不言,並不是否定「天」或輕忽「天」,只是重在人之所以能契接「天」之主觀根據(實踐根據),重人之「真正的主體性」也。重「主體性」並非否定或輕忽帝、天之客觀性(或客體性),而勿寧是更加重更真切於人之對於超越而客觀的天、天命、天道之契接與崇敬。不然,何以說「五十而知天命」?又何以說「畏天命」?孔子此步「踐仁知天」之提供,一方豁醒人之真實主體性,一方解放了王者政權得失意識中之帝、天或天命。

《詩》、《書》中的帝、天、天命雖常有人格神的意味,然亦不如希伯來民族之強烈與凸出。《詩》、《書》中之重德行已將重點或關捩點移至人身上來,此亦可說已開孔子重「主體性」之門。孔子之提出「仁」,實由《詩》、《書》中之重德、敬德而轉出也。是故《詩》、《書〉中之帝、天、天命只肯認有一最高之主宰,只凸出一超越之意識,並不甚向人格神之方向凸出。迤邏而至孔子,此方向總不甚凸出。故孔子承其以前之氣氛,其心目中之天、天命或天道亦只集中而為一超越意識,並不像希伯來宗教意識中之上帝那樣孤峭而挺立,其意味甚為肅穆,對於天地萬物甚具有一種「超越的親和性」(引曳性 transcendental affinity),冥冥穆穆運之以前進,是這樣意味的一個「天」。並不向「人格神」的方. 向走。孔子雖未說天即是一「形而上的實體」(metaphysicalreality),然「天何言哉?四時行焉,百物生焉。天何言哉!」實亦未嘗不函蘊此意味。「維天之命,於穆不已」,難說孔子未讀此詩句,亦難說其不契此詩句。前聖後聖,其心態氣氛之相感應,大

\newpage\thispagestyle{empty}\addtocounter{page}{-1}\vspace*{-12mm}\begin{center}\noindent
\includegraphics[clip, trim=140pt 115pt 128pt 269pt, height=162mm]{ocr-input/image-0139.png}\end{center}

\newpage\markright{第一部 \quad 第一章 \quad 宋明儒學之課題}

\noindent 體可見矣。是故後乎孔子之《中庸》即視天為「為物不貳,生物不測」之創生實體,而以「維天之命,於穆不已」明「天之所以為天」,此即以「天命不已」之實體視天也。此種以「形而上的實體」視天雖就孔子推進一步,然亦未始非孔子意之所函與所許。此亦是其師弟相承之生命智慧之相感應相呼招,故即如此自然地視「天」也。此亦不礙超越意識之凸出,亦不礙其對於天之崇敬與尊奉。孔子前後生命智慧之相呼應如此,則宋、明儒尤其如明道者即如孔門之呼應而亦存在地以真實生命如此呼應之,直視孔子之天為一形而上的實體而與後來之《中庸》《易傳》通而一之也。其如此看自亦不妨礙天之超越義,以及對於天之崇敬與尊奉。

天之義如此,則仁心感通之無限即足以證實「天之所以為天」,天之為「於穆不已」,而與之合而為一。在孔子,踐仁知天,雖似仁與天有距離,仁不必即是天,孔子亦未說仁與天合一或爲一,然(1)因仁心之感通乃原則上不能劃定其界限者,此即函其向絕對普遍性趨之申展。(2)因踐仁知天,仁與天必有其「內容的意義」之相同處,始可由踐仁以知之,默識之,或契接之,依是二故,仁與天雖表面有距離,而實最後無距離,故終可合而一之也。《中庸》言「肫肫其仁,渊渊其渊,浩浩其天」,此即示仁心仁道之深遠與廣大而與天為一矣。《易傳》言天道「顯諸仁,藏諸用,鼓萬物而不與聖人同憂,盛德大業至矣哉!」此亦是仁與天為一也。此亦未始非孔子意之所函與所許。如果天向形而上的實體走,不向人格神走,此種合一乃是必然者。此亦是孔門師弟相承,其生命智慧之相呼應,故如此自然說出也。宋、明儒尤其如明道即依此呼應而亦存在地呼應之,遂直視仁與天為一矣。在此,明道對於仁

\newpage\thispagestyle{empty}\addtocounter{page}{-1}\vspace*{-12mm}\begin{center}\noindent
\includegraphics[clip, trim=162pt 133pt 140pt 259pt, height=162mm]{ocr-input/image-0143.png}\end{center}

\newpage

\noindent 之體會不誤也。此須有生命智慧之存在地相感應始能知,非文字之訓詁與知解事也。自明道如此體會後,宋、明儒之大宗無人不首肯。伊川、朱子之講法(以公說仁,仁性愛情,仁是心之德愛之理),不能有此呼應也。

以上由踐仁知天,說仁與天合一,天是「實體」義的天,積極意義的天,是從正面說,從「先天而天弗違」說。(「天弗違」之天是形而下的天)。至於孔子說「知天命」,「畏天命」,「知命」,以及慨嘆語句中的「天」,則是表示一「超越的限定」義。此則不純是以「實體」言(普通所謂以理言)的天,當然亦不純是以氣言的天,乃是「實體帶著氣化、氣化通著實體」的「天」,此是從「後天而奉天時」說。此義在此不論。

2.關於仁與心性以及心性與天。孔子未說「心」字,亦未說「仁」即是吾人之道德的本心,然孔子同樣亦未說仁是理、是道。心、理、道都是後人講說時隨語意帶上去的。實則落實了,仁不能不是心。仁是理、是道,亦是心。孔子由「不安」指點仁,不安自是心之不安。其他不必詳舉。故孟子即以「不忍人之心」說仁。理義悅心,亦以「理」說仁。「仁者人也,合而言之,道也」,亦以「道」說仁。這些字都是自然帶上去的,難說非孔子意之所函,亦難說孔子必不許也,是以孟子即以道德的本心攝孔子所說之仁。

孔子亦未說仁即是吾人之「性」。子貢言「夫子之言性與天道不可得而聞也」。孔子亦偶爾言及「性相近也,習相遠也。」其心中如何意謂「性」字很難說。「性相近也」之「性」,伊川、朱子俱視為氣質之性,此大體亦不誤。劉籤山解「相近」為「相同」即指同一「於穆不已」之性體言,故性無不善。(參看《劉子全書》

\newpage\thispagestyle{empty}\addtocounter{page}{-1}\vspace*{-12mm}\begin{center}\noindent
\includegraphics[clip, trim=167pt 114pt 132pt 269pt, height=162mm]{ocr-input/image-0147.png}\end{center}

\newpage\markright{第一部 \quad 第一章 \quad 宋明儒學之課題}

\noindent 卷十九,〈答王右仲州刺〉)。吾人由此可以想孔子所說之「相近」即是孟子所說「其好惡與人相近也者幾希」之「相近」。孟子說此「幾希」之「相近」是指良心好惡之呈露言。所呈露者雖不多,然卻是與人相同者,並無異樣之良心。是則「相近」即相同。如果孔子所說之「相近」即是此意義之「相同」,則「性」當是同一的義理本然之性,不能是氣質之性。如果是同一的義理本然之性,則孔子當該想到仁就是性,就是吾人之性之實。即使想不到,亦未說到,後人(如孟子)如此說,亦無過。但孔子所說之「相近」是否必如此,則難定。即使與孟子所說之「相近」字面相同,而其實指不必相同。孟子可用「相近」指本然之性(良心)言,因而「相近」即「相同」,而孔子所用之「相近」不必指此本然之性言,而亦仍可用「相近」,因而「相近」不必即「相同」。如果與「唯上智與下愚不移」連在一起看,則此「可移」之「相近」者亦仍只是氣性,才性之類也。是則伊川、朱子說為氣質之性亦非定誤。至於子貢所不可得而聞之「性」,與「天道」連在一起說,究是指何層面之「性」,則亦難說。如果指超越面的義理之性說,則當與仁為一,仁即是吾人性體之實。如果指經驗面的氣性、才性或「生之謂性」之性說,則仁與性不能是一。而無論自那一面說,「性」之義皆是相當奧密而難聞的。在此,吾人對於孔子的態度不能確知。孔子前「性」字即已流行,然大體是「性者生也」,無自超越面言性者。「生之謂性」是一老傳統。孔子已接觸此問題,然可能一時未能消化澈,猶處於「性者生也」之老傳統中,故性是性,仁是仁,齊頭並列,一時未能打併為一。(性者生也,雖卑之無高論,說的是現實的人性,自然生命之徵象,似乎無甚難聞處,

\newpage\thispagestyle{empty}\addtocounter{page}{-1}\vspace*{-12mm}\begin{center}\noindent
\includegraphics[clip, trim=158pt 137pt 134pt 245pt, height=162mm]{ocr-input/image-0151.png}\end{center}

\newpage

\noindent 然認真討論起來,亦並不簡單。非必只同於「天道」之性或超越面之性為難聞也)。然孔子言仁如此親切,而又真切,其看人性亦斷然不會直說為惡,亦斷然不會只從人之欲性看性。然亦同樣未自覺地說到仁即是性。是則性之問題在孔子猶是敞開者,雖或偶爾觸及,然未能十分正視而著力。若依子貢之語觀之,雖難聞,而夫子未始不言,至少亦未始無其洞悟處,而結果終所以難聞而又不常言多言者,則或可如此說,即,性之問題,初次觀之,似是屬於「存有」之問題,無論卑之從「生之謂性」說,或高之從超越面說,皆然。而一涉及「存有」問題,則總是奧密的,此即法國存在主義者馬塞爾(Marcel)所謂「存有之秘密」(mystery of being)是也。此其所以為難聞乎?而一個聖者如孔子則總是多偏重於自實踐言道理,很少有哲學家之興趣去積極地思議存有問題也。即使有洞悟,亦是在踐履中洞悟之,因而多言踐履之道如仁,而少涉及存有問題如性與天道,此其所以不常言多言也。

至孟子時,性之問題正式成立。告子順「性者生也」之老傳統說性,而孟子遮撥之,則從道德的本心說,此顯然以孔子之仁為背景。在孔子,仁與性未能打併為一,至此則打併為一矣。在孔子,存有問題在踐履中默契,或孤懸在那裡,而在孟子,則將存有問題之性即提升至超越面而由道德的本心以言之,是即將存有問題攝於實踐問題解決之,亦即等於攝「存有」於「活動」(攝實體性的存有於本心之活動)。如是,則本心即性,心與性為一也。至此,性之問題始全部明朗,而自此以後,遂無隔絕之存有問題,而中國亦永無或永不會走上西方柏拉圖傳統之外在的,知解的形上學中之存有論,此孟子創闢心靈之所以為不可及也,而實則是孔子之仁有以

\newpage\thispagestyle{empty}\addtocounter{page}{-1}\vspace*{-12mm}\begin{center}\noindent
\includegraphics[clip, trim=173pt 127pt 134pt 258pt, height=162mm]{ocr-input/image-0155.png}\end{center}

\newpage\markright{第一部 \quad 第一章 \quad 宋明儒學之課題}

\noindent 啓之也。仁之全部義蘊皆收于道德之本心中,而本心即性,故孔子所指點之所謂「專言」之仁,即作為一切德之源之仁,亦即是吾人性體之實也。此唯是攝性於仁、攝仁於心、攝存有於活動,而自道德實踐以言之。至此,人之「真正主體性」始正式挺立而朗現,而在孔子之踐仁知天,吾人雖以重主體性說之,然仁之為主體性只是吾人由孔子之指點而逼近地如此說,雖是呼之欲出,而在孔子本人究未如孟子之如此落實地開出也。此即象山所謂「夫子以仁發明斯道,其言渾無罅縫,孟子十字打開,更無隱遁」之義也。孟子如此「打開」,是其生命智慧與其所私淑之孔子相呼應,故能使仁與心與性通而一之,而宋、明儒如明道與象山者即如其相呼應而亦存在地呼應之,直下視仁與心與性為一也。而伊川與朱子則去此遠矣。

仁與心、性如此,則孟子處心性與天之關係即同於孔子處仁與天之關係。孟子從道德實踐上只表示本心即性,只說盡心知性則知天,未說心性與天為一。然「萬物皆備於我矣,反身而誠,樂莫大焉」,則心即函一無限的申展,即具一「體物而不可遺」的絕對普遍性。是則心本可與天合一而為一也,能盡其心,則即可知性,是則心之內容的意義與性之內容的意義全同,甚至本心即性。蓋性即吾人的「內在道德性」之性,亦即能起道德創造大用,能使道德行為純亦不已之「性」也。由盡心(充分實現其本心)而知性,即知的這個「性」。同樣,若知了性,則即可知「天」,是則性之「內容的意義」亦必有其與天相同處,吾人始可即由知性而知天也。在孟子的語句上似表示心性與天尚有一點距離,本心即性,而心性似不必即天。然此一點距離,一因心之絕對普遍性,二因性或心性之內容的意義有同於天處,即可被撤銷。故明道云:「只心便

\newpage\thispagestyle{empty}\addtocounter{page}{-1}\vspace*{-12mm}\begin{center}\noindent
\includegraphics[clip, trim=173pt 145pt 126pt 248pt, height=162mm]{ocr-input/image-0159.png}\end{center}

\newpage

\noindent 是天,盡之便知性,知性便知天,當下便認取,更不可外求。」明道如此說,實因其生命智慧與孟子相呼應,孟子本可有此開啓,故即存在地呼應之而即如此說出也。如果「天」不是向「人格神」的天走,又如果「知天」不只是知一超越的限定,與「知命」稍不同,則心性與天為一,「只心便是天」,乃係必然者。盡心知性則知天,順心性說,則此處之「天」顯然是「實體」義的天,即所謂以理言的天,從正面積極意義看的天。所謂性之內容的意義有其與天相同處亦是從積極意義的「天」、「實體」意義的天說。此所謂「內容的意義」相同實則同一創生實體也。「天」是客觀地、本體宇宙論地言之,心性則是主觀地、道德實踐地言之。及心性顯其絕對普遍性,則即與天為一矣。明道如此呼應,宋明儒之大宗亦無一不如此呼應。惟伊川、朱子則轉成另一系統,遂亦不能有此呼應矣。

「盡其心者知其性也,知其性則知天矣」。此相當于〈乾·文言〉之「先天而天弗違」。在此,唯是一實體之徹底朗現,故心性天是一。(「而天弗違」之「天」是形而下的天,與「心性天是一」之天不同。)天地鬼神皆不能違離此實體也。

「存其心,養其性,所以事天也」。此相當于〈乾·文言〉之「後天而奉天時」。在此,「天」須帶著氣化說,而吾人心性之與天不即是一。然亦須存住吾人之本心而不放失,養住吾人之道德創造之性而不鑿喪,然後始能事天而奉天。及其一體而化,則天之氣化即吾之氣化(吾之性體純亦不已之所顯),天時之運即吾之運,知即奉,奉即知,知奉之分泯,而先後天之異亦融而爲一矣。此孟子所謂「上下與天地同流」,亦明道所謂之「一本」也。此是「大

\newpage\thispagestyle{empty}\addtocounter{page}{-1}\vspace*{-12mm}\begin{center}\noindent
\includegraphics[clip, trim=154pt 124pt 137pt 255pt, height=162mm]{ocr-input/image-0163.png}\end{center}

\newpage\markright{第一部 \quad 第一章 \quad 宋明儒學之課題}

\noindent 而化之」之聖神之境。然人畢竟亦是一現實之存在。自現實存在言,則不能不有一步謙退,因此顯出一層退處之「事天」義。不但顯出此退處之「事天」義,且可進而言「立命」。

「夭壽不贰,修身以俟之,所以立命也。」「立命」即立「超越之限定」義。在此,如說「天」,亦是帶著氣化的天,而且特重氣化對於吾人之限制,吾人之現實存在與此氣化相順相違之距離。在此,即有「命」之意義,此即所謂「立命」。知道有此限制,此是「命」之實。命本自有之。此是客觀地立。但必須真能主觀地「夭壽不貳,修身以俟之」,方始真能「立命」,此是主觀地,實踐地立。「修身」亦須以「盡心知性」、「存心養性」為根據,否則亦不能「修身」。是則「修身」即函蘊盡心知性、存心養性也。

是故「盡心知性知天」是自「體」上言。在此,心性天是一。「存心養性事天」是自人為一現實存在言,天亦是帶著氣化說。在此,心性因現實存在之拘限與氣化之廣大,而與天不即是一。自「一體而化」言,則此分別即泯。從體上說是一,帶著用說亦是一也。「立命」則是就現實存在與氣化之相順相違言,此不是說心性與天的事,而是說帶著氣化的天與吾人之現實存在間之相順相違的事。至「一體而化」之境,則一切皆如如之當然,亦無所謂「命」也。言至此,知天、事天、立命以及一體而化,全部皆備,此真所謂「孟子十字打開,更無隱遁」也。

朱子解盡心知性為致知格物,解存心養性為正心誠意,固誤;而王陽明以盡心知性為「生而知之」,以存心養性為「學而知之」,以「立命」爲「困而知之」,此種比配尤為不類。陽明《傳習錄〉義理精熟圓透,很少有不順適處,惟於此處則極顯不類,滯

\newpage\thispagestyle{empty}\addtocounter{page}{-1}\vspace*{-12mm}\begin{center}\noindent
\includegraphics[clip, trim=171pt 146pt 131pt 246pt, height=162mm]{ocr-input/image-0167.png}\end{center}

\newpage

\noindent 之甚矣。不知何故。而且此義凡三見,此非偶爾之失。吾想象山決不至此也。

3.關於「天命之謂性」。《中庸》說此語,其字面的意思是:天所命給吾人者即叫做是性,或:天定如此者即叫做是性。單就此語本身看,尚看不出此天所命而定然如此之「性」究是何層面之性。然依下句「率性之謂道」一語看,性不會是氣性之性。又依「中也者,天下之大本也」一語看,如果「中」字即指「性體」言,則作為「天下之大本」之中體性體,亦決不會是氣性之性。又依《中庸》後半部言誠、言盡性,誠是工夫亦是本體,是本體亦是工夫,誠體即性體,性亦不會是氣性之性。此可能是根據孟子言性善而來。孟子雖從道德自覺上只道德實踐地言「仁義內在」,言本心即性,言「我固有之」,似未客觀地從天命、天定言起,然孟子亦言「心之官則思,思則得之,不思則不得也。此天之所與我者。得其大者,則其小者弗能奪也。」由「此天之所與我者」看,則於此心此性,孟子亦未嘗無「天命、天定」義。又引「天生烝民,有物有則,民之秉彝,好是懿德」之詩以證性善,則「秉彝」之性亦未嘗不是天所命而定然如此者。「固有」即是先天而本有,即是天所命而定然如此者。然則《中庸》說「天命之謂性」即是與孟子相呼應而說出也。

宋、明儒如橫渠、明道、五峰、蕺山等人不但承認此呼應,且進而表示此「天所命而定然如此」之性,其內容的意義即同於「於穆不已」之天命實體。「天命之謂性」不能直解為「於穆不已」之天命實體即叫做是性,然「天所命而定然如此」之性,如進一步看其「內容的意義」,亦實函此義。從此義說性,則孟子之自道德自

\newpage\thispagestyle{empty}\addtocounter{page}{-1}\vspace*{-12mm}\begin{center}\noindent
\includegraphics[clip, trim=160pt 128pt 134pt 251pt, height=162mm]{ocr-input/image-0171.png}\end{center}

\newpage\markright{第一部 \quad 第一章 \quad 宋明儒學之課題}

\noindent 覺上道德實踐地所體證之心性,由其「固有」、「天之所與」,即進而提升為與「天命實體」為一矣。而此亦即形成客觀地從本體宇宙論的立場說性之義。如果「天」不是人格神的天,而是「於穆不已」的「實體」義之天,而其所命給吾人而定然如此之性又是以理言的性體之性,即超越面的性,而不是氣性之性,則此「性體」之實義(內容的意義)必即是一道德創生之「實體」,而此說到最後必與「天命不已」之實體(使宇宙生化可能之實體)為同一,決不會「天命實體」為一層,「性體」又為一層。依《中庸》後半部言「誠」,本是內外不隔,主客觀為一,而自絕對超然的立場上以言之的,此即「誠體」即同於「於穆不已」之天命實體也。言「天地之道」為「為物不貳,生物不測」,則天地之道即是一「於穆不已」之創生實體,而此亦即是「無內外」之誠體也。《中庸》引「維天之命於穆不已」之詩句以證「天之所以為天」,則「天」非人格神的天可知。是則誠體即性體,亦即天道實體,而性體與實體之實義不能有二亦明矣。就其統天地萬物而為其體言,曰實體;就其具於個體之中而為其體言,則曰性體。言之分際有異,而其為體之實義則不能有異。是即橫渠所謂「天所性者通極於道,氣之昏明不足以蔽之」之義。性體與道體或天命實體通而為一,故自此義言性者特重「維天之命於穆不已」之詩,遂形成客觀地超越地自本體宇宙論的立場說性之義,而與孟子之自道德自覺實踐地說性、特重「民之秉彝好是懿德」之詩句者有異,然而未始不相呼應、相共鳴,而亦本可如此上提也。由孟子之自道德自覺上實踐地說性,由其如此所體證之性之「固有」義、「天之所與」義,以及本心即性、「萬物皆備於我」、心性向絕對普遍性申展之義,則依一形而

\newpage\thispagestyle{empty}\addtocounter{page}{-1}\vspace*{-12mm}\begin{center}\noindent
\includegraphics[clip, trim=170pt 147pt 138pt 248pt, height=162mm]{ocr-input/image-0175.png}\end{center}

\newpage

\noindent 上的洞悟滲透,充其極,即可有「性體與天命實體通而為一」之提升。《中庸》如此提升,實與孟子相呼應,而圓滿地展示出。《中庸〉之如此提升與孟子並非互相敵對之兩途。此不可以西方康德之批判哲學與康德前之獨斷形上學之異來比觀。此只可以圓滿發展看,不可以相反之兩途看。

由於《中庸》之提升,宋、明儒即存在地與之相呼應,不但性體與天命實體上通而為一,而且直下由上面斷定;天命實體之下貫於個體而具於個體(流註於個體)即是性。「於穆不已」即是「天」此實體之命令作用之不已,即不已地起作用也。此不已地起命令作用之實體命至何處即是作用至何處,作用至何處即是流註至何處。流註於個體即為個體之性。此是承《中庸》之圓滿發展直下存有論地言之也。此雖與《中庸》稍有間,然實為《中庸》之圓滿發展之所函,宋、明儒如此斷定,不得謂無根也。

此斷定幾乎是宋、明儒共同之意識,即伊川、朱子亦不能外乎此,即象山、陽明亦不能謂此為歧出。惟積極地把握此義者是橫渠、明道、五峰與蕺山,此是承《中庸》、《易傳》之圓滿發展而言此義者之正宗。伊川、朱子亦承認此義,惟對於實體、性體、理解有偏差,即理解為只是理,只存有而不活動,此即喪失「於穆不已」之實體之本義,亦喪失能起道德創造之「性體」之本義。象山、陽明則純是孟子學,純是一心之申展。此心即性,此心即天。如果要說天命實體,此心即是天命實體。象山云:「萬物森然於方寸之中,滿心而發,充塞宇宙,無非斯理。」陽明云:「充天塞地中間,只有這個靈明。人只為形體自間隔了。我的靈明便是天地鬼神的主宰。天沒有我的靈明,誰去仰他高?地沒有我的靈明,誰去

\newpage\thispagestyle{empty}\addtocounter{page}{-1}\vspace*{-12mm}\begin{center}\noindent
\includegraphics[clip, trim=164pt 135pt 134pt 247pt, height=162mm]{ocr-input/image-0179.png}\end{center}

\newpage\markright{第一部 \quad 第一章 \quad 宋明儒學之課題}

\noindent 俯他深?鬼神沒有我的靈明,誰去辨他吉凶災祥?天地鬼神萬物離卻我的靈明,便沒有天地鬼神萬物了。我的靈明離卻天地鬼神萬物,亦沒有我的靈明。如此便是一氣流通的,如何與他間隔得?又問:天地鬼神萬物千古見在。何沒了我的靈明,便俱無了?曰:今看死的人,他這些精靈游散了,他的天地萬物尚在何處?」(《傳習錄》卷三)。此便是一心之申展、一心之涵蓋、一心之遍潤。自道德自覺上道德實踐地所體證之本心、所擴充推致之良知靈明頓時即普而為本體宇宙論的實體,道德實踐地言之者頓時即普而為存有論地言之者。惟不先客觀地言一「於穆不已」之實體而已。而先客觀地言之、再回歸於心以實之,或兩面皆飽滿頓時即為一以言之,亦無過,此即橫渠、明道、五峰、蕺山之路也。

4.關於「乾道變化,各正性命」。天命實體之下貫於個體而具於個體即是性,此義《中庸》雖未顯明地言之,而實已函之,而顯明地表示之者則為《易傳》之(乾彖〉。宋、明儒即會通《中庸》、《易傳》而如此斷定也。《中庸》、《易傳》是一個方向(圓滿發展)之呼應,宋、明儒即如其呼應而亦存在地呼應之。《易傳》窮神知化,正式言誠體、神體、寂感真幾,此是妙運萬物之實體。濂溪即由此而開宋儒之端。此實體即曰天道,亦曰「乾道」,此仍是「於穆不已」之天命實體之別名。

「乾道變化,各正性命」,此語字面的意思是:在乾道變化底過程中,萬物(各個體)皆各得正定其性命。此語本身並不表示所正定的各個體之性命即是以理言的性命,亦可能是以氣言的性命。但首先不管是以理言的性命,抑還是以氣言的性命,此總是從「乾道變化」說下來,此即是性命之本體宇宙論的說明。此說明之方式

\newpage\thispagestyle{empty}\addtocounter{page}{-1}\vspace*{-12mm}\begin{center}\noindent
\includegraphics[clip, trim=161pt 142pt 130pt 240pt, height=162mm]{ocr-input/image-0183.png}\end{center}

\newpage

\noindent 尚未見之於《中庸》。《中庸》只表示性體與道體通而為一,未直接表示從道體之變化中說性命之正或成。但《易傳》卻直接宜明此方式。〈乾文言〉曰:「乾元者始而亨者也。利貞者性情也。」從利貞處說性情即是從個體之成處說「各正性命」也。從利貞處見個體之成,即見性情之實,亦即見性命之正。乾道之元亨利貞即表示乾道之變化。實則乾道自身並無所謂變化,乃假氣(即帶著氣化)以顯耳。乾道剛健中正,生物不測,即是一創生實體,亦即一「於穆不已」之實體。然此實體雖是一創生的實體,雖是不已地起作用,而其自身實無所謂「變化」。「變化」者是帶著氣化以行,故假氣化以顯耳。變化之實在氣,不在此實體自身也。假氣化以顯,故元亨利貞附在氣化上遂亦成四階段,因而遂儼若成為乾道之變化過程矣。然而元亨利貞亦稱乾之四德,則隨著氣化伸展出去說為四階段,亦可收攝回來附在乾道之體上說為四德也。既是體之四德,則申展出去成為四階段而顯一「變化」相,此顯是假氣以顯耳。乾道即是元,故曰「乾元」。亨者通也,此是內通。為物不貳,生物不測,於穆不已地起作用,即是內通之亨,言誠體之不滯也。利者向也,言外通也。利而至於個體之成處,即是其「貞」相,故於個體之成處見「利貞」也。否則,乾道之於穆不已只成一虛無流,已不成其為創生實體矣。故濂溪《通書·誠上第一》云:「大哉乾元,萬物資始。誠之源也。乾道變化,各正性命。誠斯立焉。」言由「各正性命」處見誠體之利貞,即見誠體之所以立,所以「立」者即誠體(乾道實體)之於此而自立自見其自己也。否則流逝無收煞。故又云:「元亨,誠之通;利貞,誠之復。」復即立也。濂溪此點撥不誤,純就體上言四德也。所謂「變化」而顯四階段者乃假

\newpage\thispagestyle{empty}\addtocounter{page}{-1}\vspace*{-12mm}\begin{center}\noindent
\includegraphics[clip, trim=168pt 133pt 140pt 254pt, height=162mm]{ocr-input/image-0187.png}\end{center}

\newpage\markright{第一部 \quad 第一章 \quad 宋明儒學之課題}

\noindent 氣以顯耳。濂溪最後又贊之曰:「大哉易也,性命之源乎?」即就「各正性命」而說也。

然則此所正之「性命」是以理言的性命,還是以氣言的性命?濂溪之贊語只表示易道是「性命之源」,未表示此性命即是以理言的性命。然通極於「體」而言性命,衡之以儒家之道德意識,此性命不會是以氣言的性命,歷來亦無人作如此理會者。是故必是正面的、超越面的、以理言的性命。當然以氣言的性命,於個體之成時,亦自然帶在氣之凝結處。然言道德實踐之先天根據(超越的根據),卻無人以此性命為氣之凝結處之氣之性命,卻必須視為超越面的理之性命。如其是理之性命,則性即是此實體之流註於個體中。實體之流註於個體中,因而個體得正其性也。正其性即是定其性,亦即成其性。此是存有論地正、定、成也。「命」即是此性之命,乃是個體生命之方向,吾人之大分,孟子所謂「分定故也」之分。此亦是橫渠所謂「天所性者通極於道,氣之昏明不足以蔽之,天所命者通極於性,遇之吉凶不足以戕之」之義也。此顯然不就氣之凝結說氣之性命也。此當是宋、明儒之共同意識,故無人認「各正性命」為氣之性命也。此亦由於《易傳》之氣氛本自如此,不會陷落下來專言氣之性命也。即使氣之性命亦帶在內,而必以正面理之性命為主也。〈說卦傳〉云:「昔者聖人之作《易》也,將以順性命之理」,即順通此「通極於道、通極於性」之性命之理也。又曰:「窮理盡性以至於命。」「窮理」即窮性命之理,「盡性」即盡以理言的性。「至於命」,則以理言的與以氣言的俱可在內。「順性命之理」即是通「性命之源」,首先必以通正面的以理言的性命之源為主也。此一說明之方式顯明地表示於《易傳》中,亦顯

\newpage\thispagestyle{empty}\addtocounter{page}{-1}\vspace*{-12mm}\begin{center}\noindent
\includegraphics[clip, trim=161pt 142pt 129pt 242pt, height=162mm]{ocr-input/image-0191.png}\end{center}

\newpage

\noindent 明地表示於《大戴禮記・本命》篇「分於道謂之命,形於一謂之性」之語句中。「分於道」即分得於道之命(命令之命),因分得此道之命乃成個體生命之方向,即吾人之大分。「形於一」即將此道之命形著之於一個體中便叫做是「性」。此亦是從正面說性命之源也。此與《易傳》為同一思理模式。大抵先秦後期儒家通過《中庸》之性體與道體通而為一,必進而從上面由道體說性體也。此即是《易傳》之階段,此是最後之圓成,故直下從「實體」處說也。此亦當作圓滿之發展看,不當視作與《論》《孟》為相反之兩途。蓋《論》、《孟》亦總有一客觀地、超越地言之之「天」也。如果「天」不向人格神方向走,則性體與實體打成一片,乃至由實體說性體,乃係必然者。此與漢人之純粹的氣化宇宙論不同,亦與西方康德前之獨斷形上學不同。此只是一道德意識之充其極,故只是一「道德的形上學」也。先秦儒家如此相承相呼應,而至此最後之圓滿,宋、明儒即就此圓滿亦存在地呼應之,而直下通而一之也:仁與天為一,心性與天為一,性體與道體為一,最終由道體說性體,道體性體仍是一。若必將《中庸》《易傳》抹而去之,視為歧途,則宋、明儒必將去一大半,只剩下一陸、王,而先秦儒家亦必只剩下一《論》、《孟》,後來之呼應發展皆非是,而孔、孟之「天」亦必抹而去之,只成一氣命矣。孔、孟之生命智慧之方向不如此枯萎孤寒也。是故儒家之道德哲學必承認其函有一「道德的形上學」,始能將「天」收進內,始能充其智慧方向之極而至圓滿。

以上是《論》《孟》、《中庸》《易傳》之相繼承與相呼應,而宋、明儒之大宗即如此圈定,認為此是孔門之傳統,圓滿之

\newpage\thispagestyle{empty}\addtocounter{page}{-1}\vspace*{-12mm}\begin{center}\noindent
\includegraphics[clip, trim=167pt 125pt 131pt 257pt, height=162mm]{ocr-input/image-0195.png}\end{center}

\newpage\markright{第一部 \quad 第一章 \quad 宋明儒學之課題}

\noindent 發展,如其呼應而亦存在地呼應之,視為一整體,直下通而一之,而不認其有隔也。此通而為一之看法合法,則《論》《孟》、《中庸》、《易傳》之主導地位自成立。此主導地位確定,則《大學》即可得而規範矣。

宋、明儒以六百年之長期,費如許之言詞,其所宗者只不過是《論》、《孟》《中庸》、《易傳》與《大學》而已,分量並不多。即此五部經典,提綱挈領,其重要語句而為宋、明儒所反覆講說者亦甚有限。就《論》《孟》《中庸》、《易傳》之通而為一而為一整體說,其義理主脈又可繫之於兩詩:

\begin{quotation}\kaishu 1.〈大雅·烝民〉:「天生烝民,有物有則。民之秉彝,好
是懿德。」

2.〈頌·維天之命〉:「維天之命,於穆不已。於乎不顯,
文王之德之純。」\end{quotation}

\noindent 前者為孟子所引以證性善,而孔子亦贊之曰「為此詩者,其知道乎?」後者為《中庸》所引,以明「天之所以為天」以及「文王之所以為文———純亦不已」。此頌詩即是天道性命通而為一之根源,此頌詩並未表示文王之「純亦不已」是以「於穆不已」之天命之體為性,然實可開啟此門。通過孔子之言仁,孟子之言本心即性,《中庸》、《易傳》即可認性體通於天命實體,並以天命實體說性體也,故此圓滿發展即可繫之於此詩,而以此詩表示之也。此兩詩者可謂是儒家智慧開發之最根源的源泉也。孟子曰「源泉混混,不舍晝夜,有本者若是。」儒家智慧之深遠以及其開發之無窮,亦可

\newpage\thispagestyle{empty}\addtocounter{page}{-1}\vspace*{-12mm}\begin{center}\noindent
\includegraphics[clip, trim=154pt 143pt 134pt 244pt, height=162mm]{ocr-input/image-0199.png}\end{center}

\newpage

\noindent 謂「有本者若是」矣。孟子引〈烝民〉之詩,是孟子言性善(本心即性)與此詩之洞悟相呼應也。《中庸》引〈維天之命〉詩,是《中庸》作者言天道誠體與此詩之洞悟相呼應也。宋明儒能相應而契悟之,通而一之,是宋明儒之生命能與此兩詩以及《論》、《孟》《中庸》、《易傳》之智慧方向相呼應,故能通而一之也。此種生命之相呼應,智慧之相承續,亦可謂「有本者若是」矣!此與佛老有何關哉?只因秦、漢後無人理解此等經典,遂淡忘之矣。至宋儒起,開始能相應而契悟之,人久昏重蔽,遂以為來自佛老矣。若謂因受佛教之刺激而豁醒可,若謂其所講之內容乃陽儒陰釋,或儒、釋混雜,非先秦儒家經典所固有,則大誣枉。無人能因受佛教之刺激而豁醒即謂其是陽儒陰釋或儒、釋混雜。焉有不接受刺激(所謂挑戰),不正視對方,而能擔當文運學運者乎?此種誣枉亦大部由於朱子之忌諱而成。汝自家內部尚且如此,則外人更津津有辭矣。實則皆吠影吠聲,未能沉下心去,正式理會此等經典之語意,故亦無生命上之呼應也。

宋、明儒之將《論》、《孟》、《中庸》、《易傳》通而一之,其主要目的是在豁醒先秦儒家之「成德之教」,是要說明吾人之自覺的道德實踐所以可能之超越的根據。此超越根據直接地是吾人之性體,同時即通「於穆不已」之實體而為一,由之以開道德行爲之純亦不已,以洞澈宇宙生化之不息。性體無外,宇宙秩序即是道德秩序,道德秩序即是宇宙秩序。故成德之極必是「與天地合其德,與日月合其明,與四時合其序,與鬼神合其吉凶,先天而天弗違,後天而奉天時」,而以聖者仁心無外之「天地氣象」以證實之。此是絕對圓滿之教,此是宋、明儒之主要課題。此中「性體」

\newpage\thispagestyle{empty}\addtocounter{page}{-1}\vspace*{-12mm}\begin{center}\noindent
\includegraphics[clip, trim=170pt 122pt 126pt 260pt, height=162mm]{ocr-input/image-0203.png}\end{center}

\newpage\markright{第一部 \quad 第一章 \quad 宋明儒學之課題}

\noindent 一觀念居關鍵之地位,最為特出。西方無此觀念,故一方道德與宗教不能一,一方道德與形上學亦不能一。彼方哲人言「實體」(reality)者多矣,如布拉得賴(F.H.Bradley)有(現象與實體》(Appearance and Reality )之作,懷悌海(N.A.Whitehead)有《歷程與實體》(Process and Reality)之作,柏格森(H.Bergson)有《創化論》(Creative Evolution)之作,近時海德格(M.Heidegger)之存在哲學又大講「存有」,有(時間與存有〉(Being and Time)之作,即羅素(B.Russell)之《邏輯原子論》(Logical Atomism)亦有其極可欣賞之風姿。大體或自知識論之路入,如羅素與柏拉圖;或自宇宙論之路入,如懷悌海與亞里士多德:或自本體論(存有論)之路入,如海德格與虎塞爾(E.Husserl);或自生物學之路入,如柏格森與摩根(L.Morgan);或自實用論(pragmatism)之路入,如杜威(J.Dewey)與席勒(F.C.S. Schiller);或自獨斷的,純分析的形上學之路入,如斯頻諾薩(Spinoza)與來布尼茲(Leibniz)及笛卡爾(Descartes)。凡此等等皆有精巧繁富之理論,讀之可以益人心智,開發玄思。然無論是講實體,或是講存有,或是講本體(substance),皆無一有「性體」之觀念,皆無一能扣緊儒者之作為道德實踐之根據、能起道德之創造之「性體」之觀念而言實體存有或本體。無論自何路入,皆非自道德的進路入,故其所講之實體、存有或本體皆只是一說明現象之哲學(形上學)概念,而不能與道德實踐使人成一道德的存在生關係者。故一方道德與宗教不能一,一方道德與形上學不能一,而無一能開出一即函宗教境界之「道德的形上學」。其中唯一例外者是康德。彼自道德的進路接近

\newpage\thispagestyle{empty}\addtocounter{page}{-1}\vspace*{-12mm}\begin{center}\noindent
\includegraphics[clip, trim=170pt 151pt 137pt 246pt, height=162mm]{ocr-input/image-0207.png}\end{center}

\newpage

\noindent 本體界,建立「道德的神學」。意志自由、靈魂不滅、上帝存在,只有在實踐理性上始有意義,始得其妥實性。然無「性體」一觀念,視「意志自由」為設準,幾使意志自由成為掛空者,幾使實踐理性自身成為不能落實者。而其所規劃之「道德的形上學」(其內容是意志自由、物自身、道德界與自然界之合一)亦在若隱若顯中,而不能全幅展示、充分作成者·黑格爾(Hegel)言精神哲學已佳矣。吾亦常借用其辭語以作詮表上之方便,如「真實主體性」、「在其自己」、「對其自己」、「具體的普遍」,等等。然此只是表示方法上之借用,非謂其哲學內容與儒者成德之教同也。彼只籠統言精神之發展,而總無「性體」一核心之觀念,故其全部哲學總不能落實,只展現而為一大邏輯學。夫理想主義(idealism)自柏克萊(Berkeley)起至黑格爾而完成,本集中於三點:一曰觀念性(ideality),二日現實性(actuality),三曰合理性(rationality),此本不錯。凡此皆見於吾之《認識心之批判》,讀之可知其詳。然如不能落實於心性,以道德實踐證實之,則總不能順適調暢,只是一套生硬之哲學理論而已。今攝之於成德之教中,點出「性體」一觀念,則一一皆實而順適調暢矣。故宋、明儒所發展之儒家成德之教,一所以實現康德所規劃之「道德的形上學」,一所以收攝融化黑格爾之精神哲學也。而同時亦是一使宗教與道德為一,一使形上學與道德為一也。此儒家智慧方向之所以為特出,而為西方道術傳統所未及。比而觀之,其眉目自朗然矣。

又,亞里士多德有 essence 一詞。此詞,通常譯為「本質」或「體性」。此似是可類比儒者所言之「性體」;然實則不類。蓋此詞若作名詞看,其實指是一「類概念」(class-concept),又是一

\newpage\thispagestyle{empty}\addtocounter{page}{-1}\vspace*{-12mm}\begin{center}\noindent
\includegraphics[clip, trim=161pt 125pt 126pt 251pt, height=162mm]{ocr-input/image-0211.png}\end{center}

\newpage\markright{第一部 \quad 第一章 \quad 宋明儒學之課題}

\noindent 方法學上之概念,可以到處應用。而儒者所言之性體則不是一類概念。即使孟子由此以言「人之所以異於禽獸者幾希」,然此幾希一點亦不是類概念,孟子說此幾希一點亦不是視作人之定義,由定義而表示出。如當作形容詞使用或當作方法學上之概念使用,則可,此如要點、本質的一點(essential point),或人之所以為人之「本質」(the essence of human being)等皆是。此性體亦可說是人之本質的一點,是人之所以為人,乃至所以為道德的存在之本質;但即以此「本質」一詞譯此「性體」,則非是。此亦如吾人亦說此性體即是吾人道德實踐(道德行為之純亦不已)之「先天根據」或「超越的根據」,但同樣不能即以先天根據或超越根據譯此「性體」一詞。此皆是詮表方法上之詞語,可以廣泛使用,俱非足以代表此「性體」一觀念也。儒者所說之「性」即是能起道德創造之「性能」;如視為體,即是一能起道德創造之「創造實體」(creative reality)。此不是一「類概念」,它有絕對的普遍性(性體無外、心體無外),惟在人而特顯耳,故即以此體為人之「性」。自其有絕對普遍性而言,則與天命實體通而為一。故就統天地萬物而為其體言,曰形而上的實體(道體 metaphysicalreality),此則是能起宇宙生化之「創造實體」;就其具於個體之中而為其體言,則曰「性體」,此則是能起道德創造之「創造實體」,而由人能自覺地作道德實踐以證實之,此所以孟子言本心即性也。(客觀地、本體宇宙論地自天命實體而言,萬物皆以此為體,即潛能地或圓頓地皆以此為性。然自自覺地作道德實踐言,則只有人能以此為性,宋、明儒即由此言人物之別。然此區別亦非定義劃類所成之類概念中本質不同之區別,故此性體非類概念中之本

\newpage\thispagestyle{empty}\addtocounter{page}{-1}\vspace*{-12mm}\begin{center}\noindent
\includegraphics[clip, trim=167pt 133pt 139pt 259pt, height=162mm]{ocr-input/image-0215.png}\end{center}

\newpage

\noindent 質也)。故此性體譯為nature 固不恰,即譯為 essence 亦不恰,其意實只是人之能自覺地作道德實踐之「道德的性能」(moralability)或「道德的自發自律性」(moral spontaneity),亦即作為「内在道德性」(inward morality)看的「道德的性能」或「道德的自發性」也。心之自律(autonomy of mind),康德所謂「意志之自律」(autonomy of will),即是此種「性」。作「體」看,即是「道德的創造實體」(moral creative reality)也。

「性體」義殊特,則「心」亦必相應此「性體」義而成立。「心」以孟子所言之「道德的本心」為標準。孟子言心具體而生動,人或以heart 一詞釋之。此若以詩人文學家之筆出之,亦未嘗不可;然就學名言,則決不可。故孟子所言之心實即「道德的心」(moral mind)也。此非血肉之心,亦非經驗的心理學的心,亦非「認識的心」(cognitive mind),乃是內在而固有的、超越的、自發、自律、自定方向的道德本心。象山言「萬物森然於方寸之中」,以「方寸」喩心,此是象徵的指點語,言萬物皆收攝於一點,豈真是視心為血肉的方寸之心耶?此一點豈真是方寸之一點耶?劉蕺山亦言「心徑寸耳」,此亦是現象學的指點語,重在以意、知、物、家、國、天下以充實之,豈真是視心為血肉的徑寸之心耶?儒者言學喜就眼前具體字眼指點,而其實義則無盡藏。是故心即是「道德的本心」,此本心即是吾人之性。如以性為首出,則此本心即是彰著性之所以為性者。故「盡其心者即知其性」。及其由「萬物皆備於我」以及「盡心知性知天」而滲透至「天道性命通而為一」一面,而與自「於穆不已」之天命實體處所言之性合一,則此本心是道德的,同時亦即是形上的。此心有其絕對的普遍性,

\newpage\thispagestyle{empty}\addtocounter{page}{-1}\vspace*{-12mm}\begin{center}\noindent
\includegraphics[clip, trim=170pt 116pt 134pt 267pt, height=162mm]{ocr-input/image-0219.png}\end{center}

\newpage\markright{第一部 \quad 第一章 \quad 宋明儒學之課題}

\noindent 為一超然之大主,本無局限也。心體充其極,性體亦充其極。心即是體,故曰心體。自其為「形而上的心」(metaphysical mind)言,與「於穆不已」之體合一而為一,則心也而性矣。自其為「道德的心」而言,則性因此始有真實的道德創造(道德行為之純亦不已)之可言,是則性也而心矣。是故客觀地言之曰性,主觀地言之曰心。自「在其自己」而言,曰性:自其通過「對其自己」之自覺而有真實而具體的彰顯呈現而言則曰心。心而性,則堯、舜性之也。性而心,則湯、武反之也。心性爲一而不二。

客觀地自「於穆不已」之天命實體言性,其「心」義首先是形而上的,自誠體、神禮、寂感真幾而表示。若更為形式地言之,此「心」義即為「活動」義(activity),是「動而無動」之動。此實體、性體,本是「即存有卽活動」者,故能妙運萬物而起宇宙生化與道德創造之大用。與《論》、《孟》通而為一而言之,即由孔子之仁與孟子之心性彰著而證實之。是故仁亦是體,故曰「仁體」;而孟子之心性亦是「即活動即存有」者。

以上由《論》《孟》《中庸》、《易傳》通而為一以言宋、明儒之主要課題為成德之教,並言其所弘揚之成德之教之殊特。此下再就宋明儒之發展以言其分系。

\section{宋、明儒之分系}

以上言通而為一,是就宋、明儒總持地言之,並由《論》、《孟》、《中庸》、《易傳》之發展以明其通而為一為合法。然此通而為一亦不是開始時即如此。又先秦儒家是由《論》、《孟》發

\newpage\thispagestyle{empty}\addtocounter{page}{-1}\vspace*{-12mm}\begin{center}\noindent
\includegraphics[clip, trim=164pt 134pt 146pt 259pt, height=162mm]{ocr-input/image-0223.png}\end{center}

\newpage

\noindent 展至《中庸》與(易傳》,而北宋諸儒則是直接由《中庸》《易傳〉之圓滿頂峰開始漸漸向後返,返至於《論》、《孟》。人不知其通而為一之背景,遂以為北宋諸儒開始,是形而上學的意味重,似是遠離孔孟實踐之精神。固是形而上學,然卻是先秦儒家發展至《中庸》、《易傳》所本有之「道德的形上學」,固以《論》、《孟》為底據,非是空頭的「知解形上學」(teoreticalmetapysics)。惟因自此圓滿頂峰開始,一時或未能意識及,然其不自覺的背景固以通而為一為其底據也。例如濂溪對於「天道性命通而為一」一面,雖言之而略,亦有不盡處,(如言性自剛柔中而言),然此脈絡則固已顯出;而對於誠體、神體、寂感真幾則體會的極精透,太極真體亦不能外乎此。惟對於《論》、《孟》則所知甚少,至少亦未能甚注意。然彼亦云「聖人定之以中正仁義,主靜而立人極焉。」並用〈洪範〉之「思曰睿,睿作聖」而言:「無思本也,思通用也。幾動於此,誠動於彼。無思而無不通為聖人。」是則仍以實踐與聖證為根據,並未空頭言形上學。蓋其言誠體本無內外之隔也。

橫渠對於「天道性命通而為一」,言之極為精透;盛言「知虛空即氣,則有無、隱顯、神化、性命,通一無二」。首以儒家「本天道為用」之真實无妄、充實飽滿、體用不二之宇宙觀,對治佛家之緣起性空、如幻如化。此種沈雄弘偉之大手筆實不可輕侮。然而「聖人盡道其間,兼體而不累者,存神其至矣」,則亦未嘗憑空猜測料度、構畫一套外在的知解的形上學。其對於《論》、《孟》已甚能注意,不似濂溪之全未能注意,卻只轉而借用〈洪範〉語以言心。其言「天體物不遺,猶仁體事無不在」,又言「仁以敦化為

\newpage\thispagestyle{empty}\addtocounter{page}{-1}\vspace*{-12mm}\begin{center}\noindent
\includegraphics[clip, trim=170pt 116pt 126pt 260pt, height=162mm]{ocr-input/image-0227.png}\end{center}

\newpage\markright{第一部 \quad 第一章 \quad 宋明儒學之課題}

\noindent 深,化行則顯」,又言「敦篤虛靜者仁之本」,「無所繫閡昏塞,則是虛靜也」,由此可見其對於「仁體」體會之深,仁體感通之無局限已甚顯。其〈大心篇〉之言心顯有本於孟子。其對於主觀面仁與心性之注意顯已不弱。「心能盡性,人能弘道也。性不知檢其心,非道弘人也。」此語即足證其對於心之重視。其言「兼體無累」、「參和不偏」、「性其總合兩也」,又言繼善成性、盡心易氣以成性,此皆表示已回歸於《論》、《孟》,以主觀面統攝客觀面。然此一面之義理為其言太和太虛、言神言氣所掩蓋,人不易見,遂令人感覺其言主觀面,比之其言客觀面,比重猶嫌輕,不免使人有虛歉之感。是亦由《中庸》《易傳》向後返之勢然也,然其實亦並無虛歉也。

至明道則兩方面皆飽滿,無遺憾矣。明道不言太極,不言太虛,直從「於穆不已」、「純亦不已」言道體、性體、誠體、敬體。首挺立「仁體」之無外,首言「只心便是天,盡之便知性,知性便知天,當下便認取,更不可外求」,而成其「一本」之義。是則道體、性體、誠體、敬體、神體、仁體,乃至心體,一切皆一。故真相應先秦儒家之呼應而直下通而為一之者是明道。明道是此「通而一之」之造型者,故明道之「一本」義乃是圓教之模型。從濂溪、橫渠而至明道是此回歸之成熟,兩方皆挺立而一之,故是圓教之造型者。此圓教之造型亦是宋、明儒學之所以為新,此是順先秦儒家之呼應直下通而一之,調適上遂之新。如果有可以使吾人感到宋、明儒之理境有與先秦儒家不相似處,首先當從此本質的圓教之意義上去想,不可浮光撂影,從枝末點滴上去妄肆譏議也。至於造詣、意味、氣象,則是主觀的事,隨時有不同,自不會全同,亦

\newpage\thispagestyle{empty}\addtocounter{page}{-1}\vspace*{-12mm}\begin{center}\noindent
\includegraphics[clip, trim=161pt 137pt 144pt 255pt, height=162mm]{ocr-input/image-0231.png}\end{center}

\newpage

\noindent 不必能及先秦之儒家,此不必言。

由濂溪、橫渠而至明道,此為一組。此時猶未分系也。

義理間架至伊川而轉向。伊川對於客觀言之的「於穆不已」之體以及主觀言之的仁體、心體與性體似均未能有相應之體會,不同於前三家,亦不能與先秦儒家之發展相呼應。他把「於穆不已」之體(道體)以及由之而說的性體只收縮提練,清楚割截地視為「只是理」,即「只存有而不活動」的理,(明道亦說理或天理,但明道所說的天理是就其所體悟的「於穆不已」之體說,廣之,是就其所體悟的道體、性體、誠體、敬體、神體、仁體、心體皆一說,是即存有即活動者。)他把孟子所說的「本心即性」亦拆開而為心性情三分:性亦只是理,性中只有仁義禮智,仁義禮智亦只是理;仁性愛情,惻隱羞惡等亦只是情;心是實然的心氣,大體是後天心理學的心,心與性成為後天與先天、經驗的與超越的、能知與所知的相對之二。心發而為情,心亦有兩個重要的觸角:一是後天的偶然的收歛凝聚,由此說敬、說涵養;一是心知之明,由此說致知格物。孔子的仁亦只是理,以公說仁,公而以人體之便是仁。此全部與其老兄所體會者不同,實體性體只是存有論的理,而心與性不能一自此始。工夫之重點落在大學之致知格物上,總之是「涵養須用敬,進學則在致知」。此即喪失《論》《孟》《中庸》、《易傳》通而為一之境以及其主導之地位,而居主導之地位者是《大學》。彼有取於《中庸》、《易傳》者只是由之將道體提練而為一個存有論的理,彼所取於《論》、《孟》者亦只是將仁與性提練而為理,而心則沈落與傍落。此一套大體是實在論的心態,順取之路,與前三家遠矣,亦與先秦儒家《論》《孟》、《中庸》、

\newpage\thispagestyle{empty}\addtocounter{page}{-1}\vspace*{-12mm}\begin{center}\noindent
\includegraphics[clip, trim=172pt 116pt 125pt 260pt, height=162mm]{ocr-input/image-0235.png}\end{center}

\newpage\markright{第一部 \quad 第一章 \quad 宋明儒學之課題}

\noindent 《易傳》之相呼應遠矣。此一系統為朱子所欣賞、所繼承,而且予以充分的完成。此一系統,吾名之曰主觀地說是靜涵靜攝之系統,客觀地說是本體論的存有之系統,總之是橫攝系統,而非縱貫系統,此方是有一點新的意味,此是歧出轉向之新,而非調適上遂之新。此是以荀子之心態講孔子之仁,孟子之心與性,以及《中庸》、《易傳》之道體與性體,只差荀子未將其所說之禮與道視為「性理」耳。此自不是儒家之大宗,而是「別子為宗」也,此一系統因朱子之強力,又因其近於常情,後來遂成為宋、明儒之正宗,實則是以別子為宗,而忘其初也。

但南渡後,胡五峰是第一個消化者。五峰倒卻是承北宋前三家而言道體性體,承由《中庸》《易傳》回歸於《論》《孟》之圓滿發展,即承明道之圓教模型,而言以心著性、盡心成性,以明心性之所以為一為圓者。明道只是圓頓地平說,而五峰則先心性分設,正式言心之形著義,以心著性而成性,以明心性之所以一。心即孔子之仁、孟子之本心也。性即由「於穆不已」之體而言者也。故言「性天下之大本」,「性也者天地所以立也」,「性也者天地鬼神之奧也」,「誠成天下之性,性立天下之有」;而于心,則言永恆而遍在,「心也者知天地宰萬物以成性者也」,「仁者人所以肖天地之機要也」,「聖人傳心,教天下以仁也」,聖人「盡心者也,故能立天下之大本」;而於工夫,則重在「先識仁之體」,重在當下指點以求其放失之心,正式言「逆覺體證」以復其本心以為道德實踐之本質的關鍵、正因的工夫,此與伊川、朱子之順取之路根本有異,不落於大學之致知格物言也。此一系統無論是「以心著性」一面,或是「逆覺體證」一面,皆是直承明道之圓教而開出。

\newpage\thispagestyle{empty}\addtocounter{page}{-1}\vspace*{-12mm}\begin{center}\noindent
\includegraphics[clip, trim=163pt 134pt 139pt 255pt, height=162mm]{ocr-input/image-0240.png}\end{center}

\newpage

\noindent 宋、明儒中最後一個消化者劉蕺山亦是此路。北宋三家後,一頭一尾,兩人相隔如此其遠,然而不謀而合,亦云奇矣!(劉蕺山從未提過胡五峰)。惟五峰之學為朱子所不契,作〈知言疑義〉以疑之;張南軒隨朱子腳跟轉,不能弘揚其師學;堅守五峰之說而不捨者如胡廣仲、胡伯逢、吳晦叔、彪居正等,又皆作品不存,年壽不永,學力才力恐亦有所不及,皆為朱子所駁斥。是則五峰所開之湖湘學統,為朱子所掩蓋,人亦淡忘之,而不知其實蘊矣。然而吾人今日重讀《知言》,並順朱子之駁斥尋胡廣仲、胡伯逢、吳晦叔、彪居正等人之思理,則知此一系實是承明道、上蔡而來者。以五峰為準,其實義實是承明道之圓教模型而開出者,故吾正式列胡五峰與劉蕺山為一系,承認其有獨立之意義。

朱子雖將五峰系壓下,然其實在論的心態、歧出之轉向、順取之工夫入路,皆不為象山所許可。象山從《論》《孟》入手,純是孟子學,只是一心之朗現、一心之申展、一心之遍潤,是真能相應「夫子以仁發明斯道,其言渾無罅縫;孟子十字打開,更無隱遁」而開學脈者,故亦能恰當地說出此語。象山對於北宋諸家未曾多下工夫,亦不是承明道而開出,尤其不喜伊川。他根本不是順北宋前三家「由《中庸》《易傳》回歸於《論》《孟》」之路走,他是讀《孟子》而自得之,故直從孟子入,不是由明道之圓教而開出。他之特喜孟子,也許由於其心態使然,也許由於當時有感於朱子學之歧出與沉落(轉向)而豁醒,而更加重其以孟子學為宗旨。象山比朱子少九歲,鵝湖之會時,象山三十七歲,宗旨已定,而朱子四十六歲,已經過與湖湘系之奮鬥而早成熟。朱子在未與象山會面前即因夙聞而有禪之聯想,故自始即斥其為禪,後來更甚。此其

\newpage\thispagestyle{empty}\addtocounter{page}{-1}\vspace*{-12mm}\begin{center}\noindent
\includegraphics[clip, trim=161pt 108pt 125pt 265pt, height=162mm]{ocr-input/image-0244.png}\end{center}

\newpage\markright{第一部 \quad 第一章 \quad 宋明儒學之課題}

\noindent 不相契可知。此顯然是歪曲與誣枉。此與禪根本無關,問題只是伊川朱子對於先秦儒家由《論》、《孟》至《中庸》《易傳》之呼應不能有生命感應上之呼應也。吾人今日當從此著眼而觀象山之孟子學,不當再順朱子之聯想而下滾。後來陽明承象山之學脈而言致良知,亦仍是孟子學之精神,人隨朱子之聯想,吠聲吠影,更視之為禪矣。實則問題只是以《論》、《孟》、《中庸》《易傳》為主導,抑還是以《大學》為主導。時過境遷,不應再有無謂之忌諱。故問題之真相可得而明矣。

象山與陽明只是一心之朗現、一心之申展、一心之遍潤,故對於客觀地自「於穆不已」之體言道體性體者無甚興趣,對於自客觀面根據「於穆不已」之體而有本體宇宙論的展示者尤無多大興趣。此方面之功力學力皆差。雖其一心之遍潤,充其極,已申展至此境,此亦是一圓滿,但卻是純從主觀面申展之圓滿,客觀面究不甚能挺立,不免使人有虛歉之感。自此而言,似不如明道主客觀面皆飽滿之「一本」義所顯之圓教模型為更為圓滿而無憾。蓋孔子與孟子皆總有一客觀而超越地言之之「天」也。此「天」字如不能被擯除,而又不能被吸納進來,即不能算有真實的飽滿與圓滿。是則《中庸》《易傳》之圓滿發展當係必然者,明道之直下通而一之而鑄造圓教之模型亦當是必然者,而由此圓教模型而開出之「以心著性」義(五峰學與蕺山學)亦當是必然者。自象山陽明言,則不須要有此回應,但承明道之圓教模型而言,則應有此回應以明其所以爲一為圓,以真實化其「一本」與圓滿。自此而言,象山、陽明之一心遍潤、一心申展,始真有客觀的落實處,而客觀地挺立矣。自此而言,五峰、蕺山與象山、陽明是一圓圈的兩來往:前者是從客

\newpage\thispagestyle{empty}\addtocounter{page}{-1}\vspace*{-12mm}\begin{center}\noindent
\includegraphics[clip, trim=157pt 135pt 139pt 254pt, height=162mm]{ocr-input/image-0248.png}\end{center}

\newpage

\noindent 觀面到主觀面,而以主觀面形著而真實化之;後者是從主觀面到客觀面,而以客觀面挺立而客觀化之。兩者合而為宋、明儒之大宗。皆是以《論》《孟》《中庸》《易傳》為主導也。若分別言之,則五峰與蕺山是由濂溪橫渠而至明道所成之圓教模型之嫡系,而象山與陽明則只是孟子學之深入與擴大也。如不能把孔、孟之「天」擯除之,則《中庸》(易傳》之圓滿發展為合法者,明道之圓教模型亦合法者,五峰、蕺山之「以心著性」之回應亦是合法者。如不能斷此為歧途,則此兩系最好視為一圓圈之兩來往,須知在成德之教中,此「天」字之尊嚴是不應減殺者,更不應抹去者。如果成德之教中必函有一「道德的形上學」,則此「天」字亦不應抹去或減殺。須知王學之流弊,即因陽明於此處稍虛歉,故人提不住,遂流於「虛玄而蕩」或「情識而肆」,蕺山即於此著眼而「歸顯於密」也。(此為吾之判語)此為內聖之學自救之所應有者。(以博學事功來補救、相責斥,則為離題。)而象山於此稍虛歉,故啟朱子之責斥,而復不能順通朱子之蔽而豁醒之也。

依以上之疏通,宋、明儒之發展當分為三系:

1.五峰、蕺山系:此承由濂溪、横渠,而至明道之圓教模型(一本義)而開出。此系客觀地講性體,以《中庸〉、《易傳〉為主,主觀地講心體,以《論》、《孟》為主。特提出「以心著性」義以明心性所以為一之實以及一本圓教所以為圓之實。於工夫則重「逆覺體證」。

2.象山、陽明系:此系不順「由《中庸》、《易傳〉回歸於《論》、《孟》」之路走,而是以《論》《孟》攝《易》、《庸》而以《論》、《孟》為主者。此系只是一心之朗現一心之

\newpage\thispagestyle{empty}\addtocounter{page}{-1}\vspace*{-12mm}\begin{center}\noindent
\includegraphics[clip, trim=171pt 110pt 127pt 271pt, height=162mm]{ocr-input/image-0252.png}\end{center}

\newpage\markright{第一部 \quad 第一章 \quad 宋明儒學之課題}

\noindent 申展、一心之遍潤;於工夫,亦是以「逆覺體證」為主者。

3.伊川、朱子系:此系是以《中庸》《易傳》與《大學》合,而以《大學》為主。於《中庸》、《易傳》所講之道體性體只收縮提練而為一本體論的存有,即「只存有而不活動」之理,於孔子之仁亦只視為理,於孟子之本心則轉為實然的心氣之心。因此,於工夫特重後天之涵養(「涵養須用敬」)以及格物致知之認知的橫攝(「進學則在致知」),總之是「心靜理明」,工夫的落實處全在格物致知,此大體是「順取之路」。

以上1.2.兩系以《論》、《孟》、《易》《庸》為標準,可會通而為一大系,當視為一圓圈之兩來往:自《論》《孟》滲透至《易》《庸》,圓滿起來,是一圓圈;自《易》《庸》回歸於《論》、《孟》,圓滿起來,仍是此同一圓圈,故可會通為一大系。此一大系,吾名曰縱貫系統。伊川朱子所成者,名曰橫攝系統。故終於是兩系。前者是宋、明儒之大宗,亦合先秦儒家之古義;後者是旁枝,乃另開一傳統者。此第三系,若自「體」上言,則根本有偏差;順其義而成之,則亦可說是轉向,即轉成本體論的存有之系統(system of ontological being)。若自工夫言之,涵養與致知亦有補充助緣之作用,因吾人亦總有後天之心也,此亦須涵養之敬以收斂凝聚之,以使之常清明,此於道德實踐之稱體而行(純依本心性體而行)亦有助緣之作用,但「致知」方面則須有簡別。依伊川朱子,致知是通過格物知那作為「本體論的存有」的超越之理,並不是一般的經驗知識。自此而言,照顧到實然的心氣,則其所成者是主智主義之以知定行,是海德格所謂「本質倫理」,是康德所謂「他律道德」,此則對儒家之本義言根本為歧

\newpage\thispagestyle{empty}\addtocounter{page}{-1}\vspace*{-12mm}\begin{center}\noindent
\includegraphics[clip, trim=156pt 125pt 137pt 257pt, height=162mm]{ocr-input/image-0256.png}\end{center}

\newpage

\noindent 出、為轉向,此處不能說有補充與助緣之作用。但因其在把握超越之理之過程中須通過「格物」之方式,在格物方式下,人可拖帶出一些博學多聞的經驗性的知識,此則於道德實踐有補充助緣之作用。但此非伊川、朱子之主要目的,但亦未能十分簡別得開,常混在一起說。是即所謂「道問學」之意也。是則可以作為道德實踐之補充與助緣的經驗知識(科學性的知識)。問題在伊川、朱子猶未能與把握「超越之理」十分簡別得開,因而亦未能自覺地使之挺立得起。吾人今日可以分別看,就其目的在把握超越之理方面說,此於道德實踐(成德之教)根本為歧出、為轉向;就其所隱函之對於經驗知識之重視言,此處之「致知」即可視為道德實踐之補充與助緣。知識問題之引發在宋、明儒中猶未得其積極之解決,蓋其主要課題本是成德之教,不在知識問題也。吾人所以不視伊川、朱子學為儒家之正宗,為宋、明儒之大宗,即因其一、將知識問題與成德問題混雜在一起講,既於道德為不澈,不能顯道德之本性,復於知識不得解放,不能顯知識之本性;二、因其將超越之理與後天之心對列對驗,心認知地攝具理,理超越地律導心,則其成德之教固應是他律道德,亦是漸磨漸習之漸教,而在格物過程中無論是在把握「超越之理」方面或是在經驗知識之取得方面,一是皆成「成德之教」之本質的工夫,皆成他律道德之漸教之決定的因素,而實則經驗知識本是助緣者。(助緣、補充之義,象山、陽明皆表示得很清楚,非抹殺道問學也。然在伊川、朱子則成本質的,此即所以為歧出、為支離。就把握超越之理方面說,是根本上的歧出與轉向;就經驗知識之取得方面說,是枝末上的歧出與支離)。

普通只知宋、明儒有兩系,曰程、朱,曰陸、王,未嘗有說三

\newpage\thispagestyle{empty}\addtocounter{page}{-1}\vspace*{-12mm}\begin{center}\noindent
\includegraphics[clip, trim=167pt 107pt 121pt 266pt, height=162mm]{ocr-input/image-0260.png}\end{center}

\newpage\markright{第一部 \quad 第一章 \quad 宋明儒學之課題}

\noindent 系者。此是因為視朱子足以繼承北宋四家,而象山純是孟子學,不從北宋四家入手也。關於此點,吾人以為象山固不從北宋四家入手,而朱子亦並不真能繼承北宋四家也。其所真能繼承者亦只伊川而已。是以進一步,普通所以只認有兩系者,是順朱子以伊川吞沒明道,以為二程差不多,固只是一系統,即或感到明道稍有不同,或如朱子亦常示其對於明道不滿,然未知其義理之實,亦不覺其「不同」有如何嚴重之影響,朱子亦未知其義理之實,亦不知其所認為不滿者究是明道之義理之實即如此,抑或只是一時之「渾淪」與「太高」,無關於義理之實也。如是,明道乃成隱形者,義理之實全在伊川,以伊川概括二程,以為伊川即足以代表二程矣。又因為《二程遺書》中之紀錄語並無編次類聚,又屬於二先生語者又大都未分別開,不能確定是誰語,如是,人簡別為難,只順朱子所講習及者作了解,而朱子之講習固汰濾甚多,其理解亦只以伊川之思理為標準,如是,遂只以伊川代表二程矣。實則伊川並不足以代表明道,明道固有其義理之實,混稱二先生語者亦可以簡別得開,亦大體可以決定是誰語,無類聚者亦可以耐心類聚之,如是,兩系統之異,其眉目固甚顯然也。吾於此確費極大的工夫,乃見出明道確不應與伊川混而為一,明道確應與濂溪、橫渠合為一組,而為《論》《孟》《中庸》《易傳》通而一之之圓教底造型者。如是,吾人不應稱程、朱,只應稱伊川、朱子;即為與陸王對言,而稱程朱,心中亦應記住是伊川之程,非明道之程。如是,由明道之圓教模型,吾人很易看出其所開出者是五峰學,而不是朱子學,是則應有三系乃必然者。朱子雖大講(太極圖說〉,然實以伊川之思理理解太極,故對於太極真體理解有偏差,即理解為「只是

\newpage\thispagestyle{empty}\addtocounter{page}{-1}\vspace*{-12mm}\begin{center}\noindent
\includegraphics[clip, trim=160pt 128pt 132pt 254pt, height=162mm]{ocr-input/image-0264.png}\end{center}

\newpage

\noindent 理」,「只存有而不活動」者,蓋對於其所言之誠體、神體寂感真幾,無相應之體會故也,是則朱子對於濂溪所默契之道妙根本不能有相應也。至於其對於橫渠隔閔尤甚,是即其並未真能繼承北宋四家也。然以吾人觀之,濂溪、橫渠與明道實為一組,雖前二人一言太極,一言太虛,而明道俱不言,然而皆言誠體、神體、寂感真幾,則一也,皆能相應《中庸》、《易傳》所表示之創生實體、即活動即存有之實體,則一也。對此實體,雖有種種詞語,實皆表示此「於穆不已」之天命實體,故明道即由「於穆不已」體會之也,雖不言太極亦無傷,亦未嘗不可言也。雖對於橫渠之太虛神體有誤會,然誤會總是誤會也。是故對於道體之體會,彼三人者實相同,只明道能直下就《論》、《孟》《中庸》《易傳》通而一之而鑄造其圓教一本義,斯則為特殊耳,此非朱子所知也,只以渾淪、太高視之矣。是故將明道從與伊川混一中剔剝得開,以濂溪橫渠爲之先河,視為圓教之造形者,而以伊川為轉向之開始者,則明道開五峰,伊川開朱子,加上陸、王,應有三系,亦顯然矣,此亦自然之序也。觀朱子之疑《知言》,又力關五峰之後學,又力駁上蔡之「以覺訓仁」以及以「物我為一」說仁者,則其對於明道之不滿,猶不只是因其渾淪太高而已也,其義理之實不亦因此而躍然可見乎?只因湖湘學統已為朱子所壓伏,後世無傳,而朱子對於明道則為賢者諱,隱而不提,人遂不知五峰學之重要與特殊,故亦淡忘之,遂不知應有三系,而以為只有兩系矣,而明道亦成隱形者。吾詳簡《二程遺書〉,明道、伊川各有編次,又詳疏朱子與五峰系辨駁之奮鬥,則五峰學之殊特與淵源已朗然在目,而明道之「義理之實」亦脫穎而出矣。中國前賢對於品題人物極有高致,而對於義理

\newpage\thispagestyle{empty}\addtocounter{page}{-1}\vspace*{-12mm}\begin{center}\noindent
\includegraphics[clip, trim=171pt 117pt 133pt 264pt, height=162mm]{ocr-input/image-0268.png}\end{center}

\newpage\markright{第一部 \quad 第一章 \quad 宋明儒學之課題}

\noindent 形態之欣賞與評詁則顯有不及,此固由於中國前賢不甚重視義理系統,然學術有淵源,則系統無形中自亦隨之。《宋元學案》對於各學案之歷史承受,師弟關係,耙疏詳盡,表列清楚,然而對於義理系統則極乏理解,故只堆積材料,選錄多潦草不精當,至於詮表,則更缺如。

《宋元學案》卷二十九,〈震澤學案·序錄〉云:

\begin{quotation}\kaishu 信伯極為龜山所許,而晦翁最貶之,其後陽明又最稱之。予
讀《信伯集》,頗啟象山之萌芽。其貶之者以此,其稱之者
亦以此。象山之學本無所承,東發以為遙出於上蔡,予以為
兼出於信伯。蓋程門已有此一種矣。\end{quotation}

\noindent 此所云信伯即王蘋、字信伯也。

又卷五十八,〈象山學案·序錄〉云:

\begin{quotation}\kaishu 祖望謹案:象山之學先立乎其大者,本乎孟子,足以砭末俗
口耳支離之學。[……程門自謝上蔡以後,王信伯、林竹
軒、張無垢,至於林艾軒,皆其前茅。及象山而大成,而其
宗傳亦最廣。\end{quotation}

\noindent 如此追溯,見歷史上有氣味相近者則可,若謂象山「遙出於上蔡」,「兼出於信伯」,並謂「信伯、竹軒無垢、艾軒皆其前茅」,則此種強拉關係甚屬無謂,適足以蒙蔽義理系統與形態之真相。夫象山之學本無師承,乃讀孟子而自得之。象山自己表明如

\newpage\thispagestyle{empty}\addtocounter{page}{-1}\vspace*{-12mm}\begin{center}\noindent
\includegraphics[clip, trim=164pt 129pt 129pt 253pt, height=162mm]{ocr-input/image-0272.png}\end{center}

\newpage

\noindent 此,全祖望已知之矣,而又謂其源出於上蔡、信伯,何耶?象山對於北宋四家並未多加鑽研工夫,亦不走由《中庸》《易傳》回歸於《論》、《孟》之路,故象山不由明道開出,明道亦不開象山。若謂其源出於上蔡與信伯,何不直謂其源出於明道?不能謂其出於明道,則亦不能謂其出於上蔡與信伯。「程門已有此一種」是因明道圓教一本之義本有甚飽滿之言仁言心也,此與象山有相近處,然不能因此即謂象山源出於此也。順程門言者,是明道學之所開;直從孟子入者是象山學之特色。學脈之來歷、義理系統之型態,不可混濫也。(近人或有謂明道開象山,其同處是混形而上下不分,只是一個世界,此皆門外恍惚之妄言)。是則由《中庸〉、《易傳》回歸於《論》、《孟》,直下通而一之而言「一本」,以成圓教之模型,是明道學;由此開五峰之「以心著性」義,此為五峰蕺山系。直從孟子入,只是一心之申展,則是象山之圓教,此為象山、陽明系。北宋自伊川開始轉向,不與濂溪、橫渠、明道為一組,朱子嚴格遵守之,此為伊川朱子系。伊川是《禮記》所謂「別子」,朱子是繼別子為宗者。五峰、蕺山是明道之嫡系。濂溪、橫渠明道為一組,是直就《論》、《孟》《中庸》《易傳〉通而一之,從客觀面入手以成其為調適上遂之「新」者;象山、陽明是直以《論》、《孟》攝《易》、《庸〉,是從主觀面入手以成其為調適上遂之「新」者。此是宋、明儒之大宗,亦是先秦儒家之正宗也。蓋皆以《論》、《孟》《中庸》、《易傳》為主導者也。

言至此,人或覺吾此書似有貶視朱子之意。曰:非是貶視,乃如欲恰如其分而還其本來面目,則固自如此耳。吾謂伊川、朱子始

\newpage\thispagestyle{empty}\addtocounter{page}{-1}\vspace*{-12mm}\begin{center}\noindent
\includegraphics[clip, trim=170pt 116pt 128pt 264pt, height=162mm]{ocr-input/image-0276.png}\end{center}

\newpage\markright{第一部 \quad 第一章 \quad 宋明儒學之課題}

\noindent 真有點新的意味,而又恰似荀子之對孔、孟而為新,實因其所成之橫攝系統與先秦儒家所原有及宋、明儒大宗所弘揚之縱貫系統為不合。吾如此表示決非隨便說出者。吾初未嘗不欲以朱子為標準。朱子註遍群經,講遍北宋諸家。象山、陽明等人未作此工作,吾人以為朱子對於先秦儒家經典、於基本義理處必有相應,決不會有太大的出入。至少亦可以繼承北宋四家而為正宗。象山、陽明固有獨特之凸出,朱子以及朱子之後學斥其為禪固是過分,然雙方之爭論似亦無多大意義。朱子亦未嘗不尊德性,亦未嘗無「心之德」、「心具衆理」、「心理合一」、「無心外之法」等語句與議論。象山、陽明亦未嘗不重學、不處事、不讀書。雖未章句註解、考訂文獻,然何必人人都作同樣工作?道問學亦不必定在某一形態也。是則其爭論實可不必,而亦不必是兩系統之異。象山、陽明固不必為異端,而伊川、朱子亦未必不能相應先秦儒家之舊義而為大宗也。然而仔細一想,認真去處理內部之義理問題,則並不如此簡單,亦決不如此僱侗。其爭論實非無意義,亦非只門戶意氣之爭。不管以前自覺不自覺,或自覺到如何之程度,其中實有義理之根本差異處,而有足以令其雙方講不來而終於為兩路者,此非只是同一觀念而有不同之言詞表示,因而只為言詞之滯之問題也。

吾人若以朱子為標準,根據其講法去理解先秦舊典,則覺其講法於基本義理處實不相應。首先,彼以「心之德、愛之理」之方式去說仁,實不能盡孔子所說之仁之實義;彼以「心、性情三分」之格局去理解孟子,尤與孟子「本心即性」之本心義不相應;彼以「理、氣二分」之格局去理解《中庸》、《易傳》「生物不測」之天道、神體,乃至誠體,尤覺睽違重重。總之,彼之心態似根本不

\newpage\thispagestyle{empty}\addtocounter{page}{-1}\vspace*{-12mm}\begin{center}\noindent
\includegraphics[clip, trim=162pt 140pt 136pt 251pt, height=162mm]{ocr-input/image-0280.png}\end{center}

\newpage

\noindent 宜於講《論》《孟》、《中庸》與《易傳》,彼似對於由〈烝民〉詩所統繫之心、性、仁一面與(維天之命〉詩所統繫之「於穆不已」之天命之體一面根本不能有生命、智慧上之相呼應。惟一相應者是《大學》。雖不必合《大學》之原義,然畢竟是相應者。此因《大學〉在基本方向上並不明確故也。

降而至於北宋,彼對於濂溪之誠體、神體並無相應之契悟,因而對於太極之理解亦有偏差。彼對於橫渠,因二程未能了解橫渠「太虛神體」之思理,彼亦隨之而更隔閔太甚。彼對於明道本不相契,且亦不滿,然而常為之諱,或只以程子儷侗之,而歸其實於伊川,是則明道在朱子之傳承下只成為隱形的,彼似對於其妙悟道體根本未理會也。然則普通所謂「程、朱」實只是伊川、朱子也。以伊川之程子概二程非是。以伊川為主之二程再概括濂溪與橫渠尤非是。然則以為一言程、朱,即可示朱子上通北宋四家而為正宗,未盡其實。朱子真能相應者唯一伊川耳。伊川、朱子其義一也。

由上觀之,吾人不能以朱子為標準甚明。然朱子註遍羣經,講遍各家,其所反映投射之顏色沾滿一切,吾人雖不能以之為標準,實不能不以之為中心(焦點)。吾之整理疏解北宋四家與朱子實煞費精力。欲想將朱子所反映投射之顏色剔剝得開而物各付物,還其本來面目,此工作實太艱鉅。然而「求是」之心之不容已實逼迫我非如此進去不可。弄不明白,不得一諦解,實無法下手講此期之學術。如普通隨便征引幾句,隨文領義,都差不多,總無必然。此實非心之所能安。無以對北宋四家,亦無以對朱子。吾乃決心進去,予以剔剝。先整理《二程遺書》,分別編錄明道語與伊川語而確定之,凸顯明道,使其從隱形的轉為顯形的,於朱子之不解處正

\newpage\thispagestyle{empty}\addtocounter{page}{-1}\vspace*{-12mm}\begin{center}\noindent
\includegraphics[clip, trim=162pt 113pt 127pt 263pt, height=162mm]{ocr-input/image-0284.png}\end{center}

\newpage\markright{第一部 \quad 第一章 \quad 宋明儒學之課題}

\noindent 之。次對於濂溪之《通書》若干章及(太極圖說〉予以確定之疏解,而同時亦指出朱子理解之偏差,而於朱子之解語亦予以確定之詮表。次對於橫渠之《正蒙》若干篇予以確定之疏解,消除其滯辭,呈露其實義,於朱子之誤解處正之。次對於伊川予以確定之疏解,以明其為系統轉向之開始,朱子於伊川之理解大抵皆是,無可指議者。最後詳編朱子語,以中和問題與(仁說〉之辨論為中心,展開其各方面之牽連,展示其全部系統之何所是。關於朱子部,分量最多,工作亦繁重。然握其要,則其思理亦很清楚。所謂「握其要」,在客觀了解之過程上,並非是憑空從一點(譬如從〈格物補傳〉或從心之德愛之理或從敬貫動靜等)展轉引申其他。如朱子系統之成是探取西方哲學家立論之方式而形成其系統,則自可如此握住其一點,即可了解其系統之全部。然朱子並非如此者,乃是由遍註羣經、講遍北宋四家而形成其系統者。是故其要點之確義頗不易握,其思理之清楚亦不易凸顯。人初見之,或稍有深入而不能究竟,則很可以覺其為一團混雜,衝突百出,矛盾重重。然而此皆是假象,其底子固甚清晰,而其思理亦甚一貫,而且皆能充其極。此其所以為大家,而足以開創一傳統者。是故在客觀了解上,其要點確義之把握,其清晰思理之朗現,必須在比對剔剝中而把握而朗現,如是,始可得其必然而不搖蕩。吾此辦法亦可以說是堅壁清野之辦法。將其所反映投射之顏色一一剔剝得開,先將外部釐清,如是,則雙方之眉目朗然矣。雖所涉甚廣,言辭甚繁,然主要論點(關鍵處)亦並不多。列舉之,不過如下:

1.對于孟子心、性、情、才之理解;

2.對于孟子盡心知性之理解;

\newpage\thispagestyle{empty}\addtocounter{page}{-1}\vspace*{-12mm}\begin{center}\noindent
\includegraphics[clip, trim=167pt 152pt 128pt 240pt, height=162mm]{ocr-input/image-0288.png}\end{center}

\newpage

3.對于《中庸》中和之理解;

4.對於濂溪誠體、神體與太極之理解;

5.對于橫渠離明得施不得施之理解,以及對于(大心篇〉之理
解;

6.對於明道「其體則謂之易,其用則謂之神」之理解,以及其
對於其言仁之理解。

\noindent 凡此理解皆不相應者。於此等處,朱子所以必如此講,固可見其思理之何所是,而於其不相應者亦可知其所講者原義之何所是,此即所謂對比剔剝、堅壁清野之辦法也。必如此而後可以全盡,而雙方之義理系統亦朗然在目矣。此而釐清,則其必遵守伊川之思理而前進,乃係必然者;其不契不滿於明道,隱略而為之諱,亦必然者;其力駁上蔡之「以覺訓仁」亦必然者;其作(知言疑義〉並力闢五峰之後學,進而力斥象山之為禪,亦係必然者。此所謂思理清晰、一貫,而且又皆能充其極也。

以上六點,如再收縮而為一點,則只是對於道體不透,因而影響工夫入路之不同。此所謂一處不透,觸處皆異也。(所謂不透是對原有之義說。若就其自己所意謂者言,則亦甚透。)此所不透之一點,說起來亦甚簡單,即在:對於形而上的真體只理解為「存有」(Being,ontological being)而不活動者(merely being but notat the same time activity)。但在先秦舊義以及濂溪、横渠、明道之所體悟者,此形而上的實體(散開說,天命不已之體、易體、中體、太極、太虛、誠體、神體,心體、性體、仁體)乃是「即存有即活動」者。(在朱子,誠體、神體、心體即不能言)。此是差別之所由成,亦是系統之所以分。此為吾書詮表此期學術之中心觀

\newpage\thispagestyle{empty}\addtocounter{page}{-1}\vspace*{-12mm}\begin{center}\noindent
\includegraphics[clip, trim=155pt 128pt 132pt 249pt, height=162mm]{ocr-input/image-0292.png}\end{center}

\newpage\markright{第一部 \quad 第一章 \quad 宋明儒學之課題}

\noindent 念。依「只存有而不活動」說,則伊川、朱子之系統為:主觀地說,是靜涵靜攝系統;客觀地說,是本體論的存有之系統。簡言之,為橫攝系統。依「即存有即活動」說,則先秦舊義以及宋、明儒之大宗皆是本體宇宙論的實體之道德地創生的直貫之系統,簡言之,為縱貫系統。系統異,含於其中之工夫入路亦異。橫攝系統為順取之路,縱貫系統為逆覺之路。此其大較也。

吾如此詮表,亦不背於常識(一般之感覺)。依以前之說法,見道不見道,體上工夫足不足,本體透澈不透澈,端在是否能體悟「即活動即存有」之實體。支離不支離亦繫於此。心性一不一、心理一不一亦繫於此。凡此,一般皆能感覺到,吾之詮表亦如此歸結。此所謂不背常識也。惟吾能全盡而確定地說出之。此亦並非真容易透澈明白也。然則吾謂伊川、朱子之系統倒有一點「新」的意味,非隨便妄言也。此步新開,雖對先秦舊義以及宋、明儒之大宗為不合,然並非無價值。朱子之系統亦自有其莊嚴弘偉處,如其本性而明澈之,亦當屬可喜之事,非貶視也。此兩系統一縱一橫,一經一緯。經之縱亦須要緯之橫來補充。此兩系統,若對立地看,恰似西方之柏拉圖傳統與康德傳統之異。前者,海德格(Heidegger)名之曰「本質倫理」;後者,海德格名之曰「方向倫理」。此兩詞甚善,不誤也。先秦舊義及宋、明儒之大宗是方向倫理,而伊川、朱子之新開則是本質倫理也。唯在西方,本質倫理先出現,而在中國則後起也。中國以「方向倫理」為大宗,此康德傳統在西方之所以為精絕,而自中國儒學觀之,又所以為可貴也。然希臘傳統在西方為大宗,亦正有其值得吾人之崇贊與欽慕者。吾人亦如此看朱子。

\newpage\thispagestyle{empty}\addtocounter{page}{-1}\vspace*{-12mm}\begin{center}\noindent
\includegraphics[clip, trim=176pt 212pt 130pt 239pt, height=162mm]{ocr-input/image-0296.png}\end{center}

\newpage

然若謂朱子之「只存有而不活動」之理即是柏拉圖之理型,則亦非是。此須要有一簡濫之工作。此即下章之論題。

\newpage\thispagestyle{empty}\addtocounter{page}{-1}\vspace*{-12mm}\begin{center}\noindent
\includegraphics[clip, trim=161pt 445pt 146pt 75pt, height=162mm]{ocr-input/image-0300.png}\end{center}

\newpage\markright{}

\chapter{別異與簡濫}

\section{橫渠、明道之言理或天理}

依前章宋、明儒之分系,對於道體性體之體會只有兩種:

1.體會為即活動即存有。

2.體會為只存有而不活動。

又依前章之正名,宋、明儒學亦名曰「性理之學」。「性理」之得名,普通以為始自明道之言「理」或「天理」以及伊川之言「性即理」。實則「理」之一詞是就道體性體之實而帶上去的,理字並無獨立之實。又,帶上此「理」字或「天理」字亦不自明道始,横渠之《正蒙》中即已隨處皆是。又,伊川之言「性即理」,此固亦是扣緊道體性體而言「理」字,然伊川之言此語實不只此義,且有一特別之標識,即預設心性不一、心理為二、道體性體為「只存有而不活動」是也。是則「性理之學」,普通固定之於伊川、朱子之「性即理」,非是;以伊川之「性即理」概括明道,尤非是。明道固亦可言「性即理」,甚至濂溪、横渠亦可如此言。但濂溪、橫渠、明道所體會之道體性體是「即活動即存有」者,故代

\newpage\thispagestyle{empty}\addtocounter{page}{-1}\vspace*{-12mm}\begin{center}\noindent
\includegraphics[clip, trim=173pt 157pt 140pt 250pt, height=162mm]{ocr-input/image-0304.png}\end{center}

\newpage

\noindent 表此道體性體之「理」或「天理」字亦不只是「存有」義,到最後亦是「即存有即活動」者。是則「性理之學」是通名,由之亦可見有二系之分也。

濂溪對於道體(誠體、神體寂感真幾)體會甚精,然因是初創,性是自剛柔中而言,道體性體未能一;雖亦僱侗地贊曰「大哉易也,性命之源乎?」然道體與性命實未能自覺地通而為一。彼言「理」字亦不多,彼言:「德:愛曰仁,宜曰義,理曰禮,通曰智,守曰信」(《通書·誠幾德第三》),此中「理曰禮」,「理」是通泛字。《通書·理性命第廿二》:「厥彰厥微,匪靈弗瑩,剛善剛惡,柔亦如之,中焉止矣。二氣五行,化生萬物。五殊二實,二本則一。是萬為一,一實萬分。萬一各正,小大有定。」標題為(理性命〉,而文中無此三字。剛、柔、中是說「性」字無疑。其餘「厥彰厥微,匪靈弗瑩」,此兩語亦可以是說「理」字,「靈」字即代表「理」。「二氣五行化生萬物」以下八句則是說「命」,亦可以是說「理」與「命」。二與五是氣,一是太極,即是理。「萬一各正,小大有定」是命,是「各正性命」之命。如是,理與性命亦俱可在此八句中,而道體與性命亦可通而為一,而亦可以一「理」字總代表此道體與性命,此中之「性」字非剛柔中之性,而「命」亦非氣命之命,雖然氣性與氣命俱可帶在內。惟因初創,濂溪未能透澈道體性命之為一,亦未能甚顯明地以「理」字代表之也。

但至橫渠,則道體性命通而為一已甚透澈,而亦甚顯明地即能以理或天理字代表之。是則橫渠之言理或天理,除通泛意義的「理」外(如「天地之氣雖聚散攻取百塗,然其為理也,順而不

\newpage\thispagestyle{empty}\addtocounter{page}{-1}\vspace*{-12mm}\begin{center}\noindent
\includegraphics[clip, trim=161pt 129pt 137pt 249pt, height=162mm]{ocr-input/image-0308.png}\end{center}

\newpage\markright{第一部 \quad 第二章 \quad 別異典簡濫}

\noindent 妄」,此中之「理」字即通泛意義的理),實皆指道體性體而言,亦即「性命」之理也。其如此言理,大抵是根據〈說卦傳〉「窮理盡性以至於命」、「將以順性命之理」,以及(樂記〉之「不能反躬,天理滅矣」諸語中之「理」或「天理」,而即扣緊道體性體或性命之實而言之。並非於道體性體或性命之實外別有一個獨立意義的理或天理也。試看以下之文獻:

1.橫渠〈理窟·義理〉章中有一條云:

\begin{quotation}\kaishu 今之性滅天理而窮人欲,今復反歸其天理。古之學者便立天
理。孔孟而後,其心不傳,如荀揚皆不能知。\end{quotation}

\noindent 《宋元學案》卷十八〈橫渠學案下〉載此條,並附有顧諟之案語曰:「明道程子曰:天理二字是自家體貼出來。先生亦拈天理,而曰歸曰立,發明自家體貼之意,尤為喫緊。」是顧諟已知言「天理」二字不自明道始矣。茲再檢《正蒙》各篇如下:

2.〈神化篇第四〉云:

\begin{quotation}\kaishu 徇物喪心,人化物而滅天理者乎?存神過化,忘物累而順性
命者乎?\end{quotation}

\noindent 案:「順性命」即順天理也。

3.〈誠明篇第六〉云:

\begin{quotation}\kaishu 義命合一存乎理。\end{quotation}

\newpage\thispagestyle{empty}\addtocounter{page}{-1}\vspace*{-12mm}\begin{center}\noindent
\includegraphics[clip, trim=158pt 147pt 137pt 239pt, height=162mm]{ocr-input/image-0312.png}\end{center}

\newpage

\begin{quotation}\kaishu 上達反天理,下達徇人欲者與?

盡性窮理而不可變,乃吾則也。

德不勝氣,性命於氣。德勝其氣,性命於德。窮理盡性,則
性天德,命天理。【……】所謂天理也者,能悅諸心,能通
天下之志之理也。……】舜禹有天下而不與焉者,正謂天
理馴致,非氣稟當然,非志意所與也。〔……]

在帝左右,察天理而左右也。天理者,時義而已。君子教
人,舉天理以示之而已。其行己也,述天理而時措之也。

生直理順,則吉凶莫非正也。不直其生者,非幸福於回,則
免難於苟也。〔案:「回」即孟子「經德不回」之回。]

屈伸相感而利生,感以誠也。情偽相感而利害生,雜之偽
也。至誠則順理而利,偽則不循理而害。順性命之理,則所
謂吉凶莫非正也。逆理,則凶為自取,吉其險幸也。

莫非命也,順受其正。順性命之理,則得性命之正。滅理窮
欲,人為之招也。\end{quotation}

4.〈大心篇第七〉云:

\begin{quotation}\kaishu 燭天理如向明,萬象無所隱;窮人欲,如專顧影間,區區於
一物之中爾。\end{quotation}

5.〈中正篇第八〉云:

\begin{quotation}\kaishu 天理一貫,則無意必固我之鑿。[……]\end{quotation}

\newpage\thispagestyle{empty}\addtocounter{page}{-1}\vspace*{-12mm}\begin{center}\noindent
\includegraphics[clip, trim=174pt 130pt 128pt 254pt, height=162mm]{ocr-input/image-0316.png}\end{center}

\newpage\markright{第一部 \quad 第二章 \quad 別異與簡濫}

\begin{quotation}\kaishu 將窮理而不順理,將精義而不徙義,欲資深且習察,吾不知
其智也。

君子於天下達善、達不善,無物我之私。循理者共悅之,不
循理者共改之。改之者,過雖在人,如在己,不忘自訟。共
悅者,善雖在己,蓋取諸人而爲,必以與人焉。善以天下,
不善以天下,是謂達善達不善。

儒者窮理,故率性可以謂之道。〔……\end{quotation}

\noindent 據以上,橫渠之言理或天理已不少矣,而且又能緊扣道體、性體或性命而言之也。是則理或天理即道體性體之實,亦即性命之為理也。是故當吾人見到明道云:「吾學雖有所受,天理二字卻是自家體貼出來」,決不可以為「天理」二字以及此二字所指之實皆是明道所始創也。即明道自己亦決不會如此寡聞而自居。然則此語之實意只表示他真能理會這道理,並真能由此道理體會出「天理二字」之親切,而即以此二字說此道理(道體、性體性命之理)也。此是實感之事,決不是詞語發明權之事。要者是在道體、性體、性命之實。「理」或「天理」是自然帶上去的,有之不多,無之不少。「天理」二字不是義理系統之關鍵。關鍵是在對於道體之體會為如何。惟用上此二字,則此二字亦有簡括代表、豁然醒目之作用。然必須了解其所指之實。明道即以此二字大講道體、性體、性命之實,以及由此道體性體所顯發之普遍理則,以顯天理之尊嚴與深遠,天理是理體亦即奧體。此雖較橫渠更為顯豁精透與警策,然而其所指之實未有異也。執詞者以為明道前理之地位未有確定,自明道始正式言理字,此則未解其所指之實,理字好似一獨立之概

\newpage\thispagestyle{empty}\addtocounter{page}{-1}\vspace*{-12mm}\begin{center}\noindent
\includegraphics[clip, trim=155pt 149pt 137pt 238pt, height=162mm]{ocr-input/image-0320.png}\end{center}

\newpage

\noindent 念,因而亦只成一普泛之概念,此則大非其真也。在此皮相之見下,並明道所說天理之真義亦喪失矣。以下試看明道如何言天理。

\begin{quotation}\kaishu 1.萬物皆只是一個天理,己何與焉?至如言「天討有罪,五
刑五用哉。天命有德,五服五章哉。」此都只是天理,自
然當如此,人幾時與?與則便是私意。有善有惡,善則理
當喜,如五服自有一個次第以彰顯之。惡則理當惡(一作
怒),彼自絕於理,故五刑五用。曷嘗容心喜怒於其間
哉?舜舉十六相,堯豈不知?只以他善未著,故不自舉。
舜誅四凶,堯豈不察?只為他惡未著,那誅得他?舉與誅
曷嘗有毫髮廁於其間哉?只有一個義理,義之與比。
(《二程全書·遺書第二上》,二先生語二上。呂與叔東見二先
生語。〔未註明誰語,自係明道語無疑。】)\end{quotation}

\noindent 案:此由天討、天命以見天理,推之亦可由天敘、天秩、天倫以見天理。但這些天理並不只是平散在那裡,它有個收攝點,此即是性體,亦曰秉彞。故明道亦由秉彝體會天理。

\begin{quotation}\kaishu 2.「立人之道曰仁與義。」據今日合人道廢則是,今尚不廢
者,猶只是有那些秉彝卒殄滅不得。以此思之,天壤間可
謂孤立!其將誰告耶?(同上。〔未註明誰語,自係明道語無
疑。〕)\end{quotation}

\noindent 案:此有感於當時一般人皆談佛而發。明道於此一點秉彝(人所秉

\newpage\thispagestyle{empty}\addtocounter{page}{-1}\vspace*{-12mm}\begin{center}\noindent
\includegraphics[clip, trim=167pt 129pt 121pt 250pt, height=162mm]{ocr-input/image-0324.png}\end{center}

\newpage\markright{第一部 \quad 第二章 \quad 別異與簡濫}

\noindent 持之常性)確有實感,此即是判儒佛之異之本質。人若於此真有存在的實感,於此立定腳跟,則緣起性空不足惑也。明道亦常言「敬以直内,義以方外」,義由中出,貞定一切,則由「義以方外」亦可體會天理、實理。依明道,敬是工夫,亦是本體,不是拿一個外在的敬去直內也,敬只是本心性體之「純亦不已」,「敬則無間斷」。是則「敬」直通體而言,故亦得曰「敬體」,與言「誠體」同也。是則由「敬以直內」亦可體會天理、實理。此「敬以直內,義以方外」二語亦是明道所常用之以判儒佛者。平散的定然之理(天理)收于秉彝常性。「常性」,人猶易靜態地視之也。由敬體、誠體而觀之,則常性實不離動用之心體。是則由常性悟性體,秉彝常性之性體是「即活動即存有」之性體,此亦是天理也。自此而言天理,則天理已深邃化而為理體,理體即奧體。只平散地視天理,或只自常性視天理,則無以異於伊川。故上兩條判為明道語(由語脈及會通他處觀之是明道語)非謂其是別異語,非謂伊川不能如此言,非唯伊川可以如此言,且亦是宋、明儒共同之意識,此是共許,惟假明道口說出之耳。但體會性體為「即存有即活動」,此亦是天理,則為伊川、朱子所不及。此則見出明道言「天理」之殊特。試觀以下諸條:

\begin{quotation}\kaishu 3.天理云者,這一個道理更有甚窮已?不為堯存,不為桀
亡。人得之者,故大行不加,窮居不損。這上頭來更怎生
說得存亡加減?是他原無少欠,百理俱備。(二程全書·
遺書第二上〉,二先生語二上。呂與叔東見二先生語。〔未註明
誰語,《宋元學案·明道學案》及〈伊川學案〉皆未列此條,自\end{quotation}

\newpage\thispagestyle{empty}\addtocounter{page}{-1}\vspace*{-12mm}\begin{center}\noindent
\includegraphics[clip, trim=158pt 153pt 144pt 246pt, height=162mm]{ocr-input/image-0328.png}\end{center}

\newpage

\begin{quotation}\kaishu 係明道語無疑。〕)\end{quotation}

\noindent 案:此條直就性體言「天理」。「人得之者,故大行不加,窮居不損」,此顯本孟子「君子所性,雖大行不加焉,雖窮居不損焉,分定故也」而說。「人得之」即得之以為「性」也。這作為性的「這一個道理。更有甚窮已?」意即永恆常存,就其為性體總持地言之是一,然中涵萬理,「百理俱備」,其一切顯發之殊相皆已全備於此性體之中,亦即理之一切殊相皆為此性體之所顯發也。此條由永恆常存、「百理俱備」兩義觀之,似是靜態地默識天理之為「本體論的存有」。然自「理之一切殊相皆為此性體之所顯發」而言,則所謂「百理俱備」者並不是有定多之理皆並集於性體之中,實只是一理(一性)之當機而發也。而顯發是依性體之為「即活動即存有」之義而創生地顯發,非如伊川、朱子之視性理為「只存有而不活動」之義下之靜態地顯見也(此義詳見下節)。是則此條雖只靜態地默識其為「本體論的存有」,然實已預設其為一動態的、本體宇宙論的創生實體也。試看下條。

\begin{quotation}\kaishu 4.所以謂萬物一體者,皆有此理。只為從那裡來。「生生之
謂易」。生則一時生,皆完此理。人則能推,物則氣昏,
推不得。不可道他物不與有也。人只為自私,將自家軀殼
上頭起意,故看得道理小了佗底。放這身來都在萬物中一
例看,大小大快活!〔下評釋氏略】(同上。未注明誰語。
《宋元學案·明道學案》及(伊川學案〉皆未列此條。自屬明道
語無疑。)\end{quotation}

\newpage\thispagestyle{empty}\addtocounter{page}{-1}\vspace*{-12mm}\begin{center}\noindent
\includegraphics[clip, trim=180pt 134pt 122pt 257pt, height=162mm]{ocr-input/image-0332.png}\end{center}

\newpage\markright{第一部 \quad 第二章 \quad 別異與簡濫}

\noindent 案:此即動態地看此天理實體也。天理即創生實體,即宇宙之根源。「只為從那裡來」,「那裡」即指示一「根源」。因為萬物都從同一根源來,故萬物得為「一體」,猶如一家族子孫皆從一祖來,故其子孫皆是一家一體也。此「一體」非同一本體之意,乃是由於同一本體,故相連屬而為一體。

如何見出皆從同一根源來?明道即由「生生之謂易」來說。由萬物生而又生之生生不息來指點「易體」。依明道,「生而又生」之生生不息是現象的實然,此固可說是變化、變易,然若只從此現象的實然說「易」,則不能盡《易傳》(甚至《易經》)所說之「易」之實。朱子即如此說,故以為易是屬於氣之變化,是形而下者。然明道不如此體會,明道說「易」是直從「體」上說。「生生之謂易」是指點語,不是界定語。由生生不息來指點其所以不息之生理,即由此生理說易。故云「上天之載無聲無臭,其體則謂之易。」此「體」即「上天之載」之當體自己之體。以「易」為此「上天之載」之當體自己,故易即是體,故曰「易體」。此體是「密」,是「無聲無臭」的。若落於氣上說,則是有聲臭而無所謂「密」矣。(朱子解此「體」字為氣,猶言骨子,是體質之體,言氣之易與理道為體也。非本體之體。此解非是。詳見〈明道章·天道篇〉)。故明道所謂「易體」即「於穆不已」之體也,亦即誠體、神體。它是理,亦是神,乃是「即活動即存有」者。故「上天之載無聲無臭,其體則謂之易,其理則謂之道,其用則謂之神」。神用、理道、易體是一,皆直指「上天之載」而言也,皆是形而上之道體(天道實體或天命實體)也。非是只理道為形而上,而神用與易體則是形而下者也。是故易體即是「於穆不已」之體,即是生

\newpage\thispagestyle{empty}\addtocounter{page}{-1}\vspace*{-12mm}\begin{center}\noindent
\includegraphics[clip, trim=170pt 148pt 139pt 249pt, height=162mm]{ocr-input/image-0336.png}\end{center}

\newpage

\noindent 之理,簡言之,即曰「生理」或「生道」。此易體如以理或天理言之,此理即是「動理」(active reason),非「只存有而不活動」之靜理也。此作為動理之易體即是創造之真幾,亦曰創造實體,乃是「體物而不可遺」的絕對普遍的實體,故萬物皆由之而來,「生則一時〔俱】生」也。是即皆以易體為同一根源也。

依明道,不但皆從此同一根源來,而且來了皆完具此動理(創造的實體)以為性。「皆有此理」、「皆完此理」,即皆同時即具有此創造真幾、動理、於穆不已之體以為性也。不但人有之,物亦有之。此是本體論地圓具言之也。然人與物畢竟有差別,此差別之關鍵即在人能推,物不能推。能推不能推之本質的關鍵在「心」(在是否能自覺地作道德實踐),而作為助緣之形而下的底據則是氣也。物氣昏,心不能呈現,故「推不得」也。(在此,氣與心是兩範疇,不可混同於一範疇。)自此而言,人能彰顯地具此動理以為性,而物則只是潛具也。故本體論地圓具言之,皆具此理;而道德實踐地言之,則物只是潛具,而實不能真以此為性也。

依以上三層疏解,可知前條是靜態地看天理之為本體論的存有,而此條則是動態地看天理之為本體宇宙論的或道德創造的創造真幾;前條是靜態地默識天理之一相與多相(「元無少欠,百理俱備」),而此條則是動態地會通萬物之根源乃至百理之根源而見天理之一相。是故明道之言天理亦唯是就道體性體而言也,而且即以此「天理」二字說此道體性體也。「天理」非一普泛之概念,亦非一獨立之概念。

\begin{quotation}\kaishu 5.「萬物皆備於我」不獨人爾,物皆然。都自這裡出去。只\end{quotation}

\newpage\thispagestyle{empty}\addtocounter{page}{-1}\vspace*{-12mm}\begin{center}\noindent
\includegraphics[clip, trim=164pt 126pt 126pt 251pt, height=162mm]{ocr-input/image-0340.png}\end{center}

\newpage\markright{第一部 \quad 第二章 \quad 別異與簡濫}

\begin{quotation}\kaishu 是物不能推,人則能推之。雖能推之,幾時添得一分?不
能推之,幾時減得一分?百理俱在,平鋪放著。幾時道堯
盡君道,添得些君道多,舜盡子道,添得些孝道多?元來
依舊!(同上。〔未註明誰語,《宋元學案·明道學案》列有此
條,自係明道語無疑。〕)\end{quotation}

\noindent 案:此條與前條相連貫。前條自「萬物一體」說「皆從那裡來」,「皆完此理」,此條則自「萬物皆備於我」說「都自這裡出去」。因為「皆完具此動理」,故每一個體皆是一創造之中心也。在人是「萬物皆備於我」,「都自我這裡出去」,在其他任何個體亦是「萬物皆備於它的我」,「都自它們的我這裡出去」。故云:「不獨人爾,物皆然。」此種「萬物皆備於我」亦是本體論地圓具言之也。然「萬物皆備於我」並不只是「本體論地圓具之」之義,而且亦須有「道德實踐地彰顯之」之義。是故自自覺地作道德實踐以彰顯之言,則惟人能之,其他個體並不能也。而能不能之關鍵仍在能推不能推。而能推不能推之本質的關鍵在「心」,其助緣之底據則在「氣」。此能推不能推所關甚大。明道雖如此事實,然此條重點卻偏重在說「本體論的圓具」義,而對於此能推不能推之事實卻說得甚輕鬆,因此啟黃百家之疑,謂:「此則未免說得太高。人與物自有差等,何必更進一層,翻孟子案,以蹈生物平等,撞破乾坤?只一家禪詮!」(《宋元學案·明道學案》此條下黃氏之案語)。實則此條與前條相連貫。前條依「本體論的圓具」說「皆完此理」,此條即可依「本體論的圓具」說「萬物皆備於我」,此亦是應有之義。是「於穆不已」之體不但創生萬物,而且亦內具於萬

\newpage\thispagestyle{empty}\addtocounter{page}{-1}\vspace*{-12mm}\begin{center}\noindent
\includegraphics[clip, trim=161pt 139pt 144pt 254pt, height=162mm]{ocr-input/image-0344.png}\end{center}

\newpage

\noindent 物而為性,即天道性命相貫通,則「內具於萬物而為性」之義,本體論地言之,應是普遍地有效,「天道性命相貫通」亦應是普遍地有效,無理由單限於人。如是,本體論地言之,應有「皆完此理」之義,亦應有「萬物皆備於我,不獨人爾,物皆然」之義。如是,「本體論的圓具」義當是必然者,而且亦必須立此義始顯出「道德實踐地具」上之有差別。(圓具者依實體之超越又內在說,亦依實體之靜態地平鋪說,亦依一種藝術性的觀照意味說,亦依聖證之一本圓教、大而化之、渾無內外物我之分說。)是以要者在能分別「本體論的圓具」之無異與「道德實踐的具」之有異兩者分際之不同。自「道德實踐的具」而言之,人能具此理以為性,真能自覺地作道德實踐以起道德創造之大用,故能彰顯地「完具此理」,並能彰顯地作到「萬物皆備於我」。然而在其他動物以及草本瓦石則不能有此自覺,因而亦不能有此道德之創造,是即等於無此「能起道德創造」之性也。是故創造實體在此只能是超越地為其體,並不能內在地復為其性,即其他個體並不真能吸納此創造真幾於其個體內以為其自己之性也。此即立顯出人物之別矣。自「道德實踐的具」而言之,此人物之別尚與人類中有能彰顯有不能彰顯之別不同。在人類中,不能彰顯者,吾人仍承認其實有此性,此是真正的潛具——潛具此理為性,潛具其「萬物皆備於我」。然而在物處尚不能說其實踐地潛具此理為性,實踐地潛具其「萬物皆備於我」。實踐地言之,彼實根本不能有此性,亦根本不能有其「萬物皆備於我」也。是以在物處結果只有墮性、本能物質的結構之性也。此種差別實應正視。如此,可無「翻孟子案」之疑。明道雖喜言「本體論的圓具」上之無異,然依其義理客觀地觀之,此並不妨「道德實踐

\newpage\thispagestyle{empty}\addtocounter{page}{-1}\vspace*{-12mm}\begin{center}\noindent
\includegraphics[clip, trim=174pt 116pt 127pt 263pt, height=162mm]{ocr-input/image-0348.png}\end{center}

\newpage\markright{第一部 \quad 第二章 \quad 別異與簡濫}

\noindent 的具」上之有異。彼要說「本體論的圓具」義,故於「道德實踐的具」方面說得較輕鬆。然依能推不能推,彼亦仍可轉過來偏重這方面,而於那方面較輕鬆也。此與禪無關,亦未翻孟子案,只是人不透澈耳。「本體論的圓具」義是極端的理想主義之言也。是所謂一起登法界也。高則固高矣,然卻為本體宇宙論地言之之「天道性命相貫通」之義所必函,雖非道德實踐地言之者之所函。(此處所言之「本體論的圓具」義與朱子所言之「枯槁有性」義不同。詳簡見下節。)

\begin{quotation}\kaishu 6.「寂然不動,感而遂通」者,天理俱備,元無欠少。不為
堯存,不為桀亡。父子君臣,常理不易,何曾動來?因不
動,故言寂然。雖不動,感便通。感非自外也。(同上。
未註明誰語,《宋元學案·明道學案》列有此條,自係明道語無
疑。)\end{quotation}

\noindent 案:此條又從「寂然不動感而遂通」說此天理實體。前第4條,吾人說明道所說之易體是理亦是神。因其是神,故可云「寂然不動,感而遂通,天下之故。」此即是寂感真幾,亦即是誠體、心體也。是故「於穆不已」之易體是理亦是神,是誠亦是心,總之,是即活動即存有者。神、誠、心是活動義。同時亦即是理,是存有義。理是此是誠、是神、是心之於穆不已之易體之自發、自律、自定方向、自作主宰處。由此言之,即曰「動理」,亦曰「天理實體」。理使其誠、神、心之活動義成為客觀的,成為「動而無動」者,此即是存有義。是故誠神心之客觀義即是理,理之主觀義即是誠神心

\newpage\thispagestyle{empty}\addtocounter{page}{-1}\vspace*{-12mm}\begin{center}\noindent
\includegraphics[clip, trim=164pt 134pt 146pt 260pt, height=162mm]{ocr-input/image-0352.png}\end{center}

\newpage

\noindent ——誠神心使理成為主觀的,成為具體而真實的,此即理之活動義,因此曰動理,而動亦是「動而無動」者。是故此實體是即活動即存有,即主觀即客觀。其當機而發所顯之一切殊相即是所謂「百理」或萬理。「是他原無少欠,百理俱備」,是將其所顯發之一切理(實即此同一天理實體之一切當機不同之表現)皆收攝於此實體中而無剩無欠也。「天理俱備,元無欠少」,此「天理」亦是百理之天理,亦俱收攝於此寂感真幾中也。收攝於此寂感真幾中或天理實體中,只是一理,只是一個「於穆不已」之易體(天命實體)之理。然因要說它「百理俱備」,則此實體即偏於靜態的存有義。可是就其「於穆不已」之當機而發言,則又是動態的活動義。其當機而發而顯一特殊之表現,如在父子處、在君臣處等等,則即顯理之多相。此多相之理因當機而發,而各貞定一事,遂亦貞定下來而只成為靜態的存有。然其根源實只是「即活動即存有」之一理也。若只就平散的,靜態的存有義看「天理」,則不能盡明道所說之「天理」之實義。明道所說之天理是當機而發之百理統於一理之根源,統曰「天理」也。天理是就「即活動即存有」之道體性體說。道體性體固是創生之實體,自能當機而發,而顯為「只存有而不活動」之百理之多相,但卻並不只就此存有之百理之多相說天理也。至乎伊川、朱子,只就「存有」義看天理。朱子雖亦知存有之百理皆可收攝於太極之一理,然太極之一理仍是「只存有而不活動」者,是即喪失明道所體悟之「天理實體」義、「於穆不已」之實體義,因而誠、神、心與理不能一,亦因而心性不能一,心理不能一,而轉成另一系統也。此是最根本之偏差。所差只在此一點,然而影響如此其鉅!

\newpage\thispagestyle{empty}\addtocounter{page}{-1}\vspace*{-12mm}\begin{center}\noindent
\includegraphics[clip, trim=172pt 116pt 125pt 258pt, height=162mm]{ocr-input/image-0356.png}\end{center}

\newpage\markright{第一部 \quad 第二章 \quad 別異與簡濫}

以上是解釋此條就寂感真幾說「天理實體」義。惟此條「父子君臣,常理不易,何曾動來?因不動,故言寂然。」此數語語意有滑轉。「常理不易」之不動是不變動義、不改動義,只言理之永恆常在。此與「寂然不動,感而遂通」中之不動意義不同,理之永恆常在不可說「寂然」也。故此數語之實意當為:父子君臣之常理永恆常在,當吾人之性體「寂然不動」時,此常理亦寂然於性體之中而不顯,而實潛隱具在,並無少欠;而當吾人之性體「感而遂通」時,則此等常理即當機而發,粲然明著,亦無增添。而即就百理常在說「寂感」,則不諦。此是語意之不惧,言在此而意在彼也。

以上3、4、5、6四條意相貫屬,最能表示明道之就道體性體說天理,且還而即以天理二字說道體性體也。《宋元學案·明道學案》只列5、6兩條,而又不相連屬,至於3、4兩條則根本不錄。如此支解孤露,遂使人不知明道所說之天理究屬何意,而第5條又啟黃百家之疑,以爲「翻孟子案」,「只一家禪詮」矣!朱子對此四條亦未多講。《朱子語類》卷第九十七、〈程子之書三〉,只有一條涉及此四條,只催侗地以其「論萬物之一源,則理同而氣異;觀萬物之異體,則氣猶相近而理絕不同」之義以及「枯槁有性」之義解之。表面觀之,似有相同,而其實則並不相同,詳見下。近人或有拉雜抄錄之,以伊川語為標準解之,以為此幾條當屬伊川語,此亦非是。以上六條俱見《二程全書·遺書第二上》,俱為呂與叔所記,在同捲中,極易類聚。吾連同其他條類聚於一起,統名曰「明道之天理篇」,詳見分論中之明道章。吾之鑒定其為明道語之方,詳見〈明道章〉之引言。明道云:「天理二字是自家體貼出來。」彼言之如此鄭重,當有以實之。按之思理,以上四條當屬明道語。

\newpage\thispagestyle{empty}\addtocounter{page}{-1}\vspace*{-12mm}\begin{center}\noindent
\includegraphics[clip, trim=157pt 133pt 141pt 253pt, height=162mm]{ocr-input/image-0360.png}\end{center}

\newpage

\noindent 以為伊川語者未知其底實也。只浮泛作解耳。明道說理或天理亦有通泛意義的,如普通說自然的道理之類相似。如「人生氣稟,理有善惡」;「天下善惡皆天理」;「天地萬物之理無獨必有對」;「天之生物也,有長有短,有大有小。[……]天理如此,豈可逆哉?」凡此等等,吾名之曰第二義之天理,亦曰虛說的天理。吾已詳論之於〈明道章・天理篇·附識〉。不得以此意義的天理限定明道所說之天理之全部,以為明道所說者只屬於此意義的天理,至於就道體性體說者,有形而上的意義者,乃屬於伊川。若如此,則明道豈真無形而上的意義的天理乎?明道不如此之淺薄也。惟明道所體悟之形而上的實體(道體)性體與伊川、朱子不同耳。以上四條正表示是明道義,不表示是伊川義也。謂之為伊川語者未能真知該四條之實義也。是以以上四條,一、不得視為伊川語;二、不得視為玄談;三、不得視為「翻孟子案」、「只一家禪詮」;四、不得只以朱子「理同氣異」、「枯槁有性」之義解之。乃的然是明道就道體性體說天理且反而即以天理二字說道體性體之實意之所在也。

吾以上只錄六條,其餘不錄,免得多有重複。詳俱見〈明道章·天理篇〉。

\section{明道之自體上判儒、佛以及其言天理實
體與伊川、朱子之不同}

根據以上六條,明道首先將平散的定然之理(存有意義的天理)收於秉彝常性,由常性體悟性體(由「敬以直內義以方外」亦

\newpage\thispagestyle{empty}\addtocounter{page}{-1}\vspace*{-12mm}\begin{center}\noindent
\includegraphics[clip, trim=170pt 115pt 119pt 259pt, height=162mm]{ocr-input/image-0364.png}\end{center}

\newpage\markright{第一部 \quad 第二章 \quad 別異與簡濫}

\noindent 可體悟性體),由性體體悟道體,直就道體性體說天理,並反而即以「天理」二字說此道體性體也。是以天理二字有確切的意義,非通泛,亦非一獨立之概念。因此,

1.道體性體是即活動即存有者;

2.易體、誠體、心體、神體,此四者與理體是一;

3.心與性是一;

4.心與理是一;

5.理或天理是動理,即曰天理實體,亦是即活動即存有者。

明道即根據此道體性體之天理實體直下從體上判儒佛。此天理實體是能起道德創造、宇宙生化之創造真幾,亦是貞定萬事萬物使萬事萬物有真實存在之自性原則。此是支撐萬物挺立宇宙之剛骨。自此立定,自不能贊成「緣起性空」之如幻如化。此是根本之差異而不容淆混者。其餘儘有相類相似相通處,亦無妨礙也。明道由「秉彝卒殄滅不得」處立定,由「於穆不已」、「純亦不已」處立定,最為透徹而真切,把握得最堅實。此其所以奠立宋、明儒「性理之學」之規模,後之來者無有能外之者;此其所以為大家,為圓教之鑄造者。後來象山言本心即性、心即理,純是孟子學,固不能外此體上之立定。至陽明由本心進而言良知明覺,重視良知之神用,重視良知知是知非之「存在的決斷」,好似天理二字稍輕,然陽明言「無心外之理」(此本「仁義內在」言,不可混濫),良知之自發自律、自定方向、自作主宰即是理,故總言「良知之天理」(良知神用即是天理之存有處,良知即天理),是則良知並非光板之明覺,天理二字並未泯失,亦不輕矣。故黃梨洲述陽明曰:

\newpage\thispagestyle{empty}\addtocounter{page}{-1}\vspace*{-12mm}\begin{center}\noindent
\includegraphics[clip, trim=155pt 194pt 136pt 245pt, height=162mm]{ocr-input/image-0368.png}\end{center}

\newpage

\begin{quotation}\kaishu [……]而或者以釋氏本心之說頗近於心學。不知儒釋界限
只一理字。釋氏於天地萬物之理,一切置之度外,更不復
講,而止守此明覺。〔案:並非置之度外,更不復講,乃根
本不能肯定有理、有天理〕。世儒則不恃此明覺,而求理於
天地萬物之間,所謂絕異。然其歸理於天地萬物,歸明覺於
吾心,則一也。〔案:就世儒言,析心與理為二,理在心
外。就釋氏言,則根本不能肯定有理,只有空理,只有明
覺】。〔……】點出心之所以為心不在明覺,而在天理,金
鏡已墜而復收,遂使儒釋疆界,渺若山河,此有目者所睹
也。(《明儒學案·姚江學案〉)\end{quotation}

\noindent 梨洲語雖有不盡諦當處,然其以「理」以及「心理為一」為儒釋之大界,則並不誤,此仍是明道之矩蠖也。其言「金鏡已墜而復收」,「已墜」者自伊川始,朱子大成之;「復收」者上本象山而仍歸復於明道之初義也。此皆是直下以「即活動即存有」之天理實體判儒佛也。而謂之禪何哉?伊川、朱子只繼承此義之一半。

明道自「於穆不已」之體上判儒佛,而自朱子開始則漸轉而自「下學」上判儒佛,以為凡不自「下學」之路入者皆是禪,此則轉說轉遠而不切要矣。夫禪豈即不下學乎?亦各下學其所學而已。明道自「於穆不已」、「純亦不已」、「即活動即存有」、心性為一、心理為一、易誠神心理皆是一之天命實體(天理實體)判儒佛,而伊川則云:「書言天叙天秩,天有是理,聖人循而行之,所謂道也。聖人本天,釋氏本心。」(《二程全書·遺書第二十一下》、〈伊川先生語七下〉)。「聖人本天」固是,然豈不「本

\newpage\thispagestyle{empty}\addtocounter{page}{-1}\vspace*{-12mm}\begin{center}\noindent
\includegraphics[clip, trim=167pt 114pt 122pt 262pt, height=162mm]{ocr-input/image-0372.png}\end{center}

\newpage\markright{第一部 \quad 第二章 \quad 別異與簡濫}

\noindent 心」乎?是則明示其對於天或天理之體悟不同於其老兄,即只體會為「只存有而不活動」者。或云此或只是一時之輕重言,不必原則上即能斷定伊川之體會天理只為存有而不活動,然衡之其全部思理,伊川對於天理、對於性、對於道,實只體會為「只存有而不活動」者。其所言之後天的心氣之心實不能作「本」也。是以「聖人本天,釋氏本心」,在伊川實不只是一時之輕重言,乃有本質之意義也。

明道告神宗曰:

\begin{quotation}\kaishu 先聖後聖若合符節。非傳聖人之道,傳聖人之心也。非傳聖
人之心也,傳己之心也。己之心無異聖人之心:廣大無垠,
萬善皆備。欲傳聖人之道,擴充此心焉耳。(宋元學案·明
道學案》)。\end{quotation}

\noindent 此顯本孟子而言也。

胡五峰《知言》曰:

\begin{quotation}\kaishu 天命之謂性。性,天下之大本也。堯、舜、禹湯、文王、
仲尼六君子先後相詔,必曰心,而不曰性,何也?曰:心也
者知天地宰萬物以成性者也。六君子盡心者也,故能立天下
之大本。人至於今賴焉。\end{quotation}

\noindent 又曰:

\newpage\thispagestyle{empty}\addtocounter{page}{-1}\vspace*{-12mm}\begin{center}\noindent
\includegraphics[clip, trim=150pt 198pt 142pt 245pt, height=162mm]{ocr-input/image-0376.png}\end{center}

\newpage

\begin{quotation}\kaishu 聖人傳心,教天下以仁也。\end{quotation}

\noindent 此顯本明道而來者。如是,聖人豈不本心乎?伊川此言顯有偏差,亦示其思理之本質有異也。釋氏本心,聖人本天亦本心(本天即本心,非二本也),亦各本其所本而已:聖人所本之心是道德的創造之心,是與理為一、與性為一之本心;釋氏所本之心是阿賴耶之識心,即提升而為「如來藏自性清淨心」,亦並無道德的、實體性的天理以實之。伊川並無孟子之「本心」義,故只好以「本天」、「本心」來別儒佛之異;至朱子即視「以心為性」者為禪,此則真成只「本天」而不敢「本心」矣。是故伊川、朱子只繼承明道義之一半也。

伊川、朱子何以如此判儒佛?(一自本天本心判,一自下學判)。又,天理何以自心上脫落——所謂墜失?又,心神何以自道體性體上脫落而傍落,而只成為後天的實然的心氣之心?此須通過對於伊川、朱子之全部思理一一予以鑒定而後足以知之。若隨意徵引,則其彷彿依似之言多矣,未能決定其必如此也。茲總持言之如下。

溯自濂溪之言誠體、神體,乃至太極,橫渠之言太虛神體,明道之直就「於穆不已」之體言道體性體,而又易體、誠體、神體、心體、理體、仁體、忠體、敬體通而一之,總之是對於道體性體無不視為「即活動即存有」者。橫渠、明道且就此道體性體言天理,而明道且復即以天理二字說此道體性體,而天理實體仍無異指也。惟自伊川開始,承其老兄之言理或天理,遂將道體性體只簡化而為一「理」字,並誠體、神體、太虛、太極,一概不講,是則不但簡

\newpage\thispagestyle{empty}\addtocounter{page}{-1}\vspace*{-12mm}\begin{center}\noindent
\includegraphics[clip, trim=182pt 127pt 119pt 258pt, height=162mm]{ocr-input/image-0380.png}\end{center}

\newpage\markright{第一部 \quad 第二章 \quad 別異與簡濫}

\noindent 化而為一「理」字,且收縮提練、清楚割截,只剩下一「理」字,是則對於言道體性體之原初的背景已漸忘卻,而對於道體性體之內容的意義亦汰濾不少而漸喪失,總之,是成「只存有而不活動」之靜態的理,已不能相應「於穆不已」之體而言道體性體矣。彼以其道德的嚴肅感,對於此簡化與汰濾後的「只存有而不活動」之理之超越與尊嚴確有真切的實感。此卻是真能面對超越之實理而不敢放逸者。充塞天地間無適而非普遍的存有之「實理」。「天下無實於理者」。而現象地反觀吾人之實然的心,即,如其為實然的心而正視之,又顯然覺其不能常如理,如是,理盆顯其為吾人之心所攀企的對象,客觀地平置於彼而為心之所對,理益顯其只為靜態的「存有」義。而通過格物致知以言理,並由此以把握理,則理益顯其為認知心之所對。道體性體只成這個「存有」義與「所對」義之「理」字。此顯然已喪失「於穆不已」之道體、實體義,亦喪失原初言性體之實意。原初言道體性體是不能由格物窮理以知之者。明道就道體性體言天理實體亦不能與格物窮理連在一起說。(無人能由格物窮理言天命實體,亦無人能由格物窮理來肯認上帝,亦無人能由格物窮理來了解吾人之內在的道德心性。)此所以明道很少言格物,即偶爾言之,其義亦殊特,即純從體之朗現,體之直貫言,並無認知的意義,(詳見〈伊川章・格物窮理篇〉附論明道之格物義),而濂溪、橫渠言工夫且無一語道及格物也。而承明道而來之胡五峰即正式言「逆覺體證」為工夫入路,不言格物窮理以致知也。陸、王無論矣。即劉蕺山言工夫亦只在由慎獨以呈現意知心體,由心體以浸澈「於穆不已」之性體,不言認知意義的格物窮理也。惟朱子繼承伊川之思理大講致知格物,走其「順取」之路,力

\newpage\thispagestyle{empty}\addtocounter{page}{-1}\vspace*{-12mm}\begin{center}\noindent
\includegraphics[clip, trim=153pt 132pt 150pt 260pt, height=162mm]{ocr-input/image-0384.png}\end{center}

\newpage

\noindent 反「逆覺」之路。伊川、朱子所以如此者,正因其對於道體性體只簡化與汰濾而為「存有」義與「所對」義之「理」字。此為言道體性體之根本的轉向。朱子雖亦大講太極,然太極之只為「存有」義與「所對」義之「理」字則一也。此一根本轉向有以下之影響:

1.最大的影響便是「道體、性體」義之減殺。總天地萬物而本體宇宙論地言之之道體(實體)原本是「於穆不已」之天命實體、「爲物不貳生物不測」之創生之道,而今則只成靜態的存有,至多是本體論的存有,而不能起妙運萬物之創生之用者。此是「道體」義之減殺。道體具於個體而為個體之「性」,性原本是一個個體(顯明地例證是人)之道德的才能、道德的自發自律之性能,而能起道德創造(道德行為之純亦不已)之用者,而今則只成一些靜態的存有之理,平置在那裡,而不能起道德創造之用者。此是「性體」義之減殺。

2.其次是言道體性體之分際之混漫。道體性體之內容的意義雖一,而言之之分際有異。性原本是就個體而言。道德的自發自律、能起道德創造之用之性能便是此個體之性,而今道體性體混而為一,只是一些普遍的、靜態的、存有的定然之理,混漫個體與由個體而發的一切現象(事態)而不分,定然之理不但是對個體而為性,而且是對一切由個體而發的現象事態而為性,實亦無所謂對個體、對事態,只是平散地對一切事事物物凡是實然的存在者而為其性,為其定然之理即是為其性。此是「性體」義之混漫。

3.由「性體」義之混漫,人物之別不在「性」上分,而在心氣上分。明道言萬物「皆完此理」,皆可以是「萬物皆備於我」,「都自這裡出去」,是就個體而言性。「皆完此理」,皆「萬物皆

\newpage\thispagestyle{empty}\addtocounter{page}{-1}\vspace*{-12mm}\begin{center}\noindent
\includegraphics[clip, trim=173pt 110pt 124pt 269pt, height=162mm]{ocr-input/image-0388.png}\end{center}

\newpage\markright{第一部 \quad 第二章 \quad 別異與簡濫}

\noindent 備於我」,是本體論地圓具言之。自此而言,是性同。然道德實踐地言之,物實不具備此種性,此仍可從性體上別人物;人禽之辨、人物之別仍可從性上說。然依伊川、朱子之說統,則性上不能有區別。定然之理即是性,枯槁亦有性,其有此理有此性是定然地有,並無圓具地有與實踐地有之差別,人能自覺地作道德實踐,此亦不過是依心氣情變之發動當如理時,能將此理此性使之有多樣的顯見,而物則不能,故只收縮而為此物之所以為此物之定然之理。雖有此差別,然其所以為定然之理,而定然之理即是性則一也。人是一實然的存在,其一切心氣情變之發動亦是實然的存在。當此發動不如理時,理亦只收縮而為人一實然存在之所以然之定然之理,而無多樣之顯現。是以理之為定然之理,而定然之理即是性,與在枯槁處同也。此所以明道之「皆完此理」、皆「萬物皆備於我」義不能表面地即以朱子之「理同氣異」與「枯槁有性」解之也。

4.表示此理此性之方式之倒轉。原本言道體是就「於穆不已」之天命實體言,是就「為物不貳生物不測」之創生之道言,言性體是就個體言,反身自證以見吾確有能自覺地作道德實踐能起道德創造之用之超越根據,而今卻不然,卻只就「存在之然」推證其所以然之定然之理以為其定然之性。「陰陽氣也」,是形而下者,「所以陰陽」是道是理,是形而上者。陰陽氣化是實然的存在,有存在有不存在(有生滅變化),而其所以然之理則只是存有,無所謂存在不存在。「仁性愛情」,仁是對應愛之情之實然而為其所以然之定然之理,而此定然之理即是其性。任何實然的存在皆有其定然之理即皆有其性。心之知覺之實然亦有其所以為此知覺之理即性。是則性理只是實然的存在之所以存在之理。吾人名此存在之所以存在

\newpage\thispagestyle{empty}\addtocounter{page}{-1}\vspace*{-12mm}\begin{center}\noindent
\includegraphics[clip, trim=161pt 123pt 129pt 257pt, height=162mm]{ocr-input/image-0392.png}\end{center}

\newpage

\noindent 之理曰「存在之理」,即存在之「存在性」。(友人唐君毅先生首先說此理為「存在之理」,但未能簡別其與以明道為代表之一大系所說者之不同)。此是表示此理此性之方式之根本倒轉。此表示方式之倒轉函有一對於存在之然之「存有論的解析」。是以此方式中之理或性亦可曰「存有論的解析」中之理或性。而通過格物窮理(窮理即是窮的這個理)之認知的方式以把握之,此表示方式之為倒轉益顯、益著實,而其為「存有論的解析」中之理或性亦益顯、益著實。

當吾人以明道所體悟者為標準,說「於穆不已」之天命實體或「為物不貳生物不測」之創生之道為創造的實體時,此創造的實體就其所創造出之「存在之然」說,亦可以說它是「存在之然」之所以然的「存在之理」,在此或亦更可說是「實現之理」,即實現其為一「存在之然」之理,實現之即存在之;但如此說的「存在之理」或「實現之理」是從本體上直貫地說下來的,是以本體為首出而直貫地說下來是如此,不是以存在之然為首出而推證其所以然這種「存有論的解析」中之「存在之理」或「實現之理」,這是提挈宇宙之提起來說的,不是實在論式的平置地說的。就性體而言,性體是吾人能起道德創造之性能,就其所創造的道德行為之純亦不已之存在之然(如醉面、盎背、踐形等)說,此性體亦可以說是此等「存在之然」之所以然之存在之理或實現之理(超越的根據);但如此說的存在之理或實現之理亦仍是從體上直貫下來說的,是單就醉面、盎背、踐形這種道德的存在之然說,不是以存在為首出泛就存在之然而推證者,此亦是提起來說,不是平置地說。提起來體以成用,用以從體,只是其所創造的道德行為之純亦不已才是用,並

\newpage\thispagestyle{empty}\addtocounter{page}{-1}\vspace*{-12mm}\begin{center}\noindent
\includegraphics[clip, trim=155pt 106pt 134pt 269pt, height=162mm]{ocr-input/image-0396.png}\end{center}

\newpage\markright{第一部 \quad 第二章 \quad 別異與簡濫}

\noindent 不是凡存在之然皆是用也。及其大而化之,或從圓教一本說,則宇宙秩序即是道德秩序,道德秩序即是宇宙秩序,此好似一體平鋪,在此說存在之理或實現之理亦與「存有論的解析」似無以異,但仍有異,此乃是一起登法界之意,仍是從本體直貫說,不是從存在上推說。是以依明道所代表之一大系說,如要說存在之理或實現之理,只能從上說下來,不能從下說上去。此正是直貫系統與橫攝系統之異也。在直貫系統,此存在之理或實現之理是超越的、動態的,以「即活動即存有」故也。在橫攝系統,此存在之理亦可說是實現之理,但卻是靜態的(雖亦是超越的),以「只存有而不活動」故也。動態的,對萬物言,是創生而妙運之,依此而言存在之即實現之。靜態的,則只是定然而規律之,依此而言存在之即實現之,此是「存有論的解析」中之理之存在之、實現之。此種實現之、存在之,實是虛籠地說。因實際地存在之者實只在氣,心、神亦屬於氣。故理之為存在之理或實現之理只是虛籠地定然而規律之,不能實際地妙運之而創生之也。

又,此「存有論的解析」中之存在之理,因須通過格物窮理之認知的方式以把握之,此與知識問題相關聯,因而又引生以下兩問題:

(1)存在之理與形構之理之區別。

(2)存在之理與歸納普遍化之理之區別。

\noindent 此則轉說轉遠,每況愈下,皆由伊川、朱子之倒轉而引生,而易為人所想及,而又最易生誤會者,故須進一步加以簡別,以免氾濫。此為下節之論題。

5.由於性理表示方式之倒轉,性只是理,只是存在之所以為存

\newpage\thispagestyle{empty}\addtocounter{page}{-1}\vspace*{-12mm}\begin{center}\noindent
\includegraphics[clip, trim=165pt 148pt 135pt 242pt, height=162mm]{ocr-input/image-0400.png}\end{center}

\newpage

\noindent 在之理,心神義俱脫落而傍落,如是,性體之道德意義與道德力量逐減殺。原本講道體是能起宇宙生化之體,是形而上的,同時亦即有道德的涵義,是故宇宙秩序即是道德秩序;原本講性體是能起道德創造之體,是道德的,同時亦即有形而上的涵義,是故道德秩序即是宇宙秩序。而今道體性體之分際混漫而為一,只是對存在之然而為其所以然之「存在之理」、定然之理,即是其性,心神脫落而傍落,性只存有而不活動,其自身無論在人在物是不能起道德創造之用者,是即其道德力量之減殺。在物處,固不能起道德創造之用,只收縮而為一存在之理,舟車有所以為舟車之理,階磚有所以爲階磚之理,枯槁有所以為枯槁之理,此即是其定然之性本然之性。在人處,人雖能自覺地作道德實踐,然此性自身仍不能起道德創造之用。人之道德實踐並不是本性體之「能起道德創造之用」而為實踐,如孟子所謂「沛然莫之能禦」。蓋性體只是理,並無心之活動義之故,並非本心即性故,心從性體上脫落下來只成為後天的實然的心氣之靈之心。依是,人之道德實踐只是依心氣之靈之收斂凝聚,常常涵養,發其就存在之然而窮其所以然之理(即性)之知用,以便使心氣情變之發動皆可逐漸如理,如理即是如性,是則性理只是被如、被依、被合者,而其自身並不能起道德創造之用也。此即道德力量之減殺。

復次,性體只存有而不活動,只剩下理,則性之為理只能靠「存在之然」來對核其為理,並不是靠其自身之自發自律自定方向自作主宰(此即是其心義、其活動義)來核對其為理。如是,這些憑空平置的理乃是無根無著落者,只有靠由存在之然來逼顯。逼顯已,此理自身即是最後的,即是根。如是,吾人不能由反身自

\newpage\thispagestyle{empty}\addtocounter{page}{-1}\vspace*{-12mm}\begin{center}\noindent
\includegraphics[clip, trim=159pt 125pt 128pt 251pt, height=162mm]{ocr-input/image-0404.png}\end{center}

\newpage\markright{第一部 \quad 第二章 \quad 別異與簡濫}

\noindent 證,自證其道德的本心之自發自律自定方向自作主宰以為吾人之性體,以認取性體之為理,而乃是客觀地由「存在之然」逼顯出其所以然之理以為性,主觀地由心氣之靈之凝聚來把握這些理,以使心氣情變之發動漸如理。就客觀地由「存在之然」來逼顯說,是性理之道德意義之減殺;就主觀地由心氣之靈之凝聚來把握這些理說,吾人之實踐之爲道德的,是他律道德,蓋理在心氣之外而律之也。(理經由心氣之靈之認知活動而攝具之、内在化之,以成其律心之用以及心之如理,此不得視為心理為一,此仍是心理為二。其為一是關聯的合一,不是本體的即一、自一。本心即性、本體的自一,是自律道德。關聯的合一是他律道德)。

性理之道德意義減殺,則性理之尊嚴亦減殺。依明道,道體性體之當機顯發為存有義之百理,此存有義之百理是源於「於穆不已」之天命實體之顯發,是源於能起道德創造之用之性體之顯發,故其尊嚴能提得住。而今則只由存在之然來逼顯、來對核,此即成為平置的、實在論式的理,其尊嚴性即減殺。在伊川、朱子之氣氛中本是極顯這性理之尊嚴的,這是由於吾人已預設其由天命實體而來,由性體而來,然而在其自己之說統中,由於表示方式之倒轉,反不能極成其尊嚴,反成其道德尊嚴性之減殺。此亦與他律道德相呼應也。

以上五點是由伊川、朱子言道體性體之根本轉向而來的結果。吾之所述是依朱子之充其極而說者。伊川只言理、實理,不言太極,亦未言枯槁有性,然此並無妨礙。朱子之引申而充其極並不違伊川之思理。故吾得以伊川朱子為一而說之。夫伊川朱子之言道體、性體,簡之為性理,原本亦源自《中庸》、《易傳》與孟

\newpage\thispagestyle{empty}\addtocounter{page}{-1}\vspace*{-12mm}\begin{center}\noindent
\includegraphics[clip, trim=162pt 136pt 129pt 252pt, height=162mm]{ocr-input/image-0409.png}\end{center}

\newpage

\noindent 子。只因對於「於穆不已」之體不能有相應之體悟,對於孟子所言之性體不能有相應之體悟,故只簡化汰濾而為「只存有而不活動」之「存在之理」。由此步步轉向下落,遂有如上之結果。此是只順「理」之存有義而想下去,而忘其初,所不自覺地自然而轉至者。並非其初自覺地有意如此,而欲與原初之義爲敵也。此其所以亦總依附著經典說,依稀彷彿而難於董理也。然其順理之存有義而想下去,其步步解說卻是自覺的,思理亦甚清楚。順其清楚思理,不混不雜而統觀之,固應有如上所述之結果,而其與濂溪、橫渠、明道之異與其所依附之經典之異亦眉目朗然矣。未能不忘其初,則重視下學以漸磨,亦無妨礙。而實則此不只是漸教問題,乃根本是言道體性體之轉向問題也。故吾判為是橫攝系統與縱貫系統之異,是他律道德與自律道德之異,是本質倫理與方向倫理之異。而伊川、朱子之不得為儒家之大宗而為另開一傳統者亦明矣。此一系統重知識、近常情,故易為人所接受。先秦儒家並非不重學、問、思、辨,然就道德言道德,其內聖之學中言道體性體固是自律道德之縱貫系統,而與伊川、朱子異也。

\section{存在之理與形構之理之區別}

上節吾人說由於伊川、朱子表示此理此性之方式之倒轉,又因須通過格物窮理之認知的方式以把握之,此與知識問題有關聯,因而逐引生兩問題,一是存在之理與形構之理之區別,一是存在之理與歸納普遍化之理之區別。吾人現在先說第一問題。

此第一問題從何引起?曰:即從「所以然」處引起。

\newpage\thispagestyle{empty}\addtocounter{page}{-1}\vspace*{-12mm}\begin{center}\noindent
\includegraphics[clip, trim=164pt 118pt 133pt 267pt, height=162mm]{ocr-input/image-0413.png}\end{center}

\newpage\markright{第一部 \quad 第二章 \quad 別異與簡濫}

吾人平常說「所以然」即是「所以然之理」。「所以然」即是「所以之而然者」,此自然指示一個「理」字(reason)。故理字是由「所以然」而自然帶上去的。依孔子前之老傳統,「性者生也」之古訓,性生兩字雖可互易,然有「性」字出現,亦畢竟是兩個概念。就兩個概念說,「性者生也」,「生之謂性」,雖直就生之實說性,性很逼近生之實,然字面上「性」字即是生之「理」、生之「所以然」。故荀子(正名篇〉云「生之所以然者謂之性」,此直就「生之所以然」說性也。然荀子所說之「所以然」能與伊州、朱子所說之「所以然」同乎?荀子繼此語直接又云「生之和所生,精合感應,不事而自然,謂之性。」此語即上語之解析也,即對於其所說之「生之所以然」再重加以解析。「生之和」即是「生之所以然」。「和」者自然生命之絪縕也,古語所謂「陰陽沖和氣也」(楊倞註語)。自然生命之絪所生發(蒸發)之自然徵象,如生理器官之自然感應、生理欲望之自然欲求,乃至生物之自然本能、心理之自然情緒等皆是,總之即叫做是性,此即等於以自然生命之自然徵象說「生之所以然」。此種「所以然」是現象學的、描述的所以然,物理的、形而下的所以然,内在于自然自身之同質同層的所以然,而非形而上的、超越的、本體論的、推證的、異質異層的「所以然」。故此「所以然」之理即是「生之和之自然」之理也,故荀子就之說「性惡」,其為形而下的「所以然」亦明矣。故告子說「生之謂性」即就「食色性也」說,即就「性猶杞柳」、「性猶湍水」說,此取中性材質義,而此「中性」義與「性惡」義並不衝突也。而荀子亦云:「性者本始材樸也」(〈禮論篇〉)。「本始材樸」即中性義,順之而無節即「性惡」義。而後

\newpage\thispagestyle{empty}\addtocounter{page}{-1}\vspace*{-12mm}\begin{center}\noindent
\includegraphics[clip, trim=179pt 166pt 125pt 230pt, height=162mm]{ocr-input/image-0417.png}\end{center}

\newpage

\noindent 來董仲舒亦云「性之名非生與?如其生之自然之質謂之性,性者質也。」又云「質樸之謂性,性非教化不成。」(《春秋繁露》〈深察名號〉篇與〈實性〉篇以及(賢良對策〉)董子之語即是「生之謂性」之最恰當的解析,亦眩攝告子、荀子義而成也。此一傳統中所說之「所以然」即「自然」義,並無超越的意義。此種自然義、描述義、形下義的「所以然之理」,吾人名之曰「形構原則」(principle of formation),即作為形構原則的理,簡之亦即曰「形構之理」也。言依此理可以形成或構成一自然生命之特徵也。亦可以說依此原則可以抒表出一自然生命之自然徵象,此即其所以然之理,亦即當作自然生命看的個體之性也。

但伊川、朱子所說的「所以然之理」則是形而上的、超越的、本體論的推證的、異質異層的「所以然之理」。此理不抒表一存在物或事之内容的曲曲折折之徵象,而單是抒表一「存在之然」之存在,單是超越地、靜態地、形式地說明其存在,不是內在地、實際地說明其徵象,故此「所以然之理」即曰「存在之理」(principleof existence),亦曰「實現之理」(principle of actualization)。但在此曰「實現」與在明道處不同。此只是靜態地定然之之實現,不是創生地妙運之之實現也。此靜態的存在之理或實現之理其分際相當於來佈尼茲所說之「充足理由原則」,是說明一現實存在何以單單如此而不如彼者。來氏之「充足理由」最後是指上帝說,而在
. 朱子則即是太極。(伊川未說太極,亦無礙)。上帝、太極固非形構之理也。伊川說:陰陽是氣,所以陰陽是道;仁是性,愛是情;心譬如穀種,其中生之理是性,陽氣發動是情。凡此皆為朱子所繼承,而最後穀種一例所蘊函之心性情三分一格式尤為朱子所稱賞,

\newpage\thispagestyle{empty}\addtocounter{page}{-1}\vspace*{-12mm}\begin{center}\noindent
\includegraphics[clip, trim=116pt 143pt 147pt 239pt, height=162mm]{ocr-input/image-0421.png}\end{center}

\newpage\markright{第一部 \quad 第二章 \quad 別異與簡濫}

\noindent 說依此推之,物物皆然。此中之性即超越的、靜態的、本體論的推證的、形而上的所以然之理也。伊川、朱子亦常不自覺地順習慣說性發而為情,實則嚴格言之,彼所說之性理實不能發也,只是心氣依性理而發,統屬於性,遂謂性之發矣。

依朱子,此理只是一理、一太極,一個絕對普遍的、存有論的、純一的極至之理。所謂百理、萬理實只是一極至之理對應個別的存在之然而顯見(界劃出)為多相,實並無定多之理也。存在之然是多,而超越的所以然則是一。太極涵萬理實只是對存在之然顯現為多相再收攝回來而權言耳。太極無所謂動靜,但有動靜之理。實亦無所謂有動靜之理,並不是有定多之理存於太極之中,乃只是一太極對動言而為動之理以成其所以為動,對靜言而為靜之理以成其所以為靜。動靜只是氣之事。嚴格講,只是氣有動靜,非理有動靜。氣之動或靜因統屬於理(太極),為理所領有,遂謂理之動靜矣。理之有動靜與氣之有動靜意義不同也。理無所謂動靜,則說太極有動靜之理(理中有動靜之理)自亦不是有定多之理存於太極之中,只是一太極對動言而為動之理,因動之事界創出一動理相;對靜言而為靜之理,因靜之事界劃出一靜理相,此種種理相皆收攝於太極,遂謂太極有動靜之理矣,此只是收攝之權言,非實有定多之理存於太極中也。推之,仁義禮智之理相亦然。一性理(一太極)對惻隱之情言,即界劃出一仁理相;對羞惡之情言,即界劃出一義理相;對辭讓之情言,即界劃出一禮理相;對是非之情言,即界划出一智理相。實只是一性理對側隱而為仁以成其所以為惻隱,對羞惡、辭讓、是非而為義、禮、智以成其所以為羞惡、為辭讓、為是非也。故「統體一太極,物物一太極」,實只是一太極,並無

\newpage\thispagestyle{empty}\addtocounter{page}{-1}\vspace*{-12mm}\begin{center}\noindent
\includegraphics[clip, trim=184pt 164pt 140pt 242pt, height=162mm]{ocr-input/image-0425.png}\end{center}

\newpage

\noindent 多太極也,只是一「存在之理」,存在之理並無曲折之殊也。曲折之殊是在形構之理處。依朱子,即氣之凝聚所呈現之質性(徵象)也。是故形構之理是「類概念」,是氣之凝聚結構之性,是多,而存在之理則不是類概念,是純一而非多,此即所謂超越的義理之性或本然之性也。枯槁亦有此本然之性,即有其存在之理,此相應「物物一太極」而言也。

在西方,亞里士多德有本質(essence)之說。本質是由之以界定物類者,亦是一物之所以為此物之理。此所以之理由定義而表示,亦當是「形構之理」,因而亦是「類概念」,是多而非一。上推之,蘇格拉底、柏拉圖所說的「理型」主要地亦是就定義而說。能把握感覺界的具體事例依仿而成其為具體事例之理型,即算把握一概念之確定意義。故在柏拉圖,理型亦是多元的,因而亦當是形構之理,亦當是類概念。雖柏拉圖意謂之為理智界的「真實」(reality),為「真實的存有」(real being),與感覺界者分離亦無礙。亞里士多德不同意此超越說,而主張內在說,其共相內在於個體物而為本質,由之以界定物類,實由柏氏之理型說而轉出也。後來就此作為形構之理的本質又引生兩命題:

1.有此物必有此物之本質,但有此物之本質不必有此物存在。例如:有人存在,必有人之所以為人的本質(定義所表示的);但反過來,人之本質(不管對這本質如何講法,實在論的還是唯名論的)卻不函蘊人之存在。此即表示:

2.存在與本質分離。(此兩者之先後不影響此分離義)。

例如要界定人類,說人為「理性的動物」。「理性的動物」一語表示出人之所以為人的本質。此表示:如果有人類存在,他們必

\newpage\thispagestyle{empty}\addtocounter{page}{-1}\vspace*{-12mm}\begin{center}\noindent
\includegraphics[clip, trim=167pt 143pt 137pt 239pt, height=162mm]{ocr-input/image-0429.png}\end{center}

\newpage\markright{第一部 \quad 第二章 \quad 別異與簡濫}

\noindent 定合乎「理性的動物」一語所示之本質。但是,說「人為理性的動物」絕不表示就有具體的人存在。此只表示出一個人底理——形構之理。又如粉筆,從化學分析可知它的結構成分,並且可以簡潔的分子式表明之,吾人亦可依據此分子式去配製粉筆,但是只此分子式自身,即分子式底實有,並不函有一具體的粉筆存在。分子式並不等於依此式構成的物體之具體存在。此即表示存在是一事,定義所表示的本質只是一個抽象的概念(類概念),此又是一事。即,存在與本質分離。必須存在與本質結合為一,始可有具體的個體物之存在。使此兩者結合為一而產生一具體存在物者即西方哲學中所謂「實現之理」。但此實現之理卻不包含在「本質」一概念中,它是一個超越的理念。人之本質不函人之存在。人之存在是要靠一個屬於生物學的血統觀念者,而此概念卻不在人之定義中。依分子式去製造一枝粉筆,這「製造底活動」也不包含在那分子式中。此例很顯明地表示出「實現之理」是一個超越的觀念。「實現之理」之必然要引出即是西方柏拉圖傳統中宇宙論所以成立之關鍵。故實現之理是一個宇宙論的原理,它代表一個超越而絕對的真實體,使一物如是如是存在者。在西方,「實現之理」由神來充當。來布尼茲的「充足理由原則」也是實現之理,他也是意指上帝而言的。柏拉圖所說的「造物主」也是實現之理,它把理型安置在物質上。這種「製造者」之思路一轉即為中世紀基督教的上帝從無中創造世界。實現之理亦可哲學地講為「基本活動」(fundamental activity),亞里士多德即屬此類型的思路,近人懷悌海亦屬之。不管如何講法,實現之理總是一個超越的觀念。吾人當然不能置上帝於一物類之定義中,視為定義中之一成份。朱子之太極亦不能是定義中之一

\newpage\thispagestyle{empty}\addtocounter{page}{-1}\vspace*{-12mm}\begin{center}\noindent
\includegraphics[clip, trim=176pt 158pt 145pt 247pt, height=162mm]{ocr-input/image-0433.png}\end{center}

\newpage

\noindent 成分。定義所表示的「形構之理」(本質)與使本質與存在結合為一的「實現之理」實屬兩層。形構之理只負責描述與說明,不負責創造與實現。但因朱子之太極、性理亦是就存在之然而推證其「所以然」而表示,人遂誤會其太極、性理為形構之理矣。朱子雖理解太極、性理為「只存有而不活動」,無創造妙運之實現義,但卻亦是靜態的、定然而規律之實現義,故是存在之理,而非形構之理,與「即活動即存有」之創生妙運者之為實現之理為同一層次,故不可混其形上的、超越的「所以然」為定義中所表示的本質之為「所以然」也。

伊川、朱子無「形構之理」之義,但因其通過格物窮理(即物而窮其理,即存在之然而推證其所以然)之認知方式而把握其所說之太極性理(存在之理、靜態的實現之理),人可誤會其所說之「所以然」為定義中的本質之為「所以然」,因而亦可誤會為形構之理矣。此因「所以然」之歧義而生之誤會。若判開此兩種所以然,而知其代表兩種理,則此誤會即消除。

形構之理是類概念,因而亦是個知識概念(即知識問題上的概念)。伊川朱子所說之「所以然」雖不表示此概念,然其格物窮理之認知方式可以帶出此概念,吾人今日亦可以自覺地建立此概念,以與「存在之理」(靜態的實現之理)相區別。存在之理是形而上學的概念,亦是存有論的概念,此與知識概念有別。伊川、朱子俱以格物窮理之認知方式去說,此為知識問題與道德問題之混雜。在此混雜中,一方使作為道德實踐的標準之太極性理之道德意義與道德力量減殺,只成為一個認知所對的存有概念(存在之理),一方亦使積極的知識(見聞之知、形構之理所代表之知)不

\newpage\thispagestyle{empty}\addtocounter{page}{-1}\vspace*{-12mm}\begin{center}\noindent
\includegraphics[clip, trim=164pt 134pt 137pt 250pt, height=162mm]{ocr-input/image-0437.png}\end{center}

\newpage\markright{第一部 \quad 第二章 \quad 別異與簡濫}

\noindent 能有真正的建立。伊川、朱子固重視見聞之知,但其「進學」之目的固在把握此太極性理之為存在之理,而其所致之知亦是知此存在之理之「德性之知」。但「德性之知」亦依格物窮理之認知方式而表現,則此知與此知之所對之道德意義與道德力量亦減殺。此與橫渠、明道、象山、陽明等之言「德性之知」異也,亦與濂溪之言「無思而無不通」之睿知(此亦可說是德性之知)異也。(參看《通書·思第九》)。

又,在伊川、朱子,形構之理可落在氣質或氣質之性一層上說。但氣質之性不論是就氣質之剛柔緩急之殊而說一種性(橫渠、伊川是如此),或是如朱子意解為氣質裡邊的性(即義理之性之在氣質裡面濾過而受氣質之限制),俱是就道德實踐而建立,此可說是實踐上的概念,而非知識上的概念。惟客觀地單就氣之凝聚結構而說其種種曲折之相(徵象與質性),則可以逐漸開出物理知識,此即成形構之理、類概念之知識概念矣。此如《朱子語類》卷二、卷三之論天地、鬼神即是純就氣之曲折之相而說,此純為物理的,亦純為自然宇宙的。此部知識應用在道德實踐之受限制上說,即成立實踐意義的氣質或氣質之性一概念矣。

告子「生之謂性」、荀子「生之所以然者謂之性」、董子「如其生之自然之質謂之性」,凡此所說之性倒是形構之理、類概念之性,即以知識類概念之態度說人之性也。「生之謂性」是經驗地詮表性之一原則。應用於人即把人之「自然之質」舉出,應用於牛馬即把牛馬之「自然之質」舉出,雖同是「生之謂性」,而「生之自然之質」有異,是仍有人類、牛類、馬類之別也。故「生之謂性」即是「成之謂性」,即由個體之成而經驗地描述其徵象,總持之以

\newpage\thispagestyle{empty}\addtocounter{page}{-1}\vspace*{-12mm}\begin{center}\noindent
\includegraphics[clip, trim=163pt 148pt 135pt 245pt, height=162mm]{ocr-input/image-0441.png}\end{center}

\newpage

\noindent 為形構之理也。此顯然是實然的態度,亦函是一類概念。而孟子斥之者,是立於價值的觀點上說。雖其辨駁「生之謂性」之辨難推理不諦或甚至有誤,然其自價值的觀點上辨難「生之謂性」之實然的類概念之不同不足以真區別人物之異(人禽之辨),則甚顯。(孟子未知「生之謂性」可函有類概念之不同一義,因此遂有「犬之性猶牛之性,牛之性猶人之性」之難,此是實然觀點與價值觀點混一說也。)是故其所說「人之所以異於禽獸者幾希」,此「幾希」一點不是類概念之異;其由「仁義內在」而說本心即性,此性亦非視作「形構之理」之性,而當視作「實現之理」之性,故可無限地申展,函有絕對普遍性,而可與天道(天命之體)通而為一也。

孟子說「人之所以異於禽獸者幾希」,此幾希一點自是孟子所意謂的「人之所以為人之性」;但此性不是實然地看人之所以為人之類概念之性,而是應然地、理想地,亦是存在主義所謂存在地看人之所以能為一道德的存在之性,此性是能起道德創造之創造真幾。此處說「所以然」是單就人為一「道德的存在」(moralbeing)而說其所以然;說存在之理、實現之理(動態的)是單就人為一道德的存在而說其實現了的或應實現了的「實然」之「存在之理」或「實現之理」;散開說,亦是其道德行為之純亦不已所成之動靜語默之實然(如醉面盎背、施於四體、不言而喻之實然)之存在之理或實現之理。此是從本體上直貫下來說的,不是泛就「存在之然」通過即物窮理之方式而上推地說的;把握此本體亦是由反身的「逆覺體證」而把握,不是由順取的「即物窮理」而把握。故在此「本體」處說「所以然」,人不易誤會為「形構之理」之類概念;無人想到象山所說的本心即性、本心即理以及陽明所說的「良

\newpage\thispagestyle{empty}\addtocounter{page}{-1}\vspace*{-12mm}\begin{center}\noindent
\includegraphics[clip, trim=155pt 125pt 129pt 254pt, height=162mm]{ocr-input/image-0445.png}\end{center}

\newpage\markright{第一部 \quad 第二章 \quad 別異與簡濫}

\noindent 知之天理」是作為類概念的形構之理,如在伊川、朱子處所起之誤會。從「於穆不已」之天命實體處說道體說性體,對應其所創生妙運者言,說所以然、說存在之理實現之理,亦無人能誤會為形構之理。當明道與伊川簡別不開,人或誤會為其所說之天理亦可以伊川、朱子之「由然以推證所以然」之方式、並通過「即物窮理」之認知方式而把握,因而亦可誤會為「形構之理」處之「所以然」。然而經簡別得開,則吾人知明道說天理實從「即活動即存有」之實體處說(本心即性與於穆不已俱在內),並不從「由存在之然以推證其所以然」之倒轉方式說,亦不以「即物窮理」之認知方式說,因此亦不會誤會為「形構之理」處之「所以然」。但由伊川、朱子處之易引人生誤會,吾人分別開實現之理(存在之理)與形構之理之不同,則濂溪橫渠、明道、五峰蕺山象山陽明所說之道體、性體、心體,其為本體直貫之動態的實現之理存在之理自亦甚顯而無疑者。

荀子亦說:「人之所以為人者何以也?曰:以其有辨也。」又曰:「故人之所以爲人者,非特以其二足而無毛也,以其有辨也。夫禽獸有父子而無父子之親,有牝牡而無男女之別。故人道莫不有辨,辨莫大於分,分莫大於禮。」((非相篇〉)其所說之「辨」不是「智辨」義,乃是「禮以別異」之「別」義。但在荀子,此「禮之別」並不是性。此只是客觀地外在地說人之「禮辨之道」。其所說之性可以函是「形構之理」之類概念,但此禮辨之道不是性,則雖說此是「人之所以為人」,卻亦不是人之所以為一道德存在之創造真幾之性,故難說其是「形構之理」,亦難說其是實現之理或存在之理,只是客觀地、構成地屬於聖王之制作,而人當遵

\newpage\thispagestyle{empty}\addtocounter{page}{-1}\vspace*{-12mm}\begin{center}\noindent
\includegraphics[clip, trim=161pt 159pt 138pt 231pt, height=162mm]{ocr-input/image-0449.png}\end{center}

\newpage

\noindent 依之而已。此荀子之所以不透澈也。伊川、朱子本孟子之言性善,並本(中庸》《易傳》之言道體性體,而卻轉為「只存有而不活動」之普遍之理,以「然與所以然」之倒轉方式以及「即物窮理」之認知方式而肯定此理為存在之理,即存在之定然之性,是即無異於將荀子所說之「禮辨之道」推進一步普遍化而為靜態的存在之理,為一切存在之定然之性。此雖已比荀子為透澈矣,然實亦仍是以荀子之心態說道體性體也。只差荀子未將「禮辨之道」說為性耳。伊川、朱子之所取于孟子、《中庸》、《易傳》者只在本之而可以將「禮辨之道」普遍化而說為性。但此性只是存在者之靜態的「存在之理」,其道德意義與道德力量即減殺,是即已喪失孟子《中庸》、(易傳》之言道體性體(包括心體)之本義矣。

是故以明道為代表之一大系,其由道體、性體、心體所展示之形上學是真正儒家的「道德的形上學」,其內容吾人可藉康德之意志自律、物自身、道德界與存在界之合一這三者來規定,而伊川、朱子所成之形上學則頗難說。依朱子本人之辭語,可說為「理氣不離不雜」之形上學。此一整套如分別言之:就理說,是本體論的存有之系統;就氣說,是氣化的宇宙論而以只屬於存有之理以定然之;就工夫說,是認知的靜涵靜攝之系統;就道德說,自亦有道德的函義,但卻是他律道德。是以此形上學如果亦說是道德的,則當是主智主義的道德的形上學(intellectualistic moral metaphysics),簡言之,亦可直說為智的形上學或「觀解的形上學」(theoreticalmetaphysics),此已幾近於柏拉圖、亞里士多德之傳統而與之為同一類型矣。雖伊川、朱子並無形構之理一層,然亦無礙,而亦未始不隱含此一層,而亦未嘗不可自覺地由之而開出此一層。(由即

\newpage\thispagestyle{empty}\addtocounter{page}{-1}\vspace*{-12mm}\begin{center}\noindent
\includegraphics[clip, trim=159pt 143pt 133pt 239pt, height=162mm]{ocr-input/image-0453.png}\end{center}

\newpage\markright{第一部 \quad 第二章 \quad 別異與簡濫}

\noindent 物窮理而留住於氣之曲折之相上即可開出此一層)。而開出此層亦不礙其言靜態的「存在之理」之一也。其言「存在之理」是太極,而柏、亞傳統則言上帝或造物主。然屬於同一層次則無疑,同為觀解的亦無疑。(系統內部細微處之差異自甚多,此不及詳。學者自能辨之)。

在此兩種形上學中,就性體言之,在明道所代表之一大系中,性體是「即活動即存有者」,本體論地圓具言之,人與物同有;道德實踐地言之,人有而物無,此是自道德價值上判人物之異。吾人可補上「形構之理」,在此是「類不同」之異。此兩種異可以下圖表象之:

\begin{center}
\noindent\includegraphics[width=0.89\linewidth]{ocr-image-p103-10.png}
\end{center}

\begin{center}
\noindent\includegraphics[width=0.39\linewidth]{ocr-image-p103-11.png}
\end{center}

\noindent 箭頭表示道德創造之性(實現之理之性),括號表示類不同之性(形構之理之性)。在人處,箭頭能直貫於括號之內,而為人之道德創造之性;在物處則不能,是則只超越地為其體,不能內在地復為其性。

但在伊川、朱子,性理只存有而不活動,其為存在之理或實現之理亦是靜態的;靜態地為一切實然者之「存在之理」即是靜態地為其定然的、義理本然之性;枯槁有其所以為枯槁之「存在之理」,故亦有此定然的、義理本然之性,此即所謂「理同」,理同

\newpage\thispagestyle{empty}\addtocounter{page}{-1}\vspace*{-12mm}\begin{center}\noindent
\includegraphics[clip, trim=151pt 152pt 139pt 234pt, height=162mm]{ocr-input/image-0457.png}\end{center}

\newpage

\noindent 即性同,人與物同一義理本然之性也。此與明道之「本體論地圓具言之,人與物同有」之「同有」義不同。理同性同矣,則人與物之差別只在「形構之理」處。吾人亦可替伊川、朱子補上此層「形構之理」。此形構之理中即包含有氣質之異,此即朱子所謂「氣異」。因此氣異,故人依心氣之靈能多發動如理的情變,因而即能使義理本然之性有多樣的顯現,而物則不能,故定然之性在物處只收縮而為一存在之理,再不能有多樣的顯現。此即朱子「觀萬物之異體,則氣猶相近而理絕不同」一隱晦語之實意。「氣猶相近」是猶有相近處。有相近處,亦有很不相近處。「理絕不同」是理之表現絕不同。正因氣有很不相近處,始有理之表現上之「絕不同」也。不是義理本然之性自身有不同也。此義即等於說氣質裡面的性(朱子所意謂的氣質之性)之表現有異,即性通過氣質來表現而亦受氣質之限制,因而使此性理在此氣質中之表現上有各種等級之差別。即在人類亦有各種等級之差別,皆不能全盡那義理本然之性之全體(完整性);雖可變化氣質,亦仍受其拘限。即在此義下,亦可以概括說明人物之別。在物處,尚不能有人之剛柔緩急一類之氣質,它只有本能或物質的墮性,此皆是其氣之凝聚結構之所成,亦即仍是其氣與質之有殊,即吾所謂「形構之理」所表示之「類不同」也。其形構之理如此,故全不能使理有多樣的顯現。即或較靈的動物如犬馬獼猴螻蟻之類有一點子透露,亦只是本能地顯現,尚絕不能如人之自覺地顯現也。至如草木瓦石乃至枯槁則全不能有顯現,故其義理本然之性亦只好收縮而為如是個體之「存在之理」矣。是則人物之別不在義理的本然之性處,只在因「氣異」而有的不同之表現上。今以「形構之理」包括朱子所說之「氣異」,此一

\newpage\thispagestyle{empty}\addtocounter{page}{-1}\vspace*{-12mm}\begin{center}\noindent
\includegraphics[clip, trim=185pt 144pt 128pt 249pt, height=162mm]{ocr-input/image-0461.png}\end{center}

\newpage\markright{第一部 \quad 第二章 \quad 別異與簡濫}

\noindent 系統可如下圖表示:

\begin{center}
\noindent\includegraphics[width=0.55\linewidth]{ocr-image-p105-3.png}
\end{center}

\begin{center}
\noindent\includegraphics[width=0.4\linewidth]{ocr-image-p105-4.png}
\end{center}

\noindent 拱線表示靜態的存在之理(義理的本然之性),括號表示形構之理,類不同之性,中含「氣異」。在人處,義理本然之性能在氣質裡面有多樣的表現,此以中直線表之,即示本然之性能進到氣質裡面來;在物處,無此中直線,即表示本然之性不能進到其氣與質之裡面來有多樣之顯現,只收縮而為其存在之理,使之成為如是個體之定然之性。

\section{存在之理與歸納普遍化之理之區別}

上節是就「所以然」處加以簡別,此節再就格物窮理之「格」與「窮」處加以簡別。因為伊川、朱子言「存在之理」須通過「格物窮理」之認知方式而把握,又因為其特重「下學」與「道問學」,故人就「格」處「窮」處又易聯想到伊川、朱子是走歸納之路,而誤想其所言之「存在之理」是由歸納而得,因而遂與歸納普遍化之理混而為一。然吾人一經審思,乃知其所言之「存在之理」實與「歸納普遍化之理」不同,不可混而為一;而此存在之理亦不可由歸納而得。然則其於此言格物窮理,其作用究何在?其確定意

\newpage\thispagestyle{empty}\addtocounter{page}{-1}\vspace*{-12mm}\begin{center}\noindent
\includegraphics[clip, trim=179pt 152pt 148pt 263pt, height=162mm]{ocr-input/image-0465.png}\end{center}

\newpage

\noindent 義究如何?

只存有而不活動的存在之理本是一、遍、常,是在「形構之理」以上者;其顯見為多相本是由其對應各別的事事物物而被界劃出,而其自身實非多,亦無定多之理存於其中。在形構之理以上而又非多,則其非歸納普遍化之理甚顯。歸納活動只能施於「形構之理」一層上,而歸納普遍化之理亦只能在此層上而撰成。形構之理由定義而表示,實則無不以歸納普遍化為底子,即有經驗知識意義的定義無不以歸納為背景。定義無論是唯名論的定義,或是以「 本質」為根據的定義(古實在論的定義),皆然。故有經驗知識意義的定義皆可改動。此亦正因:

1.歸納普遍化是經過歸納的程序而來的普遍化,並不是那普遍
性自己;

2.歸納普遍化之真假值是概然的,並不是必然的;

3.歸納普遍化所撰成之普遍原則(一般通例、理)亦是類概
念,因而亦是多的;

4.歸納普遍化亦代表經驗知識;

5.歸納活動施於存在之然(具體事物)自身之曲折內容上,因
描述、記錄、類同、別異而推概之。

但是形而上的、超越的存在之理則只是純一而非多,絕對的普遍而非概念之相對的普遍化;其自身無迹,亦無曲折之内容,故根本不能施行描述、記錄類同、別異之歸納活動;體悟此理亦不足以代表經驗知識,因無跡無相無曲折內容,無可經驗故;體悟此理而肯認其為實有直下是一本體論的實有(存有)之自身,無所謂普遍化;由一事體悟此理是如此,再由一事所體悟的還是此理,並非

\newpage\thispagestyle{empty}\addtocounter{page}{-1}\vspace*{-12mm}\begin{center}\noindent
\includegraphics[clip, trim=171pt 128pt 128pt 260pt, height=162mm]{ocr-input/image-0469.png}\end{center}

\newpage\markright{第一部 \quad 第二章 \quad 別異與簡濫}

\noindent 另一個不同的理;因而體悟此理而肯認之即是定然、必然地肯認之,其自身亦是定然而必然的(形而上地定然必然的),無所謂概然也。然則此理不由歸納而得,亦非歸納普遍化之理,亦明矣。

然則伊川、朱子於此言即物窮理,對存在之理言,其真正作用究何在?其確定意義究如何?

即物窮理,所窮者是存在之理,則今日格一物,明日格一物,久之,自有豁然貫通處,此所謂「豁然貫通」亦不是歸納普遍化之得通例。歸納普遍化所成之通則無所謂「豁然貫通」也。其所「豁然貫通」者仍只是此存在之理之為一、為遍、為常。也不能說只格一物,便可豁然貫通,雖顏子亦不能。其實不但顏子不能,在此「即物窮理」之分際上,任何人亦不能。若照陽明所說,聖人生而知之,只是知個天理,不知能問亦是天理。是則其說天理是就本心良知說天理,而本心良知之天理是不須要亦不能通過「即物」而窮者。若就孟子盡心知性知天說,此心性天之天理亦不是即物而窮者。就「於穆不已」之天命實體說天理,此天命寶體之天理亦不是即物而窮者。此是反身逆覺體證之路,不是向外順取之路。今倒轉而為向外順取,就「即物窮理」說,則自不能只格一物。伊川、朱子在此將顏子之「聞一知十」拉來與其「即物窮理」連在一起說,好像顏子亦是此路,人或以為顏子之「聞一知十」即是伊川朱子言「格物」之根據。「聞一知十」只是表示其天資之明敏,亦不能只格一物便可「豁然貫通」,亦須多聞多格。實則此只是借用說「格物」義,並不能由此即謂顏子亦是順取之路。《論語·子罕》篇記云:「顏淵喟然嘆曰:仰之彌高,鑽之彌堅,瞻之在前,忽焉在後。夫子循循然善誘人。博我以文,約我以禮。欲罷不能,

\newpage\thispagestyle{empty}\addtocounter{page}{-1}\vspace*{-12mm}\begin{center}\noindent
\includegraphics[clip, trim=158pt 142pt 133pt 242pt, height=162mm]{ocr-input/image-0473.png}\end{center}

\newpage

\noindent 竭吾才,如有所立,卓爾。雖欲從之,末由也已。」此中博文約禮之語最為主張「道問學」者所常引用,然一般「道問學」、博文約禮,無人能反對。聖門之言博文約禮只是一般的教育,並不就體悟道體性體說。顏淵之喟然而嘆是贊聖人之道之高深莫測廣大無盡,並不表示是體悟道體性之確定的工夫入路。若以為聖門只以博文約禮、道問學、為體悟道體性體之確定的工夫入路,則孔子告顏淵亦言「克己復禮為仁」,此將如何說?又贊顏淵之好學為「不遷怒,不貳過」,此又將如何說?是以顏淵嘆語中之「博我以文,約我以禮」,不能與伊川、朱子之言「即物窮理」等同起來,視為一事。伊川、朱子將博文約禮道問學拉來與「即物窮理」合一說,是表示此是體悟道體性體之確定的工夫入路,但〈論語〉不表示此義。問題轉在體悟道體性體上,則言體悟道體性體之顯明的路數與根據是在孟子,而不在顏淵之嘆語。即就《論語〉說,亦在如何「識仁」,而「識仁」是不能就「即物窮理」來識的。(仁不離人倫日用,是就事上來表現、來驗證,但此與伊川朱子之「即物窮理」並非同一意義。)是以就識仁與體悟道體性體說,顏淵博文約禮之語之語乃成不相干者,不能於工夫入路上有若何決定之用。伊川、朱子將博文約禮、道問學與其「即物窮理」之路合一說,人於此有異議是對其「即物窮理」之路有異議,不是對於博文約禮有異議也。此問題完全是體悟道體性體之入路問題,不是一般的博文約禮問題也。

伊川、朱子走其「即物窮理」之順取之路,故一方不能只格一物,一方亦不能盡格天下之物。既「即物」而窮矣,何便只即一物?此與天資無關。蓋「即物」上即必然函著須多「即物」。及

\newpage\thispagestyle{empty}\addtocounter{page}{-1}\vspace*{-12mm}\begin{center}\noindent
\includegraphics[clip, trim=177pt 130pt 132pt 257pt, height=162mm]{ocr-input/image-0477.png}\end{center}

\newpage\markright{第一部 \quad 第二章 \quad 別異與簡濫}

\noindent 其一旦「豁然貫通」,則即一物可,即多物亦可。但在漸磨的過程上,則必須不能只即一物。若在逆覺之路,則根本不須「即物」,要者倒在「反身」。今必須「即物」,則自不能只即一物。至于亦不能說盡格天下之物,蓋「生也有涯」,何能盡格天下之物?是以不能說只格一物,亦不能說盡格天下之物。此種「即物以窮存在之理」之工夫只是漸磨之工夫;而此漸磨乃是無窮無盡者,並不是磨久了,一旦豁然貫通,便可不即物。縱使「生也有涯」,亦須盡有涯之生以赴之,乃不容一息間斷者。(此與盡格天下之物不同)。

此種無窮無盡、不容間斷之「即物窮理」之漸磨工夫,其作用只在朱子所說「心靜理明」之一語。蓋伊川朱子之說之心只是實然的心氣之靈之心,其自身常不能凝聚而清明,反常在浮動、昏沉、散亂之中。是以必須敬以涵養使之常凝聚常清明,然後始能發其明理之用。明理是明存在之理,故必須即物以明。在此,自然用得上《大學》「致知在格物」之一語。(言「用得上」者表示不必即《大學》之原義)。此即伊川「進學則在致知」一語之實義。「進學」者即是即物以明存在之理,以致其心知之明也。此種「心靜理明」之終極的亦即本質的作用即在使吾人之心氣全凝聚於此潔淨空曠無跡無相之理上,一毫不使之纏夾於物氣之交引與糾結中,然後心氣之發動始能完全依其所以然之理而成為如理之存在,此即所謂全體是「天理流行」也。此時吾人即只見有理,不見有氣,全宇宙亦只是一理之平鋪。天理、實理充塞一切,即貞定一切。是故「即物窮理」以致知並不是留住于物自身之曲折之相上而窮究其形構之理以成經驗知識(見識之知、科學之知),乃是即之而越過其

\newpage\thispagestyle{empty}\addtocounter{page}{-1}\vspace*{-12mm}\begin{center}\noindent
\includegraphics[clip, trim=161pt 134pt 146pt 263pt, height=162mm]{ocr-input/image-0481.png}\end{center}

\newpage

\noindent 曲折之相以窮究其超越的、形而上的「所以然」之「存在之理」,以便使吾人之心氣全凝聚於理上,使其發動全如理。故此知仍是「德性之知」,其目標仍在指向於道德行為上,使吾人之行為皆如理。此種知雖須即物以致,卻並不成經驗知識,因其所窮之理無跡無相、為一為遍為常故,故曰德性之知。蓋此理本為存在之理,故吾人知之而依之,即可發動如理之行為以成其德行。有德行(如理的行為)始可說有德性。此種使吾人有德行有德性,故曰「德性之知」也。依明道所代表之一大系而言,發於本心性體或誠體之知曰「德性之知」,此不須「即物窮理」以致之,而此亦無認知的意義,只表示本心性體之自主自律自定方向之自知是非,或是表示由誠起明之自知自決之朗照,並不是認知一所對之理也。蓋本心即理故,誠體即理故。發於此本心誠體之自主自決之知即證實其為理。並不是認知一所對之理(一客觀之存有)也。但在伊川、朱子,性只成存在之理,只存有而不活動,心只是實然的心氣之心,心並不即是性,並不即是理,故心只能發其認知之用,並不能表示其自身之自主自決之即是理,而作為客觀存有之「存在之理」(性理)即在其外而為其認知之所對,此即分心理為能所,而亦即陽明所謂析心與理為二者也。在此情形下,「德性之知」即不能說為「發於本心性體」之知,或「由誠起明」之知,而只能說為「知存在之理而依之以使心氣之發動皆如理,以成德行有德性」之知。此是就「即物窮理」說德性之知,不是就本心性體或誠體之明說德性之知。是故在伊川、朱子,「即物窮理」之真實作用即在「心靜理明」,其確定意義即是順存在之然而明其所以然之「存在之理」以成德性之知,以便使吾人之心氣之一切發動皆如其所以然之存在之理而成為

\newpage\thispagestyle{empty}\addtocounter{page}{-1}\vspace*{-12mm}\begin{center}\noindent
\includegraphics[clip, trim=172pt 110pt 124pt 271pt, height=162mm]{ocr-input/image-0485.png}\end{center}

\newpage\markright{第一部 \quad 第二章 \quad 別異與簡濫}

\noindent 如理之德性也。此亦可說是稱性而發,但不是稱「本心即性」之性體而發,而是心氣依性理而發,而性理自身則只存有而不活動,並無創生妙運之神用與興發道德行為「沛然莫之能禦」之大用。依格物窮理之方式去把握性理,亦只能如此。

是以「格物窮理」所函的博文約禮、道問學之意義,就把握存在之理以成「德性之知」說,只是消極的意義。其目的不在成經驗知識,即無積極的意義。所謂博、所謂問、所謂學,只是經由之以漸磨,以便使吾人「心静理明」,而所明之理是一、遍、常之「存在之理」也。然則博文約禮、道問學究有無積極的意義?曰:有,其積極的意義首先是在成經驗知識之自身。但此必須留住於「存在之然」上而窮究其自身之曲折之相始能出現。若但經由而越過之以窮「存在之理」,則不能出現。是以格物窮理亦有兩面用:用於存在之理,則成德性之知,博文下學在此無積極的意義;用於形構之理,則成經驗知識(見聞之知),博文下學在此有積極的意義。伊川、朱子言格物窮理未作此分別,常混在一起說。然其目標固在窮存在之理,而不在窮形構之理。關於形構之理之知識是在窮存在之理之過程中不自覺地帶出的。故終於是窮性理的道德學家而非是窮物理的科學家。當朱子說「衆物之表裡精粗無不到,而吾心之全體大用無不明」時,此兩語如其有真實的意義,即在窮「存在之理」,而不在窮「形構之理」;其作用即在使吾人之心氣與外物全部為天理實理(性理)所貫澈所貞定,就心氣言,是全部清明寧靜,皆如理而無一毫隱曲之私,就物(存在之然)言,雖一毛一孔亦見其為天理實理性理所澈盡。此即是其所說「用力之久而一旦豁然貫通」之境。「用力之久」即是藉「即物」之見聞以磨練。「豁

\newpage\thispagestyle{empty}\addtocounter{page}{-1}\vspace*{-12mm}\begin{center}\noindent
\includegraphics[clip, trim=163pt 131pt 134pt 259pt, height=162mm]{ocr-input/image-0489.png}\end{center}

\newpage

\noindent 然貫通」即是「表裡精粗無不到」,雖一毛一孔亦見其為天理實理性理所澈盡而無一物之或遺。若不就窮存在之理說,而就窮形構之理說,則該兩語便無真實的意義。因為你並沒有研究出物理化學乃至其他各種經驗科學之知識,何能說「表裡精粗無不到」?是以若對應科學之知言,該兩語是無意義的,甚至是廢話。人可以說你一生根本一事未作、一物未明,簡直是個懶漢,只說了一句空話,並無客觀的實效。也可以說你一生的艱苦只是不相干的徒勞,不知你忙些甚事,根本是不對題,因為如果真對題,何以沒研究出生物學、植物學來?明是一點都未到,何得侈言「表裡精粗無不到」?但是若就藉「即物」之見聞以窮存在之理說,該兩語實有真實之意義與作用,確能成德性之知以使吾人之心氣之一切發動皆可逐漸如理以成德行,以有德性,此就是其客觀的實效——實踐上的實效。至於其實效之程度如何,其在成德上之力量如何,則是另一問題。

但伊川、朱子皆如此重視博文下學,如此重視道問學,難道博文與問學就只是這消極意義乎?如果只是這消極意義,只在憑由之以窮存在之理,則博文與問學自身不能落實,其重要性即減小。吾人前說就窮存在之理說,不能只格一物,亦不能盡格天下之物。此即表示須多格、多磨練,而且不容間斷。雖然如此,但就存在之理說,格一件是此存在之理,格多件亦只是此存在之理,並無多樣之存在之理。「一旦豁然貫通」,亦確證實此義。是以於理之內容並未增加,所增加者只是心氣之寧靜與清明之程度,而於其「知」之內容亦無所增加。是即表示博文與問學已無甚重要矣。朱子說「大學始教,必使學者即凡天下之物,莫不因其已知之理而益窮之,以求至乎其極」。「因其已知之理而盆窮之」即函著從其直接所已知

\newpage\thispagestyle{empty}\addtocounter{page}{-1}\vspace*{-12mm}\begin{center}\noindent
\includegraphics[clip, trim=167pt 111pt 129pt 270pt, height=162mm]{ocr-input/image-0493.png}\end{center}

\newpage\markright{第一部 \quad 第二章 \quad 別異與簡濫}

\noindent 者而益窮其所未知者。但在「存在之理」上,此「已知」與「未知」並非因理之不同而分別,似乎只是「懵然」與「豁然」之分別,只是發蒙與明澈之分別。是即已知與未知之別在客觀內容上並無實義。如是,從已知到未知,層層推進上之博與多亦並無積極之意義,是亦表示博文與問學重要性之減殺。是以在此博文與問學之重要性似只在憑由之經歷一番,可以使吾人普遍地肯定的(遍萬物而為言的)存在之理更加落實,而於理之內容無所增加(因只一理,並無多理故),並使吾人之心氣更加寧靜與清明,而於知之內容亦無所增加(因只以存在之理為知之內容故),只是使心寧靜與清明之極只成一貞定明淨之玻璃鏡、一光板之鏡照。

吾人如何能使伊川、朱子所重視之博文與問學更有實義(積極的意義)?吾人前說其積極的意義首先是在成經驗知識之自身。此必須留住於「存在之然」上而窮其曲折之相以成形構之理。此處才可以說歸納,已知未知才有實義,不只是懵然與豁然、發蒙與明澈.之分,而且有理之内容之增加(因形構之理是多非一故),心知之明亦不只是一寧靜與明淨之光板,而且有知之內容之增加。如是,「存在之理」之一之底子可以交織上形構之理之多,而明淨光板之心之以存在之理為其知之內容者復可交織上以形構之理之內容。此是兩曆之交織。如是,方能使博文與問學挺立其自己而有實義。但只如此說,尚不能盡其實義。吾人尚須進而明此形構之理於成德上究有本質之作用否?如無,還是不能極成「道問學」之重要。伊川、朱子固是於兩層之理未曾作嚴格之分別,然見聞之知與德性之知之分別仍隱函此兩層理之分別;固是重在德性之知,窮存在之理,然其順取之路如此重視道問學,而又「因其已知之理而益窮

\newpage\thispagestyle{empty}\addtocounter{page}{-1}\vspace*{-12mm}\begin{center}\noindent
\includegraphics[clip, trim=161pt 140pt 119pt 245pt, height=162mm]{ocr-input/image-0497.png}\end{center}

\newpage

\noindent 之」,亦未曾於兩層上分別「已知未知之別」之不同,而卻是兩層混一說,(伊川尚說德性之知與見聞之知,朱子即不甚言),是則「形構之理」之知對於成德當有其本質的作用。吾人今自覺地予以分別,當於形構之理於成德上之作用予以積極之說明,以極成伊川、朱子重視「道問學」之實義。

窮「存在之理」以使吾人之心氣全凝聚於潔淨空曠無跡無相之理上固是使心氣發為純淨德行之必要條件,因而亦有其成德上之本質的作用,但須知在伊川、朱子之系統中,一、此存在之理(太極、性理)是空無內容者,其多相是對應「存在之然」而被界劃出,其自身只是純一之理,並無定多之理存於其自身之中,二、又其自身是只存有而不活動者,並無創生興發之用,是以當吾人之心氣須要依存在之理發為行動以成「存在之然」時,光此存在之理實是不足夠者。當吾人從「存在之然」推證其超越的所以然以為「存在之理」時,說凡是實然皆有其理,皆依其存在之理而來,此好像很清楚而容易,然當吾人真地(存在地)要去依存在之理而發為行動時,立時會頓感茫然,而不知如何下手。此無關於已窮未窮。即使已窮而已明澈,若只此存在之理,亦仍然會感茫然。說惻隱之情之所以然之理是仁,此甚清楚而容易。但若要依「仁」這個理去發動惻隱之情(或愛之情),落實了,便會頓感茫然。只隴侗這樣一說,好像亦很清楚而容易;但若真地要去發動,便立感並不那樣清楚,亦並不那樣容易。蓋惻隱或愛的行為都是些特殊的行為,特殊的行為對應特殊的情境而發,推之一切其他道德行為皆然。若光知一特殊行為有其所以存在之理,吾人如何能只依這空洞的存在之理去發這特殊的行為?特殊的行為有記號、有征象,但其存在之理並

\newpage\thispagestyle{empty}\addtocounter{page}{-1}\vspace*{-12mm}\begin{center}\noindent
\includegraphics[clip, trim=165pt 130pt 128pt 251pt, height=162mm]{ocr-input/image-0501.png}\end{center}

\newpage\markright{第一部 \quad 第二章 \quad 別異與簡濫}

\noindent 無記號,亦並無徵象。吾人如何能泛應曲當,單依存在之理去發這些同是側隱或同是愛的特殊行為?此確是一個難題。在此,立見出「形構之理」之知之重要。在此,形構之理是就特殊情境說。對子女是慈愛,對旁人也是慈愛,乃至有各種情境下的慈愛。難道說只依一存在之理即可發出這些同是慈愛的特殊行為而皆能泛應曲當而如理乎?我看在伊川朱子之重視「即物窮理」中並不能如此之簡單與顛預。是以在「即物窮理」中必須能窮這些特殊情境底形構之理(曲折之相),然後吾人依存在之理去行始可有一指標可資遵循。特殊情境上的形構之理是吾人發動行為底軌跡或指標,而存在之理則只是此「行為存在」之超越的所以然而定然者,所謂「所當然而不容已,與所以然而不可易」者(黃勉齋(朱子行狀〉語)。必須去作是「存在之理」之事,而如何去作或為何如此作則須有形構之理為指標或軌跡。譬如對父母當孝,倘若吾人對於父母之身體與心志之情境無所知,則簡直不知如何能恰當地去表現一孝行。汝必須去行孝以實現一孝行(使孝行存在),此是「存在之理」之事。但如何去行孝,則須賴「形構之理」作指標。又如舟有舟之存在之理,車有車之存在之理,但若吾人對於舟車之形構之理無所知,光依存在之理亦造不出舟或車來,而當吾人要發動使用舟車之行為時,亦不必能有得當之使用,也許竟茫然亂動手腳,用舟於陸,用車於水!枯槁亦有其存在之理,但若對於枯槁之物之形構之理無所知,則麻黃當作人參吃又如何?是以當即物窮理時,不但要窮其存在之理,同時形構之理亦必須在窮中,而一存在之理之「使然者然」,此「然」是存在,而其曲折之相亦同時含具在此存在之然中而為一整體(一完整之個體)。吾人窮存在之理時必須通過形

\newpage\thispagestyle{empty}\addtocounter{page}{-1}\vspace*{-12mm}\begin{center}\noindent
\includegraphics[clip, trim=146pt 133pt 142pt 255pt, height=162mm]{ocr-input/image-0505.png}\end{center}

\newpage

\noindent 構之理,但並不是只通過而越過之不理它,通過之亦須留住於其上而窮究之,由此以滲透至存在之理。否則存在之理不切不實。所謂「通過之」只是不止於此而已,並非閉目不理會也。若閉目不理會,不得謂為「即物」。所謂「眾物之表裡精粗無不到」,此語猶嫌催侗,實則只是形構之理(表粗)與存在之理(裡、精)之皆須窮方能盡朱子心中之所想。此方真是道問學之積極意義,使道問學之重要性挺立而不搖蕩,雖終極目標在窮存在之理亦無礙。從形構之理到存在之理皆是道問學精神之所貫。形構之理導引至存在之理,因此可以作為發動行為之指標或軌跡,而存在之理之一亦因此指標或軌跡而界劃出一特定之相;存在之理貫註於形構之理,如水銀瀉地,無孔不入,如是方能盡其「使然者然」之具體的任務,其所使之然者固不只是一赤裸之存在,連同其曲折之相(特殊之内容)亦皆使之然矣,因此依存在之理發動行為固必須以形構之理為指標或軌跡而後始有著手處也。窮理兩層連帶皆須窮,則兩層之理對於成德皆有本質之作用亦明矣。就窮理說,若不究知形構之理,亦不能真切地上至存在之理。就發行說,若不知形構之理,則存在之理亦下不來。此當是伊川、朱子重視道問學之實義。

此一系統徹底是漸教,亦徹底是唯智主義的他律道德。形構之理之重要即順成此他律道德。形構之理與存在之理皆所以律吾人之心氣者也。涵養上之敬亦唯是在使心氣常常凝聚而清明能完全凝聚於理上而順理。此一系統亦使一切行為活動只要是順理(順形構之理之實然與順存在之理之當然與定然)即是道德的,此是唯智論與實在論之泛道德,而道德義亦減殺。此其所以為他律道德,亦曰「本質倫理」也。實則唯是心之自主、自律、自決、自定方向方真

\newpage\thispagestyle{empty}\addtocounter{page}{-1}\vspace*{-12mm}\begin{center}\noindent
\includegraphics[clip, trim=182pt 117pt 124pt 274pt, height=162mm]{ocr-input/image-0509.png}\end{center}

\newpage\markright{第一部 \quad 第二章 \quad 別異典簡濫}

\noindent 正是道德,此是道德之本義,並不是只要順理即是道德也。是以伊川、朱子系統中作為「存在之理」之性理其所表示的「當然而不容已與所以然而不可易」實並提不住道德上之「應當」義。再加上「形構之理」之重要,「應當」全由「實然」來決定,是即「應當」全轉成平鋪之實然,實然通其所以然而定然即是應當。存在之理之存有與形構之理之本質這一實然而定然之系統,提綱說,這—存有之系統,即是應當之系統。此是以「存有」決定「善」者。此其所以為實在論、為本質倫理也。(順本質系統為指標而活動。)

然衡之先秦儒家以及宋、明儒之大宗皆是以心性為一,皆主心之自主、自律、自決、自定方向即是理;本心即性,性體是即活動即存有者;本體宇宙論地自「於穆不已」之體說道體性體,道體性體亦是即活動即存有者。活動是心、是誠、是神,存有是理。此是攝存有於活動,攝理於心神誠,此即是吾人道德創造之真幾性體。此性體不能由「即物窮理」而把握,只能由反身逆覺而體證。從此性體之自主、自律、自決、自定方向上說應當,此方真能提得住、保得住道德上之「應當」者。此是真正理想主義的自律道德,亦曰方向倫理也。此是以「意志」(康德所說的意志)決定「善」者,以「活動」義決定「善」者,而即活動即存有也。對主智主義言,此是主意論——康德式的主意論,非中世紀及來布尼茲之主意論也。宋明儒皆本先秦儒家說心說性,不說意,惟最後至劉蕺山即說這種意根獨體。其義一也。

在此自律道德之系統中,形構之理處之經驗知識(博文與問學)亦並非不重要,但並非道德之所以為道德之本質,乃只是作成一道德行為之助緣。以此知識為助緣,所極成者仍是自律道德,並

\newpage\thispagestyle{empty}\addtocounter{page}{-1}\vspace*{-12mm}\begin{center}\noindent
\includegraphics[clip, trim=155pt 143pt 137pt 244pt, height=162mm]{ocr-input/image-0513.png}\end{center}

\newpage

\noindent 未變成他律道德。此即象山所謂「尊德性」之實義。但在伊川、朱子,形構之理之重要卻終於只極成他律道德,此即是其重視「道問學」之實義。根本處只在其言性體方式之倒轉,而對於性體又只理解為「只存有而不活動」者。非泛然爭論尊德性與道問學之輕重先後也。空頭觀之,不足以知其實。

順伊川、朱子之說統,始有存在之理與形構之理之混淆,始有存在之理與歸納普遍化之理之混淆,故詳簡之如上以明其分合。順明道、象山等所代表之一大系(合兩系而成者)言,其言心體性體乃至道體不會使人有此誤會與混淆。

本章及上章是宋、明儒學之綜括。以下依明道、象山等所代表之一大系為根據來融攝康德,並藉康德之辨解以顯自律道德之實義,並進而展示其所函之全部理境,即道德的形上學之究極完成。

\newpage\thispagestyle{empty}\addtocounter{page}{-1}\vspace*{-12mm}\begin{center}\noindent
\includegraphics[clip, trim=173pt 385pt 126pt 125pt, height=162mm]{ocr-input/image-0517.png}\end{center}

\newpage\markright{}

\chapter{自律道德與道德的形上學}

\section{論道德理性三義}

依康德,自由意志所先驗構成的(自律的)普遍的道德律是屬於睿智界,用今語說,是屬於價值界、當然界,而知性範疇所決定的自然因果律則是屬於感覺界、經驗界、實然或自然界。這兩個世界間距離很大,如何能溝通而合一呢?這個問題,在康德的批判哲學中,是幽深曲折地來思索的,也可以說是相當的艱難。這個問題本也可以不必如此曲折艱思,西方哲學傳統所表現的智思與強力自始即無那道德意識所貫註的原始而通透的直悟,而其一切哲學活動皆是就特定的現象或概念,如知識、自然、道德等,而予以反省,施以步步之分解而步步建立起來的,這徵象也很顯明地表現於康德的哲學中。這樣步步分解、建立,自然不容易達到最後的融和。這是概念思考底好處,也是概念思考本身所造成的障隔。這如程明道之批評張橫渠,說他是「強探力索」,「直恁地不熟」。在這裡,我們也可以說康德是「強探力索」了,他是未至圓熟之境的。

若依宋、明儒之大宗說,道德性的天理實理是本心性體之所

\newpage\thispagestyle{empty}\addtocounter{page}{-1}\vspace*{-12mm}\begin{center}\noindent
\includegraphics[clip, trim=164pt 140pt 126pt 248pt, height=162mm]{ocr-input/image-0521.png}\end{center}

\newpage

\noindent 發。本心性體或於穆不已之道體性體是寂感真幾,是創造之源,是直貫至宇宙之生化或道德之創造。自宇宙生化之氣之跡上說以及自道德創造所引生之行為之氣之跡上說,是實然、自然,是服從自然因果律,但自創造之源上說,則是當然,是服從意志因果的。如是,則這種契合是很直接而自然的,不必須曲曲折折強探力索地去艱苦建構了。這個問題不是直接地就「道德底當然」與「自然底實然」這兩個自性絕然不同的概念而去搏鬥,一方說很難,一方說很易。依宋、明儒之大宗而說很易,(很直接而自然),關鍵不在局限於對這兩個自性絕然不同的概念之關係本身去想,而是在宋、明儒之講道德性的天理實理原不是孤離地局限地或抽象地單看「道德性」這個概念之當身,單看純然善意所先驗地自律地構成的那個「圓滿道德之理念」。他們自始就有一種通透的、具體的圓熟智慧,而把那道德性之當然滲透至充其極而達至具體清澈精誠惻怛之圓而神的境地。這里是一個絕大的原始智慧,不是概念分解的事,說艱難亦艱難,說深奧亦深奧,可是把這一關打通了,說容易亦容易。在聖人之開發此智慧原是自天而降,不是經過概念分解的,所以也可以說很簡易。但就一般人之了解這一關言,卻並不是容易的。若把這一關打通了,那道德底當然與自然底實然之契合便不是問題,而是結論了。若無這原始智慧,則只有像康德那樣認它為直接搏鬥的問題所在而去強探力索、曲折建構了。這是中國儒家言道德之當然與康德的道德哲學之最根源的而為人所不易察覺到的差異處。

我說宋明儒之大宗「把那道德性之當然滲透至充其極而達至具體清澈精誠惻怛之圓而神之境」並不是無義理指謂的憑空讚嘆之

\newpage\thispagestyle{empty}\addtocounter{page}{-1}\vspace*{-12mm}\begin{center}\noindent
\includegraphics[clip, trim=166pt 125pt 137pt 259pt, height=162mm]{ocr-input/image-0525.png}\end{center}

\newpage\markright{第一部 \quad 第三章 \quad 自律道德與道德的形上學}

\noindent 辭,乃是一、他們皆能共契先秦儒家之原始智慧之所開發而依之為矩蠖,即是說,那「達至具體清澈精誠側怛之圓而神之境」原是先秦儒家「踐仁盡性」之教所已至,也是聖人「通體是仁心德慧」之所已函,他們不過能冥契此玄微,承接之並多方闡發之而已耳。二、這「具體清澈精誠惻怛之圓而神之境」,如果想把它拆開而明其義理之實指,便是在形而上(本體宇宙論)方面與道德方面都是根據踐仁盡性,或更具體一點說,都是對應一個聖者的生命或人格而一起頓時即接觸到道德性當身之嚴整而純粹的意義,(此是第一義,)同時亦充其極,因宇宙的情懷,而達至其形而上的意義,(此是第二義,)復同時即在踐仁盡性之工夫中而為具體的表現,自函凡道德的決斷皆是存在的、具有歷史性的、獨一無二的決斷,亦是異地則皆然的決斷,(此是第三義)。

在此三義中,第一義即融攝康德《道德底形上學之基本原理》中所說之一切。因為我們不能說孔子的那個代表真實生命、代表全德、一切德所從出的「仁」是個經驗的概念,是個後天的心理學的觀念,如果我們心中默存一分解的思考方式去觀之時,孔子並沒有用「超越分解」的方式去抽象地反顯仁之為道德理性、為道德法則是先驗的,而且是普遍的。聖人是原始智慧之開發,很少用哲學家的分解建立的方式去表現道理的。就孔子的仁說,他是依其具體清激精誠側怛的襟懷,在具體生活上,作具體渾淪的指點與啟發的。我們不能說在這具體渾淪中不藏有仁道之為道德理性、之為道德的普遍法則之意,因而亦不能說這混融隱含於其中的普遍法則不是先驗的,不是對任何「理性的存在」(rational being)皆有效的。不過孔子沒有經過超越分解的方式去抽象地反顯它,而只在具體清澈

\newpage\thispagestyle{empty}\addtocounter{page}{-1}\vspace*{-12mm}\begin{center}\noindent
\includegraphics[clip, trim=158pt 137pt 132pt 247pt, height=162mm]{ocr-input/image-0529.png}\end{center}

\newpage

\noindent 精誠惻怛之真實生命中去表現它,因而仁之為普遍的法則不是抽象地懸起來的普遍法則,而是混融於精誠側怛之真實生命中而為具體的普遍,隨著具體生活之曲曲折折而如水銀瀉地,或如圓珠走盤,遍潤一切而不遺的這種具體的普遍。它的先驗性與超越性也不是反顯地孤懸在那里的先驗性與超越性,而是混融於真實生命中的內在的先驗性、具體的超越性。若說它是體,它是「全體在用」的體;若說它是用,它是「全用在體」的用。那具體清澈精誠側怛的真實生命本身就是全幅是仁道的表現。人若執滯不通,心思沈墮,執著於具體表現之一端一相而謂此即是仁或那是仁,必以為仁完全是後天的、經驗的,甚至是非常之浮淺與表面的。這當然是自己之淺陋,根本無與於孔子所言之仁。這並不可以妄假「平實」之名以為文飾,孔子之平實不如此也。「相人偶」之訓詁只是文字學教小孩的一個線索,豈是「相人偶」便能定仁之意?孟子言「仁者,人也,合而言之,道也」。此亦不錯,但若把「人道」只限於社會性的倫常道德豈能盡仁道之為人道之實義?此誠如《易傳》所言:「仁者見之謂之仁,智者見之謂之智,百姓日用而不知,故君子之道鮮矣。」對孔子之仁,我們也可以說:淺者見之謂之淺,浮者見之謂之浮,故孔子之仁陋矣。

對於道德性當身之嚴整(莊嚴)而純粹的意義,唯孔子一人是渾淪的表現,是渾然天成,是孟子所謂「堯、舜性之也」,是《中庸〉所謂「自誠明謂之性」。自孔子以下,皆有分解逆顯的意味,就孔子之渾淪表現而逆顯,把他的渾然天成打開而逆覺,是孟子所謂「湯、武反之也」,是《中庸》所謂「自明誠謂之教」。無論是孟子的「性善」,或是《中庸》的「天命之謂性」,皆是由逆覺以

\newpage\thispagestyle{empty}\addtocounter{page}{-1}\vspace*{-12mm}\begin{center}\noindent
\includegraphics[clip, trim=164pt 120pt 133pt 263pt, height=162mm]{ocr-input/image-0533.png}\end{center}

\newpage\markright{第一部 \quad 第三章 \quad 自律道德與道德的形上學}

\noindent 顯「性體」之為本,這就點出道德實踐之先天的根據,亦可曰超越的根據。孟子明由「仁義內在」以見性善,內在是表示仁義內在於超越的(非經驗的,非心理學的)道德心,是先天而固有的,「非由外鑠我也」,這是把孔子渾淪表現的仁義收攝於性體以為純粹而先天的道德理性,且不只是抽象的道德理性,而亦是必須具體呈現於那超越的道德心的。惟康德是從「自由意志」講,而中國的傳統則是喜歡從「性體」講。自由意志經由其自律性所先驗提供的普遍法則是道德行為底準繩,而依中國傳統,則是主張先驗的普遍的道德法則是性體之所展現。惟中國傳統並沒有像康德那樣,費那麼大的氣力,去分解辨解以建立它的先驗性與普遍性,而其重點則是落在「盡」字上,(盡性之盡),不是落在辨解它的先驗性與普遍性上。依儒家傳統,性體所展現的道德法則,其先驗性與普遍性,是隨著天命之性而當然定然如此的,這是不待辨解而自明的,這是由於精誠的道德意識所貫註的那原始而通透的智慧隨性體之肯定而直下肯定其為如此的。故重點不落在這種辨解上,而只落在「盡」字上。當然時下人們如沒有那種原始智慧,或不假定那原始的智慧,而有哲學辨解的趣味,則隨康德那樣去辨,也是好的。可是中國儒家傳統以前的不辨,並不能表示他們所肯定的性體以及此性體所展現的道德法則就不是先驗的與普遍的,說他們沒有這些意思,或並沒有達到康德那種崇高而嚴整的程度。其實不但並非如此,而實亦超過了康德的境界,(此義下面再說),惟沒有採取他那種辨解的方式以及其所使用的詞彙而已。非然者,孔子何以說:「有殺身以成仁,無求生以害仁」?孟子又何以說:「所欲有甚於生,所惡有甚於死」?又何以說:「君子所性,雖大行不加焉,雖窮居不損

\newpage\thispagestyle{empty}\addtocounter{page}{-1}\vspace*{-12mm}\begin{center}\noindent
\includegraphics[clip, trim=176pt 151pt 130pt 244pt, height=162mm]{ocr-input/image-0537.png}\end{center}

\newpage

\noindent 焉」?又何以說:「鄉為身死而不受,〔不受嗟來無禮之食】,今為宮室之美為之!〔竟不辨禮義而受萬鍾之祿,下同〕。鄉為身死而不受,今爲妻妾之奉爲之!鄉為身死而不受,今爲所識窮乏者得我而為之!是亦不可以已乎?此之謂失其本心!」?凡這些話俱表示在現實自然生命以上,種種外在的利害關係以外,有一超越的道德理性之標準,此即仁義、禮義、本心等字之所示。人的道德行為、道德人格只有毫無雜念毫無歧出地直立於這超越的標準上始能是純粹的,始能是真正地站立起。這超越的標準,如展為道德法則,其命於人而為人所必須依之以行,不是先驗的、普遍的,是什麼?這層意思,凡是正宗而透澈的儒者沒有不認識而斷然肯定的。宋、明儒之大宗只是把這孔、孟的規範把握得更緊、遵守得更嚴而已。其大講性體心體者,亦不過是把這超越的標準提練得更清楚更確定更不可疑而已。程明道說佛只為生死所恐動,其出發點是個利害觀念。後來陸象山就說:「看見佛之所以與儒異者,止是他底全在利,吾儒止是全在義」,這是宋明儒的共同意識。這批評,佛弟子當然不高興。然而佛教是從苦業意識出發,不從嚴整的道德意識出發,則是確然不移的。從嚴整而激底的道德意識(義)出發,是直下立根於道德理性之當身,這是不能有任何歧出與旁貸的;凡一有歧出與旁貸便是私與利。依康德,私與利是沒有先驗性與普遍性的。宋儒說佛教是私與利,當然是很深微的,並非說佛教還有世俗之私與利。依宋、明儒與康德的嚴整而澈底的道德意識出發,則普通的宗教意識俱不免私與利之情,這當然不是說的今之信教者為的是領物資與出國。

在西方哲學家中,只有康德始認真地認識了這澈底而嚴整的道

\newpage\thispagestyle{empty}\addtocounter{page}{-1}\vspace*{-12mm}\begin{center}\noindent
\includegraphics[clip, trim=163pt 133pt 137pt 250pt, height=162mm]{ocr-input/image-0541.png}\end{center}

\newpage\markright{第一部 \quad 第三章 \quad 自律道德與道德的形上學}

\noindent 德意識。在他首先說明道德法則不能從經驗得來以後,接著就說下面一段話:

\begin{quotation}\kaishu 其次,再沒有比我們想「從範例中引申出道德」這件事更是
對於道德的致命傷。因為凡是置在我面前的每一道德之範
例,其自身必須首先為道德原理所檢查,看看它是否堪充為
一原始的範例,即堪充為一範型,但它決不能即權威地供給
這道德性之概念。即使是《四福音書》中的獨一聖子,在我
們能承認祂是聖子以前,也必須先與我們的道德圓滿之理想
作比較。所以祂自己說:「你們為什麼稱你所看得見的我為
善?除你們所看不見的上帝而外,無有配稱為善者(善之模
型)!」但是我們又從那裡得有上帝為最高善之概念呢?這
只有從理性所先驗構成的,而且與自由意志之觀念不可分地
連結於一起的那道德圓滿之理念而得來。至於模倣,在道德
性中畢竟是無地位的,而範例則只可供鼓勵之用,即是說,
它們能把法則所命令者之可實行性置於無疑之地,它們能把
實踐規律所更一般地表示的使其成為可見的,但它們決不能
使我們把那存於理性中真正根源的東西置之不理,而只依範
例去指導我們自己。(《道德底形上學之基本原理〉第二節中之
一段,英人亞保特(Abbott)譯(康德的道德論〉,頁25)\end{quotation}

\noindent 這一段話很顯明地表示了道德理性之澈底透出,就是聖子與上帝也不能違背它。這正是程明道所說的「得此義理在此,甚事不盡?更有甚事出得?」一段話之意義。我們也可以說上帝就是那道德理性

\newpage\thispagestyle{empty}\addtocounter{page}{-1}\vspace*{-12mm}\begin{center}\noindent
\includegraphics[clip, trim=186pt 159pt 131pt 246pt, height=162mm]{ocr-input/image-0545.png}\end{center}

\newpage

\noindent 所先驗構成的「道德圓滿之理念」之宗教之情上的人格化。但是道德理性如真充其極,人格化與否是無足輕重的。中國儒家就是不把它人格化而為神的,不但不把它人格化,而且經由道德理性之充其極,正是把那《詩》、《書〉中原有的人格神的帝天轉化而為超越遍在的天道、天命、天理、實理或仁體、誠體、神體的。我們再看康德的另一段話:

\begin{quotation}\kaishu 要想達到這一點,〔案:即「先驗地證明實有無上命令」這
一點〕,最重要的是要記住:我們必不允許我們自己去想從
人性底特殊屬性中推演出這原則底真實性。因為義務應當是
行為底一種實踐的、無條件的必然性;所以它必須在一切理
性的存在上皆能成立,(這一切理性存在即是一定然命令所
畢竟能應用於其上者),而且只為此故,所以它亦是在一切
人類意志上皆能成立的一個法則。反之,凡是從人類之特殊
的自然特徵中推演出來的,從某種情緒與脾性(性癖)中推
演出來的,不,如其可能,甚至從適當於人類理性的任何特
殊傾向,而這不必然在每一理性存在底意志上皆有效者中推
演出來的,這雖誠可供給我們以格準,但卻決不能供給我們
以法則;可供給我們以主觀原則,使我們依之可隨脾性與性
好以行,但卻決不能供給我們以客觀原則,使我們依之必須
奉命以行,縱使一切我們的脾性、性好以及自然的性向都反
對它,我們也必須遵從之。如實言之,這義務中的命令,如
主觀衝動愈少喜愛它,或愈多反對它,其莊美性以及其內在
固具的尊嚴性就愈顯著,決不能絲毫減弱這法則底責成性或\end{quotation}

\newpage\thispagestyle{empty}\addtocounter{page}{-1}\vspace*{-12mm}\begin{center}\noindent
\includegraphics[clip, trim=146pt 134pt 147pt 249pt, height=162mm]{ocr-input/image-0549.png}\end{center}

\newpage\markright{第一部 \quad 第三章 \quad 自律道德與道德的形上學}

\begin{quotation}\kaishu 減少它的遍效性。(同上,頁34)\end{quotation}

\noindent 這一段話是非常有意義而且重要的。如果人們見到孟子講「性善」,《中庸》講「天命之謂性」,以為正宗儒家所講的人之性是康德此段話中所說的「人性底特殊屬性」之人性,「人類之特殊的自然特征」之人性,「脾性、性好以及自然的性向」或「任何特殊傾向」諸詞所表示的人性,那將是絕大的誤會。告子、荀子,下屆董仲舒、揚雄、劉向、王充乃至劉劭《人物志》所講的才性,這一系思想家所講的人性正是康德此段話中所說的人性。所以這些思想家決不說人性定然是善。他們或說是中性、「無善無不善」(告子),或說性惡(荀子),或說有善有不善(董子),或說善惡混(揚雄),或說「性不獨善,情不獨惡」(劉向),或說性分三品(王充),這些說法都可概括在「用氣為性」這一原則下,他們說的性都可以說是「氣性」。就此而言,那些說法原都是可以的,亦並不衝突。我們當然不能由這氣性建立起先驗而普遍的道德法則,或義務中的定然命令,但我們不能把孟子與《中庸》所說的性也混在這一系中一例看。孟子所說的性顯然是內在道德性當身之性,其所謂善乃是這內在道德性當身之善。此性是普遍的、先驗的,而且是純一的,並不像氣性那樣多姿多彩,個個人不同的。其善亦是定然的,並不像氣性那樣,或善或惡,或無所謂善惡的。孟子直就人的內在道德性說性,而《中庸》「天命之謂性」則推進一步把內在道德性之性通於天道、天命,不但直下是道德的,而且是本體宇宙論的,而孟子說盡心知性知天,則亦是原函蘊此義的,故云「萬物皆備於我矣,反身而誠,樂莫大焉。」這種論性顯然是從由自然生

\newpage\thispagestyle{empty}\addtocounter{page}{-1}\vspace*{-12mm}\begin{center}\noindent
\includegraphics[clip, trim=188pt 175pt 137pt 236pt, height=162mm]{ocr-input/image-0553.png}\end{center}

\newpage

\noindent 命的種種特征說性,即從氣性說性,來一個超越的大解放,從自然生命提高一層,開闢人的精神生命,建立人的理性生命。對氣性而言,這可以說是「用理為性」。(不是伊川、朱子脫落了心的「只是理」之理。)康德說的那自由自主自律而絕對善的意志,若照正宗儒家看,那正是他們說的本心即性。(康德卻並未把這視為人之「性」,註意。)明朝最後一個理學家劉蕺山講誠意慎獨正是說的這種意志。這「用理為性」一路,孟子而後,自荀子起,下屆兩漢傳統,是無人能了解的,但他們卻彰顯了「用氣為性」一路。(一般人說人性以及西方人所說的「人性」大抵都是說的氣性。)這兩路會合於宋儒,便成了他們所嚴格分別的義理之性(天地之性、本源之性)與氣質之性。(不是朱子所講者。)宋、明儒之大宗始真緊守孟子、《中庸》所開關的「超越的心性」而著力前進的。所以人們若見康德這段話,便以為儒家從人性建立道德是軟罷無力,正好是建立不起的,這意見即表示他根本未接觸到正宗儒家所說的人性是什麼。我們現在很感謝康德這段話,由他這段話,人性論底不同層面與分際完全撐開了,可使吾人簡別得很清楚。(關於此處所說,請參看拙著《才性與玄理》第一章。)

由以上,康德以為道德法則:

1.不能從經驗建立;

2.不能從「範例」引申;

3.不能從「人性底特殊屬性」、「人類之特殊的自然特征」、
「脾性(性癖)、性好以及自然的性向「propensions,
inclinations and natural dispositions)推演;

4.甚至亦不能從「上帝底意志」來建立。

\newpage\thispagestyle{empty}\addtocounter{page}{-1}\vspace*{-12mm}\begin{center}\noindent
\includegraphics[clip, trim=154pt 143pt 151pt 245pt, height=162mm]{ocr-input/image-0557.png}\end{center}

\newpage\markright{第一部 \quad 第三章 \quad 自律道德與道德的形上學}

\noindent 由這一切所建立的道德法則以決定我們的意志,都是康德所謂「意志之他律」(heteronomy of the will)。康德把「基於他律之概念而來的一切道德原則」分為兩類:「或為經驗的,或為理性的。前一類從「幸福』底原則而抽引出,它是建築於物理的(生理的)或道德的情感之上的;後一類從圓滿』底原則而抽引出,它或是建築於當作一個可能的結果看的那『圓滿』之理性概念上,或是建築於作為我們的意志之決定因的一個「獨立的圓滿』(上帝底意志)之理性概念上」。(《道德底形上學之基本原理〉第二節中「基於他律之概念而來的一切道德原則之分類」一段。英人亞保特譯本頁60)。

此中關於經驗的一類,康德說:

\begin{quotation}\kaishu 所有經驗的原則皆完全不能用來充當道德法則之基礎。因為
它們的基礎是從人性底特殊構造中或從它(人性)所處的偶
然環境中而得來時,則它們(這些道德法則)所以之以在一
切理性的存在上(並無簡別)皆成立(皆有效)的那普遍
性,並因而被安置於它們身上的那無條件的、實踐的必然
性,自然喪失無餘。但是私人幸福底原則是最有問題最該反
對的,這不只是因為它是假的,而經驗亦與「榮華富貴常正
比於善行」這假設相衝突,又亦不只是因為它對於道德底建
立無所貢獻,(因為作成一有福祿之人與作成一善良之人,
或使一人謹慎精察於其自己之利益與使他為有德,這完全是
不同的事),而且亦因為它所供給於道德的動力(興發之
力)毋寧反是暗中敗壞它,而且破壞了它的莊嚴性,因為它\end{quotation}

\newpage\thispagestyle{empty}\addtocounter{page}{-1}\vspace*{-12mm}\begin{center}\noindent
\includegraphics[clip, trim=182pt 174pt 128pt 227pt, height=162mm]{ocr-input/image-0561.png}\end{center}

\newpage

\begin{quotation}\kaishu 所供給的這些動力是把「存心於德」與「存心於壞」置放於
同類,而只教導我們去作較好的計算,而德與不德(壞)之
間的特殊差別則完全給掃滅了。另一方面,關於道德情感,
這設想的特別感覺(即道德感),當那些不能「思考」的人
相信這種「情感」將有助於他們,甚至在涉及一般法則中亦
有助於他們時,去訴諸這種情感,這實在是非常浮淺的;此
外,情感之為物,它天然在程度上有無限地差別變化,它對
於善與惡不能供給一統一的標準,而任何人也不能有權利以
其自己之情感去為他人形成一判斷:不過縱然如此,這種道
德的情感(道德感)亦尚是彌近於道德以及其尊嚴性的,
即,它將「我們對於美德所有的滿足與崇敬直接地歸給於美
德」這種光榮付於美德,而且好像是決不當她的面告訴她
說:我們不是因著她的美而卻是因著利益而親近(愛慕)
她。(同上,英譯本,頁60-61)\end{quotation}

\noindent 康德這段話是把私人幸福原則與道德情感(道德感)俱視為經驗原則,即後天的原則。這一方面是因為它有待於外,一方面亦因為它完全根據於那純主觀的「人性之特殊構造」。因此,由之而建立的道德法則自無普遍性與必然性,嚴格說這實亦不真是道德法則。關此,私人幸福原則這方面自無問題,但關於道德情感(道德感)這方面,則有申說之必要。康德所說的道德情感、道德感,是著眼於其實然的層面,其底子是發自「人性底特殊構造」,(這里所說的「人性之特殊構造」與上文所引關於「人性」脾性、性好、性向一段相呼應),而又註意其「同情他人底幸福」之意。這種落於實

\newpage\thispagestyle{empty}\addtocounter{page}{-1}\vspace*{-12mm}\begin{center}\noindent
\includegraphics[clip, trim=155pt 152pt 143pt 232pt, height=162mm]{ocr-input/image-0565.png}\end{center}

\newpage\markright{第一部 \quad 第三章 \quad 自律道德與道德的形上學}

\noindent 然層面的道德感、道德情感,有類於董仲舒一類所說的由氣性材質之性而發的仁愛之情,這當然可劃於私人幸福原則之下,因而亦當然是經驗的、後天的,而且亦無定準。但道德感、道德情感可以上下其講。下講,則落於實然層面,自不能由之建立道德法則,但亦可以上提而至超越的層面,使之成為道德法則、道德理性之表現上最為本質的一環。然則在什麼關節上,它始可以提至超越的層面,而為最本質的一環呢?依正宗儒家說,即在作實踐的工夫以體現性體這關節上,依康德的詞語說,即在作實踐的工夫以體現、表現道德法則、無上命令這關節上;但這一層是康德的道德哲學所未曾註意的,而卻為正宗儒家講說義理的主要課題。在此關節上,道德感、道德情感不是落在實然層面上,乃上提至超越層面轉而為具體的,而又是普遍的道德之情與道德之心,此所以宋、明儒上繼先秦儒家大講性體,而又大講心體,最後又必是性體心體合一之故。此時「道德感」不是如康德所說的那「設想的特別感覺」,而「道德情感」亦不是如他所說的「在程度上天然有無限地差別變化,它對於善與惡不能供給一統一的標準」這實然的純主觀的道德情感,而是轉而為超越而又內在、普遍而又特殊的那具體的道德之情與道德之心。

這種心、情,上溯其原初的根源,是孔子渾全表現的「仁」:不安、不忍之感,悱惻之感,悱啟憤發之情,不厭不倦、健行不息之德等等。這一切轉而為孟子所言的心性:其中惻隱、羞惡、辭讓、是非等是心,是情,也是理。理固是超越的、普遍的、先天的,但這理不只是抽象地普遍的,而且即在具體的心與情中見,故為具體地普遍的;而心與情亦因其即為理之具體而真實的表現,故

\newpage\thispagestyle{empty}\addtocounter{page}{-1}\vspace*{-12mm}\begin{center}\noindent
\includegraphics[clip, trim=160pt 163pt 142pt 231pt, height=162mm]{ocr-input/image-0569.png}\end{center}

\newpage

\noindent 亦上提而為超越的、普遍的、亦主亦客的,不是實然層上的純主觀,其為具體是超越而普遍的具體,其為特殊亦是超越而普遍的特殊,不是實然層上的純具體、純特殊。這是孟子磐磐大才的直悟所開發。到陸象山便直以此為道德性的本心與宇宙心:這當然不是一個抽象的乾枯的光板的智心,故理在其中,情也在其中,故能興發那純粹的道德行為、道德創造,直下全部是道德意識在貫註,全部是道德義理在支柱,全部是道德心、情在開朗、在潤澤,朗天照地,了無纖塵。到王陽明則復將此本心一轉而為良知:良知是認識此本心之訣竅,亦是本心直接與具體生活發生指導、主宰關係之指南針;而良知之内容亦不只是光板的、作用的明覺,而是羞惡、辭讓、是非、側隱全在內的心體之全,故陽明總言「良知之天理」,亦總言「精誠惻怛」之本心:這也是是理,也是情,也是心。

把這上提至超越層的心與情體現到圓而神之境的便是聖人。聖人並非無情。而凡中國以前了解聖人之情的,無不就圓而神的最高境界說。此無論就儒家或道家說皆然。王弼說「聖人茂於人者神明也,同於人者五情也。神明茂,故能體沖和以通無;五情同,故不能無哀樂以應物。然則聖人之情,應物而無累於物者也。今以其無累,便謂不復應物,失之多矣。」(參看《才性與玄理〉第三章與第四章)。此言情雖就同於人之五情而通說,似不同於康德所說的道德情感、道德感,然是聖人之情,則必非下等無色之心理學的情,雖是就同於人之五情說,然聖人之五情實已全部是道德之情,不能離乎道德感的,不過其表現的獨是圓通無礙而已。「神明茂,故能體沖和以通無」,此雖是根據道家的觀念說,然在境界上亦通儒聖。「體沖和以通無」就是先有超越之體。然體非抽象之空掛,

\newpage\thispagestyle{empty}\addtocounter{page}{-1}\vspace*{-12mm}\begin{center}\noindent
\includegraphics[clip, trim=177pt 144pt 135pt 248pt, height=162mm]{ocr-input/image-0573.png}\end{center}

\newpage\markright{第一部 \quad 第三章 \quad 自律道德與道德的形上學}

\noindent 故不能不在有中表現。「五情同,故不能無哀樂以應物」,即是處有應物,和光同塵,所以成其為圓教也。照儒家說,則聖人的生命全體是理,全體是心,亦全體是情,故為圓而神。無情不能應物,情焉可缺哉?此情之原初開始的意義當然就是道德感、道德情感,不過至此已提至超越圓熟之境而已。王弼將此境盛弘之於前,至唐李習之作〈復性書〉亦說「聖人者,豈其無情也?聖人者,寂然不動,不往而到,不言而神,不耀而光,〔……〕雖有情也,未嘗有情也。」至程明道(定性書〉則言「天地之常,以其心普萬物而無心;聖人之常,以其情順萬事而無情。」此種境界,宋、明儒者無不承認。

吾所以縷述至此者,即在明惟重視由實踐工夫以體現性體心體者始能正式正視道德感、道德情感而把它上提至超越層面而定住其道德實踐上的本質意義,然而康德則不能至此,他只把它停在實然層面上,故歸之於私人幸福原則之下,而視之為經驗原則。道德感、道德情感,如不能予以開展而把它貞定得住,則道德實踐即不能言。正因康德之道德哲學無自實踐工夫以體現性體心體一義,故亦不能正視此道德感、道德情感也。他只是由抽象的思考,以顯道德之體,他只是經驗的與超越的對翻,有條件的與無條件的對翻,此已極顯道德之本性矣,惜乎未至具體地(存在地)體現此「道德之體」之階段,故只言道德法則、無上命令(定然命令)之普遍性與必然性,而對於超越之心與情則俱未能正視也。若以儒家義理衡之,康德的境界,是類乎尊性卑心而賤情者。(當然康德並未把他所講的自由自主自律而絕對善的意志連同著它的道德法則無上命令視為人之「性」,但儒家卻可以這樣看。註意。)

\newpage\thispagestyle{empty}\addtocounter{page}{-1}\vspace*{-12mm}\begin{center}\noindent
\includegraphics[clip, trim=148pt 148pt 141pt 238pt, height=162mm]{ocr-input/image-0577.png}\end{center}

\newpage

茲且順其境界,再看其關於「理性的」一類之說明:

\begin{quotation}\kaishu 在理性的道德原則之中,本體論的「圓滿之概念」,儘管有
缺點,亦比神學的概念之從一個神性的、絕對圓滿的意志中
引申出道德為較好。前一概念無疑是空洞而不確定的,因而
對於我們在這可能的實在之無邊廣野中去尋求那最適合於我
們者,亦無多大用處;復次,在想特別去分清我們現在所說
的實在與每一其他實在之不同上,那也不可免地要落於循環
中,而且不能避免默默預定那所要去說明的道德;縱然如
此,它還是比神學的觀點較為可取。首先,因為我們對於
「神的圓滿」並無直覺,這只能從我們自己的概念中,(其
中最重要的就是道德之概念),把它推演出來,這樣,我們
的說明必陷於兜圈子中;其次,如果我們想要避免這兜圈
子,則那剩留給我們的唯一神的意志之觀念無非是以欲求榮
耀與統治這些屬性而造成,並與威力和報復這些可怕的概念
相結合,而凡建築在這基礎上的任何道德系統必直接相反於
道德。(同上,英譯本,頁6162)\end{quotation}

\noindent 這一段話即可表明上列第四點不能從「上帝底意志」來建立道德法則之意,亦與前文所引關於「範例」一段相呼應。「圓滿」底概,念,不管是那一型,雖較經驗的一類為進一步,然由之而引出的道德原則畢竟仍是屬於意志之他律的。最純淨而能保持道德自性的道德法則必須是「意志底自律」(autonomy of the will),即意志自身給它自己立法,這不涉於感覺經驗,亦不涉於任何外在的對

\newpage\thispagestyle{empty}\addtocounter{page}{-1}\vspace*{-12mm}\begin{center}\noindent
\includegraphics[clip, trim=150pt 140pt 137pt 259pt, height=162mm]{ocr-input/image-0581.png}\end{center}

\newpage\markright{第一部 \quad 第三章 \quad 自律道德與道德的形上學}

\noindent 象,即意志之遵依法則而行純是無條件的、必然的。試再看康德繼上引文而來的說明:

\begin{quotation}\kaishu 在任何情形中,凡必須要預定意志之對象,決定意志的那個
規律才能被規定出來,則這規律簡單地說就只是他律;此中
之命令必是有條件的,即是,「如果」或「因為」一個人願
望這個對象,所以他才一定要如此如此行:因此,它決不能
道德地即定然地命令著我們。對象決定意志,不管是因著性
好,如在私人幸福底原則中,或因著「導向於我們可能的決
意(一般地說)底對象」的那理性,如在圓滿底原則中,不
管是那種情形,總之這時的意志總不是因著行為之概念(即
行為本身)來直接地決定它自己,但只是因著行為底預見結
果在意志上所有的影響而決定它自己;這樣,我應當去作某
事,是因為我願望某種別的事;而在這裡,又必須在我這方
面預定另一法則以為它的主體,因著這另一法則,我必然地
意欲這「別的事」,而這個法則又需要一命令去限制這格
準。因為在我們的諸般機能所及的範圍內,一個對象底概念
在主體底意志上,結果亦在它的自然特性上,所運作的影
響,是依靠著主體底自然(本性)的,這自然本性或是感性
(性好與趣味),或是知性與理性,這諸般自然本性(自然
機能)之使用因著它們的自然本性之特殊構造而伴隨著以滿
足。這樣,恰當地說,那法則必是為自然所供給,而它也必
須因著經驗而被知與被證明,結果亦必是偶然的,因此亦不
能是一必然的實踐規律,如道德規律之所必是者。不惟如\end{quotation}

\newpage\thispagestyle{empty}\addtocounter{page}{-1}\vspace*{-12mm}\begin{center}\noindent
\includegraphics[clip, trim=147pt 149pt 144pt 243pt, height=162mm]{ocr-input/image-0585.png}\end{center}

\newpage

\begin{quotation}\kaishu 此,而且不可避免地它亦只是他律的;意志自身不能給它自
己以法則,這法則但只因著一種外來的衝動,憑藉著「主體
底特殊的自然構造適宜於去接受這衝動」,而被給與。依
是,一個絕對善的意志,(其原則必須是一定然命令),它
在關於一切對象上,將是不決定的,而且將只包含著一般說
的「決意之形式」,而這決意之形式當作自律看,(即是
說,每一善意之格準能使它們自己成為一普遍法則),其自
身就是每一理性存在底意志所安置於其自己身上的那唯一法
則,而不須去預定任何衝力(興發之力)或興趣作為一個基
礎。(同上,英譯本,頁62-63)\end{quotation}

\noindent 康德將屬於他律性的一切道德原則,或是屬於經驗的,由幸福原則而引出者,或是屬於理性的,由圓滿原則而引出者,盡皆剔除,而唯自「意志之自律」以觀道德法則,這在顯露「道德性當身之體」上說,(這是關於道德理性的第一義),可謂充其極矣。這也是「截斷衆流」句也。凡是涉及任何對象,由對象之特性以決定意志,所成之道德原則,這原則便是歧出不真的原則,就意志言,便是意志之他律。意志而他律,則意志之決意要做某事便是有條件的,是為的要得到什麼別的事而作的,此時意志便不直不純,這是曲的意志,因而亦是被外來的東西所決定所支配的意志、被動的意志,便不是自主自律而直立得起的意志,因而亦不是道德地、絕對地善的意志,而它的法則亦不能成為普遍的與必然的。不要說那屬於經驗的私人幸福原則建立不起有普遍性與必然性的道德法則,直立不起我們的道德意志,就是那屬於理性的圓滿原則,不管是本體

\newpage\thispagestyle{empty}\addtocounter{page}{-1}\vspace*{-12mm}\begin{center}\noindent
\includegraphics[clip, trim=182pt 136pt 132pt 263pt, height=162mm]{ocr-input/image-0589.png}\end{center}

\newpage\markright{第一部 \quad 第三章 \quad 自律道德與道德的形上學}

\noindent 論的圓滿概念(這是指柏拉圖傳統說),或是神學的圓滿概念,即一個屬於上帝意志的那獨立的圓滿概念,亦皆不能使吾人由之建立起有普遍性與必然性的道德法則,因而亦皆不能直立起我們的道德意志:一個是使我們的意志潛伏於客觀而外在的本質底秩序中,一個是使我們的意志蜷伏模糊於那「可怕的威權與報復」中或「榮耀與統治」中,而這後者尤其「直接相反於道德」,如康德之所說。道德是要從外在的牽連中收回來、四無傍依地單看我們自己之「存心」始能顯出來。故康德說:「一個絕對善的意志,(其原則必須是一定然命令),在關于一切對象上將是不決定的,而且將只包含著一般說的決意之形式」。這「形式」就是它的「定然命令」所表示的,這命令不是照顧著任何對象而形成的,故它一無內容(材料),而只是一個「形式」。這只是從吾人的道德決意之「最初的存心」說:不為別的,但只是理上義上應當如此;只是一個義之應當、理之必然,故無任何經驗內容也。這只是一個「方向」,意志所自律的方向。這樣的意志就叫做是「絕對善的意志」,最純正的意志,亦就是最道德的意志。康德只說到這個程度,這已經是很卓絕的了。他的辨解路數可以簡單地這樣列出,即:他由道德法則底普遍性與必然性逼至意志底自律,由意志底自律逼至意志自由底假定。

不幸地是他視「意志自由」為一假定、為一「設準」。至於這設準本身如何可能,它的「絕對必然性」如何可能,這不是人類理性所能解答的,亦不是我們的理性知識所能及的。(這點下節詳論。)這樣,意志底自律也只成了空說,即只是理當如此。這裡我們可以看出,康德的抽象思考,步步分解建立的方式,就道德言道

\newpage\thispagestyle{empty}\addtocounter{page}{-1}\vspace*{-12mm}\begin{center}\noindent
\includegraphics[clip, trim=162pt 153pt 144pt 243pt, height=162mm]{ocr-input/image-0593.png}\end{center}

\newpage

\noindent 德,是只講到理上當該如此,至於事實上是否真實如此,則非吾人所能知。因為如果自由只是一假定,則自律也不能落實,而他「截斷眾流」所建立的道德法則如何如何也只是一套空理論,都不能落實。事實上,我們是否有這樣的「意志」呢?我看,康德亦只是作到理上當該有,否則真正的道德不能講。至於這樣的意志是否有一真實,是一「呈現」,康德根本不能答覆這問題。但如果不能答覆這問題,則空講一套道德理論亦無用。但道德是真實,道德生活亦是真實,不是虛構的空理論。所以這樣的意志也必須是真實,是呈現。(儘管在感覺經驗內不能呈現。)康德在其辨解思考底過程上,對於自律與自由當然有其步驟上的區別。由道德法則底普遍性與必然性逼至意志底自律,至此為止所說的一切都只是分析的,你可以說這只是理上當該如此,只是一套空理論,但由意志底自律逼至「意志自由為一設準」,這已進到批判考察底範圍,即在批判考察中要建立這設準,這似乎不只是那屬於分析的之「理上當該如此」。試看康德直接繼上段引文而來的最後一段說:

\begin{quotation}\kaishu 這樣一個先驗實踐的綜和命題,〔案:即指上段引文末後一
句中只是作為「決意之形式」的那定然命令說〕,如何可
能?它為什麼又是必然的?這問題底解答不屬於「道德底形
上學」之範圍;而我們在這裡亦沒有肯定它的真理性,更沒
有自認說在我們的力量內能證明它。我們只是因著那普遍被
接受的道德觀念之發展而展示出意志之自律性是不可免地與
它相連結,甚或毋寧說是它的基礎。依是,不管是誰,只要
他認道德是任何真實的東西,而不是一無任何真理性的虛幻\end{quotation}

\newpage\thispagestyle{empty}\addtocounter{page}{-1}\vspace*{-12mm}\begin{center}\noindent
\includegraphics[clip, trim=171pt 133pt 128pt 254pt, height=162mm]{ocr-input/image-0597.png}\end{center}

\newpage\markright{第一部 \quad 第三章 \quad 自律道德與道德的形上學}

\begin{quotation}\kaishu 觀念,則他亦必同樣承認我們這裡所論定的道德之原則。依
是,本節也像第一節,純然是分析的。現在,只要證明道德
不是腦筋底製造物,(如果定然命令以及同著它的意志之自
律都是真的,而且作為一先驗原則又是絕對必然的,則道德
便不能是腦筋底製造物),這點便即假定了純粹實踐理性底
綜和使用之可能性,但是這一層在對於理性底這種機能未曾
先給一批判的考察以前,我們不能冒險前進。在下面最後一
節中,我們將對這種批判的考察給以大體的綱要,至對我們
的目的足夠為止。(同上,英譯本,頁63-64)\end{quotation}

\noindent 案:《道德底形上學之基本原理》共分三節,第一節與第二節,如康德所云,純然是分析的,第三節則是作大體的批判考察,標題曰「從道德底形上學轉到純粹實踐理性之批判」。在這一節中,即點出「自由」這個概念,視之為一必要的假定,(在《實踐理性批判》中即名之為「設準」),以使那「先驗實踐的綜和命題」為可能。《道德底形上學【……】》以前兩節「分析的」爲主文,以第三節「批判的」為《實踐理性批判》一書作準備。由這區別,我們可知康德對於自律與自由的想法是有步骤上的不同的。自律是在「分析的」講法中被建立,即康德所說「只是因著那普遍被接受的道德觀念之發展而展示出意志之自律性是不可免地與它(道德命令)相連結」,而「自由」則是在「批判的」講法中被假定。但這種分別並不能免除我之說他所作到的「只是理上當該如此,只是一套空理論」。他之批判地假定自由為一設準亦還只是理上逼迫著要如此的,而正因為這只是一假定、一設準,而不能講到它的真實性

\newpage\thispagestyle{empty}\addtocounter{page}{-1}\vspace*{-12mm}\begin{center}\noindent
\includegraphics[clip, trim=158pt 142pt 141pt 247pt, height=162mm]{ocr-input/image-0601.png}\end{center}

\newpage

\noindent 是一「呈現」,所以我才說他所講的「只是一套空理論」。「自由」落了空,其他分析的講法自亦全部都是空的,全部只是「理上當如此」而不能確定其是否是事實上可呈現的真實。康德雖然說:「不管是誰,只要他認道德是任何真實的東西,而不是一無任何真理性的虛幻觀念,則他亦必同樣承認我們這裡所論定的道德之原則」,但是他「論定」的只是「理上當如此」,他亦只作到以「理上當如此」來和那「虛幻觀念」相翻。我說康德的「論定」只是「理上當如此」的空理論,是再提升一層說,是就其「自由」為一假設而不是一呈現說。這樣,他所論定的仍不能恢復道德之為真實的東西。依我看,意志底自律與意志底自由,康德雖然分兩步講,其實是同意語。由道德法則底先驗性、普遍性與必然性分析地逼至意志之自律,與由意志之自律批判地假定意志之自由為一設準,實無多大的差別。故康德直說:「自由之概念是說明意志底自律之秘鑰」。(那種步驟上的不同實只是哲學專家學究式的思考工巧。)依是,如果「意志之自由」只是一假設,不是一呈現,則意志之自律是否是一呈現,即「意志自身給它自己以法則」是否是一呈現,是否真有這回事,是否真有這樣的意志,那當然要成問題。此即吾所以說「全部落空」之故。關於這層意思,我現在只這樣簡單地略提於此,下節詳論。

〔附識:在此我想乘機對於康德《道德底形上學之基本原理》一書之題名略加幾句註語。依此書之主文是「分析的」,再依《純理批判》之用語而言,「道德底形上學」(metaphysicsofmorals)實即是「道德之形上的解析」(metaphysical exposition ofmorals),或曰「道德之形上的推述」(metaphysical deduction of

\newpage\thispagestyle{empty}\addtocounter{page}{-1}\vspace*{-12mm}\begin{center}\noindent
\includegraphics[clip, trim=164pt 116pt 124pt 260pt, height=162mm]{ocr-input/image-0605.png}\end{center}

\newpage\markright{第一部 \quad 第三章 \quad 自律道德與道德的形上學}

\noindent morals)。然則這整題名底確切意義實當是:「通過道德之形上的解析而見的道德之基本原理。」這當然很絡索。依康德《純理批判》中的規定,「形而上的解析」就是對於一個概念底先驗本性之說明,依中國之思路說,就是對於一個概念之「體的說明」。依是,這書名實只應簡單地題曰:「道德之形上的解析」,便甚確切而亦足夠。只因解析的很多,成了一套,不只是對於一個詞語的簡單解析,遂錫以專名而名之曰《道德底形上學之基本原理》。這在英文底語法構造裡本不成問題,(想在德文裡限制的必更顯明),但譯成中文,當「底」、「的」不分的時候,則「道德底形上學」便與「道德的形上學」(moral metaphysics)簡直無法分別,尤其當「底」、「的」皆可省略的時候,結果都成了「道德形上學」,這是很容易混同而失旨的。實則康德只有「道德底形上學」(=「道德之形上的解析」)與「道德的神學」(moraltheology),而卻並無「道德的形上學」(moral metaphysics)。本文是想根據儒家要講出一個「道德的形上學」來,不只是「道德之形上的解析」(「道德底形上學」)。若以「道德的形上學」為準,則康德此書之題名,雖字面上為「道德底形上學」,然為避免混擾起見,實只應想為「道德之形上的解析」。此則要求讀者密切註意。至於康德何以只有「道德之形上的解析」與「道德的神學」,而卻並無「道德的形上學」之故,則將在下節中詳論。]

茲再言歸正文。

照儒家的義理說,這樣的意志自始就必須被肯定是真實、是呈現。這裡,我先就「截斷眾流」這一關說。他們是把這樣的意志視為我們的性體心體之一德、一作用。這性體心體是必須被肯定為定

\newpage\thispagestyle{empty}\addtocounter{page}{-1}\vspace*{-12mm}\begin{center}\noindent
\includegraphics[clip, trim=153pt 146pt 137pt 237pt, height=162mm]{ocr-input/image-0609.png}\end{center}

\newpage

\noindent 然地真實的,是就成德成聖而言人人俱有的。人固以道德而決定其價值,但反之,道德亦必須就人之能成德而向成聖之理想人格趨始能得其決定性之真實。在這裡,道德固然不能空講,而人亦不能只是人類學地講。人在其道德的實踐以完成其德性人格底發展上是必然要肯定這性體心體之為定然地真實的,而且即在其實踐的過程中步步證實其為真實為呈現。照正宗的儒家說,一看到康德講這樣的意志,他們馬上就能默契首肯,而且必須視為我們的性體心體之一德。其所以肯定這樣性體心體之為定然地真實的,之為人人所皆固有的「性」,其密意即在能使這樣的意志成為真實的、呈現的。(這是正宗儒家講「性」的密意。)但是康德卻未註意這一層。(康德後的發展卻是向此趨,見下第三節。)康德所說的人性只是人類所具有的諸般自然機能,如感性、知性、理性等是,即他所說的「人性底特殊屬性」、「人性底特殊構造」、「人類之特殊的自然特征」、「脾性、性好、性向」諸詞所表示的人性,但卻未以他由講道德所逼至的自律、自由的意志為人的性,故視之為假設而落了空,成為人類理性所不能及、知識所不能至的隔絕領域。

正宗儒家肯定這樣的性體心體之為定然地真實的,肯定康德所講的自由自律的意志即為此性體心體之一德,故其所透顯所自律的道德法則自然有普遍性與必然性,自然斬斷一切外在的牽連而為定然的、無條件的,因此才能有「存心純正,不為別的,但為義故」的道德行為,如:「有殺身以成仁,無求生以害仁」,「所欲有甚於生,所惡有甚於死」等語之所示。孟子說:「廣土衆民,君子欲之,所樂不存焉。中天下而立,定四海之民,君子樂之,所性不存焉。君子所性,雖大行不加焉,雖窮居不損焉,分定故也。」由

\newpage\thispagestyle{empty}\addtocounter{page}{-1}\vspace*{-12mm}\begin{center}\noindent
\includegraphics[clip, trim=165pt 122pt 126pt 254pt, height=162mm]{ocr-input/image-0613.png}\end{center}

\newpage\markright{第一部 \quad 第三章 \quad 自律道德與道德的形上學}

\noindent 「所欲」、「所樂」向裡收,直至「所性」而後止,這才真見出道德人格之尊嚴,這也就是康德所說的「一個絕對善的意志在關於一切對象上將是不決定的」一語之意,必須把一切外在對象的牽連斬斷,始能顯出意志底自律,照儒家說,始能顯出性體心體底主宰性。這是「截斷眾流」句,就是本節開頭所說的關於道德理性底第一義。其次,這為定然地真實的性體心體不只是人的性,不只是成就嚴整而純正的道德行為,而且直透至其形而上的宇宙論的意義,而為天地之性,而為宇宙萬物底實體本體,為寂感真幾、生化之理,這是「涵蓋乾坤」句,是道德理性底第二義。最後,這道德性的性體心體不只是在截斷眾流上只顯為定然命令之純形式義,只顯為道德法則之普遍性與必然性,而且還要在具體生活上通過實踐的體現工夫,所謂「盡性」,作具體而真實的表現,這就是「隨波逐浪」句,是道德理性底第三義。這是儒家言道德理性充其極而為最完整的一個圓融的整體,是康德所不能及的。道德性的實理天理之與實然自然相契合以及「道德的形上學」(不是「道德底形上學」)之澈底完成,都要靠這三義澈底透出而可能。這是以後兩節所要討論的。

\section{康德所以只有「道德的神學」而無「道
德的形上學」之故}

以上我們說明了儒家根據踐仁盡性頓時即接觸到了「道德當身之嚴整而純粹的意義」這第一義,並以為已融攝了康德的《道德底形上學之基本原理》中所說之一切。但我們前面亦屢提到儒家不只

\newpage\thispagestyle{empty}\addtocounter{page}{-1}\vspace*{-12mm}\begin{center}\noindent
\includegraphics[clip, trim=170pt 151pt 143pt 248pt, height=162mm]{ocr-input/image-0617.png}\end{center}

\newpage

\noindent 頓時接觸到了這第一義,還同時充其極而至第二義與第三義。對康德說,這境界是超過了康德而為康德所不及的。康德之達不到這境界,只就其《道德底形上學之基本原理》與《實踐理性批判》這兩本書的表現即可看出。他之所以達不到這境界,一、是因為他那步步分解建構的思考方式限制住了他,他缺乏那原始而通透的具體智慧;二、他無一個具體清澈、精誠側怛的渾淪表現之圓而神的聖人生命為其先在之矩穫,所以他只有停在步步分解建構的強探力索之境了。可是他這步步分解建構強探力索地前進卻正是向儒家這個智慧型態而趨的。我看他的系統之最後圓熟的歸宿當該是聖人的具體清澈精誠側怛的圓而神之境。他的分解工作之功績是不可泯滅的。由他開始,經過費息特、黑格爾,以至謝林這發展的傳統,即已表示出這趨勢,雖然有許多生硬不妥貼處,還待繼續淘濾與融化。

康德之達不到第二義的境界(即「同時亦充其極,因宇宙的情懷,而達至道德理性之形而上的宇宙論的意義」這第二義),具體地說出來,即在他只有《道德底形上學之基本原理)(FundamentalPrinciples of the Metaphysic of Morals)與(實踐理性批判》所建立的「道德的神學」(moral theology),而卻無(至少未充分實現)根據其分解建立的道德理性所先驗供給的客觀的道德法則再進一步展現出一個具體而圓熟的「道德的形上學」(moralmetaphysics)。「道德底形上學」與「道德的形上學」這兩個名稱是不同的。(「底」與「的」的使用從今日通行的使用。在朱子,「底」作形容詞,如是「的」字當作所有格用,與今日正相反,唯朱子很少用「的」字,馮友蘭的使用是嚴格遵守朱子的。至於「地」字,則今日與朱子同,同是表示副詞、形容動詞的。我之

\newpage\thispagestyle{empty}\addtocounter{page}{-1}\vspace*{-12mm}\begin{center}\noindent
\includegraphics[clip, trim=159pt 125pt 127pt 250pt, height=162mm]{ocr-input/image-0621.png}\end{center}

\newpage\markright{第一部 \quad 第三章 \quad 自律道德與道德的形上學}

\noindent 行文亦不嚴格地如此麻煩。惟譯文則嚴格遵守以示分別。)前者是關於「道德」的一種形上學的研究,以形上地討論道德本身之基本原理為主,其所研究的題材是道德,而不是「形上學」本身,形上學是借用。後者則是以形上學本身為主,(包含本體論與宇宙論),而從「道德的進路」入,以由「道德性當身」所見的本源(心性)滲透至宇宙之本源,此就是由道德而進至形上學了,但卻是由「道德的進路」入,故曰「道德的形上學」,亦猶之乎康德由實踐理性而接近上帝與靈魂不滅而建立其客觀妥實性,因而就神學言,即名曰「道德的神學」。但康德只就其宗教的傳統而建立「道德的神學」,卻未能四無傍依地就其所形式地透顯的實踐理性而充分展現一具體的「道德的形上學」。

這個問題底關鍵是在:

他所分解表現並且批判表現的實踐理性只是形式地建立,一方未能本著一種宇宙的情懷而透至其形而上的、宇宙論的意義,一方亦未能從工夫上著重其「如何體現」這種真正實踐的意義,即所謂「踐仁盡性」的實踐工夫,因而其實踐理性、意志自由所自律的無上命令只在抽象的理上的當然狀態中,而未能正視其「當下呈現」而亦仍是「照體獨立」的具體狀態。依儒家說,無論是「堯舜性之」,或「湯武反之」,無論是「即本體便是工夫」,或「即工夫便是本體」,這無上命令,因而連帶著發這無上命令的自由自主自律之意志、心性,都是隨時在具體呈現的。然而這境界,康德未能至,此即是人們所以常稱之曰形式主義之故。(形式主義是第一步,並不錯,只是不盡)。

\newpage\thispagestyle{empty}\addtocounter{page}{-1}\vspace*{-12mm}\begin{center}\noindent
\includegraphics[clip, trim=168pt 193pt 143pt 262pt, height=162mm]{ocr-input/image-0625.png}\end{center}

\newpage

\noindent 甲、康德視意志自由為不可解明的一個設準

康德視「意志自由」與「靈魂不滅」及「上帝存在」同為純粹的實踐理性之設準,然其實這三者雖可並列地皆視之為設準,而卻並不可同等看。意志自由之設準是與實踐理性直接相連,而其他兩者則卻遠一層。此點,康德自己已經知之。他在《實踐理性批判·序文》中說:

\begin{quotation}\kaishu 只要當自由概念底實在性因實踐理性底必然法則而被證明
時,則它便是純粹理性甚至思辨理性底全部系統之拱心石,
而一切其他概念(如上帝與靈魂不滅之概念)若當作只是理
念看,它們本是沒有任何支持的,但現在卻把它們自己連屬
於自由這個概念上,並因自由這概念而得到其一貫性與客觀
實在性;這就是說,它們的可能性是因「自由確實存在」這
事實而被證明,因為自由這理念是因道德法則而被顯露出來
的。

但是,自由是思辨理性底一切理念中唯一的一個我們先驗地
知其可能性的理念,(但卻不是理解它),因為它是我們所
知的道德法則之條件。〔康德於此有底註云:「自由是道德
法則底存在根據,而道德法則是自由底認識根據。」這裡所
謂「自由是道德法則之條件」,這條件意即「存在根據」
意。而當說「道德法則是我們在其下能意識到自由的條件」
(亦底註中語),這條件便是「認識根據」意。此亦即上段
末句「因為自由這理念是因道德法則而被顯露出來的」之\end{quotation}

\newpage\thispagestyle{empty}\addtocounter{page}{-1}\vspace*{-12mm}\begin{center}\noindent
\includegraphics[clip, trim=163pt 119pt 134pt 265pt, height=162mm]{ocr-input/image-0629.png}\end{center}

\newpage\markright{第一部 \quad 第三章 \quad 自律道德與道德的形上學}

\begin{quotation}\kaishu 意。這兩個「條件」意不可混。〕但是上帝和靈魂不滅這兩
個理念卻並不是道德法則之條件,但只是為這道德法則所決
定的意志之必然的對象之條件:這即是說,是我們純粹理性
底實踐使用之條件。因此,關於這兩個理念,甚至我們不能
肯定說我們知道了和理解了它們底可能性,更不要說它們底
現實性了。但是,它們是道德地決定的意志之應用於它的對
象之條件,這對象是先驗地給與意志的,此即所謂「最高福
善」是。因此,結果在這個實踐的觀點上,它們的可能性必
須被預定(或設定),雖然我們不能理論地知道它與理解
它。要想去安立這種預設,在一實踐的觀點中,只要看它們
不包含有内在的不可能(矛盾)便足夠。\end{quotation}

\noindent 這兩段話已顯示出意志自由與上帝存在及靈魂不滅兩者之不同。但這點尚不是我們這裡所要註意的,我們在這裡所要註意的是康德視它們三者俱為設準。他所以視它們為設準,是表示即使意志自由與實踐理性底必然法則之關係如此密切,也只是實踐理性上為道德法則之建立而有的一個必然的預定,而對於其本身之「必然性」則吾人仍一無所知。此即表示說:並不因預定它,便即放大了我們的知識。此即構成康德所說的「一切實踐哲學底極限」。(《道德底形上學之基本原理》第三節末之標題)。這是我們現在所要註意的問題。

\noindent 案:康德在該處所論的「實踐哲學之極限」即是對意志自由一設準而說。在感覺界中,一切現象都在因果鍊子中,是沒有自由可言的。因此,自由是屬於睿智界的一個理念—理想的概念,我們

\newpage\thispagestyle{empty}\addtocounter{page}{-1}\vspace*{-12mm}\begin{center}\noindent
\includegraphics[clip, trim=159pt 152pt 150pt 248pt, height=162mm]{ocr-input/image-0633.png}\end{center}

\newpage

\noindent 對之沒有積極的知識,因為它不能在經驗中給予,即我們對之不能有直覺或感覺,因此,它是超知識的。此其所以被劃歸為睿智界之故。它不是一個實現的知識,所以說它是一個設準。

但為什麼逼迫著要有這個設準呢?因為要建立道德法則底普遍妥當性之故。道德法則如果不能先驗地而且普遍有效地建立起來,則必無真正純正之道德行為可言,因受制於感覺對象或主觀之脾性、性好或性向便不純正故。

道德法則要先驗而普遍有效地建立起來,自必須要肯定我們的意志是自由的,即它自主自律,不受任何牽制影響,而甘願純自義上為這法則所決定,這是它接受決定的自由,同時它所甘願純自義上以受其決定的法則並不是外來的,乃即是它自己所供給的,此即是它自己立法的自由。此康德所以說「自由是道德法則底存在根據,而道德法則是自由底認識根據」之故。

但這樣理論地逼出來而為道德法則之「存在根據」的自由只是一個為成全道德法則之故而必然預定的設準。實踐哲學底追討只能至此而止,這就是康德所說「實踐哲學之極限」。

至於這必然要預定的設準本身又如何而可能,即「自由本身作為一意志之因果性如何而可能」,這不是人類理性所能解答的。這問題,康德以為同於「純粹理性如何能是實踐的」之問題,意即「單只是理性如何其自身就能是實踐的」之問題。這問題底確切意義究何在?康德說其不能被解答(不可理解)究是什麼意義?他就此「不可理解」而說的「實踐哲學之極限」究恰當否?究有意義否?這是本節所要充分說明的問題。我們試著看他如何表示這極限。他說:

\newpage\thispagestyle{empty}\addtocounter{page}{-1}\vspace*{-12mm}\begin{center}\noindent
\includegraphics[clip, trim=182pt 132pt 131pt 261pt, height=162mm]{ocr-input/image-0637.png}\end{center}

\newpage\markright{第一部 \quad 第三章 \quad 自律道德與道德的形上學}

\begin{quotation}\kaishu 甲之一

當實踐理性想它自己進入睿智界時,它並不因此就超越了它
自己所有的限制,好像因著直覺或感覺而進入那睿智界似
的。就感覺界而言,這睿智界只是一消極的思想,在決定意
志中它不能給理性以任何法則,而只有在這一點上它才是積
極的,即:這作為一消極性質的自由同時即與一積極機能,
甚至與理性底因果性,連合於一起,這積極機能,我們命名
為意志,即是說,它是「使行為底原則與合理的動機之本質
的特性,即與『格準當作一法則有普遍妥當性』這條件,契
合一致」這樣活動的一種機能。但是,如果它要想從睿智界
去假借一個意志底對象,即一動機,則它必會越過它的界
限,而把它對之一無所知的某事假充作很熟習。如是,睿智
界底概念只是一個觀點,這觀點是理性要想「認它自己為實
踐的」,所被迫必採取之以越過現象之外的,而如果感性底
影響對於人有一決定的力量,則這觀點必不是可能的,但
是,如果他不被否認他意識他自己為——睿智體,因而亦為
一理性的存在,因理性而鼓舞生力,即自由地運作著,則這
觀點也是必然的。這種思想確然包含有一個秩序與一個法則
底系統之觀念而不同於那屬於感覺界的自然之機械系統底觀
念;它使一個睿智界底概念成為必然的,(即是說,它使當
作物自身看的理性存在之全部系統成為必然的)。但是,它
絲毫不能使我們有理由除了只是關於它的形式條件而外,還
能對之多想一點什麼,即除了作為法則的格準之普遍性,如
果也就是意志底自律性,(單是這自律性才是與它的自由相\end{quotation}

\newpage\thispagestyle{empty}\addtocounter{page}{-1}\vspace*{-12mm}\begin{center}\noindent
\includegraphics[clip, trim=260pt 137pt 140pt 257pt, height=162mm]{ocr-input/image-0641.png}\end{center}

\newpage

\begin{quotation}\kaishu 一致的),而外,還能對之多想一點什麼;反之,凡涉及一
特定對象的一切法則皆是他律性的,這他律性只屬於自然之
法則,只能應用於感覺界。(【道德底形上學之基本原理)第
三節,「一切實踐哲學底極限」一項下,英譯本,頁78-79)\end{quotation}

\noindent 案:此段話的大意是如此:一、意志自由所表示的睿智界「只是一消極的思想」。所謂「消極」,是實踐理性雖可進入之,但不是「因著直覺或感覺而進入」,即不是以經驗知識底姿態而進入。因為依康德,凡知識所在之處,必有經驗直覺(即感覺)為知識之內容,即以經驗直覺而供給與料。但意志自由不是一個可以經驗地直覺的。它不能呈現於經驗直覺中而為一事象,它亦永不能對象化而為吾人所直覺。它是一個超越而純一的主體,永非吾人的經驗直覺(或感觸直覺)所能及。因此,它不是一個知識底對象。因此說這睿智界只是一消極的思想。但它亦有積極的意義,那就是「意志之自律」,在此自律上,意志自身給它自己立法,所立之法是普遍地妥當的。所謂「自由」就是與這自律性「相一致」或「連合於一起」的一個理念。二、這睿智界底思想(觀念或概念)就只是一個「作為法則的意志格準之普遍性」,意志之自律性,這就是「它的形式條件」,除此以外,再沒有別的。我們關於這睿智界所知的,(這「所知」是虛說),就只是這一點,除此以外「再不能多想一點什麼」。我們不要說對這睿智界能想多少,或知道些什麼,最好客觀地說這睿智界就只是一個意志之自律性,除此以外,再無別的。因此,名之為「睿智界」,而與「感覺界」相對比,依中國儒家的想法觀之,這不是一個好的表示法。因此感覺界有內容、有花

\newpage\thispagestyle{empty}\addtocounter{page}{-1}\vspace*{-12mm}\begin{center}\noindent
\includegraphics[clip, trim=167pt 118pt 134pt 269pt, height=162mm]{ocr-input/image-0645.png}\end{center}

\newpage\markright{第一部 \quad 第三章 \quad 自律道德與道德的形上學}

\noindent 樣,是一個有經驗內容的系統,而睿智界卻一無內容,只是一個「形式」。而名之曰「界」(世界),則容易使人想到好像這裡邊有什麼東西似的。其實是除了形式條件外,一無所有。(說它只是「形式」,這形式不是柏拉圖的「理型」,註意。)故康德說:「這種思想確然包含有一個秩序與一個法則底系統之觀念而不同於那屬於感覺界的自然之機械系統底觀念」,這種措辭底方式亦同樣容易使人有遐想。其實這睿智界中的一個秩序、一個法則底系統,是與感覺界中的完全不同類、不同性質、不同層次,而且亦不是可以相比對而言的。因為感覺界中的秩序、系統(自然之機械系統),是有內容的知識系統,而這個界中的秩序、系統卻是無內容的:它只是一個意志之自律性,展現而為道德法則之普遍性,這不是構造特殊內容的各種法則所組成之系統;它是一個同質的純一,而不是一個異質的系統;說它是個「系統」(法則底系統)可,說它不是個系統(是純一)亦可;它當然是「一個秩序」,但不是一個「異質的秩序」。(這只是順康德就這睿智界本身抽象地說,若依儒家而進到具體地說,則又有圓融之義。詳論見後。)因此,康德說:「如果它(實踐理性)要想從睿智界去假借一個意志底對象,即一動機,則它必會越過它的界限,而把它對之一無所知的某事假充作很熟習」,這表示法亦同樣令人有遐想。其實不是這一界裡邊還有什麼東西,因為我們對之無所知,故不能從裡邊「假借一個意志之對象」,如果一旦假借了,便是越出其界限。乃是這裡除了意志之自律性外,根本沒有東西可資假借。是以康德說「如果假借」云云,只是要表示意志底自律性,是由把落在經驗內的他律性投射到這裡來而成的虛說。其實這裡邊根本沒有對象,更亦無所謂

\newpage\thispagestyle{empty}\addtocounter{page}{-1}\vspace*{-12mm}\begin{center}\noindent
\includegraphics[clip, trim=159pt 134pt 145pt 260pt, height=162mm]{ocr-input/image-0649.png}\end{center}

\newpage

\noindent 知與不知,因而亦無所謂越界不越界。依儒家的說法,康德所說的睿智界只是一個「體」;說體用,不說兩界,這倒乾淨得多了,不容易有那些令人生遐想的贅辭。

以上所疏解的兩點意思明白了,再看康德繼上而來的下文:

\begin{quotation}\kaishu 甲之二

但是,理性如果要從事於去說明「純粹理性如何能是實踐
的」,(這問題完全同於去說明「自由如何是可能的」),
它必越過它的一切界限。

因為對一個東西,除我們能把它還原到法則,這法則之對象
能在某種可能經驗中被給予外,我們不能說明任何事。但是
自由卻是一個純然的理念【理想的概念】,它的客觀實在性
決不能依照自然之法則而被表示,結果隨亦不能在任何可能
經驗中被表示;因此之故,它亦決不能被領會或了解,因為
我們不能因任何種實例或類比來支持它。它只能在一個「相
信他自己有意志之意識,即相信他自己能意識到一個不同於
只是欲望的機能」的存在上,當作理性底一個必然假設而成
立,(這不同於「只是欲望」的機能就是一個「當作一睿智
體而決定它自己去行動」的機能,換言之,亦即是「以獨立
不依於自然本能的理性之法則來決定它自己」的一種機
能)。現在,「凡依自然之法則而有的決定」停止之處,亦
即是一切說明停止之處,結果,除消極防禦外,便一無所
有,即是說,除對那些自以為已深入事物之本性,很勇敢地
宣稱自由為不可能的人所作的反對,予以移除(撥去)外,\end{quotation}

\newpage\thispagestyle{empty}\addtocounter{page}{-1}\vspace*{-12mm}\begin{center}\noindent
\includegraphics[clip, trim=161pt 108pt 123pt 269pt, height=162mm]{ocr-input/image-0653.png}\end{center}

\newpage\markright{第一部 \quad 第三章 \quad 自律道德與道德的形上學}

\begin{quotation}\kaishu 便不能再作什麼。〔……】(同上,英譯本,頁79)\end{quotation}

\noindent 案:此段先指出「純粹理性如何能是實踐的」這問題完全同於「自由如何是可能的」一問題,並明其不可說明(解明),因而自由「亦不能被領會或了解」。次明這問題所以不能被說明之理由乃在自由並不在「可能經驗」中,即吾人對之並無一經驗的直覺,是則所謂不能被說明即是不能以說明知識對象底方式去說明它,因「自由」根本非一知識對象故,又因它本身即是最後的、無條件的,所以亦不能再把它還原到另一較高之法則上。故康德云:「凡依自然之法則而有的決定停止之處即是一切說明停止之處」。「不能被說明」之意如此,則所謂「自由不能被領會或了解」,即是不能以了解「知識對象」底方式去了解它。「說明」與「了解」即都限於經驗知識底意義,則超出此知識意義的東西便是不可說明、不可了解的。理性要想去說明與了解它,便是越過它的界限。說屬於睿智界的自由不是經驗知識方式所能說明與了解的,這本是可以的。這在現在已不成問題,凡肯認有超越實體者,皆能契此。但除經驗知識方式外,豈無另一種方式的說明與了解?康德把說明與了解之標準規定得太狹、太專一,這是很有妨礙的。正因這太狹太專一的標準,故不能有經驗知識意義的說明與了解,便是無說明與了解,因而自由便只是一假設。本來無經驗知識意義的說明與了解,不必就只是一假設,這在邏輯上就可以簡別出來的,故康德由無經驗知識意義的說明與了解便推至自由只是一個假設,這在邏輯上是有問題的;且不只這邏輯推理底問題,其真實問題乃在他所講的道德真理全部落了空。關於本段,先只作如此之疏解,至於「純粹理性如

\newpage\thispagestyle{empty}\addtocounter{page}{-1}\vspace*{-12mm}\begin{center}\noindent
\includegraphics[clip, trim=152pt 138pt 138pt 246pt, height=162mm]{ocr-input/image-0657.png}\end{center}

\newpage

\noindent 何能是實踐的」,「自由如何是可能的」,這兩個問題語句本身底意義,俟順通下引康德原文後再予以總明。

\begin{quotation}\kaishu 甲之三

「解明意志自由」之主觀的不可能性正等於「發見與解明人
何以能感興趣於道德法則」之不可能性。不過縱然如此,人
確實是感興趣於道德法則的,這興趣底在我們心中的基礎,
我們叫做是道德情感。這道德情感有時被誤認為道德判斷之
標準,其實它毋寧須被視為法則運用於意志上〔所產生〕的
主觀效果,而意志底客觀原則則單為理性所供給。\end{quotation}

\noindent 關於「興趣」一詞,康德有一底註,茲亦移譯於此以助了解:

\begin{quotation}\kaishu 興趣是理性由之以成為實踐者,即是說,它是決定意志的一
個原因。因此,我們說只有理性的存在才對一件事物感興
趣;非理性的存在則只覺有感官的噹欲。依是,理性只當它
的格準之遍效性單獨足以決定意志時,它才對於行為感着一
種直接的興趣,單只是這樣一種興趣才是純粹的。但是如果
只因着欲望底另一對象,或只在主體底特殊情感之暗示上,
它才能決定意志,則理性對於行為便只有一間接的興趣;而
因理性本身若無經驗便不能發見意志底對象,或發見激動意
志的一種特殊情感,所以那間接的典趣必只是經驗的,而不
是一純粹的理性興趣。又,理性之邏輯的興趣(意即去擴展
它的洞察)亦決不會是直接的,它總是預定了理性為之而被\end{quotation}

\newpage\thispagestyle{empty}\addtocounter{page}{-1}\vspace*{-12mm}\begin{center}\noindent
\includegraphics[clip, trim=177pt 122pt 124pt 263pt, height=162mm]{ocr-input/image-0661.png}\end{center}

\newpage\markright{第一部 \quad 第三章 \quad 自律道德與道德的形上學}

\begin{quotation}\kaishu 使用的各種目的。(以上俱見英譯本頁80)\end{quotation}

\noindent 以上的正文及底註都是在說明興趣(道德情感)一詞底意義。在這裡,康德指出「人何以能感興趣於道德法則」,這是不能被解明的,亦如「自由如何是可能的」之不能被解明。所謂「感興趣於道德法則」是直接地感,不是因着什麼別的東西而感。反過來,就是:單是這道德法則本身就足以使我們感興趣,不須任何經驗或感性的東西之助。這意思正好同於孟子所說的「理義之悅我心,猶芻豢之悅我口」。「芻豢悅口」是經驗的、感性的;但「理義悅心」卻是理性的、純粹的。但依康德,理義何以能悅我心,我心何以能直接悅理義,這卻是不可能被說明的。這問題底確切意義究竟是什麼呢?又,何以即同於「自由如何是可能的」一問題之不可能被說明?依康德,

1.純粹理性如何能是實踐的?

2.自由如何是可能的?

3.人何以能直接感興趣於道德法則?

\noindent 這三問題完全相同。我們再試看下文康德如何說明這第三問題之不可說明:

\begin{quotation}\kaishu 甲之四

實在說來,要想一個理性的存在,(他亦通過感官而被影
響),一定意欲那「單獨是理性指導著那『他們應當去意
欲』的存在」,〔案此即譬如說:指導著「人們或人類應當
去意欲」〕,那無疑地亦需要理性必有一種力量去把一種快\end{quotation}

\newpage\thispagestyle{empty}\addtocounter{page}{-1}\vspace*{-12mm}\begin{center}\noindent
\includegraphics[clip, trim=155pt 147pt 137pt 240pt, height=162mm]{ocr-input/image-0665.png}\end{center}

\newpage

\begin{quotation}\kaishu 樂或滿足底情感註入義務之充盡中,即是說,它必有一種因
果性,因著這種因果性它依照它自己的原則去決定感性。但
是,一純然的思想,其本身並不包含有任何感覺的東西,其
自身如何就能產生一種苦或樂底感覺,這是完全不可能去辨
識的,即是說,不可能使這成為先驗地可理解的;因為這是
一種特別的因果性,我們對它一如對每一其他因果性一樣,
不能先驗地決定出任何什麼東西;要決定出什麼,我們必須
只有商之於經驗。但是,因為這商之於經驗,除在兩個經驗
對象間的因與果之關係外,不能供給我們以任何因與果之關
係,而同時在這種(特種因果性)之情形中,雖然這被產生
出的結果確實存於經驗範圍內,可是那原因則是被設想為純
粹理性通過那些無對象提供給經驗的純然理念而活動,所以
要想去說明「作為一法則的格準之普遍性,即要想去說明道
德性,如何並為何能使我們感有興趣」,這對於我們人類
說,那是完全不可能的。只有這一點是確定的,即:那不是
因為它使我們感興趣,它才對於我們有妥實性,(因為這
樣,它必是他律的,而且必使實踐理性依於感性上,即依於
一種情感以為它的原則,在這情形,那決不能有道德法
則),乃是它所以使我們感興趣,是因為它對於作為人的我
們是妥實的,因為它在我們的作為睿智體的意志中,換言
之,在我們真正的自我中,有其根源,而凡屬於只是現象
者,則必因著理性而必然地隸屬於物自身之本性。(英譯
本,頁80-81)\end{quotation}

\newpage\thispagestyle{empty}\addtocounter{page}{-1}\vspace*{-12mm}\begin{center}\noindent
\includegraphics[clip, trim=156pt 176pt 142pt 260pt, height=162mm]{ocr-input/image-0669.png}\end{center}

\newpage\markright{第一部 \quad 第三章 \quad 自律道德典道德的形上學}

\noindent 案:這段話即在表明「人何以能直接感興趣於道德法則」,或反之,「道德法則本身何以就能使我們感興趣」,(理義何以能悅我心),一問題之不可解明。當我看到康德說:「要想去說明作為一法則的格準之普遍性【……】如何並為何能使我們感有興趣,這對於我們人類說,那是完全不可能的」,我心中實在有說不出的不適(不妥貼)之感,(不能釋然於懷)。一個純淨的道德動機,不附帶任何歧出的條件,而單為義之所在之故而甘願去行,這是隨時可有的,這即是「純粹理性其自身即能是實踐的」之意,亦即是「道德法則本身即使吾人感興趣」之意,這種道德真理之理解是要在踐履中理解並證實的,但康德卻說這是我們人類理性所完全不可說明、不可理解的。我不知康德何以用這種嚴重的措辭方式去表示這個問題,何以用這種不恰當的思考方式去表明這個問題。所謂不恰當的思考方式就是:「一純然的思想,其本身並不包含有任何感覺的東西,其自身如何就能產生一種苦或樂底感覺,這是完全不可能去辨識的,即是說,不可能使這成為先驗地可理解的」,這不恰當的思考方式就是用經驗知識底形態去思考這種「特別因果性」,而因為這「特別因果性」中的結果雖在經驗範圍內,而原因則卻是超經驗的純理念,我們對之並無經驗的直覺,因此這特別因果性逐亦完全不能被理解(被辨識)、被說明(被解明),遂有「要想去說明,〔……】這對於我們人類說,那是完全不可能的」這種嚴重的措辭方式。其實這問題很簡單,自由自主自律的意志連同它自給的普遍法則本非經驗知識所能及,亦本非一經驗知識之對象,即本不在經驗的事件串中,它們本非一事件,因此我們即不必以經驗知識之標準去判決它對於我們人類完全不可理解、不可說明,(這太嚴

\newpage\thispagestyle{empty}\addtocounter{page}{-1}\vspace*{-12mm}\begin{center}\noindent
\includegraphics[clip, trim=185pt 163pt 130pt 239pt, height=162mm]{ocr-input/image-0673.png}\end{center}

\newpage

\noindent 重的措辭方式,它可以使人想到道德完全被擯於人類理性能力以外),只要說明它非經驗知識所能及就夠了。這可留下一活動的餘地。而康德則以經驗知識為唯一的標準,由此而表明「實踐哲學之極限」,以為吾人雖可假設「自由」,然吾人之知識卻不因此假設而放大。經驗知識不因此而放大是可以的,然因此而說自由只是一設準,「其本身如何可能」完全不可理解、不可說明;「純粹理性如何能是實踐的」,亦完全不可理解、不可說明;「道德法則何以能使吾人感興趣」(理義何以能悅我心),亦完全不可理解、不可說明,這一切「理性要想去說明,必越過它的一切界限」,必至以其「一無所知的假充作很熟習」,這樣來划定「實踐哲學底極限」,判定「自由如何可能」等為不可說明,而置於假設、信仰之中,這是完全不恰當的思考方式。以不應如此嚴重而說為如此嚴重,這里邊必有一種不透的強探力索的工巧虛幻,看起來煞有意義,而其實多是無意義的贅辭。理性之思辨使用在經驗內有效,超出經驗,則只能提出一些空理念,以此說明思辨理性底界限,這是恰當的。而當實踐理性有權開關睿智界時,還要順經驗知識去說明它的界限,那便成為無意義。因為實踐理性在道德上所接觸到的意志自由定然命令等本已說其是超經驗的,今若再依經驗知識底標準判定其不可以經驗知識底方式去說明,那豈不是贅辭(套套邏輯)?如依這方式,我們豈不可以表明成功經驗知識的先驗條件如時空及範疇之類,亦同樣不可理解、不可說明?又,如仍是順經驗知識之標準去判定實踐理性之界限,則實踐理性之提供「自由」一理念又焉見更妥實於思辨理性之提供?自由雖對道德法則之建立為更切,然依康德判定「實踐哲學之極限」之思路,其為落空同於思

\newpage\thispagestyle{empty}\addtocounter{page}{-1}\vspace*{-12mm}\begin{center}\noindent
\includegraphics[clip, trim=149pt 143pt 143pt 237pt, height=162mm]{ocr-input/image-0677.png}\end{center}

\newpage\markright{第一部 \quad 第三章 \quad 自律道德與道德的形上學}

\noindent 辨理性之提供。故康德對於「純粹理性如何能是實踐的」,「自由如何是可能的」,「道德法則如何能使我們感興趣」,這些問題之不可說明、不可理解之表明,完全不恰當不相應,這也實足表示其對於道德真理、道德生命之不透,而陷於枯窘獃滯,只在外部指畫的境地之中,因而遂有此不恰當的思考方式:以經驗知識、思辨理性底界限誤移作實踐理性底極限,妨礙了對於實踐理性底領域之真實地開關,使道德全落於空懸之境地中。

茲再看康德順上文而來的綜論:

\begin{quotation}\kaishu 甲之五

依是,「一個定然命令如何是可能的」這問題,只能被解答
到這個程度,即:我們能指定出它依之以可能的那唯一假
設,即自由之理念這假設;我們亦能辨識這假設底必然性,
而這一點,就理性底實踐運用而言,即就對於這定然命令之
妥當性底信服,因而亦就是就對於道德律底信服而言,是很
足夠的;但是,這假設本身如何是可能的,則不能因任何人
類理性而被辨識。但依據「一睿智體底意志是自由的」這假
設,意志底自律性,(當作意志底決定之本質的形式條件
看),乃是一必然的結果。復次,這意志之自由不只是作為
一個假設它完全是可能的,(因它對於感覺界底現象之連結
中的物理必然性之原則並不包含有任何矛盾),如思辨哲學
之所示;並且進一步,一個理性的存在,他若意識到一種通
過理性的因果性,即是說,他意識到一種意志(不同於欲
望),則他也必須必然地使它(意志自由)實踐地,即在理\end{quotation}

\newpage\thispagestyle{empty}\addtocounter{page}{-1}\vspace*{-12mm}\begin{center}\noindent
\includegraphics[clip, trim=189pt 173pt 116pt 226pt, height=162mm]{ocr-input/image-0681.png}\end{center}

\newpage

\begin{quotation}\kaishu 念上,成為一切他的自願行為之條件。但是,要想去解明:
純粹理性,沒有任何從其他根源引發出來的行動之衝力(興
發之力)之助,如何其自身卽能是實踐的,即是說,只是
「一切它的作為法則的格準之普遍妥當性」底原則,(這原
則必確然是純粹實踐理性底形式),如何其自身即能供給一
興發之力,而無任何先於其中感興趣的意志之對象(材料)
以引發之:以及它如何產生一種叫做是純粹地道德的興趣;
或換言之,「純粹理性如何能是實踐的」——去解明這個問
題,實已超出人類理性底力量之外,一切尋求對於它的說明
所費之艱苦與辛勞皆屬白費。(英譯本,頁81-82)\end{quotation}

\noindent 案:這一段是以前各段底綜述,正式確定「自由之理念」只是一假設。其中「這假設本身如何是可能的,則不能因任何人類理性而被辨識」,以及最後「去解明這個問題實已超出人類理性底力量之外」,這些句子都是過甚其辭的表示,後面有一不恰當的思考方式,如對於上段所疏解者。康德以為只要我們能指定出這一假設,這「就理性底實踐運用而言,即是說,就對於這定然命令底妥當性之信服,因而亦就是對於道德律底信服而言,是很足夠的」,其實視自由為一假設,就定然命令、道德律(法則)、理性底實踐運用而言,並不足夠。因為道德律、定然命令不只是一個在理論上令人信服的東西,它必須在道德踐履上是一個呈現的現實;而理性底實踐運用亦不只是光理論地講出定然命令之普遍妥當性令人信服而已,它亦必須在道德踐履中是一個呈現的實踐運用。但如果自由只是一假設,不是一呈現,(因非經驗知識之所及),則道德律、定

\newpage\thispagestyle{empty}\addtocounter{page}{-1}\vspace*{-12mm}\begin{center}\noindent
\includegraphics[clip, trim=146pt 150pt 155pt 233pt, height=162mm]{ocr-input/image-0685.png}\end{center}

\newpage\markright{第一部 \quad 第三章 \quad 自律道德與道德的形上學}

\noindent 然命令等必全部落了空,而吾人亦不知其何以會是一呈現,這點正是康德所未能參透的。又他說「我們亦能辨識這假設底必然性」,此語表面上似與「但是,這假設本身如何是可能的,則不能因任何人類理性而被辨識」之語相衝突,但我們細看,則不是如此。前語中的「辨識」只是理論的推證,為建立道德法則而必然要推至,此即康德所說「道德法則是自由底認識根據」之意,(見本節開頭引文中);而「辨識這假設底必然性」,此所謂「必然性」亦只是這假設之理論地推證上即預定上之必然性,這是主觀地說。至於「這假設本身如何可能」,則是就「自由」本身客觀地說,或存有地說;此問題「不能因任何人類理性而被辨識」,是說其非經驗知識所能及,非經驗知識意義的說明所能解明,而康德則誇大地說為「不能因任何人類理性而被辨識」,「實已超出人類理性底力量之外」。這是兩層意思,故不衝突。這點俟下面再論。

康德上文之綜述表明「純粹理性如何能是實踐的」為不可解明後,即綜結之曰:

\begin{quotation}\kaishu 甲之六

這個問題正恰似想去發見出「自由本身作為一意志之因果性
如何是可能的」。因為要想那樣去發見,我即離開了哲學說
明之根據,我更無其他可依以前進之根據。我誠然可以自耽
於仍留存給我的那睿智界中,但是雖然我對這睿智界有一佳
構的理念,卻對之無一些知識,即以我的理性之自然機能之
一切努力,我也永不能獲得這樣的知識。這睿智界只指示一
某種仍然留存下來的東西,即,當我從我的意志之實施原則\end{quotation}

\newpage\thispagestyle{empty}\addtocounter{page}{-1}\vspace*{-12mm}\begin{center}\noindent
\includegraphics[clip, trim=179pt 164pt 116pt 226pt, height=162mm]{ocr-input/image-0689.png}\end{center}

\newpage

\begin{quotation}\kaishu 中把屬於感覺界的每一東西全數銷除以後,而仍然留存下來
的東西,其任務只是把那取自感性園地中的動機之原則保存
于其界限內不使之氾滥;固定它的限制(範圍),並表明它
在其自身範圍內並不包含著一切的一切,乃是總有越乎其外
者;但是對這越乎其外的某種東西,我卻並無再進一步的知
識·關於構成這個理想的純粹理性,在抽除一切材質,即一
切對象之知識後,所存留下來的不過是形式,即不過是格準
底普遍性之實踐法則,以及在與這法則相一致中理性底概念
在涉及一純粹的睿智界中之作為一可能的有效原因,即作為
決定意志的一個原因。在這里,一切外在的衝力必須全部不
存在;除非只這睿智界底理念本身是衝力(興發之力),或
只理性所首先對之感有興趣者;但是,要使這成為可理解,
則正是我們所不能解答的問題。(英譯本,頁82-83)\end{quotation}

\noindent 案:此段復歸於上開始所引甲之一一段之意,歸於表明睿智界本身之意義,表明它「不過是形式,不過是格準底普遍性之實踐法則」,並表明單此法則本身是衝力(興發之力),足以「悅我心」,使我們感興趣,而引發吾人之行為。但最後還是說:「要使這成為可理解,則正是我們所不能解答的問題」。(此種嚴重的措辭方式,若不知其來歷、確意,單看此兩句本身,簡直令人不愉快。)

以上是康德《道德底形上學之基本原理》第三節中「一切實踐哲學之極限」項下之主文,我把它全譯在這里,其餘前後尚有不甚相干的若干段,則略而未譯。康德在此第三節中復有一最後之〈結

\newpage\thispagestyle{empty}\addtocounter{page}{-1}\vspace*{-12mm}\begin{center}\noindent
\includegraphics[clip, trim=150pt 143pt 156pt 245pt, height=162mm]{ocr-input/image-0693.png}\end{center}

\newpage\markright{第一部 \quad 第三章 \quad 自律道德與道德的形上學}

\noindent 識〉,(亦即此全書之末尾),其文如下:

\begin{quotation}\kaishu 甲之七

理性在關於自然方面之思辨的使用,引至世界底某種最高原
因之絕對的必然性:理性之在自由方面之實踐的使用,亦引
至一種絕對的必然性,但這只是一個理性存在底行為法則之
絕對的必然性。不管如何使用,把知識推到其必然性之意識
乃是理性之一本質的原理,(無此必然性,那必不能算是理
性的知識)。但是理性既不能辨識「是什麼」或「發生什
麼」之必然性,亦不能辨識「應當發生什麼」之必然性,除
非假定一個依之以「是什麼」,「發生什麼」,或「應當發
生什麼」之條件,這情形亦同樣是理性之一本質的限制。但
是,在這路數中,若一直去追究那條件,則理性底滿足必只
是步步後設、永無了期的。因此,它不停止地要去尋求那無
條件地必然的東西,並見它自己被迫著不得不去預定這無條
件地必然的東西,雖然無任何方法足以使它成為可理解的,
但只要能發見一個概念它契合於這預定,亦就很夠愉快的
了。因此,在我們的道德底最高原理之推演中,是並無什麼
錯誤的,但是如果要有反對(異議),那反對必是對人類理
性一般而發,即:人類理性不能使我們去思議一個無條件的
實踐法則(如定然命令者是)之絕對的必然性。拒絕以一條
件,即是說,以某種「預定之以為基礎」的興趣,去解明這
種必然性,這是並不足責怪的,因為這樣,那法則便不再是
一道德法則,即是說,便不再是自由之一最高原則。如是,\end{quotation}

\newpage\thispagestyle{empty}\addtocounter{page}{-1}\vspace*{-12mm}\begin{center}\noindent
\includegraphics[clip, trim=148pt 162pt 158pt 236pt, height=162mm]{ocr-input/image-0697.png}\end{center}

\newpage

\begin{quotation}\kaishu 當我們不能理解道德命令之實踐的、無條件的必然性時,我
們猶可理解其「不可理解性」,而這亦就是我們所能恰當地
要求於一個「竭力將其原則帶至人類理性底極限」的哲學之
一切了。(英譯本,頁83-84)\end{quotation}

\noindent 案:此最後的〈結識〉表明理性的知識無論在自然方面,或是在自由方面,總以達到「絕對的必然性」為終極。此所謂「絕對的必然性」意指客觀地或存有論地(或體性學地)說的那「無條件地必然的東西」而言,順康德的詞語說,即指屬於睿智界的那「物自身」式的「最後的真實」而言。此最後的真實是絕對的,此言其是無條件的、是必然的,此言其是定然如此而不可移的。如依中國傳統的思路說,這就是「體」這個字所表示的;如依康德的道德哲學所展示,這就是自由自主自律的意志(不同於感覺界底欲望)連同著它所自給的具有普遍妥當性的道德法則。從法則方面說,它只是個「形式」,只是個「理」;從意志方面說,它是個實體,它是最後的真實,物自身式的最後真實。依是,從意志(實體)方面說,它永不能是忽起忽滅的意象或事件而可以用自然因果律去貫穿的,因而它亦永不能是經驗知識對象。它是最後的、絕對的主體,是純一、是形式,而無感性的異質以間雜之者。它不是事件、意象、經驗知識底對象,我們即不能以經驗知識意義的標準去說它「不可理解」,「不是任何人類理性所能辨識的」,因而逐謂它只是一假設。既認它是最後的真實,是自主自律自由的實體,它自然是無條件的、絕對的、必然的。它是無條件的,自然再不能通過一個條件去理解它、辨識它,把它再還原到更高一層的法則或條件上去。

\newpage\thispagestyle{empty}\addtocounter{page}{-1}\vspace*{-12mm}\begin{center}\noindent
\includegraphics[clip, trim=184pt 139pt 128pt 252pt, height=162mm]{ocr-input/image-0701.png}\end{center}

\newpage\markright{第一部 \quad 第三章 \quad 自律道德與道德的形上學}

\noindent 然如此,自亦不能因它「再不能通過一個條件而被理解被辨識」,便謂它「不可理解」,「超出人類理性底力量之外」,「非任何人類理性所能辨識」。這只能說它不是可以用「通過一條件」底方式去理解去辨識,不是可以用「概念思考」底理性去理解去辨識,但不能說不能用任何方式去理解,不能用任何理性去辨識。康德只限理解於「通過條件」底方式,於「概念思考」底理性,以之來證成「自由本身之絕對必然性」為不可理解,因而遂說自由只是一假說,這是很不恰當(不相應)的思考方式,很不相干的無意義的贅辭。從意志(實體)方面說是如此,從無條件的實踐法則、定然命令(理)方面說,亦是如此。它是無條件的,當然再不能通過一條件去理解或辨識它的「絕對必然性」。然因此便能說:「人類理性不能使我們去思議一個無條件的實踐法則(如定然命令者是)之絕對必然性」嗎?不能用條件底方式去作概念的思議,這自然是對的,但不能說不能用任何方式去思議。康德說它的「絕對必然性」(不管是自由意志的,或是定然命令的),不可理解、不可辨識,並不是老子所說的「道可道非常道」之「道」之不可說義,亦不是如佛家所說的「言語道斷、心行路絕」之「真如佛性」之不可思議義。因為「道」雖不可說,即不能用一定的概念去思考,然而它的真實性(絕對必然性)也還是呈現於我們的「虛壹而靜」的道心之前的,決不能說它超出人類理性底力量之外,非任何人類理性所能辨識。佛家的真如佛性雖是「言語道斷、心行路絕」,不是條件方式所能把握,不是概念思考所能契悟,然而它的真實性、絕對必然性,也還是真實地呈現於我們的般若智中、菩提心中,決不能說它非任何人類理性所能辨識。依此,自主自律自由的意志這道

\newpage\thispagestyle{empty}\addtocounter{page}{-1}\vspace*{-12mm}\begin{center}\noindent
\includegraphics[clip, trim=143pt 157pt 149pt 230pt, height=162mm]{ocr-input/image-0705.png}\end{center}

\newpage

\noindent 德性的最後真實以及它所自立的無條件的實踐法則、定然命令,其「絕對必然性」為什麼不可以亦依這方式在道德的踐履中去理解(證悟)去辨識(默識),因而使它真實地呈現於吾人之道德心靈之前呢?為什麼必依條件底方式,概念思考底理性,而把它擯除於人類理性底力量之外,而視之為假設呢?如果道德尚不是人類理性底力量所能及,尚不能使之成為真實的呈現,則試想我們這個人類是個什麼存在呢?他還能作什麼呢?是故康德這不恰當不相應的思考方式,不相干的無意義的贅辭,實只表示其對於道德生命、道德真理之未能透澈,未能正視道德真理與道德主體之實踐地真實地呈現之義。把一個道德實踐上的真實問題弄成一個無意義的問題,一個「只是經驗知識所不能及、條件方式概念思考所不能解,因而便謂其不可理解、只是一假設」的問題。這其實不是實踐哲學實踐理性底極限,乃只是經驗知識思辨理性底極限,而因以知識為貫通一切底標準,又因不能正視道德真理(法則)與道德主體(意志)之實踐地呈現,逐錯覺地誤移為實踐哲學之極限。實則實踐哲學、實踐理性可衝破此界限。惟衝破此界限,道德始能落實,「道德的形式上學」始能出現,而人始可真為一「道德的存在」,其最高目標是成聖。到這裡,始真可以看出康德的道德哲學之限度(即他造詣到什麼境界),亦可以看出儒家經過宋、明儒底發展與弘揚,其造詣與境界何以早超過了康德。康德是西方哲學家中正式開始認識道德真理之本性的人,然而亦只是初步。依儒者觀之,此後煞有事作,煞有奧理可說。譬如衝破康德所立的界限後,實踐哲學實踐理性還有其極限否?這將是宋明儒學中一個十分深奧的問題。此即是盡性中立命的問題。在此說極限方是恰當的。

\newpage\thispagestyle{empty}\addtocounter{page}{-1}\vspace*{-12mm}\begin{center}\noindent
\includegraphics[clip, trim=179pt 145pt 124pt 245pt, height=162mm]{ocr-input/image-0709.png}\end{center}

\newpage\markright{第一部 \quad 第三章 \quad 自律道德與道德的形上學}

於是,我們仍須回來仔細考察「純粹理性如何能是實踐的」,「自由本身如何是可能的」,「人何以能直接感興趣於道德法則」(理義何以能悅我心),這些問題底確切意義究是什麼?

\noindent 乙、意志自由如何能真實地呈現?

我們可先考察「純粹理性如何能是實踐的」一問題。此中所謂「純粹理性」不是《純粹理性批判》中所謂純粹理性,因為那是指純粹的思辨理性說,而這裡卻是指純粹的實踐理性說,其「實指」即是自主自律的意志所自給的具有普遍妥當性的道德法則、定然命令,這是沒有任何感性的成分在內的,所以是純粹理性的。這種屬於道德的純粹理性如何其自身就能是實踐的?此所謂「實踐」就是說能起用而有實效,能指導著我們人而我們人亦能承受之遵順之去行動而造成或表現出一種道德的結果。它如何其自身就能這樣生效?所謂「其自身」就是說單是它自己而不需有任何屬於感性的成分之幫助,亦不需有任何先對之感興趣的對象之引發,就能生效起作用。這種單是它自身就能生效起作用,就叫做是「透過理性的因果性」,亦曰「意志底因果性」,這是康德所說的「特種因果性」,而與貫穿或連結事件的自然因果性不同。因為雖然在行為上(言行上)所產生的道德結果是落在經驗範圍內,可以名之日事件,(所謂見諸行事,雖是行事,亦可說事件),但那由意志自律而給的道德法則、定然命令,卻不是事件,亦不在經驗內,故曰「特種因果性」。若依儒家的說法,這特種因果性就是體用底關係」。「純粹理性如何其自身就能是實踐的」,其確切的意義當該就是「這特種因果性如何能真實地呈現」。這問題完全同於「人何

\newpage\thispagestyle{empty}\addtocounter{page}{-1}\vspace*{-12mm}\begin{center}\noindent
\includegraphics[clip, trim=152pt 154pt 140pt 233pt, height=162mm]{ocr-input/image-0713.png}\end{center}

\newpage

\noindent 以能直接感興趣於道德法則」,「道德法則何以能使吾人有興趣」。這些句子都是同意語,說的是一個意思。這問題正是道德實踐底要害處,故這問題本身不是無意義的。但康德卻把這問題轉而為以經驗知識意義的標準去衡量,王顧左右而言他,說這問題不可解明、不可理解,是「超出人類理性底力量之外的,一切尋求對於它的說明所費之艱苦與辛勞皆屬白費。」這就使這問題成為無意義。因這問題本不屬於經驗知識問題,意志因果性中的那「原因」、純粹理性、儒者所謂「體」,本不是一個經驗對象,本不是一意象或事件,那麼你以經驗知識意義的標準去裁定它不可說明、不可理解,這豈不是「王顧左右而言他」,成為無意義的贅辭?這是把問題岔出去了。他把其確切意義實為「如何能真實地呈現」之問題轉而為經驗知識所不能及的問題,因而謂其不能解答、不可說明、不可理解,這正是捨要害而說那不相干的事。這尚不要緊,正因這一岔出去,遂使「單是理性命令著我們」這一十分中肯的道德真理成一不能落實的空理論,成為一無法正視的糊塗,(因不可理解,超出人類理性底力量之外故),只是理上想當然耳,而不知其何以會如此,這好像他的生命全投註在思辦的機括中而沒有真正過道德生活似的。儒者所謂「覿面相當」或「靚體承當」正是真正過道德生活而正視這道德真理的。這問題底確切意義以及其可理解、可說明,(但非經驗知識意義的),正是要在這「觀面相當」中來把握,這樣才能使那問題恢復其為具有意義的。

「純粹理性如何其自身就能是實踐的」,這問題底關鍵正在道德法則何以能使吾人感興趣,依孟子語而說,則是「理義何以能悅我心。」孟子已斷然肯定說:「理義之悅我心,猶芻豢之悅我

\newpage\thispagestyle{empty}\addtocounter{page}{-1}\vspace*{-12mm}\begin{center}\noindent
\includegraphics[clip, trim=176pt 135pt 131pt 254pt, height=162mm]{ocr-input/image-0717.png}\end{center}

\newpage\markright{第一部 \quad 第三章 \quad 自律道德與道德的形上學}

\noindent 口。」理義悅心,是定然的,本不須問如何可能。但問題是在「心」可以上下其講。上提而為超越的本心,則是斷然「理義悅心,心亦悅理義」。但是下落而為私欲之心、私欲之情,則理義不必悅心,而心亦不必悅理義,不但不悅,而且十分討厭它,如是心與理義成了兩隔,這時是可以問這問題的。因為理義悅心或心悅理義,就此語不加限制觀之,並不是分析命題,乃是一個綜和命題。故問這問題是有意義的。如是這問題底最後關鍵,是在「心」字,即康德所謂「道德感」「道德情感」,而所謂「感興趣」正是直接指這「道德情感」,最終是指這「心」字說,所以最後是「心」底問題。而這正是康德所未註意的。

關於道德感、道德情感,我在前第一節中已表明:康德是著眼於其實然的層面,其底子是發自「人性底特殊構造」,屬於才性氣性的,因而他把它劃於私人幸福原則下,而視之為經驗的、後天的,而且亦無定準。這樣的道德情感當然既不能由之建立道德法則,而它亦不必即能感興趣於道德法則,即或感之,亦不是直接的、純粹的道德興趣。本節前引康德原文關於道德興趣一段(甲之三),康德亦知人確實是感興趣於道德法則的,但「人何以能感」,則彼以為不可解明。此即示康德對於「感興趣」所直指的「道德之心」與「道德之情」不能正視,因而遂使這「感興趣」之感成為偶然的現象,並不能使之挺立起而有心體上之必然性。康德說:「這道德情感有時被誤認為道德判斷之標準,其實它毋寧須被視為法則運用在意志上的主觀效果,而意志底客觀原則,則單為理性所供給。」但是因為「法則何以能運用在意志上而產生這樣的效果」既不可解明,則道德情感之地位即不穩定,無心體上之必然

\newpage\thispagestyle{empty}\addtocounter{page}{-1}\vspace*{-12mm}\begin{center}\noindent
\includegraphics[clip, trim=167pt 152pt 140pt 245pt, height=162mm]{ocr-input/image-0721.png}\end{center}

\newpage

\noindent 性,亦可以產生,亦可以不產生,如是,則意志之因果性或透過理性的因果性即不能真實地,必然地呈現。〔人何以能直接感興趣於道德法則,道德法則何以能使吾人感興趣(悅我心),這問題底正當而確切的意義是道德法則、定然命令、意志之因果性等如何能在踐履中真實地、必然地呈現之問題,不是康德岔出去而視為不可解明的知識問題。】但是康德在那段文之底註中亦說:「興趣是理性由之以成為實踐者,即是說,它是決定意志的一個原因」。這話很有意義。由前一句,如果我們能把道德情感(興趣)上提而講出其心體上的必然性,則「理性如何能是實踐的」一問題即算得到其解答。由後一句,道德情感是「決定意志」的一個原因,這「決定」當然是從「心」說的主觀實現的決定。法則決定意志,這決定是從「理」上說的客觀的決定,這只是當然,不必能使之成為呈現的實然。要成為呈現的實然,必須註意心——道德興趣、道德情感。心(興趣情感)是主觀性原則、實現原則;法則是客觀性原則、自性原則。關於這主觀性原則(實現原則,即真實化、具體化底原則),康德並未能正視而使之挺立起,到黑格爾才正式予以正視而使之挺立起。(因黑格爾正重視實現故)。康德只著力於客觀性原則之分解地建立,未進到重視實現問題,故彼雖提出之而實並未能知「純粹理性如何其自身即能是實踐的」一問題之正當而確切的意義。故彼視其為不可理解,不可說明也。張橫渠云:「心能盡性,人能弘道;性不知檢其心,非道弘人也。」前句正是主觀性原則(實現原則),故重心;後一句正是客觀性原則,但不必能呈現,故「不知檢其心」,亦是「非道弘人」之意也,此正是康德之境界,所以視「道德法則何以能悅我心」為不可理解也。他若真能正

\newpage\thispagestyle{empty}\addtocounter{page}{-1}\vspace*{-12mm}\begin{center}\noindent
\includegraphics[clip, trim=156pt 124pt 138pt 260pt, height=162mm]{ocr-input/image-0725.png}\end{center}

\newpage\markright{第一部 \quad 第三章 \quad 自律道德與道德的形上學}

\noindent 視「興趣是決定意志的一個原因」,而進到重視主觀性原則,則「理義悅心」即得解矣,「理性如何能實踐」亦得解矣。此正是孟子、象山、陽明之所著力者。將心(興趣、情感)上提而為超越的本心,不是其實然層面才性氣性中之心,攝理歸心,心即是理;如是,心亦即是「道德判斷之標準」:同時是標準,同時是呈現,此為主客觀性之統一;如是,理義必悅我心,我心必悅理義,理定常、心亦定常、情亦定常,此即是「純粹理性如何其自身即能是實踐的」一問題之真實的解答。此非康德所能至。康德對於道德法則、定然命令之了悟尚只停在抽象的階段中,不知其如何能具體實現也。實則法則決定意志這客觀的決定與興趣決定意志這主觀實現的決定只是由對於意志平看,即不加任何規定而只作實然的意志看(will as such),而來的從外面(主客觀)分別說。及至說意志是自主自律自給法則的意志,法則不從外來,則法則決定意志即是意志自己決定自己,不是由外來的法則決定其自己。而自給法則、自己決定自己的意志即是超越的本心之自律活動。此意志就是本心。它自給法則就是它悅這法則,它自己決定自己就是它甘願這樣決定。它願它悅,它自身就是興趣,就是興發的力量,就能生效起作用,並不須要外來的興趣來激發它。如果還須要外來的低層的興趣來激發它,則它就不是本心,不是真能自律自給法則的意志。康德言意志自律本已函著這個意思。只是他不反身正視這自律的意志就是心,就是興趣、興發力,遂把意志弄虛脫了,而只著實於客觀的法則與低層的主觀的興趣。

次再考察「自由本身作為一意志之因果性如何是可能的」一問題之確切意義。康德以為這問題完全同於前一問題,此不錯。「純

\newpage\thispagestyle{empty}\addtocounter{page}{-1}\vspace*{-12mm}\begin{center}\noindent
\includegraphics[clip, trim=177pt 152pt 128pt 245pt, height=162mm]{ocr-input/image-0729.png}\end{center}

\newpage

\noindent 粹理性如何能是實踐的」以及「道德法則何以能使吾人感興趣」這兩者完全是同意語,其關鍵在心(興趣、情感)之上提而為超越的本心。現在「自由本身作為一意志之因果性如何可能」,亦復如此。「自由」是論謂「意志」的一個屬性,與自主、自律為同意詞,而「意志」則是實體字,它是心體底一個本質的作用,即定方向的作用,劉蕺山所謂「心之所存,淵然有定向」者是。亦可以說它就是本心。理性、法則、定然命令等即由這心之自主、自律、有定向而表示,這就是所謂理或理義。心即是理。理義悅心,心悅理義(心之所同「然」,「然」是動字),純粹理性就能是實踐的,而悅理義之心與情必須是超越的本心本情,如是它自然非悅不可,即這「悅」是一種必然的呈現。它自給法則就是悅,就是興發力。心與理義不單是外在的悅底關係,而且即在悅中表現理義、創發理義。理義底「悅」與理義底「有」是同一的。悅是活動,有是存在,即實理底存在(存有或實有)。如是,「這自主自律的意志,即自由的意志,如何是可能的」,其意即是「如何能是呈現的」。前一問題是「道德法則如何能是實踐的」,而這一問題則是「自由的意志如何能呈現」,其義一也。道德法則即由自主自律的意志所供給,因此才是純粹而普遍的,故「道德法則如何能是實踐的」一問題即函「自由的意志如何是可能的」一問題。如果屬於自律的道德法則能呈現,則自由意志自必亦呈現。如果自由的意志不能呈現,則其所自律的道德法則自亦空懸不能呈現。故此兩者實是同一問題。康德何不反身正視這自由自律的意志即是本心,即是興趣之源,其自身就能實踐,生效起作用,所謂沛然莫之能禦?尚待何處去找興發力呢?

\newpage\thispagestyle{empty}\addtocounter{page}{-1}\vspace*{-12mm}\begin{center}\noindent
\includegraphics[clip, trim=156pt 130pt 140pt 248pt, height=162mm]{ocr-input/image-0733.png}\end{center}

\newpage\markright{第一部 \quad 第三章 \quad 自律道德與道德的形上學}

意志自由本身如何而可能,依《道德底形上學之基本原理》最後的〈結識〉觀之,即是它本身底「絕對必然性」如何而可能。這「必然性」不是我們為成全道德法則之故而去預定它這主觀的或預定上的必然性,乃是它自身客觀的存在上的必然性。康德以為這意義的「必然性」不是我們人類理性所能辨識、思議或理解的。這就是哲學思考底界限。前說「道德法則是自由底認識根據」,這認識其實是虛說,意即:「道德法則是我們在其下能意識到自由的條件」,即至多根據道德法則我們能意識到自由這個概念,能意識到必有自由之預定。所謂「認識」只是此意,並非說我們對於自由本身之客觀而存在上的必然性真有積極的知識或理解。

依此,在康德步步分解建構的哲學中,自由只是在抽象地被預定中,因而亦只是在抽象的懸空中,只是一個理念,理想的概念。至於其真實的存在上的絕對必然性,則對於我們的理性完全隔絕,不可理解,這是屬於康德所說的「物自身」式的睿智界,我們也可以說,對人類理性言,這是屬於「存有底神秘」(mystery ofbeing)的。說這是哲學底界限,本也是可以的。但若把自由完全歸諸信仰,視作被預定的理念,不能落實,不能真實呈現,這等於說道德不能落實,不能真實呈現。如是,康德所佳構的道德真理完全是一套空理論。這似乎非理性之所能安,不,簡直是悖理!

他之所以至此,是因他把這問題滑轉而為經驗知識所不能及,因而謂其不可解明、不可理解。他不知這問題本是「自由本身之絕對的必然性如何呈現」的問題,卻王顧左右而言他,說我們「對它無一些知識」。這不是經驗知識能及不能及的問題,這是一個在道德實踐中它如何能真實地呈現的問題。因為這樣自主自律因而亦是

\newpage\thispagestyle{empty}\addtocounter{page}{-1}\vspace*{-12mm}\begin{center}\noindent
\includegraphics[clip, trim=182pt 152pt 145pt 258pt, height=162mm]{ocr-input/image-0737.png}\end{center}

\newpage

\noindent 自由的實體性的意志本是個最後的實體,它本不是經驗知識底對象,本不是可以被直覺的事件,你說它非經驗知識所能理解,這於「它本身之絕對必然性如何可能」一問題有何相干呢?如果以經驗知識為標準,則如果我們對它有知識,它即不是那「最後的實體」,不是我們所說的自由的意志。如果它要是最後的實體,它即自非這種知識所能達到,自亦非這種知識所能裁定,即不因這種知識達不到它,便謂其不可理解,不可說明。康德這一歧出,遂使這問題底確切意義完全弄糊塗了。

依是,「自由本身之客觀存在上的絕對必然性如何可能」之問題就是「它的絕對必然性如何能真實地必然地呈現」之問題,這是不可以經驗知識底尺度來衡量的,這是一個實踐問題,不是一個知識問題。因此,它的絕對必然性如能在實踐中真實地呈現,則我們的理性即能與它觀面相當而理解之。這種理解是不要通過「感性」的,因自主自律自由的意志是一實體,不是一對象一事件故。因此,這種理解只是與它「觀面相當」的親證,是實踐的親證;理解之即是證實之,即是呈現之;這不是知「特定經驗內容」的普通知識,而單是實踐地知這「實體」之知。

依宋、明儒說,知不只是「知性之知」(麗物之知、見聞之知),還有實踐的德性之知。理解不只是知識意義的理解,還有實踐意義的理解。我們不只是思辨地講理性之實踐使用,還有實踐地講理性之實踐使用。不只是外在的解悟,還有內在的證悟,乃至澈悟。知性之知展開自然界,成功知識系統,如物理學等。實踐的德性之知(證悟)展開價值界,成功德性人格的發展,最高目標是成聖。

\newpage\thispagestyle{empty}\addtocounter{page}{-1}\vspace*{-12mm}\begin{center}\noindent
\includegraphics[clip, trim=158pt 130pt 141pt 254pt, height=162mm]{ocr-input/image-0741.png}\end{center}

\newpage\markright{第一部 \quad 第三章 \quad 自律道德與道德的形上學}

在這基本觀點上轉一下,如是乃得進而解明「自由本身」之客觀的存在上的絕對必然性。此將如何而可能?

「自由本身之客觀存在上的絕對必然性如何而可能」,此中所謂「可能」當不再是康德提出問題解答問題時所擬議的「可能」之意義,即不再是依一法則或形式條件而得「可能」的那「可能」之意義。此處「可能」之解答,不再是依一條件之預定,因為「自由」本身已被預定為最後的、無條件的。因此,它的客觀而存在上的絕對必然性之如何可能的問題就是「它的真實性如何呈現」的問題。此處「可能」等於其真實性之「呈現」。

宋、明儒所講的性體心體,乃至康德所講的自由自律的意志,依宋、明儒看來,其真實性(不只是一個理念)自始就是要在踐仁盡性的真實實踐的工夫中步步呈現的:步步呈現其真實性,即是步步呈現其絕對的必然性;而步步呈現其絕對的必然性,亦就是步步與之靚面相當而澈盡其內蘊,此就是實踐意義的理解,因而亦就是實踐的德性之知,此當是宋、明儒所說的證悟、澈悟,乃至所謂體會、體認這較一般的詞語之確定的意義。這自然不是普通意義的知識,不是宋、明儒所謂「見聞之知」「麗物之知」,因為它不是感觸經驗的,它無一特定的經驗對象為其內容,因為性體心體不是一個可以感覺去接觸的特定對象。從知識方面說,這知是實踐意義的體證;從性體心體本身方面說,這種體證亦就是它的真實性之實踐的呈現。步步體證就是步步呈現。但說步步,則這體證只是分證(部分的滲透悟入),而其真實性之呈現亦只是部分的呈現。但這無礙於它的真實性即絕對的必然性之呈露。如果一旦得到滿證,則它的真實性(絕對必然性)即全體朗現,此就實踐的成就說,這

\newpage\thispagestyle{empty}\addtocounter{page}{-1}\vspace*{-12mm}\begin{center}\noindent
\includegraphics[clip, trim=164pt 157pt 149pt 244pt, height=162mm]{ocr-input/image-0745.png}\end{center}

\newpage

\noindent 就是理想人格的聖人了。例如就王陽明所講的良知說,洒掃童子的良知與聖人的良知,雖在體證上有分全的不同,良知萌芽與良知本體雖亦有體證上的分全之不同,然既同為良知,則即是同具真實性與絕對必然性。此即是以前所說「無論一錢金子或一兩金子畢竟同屬金子」一喻之意。當然我們也可以反過來說,雖同屬金子,然畢竟有分量的不同,此就是聖人與普通人之不同,乃至聖人中堯舜與孔子之不同。他們當然也知道這性體心體是無邊的大海,雖說步步體證,乃至全體朗現,但亦無礙於其存有論上的奧密或超越的奧密。全體朗現了,則便轉而為「聖格全體是奧密」,此就是孟子所說的「大而化之之謂聖,聖而不可知之謂神」,亦就是羅近溪所謂「抬頭舉目,渾全只是知體著見;啟口容聲,纖悉盡是知體發揮。」

對於性體心體之體證,或性體心體本身之呈現,不只是隔絕一切經驗而徒為抽象的光板的體證與呈現,而且還需要即在經驗中而為具體的有內容的體證與呈現。「具體的」即是真實的,它不只是一抽象的光板、純普遍性,而且是有內容充實於其中而為具體的普遍。普遍性不因有內容而喪失,故雖是有內容,而卻「渾是知體著見」。這樣,倒因有內容而為具體而真實的普遍、落實平平的普遍,不是凸起抽離的光板所謂「光景」的普遍。「有內容」,這內容固是因與經驗接觸而供給,但由經驗供給而轉成性體之內容,則此內容即不是經驗與料本身而待吾人去客觀地了解它以成為「知性之知」的內容,而卻只是在這種知中、行中,乃至一切現實生活中,使性體心體之著見更為具體而真實,因而轉成「德性之知」之內容,亦即是性體心體本身之真實化的內容,此即喪失了其為「麗

\newpage\thispagestyle{empty}\addtocounter{page}{-1}\vspace*{-12mm}\begin{center}\noindent
\includegraphics[clip, trim=167pt 124pt 131pt 257pt, height=162mm]{ocr-input/image-0749.png}\end{center}

\newpage\markright{第一部 \quad 第三章 \quad 自律道德與道德的形上學}

\noindent 物之知」的內容之意義,而轉為性體心體具體化真實化之具體而真實的脈絡。(故在此種體證與呈現中,所成的不是知識系統,而是德性人格底真實生命之系統。)就性體說,固已因有內容而具體化了,但就內容說,這內容已不是「麗物之知」中那只是特殊意義的內容,而是為性體心體之普遍性所通澈潤澤了的特殊,因而亦具有普遍的意義、永恆的意義,此亦可說是普遍的特殊。因而亦即是具體而真實的特殊,不是「麗物之知」中那純然的、抽象的特殊。此即是特殊不作特殊觀,「渾是知體著見」,雖特殊而亦普遍,雖至變而亦永恆。此即羅近溪所謂「捧茶童子是道」也。亦程明道所謂「道亦器,器亦道。但得道在,不繫今與後、己與人。」今與後、己與人,在麗物之知中,當然是純然的特殊;但在道中、德性之知中,則雖特殊而亦普遍,雖至變而亦永恆,渾不見有今與後、己與人,而亦不離今與後、己與人,此即所謂化境:一起都是真實的,絕對地必然的。性體心體乃至意志自由就是這樣在體證中、在真實化、充實化中而成為真實生命之系統里得到其本身的絕對必然性。孔子是如此渾全地表現,孟子、象山、陽明、龍溪、近溪以及濂溪、橫渠、明道、五峰、蕺山等的分途解說亦不過是向此境界趨。(伊川、朱子是另一系統,不屬此自律道德之系統。)這不是普通所說的神秘主義,乃是實踐理性之實踐的必然。就孟子、象山、陽明說,亦不是普通所說的直覺主義,這也是實踐理性之實踐的必然。若說是直覺,則凡體證皆是直覺。但若不知其來歷、問題與境界之何所是,冒然憑空說某某是直覺主義,某某是理性主義,皆是不知甘苦的妄說。若只講成是直覺主義,乃是極大的誤解。

由以上即可看出宋、明儒者實早已超過了康德。若謂康德講的

\newpage\thispagestyle{empty}\addtocounter{page}{-1}\vspace*{-12mm}\begin{center}\noindent
\includegraphics[clip, trim=170pt 163pt 150pt 242pt, height=162mm]{ocr-input/image-0753.png}\end{center}

\newpage

\noindent 是哲學,那麼,這也是儒者成德之教之超過哲學處。若謂康德倒顯得謙遜,你所說的未免太樂觀、太狂大了。其實這不相干。這不是謙遜與否的問題,乃是對於實踐理性是思辨地講,抑還是實踐地講之問題,是「實踐理性如何能真實呈現」的問題,這是實踐理性之實踐地必然的。茫然不知其來歷,據淺陋為平實,視歧出者為謙遜,指其所不知者為狂大,此乃正是狂妄之言。大音不入于俚耳,視大音為狂妄,此乃真狂妄也。是以非過來人,未可輕議。

\section{「道德的形上學」之完成}

自由本身之絕對必然性可實踐地體證之,則在自由處所表示的「意志之因果性」(透過理性而表現的因果性)亦自因這體證而呈現,此即接觸到了康德所說的「是什麼」,「發生什麼」,或「應當發生什麼」之必然性底問題,這也就是我們原初之問題,即:實現之理與自然、實然者之契合問題。(參看本章第二節甲之七)。

意志之因果性,康德亦說它是一種特種因果性。我們已指出,依儒者觀之,這「特種因果性」就是「承體起用」的一種因果性。自由、自主、自律的意志是體,由它直接所指導,不參雜以任何感性的成分,而生的行為、德業或事業便是用。「應當發生什麼」是自由意志所直接決定的。意志所直接決定的「應當」,因心、情感、興趣,即因心之悅理義發理義,而成為「實然」,此即是「是什麼」或「發生什麼」之必然性。由應當之「當然」而至現實之「實然」,這本是直貫的。這種體用因果之直貫是在道德踐履中必

\newpage\thispagestyle{empty}\addtocounter{page}{-1}\vspace*{-12mm}\begin{center}\noindent
\includegraphics[clip, trim=171pt 131pt 127pt 254pt, height=162mm]{ocr-input/image-0757.png}\end{center}

\newpage\markright{第一部 \quad 第三章 \quad 自律道德與道德的形上學}

\noindent 然地呈現的。其初,這本是直接地只就道德行為講:體是道德實踐上的體,用是道德實踐上的用。但在踐仁盡性底無限擴大中,因著一種宇宙的情懷,這種體用因果也就是本體宇宙論上的體用因果,兩者並無二致。必貫至此境,「道德的形上學」(不是「道德之形上的解析」)始能出現。這種意義的形上學,本亦可原為康德思想所函蘊,但因他自由為假設,不是一呈現,又因他忘掉意志即本心,即是興發力,他遂只成了一個「道德的神學」,而並未作出這種道德意義的形上學,即由道德進路而契接的形上學。簡言之,即並未作成一個「道德的形上學」。他的屬於純睿智界的意志之因果性與屬於感覺界的自然因果性並非是直貫,乃是兩不相屬,而需有一第三者為媒介以溝通之,這是他的哲學中之一套。同時還有另一套,即在「先驗地給與意志」的最高福善這個對象之上所設定的「上帝存在」與「靈魂不滅」這兩個設準,這是屬於「道德的神學」的。這兩套不相統屬,造成許多支離。假定依儒者的襟懷,「道德的形上學」激底完成,這些支離便可全部融化。以下試道其詳。

依康德,是什麼,發生什麼,或應當發生什麼之「必然性」不能為吾人的理性所辨識。這「必然性」仍是那「是什麼」者等之客觀而存在上的那超越的必然性,即那來自睿智界而使它必然「是什麼」、必然「發生什麼」或必然「應當發生什麼」之必然性,這即相當於由「動態的實現之理」而來的「必然性」。「那無條件必然的東西」亦即相當於此動態的「實現之理」。但康德以為這只是順條件的追問而被預定的,我們的理性「無任何方法足以使它成為可理解」。即,不能理解那被預定的「無條件必然的東西」本身何以

\newpage\thispagestyle{empty}\addtocounter{page}{-1}\vspace*{-12mm}\begin{center}\noindent
\includegraphics[clip, trim=149pt 137pt 142pt 251pt, height=162mm]{ocr-input/image-0761.png}\end{center}

\newpage

\noindent 是絕對地必然的,即不能理解其本身之客觀而存在上的絕對必然性如何是可能的。這不能理解,一、因它不是一可規定的對象,可還原到一更高之法則,因為它已被預定為最後的,無條件的。二、尤其重要的,是因我們對它並沒有一直覺或感覺,即不在經驗中。是以不能理解它的「絕對必然性如何可能」就等於說它的真實性不能因感觸直覺而成為呈現的。這「無條件必然的東西」之絕對必然性,用在自由之概念上,即為自由本身之絕對的必然性,意志因果性之絕對的必然性,道德命令(定然命令)之絕對的必然性,而這些概念都只是一預定,其客觀而存在上的絕對必然性不能被理解,我們對之不能有一點知識,即其本身之真實性不能因感觸直覺而被呈現。這是屬於睿智界的。康德所謂「可能」就等於說可套在一定法則中,其所謂「其真實性如何可能」也可以說就等於「如何呈現」,但卻是可套在一定法則中的對象只因感觸直覺而為呈現,他只有這一種呈現之意義。本來對「這無條件必然的東西之絕對必然性如何可能」底問題是不能這樣思考的。這是不恰當的岔出去的不相干的思考。它的絕對必然性之可理解與不可理解,可呈現與不可呈現,不是可因有「感觸直覺」否而決定的,乃是單在是否能在踐履中與它「觀面相當」。是以除因感觸直覺而呈現外,還有一種因踐履而呈現。可理解否亦如此。奇怪的是單單這一點卻是康德所不知的。

依康德,屬於睿智界的「意志之因果性」(即意志之目的論的判斷)與自然系統之「自然因果性」,根本是兩個獨立的世界。我們對於後者有積極的知識,(知識只限於此),對於前者無積極的知識,(即不因預定它而即擴大了我們的知識,說對它亦有知

\newpage\thispagestyle{empty}\addtocounter{page}{-1}\vspace*{-12mm}\begin{center}\noindent
\includegraphics[clip, trim=170pt 121pt 122pt 259pt, height=162mm]{ocr-input/image-0765.png}\end{center}

\newpage\markright{第一部 \quad 第三章 \quad 自律道德與道德的形上學}

\noindent 識)。這兩個世界如何能接合呢?這在宋、明儒者的學問里,本不成問題。因為如果自由本身因實踐的體證而呈現,意志之因果性自亦因這體證而呈現,不只是一個隔絕的預定,如是,則意志之目的論的判斷本是可以直貫下來的。如其如此,則它自然而然地即與自然系統之自然因果性相接合,這是一個結論,不是一個問題,這就是我們前文第一節開頭時所說:問題不在直接就這兩個世界的關係本身上去奮鬥。但在康德,自由、意志之因果性,甚至道德命令,皆只是一個預定,不是一個呈現,因此,意志之目的論的判斷不能直貫下來,因而這兩個世界如何能接合逐形成了一個問題,而且是一個「其解答即寄託在直接就這兩個世界的關係本身上去構想」的問題。這就是康德第三《批判》(《判斷力批判》)之工作。

康德是以美的判斷之無所事事之欣趣所預設的一個超越原理即「目的性原理」來溝通這兩個絕然不同的世界的。這固是一個巧妙的構思,但卻是一種技巧的湊泊,不是一種實理之直貫,因而亦不必真能溝通得起來。美的判斷對於自然無所事事,它只是一種欣趣。知性判斷決定自然底質、量、關係等,道德判斷決定行為底方向,這些都是「決定性的判斷」,而唯美的欣趣無所決定,所以它是「反身性的判斷」。然而欣趣必接觸於具體,這就是說,必落在自然之實然上,此是其與自然系統相接頭處。然而自然界底複雜內容必有一種諧和的統一它始能這樣成其為欣趣。這諧和的統一,各種成分之絲絲入扣、相適應相順成所表示的一種目的性就是欣趣判斷之超越原理。但這目的性不是欣趣判斷所決定的,因它原是無所事事故。是以這目的性原理只是主觀的、形式的、反省的,順欣趣判斷之無所事事而反顯的。可是它需要這目的性原理,則此目的

\newpage\thispagestyle{empty}\addtocounter{page}{-1}\vspace*{-12mm}\begin{center}\noindent
\includegraphics[clip, trim=153pt 139pt 137pt 246pt, height=162mm]{ocr-input/image-0769.png}\end{center}

\newpage

\noindent 性原理即與意志之目的論判斷——決定方向的那目的性相接頭。此即以美的判斷為兩界溝通之媒介。由美的判斷所反顯的目的性原理,其為目的性是無向的目的性,無目的之目的,因此非決定故。但就是這「無向的目的性」即可與意志決定之「有向的目的性」相接頭。美的判斷,從其背後所經由反顯而預設的目的性原理說,它與睿智界接頭;從其落實於具體之自然上說,它與自然界接頭,所以它可以溝通兩界而使之接合。

這思路看起來誠然有巧妙之處,但亦實是一種工巧的湊泊,真能湊泊得上嗎?如果意志之因果性只是一預定之理念,不是一呈現,它下不來,只憑一外鑠的第三者去湊泊它,亦未必能接得上,它還是下不來。如果它不只是一預定之理念,而且是呈現之真實,則不需要一第三者去湊泊它,它還是下得來。它若下得來,即自然可與「自然系統」相接合。這不是一個單就絕然不同的兩界之關係上直接去搏鬥的問題,不是一個可以用工巧的構思去解決的問題。

美的欣趣誠然可以不接觸地接觸自然之具體而微妙處,(不接觸地接觸即是不著之欣賞,不關心的觀照),然而這不一定就能接上意志決定之有向的目的性。亦如中國術數之路之知幾亦可以接觸《易經》陰陽造化之妙,但不一定就能接上孔門那道德意識所貫註的窮神知化與盡性至命。孔門那道德意識所貫註的窮神知化自亦牽連著陰陽造化,但卻不是術數家眼中的陰陽造化。術數家之知幾亦可以窺測到神化,但不必是孔門義理中的神化。術數家常是自然主義與命定主義,而孔門義理則卻必須是道德的理想主義·此所以宋、明儒只講理,不講數,而邵堯夫不入宋、明儒正宗之故。術數家之知幾並不是科學判斷,也類乎一種藝術性的觀照、智的直覺。

\newpage\thispagestyle{empty}\addtocounter{page}{-1}\vspace*{-12mm}\begin{center}\noindent
\includegraphics[clip, trim=166pt 112pt 122pt 264pt, height=162mm]{ocr-input/image-0773.png}\end{center}

\newpage\markright{第一部 \quad 第三章 \quad 自律道德與道德的形上學}

\noindent 所以凡此型心態亦常含有一種洒脫的襟懷,邵堯夫以及道家俱表現這種襟懷,但亦俱缺乏那嚴整的道德意識與精誠惻怛的仁者襟懷。此所以有「易之失賊」之誡。我用中國術數家講《易經》方面的道理,即可烘托出康德以美的判斷溝通兩界之構思只是一種工巧的湊泊,並不真能湊得上。儒家的精神是孔子所說的「興於詩,立於體,成於樂。」經過了嚴整的道德意識之支柱(立於禮),最後亦是「樂」的境界,諧和藝術的境界(成於樂)。但這必須是性體、心體、自由、意志之因果性徹底呈現後所達到的純圓熟的化的境界、平平的境界,而不是以獨立的美的判斷去溝通意志因果性與自然因果性。踐仁盡性到化的境界、「成於樂」的境界,道德意志之有向的目的性之凸出便自然融化到「自然」上來而不見其「有向性」,而亦成為無向之目的,無目的之目的,而「自然」亦不復是那知識系統所展開的自然,而是全部融化於道德意義中的「自然」,為道德性體心體所通澈了的「自然」:此就是真美善之真實的合一,而美則只是由這化的境界而顯出,而不是一獨立的機能,這亦正合康德所說的「無所事事」(非決定的)。這層意思是明道以及陽明門下泰州派王艮父子及羅近溪所最喜歡談而亦達至最精微者,尤其是羅近溪為最圓融。這本亦是孔門所已含,孔聖襟懷所已至者。但這卻不是康德的構思。康德的構思只是一旁蹊曲徑,不是一康莊的大道,只有輔助指點的作用,不足以盡擔綱的說明。

我這裡只簡略地用我的較輕鬆的話,順術數家之知幾底路數以及孔子「立於禮成於樂」之義,把康德的工巧構思路數之不足表明出來,並未內在於康德之書就其詞語來評述,因為那樣將太煩重。我在《認識心之批判》最後一章中,曾有較專門之評述,但我覺得

\newpage\thispagestyle{empty}\addtocounter{page}{-1}\vspace*{-12mm}\begin{center}\noindent
\includegraphics[clip, trim=167pt 135pt 126pt 247pt, height=162mm]{ocr-input/image-0777.png}\end{center}

\newpage

\noindent 那樣評述,在表示康德思想之限度上並不比這里所說更顯豁。我這里是專以表示康德思想之限度所由成之關鍵為主題。

由以上即可充分表示出康德對於實踐理性之思想義理並未充其極。他缺乏一個「道德的形上學」,因而他只對於實踐理性之第一義能充分地展現出來,(亦只是抽象地思考的),可是對於其第二義與第三義,則因自由只是一被預定之理念,不是一呈現之故,根本不能接觸到。這樣遂使他的全部道德哲學落了空。這是他的哲學思考把他限住了,因而遂有他的「實踐哲學之極限」之想法。就是這一極限,遂使他不能有一個「道德的形上學」出現。這是要有超過哲學的儒者襟懷才能作到的。如果這「道德的形上學」亦是一實踐哲學,即亦可以哲學地講出來,則它當是相應儒家成德之教的實踐哲學,它是衝破康德所立的界限而將其所開關的實踐理性充其極的。以下試言儒者義。

依原始儒家的開發及宋、明儒者之大宗的發展,性體心體乃至康德所說的自由、意志之因果性,自始即不是對於我們為不可理解的一個隔絕的預定,乃是在實踐的體證中的一個呈現。這是自孔子起直到今日的熊先生止,凡真以生命滲透此學者所共契,並無一人能有異辭。是以三十年前,當吾在北大時,一日熊先生與馮友蘭氏談,馮氏謂王陽明所講的良知是一個假設,熊先生聽之,即大為驚訝說:「良知是呈現,你怎麼說是假設!」吾當時在旁靜聽,知馮氏之語底根據是康德。(馮氏終生不解康德,亦只是這樣學著說而已。至對於良知,則更茫然。)而聞熊先生言,則大為震動,耳目一新。吾當時雖不甚了了,然「良知是呈現」之義,則總牢記心中,從未忘也。今乃知其必然。

\newpage\thispagestyle{empty}\addtocounter{page}{-1}\vspace*{-12mm}\begin{center}\noindent
\includegraphics[clip, trim=162pt 122pt 143pt 263pt, height=162mm]{ocr-input/image-0781.png}\end{center}

\newpage\markright{第一部 \quad 第三章 \quad 自律道德與道德的形上學}

陽明的良知、後來劉蕺山的意,乃至康德的自由、意志之因果性,都是這性體心體之異名,各從一面說而已。性體心體不只是在實踐的體證中呈現,亦不只是在此體證中而可被理解,而且其本身即在此體證的呈現與被理解中起作用,起革故生新的創造的作用,此即是道德的性體心體之創造。依儒家,只有這道德的性體心體之創造才是真實而真正的創造之意義,亦代表著吾人真實而真正的創造的生命,所謂「於穆不已」者是。這是吾人理解「創造性原則」最重要的法眼,切不可忘記。這也是創造性原則之最基本、最原初而亦最恰當的意義,它不是生機主義的生物學的生命之創造,亦不是宗教信仰上的上帝之創造,更復不是文學家所歌頌的天才生命之創造。因為生物學的生命之創造,是實然的自然生命之本能,不真是能創造的;文學家所歌頌的天才生命是情感生命底光彩,其底子還是實然而自然的生命,這還是才性的,所以講天才,亦講江郎才盡,這都不是經過逆覺而翻上來的道德生命、真實而真正的精神生命之創造。就是宗教信仰所說上帝之創造,若真是落實了,還是這道德的性體心體之創造。

性體心體在個人的道德實踐方面的起用,首先消極地便是消化生命中一切非理性的成分,不讓感性的力量支配我們;其次便是積極地生色踐形、醉面盎背,四肢百體全為性體所潤,自然生命底光彩收飲而為聖賢底氣象;再其次,更積極地便是聖神功化,仁不可勝用,義不可勝用,表現而為聖賢底德業;最後,則與天地合德,與日月合明,與四時合序,與鬼神合吉凶,性體遍潤一切而不遺。性體心體在這樣體證之呈現中的起用便是以前所謂「繁興大用」,用今語說,則是所謂「道德的性體心體之創造」,亦即康德所謂

\newpage\thispagestyle{empty}\addtocounter{page}{-1}\vspace*{-12mm}\begin{center}\noindent
\includegraphics[clip, trim=165pt 139pt 129pt 247pt, height=162mm]{ocr-input/image-0785.png}\end{center}

\newpage

\noindent 「因理性而鼓舞生力」,「只這睿智界底理念本身即是興發之力」。(惟康德如此說,只是理之當然之預定,而不是呈現。因為這不是我們人類理性所能理解的)。

在體證中如此呈現如此起用之性體自始即不限於人類而單為人類之性體,或甚至亦不限於康德所說一切理性的存在,而是頓時即通「天地之性」,「天地之中」,而為宇宙萬物之性體,因而亦就是宇宙萬物底本體、實體,此是絕對地普遍的,亦是道德實踐上絕對地必然的。此無論就孟子的「性善」之心性說,或就《中庸〉的「天命之謂性」之性以及誠說;無論就濂溪之誠、太極寂感之神說,或就橫渠之太和、太虛、天地之性說,或就明道之仁、天理、實體、於穆不已之體說,或就象山之本心即性即理說,或就陽明之良知說,或就蕺山之意說皆然。而總之曰:性即是道,性外無道;心即是理,心外無理。性、道(亦曰性、天)是道德的亦是宇宙性的性、道,心、理是道德的亦是宇宙性的心、理;而性、道與心、理其極也是一,故吾人亦總性體心體連稱。

此道德的而又是宇宙的性體心體通過「寂感真幾」一概念即轉而為本體宇宙論的生化之理、實現之理。這生化之理是由實踐的體證而呈現,它自必「顯諸仁,藏諸用,鼓萬物而不與聖人同憂,盛德大業至矣哉!」它自然非直貫下來不可。依是,它雖是超越的,而卻不是隔絕的。它與自然系統之「實然、自然」相接合不是一個待去構思以解決的問題,而是它的創造性之呈現之結果,是它的繁興大用之自然如此。這樣,「是什麼」或「發生什麼」或「應當發生什麼」底那「超越的必然性」全部透徹朗現,而不是一個隔絕的預定,無法為我們的理性所了解者。這樣,實然自然者通過「定然

\newpage\thispagestyle{empty}\addtocounter{page}{-1}\vspace*{-12mm}\begin{center}\noindent
\includegraphics[clip, trim=156pt 128pt 140pt 255pt, height=162mm]{ocr-input/image-0789.png}\end{center}

\newpage\markright{第一部 \quad 第三章 \quad 自律道德與道德的形上學}

\noindent 而不可移」,便與那超越的動態的「所以然而不容已」直下貫通於一起而不容割裂。儒家惟因通過道德性的性體心體之本體宇宙論的意義,把這性體心體轉而為寂感真幾之「生化之理」,而寂感真幾這生化之理又通過道德性的性體心體之支持而貞定住其道德性的真正創造之意義,它始打通了道德界與自然界之隔絕。這是儒家「道德的形上學」之徹底完成。康德的「目的論判斷」通過美的判斷作媒介而與自然系統相接合,這層意思,若不是出之以這樣湊泊的方式,而真能實現出來,則必然就是這種「道德的形上學」。可惜他一間未達,一層未透,(自由為一隔絕之預定、設準,其本身之必然性不可理解,是一本質的關鍵),「道德的形上學」不能出現,而只完成了一個「道德的神學」。

撥開這「一間」,打通那一層隔,是要靠那精誠的道德意識所貫註的原始而通透的直悟的,亦即靠那具體清澈精誠惻怛之圓而神的渾全襟懷,這是儒聖底德慧生命之所開發。西方自始即無這種生命。以步步分解建構的方式而達至康德底造詣,亦算不易了。

康德由步步分解建構的方式給實踐哲學立下的限度實已隱函著這種「道德的形上學」之要實現。(在康德只是一間未達,一層未透,如適所說。)康德後德國理想主義底發展即向此「道德的形上學」之實現而趨。在此趨勢上,康德所開的「道德的神學」便與這「道德的形上學」合而為一,而打通了那一層隔。隔是「道德的神學」,不隔即是「道德的形上學」。試看現在德人謬勒(MaxMüller)的叙述:

\begin{quotation}\kaishu 康德以為倫理行為的意義可以下列格言形式表出:你要成為\end{quotation}

\newpage\thispagestyle{empty}\addtocounter{page}{-1}\vspace*{-12mm}\begin{center}\noindent
\includegraphics[clip, trim=174pt 147pt 125pt 243pt, height=162mm]{ocr-input/image-0793.png}\end{center}

\newpage

\begin{quotation}\kaishu 普遍的,從而成為超個別的。這裡所突破的界限是「個體
性」的界線。你要作任何人在你的立場上都要作的事,康德
的這一基本原則恰和存在論的一個格言成為對照:你要作你
所能作的,作任何人不能替你作的事。〔案:此兩格言不相
衝突,乃相補成。存在論的格言即函攝在上第一節所說儒家
踐仁盡性之第三義中。康德的分解建立只作到了上第一節所
說的第一義,而第二義與第三義則未能至。此其所以未充其
極。】那「超個體的普遍性」便是康德所謂的意志的本質所
在,這意志於是可以擺脫一切內容的制約,避開一切「異律
管制」,由讓自己去取決。在意志裡,在自由裡,現象世
界、有條件、受限定相對性的世界要被衝破;而自由意志
的無上命令是不受制約的,絕對的。因為在康德以後,德國
的理想主義裡(菲希德、謝林、和黑格爾),「絕對」就等
於「無限」,於是那原來的格言不復是:成為普遍、成為絕
對,而變為:要在無限裡成為絕對的,要成為無限者。精神
意志(意志是精神的本質,正如精神是意志的本質)的圓滿
是爭取那已經內在於自身的絕對性和無限制。在這特別的理
想主義及其倫理思想中,於是乃生出一種沒有內容的動力
論:倫理道德的努力便是無限的努力,努力一有限制,便是
不道德。

[……]

既然在「本質倫理學」的倫理裡,個人潛伏於「實有」的各
種秩序中,那麼這倫理也就是一種以潛隱為主的倫理;理想
主義的倫理卻正相反,基本上是一種「展現」的倫理;它脫\end{quotation}

\newpage\thispagestyle{empty}\addtocounter{page}{-1}\vspace*{-12mm}\begin{center}\noindent
\includegraphics[clip, trim=157pt 128pt 138pt 249pt, height=162mm]{ocr-input/image-0797.png}\end{center}

\newpage\markright{第一部 \quad 第三章 \quad 自律道德與道德的形上學}

\begin{quotation}\kaishu 離穩固安全的本質秩序,為能向無限的、無內容的境界突
進。在本質哲學的實在論裡,「實有」便是各種本質的秩
序,在這些本質秩序後面的「實有」卻不可見,它已經被認
為是本質秩序本身。〔案:本質哲學指柏拉圖、亞里士多德
下駭中世紀的聖多瑪這傳統哲學講。】在理想主義裡,實有
等於精神,卻是那從感性、從有限解脫出來的無限性,是絕
對自由的本身。自由的即是不屬別的存在管轄,自己保有自
己。無限、實有、絕對,我通稱為精神,因為它保有自己,
它存在於自己內,換句話說,它是自由的。那保有自由以
自我的保有來發展其天賦的無限性及其應承當的自由的那一
種「實有」的名稱便是精神。「成為無限」也就是「成為精
神」,也就是「成為自由」的。因此,在人走向無限的自
由、爭取他那天賦同時又應當自己承當和追求的「神性」途
徑上,理想主義主張一種無止境進步的樂觀主義。理想主義
的「展現」主張把人直接置於他本有的無限性之前:人本身
內那導引他、率領他、同時又約束他的是那突破界限的,那
超越的因素。人只是受他自己的一種可能性的束縛:成為絕
對的可能性。惟一的一種具有束縛力的連繫是他和他本身內
的那一上主(上帝)的連繫,人本身便是一潛勢的上主,現
下應當成就的上主。因此,他的法則並不是什麼具有內容的
法律,也不是什麼可以認知的客觀秩序,而僅僅是一種很明
確的方向:正是那向絕對進行的方向。這一方向的消極實現
是對各種感性的天賦、動機、和連繫的克服(依康德的意
見),積極方面則是無止息地求取精神化的行動(依菲希\end{quotation}

\newpage\thispagestyle{empty}\addtocounter{page}{-1}\vspace*{-12mm}\begin{center}\noindent
\includegraphics[clip, trim=281pt 163pt 123pt 234pt, height=162mm]{ocr-input/image-0801.png}\end{center}

\newpage

\begin{quotation}\kaishu 德),或者作為一種向著絕對的知識進行的辯證法式的過渡
以求克服一切片面的界定或暫時的安頓(依黑格爾),終於
達到在理性的直觀中和絕對本體作神秘的契合的極致(依謝
林)。\end{quotation}

\noindent 以上兩段文見《現代學人》第4期張康譯《存在哲學在當代思想界之意義》第四節。譯文中專門詞語原附有德文,略。張君告予,謬勒此文中的意思皆是海德格底意思。此兩段話中亦牽涉著「本質倫理」與存在哲學底「存在倫理」而作比論。但這一點,讀者可暫不過問。如有興趣,可詳看該譯文的全文。

我們所註意於這兩段話的是它表示著對於德國由康德而開的理想主義傳統之簡括敘述。這敘述大體是對的,名之曰「方向倫理」,或「展現的倫理」亦很恰當。(這當是海德格本人底意思,謬勒只作介述而已。)我們從這簡述中,很可以看出德國康德後理想主義底發展是向打通那一層隔而期完成「道德的形上學」之方向趨。關鍵是在由「自由」所表示的絕對性與無限性而直通那無限而絕對的神性以為我們自己最內在的本質、本性。(這本性就是正宗儒家所說的「性」之意義)。說「實有」,這就是最高的實有,宋、明儒之大宗所謂道體性體、心體、神體、仁體、誠體等;說精神,這就是最真實最內在的精神。這樣,意志自由與上帝存在不再是並列的兩個設準,像在康德本人那樣,而是打成一片而在「展現」中呈現。「人本身便是一潛勢的上帝,現下應當成就的上帝」,這話尤其中肯,這是東方宗教因而亦是儒教「人而神」的精神,(儒家所謂「人人皆可以為聖人」,佛家說「一切衆生皆可成

\newpage\thispagestyle{empty}\addtocounter{page}{-1}\vspace*{-12mm}\begin{center}\noindent
\includegraphics[clip, trim=145pt 138pt 152pt 245pt, height=162mm]{ocr-input/image-0805.png}\end{center}

\newpage\markright{第一部 \quad 第三章 \quad 自律道德與道德的形上學}

\noindent 佛」),這是與基督教「神而人」底教義不同的。但這卻是實踐理性充其極,「道德的形上學」實現後所必然要至的。在該譯文第六節又有這樣幾句話:

\begin{quotation}\kaishu 在康德以後的所謂「德國理想主義」裡,尚未有任何可以辨
認的本質秩序被宣佈為行為的準繩,而只是把那向無界限、
無輪廓、因此也就無內容、內在的絕對性,向著內在於人的
神性的突進宣佈為標準。不過在這無本質無輪廓、無可捉
摸、無限、絕對、整全的「實有」被宣佈為我們內在的標
準,這向著人的最內在部分的突進即是向著我們的最本質的
本性,在人的本然行為(那超越的行為)即是爭取自己、爭
取人內部天賦然而隱而不見的神性和無限性的當兒,「實
有」本身不是就成為我們人的「本性」(Natur),成為我
們最內在絕對的天賦素質,換句話說,成為我們的本質了
嗎?於是,在主體性裡便有神性在,於是在向後方進入主體
時便找到絕對;無限便也在主體裡得到發展。\end{quotation}

\noindent 在這幾句話裡,我們所註意的是:那無限、絕對、整全的「實有」本身就是我們人底「本性」,我們最內在絕對的天賦素質。這本性、這最內在絕對的天賦素質,很顯然就是正宗儒家所說的道德性而又是宇宙性的性體心體,當然不是才性氣性之類的人性,所以它是「人內部那天賦然而隱而不見的神性和無限性」。因為正宗儒家所說的性體心體同時是道德的,同時又是本體宇宙論的;它是我們的性,同時亦普遍而為「天地之性」,而為宇宙萬物的本體、實

\newpage\thispagestyle{empty}\addtocounter{page}{-1}\vspace*{-12mm}\begin{center}\noindent
\includegraphics[clip, trim=179pt 172pt 132pt 231pt, height=162mm]{ocr-input/image-0809.png}\end{center}

\newpage

\noindent 體,即生化萬物的寂感真幾。這也是「在主體性裡便有神性在」,「在向後方進入主體時便找到絕對」。因為這性體心體就是我們的真實而真正的主體性。在我們實踐的體證中,這「隱而不見的神性和無限性」即逐步朗現或頓時朗現,(性體心體只有隱顯,並無生滅),這就是所以名為「展現」的倫理之故。

由這對於德國理想主義底方向倫理、展現倫理的簡括敘述,我們很顯然可以看出理想主義在基本義理與方向上與儒家的成德之教並無二致。這方向倫理展現倫理是屬於儒家型的。這卻不是表面地枝枝節節地附會說它們同,或表面地枝枝節節地說它們根本不同,這是看穿了它們兩者的內部義理及最後的企向而見出其屬於同一類型。當然德國「理想主義傳統」底哲學思辨並沒有達到儒家那種具體清澈精誠側怛的溫潤平實而又高明圓熟之境,反過來,中國儒家傳統由踐仁盡性的體證表現出來的義理講說,東一句、西一句,亦沒有德國理想主義傳統底哲學思辨那樣嚴整而有系統的概念架構。要說不同,可以說出種種不同來:氣氛上、情味上、詞語上、思辨入路與方式上俱有不同。然而這不礙於其基本義理最後方向之屬於同一類型。只看「人而神」與「無限絕對的實有本身即為我們人的本性」這兩點即足夠。

自十九世紀末進入二十世紀以來,當代哲學底趨勢是反這種以實踐理性上的自由為中心而展開的主體主義的理想主義的。其他瑣瑣者不足論,與此相干的,我們註意英國的懷悌海與德國的海德格。前者是從宇宙論方面向外開,脫離那主體主義的中心而向客觀主義走,建立那客觀建構的宇宙論:以二十世紀來的物理、數學、邏輯底成就為底子,以美學情調為基本靈魂,以柏拉圖、亞里士多

\newpage\thispagestyle{empty}\addtocounter{page}{-1}\vspace*{-12mm}\begin{center}\noindent
\includegraphics[clip, trim=160pt 146pt 146pt 242pt, height=162mm]{ocr-input/image-0813.png}\end{center}

\newpage\markright{第一部 \quad 第三章 \quad 自律道德與道德的形上學}

\noindent 德式的原始基型為接合傳統的歸宿點,而建立起近代式的客觀主義的宇宙論。後者則是從存有論方面向外開,脫離那主體主義的中心而向客觀的獨立的存有本身之體會走,建立那客觀自性的存有論,上而拉開與宗教的距離,使宗教超然而獨存,不與哲學糾纏於一起,內而倒轉那以自由、無限、神性為中心的方向倫理、展現倫理而為以「存有」(實有)為中心的「存在倫理」:面對實有而站出來,把自己掏空,一無本性,一無本質,然而完全服役於實有便是人的本性、人的本質,即真實存在的人。

這兩種向外開的客觀建立,其基本靈魂都不是以道德意識為主的。懷悌海是美學情調,固無論矣,即海德格底靈魂深處恐亦是一種英雄氣的美學情調。(友人唐君毅先生曾提此意。)這且不論,我的意思是如此:如果實踐理性充其極而達至「道德的形上學」之完成,(在中國是儒家的形態,在西方是德國理想主義的形態),則這一個圓融的智慧義理本身是一個圓輪,亦是一個中心點,所謂「道樞」。說它是個圓輪,是說在這輪子底圓轉中,人若不能提得住,得其全,則轉到某方面而停滯了,向外開,亦都是可以的,上下、內外、正負,皆可開合。「道德的形上學」一旦完成,康德的那一層隔打通了,此就上帝說,雖超越而亦內在化了,人若順內在化的落實,提不住而真落下來了,則多從人的負面性(如罪)與有限性著眼,而再把上帝推遠一點,以保持其尊嚴,這也是可以的,這便是基督教的形態。這是上下的開,但不能憑這開來反對那實踐理性充其極的合。復次,那圓輪子本不外於「外」,若轉到外面而停滯了,見到外面亦有獨立性,就此而向外開,或開懷悌海式的宇宙論,或開海德格式的存有論,皆無不可,但若執此而與那圓輪子

\newpage\thispagestyle{empty}\addtocounter{page}{-1}\vspace*{-12mm}\begin{center}\noindent
\includegraphics[clip, trim=186pt 173pt 131pt 229pt, height=162mm]{ocr-input/image-0817.png}\end{center}

\newpage

\noindent 為對立,則非是。懷悌海的宇宙論終必收攝於這以實踐理性為中心的圓輪子內方能站得住。就海德格說,當「後天而奉天時」的時候,就是他的「存在倫理」。可是「後天而奉天時」原與「先天而天弗違」連在一起的。良知的當下決斷亦就是他的「存在倫理」中之存在的決斷,獨一無二的決斷,任何人不能替你作的決斷。可是良知的當下決斷原是本良知本體(即性體心體)而來,原是本「先天而天弗違」的道體性體而來,原不與康德所宣稱的格言相衝突,乃是本體以成用。若執著「後天而奉天時」一義而與「先天而天弗違」為對立,執著存在的決斷而忘其體,那便不對。此是內外的開合。復次,從正面踐仁盡性到圓熟之境,一切都平平,一切都落實,人若在此平平落實處,只見到那形而下的器而膠著於事相上,則從負面來觀察人生,多從空、無一面入,也是可以的:無卻那相對事相的執著、人為造作的不自然,而超顯那自然無為的境界,這便是道家;空卻那事相緣起流轉的自性而當體證空,這便是佛教。因為這負面的生命原也是那圓輪子所要化掉的。若執著於這從負面入手之所證而與那圓輪子為對立,便不對。此是正負之開合。最後,在踐仁盡性到圓熟平平之境,如羅近溪所謂「抬頭舉目渾全只是知體著見」,(「知體」即良知本體),人若在此提不住,見不到是「知體著見」,而只見到「抬頭舉目」之生理活動,如是,只去研究這生理活動本身也可以,這便是所謂科學,但若在此執持只有這生理活動是真實,並無你所說的良知本體,那便成了荒謬的唯物論,此即馬克斯是也。馬克斯的罪惡與黑格爾全無關,(乃至希特拉的罪惡也不能牽連到黑格爾等身上去),也只是黑格爾底哲學到圓融落實之境(思辨上的),馬克思這低劣的根器,只見到物,

\newpage\thispagestyle{empty}\addtocounter{page}{-1}\vspace*{-12mm}\begin{center}\noindent
\includegraphics[clip, trim=156pt 150pt 140pt 230pt, height=162mm]{ocr-input/image-0821.png}\end{center}

\newpage\markright{第一部 \quad 第三章 \quad 自律道德與道德的形上學}

\noindent 不見到其老師所說的精神,遂有那叛師背道的墮落。(共產黨因自己講唯物辯證法,遂稍偏愛及黑格爾,其實這全不相干。而自由世界則因希特拉的極權與馬派之講唯物辯證法,遂深惡黑格爾,造成自由世界之禁忌,這也是無謂的遷怒與怨尤。)是以這圓輪子在其圓轉底過程中可容納一切開合,而唯不能容納罪惡的唯物論。罪惡只能被消除,不能被容納。以上是就圓輪說。說它是一個中心點,是說由此收攝一切,由此開發一切。康德在其《實踐理性批判》序文中說:「只要當自由概念底實在性因實踐理性底必然法則而被證明時,則它便是純粹理性甚至思辨理性底全部系統之拱心石。」人生真理底最後立場是由實踐理性為中心而建立,從知性,從審美,俱不能達到這最後的立場。宗教信仰只是這中心底開合。中國儒家正是握住這「拱心石」的,而宋、明儒之大宗則是盛弘這拱心石而充其極而達圓熟之境者。

以上的縷述已經很長了,至此當止。本章的主要意思是在:

1.打通康德的那一層隔,而完成「道德的形上學」。

2.「道德的形上學」之完成,在一切問題性的辨論以外以上是有一個精誠的道德意識所貫註的原始而通透的直悟的。

3.這原始而通透的直悟是以儒聖的具體清澈精誠惻怛的圓而神之境為根據,也可以說是聖人所開發。這是一個絕大的原始智慧,不是概念分解的事。這是中國儒家傳統與德國理想主義底哲學傳統所不同的地方,雖在客觀義理與最後方向上屬同一類型。〔關於這一點,我請讀者參看唐君毅先生《人生之體驗》中〈自我生長之途程〉一文以及《人文精神之重建》中〈孔子與人格世界〉一文。我即從此兩文悟到孔子的精誠側怛的渾全表現所代表的那原始的智

\newpage\thispagestyle{empty}\addtocounter{page}{-1}\vspace*{-12mm}\begin{center}\noindent
\includegraphics[clip, trim=164pt 159pt 137pt 232pt, height=162mm]{ocr-input/image-0825.png}\end{center}

\newpage

\noindent 慧,並見到儒家何以一下子即能使實踐理性充其極而徹底完成了那「道德的形上學」,而康德則不能之故。唐先生此兩文都在多年以前發表,前文尤早,尚在抗戰時期。我由此兩文所悟到的意思蓄之已久,今始正式說出,聊作「辯以相示」。(莊子云:「聖人懷之,衆人辯之以相示也。」)讀者於此或可得一眉目。

\newpage\thispagestyle{empty}\addtocounter{page}{-1}\vspace*{-12mm}\begin{center}\noindent
\includegraphics[clip, trim=176pt 439pt 139pt 84pt, height=162mm]{ocr-input/image-0829.png}\end{center}

\newpage\markright{}

\chapter{道之本統與孔子對於本
統之再建}

\section*{引言}\addcontentsline{toc}{section}{引言}

自韓愈作〈原道〉而有道統之說,其言曰:「夫所謂先王之教者何也?博愛之謂仁,行而宜之之謂義,由是而之焉之謂道,足乎己無待於外之謂德。其文《詩》、《書》、《易》《春秋》,其法禮樂刑政。〔……】斯道也,何道也?曰:斯吾所謂道也,非向所謂老與佛之道也。堯以是傳之舜,舜以是傳以禹,禹以是傳之湯,湯以是傳之文、武、周公,文、武、周公傳之孔子,孔子傳之孟軻。孟軻之死不得其傳焉。」

此堯、舜、禹、湯、文、武、周公、孔子、孟子一線相承之道,其本質內容為仁義,其經典之文為《詩》、《書》《易》、《春秋》,其表現於客觀政治社會之制度為禮樂刑政。此道通過此一線之相承而不斷,以見其為中華民族文化之命脈,即名曰「道統」,自韓愈為此道統之說,宋、明儒興起,大體皆繼承而首肯之。其所以易為人所首肯,因此說之所指本是一事實,不在韓愈說之之為「說」也。(此非是一個人之學說之問題)唯韓愈說之,有

\newpage\thispagestyle{empty}\addtocounter{page}{-1}\vspace*{-12mm}\begin{center}\noindent
\includegraphics[clip, trim=157pt 137pt 138pt 257pt, height=162mm]{ocr-input/image-0833.png}\end{center}

\newpage

\noindent 點醒之用耳。

此一線相承之事實,孟子已有此自覺。孟子曰:「由堯舜至於湯,五百有餘歲,若禹、臯陶,則見而知之;若湯,則聞而知之。由湯至於文王,五百有餘歲,若伊尹、萊朱,則見而知之;若文王,則聞而知之。由文王至於孔子,五百有餘歲,若太公望、散宜生,則見而知之;若孔子則聞而知之。由孔子而來,至於今,百有餘歲。去聖人之世,若此其未遠也;近聖人之居,若此其甚也。然而無有乎爾,則亦無有乎爾。」(〈盡心〉篇末),此即一線相承之意也,末後之慨嘆即示孟子自己欲以道之傳承自任也。其實顏淵、曾子,皆見而知之,而孟子亦可說是聞而知之也。對文王而言,不獨太公望、散宜生為見而知之也,即周公、召公亦可說是見而知之也。

此一線相承之事實,即孔子亦有此自覺。子曰:「殷因於夏禮,所損益可知也。周因於殷禮,所損益可知也。其或繼周者,雖百世可知也。」(《論語·為政》第二)有因、有損益,此三代王者之相承也。此雖就禮言,然亦可洞悟其相承之立國之道也。孔子甚嚮往周公制禮作樂之業績,有云:「甚矣,吾衰也!久矣,吾不復夢見周公!」(〈述而〉第七)又盛贊堯舜、禹之至德。由此即可見孔子實已洞悟到堯、舜三代一線相承之立國之道也。又曰:「文王殁,文不在兹乎?」(〈子罕)第九)此示孔子亦欲以繼承此道為己任也。

孔孟之如此稱述是後起者之反省,但亦自有其生命上之承當,(此之韓愈以及宋、明儒之如此稱述皆然),至於堯、舜三代之當事人尤其易意識到其世代之相承,故當湯武革命之際必歷舉前代為

\newpage\thispagestyle{empty}\addtocounter{page}{-1}\vspace*{-12mm}\begin{center}\noindent
\includegraphics[clip, trim=173pt 143pt 138pt 250pt, height=162mm]{ocr-input/image-0837.png}\end{center}

\newpage\markright{第一部 \quad 第四章 \quad 道之本統與孔子對於本統之再建}

\noindent 鑑戒也。

然自堯、舜三代以至於孔子乃至孔子後之孟子,此一系相承之道統,就道之自覺之內容言,至孔子實起一創闢之突進,此即其立仁教以闢精神領域是。孔子並非一王者,故其相承堯、舜三代之道,並非與三代之王者為同質地相承。此是其虛歉處。然亦正因此,而使道有「直方大」之解放,此又是其充盈處。此一創闢之突進,與堯、舜三代之政規業績合而觀之,則此相承之道即後來所謂「內聖外王之道」(語出《莊子·天下》篇)。此「內聖外王之道」之成立即是孔子對於堯、舜三代王者相承之「道之本統」之再建立。內聖一面之彰顯自孔子立仁教始。曾子、子思、孟子、《中庸》、《易傳》之傳承即是本孔子仁教而展開者。就中以孟子為中心,其器識雖足以籠罩外王,然重點與中點以及其重大之貢獻實落在內聖之本之挺立處。宋儒興起亦是繼承此內聖之學而發展。其器識雖足以籠罩外王,亦從未忽視於外王,然重點與中點亦仍是落在內聖之本之挺立處。此內聖之學,就其為學言,實有其獨立之領域與本性,此即彰著道德之本性(自性)以及相應道德本性而為道德實踐所達至之最高歸宿為何所是者是。自孔子立仁教後,此一系之發展是其最順適而又是最本質之發展,亦是其最有成而亦最有永久價值之發展,此可曰孔子之傳統。

然此道是內聖外王之道,則外王一面亦是器識上所必應函攝到之本質的一面,此即《大學》所謂治國平天下者是。「外王」者,即客觀而外在地於政治社會方面以王道(非霸道)治國平天下之謂也。此所謂「王道」已不是指三代王者所實際表現者而言,而是已成一有確定意義之政治上之最高原則。粘附於三代王者之名而

\newpage\thispagestyle{empty}\addtocounter{page}{-1}\vspace*{-12mm}\begin{center}\noindent
\includegraphics[clip, trim=154pt 157pt 146pt 239pt, height=162mm]{ocr-input/image-0841.png}\end{center}

\newpage

\noindent 名之曰「王道」者,是因三代之王者稍能幾近於此原則,或能表現此原則於幾分之幾之故也。然此「外王」一名,隨時代之發展與需要,其函義甚廣泛。初不只是指個人為君之道言,尤其不是指要人去為王言。如適所言,乃是於政治社會方面以王道治國平天下之謂,此是客觀地就治國平天下之最高原則之實施言。然隨時代之發展與需要亦常不只此義。是以外王一名,其函義,若總持言之,大體可分為三層:

一、客觀而外在地於政治社會方面以王道治國平天下:此是其初義,亦是其基本義。就「以王道治國平天下」言,此中含有政治之最高原則如何能架構成而可有實際之表現之問題,亦含有政體國體之問題。

二、在此最高原則以及此最高原則所確定之政體國體之下各方面各部門開展進行其業務之制度之建立:此是其第二義,亦即永嘉派所謂「經制事功」者是。

三、足以助成此各方面各部門業務之實現所需有之實際知識之研究與獲得:此是其第三義,此大體是顧亭林與顏、李等之所嚮往。

以上三義俱為外王一名所函攝。亦可以說是相連而生者,然而卻有其層次之不同。第一層為政治,踐之者為政治家。第二層為事功,踐之者為百官眾有司以及社會上之各行業。第三層為知識,踐之者(言實際去研究)為專家為學者。從問題言,此三層中之問題俱屬外王之問題。從學言,此三層之內容俱為外王學。

自宋儒興起,重新確認並展開孔子之傳統後,此外王學一面亦常因迫切之需要而為人所註意,而且常在華族受欺凌於夷狄而覆亡

\newpage\thispagestyle{empty}\addtocounter{page}{-1}\vspace*{-12mm}\begin{center}\noindent
\includegraphics[clip, trim=172pt 133pt 118pt 248pt, height=162mm]{ocr-input/image-0845.png}\end{center}

\newpage\markright{第一部 \quad 第四章 \quad 道之本統與孔子對於本統之再建}

\noindent 於夷狄時,如南宋時永嘉派之薛士龍、陳君舉、葉水心,以及永康之陳同甫,明末時之顧亭林黃梨洲、王船山,以及顏習齋李恕谷等皆甚著重此一面。如能相應外王學之第一義而解決中國之政治問題,相應其第二義與第三義而能開出各種科學知識以引發並充實各行業各部門之事功;此豈非佳事?然而著重此方面而反對談內聖之學者卻甚無所成。積極地不能就外王學之第一義開出中國政治之方向,復不能就第二義與第三義開出各種科學知識之規模,而只知消極地泛言事功與實用以為反對談內聖之學者(談性命天道者)之藉口。彼等不知彼等所要求之事功與實用,其最大之症結與關鍵是在政治問題(政治之最高原則如何架構之問題)之不得決,家天下之私是其先天之限制,彼等於此方面之意識皆甚差,且根本不甚能接觸此問題,不及黃梨洲、王船山遠甚,即比程、朱、陸、王之專致力於內聖之學者亦不及。彼等結果只是落於第二義第三義泛言事功實用以詬詆談性命天道者。此不知孔子仁敎之意義,復不知外王之根本。平心而論,中國之政治問題,在以往之思路中本不易解決者。此固非内聖之學所能直接推演而決者。程、朱、陸、王只知不滿於漢、唐,但亦開不出解決之道。即黃梨洲、王船山已甚接觸及之,但亦有至此而窮之感。此是中國以往學術傳統之限制。善乎友人徐復觀先生之言曰:「凡是想要把政權固定為一身一家一集團的千秋之業的想法,結果必走上李承晚、吳廷琰之路。而這一罪惡且悲慘的局面,在中國傳統文化中,追索到底,竟找不出答案,這是東方人的良心呈現所受的最大的限制」。(見〈良心、政治、東方人〉,《民主評論》第14卷第23期)。然無論如何,就儒家「內聖外王之道」之全體言,宋、明儒講內聖之學者對於此全體更

\newpage\thispagestyle{empty}\addtocounter{page}{-1}\vspace*{-12mm}\begin{center}\noindent
\includegraphics[clip, trim=173pt 159pt 135pt 241pt, height=162mm]{ocr-input/image-0849.png}\end{center}

\newpage

\noindent 能契接而相應;就外王一面言,講內聖之學者對於此一面更易有開發、有實感、有貢獻;即就顧、黃、王言,雖並稱為明末三大儒,而不反談性命天道之黃梨洲與王船山,即較顧亭林為更能實感於外王之癥結,因而亦更能接近於事功問題之解決。以往之限制只是一間未達、一結未解。一旦解開此糾結,則此問題之解決仍須順能籠罩而相應於「內聖外王之道」之全體者之創造生命與綜和意識而前進,而只泛言事功與實用以詬詆談性命天道者決無助於此問題之解決,以其方向與眼目不清故也。其歸也只是詞章考據而已。須知事功不立;國族淪亡,豈因談內聖之學而至此耶?徒見其不知類而已矣。

泛言事功與實用而詬詆談性命天道之內聖之學者以葉水心為最極端而澈底。彼不但反其並世周張二程,且並曾子、子思、孟子、《中庸》、《易傳》而一起詬詆之,甚至連孔子亦為其所不滿。彼根本無所知於孔子之仁教,自亦無所知於承孔子仁教而展開之「孔子之傳統」。彼以堯舜三代王者之業績為「道之本統」之所在,且只落於外王學之第二義與第三義而觀之,以此定其即事達義,即器明道之〈講學大旨〉。彼所以詬詆孔子之傳統者,正為其不合其所說之上世之「本統」與古人之「禮統」之故。彼以為心性、性命、天道等問題皆是「茫昧冥惑」之事,皆非上世所曾有,皆是子思、孟子所造作之「新說奇論」以冥惑後世者。後之顧亭林與顏、李,乃至戴東原諸反對宋、明儒之談性命天道者皆不能出其規模之外,而皆不及其勇悍。本文之作,是欲就其論點澈底疏導此問題以明堯舜三代道之本統之何所是以及孔子之創闢突進對於道之本統再建之意義,並進而就其〈講學大旨〉之詬詆曾子、子思、孟子、《中

\newpage\thispagestyle{empty}\addtocounter{page}{-1}\vspace*{-12mm}\begin{center}\noindent
\includegraphics[clip, trim=164pt 134pt 126pt 250pt, height=162mm]{ocr-input/image-0853.png}\end{center}

\newpage\markright{第一部 \quad 第四章 \quad 道之本統與孔子對於本統之再建}

\noindent 庸》、《易傳》予以澈底之糾正,以明內聖之學之獨立的意義與自性,兼為外王之學進一解。「道」者精神生命之方向之謂也。一民族如不能對此點有澈底清醒之確立與挺立,則必永停於軟塌恣肆、顛倒搖擺,甚至凍結,而不能暢達屹立其自己之境。

見本章先論「道之本統與孔子對於本統之再建」,下章再衡定葉水心(講學大旨〉之紕謬。茲為方便起,先從「性」字說起。

\section{孔子前性字之流行及生、性二字之互用
與不互用}

\noindent I、見於《詩》《書》中者

1.《詩·大雅·卷阿》:

\begin{quotation}\kaishu 伴奐爾游矣,優游爾休矣。豈弟君子,俾爾彌爾性,似先公
酋矣。〔朱注:「言使爾終其壽命,似先君善始而善終也」。一解
「似」,嗣也,紹也,續也。「酋」就也,成就也。〕爾土字
章,亦孔之厚矣。豈弟君子,俾爾彌爾性,百神爾主矣。
〔朱註:「使爾終其身常為天地山川鬼神之主也」〕爾受命長
矣,茀祿爾康矣,豈弟君子,俾爾彌爾性,純嘏爾常矣。\end{quotation}

2.附錄金文:

\newpage\thispagestyle{empty}\addtocounter{page}{-1}\vspace*{-12mm}\begin{center}\noindent
\includegraphics[clip, trim=157pt 200pt 140pt 244pt, height=162mm]{ocr-input/image-0857.png}\end{center}

\newpage

\begin{quotation}\kaishu 永令彌厥生。.\end{quotation}

\noindent 照朱註,上條「彌爾性」與此條金文「彌厥生」同,生性互用。依友人徐復觀先生解,字雖互用,實即「性」義。性作欲望講。「彌爾性」,「彌厥生」,皆言滿足其欲望。彌者滿也。猶言遂心滿意,事事如意也。表善頌善禱之吉祥辭。「俾爾彌爾性,似先公酋矣」,猶言「使你百事如意,來紹續先公之成就」。「似」,嗣也,紹也。「酋」,就也,言成就。此解較優。(參看《中國人性論史·先秦篇》,第一章頁10。)

3.《商書·西伯戡黎》(今文):

\begin{quotation}\kaishu 祖伊[……]曰[……]非先王不相我後人,惟王淫戲用自
絕。故天棄我,不有康食,不虞天性,不迪率典。\end{quotation}

\noindent 蔡沈注:「不虞天性,民失常心也」。「不虞」猶不慮、不顧。此天性常心是一般言之,即生命生活中自然有者之常態。失其常態,則自然不顧一切,盲動亂行。故下句即繼之言「不迪率典」,言「廢壞常法也」(蔡沈注)。

4.《周書·召誥》(今文):

\begin{quotation}\kaishu 王先服殷御事,比介於我有周御事。節性,惟日其邁。\end{quotation}

\noindent 「節性」,蔡注:「節其驕淫之性。」此性字即生命中自然有的欲望本能等。此項節、導,而不能縱,如此方能日進其德。此處是

\newpage\thispagestyle{empty}\addtocounter{page}{-1}\vspace*{-12mm}\begin{center}\noindent
\includegraphics[clip, trim=165pt 122pt 129pt 257pt, height=162mm]{ocr-input/image-0861.png}\end{center}

\newpage\markright{第一部 \quad 第四章 \quad 道之本統與孔子對於本統之再建}

\noindent 性、德對言,此見性在下,屬自然而實然,德在上,(儘管是外在的),屬當然而應然。「節性」,字亦可寫為「節生」,然其實只是性義,不是今日所謂「節制生育」之「節生」也。以上是今文《尚書》,無問題者。

5.《商書·仲虺之誥》(古文):

\begin{quotation}\kaishu 嗚呼!惟天生民有欲,無主乃亂。惟天生聰明,時又。天乃
錫王勇智,表正萬邦。王懋昭大德,建中於民。以義制事,
以體制心。\end{quotation}

\noindent 案:此為古文《尚書》,文中無「性」字。然「生民有欲」句可助解生、性,故併錄之。據此文,天不但「生民有欲」,且賦人以「聰明」,錫人以「勇智」,此亦性也,此是所以主宰欲者。此文中之觀念可能較晚,此其所以為偽與?然其觀念並不與傳統相悖。

6.《商書·太甲上》(古文):

\begin{quotation}\kaishu 伊尹曰:茲乃不義,習與性成。\end{quotation}

\noindent 案:此言習性,習久成性也。此習、性對言。此可表示凡性字無論是何層面之性,是何意義之性,皆是指那自然而本然者言,即自然如此本然如此之性向、性能、性好、質性或質地。此是性字之通義,但視其應用於何層面而定其殊指。大抵性之層面有三:一、生物本能、生理欲望、心理情緒這些屬於自然生命之自然特徵所構成的性,此為最低層,以上各條所說之性及後來告子、荀子所說之性

\newpage\thispagestyle{empty}\addtocounter{page}{-1}\vspace*{-12mm}\begin{center}\noindent
\includegraphics[clip, trim=166pt 142pt 143pt 252pt, height=162mm]{ocr-input/image-0865.png}\end{center}

\newpage

\noindent 即屬於此層者;二、氣質之清濁、厚薄、剛柔、偏正、純駁、智愚、賢不肖等所構成之性,此即後來所謂氣性才性或氣質之性之類是,此為較高級者,然亦由自然生命而蒸發;三、超越的義理當然之性,此為最高級者,此不屬於自然生命,乃純屬於道德生命精神生命者,此性是絕對的普遍,不是類名之普遍,是同同一如的,此即後來孟子、《中庸》、《易傳》所講之性,宋儒所謂天地之性、義理之性者是。本條伊尹所說之「習與性成」之性大體可說是概括前兩層俱在內。太甲之自然生命所蒸發之氣質本有其駁雜,或一時陷溺而轉不過,故常習於不善不順,習久而成惡性,儼若其性如此。其實這種性總有其可導性與可化性,故伊尹放之於桐宮,「密邇先王其訓,無俾世迷。」此即函可使其一旦覺悟而向善,開擴其心靈而不終于沉迷。而太甲亦終于覺悟而迴向,故曰「王徂桐宮居憂,克終允德。」此條習與性對言,雖是直接關涉著氣質之性說,然亦有向道德心靈開發之透視。故此文之觀念亦當屬後起,大抵最早所注意之性只是就自然生命之自然特徵而說者。

7.《周書·施獒》(古文):

\begin{quotation}\kaishu 犬馬非其土性不畜。\end{quotation}

\noindent 「土性」即方土之質地,適宜于生長或畜養某物者。

8.《商書·湯誥》(古文):

\begin{quotation}\kaishu 惟皇上帝,降衷於下民,若有恆性。克綏厥猷惟后。\end{quotation}

\newpage\thispagestyle{empty}\addtocounter{page}{-1}\vspace*{-12mm}\begin{center}\noindent
\includegraphics[clip, trim=167pt 176pt 127pt 254pt, height=162mm]{ocr-input/image-0869.png}\end{center}

\newpage\markright{第一部 \quad 第四章 \quad 道之本統與孔子對於本統之再建}

\noindent 案:此文語脈本甚順,而古注則纏夾不明,遂不順矣。問題在「若有」二字,古注皆訓若爲順。《孔傳》云:「衷,善也。」正義曰:「天生蒸民,與之五常之性,使有仁義禮智信,是天降善於下民也。」此解「降衷」句。類比「天命之謂性」也。《孔傳》又謂:「順人有常之性,能安立其道教,則惟為君之道。」是則將「若」字講為動詞之「順」,而把原為動詞之「有」字與「恆」字合,解為「有常」,此是一形容詞,「有常之性」,即是「恆性」;而又與「克綏」句連為一氣,類比「修道之謂教」,此於語句不合。若如《孔傳》,則經文當如此標點:「惟皇上帝,降衷於下民。若有恆性,克綏厥猷,惟后。」此甚別扭。蔡沈註曰:「若,順也。〔……】以降衷而言,則無有偏倚。順其自然,固有常性矣。以稟受而言,則不無清濁純雜之異,故必待君師之職而後能使之安於其道也。故曰克綏厥猷惟后。」此則將「若」字之「順」義承上「降衷」而言,較《孔傳》爲合矣。然將一「若」字措為一整句,亦與「若有恆性」句之辭意不合。依蔡注,此文之句意是:若能順所降之衷之自然,則固有常性矣。把「若」字措為一條件句橫插於其中,此甚不合。此皆視「若」字之「順」義太死殺,「若」固有順義,亦別處亦可直解為動詞之順,如「不若德」之類。但此處之「若」字只是作為一順承上句之連繫字,即訓「順」,亦是連繫詞「因而」之順,「遂亦」之順,「似若」之順,不能固定為動詞之順也。「若」本亦發語辭,如「曰(越)若稽古」之若,召誥「越若來」之若。此處之若即連繫辭發語辭合用之「若」。言上帝降衷于下民,下民因而有其恆性。恆性即定常之性,亦即〈西伯戡黎〉中「不虞天性」之「天性」。

\newpage\thispagestyle{empty}\addtocounter{page}{-1}\vspace*{-12mm}\begin{center}\noindent
\includegraphics[clip, trim=10pt 115pt 140pt 193pt, height=162mm]{ocr-input/image-0873.png}\end{center}

\newpage

此「天性」、「恆性」之內容在此處究如何解析亦不易有明確之規定。孔穎達《正義》順《孔傳》解「衷」為善,將此恆性直解為仁義禮智信「五常之性」。蔡沈註則謂:「天之降命而具仁義禮智信之理,無所偏倚,所謂衷也。」衷即「中」,故以「無所偏倚」解之,而其實指則是「仁義禮智信之理」。人稟受之即為一己之性。如是,則「恆性」之內容即為義理當然之性。然自湯「克夏誕告萬方」而言,則此「恆性」之內容恐不能直指此「義理當然之性」而說。天生民有其發於生性之所需與所欲以及其一定之生活軌道,此皆不能虐亂者。「夏王滅德作威,以敷虐於爾萬方百姓」,此即虐亂其生性也。故湯起而革之,以安生民。故曰:「凡我造邦,無從匪彝,無即慆淫,各守爾典,以承天休。」此種申述,即在呼應「克綏厥猷惟后」一句也。然則此處「恆性」之內容恐不能即是義理當然之性,恐是就人之生命生活綜和而言之。其實指當是指發於生性之所需、所欲以及一定之生活軌道、一般之生活常態而言。此亦是生民之恆性常性也。

天之「降衷於下民,若有恆性」,在語句上,就「恆性」言,極似「天命之謂性」,就「降衷」言,又似《左傳》劉康公所謂「民受天地之中以生」。(劉子語詳解見下第二節)。天降衷,民受中,因而有其生也。即有其個體生命之存在。有其生命之存在,即有其恆性。此「中」,在劉康公是就之以說個體之存在(生)。有個體之存在即有個體生命之「命」,即根命、氣命之命。故「天地之中」是偏就中之氣一面說,即中和之氣。而此處之「降衷」,則是就「衷」以說性。此「衷」與「性」究如何說,則不易有明確之決定。此處不說「受天地之中」。而說「上帝降衷」,此於解析

\newpage\thispagestyle{empty}\addtocounter{page}{-1}\vspace*{-12mm}\begin{center}\noindent
\includegraphics[clip, trim=163pt 124pt 128pt 251pt, height=162mm]{ocr-input/image-0877.png}\end{center}

\newpage\markright{第一部 \quad 第四章 \quad 道之本統與孔子對於本統之再建}

\noindent 上亦有影響。上帝有人格神、最高主宰之意。祂之降衷於下民以使民有恆性,若說單是降中和之氣,則似乎不可通。若簡單地只說天生民,或上帝造民,則無論賦人以生命,或賦人以性,問題皆簡單。但現在是由降衷說性,其所降而賦予人者有特指,則問題便麻煩。當然上帝亦可賦人以中和之氣(至於如何賦法,則是上帝之事,非吾人所能知),但若如此,則直接來的是「生」(存在),而不是「性」。故就「性」言,依中國之傳統想法,只能說是降衷善之德或理。故《孔傳》云:「衷,善也。」而孔穎達《正義》直以仁義禮智信之德解之,而蔡註則以「仁義禮智信之中理」說之。此皆類比《中庸》之首句以解此文。但若如此,則下句「克綏厥猷惟后」便不甚好講。《孔傳》將「若有恆性」連下「克綏」句為一氣,而謂:「順人有常之性,能安立其道教,則惟為君之道」,此對於「若有恆性」句,決不可通,說已見前。但以「能安立其道教」解「克綏厥猷」,卻尚切合。此整文,若隨《孔傳》,改為如下之解說,當可通順,即:「惟皇上帝降衷善之德於下民,下民因而有其恆性。〔順人之恆性】而能安立其道教者,則惟賴在上之君后」。此中「順人之恆性」是補充句,不是解詁「若有恆性」。「能立其道教」即《中庸》「修道之謂教」也。如是,「惟皇上帝降衷於下民,若有恆性」,此兩句合起來即是「天命之謂性」。「克綏厥猷惟后」一句則是「修道之謂教」。此中並無「率性之謂道」一義。自政治立場言,跨過此句亦可。而《孔傳》則以「惟皇上帝降衷於下民」一句為「天命之謂性」,而以「若有恆性,克綏厥猷惟后」兩句相連屬為「修道之謂教」,故差謬不通也。

蔡沈注,將「若有恆性」屬上「降衷」句,固較《孔傳》為

\newpage\thispagestyle{empty}\addtocounter{page}{-1}\vspace*{-12mm}\begin{center}\noindent
\includegraphics[clip, trim=161pt 147pt 136pt 244pt, height=162mm]{ocr-input/image-0881.png}\end{center}

\newpage

\noindent 佳,但解「克綏」句,則不如《孔傳》之切合。蔡注云:「猷,道也。由其理之自然而有仁義禮智信之行,所謂道也。〔……]以稟受而言,則不無清濁純雜之異,故必待君師之職而後能使之安於其道也。故曰:克綏厥猷惟后。」實則原是后「能安立其道教」,(如《孔傳》),或「能綏寧其道」,(如此,「道」須別解,見下),而不是「能使之安於其道」。夫民有仁義禮智之常性矣,能不能率性而行而「安於其道」,豈「必待君師之職而後能」耶?依《中庸》,將道修之於家國天下而成教(教化、風教),所謂「修道之謂教」,須「待君師之職而後能」,然「率性之謂道」,則不須「待君師之職而後能」也。故此解於語句為不切,又於義理為不通也。蔡沈之註是根據朱子之義理而說,並想補上「率性之謂道」一義,故如此解「克綏厥猷」也。實則「猷」字雖可訓道,而不必即是「率性之謂道」之「道」。「猷」字之「道」義不必如此著實也。故蔡註解此句不如《孔傳》較為合於政治立場也。

以上兩解皆類比《中庸》而言之,又對於「若」字之「順」義,看得太死寂,故有許多差謬並參差不齊處。修改而通之,當以《孔傳》爲佳,或兩解皆當歸於順《孔傳》而修改之。

但若衡之〈湯誥〉之「誕告萬方」之政治立場,此文似可另作一稍為虛籠之解析,使之較為直接而顯豁,不必類比於《中庸》,如該兩解之著實。此稍為虛籠之解析當如此:上帝以生民愛民為衷。此衷可指上帝之衷善之心意說。此衷善之心意不能著實為民之義理當然之性。上帝其衷善之意於下民,使民有其恆性故不可讓殘暴者隨意虐待而擾亂之。是則民之有此恆性固是關聯著上帝之衷善之意而說,但此恆性自身之內容卻當就民另講,不能即由上帝之

\newpage\thispagestyle{empty}\addtocounter{page}{-1}\vspace*{-12mm}\begin{center}\noindent
\includegraphics[clip, trim=160pt 125pt 129pt 257pt, height=162mm]{ocr-input/image-0885.png}\end{center}

\newpage\markright{第一部 \quad 第四章 \quad 道之本統與孔子對於本統之再建}

\noindent 「衷」定,亦不能直指仁義禮智之理說。如是,則「衷」字稍為虛籠。由仁義禮智說義理當然之性,只當人註意到個人自己有道德自覺而能自主地作道德實踐時,始能出現。此是通過孔子之仁教後,孟子、《中庸》、《易傳》之所發揚。湯克夏,誕告萬方,不必說此義。當然,此是古文《尚書》,人以為偽者也。其編輯此文,或許夾雜有《中庸》之觀念,故順之而輯成此句,而解者亦順《中庸》而解之。人可以此而益譏其偽。吾在此不甚注意此問題。吾只因舊解之纏夾,而想到此虛籠之解析。如依舊解,則當順《孔傳》而修改,如上所說。依「誕告萬方」之政治立場,此虛籠之解析亦可通。如此即不必用《中庸》來解此〈湯誥〉之語矣。是故此處順上帝之降衷而言民有恆性,其內容當指「發于生性之所需所欲以及一般生活之常態」而言。民有此恆性,即須有安定之社會秩序以遂其生達其欲,使之「不失其性」。故此「恆性」即是〈西伯戡黎〉中之「天性」。如此,恆性即是「實然之生性」。而不是超越之義理當然之性。能安定社會秩序使不失其性者,則是君后之職責,以今語言之,即是政治。故曰「克綏厥猷惟后」。猷有多義,謀、圖、道、功皆可。古註訓道。若訓道,即是生活之道或常規。言「能綏寧其生活之道者惟后也」。此「道」(猷)字無超越之意義。而是表現在日常生活之事中之常規。此是屬於客觀的社會的一面者。故此處不必直以《中庸》「天命之謂性,率性之謂道,修道之謂教」連三語之義解之。其語脈思路雖近之,而理趣不必同也。《中庸》「天命之謂性」即是超越的義理當然之性,故下句即云「率性之謂道」,此則是個人自己率性盡性之事,不須「克綏厥猷惟后」也。但是發於生性之生活常規之維持與安排,不能不有待於

\newpage\thispagestyle{empty}\addtocounter{page}{-1}\vspace*{-12mm}\begin{center}\noindent
\includegraphics[clip, trim=149pt 144pt 143pt 243pt, height=162mm]{ocr-input/image-0889.png}\end{center}

\newpage

\noindent 政治「即君后」。如此解,則直接而顯豁。《中庸》則複雜。此解亦有缺點,即對於「降衷」之解析,不合傳統之習慣。如依舊解,則當順《孔傳》而修改之。如此,兩解可並存。吾之對於「上帝降衷」作稍為虛籠之解析,是想保存〈湯誥〉此文與《左傳〉劉康公「民受天地之中以生所謂命也」以及《中庸》「天命之謂性」此三者各自之殊義以及此三者所形成之發展性,即生性、根命與超越的義理當然之性三者各自之殊義以及其發展性。

\noindent Ⅱ、見於《左傳》中者

1.《左傳〉襄公十四年晉師曠答晉侯「衛人出其君」中有云:

\begin{quotation}\kaishu 天生民而立之君,使司牧之,弗使失性。有君而爲之貳,使
司保之,勿使過度。〔又云:】天之愛民甚矣。豈其使一人
肆於民上,以從其欲,而棄天地之性?必不然矣!\end{quotation}

\noindent 此言天地以愛民為性。此是超越意義之性。「弗使失性」即使人民各遂其生,各適其性,此指生活欲望等言。此條可與前(湯誥〉「降衷、恆性」之文合觀。

2.《春秋》襄公二十六年經:「冬楚子蔡侯陳侯伐鄭。」《左傳〉:

\begin{quotation}\kaishu 冬十月楚子伐鄭,鄭人將禦之。子產曰:晉楚將平,諸侯將
和,楚王是故昧於一來。不如使逞而歸,乃易成也。夫小人
之性釁於勇,嗇於禍,以足其性而求名焉者。非國家之利\end{quotation}

\newpage\thispagestyle{empty}\addtocounter{page}{-1}\vspace*{-12mm}\begin{center}\noindent
\includegraphics[clip, trim=170pt 125pt 116pt 252pt, height=162mm]{ocr-input/image-0893.png}\end{center}

\newpage\markright{第一部 \quad 第四章 \quad 道之本統與孔子對於本統之再建}

\begin{quotation}\kaishu 也,若何從之?子展說,不禦寇。\end{quotation}

\noindent 「小人之性」,此性字是本性義,是其生命特徵之總說。「以足其性」,此性字是欲望,是發於其特有之本性之特有的欲求。

3.昭公二十五年鄭子太叔引子產語答趙簡子問禮:

\begin{quotation}\kaishu 夫禮天之經也,地之義也,民之行也。天地之經,而民實則
之。則天之明,因地之性,生其六氣,用其五行。氣為五
味,發為五色,章為五聲。淫則昏亂,民失其性。是故為禮
以奉之。〔……】民有好惡喜怒哀樂。〔……】哀樂不失,
乃得協於天地之性,是以長久。\end{quotation}

\noindent 此言「天地之性」是以禮則為性,此亦是超越意義的性。前第一條以「愛民」為性,是就天地之心言性,此條是就天地之理道言性。「民失其性」,此性字即「好惡喜怒哀樂」等之情性,亦自然生命所自然而有者。不可絕,亦不可淫,須有禮以節導之。此條亦當與前〈湯誥〉「降衷、恆生」之文合觀。

4.昭公十九年《左傳》:

\begin{quotation}\kaishu 楚人城州萊·沈尹戌曰:[……】吾聞撫民者節用於內,而
樹德於外,民樂其性,而無寇讎。〔……]\end{quotation}

\noindent 樂其性即樂其生也。樂其生當然是其生活之欲望有相當之滿足。

5.昭公八年《左傳》:

\newpage\thispagestyle{empty}\addtocounter{page}{-1}\vspace*{-12mm}\begin{center}\noindent
\includegraphics[clip, trim=156pt 143pt 143pt 246pt, height=162mm]{ocr-input/image-0897.png}\end{center}

\newpage

\begin{quotation}\kaishu 八年春,石言於晉魏榆。晉侯問於師曠曰:石何故言?對
曰:石不能言,或.焉。不然,民聽濫也。抑臣又聞之曰:
作事不時,怨識動於民,則有非言之物而言。今宮室崇侈,
民力彫盡,怨讥並作,莫保其性。石言,不亦宜乎?\end{quotation}

\noindent 案:「莫保其性」即莫保其生。此則只應作「生」字讀解,雖則其義亦通上第三條「民失其性」之句意。

6.附錄:《周禮·大司徒》:

\begin{quotation}\kaishu 辨五土之物生。\end{quotation}

\noindent 案:此「物生」實即「物性」,亦如前引〈旅獒〉「犬馬非其土性不畜」之土性,此則只應作性字讀解。
* * *

綜觀以上所引《詩》、《書》及《左傳》明言「性」字之諸文,自人言,皆指實然之生性而言。(〈湯誥〉「降衷、恆性」,若從吾解,亦然。)字面雖可互用,然有時義同於生,有時義同於性;自義(觀念)言,生性究是兩義。大抵造字先有生字,後漸孳乳性字。自性之觀念言,其初只是直接就生而言性。所謂實然之性即是自生而言性也。自生言性,性非即生也。初民文字簡略,字可互代字。雖可通用互代,而觀念生,則義實有別。生與性各自有義。究從生,抑從性,則由上下文語脈決定。不能消滅性字之獨立義,而謂性即是生也。自生言性只表示自「自然生命」之特徵言之耳。董仲舒云:「性之名非生與?」此固不錯,然下句即繼之云:

\newpage\thispagestyle{empty}\addtocounter{page}{-1}\vspace*{-12mm}\begin{center}\noindent
\includegraphics[clip, trim=173pt 126pt 125pt 254pt, height=162mm]{ocr-input/image-0901.png}\end{center}

\newpage\markright{第一部 \quad 第四章 \quad 道之本統與孔子對於本統之再建}

\noindent 「如其生之自然之資謂之性」,此即示「性」字有自義。自生言性,性非即生。說本質本性,亦是就其自然之特徵而總說。此總說是描述地總說,故總為實然之性也。

自物言,「物生」、「土性」,字雖可互用,然實只是「性」義。「物生」即物之質性,「土性」即土之性能,亦是指自然實然之質性或性能而言。

惟《左傳》師曠及子產所言之「天地之性」(就天地而言其性)則有超越的意義與道德價值的意義,此開後來就人言超越之性或義理當然之性之門,此是自德而升進者,此超越乎「自生而言性」之上,此非是經驗的、實然的,此須有一種道德的、形上的洞見。

\section{孔子後言超越意義之性之傳統背景—
《詩》《書》中所表現之道德總規與
政規}

除上明言性字外,《詩》、《書》、《左傳》中復有另一組超越意義與道德意義的觀念,如下:

1.《詩·大雅·烝民》:

\begin{quotation}\kaishu 天生烝民,有物有則。民之秉彝,好是懿德。

〔孟子引此詩以證性善。]\end{quotation}

\newpage\thispagestyle{empty}\addtocounter{page}{-1}\vspace*{-12mm}\begin{center}\noindent
\includegraphics[clip, trim=170pt 191pt 122pt 246pt, height=162mm]{ocr-input/image-0905.png}\end{center}

\newpage

2.《詩·周頌·維天之命》:

\begin{quotation}\kaishu 維天之命,於穆不已。於乎不顯,文王之德之純。

〔《中庸〉引此詩明天之所以為天以及文王之所以為文,純亦不
已。]\end{quotation}

3.《詩·周頌·昊天有成命》:

\begin{quotation}\kaishu 昊天有成命,二后受之。成王不敢康,夙夜基命宥密。於緝
熙,單厥心,肆其靖之。\end{quotation}

4.《周書·召誥》(今文):.

\begin{quotation}\kaishu 王乃初服。鳴呼!若生子,罔不在厥初生,自貽哲命。今天
其命哲?命吉凶?命歷年?知〔語詞】今我初服,宅新邑,
肆惟王其疾敬德。王其德之用,祈天永命。\end{quotation}

5.《周書·洛誥》(今文):

\begin{quotation}\kaishu 王如弗敢及天基命定命,予乃胤保,大相東土,其基作民明
辟。\end{quotation}

6.〈堯典〉(今文):

\newpage\thispagestyle{empty}\addtocounter{page}{-1}\vspace*{-12mm}\begin{center}\noindent
\includegraphics[clip, trim=158pt 178pt 146pt 261pt, height=162mm]{ocr-input/image-0909.png}\end{center}

\newpage\markright{第一部 \quad 第四章 \quad 道之本統與孔子對於本統之再建}

\begin{quotation}\kaishu 克明俊德。\end{quotation}

7.《商書·太甲上》(古文〉:

\begin{quotation}\kaishu 伊尹作書曰:先王顧諟天之明命,以承上下神祇。社稷宗
廟,罔不祗肅。\end{quotation}

8.《商書·咸有一德》(古文):

\begin{quotation}\kaishu 惟尹暨湯咸有一德,克享天心,受天明命,以有九有之師,
爰革夏正。

〔又:〕德惟一,動罔不吉。德二三,動罔不凶。

〔又:】德無常師,主善為師。善無常主,協於克一。\end{quotation}

9.《周書·康誥》(今文):

\begin{quotation}\kaishu 惟乃丕顯考文王,克明德慎罰。\end{quotation}

10.《周書·召誥》(今文):

\begin{quotation}\kaishu 惟王受命,無疆惟休,亦無疆惟恤。鳴呼!曷其奈何弗敬!
〔又:]鳴呼!天亦哀於四方民,其眷命用懋,王其疾敬
德!

〔又:]王敬作所,不可不敬德。\end{quotation}

\newpage\thispagestyle{empty}\addtocounter{page}{-1}\vspace*{-12mm}\begin{center}\noindent
\includegraphics[clip, trim=231pt 140pt 130pt 244pt, height=162mm]{ocr-input/image-0913.png}\end{center}

\newpage

\begin{quotation}\kaishu 〔又:]惟不敬厥德,乃早墜厥命。

〔又:]宅新邑,肆惟王其疾敬德。王其德之用,祈天永
命。\end{quotation}

11.《左傳》成公十三年:

\begin{quotation}\kaishu 公及諸侯朝王。遂從劉康公成肅公會晉侯伐秦·成子受脈
於社,不敬。劉子曰:吾聞之,民受天地之中以生,所謂命
也。是以有動作禮義威儀之則,以定命也。能者養之以福,
不能者敗以取禍。是故君子勤禮,小人盡力。勤禮莫如致
敬,盡力莫如敦篤。敬在養神,篤在守業。國之大事,在祀
與戎。祀有執勝,戎有受脈。神之大節也。今成子惰,棄其
命矣。其不反乎?\end{quotation}

\noindent 案:以上共十一條,首條「民之秉彝,好是懿德」,由「秉彞」已十分接近於說「性」,故孟子引之以證性善。「有物有則」是客觀地說。「民之秉彝,好是懿德」,則是主觀地說,即由好懿德以見人所秉持之常性。為此詩者確有道德的洞見,亦有道德的真實感,故能直下從則、道,說到內心好德之實,即說到定然之乗彝之性。雖未明言性字,亦必然要逼至矣。故孟子直引之以證性善也。

末條劉康公之言,從「民受天地之中以生」說命,不是說性。此「命」是生命性命之命,「性命」是通常所謂「性命根子」之性命,故此命亦即是「根命」之命。不是「維天之命」之天命之命,亦不是「天命之謂性」之命令之命。「民受天地之中以生」言稟受

\newpage\thispagestyle{empty}\addtocounter{page}{-1}\vspace*{-12mm}\begin{center}\noindent
\includegraphics[clip, trim=156pt 125pt 143pt 257pt, height=162mm]{ocr-input/image-0917.png}\end{center}

\newpage\markright{第一部 \quad 第四章 \quad 道之本統與孔子對於本統之再建}

\noindent 天地之中以得其「存在」也。得其存在自然是得其「個體生命之存在」。此顯是以個體生命之存在規定「命」,故此命即是根命之命。劉康公是要說此義,尚未直下從「天地之中」說那超越的義理當然之性。人由天地之中而得有一命,便須有禮敬以凝定或貞定其命,故曰「君子勤禮」,又曰:「勤禮莫若致敬」。怠惰放肆,不敬無禮,則示其由天地之中所受之命搖動而不貞固,其命亦不可久矣。故劉康公由成肅公之「不敬」而預言其「棄其命矣,其不反乎?」

此命既是根命之命,則「天地之中」,若依後來之詞語釋之,則當是偏於「氣」之一面說。天地之中即天地沖虛中和之氣,或元一之氣。若由此義之「中」說性,則性即是後來所謂氣性才性之類。此亦自生而言性,但此「生」卻是根源地說,即形上地宇宙論地偏就氣之一面說,不是前節所引明言「性」字者(除「天地之性」外)之大抵皆就自然生命之徵象而為經驗地描述地說。此比描述地說為推進一步,此中亦有一種形上的超曠的洞見。性命根子之命即是此「氣命」。如果此命即是性,即性命根子之性,則此性即是氣命之性。

若由「天地之中」說道德意義之超越之性,如《中庸》「天命之謂性」之性,則是偏就天地之心、理道一面說。此種意義之性即是義理當然之性、內在道德性之性,此是萬善萬德之所從出,此則只應「盡」。人應盡性即盡其義理當然之性所有之義理當然之要求,即盡其性分所命令汝必須為者。劉康公尚未進至說此種性之境,禮敬尚在外在的作用中,尚未能內在化稱義理當然之性體而說。故只言由禮敬以定人之氣命或根命,而不能言克己復禮以盡性

\newpage\thispagestyle{empty}\addtocounter{page}{-1}\vspace*{-12mm}\begin{center}\noindent
\includegraphics[clip, trim=173pt 153pt 128pt 245pt, height=162mm]{ocr-input/image-0921.png}\end{center}

\newpage

\noindent 也。

此雖未至由「天地之中」以說義理當然之性,然由其推到根源,形上地宇宙論地就中和之氣以說氣命,亦函氣性,則已開由天地之中偏就心理道一面而說義理當然之性之門,而可使人逐步逼近之矣。通過孔子之仁教後,此義即出現。〈湯誥〉「惟皇上帝,降衷於下民,若有恆性」,即已由「衷」(中)以說「恆性」矣。惟此由「降衷」所說之「恆性」,究屬氣性,抑屬義理當然之性,在〈湯誥〉之語脈里,尚不能明確地表現出。自下句「克綏厥猷惟后」觀之,亦可屬氣性生性,如吾前文第一節所解。民有恆常之生性。即一般生活所需所欲之常規,而能綏靖而安理之者,則惟賴在上之君后。若屬義理當然之性,則不必賴君后矣。至後來《中庸》說「天命之謂性」,則其為義理當然之性已確定矣。此皆與劉康公之語同一思路,同一語脈,但內容卻甚不同,由天地之中說命,進到說「天命之謂性」,甚至由「上帝降衷說恆性」進到明確地說「天命之謂性」,尚須有思想上之進一步的啟迪與發展。故劉康公之言固不能認為即是「天命之謂性」,即〈湯誥〉之語亦不能明確地認為即與《中庸》之語同其理趣也。此只能是《中庸》說性之預備,而已幾近之矣。

2、3、4、5四條為一組。

第2條天命於穆不已,文王之德純亦不已,與天同其精誠不息也。此將「天命」理解為天道健行之不息,「命」理解為流行之命,後來宋儒理會為「天命流行之體」不誤也。文王精進其德亦如此,法天也。為此詩者確有其形而上的深遠之洞悟,亦有其對於道德踐履之真實感與莊嚴感。此詩影響甚大,於儒家對於天道之體悟

\newpage\thispagestyle{empty}\addtocounter{page}{-1}\vspace*{-12mm}\begin{center}\noindent
\includegraphics[clip, trim=158pt 129pt 137pt 252pt, height=162mm]{ocr-input/image-0925.png}\end{center}

\newpage\markright{第一部 \quad 第四章 \quad 道之本統與孔子對於本統之再建}

\noindent 與對於德性人格之嚮往有決定性之影響,此確能反映出儒家心靈之核心。後來通過孔子而進一步發展的《中庸》與《易傳》皆可說是承此詩之理境而為進一步之闡揚。其進一步處即在認此「於穆不已」為性體,天道與性命打成一片。但此詩則尚未至此。此詩只是對於天道有此洞悟,只是贊美文王之德行,尚未至即以此「於穆不已」之體為吾人之性體也。就德行言,尚只是作用地或從成就上(所謂丕顯)說,尚未至內在化點出吾人所以能日進其德之內在而固有的性體,即內在而固有的道德創造之真幾。然由此詩之理境而向此進一步之義而趨亦是理上應有之發展。

第3條贊成王「夙夜基命宥密,於緝熙,單厥心,肆其靖之。」朱子注此詩非常恰當,注云「基,積累於下以承藉乎上者也。宥,宏深也。密,靜密也。」「此詩〔……】言天祚周以天下,有定命,而文武受之矣;成王繼之,又能不敢康寧,而其夙夜積德以承藉天命者,又宏深而靜密,是能繼續光明文武之業,〔此句解「緝煕」不諦】,而盡其心,故今能安靜天下而保其所受之命也。」實則可簡單化如此說:「成王能早晚宏深而靜密地積德以基其所受于天及祖之成命。常明不昧〔緝熙】以盡其心,故能靖其成命而不搖動也。」此實亦「純亦不已」之意。「於緝煕」是「宥密」之轉換語,皆副詞。「宥密」是形容其「基命」,而「緝煕」則是承接之以形容其「盡心」。前條贊文王者弘大,而此詩則深密。緝煕以盡心,宥密以基命,皆是作用地關聯地言其德之精進,亦尚未至轉「宥密、緝煕」為形容名詞以直指吾人之性體與心體。然而後來通過孔子之發展,則向此而趨,直以「於穆不已」之真幾或宥密緝熙之「純亦不已」者以為吾人之性體心體矣。

\newpage\thispagestyle{empty}\addtocounter{page}{-1}\vspace*{-12mm}\begin{center}\noindent
\includegraphics[clip, trim=184pt 162pt 120pt 235pt, height=162mm]{ocr-input/image-0929.png}\end{center}

\newpage

第4條「罔不在厥初生自貽哲命」,此亦教訓地作用地說,即「靡不有初,鮮克有終」之意。一切皆決於「初服」。在「初服」之時,應當戒慎恐懼,「疾敬其德」,以自貽其「哲命」。「德」亦不是吾人內在之性德,「敬德」亦不是居敬以盡吾人之性德,乃是行事之合理(率典),此猶是外在者,故是作用地關聯著「命哲」、「命吉凶」、「命歷年」而說。然到後來則內在化而言盡性以「自貽哲命」矣。能當下盡性即是「初服」。是則初服是稱體而說,不只是教訓地作用地說。此是道德意識道德理境之進一步的挺立。此須通過孔子之仁始能至此。

第5條是陪襯。取「基命定命」二詞以作助解。基命定命皆是作用地基其所受於天之命,定其所受於天之命。此「命」是受命為王之命,尚不是如劉康公所說定每人自己所受於「天地之中」之根命。更不是後來所說盡吾人所受於「天地之中」之性。然通過孔子,則步步向此而趨。

6、7、8、9、10,五條為一組。

第6條「克明俊德」,意即能明大德,此尚不是吾人之本心性德。後來《大學》引之以言「明明德」亦不足以表示「明德」即是吾人之本心性德。宋、明儒皆解為每人固有之本心性德,此是根據孟子、《中庸》《易傳》之言心體性體以解《大學》之「明德」,此已將所成就之德行內在化而點出其所以能成此德行之內在根據,即超越之心體性體,指名曰「明德」。故「明德」已成一實體字。此固不是《尚書》中言「俊德」之原意,亦不必是《大學》言「明德」之意。〈堯典〉「克明俊德」是形容堯之德行之成就,承上文「欽明、文思、安安、允恭克讓,光被四表,格於上下」,

\newpage\thispagestyle{empty}\addtocounter{page}{-1}\vspace*{-12mm}\begin{center}\noindent
\includegraphics[clip, trim=137pt 135pt 149pt 242pt, height=162mm]{ocr-input/image-0933.png}\end{center}

\newpage\markright{第一部 \quad 第四章 \quad 道之本統與孔子對於本統之再建}

\noindent 並貫下文「以親九族」等等而言。此種形容古代帝王之人格正恰是反映中華民族所嚮往之最高德性人格何所是之心靈方向,而此心靈方向正恰為儒家之道德意識所代表。後人如此塑造古聖正表示後人之心向,而後人雖不能至,亦必黽勉以此為準。而後之繼起者亦正表現此型範,為此型範作見證,如文王,如孔子,皆是其著者。〈舜典〉稱舜曰:「濬哲文明,溫恭允塞,玄德升聞,乃命以位。」德性人格必以「溫潤瑩澈如玉」為準,決非如外邦之形容其聖人之熱烈與誇誕也。此則重德行修養者與重信仰之激情者之異也。如《詩·大雅》之歌頌文王曰:「穆穆文王,於緝熙敬止。」(〈文王〉章)。「離雝在宮,肅肅在廟。不顯亦臨,無射亦保。肆戎疾不殄,烈假不瑕。不聞亦式,不諫亦入。」(〈思齊〉章)。「無然畔援,無然歆羨,誕先登於岸。」「不大聲以色,不長夏以革〔此句未詳】,不識不知,順帝之則。」(〈皇矣)章)。此皆言其穆穆、緝煕精純溫恭安安、宥密之德行,皆是如如之實,無絲毫露相之光彩。至於《論語》所表現之孔子尤其臻於廣大精緻、高明平實之化境,宋儒所謂天地氣象也。「望之儼然,即之也溫」,其德性之純備不可以一端論,要以既通且化之「溫潤瑩澈如玉」為準的,此則前聖後聖其揆一也。〈皐陶謨〉言九德:「寬而栗,柔而立,愿而恭,亂而敬,擾而毅,直而溫,簡而廉,剛而塞,彊而義」,皆以相反者之融化為德之成與真,偏於一端皆非真德,此只有對於道德踐履有真實感者方能知之。此種「在辯證的融化中以成其實德」之工夫與造詣自必消除一切虛妄與誇誕而自臻於無聲無臭之化境。此之謂「玄德」。有此德性人格之嚮往,則通過孔子之仁教後,言道範者自必要點明其所以,而挺

\newpage\thispagestyle{empty}\addtocounter{page}{-1}\vspace*{-12mm}\begin{center}\noindent
\includegraphics[clip, trim=166pt 160pt 136pt 231pt, height=162mm]{ocr-input/image-0937.png}\end{center}

\newpage

\noindent 立起人人所以能臻此境之超越根據(性體心體),此即孟子、《中庸》、《易傳》之所為也。

第7條〈太甲〉上,伊尹所說:「顧諟天之明命」固是古文《尚書》所輯錄,然《大學》亦引此語以言「明明德」之「皆自明」。凡「明」之工夫固皆須自明,然《尚書》此句動詞在「顧」字,而「明命」之明為狀詞,亦如「哲命」之「哲」「成命」之「成」,皆指王者得有政權所受於天之命言,非個人自己所棄受於天之明德或心體性體也,亦非明此明德或盡此心盡此性也。《大學》言「明明德」亦不能表示此義。「顧諟天之明命」言先王(湯)能常顧念正視此上天所授于我之明命而不敢怠忽。能常如此顧念而目在之,自能提醒吾人之道德意識,然此猶是外在他律之德行。通過孔子以後之發展,講出一超越之性體,則是向內在自律而發展。此是孟子、《中庸》、《易傳》之所言,而《大學》則不必能表示是此義。

第8條〈咸有一德〉之一德即純一無間之德,與「文王之德之純」同。此亦就精進工夫之純一言,非實體字之「一」也,亦非一以貫之」之「普遍之一」也。

第9條〈康誥〉「克明德慎罰」,《大學》亦引之,然皆不表示「德」是本心性德。「克明德」不是能明「明德」,乃是能明于「德」,此德亦不是內心固有之性德,乃是合理率典之德行,故下文云:「不敢侮鳏寡,庸庸、祗祗、威威、顯民,用肇造我區夏」,此即是明於德謹於罰也。此亦是作用地關聯著天之大命而言德行,非是稱體自律而言本心性德也。《大學》轉言「明德」亦不必是本心性德,亦可能只是光明的德行。然由此道德意識之常精進

\newpage\thispagestyle{empty}\addtocounter{page}{-1}\vspace*{-12mm}\begin{center}\noindent
\includegraphics[clip, trim=159pt 139pt 132pt 237pt, height=162mm]{ocr-input/image-0941.png}\end{center}

\newpage\markright{第一部 \quad 第四章 \quad 道之本統與孔子對於本統之再建}

\noindent 必逐步要逼近於內在自律稱體而行之盡性之德行。

第10條〈召誥〉是《詩》、《書》中就三代王者受命而言德之總規。此道德之總規是作用地關聯著明命、哲命、成命、大命而言,總之是關聯著其受命與永命而言,故曰:「王其德之用,祈天永命。」此道德之總規即是其「政規」,所謂原始之「王道」者是也。此三代王道之「政規」,其德與道之表現是在王者團聚群體以開物成務中表現,即在原始的綜和構造中表現。葉水心所謂「道之本統」、「古人之體統」,即是如此之本統、如此之體統。彼即據此以攻擊曾子、子思、孟子、《易傳》,以及其同時代之周、張、程、朱也,如此隔絕,勢必連孔子亦須抹倒。此未能真明道之本統者也。三代之本統固不錯,孔子亦有承於此本統,但真正之本統是在通過孔子而成之本統之重建中,並不在限於「政規」之原始的綜和構造也。即就堯舜三代之本統言,孔孟所註意者,其重點亦是在「其德之用,祈天永命」處,即德之總規、政規之「規」字,此乃見德之所以為德、道之所以為道者,此仍是提起來以見道之本統者。此道之本統是由主觀方面之「敬德」與客觀方面之帝、天、天命、天道而規定者。其表現雖是作用地關聯著天之大命而表現,然畢竟亦是其有德與有道處,故言道之本統,其重點不能不在此處著眼也。此只有在聖者生命之通契中,始能自我作主地在存在之實感中如此認識道也。而葉水心之言本統,其重點卻是落在此道德之總規、政規所成之業績,即依之而起之原始綜和構造,單重其所凝聚,而忽其所以凝聚,以所凝聚之業績為道之所在,此則正是重其末而忽其本,羨其果而忘其因,單以經制事功之自身為道,故流於現象主義、實證主義而不自知。夫道可以傳,而業績不可傳。傳其

\newpage\thispagestyle{empty}\addtocounter{page}{-1}\vspace*{-12mm}\begin{center}\noindent
\includegraphics[clip, trim=164pt 158pt 130pt 225pt, height=162mm]{ocr-input/image-0945.png}\end{center}

\newpage

\noindent 道,而業績則隨時。葉水心以不可傳之三代業績為道之本統之所在,故一方不知孔子之仁教為道之本統之再建,一方復不知曾子、子思、孟子、《易傳》乃承孔子之仁教而展開,故極力輕薄而詬詆之,而道之傳承遂泯滅而不見,此真斷潢絕港之死見,而猶侈言道之本統耶?關於葉水心之思想詳見下章。茲略提於此。

以上是關於明言「性」字之文獻以及關於天命、天道敬德、祈天永命之文獻。前者自生而言性,是一個暗流,不及後者之彰顯,而後者則是通過孔子後孟子、《中庸》、《易傳》言性命天道之先在背景。由此背景言性是自理或德而言性,是超越之性,是理想主義的義理當然之性,是儒家人性論之積極面,亦是儒家所特有之人性論,亦是正宗儒家之所以為正宗之本質的特徵。自生而言性是實在論態度的實然之性,是後來所謂氣性、才性、氣質之性,是儒家人性論之消極面,不是儒家所特有,如是儒家而又只如此言性,便是其非正宗處。積極面之人性論之成立,孔子之仁是其重要的關鍵。如是,吾人須進而論孔子何以是道之本統之重建,而孔子所以不常言「性與天道」以及子貢之所以言「不可得而聞」之故亦有可得而言者。

\section{孔子所以不常言「性與天道」以及子貢
所以言「不可得而聞」之故}

前言「自生而言性」是一暗流之老傳統,在孔子以前就流行。而子貢所說「夫子之言性與天道不可得而聞」之性,孔子對此性之態度究如何,現在雖無明文可徵,恐亦不即是「自生而言性」(後

\newpage\thispagestyle{empty}\addtocounter{page}{-1}\vspace*{-12mm}\begin{center}\noindent
\includegraphics[clip, trim=157pt 146pt 134pt 232pt, height=162mm]{ocr-input/image-0949.png}\end{center}

\newpage\markright{第一部 \quad 第四章 \quad 道之本統與孔子對於本統之再建}

\noindent 來告子所謂「生之謂性」)之性。縱亦有此意,孔子究亦未十分正面去談它。「性相近也,習相遠也」,伊川謂此是屬於氣質之性,蓋就「相近」而想。因義理當然之性人人皆同,只是一,無所謂「相近」。惟古人辭語恐不如此嚴格。孟子言:「其日夜之所息,平旦之氣,其好惡與人相近也者幾希。」孟子此處所言之「相近」恐即是孔子「性相近」之「相近」。如是,「相近」即是發于良心之好惡與人相同。孔子恐亦即是此意。如是,孔子此句之「性」當不能是「自生而言性」之性,亦不必如伊川講成是氣質之性。但上智下愚不移之類則是屬於後來所謂氣性才性者。然此究非孔子所積極正視而討論之之問題。又,如果《易》之〈彖〉〈象〉真是孔子所作,則〈乾彖〉「乾道變化各正性命」語中之「性」正是上節所謂積極面之性,是自理道或德而言之「超越之性」,此性是與天道天德貫通於一起的。如此,則孔子對於「性與天道」並非不言,亦並非無其超曠之諦見。子貢不可得而聞自是子貢之事。但若以《論語》爲準,衡之孔子之真精神乃在仁,仁是其真生命之所在,是其生命之大宗,如是則說此積極面之「性」非其所常言,非其所積極正視而討論之之問題,亦並非不可。因為此問題畢竟是其後繼者孟子《中庸》、《易傳》之所積極弘揚者。衡之思想之發展亦應是如此。孔子不能一時俱言也。此積極面之性,其傳統背景是《詩》、(書》中所表現之道德總規(亦即政規):「王其德之用,祈天永命。」但這一總規中之諸觀念要發展到此積極面之性之建立,非通過孔子之仁不能出現。孔子本人對此或許已有憧憬,然正式消化而建立此種積極面之性,說是孔子後繼者之工作,則較妥當而順適。因為這種意義之性並不通常,乃是一新創造,而促成此

\newpage\thispagestyle{empty}\addtocounter{page}{-1}\vspace*{-12mm}\begin{center}\noindent
\includegraphics[clip, trim=167pt 160pt 132pt 228pt, height=162mm]{ocr-input/image-0953.png}\end{center}

\newpage

\noindent 創造,孔子之仁是一本質而重要之關鍵。至於復於道德實踐中來消化並安立「自生而言性」之實然層面之性之意義與作用,則更是後來的事。此決非孔子一人所皆能言及者。本節目的不在就子貢之語之字面上去猜測,而在以仁為宗斷定性之問題不是孔子所積極正視者,如此說其不常言可,縱或言之,而子貢說「不可得而聞」亦可,此「可」之故亦有可得而言者。本節即在進一步說其所以然之故。

天道與性稍不同。帝、天、天道、天命之觀念是顯著之老傳統,孔子對之自極親切而熟習,何以亦可說孔子不常言?縱或言之,何以子貢竟亦不可得而聞?此亦必有可得而言者。非必孔子之不言,亦非必子貢之低劣。此非言不言之問題,亦非子貢低劣否之問題。決非如後世之隔絕論者、冥惑論者,以為此屬渺茫之事,孔子尚且不言,子貢尚且不可得而聞,而後世之清談君子卻不務實學,專作此無謂之空談,此豈聖人之道之所在乎?吾以為此斷然是隔絕論冥惑論者之陋,決非此問題之實義。孔子決非隔絕論者,亦非冥惑論者。故本節願將天道與性連同一起進一步而明其所以不常言與子貢所以「不可得而聞」之故。

無論對「性」字作何解析,深或淺,超越或實然(現實),從生(從氣)或從理,其初次呈現之意義總易被人置定為一客觀之存有,而為一屬於「存有」之事。凡屬存有,若真當一客觀問題討論之,總須智測。事物之存有與內容總是複雜、神秘而奧密。何況人、物、天地之性?天命天道是超越的存有,其為神秘而奧密(不說複雜),自不待言。縱使性字所代表者是比較內在而落實的存有,邵堯夫所謂「性者道之形體」,亦仍然是神秘而奧密。(在此

\newpage\thispagestyle{empty}\addtocounter{page}{-1}\vspace*{-12mm}\begin{center}\noindent
\includegraphics[clip, trim=158pt 140pt 140pt 243pt, height=162mm]{ocr-input/image-0957.png}\end{center}

\newpage\markright{第一部 \quad 第四章 \quad 道之本統與孔子對於本統之再建}

\noindent 亦不說複雜)。此是屬於康德所謂「物自體」者。至於自生而言性,淺言之,雖可極淺,而若深觀,則氣性才性亦非簡單,此不但神秘而奧密,且亦有無窮之複雜。此是屬於自然生命之事、個性之事。明夫此,則知孔子所以不常正式積極言之,縱或言之,而亦令人有「不可得而聞」之嘆之故矣!因孔子畢竟不是希臘式之哲人。性與天道是客觀的自存潛存,一個聖哲的生命常是不在這裡費其智測的,這也不是智測所能盡者。因此孔子把這方面——存有面——暫時撇開,而另開關了一面—仁、智、聖。這是從智測而歸於德行,即歸於踐仁行道,道德的健行。這是從德行盡仁而開闢了精神領域,這似乎是自己所能把握的:「我欲仁,斯仁至矣」,「一日克己復禮,天下歸仁焉。」孔子對仁似乎極有清晰的觀念,亦有極旺盛的興趣。雖對之無定解,無確詁。看似無把柄,然亦可以說任說任通,句句精熟,這是圓音,並非滯辭。他在這裡表現了開朗精誠、清通簡要、溫潤安安、陽剛健行的美德與氣象,總之他表現了「精神」、生命、價值與理想,他表現了道德的莊嚴。性與天道是自存潛存,是客觀的、實體性的、第一序的存有,而仁智聖則似乎是凌空的、自我作主地提起來的生命、德性,其初似乎並不能直接地把它置定為客觀的、實體性的、自存潛存的存有,因此它似乎是他自己站起來自己創造出的高一層的價值生命。他的渾淪表現,也沒有定說仁是本體性的心,或是什麼自存潛存的本體性的道,尤其沒有想到這就是我們的實體性的性。但在孔子,仁也是心,也是道,雖然《論語》中並沒有講到「心」字。至於說它就是我們的性,那是孟子的事。所以這是在第一序的存有——客觀的或主觀的—外,凌空開闢出的不著跡的「虛室生白吉祥止止」的居間領

\newpage\thispagestyle{empty}\addtocounter{page}{-1}\vspace*{-12mm}\begin{center}\noindent
\includegraphics[clip, trim=159pt 159pt 132pt 227pt, height=162mm]{ocr-input/image-0961.png}\end{center}

\newpage

\noindent 域,但這卻是由其自我作主、自己站起來、自己創造出的陽剛天行而有光輝的領域,這是德行上的光輝,價值、生命、精神世界的光輝。人的生命在這裡是光暢的、挺立的。他的心思是向踐仁而表現其德行,不是向「存有」而表現其智測。他沒有以智測入於「存有」之幽,乃是以德行而開出價值之明,開出了真實生命之光。在這裡也有智,但這智是德行生命的瑩澈與朗照:它接於天,即契合了天的高明;它接于地,即契合了地的深厚;它接于日月,即契合了日月之明;它接于鬼神,即契合了鬼神的吉凶。在德性生命之朗潤(仁)與朗照(智)中,生死晝夜通而為一,內外物我一體咸寧。它澈盡了超越的存有與內在的存有之全蘊而使它們不再是自存與潛存,它們一起彰顯而挺立,朗現而貞定。這一切都不是智測與穿鑿。故不必言性與天道,而性與天道盡在其中矣。故曰「五十而知天命」,又曰「下學而上達,知我者其天乎?」又曰:「天何言哉?四時行焉,百物生焉。」而孟子,便說盡心知性知天,存心養性事天了。原來存有的奧密是在踐仁盡心中彰顯,不在寡頭的外在的智測中若隱若顯地微露其端倪。此就是孔、孟立教之弘規,亦就是子貢所以有「不可得而聞」之歎之故了。

孔子實不曾多就「存有」而窺測其「是什麼」,而只環繞聰明、勇智、敬德而統之以仁,健行不息以遙契天命,是猶是繼承《詩》、《書》中「疾敬德」,「祈天永命」之道德總規而使之益為深遠宏顯者。《詩》、《書》中是就夏、商、周三代王者之受命永命言。天能授命,亦能改命,故在人分上必須「疾敬德」,以「祈天永命」。而孔子則未受命而為王,有其德,無其位,故由「疾敬德、祈天永命」轉而為踐仁以知天,(事天畏天奉天俱在

\newpage\thispagestyle{empty}\addtocounter{page}{-1}\vspace*{-12mm}\begin{center}\noindent
\includegraphics[clip, trim=155pt 135pt 131pt 245pt, height=162mm]{ocr-input/image-0965.png}\end{center}

\newpage\markright{第一部 \quad 第四章 \quad 道之本統與孔子對於本統之再建}

\noindent 內)。此為聖者獨闢精神領域之盡倫立教,而非王者開物成務之盡制施政。在此有一現實上之委屈,而卻有一理想上之「直方大」;在此亦照察出現實與理想之距離,而必由聖者之盡倫立教以為超越之規範,籠罩駕臨於囿於現實之王者之盡制施政之上,而不能讓人類歷史永停於君師合一政教不分之原始的王者受命之不自覺的、渾淪的、囿於現實的之綜和構造中。原始的綜和構造固不錯,然唯是一人生命之凸出,而其他則唯在其廣被噓拂之中,則人道之尊嚴,每一德性生命之光輝,仍不能普遍地開其挺立之門。又囿於現實矣,則在綜和構造中雖有道之表現,而其所表現者已有限矣,而公私纏夾,義利混雜,則道之為道未可知也,縱使是道,其為道亦有限矣。歷史演進至孔子,似是冥冥之中必然要天降一「不有天下」之德性生命以闢精神、價值之源,以開生民光明其自己之門。此即仁教之所以立,而由践仁以知天也。践仁以光明每一生命之自己必落於個人之進德修業,而王者受命奉天承運以為綜和的構造,則必團聚群體以開物成務。一是散開,重個體;一是統聚,重群體。若無仁教以光明每一生命之自己,開理想、價值之源,予奉天承運者以限制與折衝,則此後者之團聚群體以居民上未有不強人從己,立理限事,私其位,縱其欲,肆於民上,以為極權專制者也。三代之綜和構造猶原始而渾樸,雖是家天下,猶未盡極其私,藏天下於筐筴,牽率天下以從己,然以此為至善,不准開闢精神領域以為價值、理想之源,以立生命德行之本,則未來之歷史以及進進不已之開合構造便不可能。此非真能知天運人紀者也。孔子之使命即在本「疾敬德」、「祈天永命」之政規轉而為「踐仁以知天」之道範以導夫政節乎君而重開文運與史運者也。葉水心淺陋無知,以其淺狹

\newpage\thispagestyle{empty}\addtocounter{page}{-1}\vspace*{-12mm}\begin{center}\noindent
\includegraphics[clip, trim=158pt 161pt 149pt 242pt, height=162mm]{ocr-input/image-0969.png}\end{center}

\newpage

\noindent 之聰明彷彿一二,渾不知孔子雖承於三代而與三代有別,遂滯道之本統於三代,停於原始的綜和構造為已足。看似開明,而實淺陋;看似平實,而實庸狹;看似內外兼備,心物交成,而實囿於現實。未知孔子實是本統之重建,因而遂落於現象主義、實證主義之末用平鋪,而不知立體創造之軒波,落於政規皇極之一元,而不知政規道範太極人極皇極之並建與本末之不可泯,不知一元只能自道範說,不能自政規說也。以原始的直接的綜和構造之即器明道、即事達義,內外相成為至極,而不知主體之不立,即無有真正道德之可言。故停於原始諧和(所謂原始的綜和構造)為已足,而不知「再度諧和」始為道之真正實現也。孔子獨關精神領域以立本源正是開再度諧和之關鍵。故道之本統只能斷自孔子,前乎孔子是其預備,後乎孔子是其闡發與其曲折之實現。焉可混抹孔子之開闢,而唯斷自三代王者受命之政規耶?

自孔子立仁教以闢精神領域,將「疾敬德」以「祈天永命」之王者受命之政規轉而為「踐仁以知天」之個人進德之道範以後,其門下受其啟迪而繩繩相繼者莫不本此道範以宏揚,雖或有未能盡夫子之全處,然而亦未有能離此本統者。曾子、子思(《中庸》)、孟子、《易傳》,其選也。孔子雖未就性與天道而作智測,然而其環繞聰明、勇智、敬德而統之以仁,由踐仁以知天,則實已逼顯出「不自生以言性而自德以言性」之途徑,孟子、《中庸》、《易傳》即順此途徑以進,此為仁教踐仁知天應有之義。孔子不言,而其教未始不函,則後人言之有何傷?後人之言性與天道亦不是當作一客觀置定之存有而智測之,而是統於踐仁知天以言之,如是,由孔子「踐仁以知天」乃轉進而為「踐仁盡性以知天」:孟子之盡心

\newpage\thispagestyle{empty}\addtocounter{page}{-1}\vspace*{-12mm}\begin{center}\noindent
\includegraphics[clip, trim=179pt 137pt 121pt 248pt, height=162mm]{ocr-input/image-0973.png}\end{center}

\newpage\markright{第一部 \quad 第四章 \quad 道之本統與孔子對於本統之再建}

\noindent 知性知天、存心養性事天,《中庸》之至誠盡性參天地贊化育是也。性天統于一仁中,如此之言性命與天道有何不可而必據孔子、子貢以為忌諱哉?孔子環繞古統聰明、勇智、敬德而統之以仁,由踐仁以知天,則後人再循其所環繞之觀念而連及降衷、恆性、天命、乗彞、於穆不已、基命宥密,而統性天於仁,由「德之純、純亦不已」,與「於緝熙單厥心」以言內在道德性之性或真實創造性之性,進而為「盡心知性以知天」或「至誠盡性以參天」,乃正是極順理成章者,此其眉目極顯豁,理路極清楚,即孔子不言而亦呼之欲出,躍如也。此乃仁教圓滿上之必然應有的發展,而葉適無知,必欲盡予以誹薄而抹除之,以復其三代卽器明道即事達義之本統,落於原始諧和而以現實平鋪為已足,而不知孔子重建本統使命之重大,亦復不知孔子踐仁知天以及孟子、《中庸》盡性知天亦正是本於古統王者受命之格局中作用地所函之政規——皇極疾敬德以祈天永命,再推進一步而為自體上以言之,由政規解放而為道範,以闢精神領域,以開價值、理想之源,以立人道之尊,以期重開文運與史運者也。若如彼所言,則不但曾子、子思、孟子、《易傳》已失古人體統(非正統),即孔子亦非古人體統也。如此愚悍狂言,何足語於論學論政哉?

正宗儒家本孔子「踐仁以知天」自德以言性,此是儒家之所以為理想主義之特別凸出者。外此,自生以言性者,則有道家、告子、荀子、世碩、公孫尼子,下及兩漢董仲舒、王充之言氣性,以及劉劭之言才性,此亦源遠流長之傳統,而統攝於宋儒所言之氣質之性,此為踐仁知天中消極面之性論,亦皆由古統中通過孔子所開出之傳統而見其意義之切者也。性與天道在孔子可以暫時撇開而不

\newpage\thispagestyle{empty}\addtocounter{page}{-1}\vspace*{-12mm}\begin{center}\noindent
\includegraphics[clip, trim=173pt 155pt 137pt 241pt, height=162mm]{ocr-input/image-0977.png}\end{center}

\newpage

\noindent 言,而經孔子之重建本統後,通過孔子而進言之,則不復是對存有之智測,(「自生以言性」之性亦是「存有」,此可曰氣之存有),而是統於踐仁知天中,而為道德實踐之圓滿所必應道德地察及之者。凡是存有,無論超越的存有,或是形氣的存有,客觀智測言之,皆是冥惑不定者,而統於踐仁知天中,則其奧密不復是智測中之冥惑,乃是在踐仁盡心中而朗現,此乃是形成道德之莊嚴與嚴肅乃至敬畏所必應具備者。陋儒膠著於事功,急切於實用,而必據孔子與子貢以為拒斥言性命天道者之口實,徒見其無真切之道德意識而已矣。

\newpage\thispagestyle{empty}\addtocounter{page}{-1}\vspace*{-12mm}\begin{center}\noindent
\includegraphics[clip, trim=165pt 406pt 124pt 97pt, height=162mm]{ocr-input/image-0981.png}\end{center}

\newpage\markright{}

\chapter{對於葉水心〈總述講學
大旨〉之衡定}

\section*{引言}\addcontentsline{toc}{section}{引言}

葉水心不滿曾子、子思、孟子、《中庸》、《易傳》以及北宋諸儒所弘揚之「性理」,而另開講學之大旨,以期有合於二帝三王之「本統」。然而不解孔子對於道之本統再建之意義,孔子之傳統全被抹殺,是則其歸也終於成為隔絕論與冥惑論。故真正輕忽孔子而與孔子傳統為敵者葉水心也。彼繼永嘉功利之說推進一步,以三代本統為原則性之劃定,彼自不無所見。其如此劃定以為講學之標的,其心態特別凸出,由此心態所決定之觀念形態亦是一特別之形態,吾名之曰「皇極一元論」。其論皇極亦甚好,然而由此而抹殺孔子之傳統,成爲隔絕論與冥惑論,則其愚悍狂悖亦云極矣。

薛士龍(艮齋)「學主禮樂制度,以求見之事功。」(《宋元學案》卷五十二(艮齋學案〉「敘錄」語)。發揮此義亦不錯。永康陳同甫言事功,則重英雄之生命,此義亦有其中肯處,彰顯而著之,亦有其好處。然皆只說一義,未曾抹殺孔子之傳統。惟水心繼承「永嘉以經制言事功」,(《宋元學案》卷五十六,〈龍川學

\newpage\thispagestyle{empty}\addtocounter{page}{-1}\vspace*{-12mm}\begin{center}\noindent
\includegraphics[clip, trim=152pt 144pt 142pt 248pt, height=162mm]{ocr-input/image-0985.png}\end{center}

\newpage

\noindent 案〉「叙錄」語)推進一步,以三代本統為標的,以「古人體統」壓孔氏,並曾子孟子、《中庸》、《易傳》而抹殺,另開講學宗旨,而成為皇極一元論,則其原則益透,形態益定,而其過亦益顯而愈甚。吾讀其書極不懌,然而極力忍耐,平其心、靜其氣,鄭重認識其所見究為何,關其謬而彰其是,重新確定本統之發展,使人得有正確之認識,庶不陷於乖妄之論,以似之而非者為真也。茲衡定其〈總述講學大旨〉如下:

\section{堯與舜:以器知天與人心道心}

\begin{quotation}\kaishu 道始於堯:「欽明文思安安,允恭克讓。」命義、和「曆象
日月星辰,敬授人時。」

堯敬天至矣。曆而象之,使「人時」與天行不差。若夫以
術下神,而欲窮天道之所難知,則不許也。

次舜:「濬哲文明,溫恭允塞。」「在璿璣玉衡,以齊七
政。」

舜之知天,不過以器求之。日月五星齊,則天道合矣。

其微言曰:「人心惟危,道心惟微。惟精惟一,允執厥
中。」

人心至可見,執中至易知、至易行。不言性命。子思贊
舜,始有大知,執兩端用中之論。孟子尤多。皆推稱所
及,非本文也。\end{quotation}

\noindent 案:以上言堯及舜。正文括弧中語,見〈堯典〉、〈舜典〉及〈大

\newpage\thispagestyle{empty}\addtocounter{page}{-1}\vspace*{-12mm}\begin{center}\noindent
\includegraphics[clip, trim=171pt 122pt 127pt 257pt, height=162mm]{ocr-input/image-0989.png}\end{center}

\newpage\markright{第一部 \quad 第五章 \quad 對於葉水心〈總述講學大旨〉之衡定}

\noindent 禹謨〉。低一格者乃葉適自註文。

堯命羲、和「暦象日月星辰,敬授人時」,此「制曆明時」也。亦《周禮》言史官「正歲年以叙事」之意。羲、和即後來所謂天官或史官。舜「在璿璣玉衡,以齊七政」,七政即日月五星七者運行之事,與「曆象日月星辰」同也。此為原初之科學知識,吾名此為羲和傳統。此為初民政治措施中關聯著自然所應首先註意及者。此所謂「天」即自然之天象,非帝、天、天命中之「天」也。制曆明時,窺測自然,乃純為「敬授人時」之實用者,此為科學知識之開端。「堯之敬天」,「舜之知天」,乃政治家開物成務順羲和之官之「制曆明時」之旨趣而敬天、知天,非後來順超越意義之帝、天、天命、天道而敬天、畏天、知天、奉天,乃至法天也。乃葉適則云:「舜之知天,不過以器求之。日月五星齊,則天道合矣。」「以器求之」,乃窺測自然以成科學知識之事,儒者不非此義,然謂「不過以器求之」,此能盡古人言天之體統乎?言天道之本統只如此乎?「日月五星齊,則天道合矣。」此所謂「天道」乃日月五星運行之自然現象,非帝、天、天命、天道中之「天道」也。日月五星之運行乃實然而自然者,並無超越之意義,而後一組概念中之「天道」乃有超越意義與道德價值意義者。此豈可混而同之乎?於以見其並不能認識「古人體統」中道德宗教之莊嚴意識也。徒封囿於政治措施之即事達義,以器知天,而謂能盡古人言天之體統乎?看似平實,實乃器識之陋也。至若「以術下神,而欲窮天道之所難知」,則是後來術數家之所為,非純正儒者之所言也。然不言「以術下神」,非並「窮神知化」,「窮理盡性以至於命」,亦不許言也。亦非並「神之格思,不可度思,矧可射思」

\newpage\thispagestyle{empty}\addtocounter{page}{-1}\vspace*{-12mm}\begin{center}\noindent
\includegraphics[clip, trim=147pt 136pt 146pt 245pt, height=162mm]{ocr-input/image-0993.png}\end{center}

\newpage

\noindent 《詩·大雅·抑》篇之精誠以來神亦不許言也。孔子之敬天、畏天、知天,孟子之知天、事天非「以器知天」,亦非「以術下神」,乃正是由「踐仁盡性」而敬之、畏之、知之、事之也。葉水心能純以「以器求之」為標準而抹殺之乎?能謂凡非「以器求之」者,皆非古人之體統乎?皆非「道之本統」乎?孔、孟之由踐仁盡性以知天,《大傳》之言「窮神知化」,以及「窮理盡性以至於命」,皆是本三代聰明、勇知、敬德以及帝、天、天命兩組觀念而轉化出者,能盡排於「古人體統」、「道之本統」之外乎?是以其並曾子、孟子、《中庸》、《易傳》而菲薄之,實即並孔子而亦非之也。惟不便明言之耳。蓋如其所言,自孔子始,即已失古人體統矣。此愚悍為何如!

孔子之稱堯舜、禹,皆稱其至德。《論語·泰伯》第八,子曰「大哉堯之爲君也!巍巍乎唯天爲大,唯堯則之。蕩蕩乎民無能名焉,巍巍乎其有成功也,焕乎其有文章。」又曰:「巍巍乎舜禹之有天下也,而不與焉。」又曰:「禹,吾無間然矣。菲飲食,而致孝乎鬼神,惡衣服,而致美乎黻冕,卑宮室,而盡力乎溝洫。禹、吾無間然矣!」〈衛靈公〉第十五又曰:「無爲而治者,其舜也與?夫何為哉?恭己正南面而已矣!」〈堯典〉稱堯曰:「欽明文思安安,允恭克讓。光被四表,格於上下。克明俊德,以親九族。九族睦,平章百姓。百姓昭明,協和萬邦。黎民於變時雍。」〈舜典〉稱舜曰:「濬哲文明,溫恭允塞。玄德升聞,乃命以位。慎徽五典,五典克從。納於百揆,百揆時敘。賓於四門,四門穆穆。納於大麓,烈風雷雨弗迷。」此皆言其能明德以致治,此足示後人追述古人人格之道德心靈之嚮往以及道之本統中心之所

\newpage\thispagestyle{empty}\addtocounter{page}{-1}\vspace*{-12mm}\begin{center}\noindent
\includegraphics[clip, trim=170pt 120pt 122pt 255pt, height=162mm]{ocr-input/image-0997.png}\end{center}

\newpage\markright{第一部 \quad 第五章 \quad 對於葉水心〈總述講學大旨〉之衡定}

\noindent 在,而葉水心則引而置之,不復贊一辭,獨於「曆象日月星辰,敬授人時」,「在璿璣玉衡,以齊七政」,特著而明之,謂其知天「不過以器求之」,以為古人體統不過「即事達義」,「以器明道」,獨以羲、和傳統為中心,不以堯、舜之德為中心,可謂忽其本而著其末,正是不明道之本統為何物者也。(後來通過孔子後,亦未有離事言義,離器明道者,然此即事即器,乃本乎超越者圓融而言之,非葉水心之只現象地外在地平面地言之也。此不可不辨。魚目混珠,遂藉以為拒談性命天道之口實矣)。

至人心道心四句乃古文《尚書·大禹謨》舜命禹之語。此觀念當屬後起。《論語·堯曰》第二十載:「堯曰:咨爾舜,天之歷數在爾躬,允執其中。四海困窮,天祿永終。舜亦以命禹。」此只言堯舜禹以「允執其中」一語相授受,未言人心道心也。輯〈大禹謨〉者大抵根據此語,補之以人心道心,擴充而為四句。宋儒認為此是二帝三王「相傳之心法」;考據家以為此是偽古文,不足信。

《荀子·解蔽篇》云:

\begin{quotation}\kaishu 昔者舜之治天下也,不以事詔而萬物成。處一危之,其榮滿
側。養一之微,榮矣而未知。故《道經》曰:人心之危,道
心之微。危微之幾,惟明君子而後能知之。\end{quotation}

阮元曰:

\begin{quotation}\kaishu 案:後人在《尚書》內解此者姑弗論,今但就《苟子》言
《荀子》,其意則曰:舜身行人事而處以專一,且時加以戒\end{quotation}

\newpage\thispagestyle{empty}\addtocounter{page}{-1}\vspace*{-12mm}\begin{center}\noindent
\includegraphics[clip, trim=164pt 144pt 132pt 245pt, height=162mm]{ocr-input/image-1001.png}\end{center}

\newpage

\begin{quotation}\kaishu 懼之心,所謂危之也。惟其危之,所以滿側皆獲安榮。此人
所知也。舜心見道而養以專一,在於幾微。其心安榮,則他
人未知也。如此解之,則引《道經》及明君子二句與前后各
節皆相通矣。楊注謂危之當作之危,非也。危之者,懼蔽於
欲而慮危也。之危者,已蔽於欲而陷危也。謂榮爲安榮者,
〈儒效篇〉曰:爲君子則常安榮矣。爲小人則常危辱矣。凡
人莫不欲安榮而惡危辱。據此,則荀子常以安榮與危辱相對
而言。此篇言處一危之,其榮滿側。若不以本書證之,則危
榮二字難得其解矣。故解《道經》當以《荀子》此說為正,
非所論於古文《尚書》也。(王先謙《荀子集解〉引)\end{quotation}

\noindent 案:此解《荀子》不誤。《荀子》引《道經》「人心之危,道心之微」之語,楊註云:「《道經》,蓋有道之經也」。郝懿行曰:「《道經》,蓋古言道之書。」(《荀子集解》引)。古文《尚書·大禹謨》可能根據《荀子》所引之《道經》語,將《論語》「允執其中」一語擴充為四句。此觀念自屬後起,蓋非有道德自覺真作道德修養工夫者,不能有此人心道心危微精一之別。堯命舜「允執其中」是指行事言。《中庸》引子曰:「舜其大知也與!舜好問而好察邇言,隱惡而揚善。執其兩端,用其中於民。其斯以爲舜乎?」亦是就行事言,而《道經》之語則直就心上作工夫,此非有真實而嚴肅之道德自覺者不能也。此義推之於二帝三王,固是過早,然確是儒家義則無疑。古文《尚書》雖可謂偽造,然其輯錄之語固有據,於義理亦不乖也。宋儒重視此語,不在古文《尚書》之偽不偽,而在其道德自覺上義理之精當。二帝三王之自政治措施上

\newpage\thispagestyle{empty}\addtocounter{page}{-1}\vspace*{-12mm}\begin{center}\noindent
\includegraphics[clip, trim=158pt 121pt 132pt 254pt, height=162mm]{ocr-input/image-1005.png}\end{center}

\newpage\markright{第一部 \quad 第五章 \quad 對於葉水心〈總述講學大旨〉之衡定}

\noindent 言「中」,固尚不能進至此。

葉水心當時尚未發現古文《尚書》之為偽,然謂「人心至可見,執中至易知、至易行,不言性命」,則輕而率矣。夫人心道心以及危微之別,真「至可見」乎?何言之易耶?「執中」真「至易知、至易行」乎?何言之易耶?就心上危微之幾作精一工夫,已是不易,故《荀子》謂「危微之幾,惟明君子而後能知之。」而葉水心則謂「人心至可見」,其無真切之道德自覺可知。本精一工夫以通體達用而「用其中於民」,則尤不易,而葉水心則謂「執中至易知、至易行」,其輕率淺躁可知。《中庸》引子曰「舜其大智也與!〔……】執其兩端,用其中於民」,此猶只是就舜之德行而贊之,未推進至自心上而言之,此猶近古,而葉水心則謂「子思贊舜,始有大知,執兩端用中之論,孟子尤多。皆推稱所及,非本文也」。《論語·堯曰》篇明言「允執其中」,《中庸》就之而言「執兩用中」,贊之為「大智」。只是一「執中」,何以在古文《尚書》便為「本文」,在《中庸》便為「推稱所及」?豈因汝之言「至易知、至易行」,《中庸》贊「大智」,便為「推稱所及,非本文也」?其淺躁無思理有如此!不知《論語》之語,《中庸》之贊,始是近古之「本文」,而自人心道心危微精一之辨以言「允執厥中」,才是「推稱所及」。不但是「推稱所及」,而且是推進一步,將由行事而言之者,內在化深刻化就心而言之。此顯是後來之發展。葉氏之言,可謂「本、推」倒置矣。徒因不慊於《中庸》,故到處譏議之,而又無嚴正之道德意識,故妄解人心道心也。

人心道心之辨,危微之幾,精一之工,正是後來自道德實踐上

\newpage\thispagestyle{empty}\addtocounter{page}{-1}\vspace*{-12mm}\begin{center}\noindent
\includegraphics[clip, trim=170pt 149pt 130pt 241pt, height=162mm]{ocr-input/image-1009.png}\end{center}

\newpage

\noindent 言性命天道者,始能正視此心上之工夫。而葉水心則深厭性命之談,故輕率淺躁,而言「人心至可見,執中至易知、至易行,不言性命。」蓋以為「性命」是渺茫冥惑之事,故只言「人心、執中」之顯明「可見而易知易行」者,不言幽渺難測之「性命」也。此其輕率愚妄為何如!

\section{禹與皋陶:〈皋陶謨〉之天敘天秩、
天命、天討、天聰明、天明畏}

\begin{quotation}\kaishu 次禹:「后克艱厥后,臣克艱厥臣」。「惠迪、吉,從逆、
凶。惟影響」。〔引號中者,皆(大禹謨〉中語。】

〔注文言〈洪範〉,不相干,略】

次皋陶:訓人德以補天德,觀天道以開人治。能教天下之多
材,自皋陶始。

禹以才難得、人難知為憂。皋陶言「亦行有九德,亦言其
人有德」。卿大夫諸侯皆有可任。「翕受敷施,九德咸
事」,以人代天。〔案:〈臯陶謨〉原文:「天工人其代
之」。]

典禮賞罰,本諸天意。禹相與共行之,夏商周一遵之。\end{quotation}

\noindent 案:此言禹及臯陶無問題,而言後者尤好。「訓人德以補天德,觀天道以開人治」,皆極佳語,注文皆綜括〈臯陶謨〉(今文)而成。「典禮賞罰,本諸天意」兩語即綜括「天敘有典,勅我五典五

\newpage\thispagestyle{empty}\addtocounter{page}{-1}\vspace*{-12mm}\begin{center}\noindent
\includegraphics[clip, trim=158pt 128pt 131pt 252pt, height=162mm]{ocr-input/image-1013.png}\end{center}

\newpage\markright{第一部 \quad 第五章 \quad 對於葉水心 \quad 〈總述講學大旨〉之衡定}

\noindent 惇哉。天秩有禮,自我五禮有庸哉。同寅協恭和衷哉。天命有德,五服五章哉。天討有罪,五刑五用哉。」而成。程明道喜就此而言「天理」。宋明儒所言之道德性之性理皆不外就此而體證之。〈皐陶謨〉確能表現超越的道德意識之莊嚴。復有「天聰明,自我民聰明。天明畏,自我民明威〔畏〕。達於上下,敬哉有土。」此亦自政治上言天人之相感應也。此中天叙、天秩、天命、天討、天聰明、天明畏,亦能「以器求之」乎?天叙雖不離五典,天秩雖不離五禮,天命雖不離五服五章,天討雖不離五刑五用,天聰明雖不離我民聰明,天明畏雖不離我民明畏,此亦可說「即事達義」,然若無道德上真實感與超越感,亦不能真切乎此義。故此種「即事達義」非現象主義之「即事達義」也。宋、明儒所言之道德性之實理、天理、性理,乃至性命天道,亦不過就此擴大而肯證之。此種性命天道有何冥惑可言,而必隔絕而非之?於以見葉氏所謂「訓人德以補天德,觀天道以開人治」,亦是浮言,其對於天德、天道,並無實感。非然者,何至對於曾子、孟子、《中庸》、《易傳》乃至周、張、二程如此之深閉而固拒之哉?豈只准限於帝王之措施,而不准孔、孟傳統之自覺地言之以開價值創造之源乎?

\section{湯與伊尹:〈湯誥〉「降衷、恆性」與
《中庸》「天命之謂性」}

\begin{quotation}\kaishu 次湯:「惟皇上帝,降衷於下民,若有恆性。克綏厥猷惟
后」。其言性蓋如此。\end{quotation}

\newpage\thispagestyle{empty}\addtocounter{page}{-1}\vspace*{-12mm}\begin{center}\noindent
\includegraphics[clip, trim=155pt 143pt 153pt 254pt, height=162mm]{ocr-input/image-1017.png}\end{center}

\newpage

\begin{quotation}\kaishu 次伊尹:言「德惟一」,又曰「終始惟一」,又曰「善無常
主,協於克一」。〔皆古文〈咸有一德〉中語。】

嗚呼!堯、舜、禹皋陶、湯伊尹,於道德性命,天人
之交,君臣民庶,均有之矣。\end{quotation}

\noindent 案:此暫綜結。謂「均有之」,固不錯,然不能封於此原始之綜和型態也,茲就(湯誥〉(古文)之語以明葉適誹議《中庸》之謬。關此四語,吾已詳解之於前章第一節。葉氏於其《習學記言》中有一條即本此湯誥之文以議《中庸》為不諦。可見其對於《中庸》之深惡也。

《習學記言》曰:

\begin{quotation}\kaishu 《書》稱「惟皇上帝,降衷於下民」,即天命之謂性也。然
可以言降衷,不可以言天命。蓋物與人生於天地之間,同謂
之命。若降衷,則人固獨得之矣。降命而人獨受,則遺物。
若與物同受命,則物何以不能率,而人能率之哉?\end{quotation}

\noindent 案:此皆妄加分別,不解《中庸》,亦不解(湯誥〉。〈湯誥)「惟皇上帝,降衷於下民,若有恆性」,此三句(實兩整句)合起來可相當於「天命之謂性」,而乃順《孔傳》之讀,只言「降衷」為「天命之謂性」,而將「若有恆性」比配「率性之謂道」,(見下),此皆誤也。「降衷、恆性」,若解為超越的義理當然之性,「降衷」即「天命」,意旨實同,而乃妄肆分別,謂只可言降衷,不可言天命,此猶知二五不知十也。〈湯誥〉「降衷」但就人而言

\newpage\thispagestyle{empty}\addtocounter{page}{-1}\vspace*{-12mm}\begin{center}\noindent
\includegraphics[clip, trim=175pt 123pt 119pt 257pt, height=162mm]{ocr-input/image-1021.png}\end{center}

\newpage\markright{第一部 \quad 第五章 \quad 對於葉水心〈總述講學大旨〉之衡定}

\noindent 之,固也,《中庸》言天命可普就人物而總言之,亦固也。然《中庸》之普言「天命」,乃承《詩·大雅》「維天之命,於穆不已」之將天命轉為流行之命,所謂「天命流行之體」,吾所謂「形而上之實體」者,而言之。其流行於人而命於人,而人能受之,即為人之「性」,是即為「天命之謂性」。其流行於物而命於物,而物不能受之而為性,則於此只好說「天命之謂在」。「同謂之命」,即程、朱所謂「同體」也。雖同體而有人禽之辨,則人之所以異於禽獸者乃在心之自覺。明道謂「萬物皆備於我,不獨人耳,物皆然。都自這裡出去。只是物不能推,人則能推之。」能推不能推之關鍵即在「心」。能推,則可以將天所命者吸納於自己之生命中而為自己之呈現而定然之性,不能推,則天外在而超越地命之,而卻不能吸納此天命流行之體於其自己之生命中而為其呈現而定然之性,是則天命只是超越地為其體,而不能內在地復為其性,此亦即唯人「獨得之」也。然則言「天命之謂性」有何不可乎?「天命」非即「降衷」乎?人物之別不以降衷與天命判,而以能不能推判。人能推而能實有此天命流行之體以為己性,則就其為性之為先天而定然的言,亦即等於天所命也。是則天命之流行於人而命於人不獨命人之存在,而且命以超越的義理當然之性也。物不能推,則物即不能實有天所命者以為己性,結果物只有物質結構之性、墮性,或本能之性,而不能有「道德的創造性」之性,是則就此性言,物不能吸納而實有之,天對之亦即無所命也。然其個體之存在仍是天命流行之體之所實現(生化),此亦是「天命」也。此則天只命其有個體之存在,而不能命其有「道德的創造性」之性也。(不能命是因物不能推而不能命。)就物言,只能說「天命之謂在」,或說氣化

\newpage\thispagestyle{empty}\addtocounter{page}{-1}\vspace*{-12mm}\begin{center}\noindent
\includegraphics[clip, trim=153pt 142pt 147pt 249pt, height=162mm]{ocr-input/image-1025.png}\end{center}

\newpage

\noindent 之謂物質結構之性,或墮性,或本能之性。朱子亦言在人為性,在物為理,皆由天命而來也。然言性與只言理,確有不同。明道言「物皆然」,是形而上地就「同體」而言,亦「物物一太極」之意。然因物不能推,無心之自覺,不能實有此創造性之性,則所謂「物皆然」只是潛存地函備一切理,而並不能實現之以為己性乃至盡性以有此性體之創造也。故其言「萬物皆備於我,不獨人耳,物皆然」,亦只是一種靜觀之境,乃極端道德的理想主義之言,亦為《中庸》、《易傳》之所函,惟宋儒能自然而順適地引發之而已耳。葉水心輕浮淺躁,不能察此中之蘊,而妄據〈湯誥〉之文疑《中庸》,其於孔子以後之學全無所知亦明矣。若能真知「降衷」之性之為超越的義理當然之性,必不發此妄議也。《中庸》言天命雖可普就人物而總言之,然天命豈只是命人與物之「生」或「存在」耶?就人而言「天命之謂性」,則不但命人之生,亦命其性,是「天命」即「降衷」也,而人亦獨得之矣。就物而言天命,因物不能推,則天只命其生或存在,而不命其義理當然之性,是則只成「天命之謂在」,而人與物亦區以別矣。是則言「天命」可言同體,亦可言差別。同體,則不「遺物」;差別,則人「獨得」。焉有如葉水心之疑問耶?

《習學記言》繼上又云:

\begin{quotation}\kaishu 《書》又稱「若有恆性」,即率性之謂道也。然可以言若有
恒性,而不可以言率性。蓋已受其衷矣,故能得其當然者。
若人而有恆,則可以為性。若止受於命,不可知其當然也。
而以意之所謂當然者率之,則道離於性而非率也。\end{quotation}

\newpage\thispagestyle{empty}\addtocounter{page}{-1}\vspace*{-12mm}\begin{center}\noindent
\includegraphics[clip, trim=166pt 123pt 125pt 254pt, height=162mm]{ocr-input/image-1029.png}\end{center}

\newpage\markright{第一部 \quad 第五章 \quad 對於葉水心〈總述講學大旨〉之衡定}

\noindent 案:此議尤其語無倫次,莫知所云。蓋對於「若有恆性」視為獨立之一句,以比配「率性之謂道」,全不通也。「若人而有恆,則可以為性」,此能落實於「若有恆性」之語句上而視為此句之解詁乎?「降衷」之爲性,豈待人之有恆無恆而始「可以為性」乎?其不成辭意有如此!而下文復將此句隨意解為「若其恆性」,「有」字復又變爲「其」矣。蓋順《孔傳》之訓「若」為「順」而言也,不知《孔傳》對此句已誤讀誤解,而又順之復為此不通之論也。若知「若有恆性」乃上承「降衷」句而為一氣,只可比配「天命之謂性」,〈湯誥〉並無「率性之謂道」之一義,則焉有此不通之解乎?「恆性」即常性,義同「秉彝」,非人之有恆無恆之恆也。此尚是語句之問題。至若「率性」正是承天所命之義理當然者而率之,何言「若止受於命,不可知其當然」耶?汝以為此「性」是何性耶?豈是空白或無色無記中性之性乎?汝以為天之命是何命耶?豈是空白之命乎?若然,則不得就天命而言性矣。天命人以義理當然之性,故率之即為道也。焉有所謂「以意之所謂當然者率之,則道離於性而非率也」之謂乎?

又云:

\begin{quotation}\kaishu 《書》又稱「克綏厥猷惟后」,即修道之謂教也。然可以言
綏,而不可以言修。蓋民若其恆性,而君能綏之,無加損焉
爾。修則有所損益,而道非其真,則教者強民以從己矣。\end{quotation}

\noindent 案:此又將綏與率性合而為一,此非《孔傳》之意,亦非「修道之謂教」之意。蓋如果「民若其恆性」,即是「率性」之意,則何

\newpage\thispagestyle{empty}\addtocounter{page}{-1}\vspace*{-12mm}\begin{center}\noindent
\includegraphics[clip, trim=181pt 145pt 118pt 241pt, height=162mm]{ocr-input/image-1033.png}\end{center}

\newpage

\noindent 須君之綏耶?《孔傳》是說「順人之常性而能安立其道教,則惟為君之道。」此雖解「若有恆性」為非是,然若將「順人之常性」視為補充句,而不視為「若有恆性」之解詁句,(參看上章第一節),則「能安立其道教」句猶可比配「修道之謂教」。此是說君「順人之常性而能安立其道教」,不是「民若〔順】其恆性,而君能綏之」也。「率性之謂道」是一義,「修道之謂教」是另一義。前者是個人之事,後者是政教之事。焉可混而一之?「修道之謂教」者,是將道修之於家國天下而成教也。此是就社會生活客觀關係而言之,故須有賴於政治(君后)之綏寧。正因道不離性,故雖修之於家國天下之社會生活之間、客觀關係之中,亦不能背乎人情之自然與性理之當然,焉有所謂「修則有所損益,而道非其真,則教者強民以從己矣」之病乎?今之極權專制者正是不肯定人之義理當然之性,而以外在之主義牽率天下強民以從已也。焉有順「天命之謂性,率性之謂道」而言「修道之謂教」者而有此病乎?葉水心能注意「強民從己」之為害,此可取處。(彼言政治、言皇極,皆極好。)惜其不解《中庸》,而又不知「率性」與「修道」之為兩事也。社會生活之間,客觀關係之中,不皆能合理也。否則當無社會問題矣。然則以由率性而見之道為標準,就現實之衝突偏激處,調整而損益之,提撕而綏寧之,使其順理而合道,此種「損益」有何不同,焉有所謂「修則有損益,道非其真」之說乎?天命之性不能有損益,故只可言「率」,而不可言「修」。只聞有「修身」,未聞有「修性」者。「修身」者以道修持自己也。但「修道」卻不是對於道本身有所修,乃是將道修之於家國天下,亦即以道修客觀之事也。是則所損益者是事,不是道。修事、修身,從「所」言;

\newpage\thispagestyle{empty}\addtocounter{page}{-1}\vspace*{-12mm}\begin{center}\noindent
\includegraphics[clip, trim=154pt 128pt 147pt 254pt, height=162mm]{ocr-input/image-1037.png}\end{center}

\newpage\markright{第一部 \quad 第五章 \quad 對於葉水心〈總述講學大旨〉之衡定}

\noindent 修道從「能」言,即以能修之道修之於事也。以道修身靠自己,修道成教(以道修客觀社會之事)賴政治。焉可詞意不分,而妄意「修道」為非耶?

\section{文王與「無聲無臭」}

\begin{quotation}\kaishu 次文王:「肆戎疾不殄,烈假不瑕。不聞亦式,不諫亦
入。」「難離在宮,肅肅在朝。不顯亦臨,無射亦保。」
〔案:以上為《詩・大雅·思齊》章語。】「無然畔援,無
然歆羨,誕先登於岸。」「不大聲以色,不長夏以革。不識
不知,順帝之則。」〔案:以上為《詩・大雅・皇矣》章
語。〕文王備道盡理如此。豈特文王為然哉?固所以成天下
之材,而使皆有以充乎性全乎命也。

案:《中庸》言:「鳶飛戾天,魚躍於淵,言其上下察
也。」「德輶如毛,毛猶有倫。上天之載,無聲無臭,至
矣。」夫鳥至於高,魚趨於深,言文王作人之功也。「德
輶如毛」,舉輕以明重也。「上天之載,無聲無臭」,言
天不可即,而文王可象也。古人患乎道德之難知而難求
也,故自「允恭克讓」,以至「主善協一」,皆盡己而
非無所察於物也,皆有倫而非無聲臭也。今傾倒文義,指
其至妙以示人。後世冥惑於性命之理,蓋自是始。不可謂
文王之道固然也。\end{quotation}

\noindent 案:葉適之誹議《中庸》,正為其「指至妙以示人」,而不知《中

\newpage\thispagestyle{empty}\addtocounter{page}{-1}\vspace*{-12mm}\begin{center}\noindent
\includegraphics[clip, trim=178pt 147pt 128pt 249pt, height=162mm]{ocr-input/image-1041.png}\end{center}

\newpage

\noindent 庸〉此處之引「無聲無臭」正喻進德之默成,並未指天道之至妙者以示人,並未就天道本身以言其至妙。其就鳶飛魚躍以言上下察是喻進德之造端與極至,亦無所謂「指其至妙以示人」。《中庸》之言天道至妙者多矣,而此處言無聲無臭,則非言天道之至妙。今謂其「顛倒文義,指其至妙以示人」,則誣妄之言也。試詳解《中庸》所引詩如下。

《詩·大雅·旱麗》:「鳶飛戾天,魚躍於淵。豈弟君子,遐不作人?」朱注:「興也」。又引:「李氏曰,《抱朴子》曰:『鳶之在下無力,及至乎上,聳身直翅而已。』蓋鳶之飛全不用力,亦如魚躍,怡然自得,而不知其所以然也。」鳶之飛、魚之躍,而不知其所以然,實皆出於性之自然而然也,由此以興起人之進德修業以「作人」亦當是性之自然而不容已。然則「豈弟君子,而何不作人乎?言其必作人也。」(朱注語)。此為此興體詩之本意。《中庸》引此「興」語而言「言其上下察也」,固與原意有出入,然由之以言「君子之道,造端乎夫婦,及其至也,察乎天地」,亦是表示人之進德之「造端」與「其至」,而無論是其造端或其極至,亦皆出於性之不容已,一如鳶飛之至於高,魚躍之「出於淵」(朱注語,或如葉適言「趨於深」亦得),皆其性之自然也。(中庸》不過將此「戾天」之高與趨淵之深,綜和而為「上下察」之一義,以明「造端」與「其至」之終始而已。此有何一定不可乎?詩之興只言「作人」,而《中庸》藉之以興喻「造端」與「極至」,皆言進德作人,興喻雖有出入,而進德之實則同,此足以成為古人引《詩》之過患乎?古人引《詩》大抵皆引之以喻己意,不必盡同於原意也。夫人之進德,其「造端」雖極淺近,「夫

\newpage\thispagestyle{empty}\addtocounter{page}{-1}\vspace*{-12mm}\begin{center}\noindent
\includegraphics[clip, trim=143pt 122pt 143pt 259pt, height=162mm]{ocr-input/image-1045.png}\end{center}

\newpage\markright{第一部 \quad 第五章 \quad 對於葉水心 \quad 〈總述講學大旨〉之衡定}

\noindent 婦之愚可以與知」,「可以能行」,然人之進德寧有止境乎?「及其至也,雖聖人亦有所不知焉」,「亦有所不能焉」。蓋道無所不在,始於日用,「察乎天地」。「察,至也」(《廣雅》)。朱子則言「著也」,亦通。人之進德造道必以至乎天地與天地合德為極至。此乃通過孔子之仁教後,所必至之理境,而亦不悖於《詩・周頌》「維天之命,於穆不已。於乎不顯,文王之德之純」之超越的洞見也。葉適淺陋,而必於此致其不滿,此豈非成見作祟乎?葉適謂「夫鳥至於高,魚趨於深,言文王作人之功也。」此固不錯。然《中庸》就之言上下察以喻進德之造端與極至。又豈非君子作人之實功乎?且亦顯出道之廣大(費)與深微(隱)。此有何「冥惑」之可言?「後世」程子曰:「此一節【君子之道費而隱】,子思吃緊為人處,活潑潑地,讀者其致思焉。」(朱註引)。又「〔尹和靖〕先生嘗問伊川:鳶飛戾天,魚躍於淵,莫是上下一理否?伊川曰:到這裡只得點頭」。(《二程全書·外書第十二》)。明道就此而言「活潑潑地」,伊川首肯「上下一理」之義,皆是就《中庸》「君子之道,費而隱」一章說。此固比(旱麓〉詩推進一步,然由造端與極至道德踐履之實功以實之,又有何冥惑之可言?只因葉適無道德踐履之實功,無洞明之心胸,故一見「費而隱」便頭腦昏漲,遂並《中庸》言「上下察」亦譏議之。此亦自己之冥惑而已矣。

《詩·大雅·烝民》:「人亦有言,德輶如毛,民鮮克舉之。我儀圖之,唯仲山甫舉之。愛莫助之。」朱注:「言人皆言德甚輕而易舉,然人莫能舉也。我於是謀度其能舉之者,則惟仲山甫而已。是以心誠愛之,而恨其不能有以助之。蓋愛之者,秉彝,好德之性

\newpage\thispagestyle{empty}\addtocounter{page}{-1}\vspace*{-12mm}\begin{center}\noindent
\includegraphics[clip, trim=173pt 156pt 126pt 237pt, height=162mm]{ocr-input/image-1049.png}\end{center}

\newpage

\noindent 也。而不能助者,能舉與否,在彼而已,固無待於人之助,而亦非人之所能助也。」此為言「德輔如毛」之原意。《中庸》末章末段:「詩曰:『予懷明德,不大聲以色。』〔《詩·大雅・皇矣》】。子曰:「聲色之於以化民,末也』。《詩》曰:『德輶如毛。』毛猶有倫。「上天之載,無聲無臭。』至矣。」此末段三引《詩》以作結。「上天之載,無聲無臭。儀刑文王,萬邦作孚。」此是《詩·大雅·文王》篇之句。《中庸》之所以三引《詩》句作結乃是因為此末章是從「君子之道闇然而日章」說起。繼之即言「君子之所不可及者,其唯人之所不見乎?」此仍是慎獨之旨。又言:「故君子不動而敬,不言而信。」又言:「是故君子不賞而民勸,不怒而民威於鈇鉞。」又言:「是故君子篤恭而天下平。」每說一義,皆引《詩》句作證,皆表示「默默修德,不露形跡」之義。此是言個人自己作修養工夫之最真切最平實處。故此末段即首引《詩·大雅·皇矣》篇「予懷明德,不大聲以色」之句以證之。復引子曰:「聲色之於以化民末也」以明之。此亦是《易傳》「默而成之,不言而信,存乎德行」之意。德行而能自己宣揚自我表白乎?個人進德是在無聲無臭中進,而其感應亦在無聲無臭中潛移默化。此即是「君子之道閣然而日章」,「不動而敬,不言而信」,「不賞而民勸,不怒而民威於鈇鉞」,「篤恭而天下平」之意。正合「穆穆文王,於緝煕敬止」,「不大聲以色,不長夏以革,不識不知,順帝之則」之意,蓋文王正是此所謂篤恭君子之典型也。此引「上天之載無聲無臭」正在明君子進德之無聲臭。「上天之載,無聲無臭」亦猶孔子曰:「天何言哉?四時行焉,百物生焉,天何言哉?」此有何「顛倒文義,指其至妙以示人」之可言?又有何「冥

\newpage\thispagestyle{empty}\addtocounter{page}{-1}\vspace*{-12mm}\begin{center}\noindent
\includegraphics[clip, trim=152pt 134pt 145pt 250pt, height=162mm]{ocr-input/image-1053.png}\end{center}

\newpage\markright{第一部 \quad 第五章 \quad 對於葉水心 \quad 〈總述講學大旨〉之衡定}

\noindent 惑」之可言?至於〈烝民〉詩「德輜如毛」之句,是以毛比德之甚輕而易舉,而人終鮮能舉之。《中庸》則引之進一步以明進德之無聲臭。故曰:「毛猶有倫。」蓋毫毛雖甚輕,猶是有形跡之物,猶可倫比,至於就個人進德言,則根本是「默成」之事,根本不應有任何形迹聲色之表露,只應慎獨以潛修,自能「閣然而日章。」故以「上天之載無聲無臭」證之也。《詩》與《中庸》各喻一義。《詩》之言「德輜如毛」,是直接喻德之輕而易舉,而人卒莫能舉。亦猶孔子言「仁遠乎哉?我欲仁,斯仁至矣。」此輕而易舉也。然孔子亦曰:「民之於仁也,甚於水火。水火,吾見蹈而死者矣。未見蹈仁而死者也。」民之於仁如此之甚,然而人卻不肯爲也!又曰:「我未見好仁者,惡不仁者。」又曰:「有能一日用其力於仁矣乎?我未見力不足者。蓋有之矣,我未之見也。」此皆亦勉亦勸,而亦表示「民鮮克舉之」之意也,而《中庸》之引「德輔如毛」,而言「毛猶有倫」,則是進一步喻君子進德之默成。各喻一義,有何「顛倒文義」之可言?進德默成,是最真實最切實者,又有何「指其至妙以示人」以「冥惑」後世乎?冥惑者自冥惑,葉水心之類是也。真實者自真實,蓋無人能否認此「默成」之義也。《中庸》言「默成」,而「穆穆文王」亦正是無聲無臭之默成。「文王之道」何以不可謂「固然」耶?葉水心既不理會「默成」之義,復不理會「德輜如毛,民鮮克舉之」之義,而卻說「古人患乎道德之難知而難求也,故自允恭克讓,以至主善協一,皆盡己而非無所察於物也,皆有倫而非無聲臭也。」此正是王顧左右而言他,說那不相干的事。此無當於「默成」,亦無當於「德輪如毛」。此才真誤引文義而自說其「即事達義」「以器明道」之一義。故重

\newpage\thispagestyle{empty}\addtocounter{page}{-1}\vspace*{-12mm}\begin{center}\noindent
\includegraphics[clip, trim=193pt 161pt 124pt 244pt, height=162mm]{ocr-input/image-1057.png}\end{center}

\newpage

\noindent 察物有倫。夫察物有倫是一義,而個人進德之默成無聲臭又是一義,此兩相礙乎?而必反對無聲臭何也?此正是自己之不解又誤引,而反責《中庸》乎?穆穆文王不正是無聲無臭,默成其德乎?然文王是一具體之生命,是一德性人格之存在,其本身進德雖默成,如上天之無聲無臭,然而卻可象可刑,而上天則根本無象可象,無形可型,故欲知天,則看人可也。如曰:「上天之載無聲無臭。儀刑文王,萬邦作孚。」此是通過有象者以證無象。此是此詩之原意也。《中庸》則引之以明個人進德之默成,此是說另一義。有何「顛倒文義」之可言?而此義亦正合文王之道,又有何「冥惑」之可言?此亦只是葉水心成見作怪而已矣!「後世」程明道言:「忠信所以進德,終日乾乾,君子當終日對越在天也。蓋上天之載無聲無臭。其體則謂之易,其理則謂之道,其用則謂之神。其命於人則謂之性。率性則謂之道。修道則謂之敎。孟子在其中又發揮出浩然之氣,可謂盡矣。故說神如在其上,如在其左右。大小疑事,而只是誠之不可掩。澈上澈下,不過如此。」(《二程全書·遺書第一》)。案:此是後世程明道本《中庸》、《孟子》、《易傳》而言天道性命之為一,此比《中庸》就無聲無臭言進德之默成進一步。然皆是實見實理,亦無冥惑之可言,冥惑者只是葉水心之不解耳。

\section{周公與原始的綜和構造}

\begin{quotation}\kaishu 次周公:治教並行,禮刑兼舉。百官眾有司雖名物卑瑣,而
道德義理皆具。自堯舜以來,聖賢繼作。措於事物,其該括\end{quotation}

\newpage\thispagestyle{empty}\addtocounter{page}{-1}\vspace*{-12mm}\begin{center}\noindent
\includegraphics[clip, trim=141pt 134pt 154pt 248pt, height=162mm]{ocr-input/image-1061.png}\end{center}

\newpage\markright{第一部 \quad 第五章 \quad 對於葉水心〈總述講學大旨〉之衡定}

\begin{quotation}\kaishu 演暢,皆不得如周公。不惟周公,而召公與焉。遂成一代之
治。道統歷然如貫聯,不可違越。\end{quotation}

\noindent 案:葉水心所謂「道統」即堯舜以來三代開物成務之原始綜和構造之過程也。此原始綜和構造之過程結集於周公。其形態為「治教並行,禮刑兼舉」。而「道德義理」即在「百官衆有司」之名物中。故云:「措於事物,其該括演暢,皆不得如周公。」此所謂「道德」即是表現於典章制度中之道德,所謂「義理」即是表現於名物度數中之義理。故著重於察物有倫,而張即事達義,即器明道,以明「內外交相成」,實即政治措施之綜和構造。其所謂道之本統即此開物成務之綜和構造也。此義固不錯,一切學問理想未有不期其向現實有所構造者,亦必終落於綜和構造而始得其真實與客觀化。然歷史是在發展中,綜和構造亦在歷史發展中為一期一期之形成,故綜和構造有其歷史階段中之形態。此是強度的、歷史的,非邏輯的(數學的)、永恆的也。如是數學的、永恆的,何不直線相續,永停於周公之明備,而復有春秋戰國之轉變,乃至秦、漢之新局耶?歷史既是如此矣,綜和構造是有其歷史階段中之形態矣,而如果複仍是停於此原始之綜和構造中,總是直接就此原始之綜和構造以明道統,以為只此才是「本統」,才是「古人體統」,而不准有任何相應歷史發展之開合,凡離此綜和構造而有所開合以關理想、價值之源,以期重開史運文運者,皆非道之本統,皆失古人之體統,如是,則即為現象主義之不見本源,落於皇極一元論之封閉隔絕而不自知,雖曰內外交相成,而實永不開眼者也。雖曰即事達義,即器明道,而實永粘著於名物度數而並不知何為義何為道者

\newpage\thispagestyle{empty}\addtocounter{page}{-1}\vspace*{-12mm}\begin{center}\noindent
\includegraphics[clip, trim=176pt 167pt 141pt 236pt, height=162mm]{ocr-input/image-1065.png}\end{center}

\newpage

\noindent 也。古人之原始生命往矣,原始之綜和構造不可復見,而若不闢理想、價值之源,重開文運與史運,則綜和構造不可再見,雖念念不忘即事達義、即器明道,而實百事無成、一器不備,徒騰口說而已。此所以言實用者終無用,重事功者總無功。墨子不及孔子之能繁興大用也。永嘉之經制事功不及程、朱、陸、王之更有事功也。乾嘉之考據更是無用而已矣。能成科學乎?能開物成務乎?樸學云乎哉?此是樸學、實學之名之偽似,真正之樸學實學不在此也。假名久矣,歷史照然不爽,而猶不自覺。葉水心之蔽正在停於原始之綜和構造而不知孔子之開合,落於皇極一元論,而不知孔子對於道之本統之再建。凡後來之言事功、言實用、言樸學,而斥宋明儒之談性命天道為無用者,皆不出葉水心之規模。吾故不厭辭費,詳闢葉水心之謬而祛其蔽也。善哉王弼之言曰:「凡物之所以存,乃反其形。功之所以克,乃反其名。」(《老子微旨例略》)。此智慧之言也,聖人者乃此大智慧之表現者也。凡永粘著于物之形、功之名,而不見「反其形、反其名」之道(本源)者,皆永無事功也。世之無真生命、真智慧、真學問,而空言事功以譯世取寵者,其念之哉!其戒之哉!

\section{孔子與仁教}

\begin{quotation}\kaishu 次孔子:周道既壞,上世所存皆放失。諸子辯士,人各為
家。孔子蒐補遺文墜典,《詩》、《書》、《禮》、
《樂》、《春秋》,有述無作,惟《易》著〈象〉、
〈象〉。\end{quotation}

\newpage\thispagestyle{empty}\addtocounter{page}{-1}\vspace*{-12mm}\begin{center}\noindent
\includegraphics[clip, trim=159pt 146pt 138pt 234pt, height=162mm]{ocr-input/image-1069.png}\end{center}

\newpage\markright{第一部 \quad 第五章 \quad 對於葉水心〈總述講學大旨〉之衡定}

\begin{quotation}\kaishu 舊傳刪《詩》定《書》作《春秋》,予考詳,始明其不
然。

然後唐虞三代之道,賴以有傳。

案:《論語》「子罕言利與命與仁」,而考孔子言仁多於
他語。豈有不獲聞者,故以為罕耶?\end{quotation}

\noindent 案:此言孔子。其於孔子之仁教全無所知甚顯。若如葉適所言,則孔子只是「蒐補遺文墜典」,使「唐虞三代之道,賴以有傳」,而其本身則對於道並無貢獻。對於《詩》、《書》、《禮》、《樂》、《春秋》,無論是刪、定、作,或只是蒐補,有述無作,皆不關重要。要者是在仁。仁是其真生命之所在,亦是其生命之大宗。不在其蒐補文獻也。有了仁,則其所述而不作者一起皆活,一切皆有意義,皆是真實生命之所流注。然則唐虞三代之制度之道與政規之道惟賴孔子之仁教始能成為活法,而亦惟賴孔子之仁教,始能見其可以下傳以及其下傳之意義。自其可以下傳言,是孔子之所以承繼唐虞三代之道德總規與政規者,自其下傳之有意義言,乃見其必有一開合以期新的綜和構造之再現,所謂重開文運與史運者。是則仁教者乃對於道之本統之重建以開創造之源者也。《詩》、《書》《禮》《樂》、《春秋》可以述而不作,而仁教則斷然是其創造生命之所在,此不可以通常著書立說之創造視之也。葉水心既不知仁教之意義,復只視孔子為蒐補遺文者,然則孔子之業一史官之檔案家足以優爲之,何必賴孔子而後傳耶?是則孔子一無所有,但何因而又被稱為大聖耶?不應孔子以後,二千年來,皆妄語妄稱也!於以見葉適對於孔子之仁教全無所解。是其外在之頭腦,

\newpage\thispagestyle{empty}\addtocounter{page}{-1}\vspace*{-12mm}\begin{center}\noindent
\includegraphics[clip, trim=185pt 178pt 146pt 233pt, height=162mm]{ocr-input/image-1073.png}\end{center}

\newpage

\noindent 只看王者業績之心靈,固只能成為皇極一元論,而不能知孔子仁教之意義以及其對於道之本統之再建之作用也。

命與仁明是孔子真精神之所在。「子罕言利,與命與仁」,明須利字點句,「與」字明非連結詞,乃是「與於」之與。以往雖大皆作連結詞講,然對於命與仁則必曲為之說。是則反不如直解作「與於」之與為直接而明顯也。葉水心對於孔子之仁教並無真解,故亦無多重視,只見出「孔子言仁多於他語」,而以「豈有不獲聞者,故以為罕耶」之不相干之疑問語句輕輕略過。仁為孔子真正生命之所在,如此顯明而重大之事,而猶心存「冥惑」,而不敢斷定,則其平素之不解孔子無視孔子也明矣,而亦甚矣。對于孔子如此,則其菲薄曾子、子思、孟子、《中庸》、《易傳》又何怪焉!

仁是全德,是真實生命,以感通為性,以潤物為用;它超越乎禮樂(典章制度、全部人文世界)而又內在於禮樂;在仁之通潤中,一一皆實。體現仁之最高境界是「欽思、文明、安安」,是天人不隔,是圓融無礙。孔子講仁是敞開了每一人光明其自己之門,是使每一人精進其德性生命為可能,是決定了人之精神生命之基本方向,是開關了理想、價值之源。是謂理想之「直、方、大」。

孔子自道曰:「若聖與仁,則吾豈敢?抑爲之不厭,誨人不倦,則可謂云爾已矣。」(〈述而〉第七)。

又曰:「默而識之,學而不厭,誨人不倦,何有於我哉?」(同上)。(「何有於我哉」與上文「則可謂云爾已矣」相呼應,言這於我沒有什麼,意函我能作到,亦意函我亦只是如此而已,然若能不厭不倦,即示其德性生命之精進,此亦幾近於仁聖,而仁聖

\newpage\thispagestyle{empty}\addtocounter{page}{-1}\vspace*{-12mm}\begin{center}\noindent
\includegraphics[clip, trim=162pt 149pt 143pt 235pt, height=162mm]{ocr-input/image-1077.png}\end{center}

\newpage\markright{第一部 \quad 第五章 \quad 對於葉水心〈總述講學大旨〉之衡定}

\noindent 亦不離乎此也。)

孟子引述曰:「孔子曰:聖則吾不能,我學不厭而教不倦也。子貢曰:學不厭,智也。教不倦,仁也。仁且智,夫子聖矣。」(〈公孫丑〉篇)子貢以「仁且智」贊不厭不倦,並以「仁且智」規定聖,甚為諦當。仁是真實生命之覺與健,智是真實生命所發之光照。智以覺照為性,以及物為用。(智及仁守。《中庸》亦言「成己仁也,成物智也。」)

自孔子開啟真實生命之門,由其踐仁知天示人以精進其德性生命之型範,則目擊而道存,道之「直方大」于茲顯矣。

首先儀封人曰:「二三子何患於喪乎?天下之無道也久矣。天將以夫子為木鐸。」(〈八佾〉第三)。木鐸是「施政教時,所振以警衆者也。」(朱注語)。儀封人見人多矣,而獨於孔子則說「天將以夫子為木鐸」,此見其於真實人格確有其洞見。孔子即是當時一個真實生命,他自有其振動,乃所以警醒世人之昏沉而示人以方向者也。

其次,顏淵喟然歎曰:「仰之彌高,鑽之彌堅,瞻之在前,忽然在後。夫子循循然善誘人。博我以文,約我以禮。欲罷不能,竭吾才,如有所立,卓爾!雖欲從之,末由也已!」(〈子罕〉第九)。此是顏淵直接面對孔子之真實生命而來之契會與讚嘆。

《論語·子張》第十九又記曰:「叔孫武叔毀仲尼。子貢曰:無以為也。仲尼不可毀也。他人之賢者丘陵也,猶可踰也。仲尼日月也,無得而踰焉。人雖欲自絕,其何傷於日月乎?多見其不知量也。」

又記曰:「陳子禽謂子貢曰:子為恭也,仲尼豈賢於子乎?子

\newpage\thispagestyle{empty}\addtocounter{page}{-1}\vspace*{-12mm}\begin{center}\noindent
\includegraphics[clip, trim=173pt 175pt 143pt 231pt, height=162mm]{ocr-input/image-1081.png}\end{center}

\newpage

\noindent 貢曰:君子一言以爲知,一言以為不知,言不可不慎也。夫子之不可及也,猶天之不可階而升也。夫子之得邦家者,所謂立之斯立,道之斯行,綏之斯來,動之斯和。其生也榮,其死也哀,如之何其可及也?

以上是子貢對於孔子之真實生命之契會。一則喻之如日月之「無得而踰焉」,再則喻之如「天之不可階而升也。」此與顏淵同其贊嘆。「立之斯立」四句,朱注引程子曰:「此言聖人之神化,上下與天地同流者也。」一個真實生命自然有其感通與潤澤之用。

百有餘歲後,孟子繼起曰:「非其君不事,非其民不使,治則進,亂則退,伯夷也。何事非君?何使非民?治亦進,亂亦進,伊尹也。可以仕則仕,可以止則止,可以久則久,可以速則速,孔子也。皆古聖人也,吾未能有行焉。乃所願,則學孔子也。

伯夷、伊尹於孔子,若是班乎?曰:否!自有生民以來,未有孔子也。

曰:然則有同與?曰:有!得百里之地而君之,皆能以朝諸侯,有天下。行一不義,殺一無辜,而得天下,皆不為也。是則同。

曰:敢問其所以異?曰:宰我、子貢、有若智足以知聖人,汙不至阿其所好。宰我曰:以予觀於夫子,賢於堯舜遠矣。子貢曰:見其禮而知其政,聞其樂而知其德,由百世之後,等百世之王,莫之能違也。自生民以來,未有夫子也。有若曰:豈惟民哉?麒麟之於走獸,鳳凰之於飛鳥,泰山之於丘垤,河海之於行潦,類也。聖人之於民亦類也,出於其類,拔乎其萃,自生民以來,未有盛於孔子也。」(〈公孫丑)篇)。

\newpage\thispagestyle{empty}\addtocounter{page}{-1}\vspace*{-12mm}\begin{center}\noindent
\includegraphics[clip, trim=155pt 133pt 135pt 248pt, height=162mm]{ocr-input/image-1085.png}\end{center}

\newpage\markright{第一部 \quad 第五章 \quad 對於葉水心 \quad 〈總述講學大旨〉之衡定}

此孟子引述宰我、子貢、有若之言以明孔子之特異。

〈滕文公〉篇又引述曰:「他日,子夏、子張、子游以有若似聖人,欲以所事孔子事之,強曾子。曾子曰:不可!江漢以濯之,秋陽以曝之,らら乎不可尚已。」

此引述曾子贊孔子之言也。

而孟子曰:「伯夷,聖之清者也。伊尹,聖之任者也。柳下惠,聖之和者也。孔子,聖之時者也。孔子之謂集大成。集大成也者,金聲而玉振之也。金聲也者,始條理也。玉振之也者,終條理也。始條理者,智之事也。終條理者,聖之事也。智,譬則巧也。聖,譬則力也。由射於百步之外也,其至,爾力也;其中,非爾力也。」(〈萬章〉篇)。

至稍後之《中庸》(指後半部說),則直贊曰:「仲尼祖述堯舜,憲章文武,上律天時,下襲水土。辟如天地之無不持載,無不覆幬,辟如四時之錯行,如日月之代明。萬物並育而不相害,道並行而不相悖。小德川流,大德敦化。此天地之所以為大也。

唯天下至聖為能聰明睿智,足以有臨也;寬裕溫柔,足以有容也;發強剛毅,足以有執也;齋莊中正,足以有敬也;文理密察,足以有別也。

溥博淵泉,而時出之。溥博如天,淵泉如淵。見而民莫不敬,言而民莫不信,行而民莫不說。〔……]

唯天下至誠,為能經綸天下之大經,立天下之大本,知天地之化育,夫焉有所倚?肫肫其仁,淵渊其淵,浩浩其天。苟不固聰明聖知,達天德者,其孰能知之?」

自顏淵之喟然歎其高深,子貢之喻之如天、如日月,曾子之以

\newpage\thispagestyle{empty}\addtocounter{page}{-1}\vspace*{-12mm}\begin{center}\noindent
\includegraphics[clip, trim=109pt 140pt 184pt 223pt, height=162mm]{ocr-input/image-1089.png}\end{center}

\newpage

\noindent 江漢之濯、秋陽之曝喩其「ら乎不可尚」,直至孟子之喻之以「金聲而玉振」,謂之為「集大成」,皆是相應其真實之仁者生命之渾化而言之。故孟子曰:「充實之謂美,充實而有光輝之謂大,大而化之之謂聖,聖而不可知之之謂神。」(〈盡心〉篇)。充實不可以已,自然能大能化,即《中庸》「至誠」之事也,故聖人之人格,總之即是「肫肫其仁,淵淵其淵,浩浩其天」三語之所示,再總之,亦即是「大德敦化」一語之所示。顏淵之歎,子貢之喻之如玉如日月,孟子之謂之為集大成,皆是表示聖人生命已達至「敦化」之境,皆是相應仁智之聖而言也。人之精神生命之基本方向及其最高歸宿,皆在孔子真實之仁者生命中呈現,此即人類之「木鐸」也。儀封人之語可謂得之矣。夫仁智之聖固須有創造之真實生命以達之,即契悟聖境亦須有創造之解悟以會之。孔子之弟子及再傳弟子皆有其創造之解悟者也,皆能相應聖人之真實生命而以創造的解悟相契接者也。創造的解悟者,真實生命之共鳴而閣與相會之謂也。故《中庸》曰:「苟不固聰明聖智,達天德者,其孰能知之?」自家若無此真實生命亦不能契接真實生命也。「由百世之後,等百世之王,莫之能違」,此亦真實生命之共鳴而闇與相會也,此亦創造的解悟(存在的證悟)之謂也。此之謂「相應」。

兩漢經生博士之業、災異讖緯之學,固亦推尊於孔氏,但其推尊,不流於粃糠,即流於巫魔,無一能有此「相應」之解悟也。漢人之孔子,蓋已脫離其真實的仁者生命之光暢,而為濁氣(迂)與巫氣(怪)所撐架,愚蔽漂蕩而為一不實之幻影。蓋漢人只是外在地視孔子,故無真實生命之共鳴也。

至魏、晉之玄學,雖一洗漢人之濁氣與巫氣,但王弼之「聖人

\newpage\thispagestyle{empty}\addtocounter{page}{-1}\vspace*{-12mm}\begin{center}\noindent
\includegraphics[clip, trim=86pt 144pt 201pt 223pt, height=162mm]{ocr-input/image-1093.png}\end{center}

\newpage\markright{第一部 \quad 第五章 \quad 對於葉水心〈總述講學大旨〉之衡定}

\noindent 體無」論,以及向、郭注《莊》之「迹冥論」,演變而為梁阮孝緒之「跡本論」,亦皆不能相應聖人踐仁以知天之真實生命之化境而有共鳴之契悟。聖人踐仁到「大而化之」之境固可說無,此無是以「化」來顯。「無適無莫」是無,「毋意,毋必,毋固,毋我」是無,「天何言哉」是無,「蕩蕩乎民無能名焉」是無,「無為而治者其舜也與」是無。然此種無皆是由德性生命之沛然與渾化而顯,與道家之只以有為與無為對遮而顯之「無」,復直以「無」為道為本,固有間矣。魏、晉人是以道家之無為道,而以為堯、舜與聖人(孔子)能體之而不言,老、莊言之而不能體,此固已推尊聖人矣,然而所言之道固以道家為準也。自聖人之化境言,「無」亦可用得上,魏、晉人於此非無是處,而且其如此觀望人,已體悟到精神生命最高境界之何所是,比兩漢經生只是外在地觀聖人落實而真實得多矣。然而聖人之所以至此,其實體之道是仁,其踐行之道是踐仁以知天,「無」只是其化境,而不是直接以「無」為道、為本、為冥,而以周、孔之德業為此本之跡也。德業縱可說為跡,亦是仁之實德、實理、實體之跡,唯踐仁至化境始有此如天之盛德大業耳。老、莊只知就此化境而言「無」以為道,而實體之道(仁、天)則蘊而不出,隱而不彰,其生命總不肯向此鞭辟入裡而觀體承當,此其所以一間未達而流於偏支之故也。魏、晉人正是順道家之思路而會通孔、老,雖推尊儒聖(因其能體無),而道在老、莊,是則對於儒聖之仁教全無所知也。故其創關的解悟只能契接老、莊,而不能於真實生命上與孔、孟相共鳴而閣與相會也,此亦是不相應。關此,欲所其詳,請參看《才性與玄理》第四章第七節以及第六章第四節。

\newpage\thispagestyle{empty}\addtocounter{page}{-1}\vspace*{-12mm}\begin{center}\noindent
\includegraphics[clip, trim=101pt 146pt 210pt 236pt, height=162mm]{ocr-input/image-1097.png}\end{center}

\newpage

經過魏晉南北朝、隋唐長期之歧出,至宋儒興起始復返本而重歸於相應之解悟,此謂孔、孟傳統之重振。以下試看二程對於聖賢人格之契悟與品鑒。

\begin{quotation}\kaishu 1.程子:「昔受學於周茂叔,每令尋顏子仲尼樂趣,所樂何
事。」(《遺書・二先生語二上》。〔未定誰語】)

2.伊川:「用休問老者安之,少者懷之,朋友信之。曰:此
數句最好。先觀子路顏淵之言,後觀聖人之言,分明聖人
是天地氣象。」(《遺書・伊川先生語八上〉,〈伊川雜
錄〉。)

3.程子:「顏子所言不及孔子。無伐善,無施勞,他是顏子
性分上事。孔子言安之、信之、懷之,是天理上事。」
(〈遺書・二先生語五》。〔未定誰語】)

4.又:「仲尼,元氣也。顏子,春生也。孟子,並秋殺。盡
見仲尼無所不包;顏子示不違如愚之學於後世,有自然之
和氣,不言而化者也;孟子則露其才,蓋亦時然而已。仲
尼,天地也;顏子,和風慶雲也;孟子,泰山巖巖之氣象
也。觀其言,皆可以見之矣。仲尼無迹;顏子微有迹;孟
子其迹著。」(同上)

5.又:「孔子言語句句是自然,孟子言語句句是實事。」
(同上)

6.又:「孟子有功於道,為萬世之師。其才雄。只見雄才,
便是不及孔子處。人須當學顏子,便入聖人氣象。」(同
上)\end{quotation}

\newpage\thispagestyle{empty}\addtocounter{page}{-1}\vspace*{-12mm}\begin{center}\noindent
\includegraphics[clip, trim=119pt 140pt 173pt 228pt, height=162mm]{ocr-input/image-1101.png}\end{center}

\newpage\markright{第一部 \quad 第五章 \quad 對於葉水心〈總述講學大旨〉之衡定}

\begin{quotation}\kaishu 7.又:「孔子儘是明快人,颜子儘豈弟,孟子儘雄辨。」
(同上)

8.明道日:「颜子合下完具,只是小,要漸漸恢廓。孟子合
下大,只是未粹,索學以充之。」(《遺書・二先生語
三》,謝顯道記憶平日語。)

9.又:「學者要學得不錯,須是學顏子。」(同上)

10.又:「孟子才高,學之無可依據。學者當學顏子,入聖
人為近,有用力處。」(《遺書・二先生語二上〉)\end{quotation}

\noindent 由此八、九、十、三條明道之語觀之,則上三、四、五、六、七、五條當亦明道語。

\begin{quotation}\kaishu 11.程子曰:「聖人之德行,固不可得而名狀。若顏子底一
個氣象,吾曹亦心知之。欲學聖人,且須學顏子。」
(同上。〔未定誰語,亦當是明道語。】)

12.又:「孔孟之分只是要別個聖人賢人。如孟子,若為孔
子事業,則儘做得,只是難似聖人。譬如翦綵以為花,
花則無不似處,只是無他造化功。綏斯來,動斯和,此
是不可及處。」(同上。〔未定誰語】)

13.又:「孔子爲宰則爲宰,爲陪臣則為陪臣,皆能發明大
道。孟子必得賓師之位,然後能明其道。猶之有許大形
象,然後為大山,許多水,然後為海。」((遺書・二先
生語五》。〔未定誰語〕)

14.又:「顏子作得禹稷、湯、武事功,若德則別論。」\end{quotation}

\newpage\thispagestyle{empty}\addtocounter{page}{-1}\vspace*{-12mm}\begin{center}\noindent
\includegraphics[clip, trim=173pt 158pt 149pt 250pt, height=162mm]{ocr-input/image-1105.png}\end{center}

\newpage
(同上。〔未定誰語】)

\begin{quotation}\kaishu 15.明道:「顏子默識,曾子篤信。得聖人之道者二人
也。」((遺書第十一・明道先生語一》。師訓,劉質夫
錄。)

16.又:「顏子不動聲氣,孟子則動聲氣矣。」(同上)

17.又:「學者須識聖賢之體。聖人化工也,賢人巧也。」
(同上)

18.又:「聖人之言,沖和之氣也,貫澈上下。」(同上)

19.又:「人須學顏子。有顏子之德,則孟子之事功自有,
孟子者,禹稷之事功也。」(同上)

20.又:「顏子短命之類,以一人言之,謂之不幸可也。以
大目觀之,天地之間無損益,無進退。譬如一家之事,
有子五人焉,三人富貴,而二人貧賤。以二人言之,則
不足;以父母一家言之,則有餘矣。若孔子之至德,又
處盛位,則是化工之全爾。以孔、顏言之,於一人有所
不足,以堯、舜禹湯、文武、周公群聖人言之,
則天地之間亦富有餘也。」(同上)

21.明道曰:「曾子易簀之意,心是理,理是心,聲爲律,
身為度也。」(《遺書第十三・明道先生語三〉。亥八月見
〔伯淳】先生於洛所聞。劉質夫錄。)

22.伊川曰:「問橫渠之書有迫切否?曰:子厚謹嚴。繞謹
嚴,便有迫切氣象,無寬舒之氣。孟子卻寬舒,只是中
間有些英氣。纔有英氣,便有圭角。英氣甚害事。如顏
子,便渾厚不同。顏子去聖人只毫髮之間。孟子大賢,\end{quotation}

\newpage\thispagestyle{empty}\addtocounter{page}{-1}\vspace*{-12mm}\begin{center}\noindent
\includegraphics[clip, trim=184pt 152pt 120pt 239pt, height=162mm]{ocr-input/image-1109.png}\end{center}

\newpage\markright{第一部 \quad 第五章 \quad 對於葉水心〈總述講學大旨〉之衡定}
亞聖之次也。或問:英氣於甚處見?曰:但以孔子之言
比之,便見。如冰與水精,非不光。比之玉,自是有溫
潤含蓄氣象,無許多光耀也。」((遺書·伊川先生語
四〉)

\noindent 以上共二十二條,吾皆類聚於此。除明標其為明道語或伊川語者,其餘則只云程子。其實此種品鑑,大體皆發之明道。蓋宋、明儒中,說到具體解悟,以明道為最強最高。此亦是其創關心靈之所發,故其具體解悟所成之品鑑亦是創關的品鑑,是劃時代者。宋、明儒六七百年之傳統無有能出此品鑑之規範以外者。此種品鑑亦是創造,亦是承續。蓋其所說皆是本顏淵之嘆,子貢、曾子之喻,孟子之謂集大成(金聲而玉振),《中庸》之謂「肫肫其仁,淵淵其淵,浩浩其天」,而進一步作反省的品鑑。其品鑑之標準是聖人「大而化之」之化境。其辭語已不是魏晉人就之說「無」以明道家之道,而是就之以論化不化、粹不粹、大不大,是否有英氣,是否動聲氣,是否有跡,是否是天地氣象,是否有光耀。此種品鑑是以「踐仁以知天」之仁教為根,純是對於個人德性生命之精進所達至之境界之品題。此種品題固是創關心靈之具體解悟,亦是一種藝術之欣賞,乃是一種欣趣判斷者。故以「周茂叔每令尋顏子仲尼樂趣」開端也。自其有本於先秦孔子傳統之贊聖而言,是其承續;而若非有真實生命之共鳴而閣與相會,則亦發不出此種品鑑,由此而言,是其創關。明道對於「曾子易簀之意」之品題而曰:「心是理,理是心,聲為律,身為度」,此確是千古之絕唱。若非對於內聖之學真有實感,焉能發此創關之具體解悟?內聖之學之全部律度

\newpage\thispagestyle{empty}\addtocounter{page}{-1}\vspace*{-12mm}\begin{center}\noindent
\includegraphics[clip, trim=163pt 151pt 150pt 250pt, height=162mm]{ocr-input/image-1113.png}\end{center}

\newpage

\noindent 不過三語盡之:

1.義理骨幹:天道性命相貫通。

2.踐履歸宿:踐仁以知天,即成聖。

3.践履之最高境界:「大而化之」之化境。

品鑑是就化境說,至於前兩者則是內聖之舉之內容。此皆由孔子之仁教所開啓,由孔子傳統所傳承,而為宋、明儒所繼之以發展者。此一系列無一不是真實生命之共鳴。孔子之仁教確為中國文化生命中自本自根之「直方大」而光暢之精神生命之方向之決定。繼之而發展者,不惟曾子、子思、孟子、《中庸》、《易傳》是其自本自根之發皇,即隔千餘年而興起之宋、明儒亦是其自本自根之發皇。其真實義理之內容與相共鳴之真實生命之基點無一不是自本自根者,無一是來自佛老者。只要對於孔子之仁教有實感,對於其所遺傳以及其直接繼承者之經典能逐句理會有實感,當知此自本自根之發皇決是真實,而非虛妄。謂之為陽儒陰釋者,皆是浮光掠影無真實感者之膚談,不負責任之妄語。此皆是耳食之輩於中國文化生命之長期歧出中,不知孔子之仁教為何物,不知內聖之學為何物,而忘其固有之「精神生命之方向」者之言。一個民族如長期不自知其精神生命之方向,不自知其文化創造之真實動力,則此民族即無真實精神生命之可言,只在停滯推移中與俗浮沉,此則大可哀憐者也。而浮沉久,則畏見赫日之明,而自甘于卑下,凡稍涉于玄遠,精神生命之理想之境者,概視之為佛老,如是,遂流於自貶自抑自賤而不自知,此則尤其大可哀憐者也。如葉水心者即此類之人也。彼不但無視於其並世之周張二程之業之價值,甚至並曾子、子思孟子、《中庸》、《易傳》而一起詬詆之,以為與佛老同其

\newpage\thispagestyle{empty}\addtocounter{page}{-1}\vspace*{-12mm}\begin{center}\noindent
\includegraphics[clip, trim=180pt 124pt 117pt 258pt, height=162mm]{ocr-input/image-1117.png}\end{center}

\newpage\markright{第一部 \quad 第五章 \quad 對於葉水心〈總述講學大旨〉之衡定}

\noindent 「茫昧」而「冥惑後世」,視之為「自亂」之始作俑者。此其自貶自抑自賤而癲癇狂悖竟如此,則其視孔子只為堯、舜、禹、湯、文、武之檔案家,對其仁教全無所知,又何怪焉。

程顥卒,文彥博題其墓曰明道先生。伊川為之序曰:「周公沒,聖人之道不行。孟軻之死,聖人之學不傳。道不行,百世無善治。學不傅,千載無真儒。無善治,士猶得以明夫善治之道,以淑諸人,以傳諸後。無真儒,則天下貿貿焉莫知所之,人欲肆而天理滅矣。先生生乎千四百年之後,得不傳之學於遺經,以興起斯文為己任。辨異端,闢邪說,使聖人之道煥然復明於世。蓋自孟子之後,一人而已。然學者於道不知所向,則孰知斯人之為功,不知所至,則孰知斯名〔案:即「明道」之名】之稱情也哉?」若於孔子之仁教有實感,於內聖之學有真契,則知伊川之語決非虛誕。「天下貿貿焉〔然】莫知所之」即不知其「精神生命之方向」者是也。「學者於道不知所向,則孰知斯人之為功?」此豈非葉水心之類乎?葉水心不自知其「精神生命之方向」為何物,故亦不知周、張、二程之功之大也。尤可惡者是其詬詆曾子、子思、孟子、《中庸》、《易傳》也。周、張、二程雖或不必能至先秦孔子傳統之「直方大」與光暢之境,然謂其不於真實生命上與之相共鳴而重振此精神生命之方向,則決然不可也。

\section{曾子與「孔子之傳統」兼論忠恕一貫}

\begin{quotation}\kaishu 孔子歿,或言傳之曾子,曾子傳子思,子思傳孟子。

案:孔子自言「德行顏淵」而下十人,無曾子,曰:「參\end{quotation}

\newpage\thispagestyle{empty}\addtocounter{page}{-1}\vspace*{-12mm}\begin{center}\noindent
\includegraphics[clip, trim=158pt 153pt 150pt 244pt, height=162mm]{ocr-input/image-1121.png}\end{center}

\newpage

\begin{quotation}\kaishu 也魯」。若孔子晚歲獨進曾子,或曾子於孔子歿後,德加
尊,行加修,獨任孔子之道,然無明據。

又案:曾子之學以身為本,容色辭氣之外不暇問,於大道
多遺略,未可謂至。

又案:孔子嘗言《中庸》之德民鮮能,而子思作《中
庸》。若以為遺言,則顏、閔猶無是告,而獨閟其家,
非是。若所自作,則高者極高,深者極深,非上世所傳
也。

然則言孔子傳曾子,曾子傳子思,必有謬誤。\end{quotation}

\noindent 案:此言孔子並不傳曾子,曾子亦並不能傳孔子之道。首先,吾人須知以前儒者所謂「傳」,並不像佛家禪宗祖師之相傳,孔子亦未秘傳其道於曾子。從曾子方面說,說曾子能傳聖人之道於後,只因子思是曾子弟子,而孟子又是子思弟子,孔子之道至孟子而大顯,故如此云耳。至於曾子究能傳多少,則是另一問題。此中自有一相承續之線索。凡道之傳與技藝之傳不同,此是真實生命之事。師生相承只是外部之薰習,若夫深造自得,則端賴自己。然大端方向亦必有相契,方能說傳,否則倍師叛逆,不得云傳。生命之事至為殊特,亦至為共通。然後謂之傳與不傳耶?曾子根據孔子之仁教,確有其深造自得者,此即道德意識之加強是。試看《論語·泰伯》第八記載曾子之言曰:「士不可以不弘毅,任重而道遠。仁以為己任,不亦重乎?死而後已,不亦遠乎?」仁道全而至,孔子不輕許人以仁,而亦不敢以仁自居。子曰:「若聖與仁,則吾豈敢?抑爲之不厭,誨人不倦,則可謂云爾已矣。」(〈述而〉第七)。曾子

\newpage\thispagestyle{empty}\addtocounter{page}{-1}\vspace*{-12mm}\begin{center}\noindent
\includegraphics[clip, trim=174pt 128pt 119pt 250pt, height=162mm]{ocr-input/image-1125.png}\end{center}

\newpage\markright{第一部 \quad 第五章 \quad 對於葉水心〈總述講學大旨〉之衡定}

\noindent 若非對其師教有真實感,焉能言之如此真切而嚴肅?又曰:「可以託六尺之孤,可以寄百里之命,臨大節而不可奪也,君子人與?君子人也。」(〈子罕〉第九)孔子亦曰:「三軍可奪帥也,匹夫不可奪志也。」志節堅貞,不可搖動。若非有真切之道德意識,真能通過道德自覺而作工夫者,亦何能至此?孟子言曾子「守約」。大抵「守約」二字可以代表曾子之精神。曾子言:「吾日三省吾身。為人謀而不忠乎?與朋友交而不信乎?傳不習乎?」(〈學而〉第一)。此即「守約」之表現,而守約之確義則在通過道德自覺而唯是稱仁體以動,用心於內以清澈自己之生命而期無一事之非理。至於其所就之以表現此精神之「事」,則看生活所處之環境與時代,此則不能拘定,要者在認識此道德自覺之精神。又「曾子有疾,召門弟子曰:啓予足,啟予手。《詩》云:戰戰兢兢,如臨深淵,如履薄冰。而今而後,吾知免夫!小子!」((泰伯〉第八)。歷來以為曾子重孝道,「身體髮膚,受之父母,不敢毀傷」,(《孝經》語),故臨終使其弟子開其衾而視其手足,看有無毀傷否。此解太呆笨,好像曾子一生專在保護一己之身體。當然,「父母全而生之,子全而歸之」,吾人亦不能隨便糟塌吾人之身體。在此表示「孝道」之意,固亦是一崇高之道德意識,但身體之應當保存、尊重亦不能只限於此一義。人之一已之自然生命所具之自然官能與天賦材能,人對之亦應當有一善予運用之義務,不應當自甘暴棄,妄自摧殘、墮落,以陷於「不自愛」之境,此即孟子所謂「踐形」。「踐形」當然不容易,故孟子說「惟聖人為能踐形」。此種「踐形」之義務,據康德說,亦能成一普遍的道德法則,此即其所說之第三類義務,即關於「對一個人自己之偶然性的(有功效的)義

\newpage\thispagestyle{empty}\addtocounter{page}{-1}\vspace*{-12mm}\begin{center}\noindent
\includegraphics[clip, trim=166pt 146pt 128pt 241pt, height=162mm]{ocr-input/image-1129.png}\end{center}

\newpage

\noindent 務」。反之一個人自甘暴棄,縱使出諸其自願,亦決不能成一有普遍性的道德法則,或甚至根本是不道德者。然則,曾子之「身體髮膚,受之父母,不敢毀傷」之孝道的道德意識,實應著重其「踐形」之義,不自甘暴棄之義,此即通於普遍的道德心靈之戒懼,並不是專限於自愛身體也。縱使其臨終之時,在某特殊機緣下,一時想到「啓予足啟予手」,然而其引《詩》云「戰戰兢兢,如臨深淵,如履薄冰」,又繼之云:「而今而後,吾知免夫」,此一嚴肅之道德意識卻決不能專限於「身體髮膚」之全而歸之。彼一生作自省慎獨之工夫,豈僅限於「免夫身體之毀傷」耶?故知其「戰戰兢兢」之戒慎恐懼即其「守約」、「慎獨」工夫之表示,而「而今而後,吾知免夫」,實亦函庶免於罪戾可至寡過而已矣之意。此實為一嚴肅艱苦之道德奮鬥至臨終時松一口氣撒手歸去之慨嘆。此誠是「君子曰終,小人曰死」(《禮記・檀弓》)之所謂「終」之真實表現,亦子貢所謂「君子息焉,小人休焉」(《荀子·大略篇》)之「息」之真實表現。觀其「易簀」時之仍不苟且,則其道德心靈之常明不昧可知矣。難說此非仁教之所應有也。

又「曾子有疾,孟敬子問之。曾子曰:鳥之將死,其鳴也哀,人之將死,其言也善。君子所貴乎道者三:動容貌,斯遠暴慢矣。正顏色,斯近信矣;出辭氣,斯遠鄙倍矣。籩豆之事,則有司存。」(〈泰伯〉第八)「遠暴慢」是自己可遠於粗暴放肆,不是可遠於人對我之「暴慢」。「近信」是自己近於信實不妄,不是可近於他人對我之不詐。「遠鄙倍」是自己遠於鄙俗悖戾,不是可遠他人對我之「鄙倍」。此三者皆是指自己作工夫以化自己言,若是轉移而自他人對我言,則成何義理!若如此,焉得謂之善言?只是

\newpage\thispagestyle{empty}\addtocounter{page}{-1}\vspace*{-12mm}\begin{center}\noindent
\includegraphics[clip, trim=163pt 130pt 128pt 249pt, height=162mm]{ocr-input/image-1133.png}\end{center}

\newpage\markright{第一部 \quad 第五章 \quad 對於葉水心〈總述講學大旨〉之衡定}

\noindent 適應與規避而已矣。曾謂一生作守約慎獨工夫之曾子而有此「鄙倍」之胸襟乎?然而葉水心完全缺乏慎獨之道德意識,就此力言曾子不能傳孔子之道。其《習學記言》有如下之一段:

\begin{quotation}\kaishu 曾子有疾,孟敬子問之。近世以曾子爲親傳孔子之道,死後
傳之於人,在此一章。案曾子末後語,不及正於孔子。以為
曾子自傳其所得之道則可,以爲得孔子之道而傳之則不可。
自堯舜、禹、湯文、武、周公孔子所傳皆一道。孔子
以教其徒,而所受各不同。以為雖不同,而皆受之孔子則
可;以爲堯舜、禹、湯、文、武、周公、孔子之所以一
者,而曾子獨受而傳之人,大不可也。

孔子曾告曾子吾道一以貫之,曾子既唯之,而自以爲忠恕。
案:孔子告顏子一日克己復禮天下歸仁焉。蓋己不必是,人
不必非。克己以盡物可也。若動容貌而遠暴慢,正顏色而近
信,出辭氣而遠鄙倍,則專以己為是,以人為非,而克與未
克,歸與不歸,皆不可知,但以己形物而已。且其言謂君子
所貴乎道者三,而籩豆之事則有司存。尊其所貴,忽其所
賤,又與一貫之指不合。故曰:非得孔子之道而傳之也。

夫堯舜、禹湯、文武、周公孔子之所以一者,非特
以身傳也。存之於書,所以考其德。得之於言,所以知其
心。故孔子稱天之未喪斯文為己之責。獨顏淵博我以文,約
我以禮,欲罷不能,既竭吾才,餘無見焉。夫託孤寄命雖曰
必全其節,任重道遠,可惜止於其身。然則繼周之損益為難
知,六藝之統紀為難識。故曰:非得堯舜、禹、湯、文、\end{quotation}

\newpage\thispagestyle{empty}\addtocounter{page}{-1}\vspace*{-12mm}\begin{center}\noindent
\includegraphics[clip, trim=164pt 145pt 130pt 242pt, height=162mm]{ocr-input/image-1137.png}\end{center}

\newpage

\begin{quotation}\kaishu 武、周公、孔子之所以一者受而传之也。

傳之有無,道之大事也。世以曾子為能傳,而予以為不能。
予豈與曾子辯哉?不本諸古人源流,而以淺心狹志自為窺測
者,學者之患也。\end{quotation}

\noindent 案:以上為一整段,茲略予節次,分四點言之。

第一點,水心言「以為曾子自傳其所得之道則可,以為得孔子之道而傳之則不可。」關此,若知曾子之守約慎獨之工夫,(即本道德自覺而為道德實踐之工夫),是根據孔子之仁教而來者,則謂「曾子自傳其所得之道」可,謂其「得孔子之道而傳之」亦可。「自傳其所得」即得之於孔子也。「自堯、舜、禹、湯、文、武、周公、孔子所傳皆一道」,自孔子有承於古德之道德總規與政規而言之,說「皆一道」(言同一之道),固無不可,然自孔子對於道之本統之再建言,則亦可以說一而不一。堯、舜、禹、湯、文、武、周公是王者開物成務之盡制,是原始的綜和構造,是皇極之一元,而孔子對於道之本統之再建則是太極人極與皇極三者之並建,而以太極人極為本,以皇極為末;太極是天道,人極是仁教,皇極是君道;太極是本,人極是主,皇極是用;仁者人之所以立,證實天之所以為天,而皇極則是其廣被於客觀政治社會之用。此則推進一步、開擴一步,道之中心在人極與太極,(踐仁以知天),而不在皇極也。自此言之,則不一。皇極者,自人言,是隨時隨分重點之一,自道言,則為仁教所範圍而不能外。至於人極太極則為每一人盡人道(簡言之即盡道)之必然的本質的義務,孟子所謂「求則得之,舍則失之,是求有益於得也,是求之在我者也。」而皇極則

\newpage\thispagestyle{empty}\addtocounter{page}{-1}\vspace*{-12mm}\begin{center}\noindent
\includegraphics[clip, trim=163pt 128pt 131pt 253pt, height=162mm]{ocr-input/image-1141.png}\end{center}

\newpage\markright{第一部 \quad 第五章 \quad 對於葉水心〈總述講學大旨〉之衡定}

\noindent 有命焉。故言「是隨時隨分重點之一」。然客觀地自道而言之,則皇極之道亦為仁教之本質的一環。是故儒者必應有此意識。得之雖有命,而器識則必應及此。是以儒者隨時隨緣必論政也。曾子守約慎獨之工夫固是仁教盡道之本質的義務,其皇極之意識容或有不足,然謂其所傳非孔子仁教之本質的必然的一面不可也。堯、舜、禹、湯、文、武、周公之道,一是屬於政治,一是得之有命。孔子即已不得其位矣。然其器識宏大,心願宏深,規模宏闊,故雖不得其位,而其德足以籠罩及之。若謂曾子之規模不及孔子之萬一可也。(有誰能及其萬一乎?)若必謂其不傳孔子之道則大不可也。人各有才,才不盡同。孔子之道固在也,後之起者,若有其才與識,則盡傳之無遺略可也。曾子固未嘗封閉孔子之道於一己也。若謂曾子、子思、孟子、《中庸》、《易傳》皆不能「得孔子之道而傳之」,皆非上世所傳之道,則孔子後根本無知道者,僅得一葉水心而傳之,不亦太孤單乎?此真為「淺心狹志,自爲窺測」之愚妄之言,而反責人乎?此不知孔子對於道之本統之再建,復不知曾子之守約慎獨之道德意識(所謂內聖工夫)乃仁教本質之一面,而只將道限於上世皇極之一元,以為凡不能見之於政制之行事,便為一無所有,便非上世所傳之道,是則根本不知道為何物,根本不知仁教中道德意識之自覺,根本不知自孔子始,已不止於上世之所傳,而徒以事業之意識(所謂外王)到處混抹,此種平面顛預之見而謂能知孔子之道,毋乃太過愚妄乎!儒家自孔子始,內聖外王為一綜體,內聖為本為體,外王為末為用,內聖是求之在我,是每一人之必然的義務,而外王是得之有命,是每一人之偶然的(有功效的)義務(康德語)。孔子之集團本為一士人之集團,而非一得其

\newpage\thispagestyle{empty}\addtocounter{page}{-1}\vspace*{-12mm}\begin{center}\noindent
\includegraphics[clip, trim=159pt 142pt 151pt 252pt, height=162mm]{ocr-input/image-1145.png}\end{center}

\newpage

\noindent 位之王者集團,立仁教以闢理想價值之源,弘道規以端文運史運之向,曾子雖不及於皇極之事,亦或其時尚無足以引發其政治意識之機緣,退一步,縱使其政治意識本不足,然能守約慎獨,彰著仁教中內聖之一面,亦是其本質的必然的一面,則如謂其非傳孔子之道決不可也,此乃非盲即妄之論也。

第二點,葉水心又以為曾子之動容貌、正顏色、出辭氣,乃專為規避他人對我之暴慢、欺詐與鄙倍而為,故云:「專以己為是,以人為非,而克與未克,歸與不歸,皆不可知,但以己形物而已。」若如此,則誠何心哉?焉得謂為守約慎獨之工夫?水心此言,太過無忌憚。焉可如此厚誣古人?斯則誤解「遠暴慢、近信、遠鄙倍」之遠近字,而自己又無真切之道德自覺之工夫,故有如此「鄙倍」之言。曾謂每日三省其身,臨終言「而今而後,吾知免夫」之曾子,而竟「專以己為是,以人為非」,「但以己形物而已」乎?若知「遠暴慢」是自己可遠於(免於)暴慢,不是遠他人之暴慢;「近信」是自己可近於誠信無妄,不是近他人之不詐或信我;「遠鄙倍」是自己可遠於(免於)鄙俗背理,不是遠他人之「鄙倍」,則知「動容貌,正顏色,出辭氣」正是克己慎獨之工夫,焉有「以己之是形人之非」之自矜之事乎?此誠小人之心,而謂曾子竟如此乎?葉水心無克己慎獨之工夫,不知道德自覺之嚴肅,故忍得下此話頭以詬詆曾子。若稍知「慎獨」者,有誰忍說此「肆無忌憚」之言乎?當然,道德自覺愈深愈強者,其覺自己之罪戾愈多,故「戰戰兢兢,如臨深淵,如履薄冰」,有誰能保自己之純是無非乎?面對罪惡之「深淵」,一失足成千古恨,故真做克己慎獨之工夫者,必終生「戰戰兢兢」,「戒慎恐懼」,而能保

\newpage\thispagestyle{empty}\addtocounter{page}{-1}\vspace*{-12mm}\begin{center}\noindent
\includegraphics[clip, trim=173pt 123pt 118pt 256pt, height=162mm]{ocr-input/image-1149.png}\end{center}

\newpage\markright{第一部 \quad 第五章 \quad 對於葉水心〈總述講學大旨〉之衡定}

\noindent 「己私」之真能克也。至於「天下歸仁」與否,則更未敢必也。若如此說「克與未克,歸與不歸,皆不可知」可也。但葉水心之說此三語,一似曾子根本未能作到克己,但只貿然「以己為是,以人為非」,「以己形物而已」,此則太過鄙倍。如此輕慢,厚誣古人,不可恕也。

第三點,彼又以為曾子言「君子所貴乎道者三,〔動容貌,正顏色,出辭氣〕,而籩豆之事,則有司存。尊其所貴,忽其所賤,又與一貫之旨不合。」夫言克己慎獨,「籩豆之事」,自不相干。習于籩豆之事,即能克己慎獨乎?言各有宗,事各有本,所主在此而不在彼也。克己慎獨,非必不肯認籩豆之事之價值,此與一貫之旨有何妨礙?豈必事事皆作方為一貫乎?於以知葉水心之不知本也。或曰:一貫之旨雖不必事事皆作,然亦不能「止於其身」一事不作。曰:此固也。夫孰謂克己慎獨即一事不作乎?君子之道在乎自己振拔,以清澈自己之生命,不在徒習於名物度數也。籩豆之事,歸諸有司,此只言此種事數於個人自己之道德自覺以精進其德性生命非本質的相干者,不謂可隔絕此種事也,亦不謂此種事無客觀之價值也。此只有真作慎獨工夫以精進其德性生命者方能見及此義。此是提升一步、推進一步說。此是承孔子之仁教以開關其德性生命而來者,而葉水心乃以「尊其所貴,忽其所賤」責之,此真徒習不察,冥罔不覺,不知德性生命為何物者之妄言,而反以不能一貫為口實乎?若如此責曾子,則孔子何以說子路「可使治其賦也,不知其仁也」,說冉求「可使為之宰也,不知其仁也」,說公西華「可使與賓客言也,不知其仁也」?子路之可治賦,冉求之可為宰,公西華之可與賓客言,不更是「克己以盡物」,更與一貫之旨

\newpage\thispagestyle{empty}\addtocounter{page}{-1}\vspace*{-12mm}\begin{center}\noindent
\includegraphics[clip, trim=160pt 144pt 142pt 248pt, height=162mm]{ocr-input/image-1153.png}\end{center}

\newpage

\noindent 為合乎?然而孔子卻謂其「不知其仁也」,此即示仁道之實不易,一貫之實有本,非膠著於事數即為一貫也。

葉水心於此力言曾子不能得孔子一貫之道而傳之。於是,吾人須看孔子所謂一貫究為何,葉水心心目中之一貫究為何。此是分別道之本統與對於道之本統之再建之大關節,不可不詳抉而出之也。

曾子容或不能及孔子之一貫,然一貫之旨亦決非葉水心心目中之所想也。不能及孔子之一貫,亦決非即不能傳孔子之道也。

《論語·衛靈公》第十五:「子曰:賜也,汝以予為多學而識之者與?對曰:然。非與?曰:非也。予一以貫之。」〈里仁〉第四:「子曰:參乎!吾道一以貫之。曾子曰:唯。子出,門人問曰:何謂也?曾子曰:夫子之道,忠恕而已矣。」此兩章皆記孔子鄭重言一貫,一以告子貢,一以告曾子。子貢無所表示,曾子直應曰「唯」,復以忠恕解之。葉水心以為「曾子之易聽,而不若子貢之難曉」。《習學記言》中,彼有一段言此問題云:

\begin{quotation}\kaishu 舜言精一而不詳,伊尹言一德詳矣。至孔子,於道及學,始
皆言一以貫之。\end{quotation}

\noindent 案:舜言「惟精惟一,允執厥中」,伊尹言「咸有一德」,此一皆專一、純一之一,與孔子一貫之一不同,不可混。

\begin{quotation}\kaishu 夫行之於身,必待施之於人,措之於治,是一將有時而隱。
孔子不必待其人與治也。道者,自古以為微眇難見,自古以
為纖悉難統。今得其所謂一,貫通上下,萬變逢原,故不必\end{quotation}

\newpage\thispagestyle{empty}\addtocounter{page}{-1}\vspace*{-12mm}\begin{center}\noindent
\includegraphics[clip, trim=176pt 128pt 125pt 259pt, height=162mm]{ocr-input/image-1157.png}\end{center}

\newpage\markright{第一部 \quad 第五章 \quad 對於葉水心〈總述講學大旨〉之衡定}

\begin{quotation}\kaishu 其人之可化,不必其治之有立。雖極亂大壞,絕滅蠹朽之
餘,而道固常存,學固常明,不以身歿而遂隱也。\end{quotation}

\noindent 案:此段語意閃爍難明。其究稱贊孔子乎?抑諷刺孔子乎?揆其語意,案其心念,似不以孔子之一貫為可贊也。蓋孔子亦並未得其位,使之可以將道「施之於人,措之於治」,若「必待施之於人,措之於治」,而後始可以言一貫,則「一將有時而隱」。今孔子未得其位,而竟能言一貫,是則「孔子不必待其人與治也」,依此而言,似是贊孔子。然衡之水心之心念,以及下文「今得其所謂一,貫通上下,萬變逢原」云云,語意之間,似又以孔子之一貫為不足,而隱寓諷刺之意。若誠如此,則其謬妄狂悖亦甚矣;若非然者,則亦何憾恨曾子、孟子、《中庸》、《易傳》以及其並世之周、張、程、朱如此其深耶?蓋彼以為上世開物成務、「施之於人、措之於治」之綜和構造為道之本統,此方是真一貫,而孔子之一貫,則只是停於主觀狀態中,並未能見之於行事以成客觀之構造;若以孔子之一貫爲是,則即不必憾恨曾子等等矣。蓋畔援歆羡於三代之構造,而於孔子之一貫又無的解,故閃爍其辭,貌雖稱贊,而實隱寓不滿。

\begin{quotation}\kaishu 然予嘗疑孔子既以一貫語曾子,直唯而止,無所問質,若素
知之者。以其告孟敬子者考之,乃有粗細之異,貴賤之別,
未知於一貫之理果合否?\end{quotation}

\noindent 案:關此辨見上,其疑無是處。

\newpage\thispagestyle{empty}\addtocounter{page}{-1}\vspace*{-12mm}\begin{center}\noindent
\includegraphics[clip, trim=170pt 143pt 128pt 243pt, height=162mm]{ocr-input/image-1161.png}\end{center}

\newpage

\begin{quotation}\kaishu 曾子又自轉為忠恕,忠以盡己,恕以盡人,雖曰內外合一,
而自古聖人經緯天地之妙用,固不止於是。疑此語未經孔子
是正,恐亦不可便以為準也。\end{quotation}

\noindent 案:由此即可見其不解不滿孔子之一貫,而畔援歆羨於三代之綜和構造之意甚顯矣。〈衛靈公〉第十五:「子貢問曰:有一言而可以終身行之者乎?子曰:其恕乎?己所不欲,勿施於人。」〈雍也〉第六:「子貢曰:如有博施於民,而能濟衆,何如?可謂仁乎?子曰:何事於仁?必也聖乎!堯、舜其猶病諸!夫仁者,己欲立而立人,己欲達而達人。能近取譬,可謂仁之方也已。」〈衛靈公〉第十五又載:「子張問行。子曰:言忠信,行篤敬,雖蠻貊之邦行矣。言不忠信,行不篤敬,雖州里行乎哉?立則見其參於前也,在輿則見其倚於衡也,夫然後行。子張書諸紳。」(子路〉第十三:「樊遲問仁。子曰:居處恭,執事敬,與人忠,雖之夷狄不可棄也。」衡之孔子此四段之言,曾子以忠恕解一貫,亦不算悖。後來《中庸》亦引孔子言:「忠恕違道不遠,施諸己而不願,亦勿施於人。」孟子於〈盡心〉篇第七則直曰:「萬物皆備於我矣。反身而誠,樂莫大焉。強恕而行,求仁莫近焉。」此皆環繞孔子之言而為同一思路、同一語脈。有真生命者必自然相契。蓋孔子之真生命在「仁」,體現仁之真實而落實之工夫在忠恕,(或至少從忠恕說亦無妨。)忠以盡己,恕以及人,此亦是克己慎獨之工夫,自覺地清澈自己之生命,使仁體昭然呈露,承體而行,則無往不是仁道。「己欲立而立人,己欲達而達人」,亦是忠恕之轉語,故曰「能近取譬,可謂仁之方也已。」仁是由自己之覺悟(慎獨)而超越其軀

\newpage\thispagestyle{empty}\addtocounter{page}{-1}\vspace*{-12mm}\begin{center}\noindent
\includegraphics[clip, trim=156pt 129pt 134pt 251pt, height=162mm]{ocr-input/image-1165.png}\end{center}

\newpage\markright{第一部 \quad 第五章 \quad 對於葉水心〈總述講學大旨〉之衡定}

\noindent 之私而呈現。此是精神領域價值之源之開關。仁心呈露,承仁心而行,謂之仁道。念茲在茲而不放失,謂之德性生命之精進。離開仁心仁道,無有足以一貫之者。以忠恕說一貫,即是以仁道說一貫。仁道之表現不是「博施於民,而能濟衆」。汝能以財物遍施天下乎?此正是孟子所謂「惠而不知為政」也。夫藏富於民,藏天下於天下,「不塞其源,不禁其性」(王弼注《老》語),方是真仁道。不是聚歛于己反而再博施于民以徼仁之名也。尤不是搜刮而藏之於政府,反而再配給之於人民也。如此正是不仁之甚。「己欲立而立人,己欲達而達人」,方是真仁道。立人達人,就德性生命言,自己克己慎獨精進其德性生命(立己),亦欲他人亦能克己慎獨以精其德性生命也。互相啟沃勸勉,工夫仍須自已作;自己不作,無有能使之立者。就現實生活言,政治之措施即在不塞不禁,開其路以使之自行,順其生事所需所欲以使之自立自達,不在博施於民,亦不在騷擾干禁也。三代雖是家天下,而此政規猶不失,故黃梨洲謂之藏天下於天下,非如後世之藏天下於筐筴也。孔子繼承此政規,推進一步立仁教以重新自覺地肯定之,為政治立一最高之規範,此即吾所謂敞開散開之原則,物各付物之精神也(參看《政道與治道》)。吾亦欲名此為「超越的自由主義」,或「超越的個體主義」。「超越」者,蓋自仁教而超越地言之也。非如西方之純自政治範圍內,自階級對抗而成之經驗的自由主義、經驗的個體主義也。然一落於政治內,所謂「施之於人,措之於治」,則必與此經驗的自由主義、個體主義相接頭,決不會「立理限事」以成為「封閉之社會」也。是以仁者,各人自己自立之道也,(無論是德性的或是現實的),亦是敞開之道。也惟孔子立仁教,不純囿於皇

\newpage\thispagestyle{empty}\addtocounter{page}{-1}\vspace*{-12mm}\begin{center}\noindent
\includegraphics[clip, trim=170pt 141pt 125pt 244pt, height=162mm]{ocr-input/image-1169.png}\end{center}

\newpage

\noindent 極之政規而言之,乃直下就人之當然之道而言之,乃普遍地開出理想、價值之源,開出德性生命之所以立,開出每一人直下對己對人之必然的義務與偶然的(有功效的)義務(順康德之分類),而政治之最高原則亦函攝於其內,此即所謂對於道之本統之再建也。依是,一貫之旨當如下:

1.不離經驗之學而必消化經驗之學以轉為自己之智慧,決非只「多學而識之」,停於荀子所謂「雜而無統」者。此為「一貫」之直接意義,乃與「雜而無統」相對揚。亦是「學而不思則罔,思而不學則殆」之意。

2.一貫之實即仁道,體現仁道之真實而落實之工夫為忠恕。

3.德性生命之精進上之一貫即是踐仁以知天,孟子所謂「萬物皆備於我,反身而誠,樂莫大焉。」此為內聖之一貫。

4.仁教必函攝政治上最高原則,此即「超越之自由主義」,物各付物,順個體而順成之「敞開之原則」。此為內聖外王之一貫。然此一貫只是器識上的,不必亦不能限於一人而為之。內聖之工夫為每一人對己對人之必然的義務,而貫於外王則有緣有命,此即示政治為一獨立之領域,(雖在最高原則上與仁教相貫通),而在此方面體現仁道亦常須有一獨特之生命。能將此獨特之生命納於仁教之在政治方面之最高原則而不悖,即為仁教之治,以前所謂王道,亦即是開太平之治,此為有道之世,反之則為霸道、苟偷之道,不仁之治,衰世亂世,無道之世。

孔子之一貫,其終始條理不過如此。此為對於道之本統之再建後的一貫,非只如上世皇極一元論之一貫也。道之本統,經孔子之再建後,中心仍由皇極一元轉而為直下就德性生命之所以立而言

\newpage\thispagestyle{empty}\addtocounter{page}{-1}\vspace*{-12mm}\begin{center}\noindent
\includegraphics[clip, trim=153pt 122pt 133pt 252pt, height=162mm]{ocr-input/image-1173.png}\end{center}

\newpage\markright{第一部 \quad 第五章 \quad 對於葉水心〈總述講學大旨〉之衡定}

\noindent 之。太極、人極、皇極三者並建。踐仁以知天乃為每一人之必然的義務,而皇極之及不及則待緣而有命。中心既提升而轉移,則曾子之以忠恕說一貫,正就人之必然義務而言者,此何以不能傳孔子之道耶?豈必皆如上世之綜和構造,所謂「自古聖人經緯天地之妙用」,而後始為一貫耶?若必如此而後為一貫,則凡不得皇極之位者皆不得與於一貫之道與學矣,孔子亦非一貫也,是真一貫「將有時而隱」矣。惟因孔子立仁教,普遍地開出理想、價值之源,開出德性生命之所以立,開出每一人直下對己對人之必然的義務與偶然的(有功效的)義務,而後一貫之道常明常在,因而重開文運與史運,而隨時期望有綜和構造之再現。曾子克己慎獨,明忠恕一貫之旨,久久傳習,道賴以存,學賴以明,人得賴以常目在政治之最高原則,以為必如是始可為王道,人得以知王道之真義,此如何不為傳聖人之道耶?曾子子思(《中庸》)、孟子、《易傳》皆環繞孔子之仁教而展開。內聖之道明,則外王之道亦可得而明。居常講習其本者又如何不可乎?而必因急切膠著於事功,畔援歆羡于皇極之綜和構造,遂轉而輕薄克己慎獨者之講明內聖之學(即講明每一人應盡其必然的義務之學)為不傳聖人之道,此誠何心哉?須知歷史之進展,並非易事,政治之綜和構造之實現尤其險阻重重。此非是只就經制言事功即可一蹴而幾也。能經常講明聖人之仁教,開闢生命價值之源,則政治之原則與方向庶可開出之,逐步以彰顯之也。葉水心不知仁教為何物,不知「踐仁以知天」之內聖之學之重要,故不解孔子之一貫,亦不知曾子、子思、孟子、《易傳》所說者之為何事也。

\newpage\thispagestyle{empty}\addtocounter{page}{-1}\vspace*{-12mm}\begin{center}\noindent
\includegraphics[clip, trim=173pt 206pt 134pt 245pt, height=162mm]{ocr-input/image-1177.png}\end{center}

\newpage

\begin{quotation}\kaishu 子貢雖分截文章性命,自絕於其大者而不敢近,孔子丁寧告
之,使決知此道雖未嘗離學,而不在於學,其所以識之者,
一以貫之而已。是曾子之易聽,反不若子貢之難曉。至於退
言之學,但夸大曾子一貫之說,而子貢之所聞者殆置而不
言,此又予之所不能測也。\end{quotation}

\noindent 案:「雖未嘗離學,而不在於學」云云,正是曾子忠恕踐仁之一貫。子貢「分截文章性命,自絕於其大者而不敢近」,而曾子由忠恕說一貫正是表示由忠恕踐仁以通於性命天道之大者,此有何「曾子之易聽,反不若子貢之難曉」之謂乎?「曾子之易聽」有何背於孔子之一貫乎?豈是離開忠恕踐仁而別有性命天道之可貫乎?于以知其並不解克己慎獨忠恕踐仁為何物,亦不解性命天道為何物,只是恍惚其辭隨便作文章而已耳。至於其所斥之「退言之學」乃是指當時周、張、程、朱等之順曾子忠恕踐仁之一貫之內聖之學而談性命天道者而言,彼不滿於此輩人專講性命天道之一貫,而忽略「夫子之文章」,故遂並曾子子思(《中庸》)、孟子《易傳》而一起輕薄之,以為此皆不能傳堯、舜、禹、湯、文、武、周公、孔子之道,皆非上世之所傳,皆是「子思、孟子之新說奇論」(見下)。殊不知此一傳統正是根據孔子「踐仁以知天」而來者,正是傳孔子之道者。謂其非只就上世之原始的綜和構造而言道可也,謂其不本於孔子不可也。如以為彼輩專講性命天道之一貫為一偏,則汝補之可也,何必定謂其非孔子之道乎?歷來講性命天道內聖之學者向忽視禮樂制度之外王。伊川作〈明道行狀〉云:「盡性至命必本於孝弟,窮神知化由通於禮樂。」此即內外本末之一貫。雖不詳

\newpage\thispagestyle{empty}\addtocounter{page}{-1}\vspace*{-12mm}\begin{center}\noindent
\includegraphics[clip, trim=163pt 131pt 138pt 253pt, height=162mm]{ocr-input/image-1181.png}\end{center}

\newpage\markright{第一部 \quad 第五章 \quad 對於葉水心〈總述講學大旨〉之衡定}

\noindent 究其名物度數之細節,蓋以此為專家之學,非必人人皆能講也,有能講者,則講之可也,彼輩決不抹殺其價值也。惟講多學而識之,註意禮樂刑政名物度數者,則必反對講性命天道之一貫,以為是無用之空談,蓋以其對於克己慎獨忠恕踐仁之內聖之學並無真實感,故不認其為學,亦不認其為聖人之道也。是則聖人之道全成為外在化之實然之知識,此豈孔子仁教之意乎?以此觀之,究誰能傳聖人之道,不亦昭然甚明乎?

《習學記言》復有一段,亦明曾子不能傳聖人之一貫。茲錄之于下,以見水心所謂一貫之意。

\begin{quotation}\kaishu (曲禮〉中三百餘條,人情物理,的然不違。餘篇如此要切
言語,可併集為上下篇,使初學者由之而入。豈惟初入?固
當終身守而不畔。蓋一言行,則有一事之益。如鑒睹像,不
得相離也。古人治儀,因儀以知事。曾子所謂籩豆之事,今
《儀禮》所遺與《周官》戴氏雜記者是也。然孔子教顏淵非
禮勿視,非禮勿聽,非禮勿言,非禮勿動,蓋必欲此身常行
於度數折旋之中。而曾子告孟敬子,乃以為所貴者動容貌、
正顏色、出辭氣,三事而己。是則度數折旋皆可忽略而不
省,有司徒具其文,而禮因以廢矣。故予以為一貫之語,雖
唯而不悟也。今世度數折旋既已無復可考,則曾子之告孟敬
子者,宜若可以遵用。然必有致於中,有格於外,使人情事
理,不相踰越,而後其道庶幾可存。若他無所用力,而惟三
者之求,則厚者以株守爲固,而薄者以捷出爲偽矣。\end{quotation}

\newpage\thispagestyle{empty}\addtocounter{page}{-1}\vspace*{-12mm}\begin{center}\noindent
\includegraphics[clip, trim=181pt 222pt 137pt 239pt, height=162mm]{ocr-input/image-1185.png}\end{center}

\newpage

\noindent 案:此種辯難全不相干。彼對於克己慎獨、忠恕踐仁、清澈一人之道德動機之內聖工夫全無所知,故對於曾子所貴之道,已誤認其為「以己為是,以人為非」,茲復謂其只是自身之三件事,而自身以外之「度數折旋,皆可忽而不省,有司徒具虛文,而禮因以廢矣。」夫真正之道德實踐惟在能自習慣、不自覺中,反身而誠,自覺地本之內在之天理而行。此義於孔門弟子中正見之而且彰著之於曾子之守約慎獨、戰戰兢兢之敬畏工夫中。在孔子之渾全啟發話頭中,尚不凸顯此義,唯在曾子之凝歛弘毅之精神中始真凸顯此義。此義之凸顯是道德意識、道德性自身正式挺立起而自見其自己之本質的關鍵。謂曾子尚未至聖人之化境可也。謂此義不是內聖工夫(真正的道德行為)之根本義不可也。葉水心根本無此意識。無此意識即不得謂之為知聖人之道者。徒自外面看聖人之德業文章或王者之制度功業以為道耳。此種道只是客觀事業之秩序與條理,客觀生活之軌道,社會關係、業務之制度。葉水心所知之道只是此種道。此是外在化客觀化之實然平鋪之道。由此再推而下之,人們遂以為生活之方式即為道,因而遂有堯、舜有堯、舜之道,桀、紂有桀、紂之道,甚而有所謂苟偷之道,盜亦有道之種種說法,此皆順實然平鋪而忘其本源者而來之下委。實則桀、紂並無道,苟偷根本不是道,盜則只是盜,凡此等等焉可以道謂之耶?葉水心之所謂道尚不至於此。雖實然平鋪,亦必其合眾情合天理者方可謂之道。然則其所以合眾情合天理必有其義理當然之理想根源以創發而支持之,而此義理當然之理想根源之根復何在耶?曰:即在生命之躍起、內在天理之呈現,此即孔子之仁教,而仁教復即為精神領域、價值理想之源之開闢也。而將此教通過克己慎獨之凝斂工夫以相應

\newpage\thispagestyle{empty}\addtocounter{page}{-1}\vspace*{-12mm}\begin{center}\noindent
\includegraphics[clip, trim=146pt 134pt 144pt 243pt, height=162mm]{ocr-input/image-1189.png}\end{center}

\newpage\markright{第一部 \quad 第五章 \quad 對於葉水心〈總述講學大旨〉之衡定}

\noindent 道德之本性而體現之者則首先見之而且彰著之於曾子,故曾子「所貴乎道者三」之工夫與涵義即是此精神領域、價值理想之源之開闢,此即為義理當然之理想根源之所從出。此是從習慣不自覺中躍起之工夫,此是生命之提升,故是真實生命之所在,亦是真正的道、道之本義之所在。其見於客觀之事業而成為實然之平鋪只是此根本義之道在特殊境況中之實現,亦即是在客觀的社會的生活關係或業務中之客觀化。此義只能通過孔子之仁教後始能說,在三代王者之開物成務之盡制中尚不能說,因其為原始的、不自覺的故也。其道德總規尚只在作用中,關聯著祈天永命中,即只在他律中。自孔子之仁教始正式從作用中轉為承體而起,從關聯中轉為義理當然之不容已,從他律中轉為自律,而曾子之慎獨卻正是自覺地要直接承當此依自律而行者。此種提升一步相應道德之本性而直接承當此依自律而行之慎獨工夫與禮之度數折旋不相干,亦不相礙。豈是直接承當此依自律而行便不可與於禮之度數折旋耶?而只與於禮之度數折旋中卻並無關亦無助於克己慎獨之直接承當此依自律而行也。然則葉水心所謂曾子止於自身之三事,而自身以外之「度數折旋皆可忽而不省」云云之疑難尚有任何意義耶?徒見其不知道而已矣。又曰:「今世度數折旋已無復可考,則曾子之告孟敬子者宜若可以遵用。然必有致於中,有格於外,使人情事理不相踰越,而後其道庶幾可存。.」此兩整句,前一整句最為失旨。曾子之告孟敬子者豈因「度數折旋無可考」而始可用耶?宋世豈無其禮樂之度數與折旋?任何時皆有其度數與折旋,豈因有度數折旋便不可用耶?此根本是兩層之道德行為,焉可混為一談?於以見其對於克己慎獨直接承當此依自律而行之內聖工夫根本一無所知也。然則其所謂一

\newpage\thispagestyle{empty}\addtocounter{page}{-1}\vspace*{-12mm}\begin{center}\noindent
\includegraphics[clip, trim=167pt 162pt 126pt 223pt, height=162mm]{ocr-input/image-1193.png}\end{center}

\newpage

\noindent 貫只是社會生活禮樂生活中個人與他人與度數之實際地關聯於一起而已。其所謂道只是客觀的社會關係中實然平鋪之軌制而已。而曾子之忠恕踐仁之一貫則是承體起用,由本開末之一貫,是從體上說之一貫,不只實然平鋪中個人、他人、度數關聯於一起之關聯的一貫。曾子本人雖或未能充其極,然其方向是在此,不可誣也。而曾子所傳承之道是自克己慎獨、相應道德之本性直接承當此依自律而行,所見之「義理當然之理想根源」之道、道之根本義之道,自真實生命處說之道,此正是承孔子之仁教而來者,不是寡頭的客觀關係社會業務中之實然平鋪之規制之道也。孔子教顏淵非禮勿視聽言動之「克己復禮為仁」豈是只教他「治儀,因儀以知事」耶?豈是只教他「常行於度數折旋之中」而已耶?若如此,則習儀生不更為一貫乎?而孔子亦不必不輕許人以仁矣。孟子曰:「行之而不著焉,習矣而不察焉,終身由之而不知其道者,衆也。」(〈盡心)篇)此正是葉水心之類也。而孔子之由克己復禮以指點仁,曾子之克己慎獨忠恕以踐仁,正是「行之而著,習矣而察」,以開關生命、價值之源,以求知夫義理當然之道者也。夫如此而後可以損益禮儀,調整禮儀,並隨時創造禮儀也。不是僵滯於「度數折旋」之中而為永不開眼之不相離也。寧有「動容貌,正顏色,出辭氣」而不能折旋於禮樂度數之中而為狂悖之行耶?

以上是葉水心言曾子不能傳聖人一貫之道之第三點,亦是其譏議曾子之關鍵所在,故吾復引兩段《習學記言》之文以明之。順之蔓延,不覺其辭之長也。此點明,則下第四點只結成而已。

第四點,彼以為曾子「託孤寄命,雖曰必全其節,任重道遠,可惜止於其身。然則繼周之損益為難知,六藝之統紀為難識。故

\newpage\thispagestyle{empty}\addtocounter{page}{-1}\vspace*{-12mm}\begin{center}\noindent
\includegraphics[clip, trim=156pt 152pt 141pt 231pt, height=162mm]{ocr-input/image-1197.png}\end{center}

\newpage\markright{第一部 \quad 第五章 \quad 對於葉水心〈總述講學大旨〉之衡定}

\noindent 曰:非得堯、舜、禹、湯、文、武、周公、孔子之所以一者受而傳之也。」其意是曾子之「動容貌,正顏色,出辭氣」只是限於一己之自身,自身以外一概輕忽不問,因而繼周之損益與六藝之統紀遂被泯沒而使人無法知聞也。故以為此非「堯、舜、禹、湯、文、武、周公、孔子之所以一者」。此即〈講學大旨〉主文所謂「曾子之學以身為本,容色辭氣之外不暇問,於大道多遺略,未可謂至」之意也。彼以為「堯、舜、禹、湯、文、武、周公、孔子之所以一者,〔言所以為同一道者】,非特以身傳也,〔言非特以己身傳也〕。存之於書,所以考其德。得之於言,所以知其心。」「存之於書」者即存之於六藝也。「考其德」者即由六藝以考其開物成務之道德總規與政規也。「得之於言」即得之於六藝之文也。「知其心」即由六藝之文以知上世帝王「經緯天地」之心志也。彼以為孔子只是「蒐補遺文墜典」,「唐虞三代之道賴以有傳」。孔子之得以與堯、舜、禹、湯、文、武、周公劃為一系而為「一道」,只是一檔案家之附驥尾。其如此看孔子,固不知孔子之真生命究為何在也。孔子一無所有矣,則曾子、子思、孟子之所言固不類上世之所傳,亦全無所本矣。因上世帝王實並未言及此也,故只是「子思、孟子之新說奇論」也。其對於上世帝王開物成務,原始綜和構造之本統固稍有了解,能進窺及此,已甚不易,能欣賞此道之本統亦甚美。然其對於孔子傳統之忽視與無知如此其甚,則不可恕也。蓋如此,實不必推崇孔子,只推崇堯舜禹湯、文、武而已矣。如此豈不較明朗而更率直?何必口是心非,徒為此口頭之虛稱?凡言經制事功者,其稍帶孔子皆只是敷衍門面,不敢或不便公開輕薄之而已耳。其心目中對於孔子實皆輕忽而一無所知也。此皆功業之念

\newpage\thispagestyle{empty}\addtocounter{page}{-1}\vspace*{-12mm}\begin{center}\noindent
\includegraphics[clip, trim=184pt 176pt 128pt 225pt, height=162mm]{ocr-input/image-1201.png}\end{center}

\newpage

\noindent 急切於心,故思英雄豪傑智士武夫乃至帝王出而撥亂反正重開太平也。每當衰亂之世,此念便起。此其用心,固無可非。孔子亦不抹殺管仲之功業,而亟稱「如其仁,如其仁。」惜乎自己既非英雄,亦非豪傑,更非帝王,亦仍不過一書蟲之知識分子耳。乃轉其急切之思而為怨天尤人。帝王不可得,乃畔援歆羨,轉而言經制事功,言皇、帝、王、霸之略,言博文有用之學。夫言之可也,無人薄而非之。而怨天尤人則不可,輕忽孔子之傳統尤不可。即暫時撇開孔子傳統而不言,真切正視歷史盛衰大運之所由,政治治亂大關之所在,亦極是有用者,而乃徒陷於直接之實用主義、博聞雜識之經驗主義、原始之體力主義,美其名曰朴、曰實、曰經制事功、曰三王之道,而終於一無所有、一無事功,並未知歷史盛衰大運之所由,政治治亂大關之所在,亦未研究出一個經制事功之方向與夫合理政治之原則,只落得以詞章考據終其身,經制事功只成為攻擊內聖之學之藉口,何其大背於初衷耶?此中之故,言經制事功者不可不自反而深長思之也。自南宋永嘉、永康言經制事功、皇帝王霸之學以後,明末顧亭林與顏、李,其思路語脈與規模無一能出葉水心之外者,皆不能知孔子傳統內聖之學之意義,亦皆是怨天尤人反對談性命天道者。惟顧亭林最篤實典雅,只反陸、王,不反程、朱;雖不反程、朱,但亦不講程、朱之學。顏、李則並程、朱而反之。然而實皆不及葉水心之勇敢與一貫,並會子、子思、孟子、《易傳》而一起皆反之也。孔子成了孤家寡人,只成堯、舜、禹、湯、文、武之檔案家,則實並孔子而亦抹之也。若能稍知孔子之仁教,何至如此狂悖哉?是以孔子者對于道之本統之再建者也,曾子、子思、孟子、《易傳》乃本孔子之仁教而展開者,此為孔子之傳統。程、

\newpage\thispagestyle{empty}\addtocounter{page}{-1}\vspace*{-12mm}\begin{center}\noindent
\includegraphics[clip, trim=158pt 159pt 143pt 228pt, height=162mm]{ocr-input/image-1205.png}\end{center}

\newpage\markright{第一部 \quad 第五章 \quad 對於葉水心〈總述講學大旨〉之衡定}

\noindent 朱、陸、王者則又繼承此傳統而前進者也。故真傳孔子之道者,曾子子思、孟子、《易傳》以及程、朱、陸、王也,決不在葉水心也。此一傳統在以往雖較偏重內聖一面,然此一面卻是仁教之本質的一面,不可忽而抹之也。外王一面雖不彰顯,然此一面之解答決非只膠著於經制事功、皇帝王霸之略、原始之綜和構造者所能勝任也。此還須求之於講內聖之學者。此須大開大合以觀之。凡此吾已言之於《政道與治道》。然則經制事功並非不可言,綜和構造並非不可期,然決非怨天尤人抹殺內聖之學者所能辦。吾非反其言經制事功,乃反其反孔子傳統也。自己於孔子之仁教無所知,而反以「淺心狹志自為窺測」責曾子,豈不謬哉?豈不狂悖矣哉?

\section{孟子之開德與言治}

\begin{quotation}\kaishu 孟子亟稱堯、舜、禹、湯、伊尹、文王、周公,所願則孔
子,聖賢統紀則得之矣。養氣知言,外明內實。文獻禮樂,
各審所從矣。夫謂之傳者豈必曰授之親而受之的哉?世以孟
子傳孔子,殆或庶幾。然開德廣,語治驟,處己過。涉世
疏。學者趨新逐奇,忽亡本統,使道不完而有跡。

案:孟子言性言命言仁言天,皆古人所未及,故曰:開德
廣。齊滕大小異,而言行王道,皆若建瓴,故曰:語治
驟。自謂庶人不見諸侯,然以彭更言考之,後車從者之
盛,故曰:處己過。孔子亦與梁邱據語,孟子不與王驩
言,故曰:涉世疏。學者不足以知其統,而襲其迹,則以
道為新說奇論矣。\end{quotation}

\newpage\thispagestyle{empty}\addtocounter{page}{-1}\vspace*{-12mm}\begin{center}\noindent
\includegraphics[clip, trim=170pt 167pt 131pt 224pt, height=162mm]{ocr-input/image-1209.png}\end{center}

\newpage

\noindent 案:此對於孟子有稱許,亦有批評。但其稱許與批評俱不相干。曾子、子思、孟子俱本孔子之仁教而發展。孟子亟稱堯、舜、禹、湯、伊尹、文王、周公是稱其道德總規,亦即政規。(參看上章第三節)。此種「聖賢統紀」,曾子、子思未必不知也,亦未必不承認也,豈只孟子始「得之」哉?而「所願則孔子」,正示其本孔子之仁教而展開也。孔子立仁教,是對於道之本統之再建。「孟子傳孔子」是以仁教為中心而傳之也,非籠統不分,視孔子為堯、舜、禹、湯、文、武之檔案家而傳之也。「養氣知言,外明內實」,此是仁智雙彰。就養氣言,亦是本仁義內在之性善而來者,非是葉水心之膠著於名物度數,個人他人與度數關聯於一起之「內外交相明」(一貫)也。「文獻禮樂,各審所從」,此句說孔子可,說孟子則為不相干。正因對於孟子之真生命無所解,故浮泛其辭,亂說一通耳。是其稱許無是處也。彼謂「孟子言性言命言仁言天,皆古人所未及,故曰開德廣。」此是無知乎?是閉眼瞎說乎?焉得如此隨意妄言!帝、天、天命、天道不是上世本統中之主要概念乎?命、仁、天不是孔子「践仁以知天」中之重要概念乎?惟「性」之問題是孟子時特顯之問題,而孟子亦積極地創關地盛言此問題,遂奠定儒家中內聖之學之基礎。其不順「生之謂性」(自生言性)之老傳統言性,而創闢地自仁義內在以言超越的義理當然之性、內在道德性之性,或道德的創造性之性,正是本上世道德總規(政規)中道德意義之概念,如聰明、勇智、敬德之類,以及超越意義之概念,如帝、天、天命,天道之類,通過孔子之仁教而如此言之者。是其言性是以上世道德總規為背景,以孔子之仁為關鍵,此點雖「古人所未及」,然正是本古人(上世及孔子)所應有之發展。此

\newpage\thispagestyle{empty}\addtocounter{page}{-1}\vspace*{-12mm}\begin{center}\noindent
\includegraphics[clip, trim=155pt 138pt 131pt 238pt, height=162mm]{ocr-input/image-1213.png}\end{center}

\newpage\markright{第一部 \quad 第五章 \quad 對於葉水心〈總述講學大旨〉之衡定}

\noindent 應有之發展是通過孔子對於道之本統之再建後,所必然應有之義理發展,由此發展可以見出孔孟真生命之所在。若由仁教之氣象與境界言,孟子此步開擴之發展,其「開德」並不能更廣大於孔子,亦不能超越孔子「踐仁以知天」之範圍,不過由「踐仁以知天」轉進而為「踐仁盡性以知天」而已。此是一種義理充實之發展,使仁教中直接所函之內聖之學,即每個人精進其德性生命之學,更有系統,更有其自覺的「可能之基礎」。孔子不言不為狹,而孟子言之不為廣。孔子不能一時俱言也。至於「言命言仁言天」皆古人所已及,何言「皆古人所未及」耶?在此系中,上世只言命言天,至孔子益之以仁,至孟子再進之以性,皆有所承本,亦皆有所開展。若云孟子「開德廣」,則孔子已「開德廣」矣。「廣」豈不甚佳矣哉?豈必永封於「祈天永命」之他律境界中,始為盡美盡善耶?「開德廣」豈是空說大話,空為冥惑之論耶?葉水心之為此言以示其不滿之意,實足以見其對於孔子之仁以及孟子之真生命全無所知(無真實感)而已矣,故遂視為「新說奇論」也。此其愚妄為何如!

抑孟子不只進之以「性」,而且即心以言性,又盛言「心」,心之地位自孟子始正式挺立起。蓋若真正視人之內在道德性之真實呈現因而得以有自覺的純淨的道德行為之可能,不能不正視到心。性不只是一個抽象的空概念,其具體的表現而可以為吾人之道德實踐之所以可能之先天的超越根據者即在「心」。心是具體化原則,亦是實現原則。它是吾人生命得以物物而不物於物之真正的主宰,它指導並決定吾人行為之方向,它是吾人之真正的主體。吾人即由心之「悅理義」而同時亦即是理義以見吾人之性,即精進德性生

\newpage\thispagestyle{empty}\addtocounter{page}{-1}\vspace*{-12mm}\begin{center}\noindent
\includegraphics[clip, trim=163pt 162pt 148pt 236pt, height=162mm]{ocr-input/image-1217.png}\end{center}

\newpage

\noindent 命、發展德性人格之所以可能之先天的超越根據,故此性是具體的、真實的道德性之性,亦即「道德的創造性」之性,孟子亦稱之曰「本心」。葉水心對於孟子之如此重視心,當然更不能滿意,更須視為「新說奇論」矣,蓋古之人誠未有如此言之者也。

《習學記言》中於前引批評曾子「動容貌,正顏色,出辭氣」一段後有一案語云:

\begin{quotation}\kaishu 案:〈洪範〉耳目之官不思,而為聰明,自外入以成其內
也。思曰睿,自内出以成其外也。故聰入作哲,明入作謀,
睿出作聖,貌言亦自內出而成於外。古人未有不內外交相
成而至於聖賢,故堯舜皆備諸德,而以聰明為首。〔……
夫古人之耳目安得不官而蔽於物?而思有是非邪正,心有人
危道微,後人安能常官而得之?舍四從一,是謂不知天之所
與,而非天之與此而禁彼也。蓋以心為官,出孔子之後。以
性為善,自孟子始。然後學者盡廢古人之條目,而專以心為
宗主,致虛意多,實力少,測知廣,凝聚狹,而堯舜以來
內外相成之道廢矣。\end{quotation}

\noindent 案:此即批評孟子「耳目之官不思而蔽於物」,「心之官則思」一段文之義也。(見《孟子・告子》篇)葉適只是五官與心官平列而實然地平視其交互之作用。耳目得其為耳目之官而盡其視聽,心得其爲心之官而盡其思。盡其視聽而為聰明,盡其思而爲睿。盡其視之明而為哲(「明作哲」),盡其聽之聰而為謀(「聰作謀」),盡其思之睿而為聖(「睿作聖」)。此是順(洪範〉就事論事而只

\newpage\thispagestyle{empty}\addtocounter{page}{-1}\vspace*{-12mm}\begin{center}\noindent
\includegraphics[clip, trim=165pt 143pt 132pt 239pt, height=162mm]{ocr-input/image-1221.png}\end{center}

\newpage\markright{第一部 \quad 第五章 \quad 對於葉水心 \quad 〈總述講學大旨〉之衡定}

\noindent 說其事實之當然,此是原始的自然主義之講法。耳固有其聽之聰,目固有其視之明,此官之自然者也。然此自然之聰能必其「入而作謀」乎?(葉氏謂「聰入作哲」,依〈洪範〉當為「聰入作謀」。)此自然之明能必其「入而作哲」乎?(葉氏謂「明入作謀」,依〈洪範〉當為「明入作哲」。)此非分析命題也。若非心之清明,聰入不必能作謀,明入不必能作哲。然則聰之自外入,以成其內之謀,明之自外入以成其內之哲,固須賴心之清明為其條件始能成其為謀為哲也。「古人之耳目安得不官而蔽於物?」此固也,然此「古人」是何種之「古人」耶?必是能「不蔽於物」之古人也。某人能不蔽于物,不保人皆能不蔽于物也;有不蔽于物之時,不保其能常不蔽於物也。「耳目之得以為官而不蔽於物」亦非分析命題也。葉水心徒作實然之平說,有何益哉?耳目固有其自然之聰明,然亦有其天然之封限。囿于其封限,其本身不必能入而作謀作哲也。作謀作哲是心之超曠與綜和,而不為耳目所限者也。不必說作謀作哲,即其本身之聰明,如能正當地成其為聰,成其為明,而不誤用濫用其聰明,亦非易事。此孔子所以言「非禮勿視,非禮勿聽」也。〈洪範〉亦言「敬用五事」。放肆怠傲,不敬用其視聽,邪視邪聽以誤用其聰明者多矣。視聽之禮不禮、敬不敬不在視聽本身之中也。視聽之「敬用」以期合禮中度,不濫用其聰明,以正當地成其為聰成其為明,即孟子所謂「踐形。」孟子曰:「形色天性也。惟聖人為能踐形。」此即葉水心所謂「古人之耳目安得不官而蔽於物」疑問語句中之「古人」也。雖不必如孟子所言「唯聖人為能踐形」,然至少「踐形」並非易事,此須有一超越者作根據始可能也。然則聰明正當地成其為聰明,乃至順此正當的聰明之

\newpage\thispagestyle{empty}\addtocounter{page}{-1}\vspace*{-12mm}\begin{center}\noindent
\includegraphics[clip, trim=163pt 159pt 137pt 234pt, height=162mm]{ocr-input/image-1225.png}\end{center}

\newpage

\noindent 入而作謀作哲,皆非囿於耳目之官本身所能濟。孟子即因此而開出大體與小體。對超越之心言,視耳目之官為小體有何不可乎?小之所以爲小,以其有封限也,以其自身並不能即作主而成其為正當之聰明,乃至入而作謀作哲也。其本身乃為自然無色者,吾人之道德生命能由之而建立乎?若無主之者,其與物接而為物所牽引者多矣。其視聽之悖禮亂度而誤用濫用其聰明者亦多矣。此一顯明之道德真理本是無人能有異議者,本不須置辯。而葉水心於此橫起小慧,正見其滿腦子是經制事功而不知所以成此事功者之道德意識為何物也。自覺地作道德實踐者始見出此大體小體之分別,知小體之不足恃,而必有其大者以主之。當然天資高者自能「內外交相成」,而似不必真自覺到此分別,以自覺地作工夫,此所謂天才也,亦是原始的自然生命之健旺有以致之;然天才可遇而不可求,自然生命有健旺之時,亦有其衰退之時。當其健旺也,可無往而不順適而偶合於道;及其衰退,則昏庸頹肆,不能自持,能保其內外「必」相成乎?是則未可便以原始之自然生命為足恃也。若純以天才與自然生命為足恃,而不知有自覺地作克己慎獨工夫之一義,則不能有自覺地作道德實踐之可能,亦必落於命定主義之斷滅,是則事功者真成望洋興嘆之事耳,亦只成天才家之縱橫揮洒之事與夫原始的自然生命之偶然之事,此可以為常法乎?自孔子起,立仁教以開關生命價值之源,示人以精進其德性生命之常度,不使人停於自然主義、天才主義、命定主義之自然狀態中,此所以為對於道之本統之再建,而為人類開眼目者也,曾子、子思、孟子繼之,為此「常度」展開其可遵循之定規,使人可自覺地作工夫以精進其德性生命之發展,使「內外交相成」真有其可能之基礎,而不只是一偶

\newpage\thispagestyle{empty}\addtocounter{page}{-1}\vspace*{-12mm}\begin{center}\noindent
\includegraphics[clip, trim=170pt 140pt 123pt 242pt, height=162mm]{ocr-input/image-1229.png}\end{center}

\newpage\markright{第一部 \quad 第五章 \quad 對於葉水心〈總述講學大旨〉之衡定}

\noindent 然,此有何廢於「古人之條目」,又有何背於孔子之仁教乎?只因葉水心不知德性生命為何物,停於自然主義、天才主義、命定主義而不進,妄以「古人之耳目安得不官而蔽於物」疑難孟子大體小體之分,此真所謂「不知類」者也。(不知義理之分際與相成)。

以上是就耳目之官言,兹再就心之官言。「思有是非邪正,心有人危道微」,此自不錯,孟子亦不謂「後人能常官而得之」。不但後人不能,即古人亦不必能。正因不必能常官而得之,故(洪範〉之「思曰睿」,亦須「敬用」其「思」,始能睿,「睿作聖」也;正因心有人心之危,道心之微,故须「惟精惟一」之工夫以贞定其道心。道心即孟子所謂「本心」也。孟子言「心之官則思,思則得之,不思則不得也」,此所謂「思」正是本心所發之超越而總持之妙用,能提住耳目、主宰耳目,而不為耳目所囿所拖累者也。並非是無色之可邪可正可是可非之罔思也。思而不正,焉能成其為大體?正是隨軀殼起念,而以耳目之官為主,從而成其不正之聰明耳。此是一起皆小,而「内外交相成」亦不可得而言矣。「其內外交相成」是成小人,而非「至於聖賢」也。然則徒心與耳目平列而只實然地平視其「內外交相成」,而不知「敬用」之工夫、「精一」之工夫、大體小體之分、主從本末之分,未必能「至於聖賢」也。而孟子本「敬用」、「精一」之工夫,推進一步而言大體小體之分、主從本末之分,此於「至於聖賢」之「內外交相成」有何妨礙?不但無妨礙,且正是分解出「至於聖賢」之「內外交相成」之所以可能之基礎亦即實現之基礎。葉水心看不到「敬用」字,看不到「精一」字,看不到堯之「欽明文思安安、尤恭克讓,〔……克明俊德」,看不到舜之「濬哲文明,溫恭允讓」,看不到夏禹之

\newpage\thispagestyle{empty}\addtocounter{page}{-1}\vspace*{-12mm}\begin{center}\noindent
\includegraphics[clip, trim=168pt 156pt 133pt 235pt, height=162mm]{ocr-input/image-1233.png}\end{center}

\newpage

\noindent 「后克艱厥后,臣克艱厥臣」,看不到「穆穆文王,於緝煕敬止」,看不到(康誥〉之「文王克明德慎罰」以及(召誥〉之「敬德」以「祈天永命」,只著眼於思之睿與視聽之聰明之「內外交相成」之實然之成果,而不知其使之所以然之戒慎敬德之工夫,反妄謂「堯舜皆備諸德,而以聰明為首。」夫堯之「欽明文思」,舜之「濬哲文明」,豈是指耳目之自然聪明而言耶?孟子本「敬用」、「精一」之工夫進而言大體小體之分,正是相應上世諸賢聖之戒慎敬德而展開。以此觀之,究誰更切合於「古人之條目」,究誰更合于道之本統、古人之體統,不亦甚顯明乎?而反妄謂孟子「捨四從一,是謂不知天之所與!」夫大體小體之分非捨耳目等五官(非四官)而廢之也,亦非專「與此而禁彼也」。孟子明言「此天之所與我者」。孟子非不知耳目與心皆天之所與,天亦自不專「與此而禁彼」。然雖皆天之所與,豈無大小主從之別乎?言大小主從,豈即專與心而禁耳目乎?雖不禁耳目,然不可不有主于耳目。否則,何必言戒慎敬德乎?自此而言,「專以心為宗主」有何不可乎?汝能本末顛倒以耳目為宗主乎?設曰當皆為宗主,不可「專以心為宗主」。曰:皆為宗主,即是無宗主。皆有其用可也,皆為宗主不可也。此是就大小本末主從而言宗主問題,非是泛言皆有其用也。「專」字是就宗主言,不就皆有其用言也。專一其宗主,非即廢耳目而不用也。何來「虛意多、實力少、測知廣凝聚狹」之病乎?後之不能如上世綜和構造之「實力」與「凝聚」,乃是政治問題之不能決,豈是因分主從本末,以心為宗主而然耶?本上世戒慎敬德之規模而分辨出大小本末主從之關係,使自覺地作道德實踐為可能,開出理想價值之源,使開物成務,重現綜和構造為可能,此正

\newpage\thispagestyle{empty}\addtocounter{page}{-1}\vspace*{-12mm}\begin{center}\noindent
\includegraphics[clip, trim=158pt 139pt 140pt 245pt, height=162mm]{ocr-input/image-1237.png}\end{center}

\newpage\markright{第一部 \quad 第五章 \quad 對於葉水心〈總述講學大旨〉之衡定}

\noindent 是重新恢復創造之生命以期有新的「實力」與繼起之「凝聚」,此豈是蹈虛之空論,冥惑之「測知」耶?徒因葉氏兩眼只看現成,而不正視道德實踐所以可能之本源,故以冥惑測知視之耳。是亦自己冥惑而已矣。

《習學記言》復有一條云:

\begin{quotation}\kaishu 耳目者,視聽之官也。心而無與乎視聽之事,則官得守其
分。夫心有欲者,物過而目不見,聲至而耳不聞也。故曰:
上離其道,下失其事。心術者無為而制竅者也。〔案:此正
是心為主之意】。案:孟子稱耳目之官心之官,予論之已
詳。然則執心既甚,形質塊然,視聽廢而不行。蓋辯士之言
心也,其爲心之害大矣。〈洪範〉「思曰睿,睿作聖。」各
守身之一職,與視聽同謂之聖者,以其經緯乎道德仁義之
理,流通於事物變化之用,融暢淪浹,卷舒不窮而已。惡有
守獨失類,超忽惝悅,狂通妄解,自矜鬼神也哉?\end{quotation}

\noindent 案:此條所言:「案」字以下意與前同。其稱孟子之言心為「辯士之言心」,其狂悖為何如!此足見其憾恨之深,而不知自己之無知也。其憾恨之焦點或在其並時之周、張、程、朱等,然而因此便憾恨曾子、子思與孟子,亦概歸於無知而已矣;即周、張、程、朱等亦無不超過葉適遠甚,其憾恨之反動亦終歸於無知而已矣。夫言學在能降心與平心,如此狂悖反動,何有于事功?此真是「狂通妄解,守獨失類」,而不知學問事業之艱難之鄙夫!而反以其蛙見責人耶?既知「上離其道,下失其事,心術者無為而制竅」,又何憾

\newpage\thispagestyle{empty}\addtocounter{page}{-1}\vspace*{-12mm}\begin{center}\noindent
\includegraphics[clip, trim=167pt 152pt 147pt 248pt, height=162mm]{ocr-input/image-1241.png}\end{center}

\newpage

\noindent 于大體小體之分耶?心之「無欲」而不影響耳目之聰明,使之「得守其分」,汝以為不須戒慎敬德即可自然而能耶?若如此,則世之「有欲」以使其「目不見,耳不聞」者又何其多耶?「思曰睿,睿作聖。各守身之一職,與視聽同謂之聖者,以其經緯乎道德仁義之理,流通于事物變化之用,融暢淪浹,卷舒不窮而已。」此言固已甚美矣,此正是聖人之境界,汝以為人人皆可自然而能耶?自孔子立仁教後,經過曾子子思、孟子、《中庸》、《易傳》下及周、張、程、朱等正思講明此事,自覺地作工夫,以期達乎此境耳。汝不降心用功,自求聞達,而妄肆反動之鄙心,詬詆先賢,多見其不知量而已矣。

《習學記言》又有一條云:

\begin{quotation}\kaishu 古之聖賢無獨指心者。舜言人心道心,不止於治心。孟子始
有盡心、知性、〔貴】心官賤耳目之說。蓋辯士索隱之流
多論心,而孟、荀為甚。\end{quotation}

\noindent 此與上為同一低劣,誠可恥之尤!

又有一條云:

\begin{quotation}\kaishu 孔子講道無內外,學則內外交相明。〔……]近世又偏墮太
甚,謂獨自內出,不由外入。往往以為一念之功,聖人可招
而致。不知此心之稂莠,未可遽以嘉禾自名也。\end{quotation}

\noindent 然則汝知心之稂莠乎?更不知也!

\newpage\thispagestyle{empty}\addtocounter{page}{-1}\vspace*{-12mm}\begin{center}\noindent
\includegraphics[clip, trim=164pt 128pt 133pt 256pt, height=162mm]{ocr-input/image-1245.png}\end{center}

\newpage\markright{第一部 \quad 第五章 \quad 對於葉水心〈總述講學大旨〉之衡定}

順其反動之鄙心,不但曾子、子思、孟子不在其眼下,即孔子亦不在其眼下。《習學記言》有一條云:

\begin{quotation}\kaishu 志學至從心所欲為限節者,非所以為進德之序,疑非孔子之
言。由後世言之,祖習訓故、淺陋相承者,學而不思之類
也。穿穴性命、空虛目喜者,思而不學之類也。士不越此二
途。\end{quotation}

\noindent 孔子明言「五十有五而志於學,〔……】七十而從心所欲不踰矩」,而葉水心亦竟疑之。此其狂悖無知太甚!

又有一條云:

\begin{quotation}\kaishu 禮非玉帛所云,而終不可以離玉帛。樂非鐘鼓所云,而終不
可以舍鐘鼓。(仲尼燕居〉乃以几筵升降酌廚酬酢不必謂之
禮,而以言而履之爲禮。以綴兆羽籥鐘鼓不必謂之樂,而偶
行而樂之為樂。是則離玉帛舍鐘鼓,而寄之以禮樂之虛名,
天下無禮樂矣。\end{quotation}

\noindent 此並對孔子亦不滿。蓋必天下人皆依仿葉適所說,始可謂之道。若有一字不同於彼,皆非知道者。其反動無知已近於白癡癲癇之狀態。言學至此,誠可悲也!

又有一條云:

\begin{quotation}\kaishu 不遷怒,不貳過,以是爲顏子之所獨能,而凡孔氏之門皆輕\end{quotation}

\newpage\thispagestyle{empty}\addtocounter{page}{-1}\vspace*{-12mm}\begin{center}\noindent
\includegraphics[clip, trim=158pt 150pt 136pt 236pt, height=162mm]{ocr-input/image-1249.png}\end{center}

\newpage

\begin{quotation}\kaishu 愠頻復之流與?是孔子誣天下以無人也。蓋置身於喜怒是非
之外者,始可以言好學。而一世之人常區區乎求免於喜怒是
非之內而不獲,如泥而揚其波也。嗚呼!必若是,則惟顏
子耳。\end{quotation}

\noindent 因孔子所言之「好學」不同於彼,故亦罵之!天下寧有此蠻橫不講理之人乎!何其根性如此之恶劣耶?

由以上觀之,其責孟子為「開德廣」者,實不是廣不廣之問題,乃實是根本不應談仁、命、性、天乃至於心也。總之是不應談內聖之學。吾人只應停於原始不自覺之狀態中。其所謂聖賢只是上世自然之直接行動之人。其所謂道只是此自然之直接行動之所表現者。以上乃關於「開德廣」者。

至於「齊滕大小異,而言行王道,皆若建瓴,故曰語治驟。」茲察孟子之言,豈其「驟」乎?亦閉眼瞎說而已。「滕文公問曰:『滕小國也,間於齊楚,事齊乎?』孟子對曰:『是謀,非吾所能及也。無已,則有一焉。鑿斯池也,築斯城之,與民守之,效死而民弗去。則是可為也。』滕文公問曰:『齊人將築薛,吾甚恐,如之何則可?』孟子對曰:『昔者大王居邠,狄人侵之,去之岐山之下居焉。非擇而取之,不得已也。苟爲善,後世子孫王者矣。君子創業垂統為可繼也。若夫成功,則天也。君如彼何哉?強爲善而已矣。』滕文公問曰:「滕小國也,竭力以事大國,則不得免焉。如之何則可?』孟子對曰:昔者大王居邠,狄人侵之。事之以皮幣,不得免焉。事之以犬馬,不得免焉。事之以珠玉,不得免焉。乃屬其耆老而告之曰:狄人之所欲者,吾土地也。吾聞之也,君子

\newpage\thispagestyle{empty}\addtocounter{page}{-1}\vspace*{-12mm}\begin{center}\noindent
\includegraphics[clip, trim=169pt 128pt 129pt 251pt, height=162mm]{ocr-input/image-1253.png}\end{center}

\newpage\markright{第一部 \quad 第五章 \quad 對於葉水心〈總述講學大旨〉之衡定}

\noindent 以其所以養人者害人。二三子何患乎無君?吾將去之。去邠,踰梁山,邑於岐山之下居焉。邠人曰:仁人也,不可失也。從之者如歸市。或曰:世守也,非身之所能為也。效死勿去!君請擇於斯二者。』」(〈梁惠王〉篇)。試看此三問三答有一非真實語乎?小國亦有小國自處之道。不知所以自處之道,不實行其所以自立者,而唯畔援歆羨,日視他人顏色以行事,一方怨天尤人,一方投機取巧,自己腳跟先站不穩,未有能屹立于斯世者也。孟子答滕文公一則曰「效死而民弗去」,一則曰「強為善而已矣」。以大王居邠為例以告人,正是告以自處自立之道,此是經制事功之切實者,捨此寧有巧便之法乎?巧便者責之為迂闊可也,而責之為「語治驟」,則隨便亂說而已矣。試問其不驟者又如何?孟子明言「若夫成功,則天也」此豈有一毫「骤」之意哉?至於滕文公為世子時,自「孟子道性善,言必稱堯舜」起,與之言「三年喪」,言「民事」,言「取於民有制」,言「庠序學校」,言「仁政必自經界始」,言「助法」,言「井田」,皆經國之大者,亦經制事功之切者,皆所以自處自立之正道大道,故曰「子力行之,亦以新子之國」。雖小國亦有自處自立之道,「行王道」豈在國之大小乎?如此而言治,尚謂之為驟,試問其不驟者又如何?葉水心之經制事功能外此而復有不驟之法乎?

至於齊大國也。孟子曰:「以齊王,由〔猶】反手也」。公孫丑曰:「若是,則弟子之惑滋甚。且以文王之德,百年而後崩,猶未洽於天下。武王、周公繼之,然後大行。今言王若易然,則文王不足法與?」曰:「文王何可當也?由湯至於武丁,賢聖之君六七作,天下歸殷久矣。久則難變也。武丁朝諸侯,有天下,猶運之掌

\newpage\thispagestyle{empty}\addtocounter{page}{-1}\vspace*{-12mm}\begin{center}\noindent
\includegraphics[clip, trim=182pt 157pt 142pt 249pt, height=162mm]{ocr-input/image-1257.png}\end{center}

\newpage

\noindent 也。紂之去武丁未久也。其故家遺俗,流風善政,猶有存者。又有微子、微仲、王子、比干、箕子、膠鬲,皆賢人也。相與輔相之,故久而後失之也。尺地莫非其有也。一民莫非其臣也。然而文王猶方百里起,是以難也。齊人有言曰:雖有智慧,不如乘勢,雖有镃基,不如待時。今時則易然也。夏后、殷、周之盛,地未有過千里者也,而齊有其地矣,雞鳴狗吠相聞,而達乎四境,而齊有其民矣。地不改辟矣,民不改聚矣,行仁政而王,莫之能禦也。且王者之不作,未有疏於此時者也。民之憔悴於虛政,未有甚於此時者也。飢者易為食,渴者易為飲。孔子曰:德之流行,速於置郵而傳命。當今之時,萬乘之國行仁政,民之悅之,猶解倒懸也。故事半古之人,功必倍之,惟此時爲然。」(〈公孫丑〉篇)

由此觀之,以齊之大,發政施仁,行王道,豈不甚易乎?若以此爲驟,則試問不驟者將如何?至其於〈梁惠王〉篇,答齊宜王之問而言王道,層層辨詰剖白,務在啟沃其心志,使之明於為君之分、爲政之道,儒者對於政治之最高原則得以明,亦不悖於三代之政規。只是齊宣王「惛,不能進於是」。只聞以迂闊譏之者,未聞以「語治驟」責之者。葉水心之言可謂全不相應矣。孟子曰:「三代之得天下也以仁,其失天下也以不仁。因之所以廢興存亡者亦然。」又曰:「孔子曰:仁,不可爲衆也。夫國君好仁,天下無敵。」又曰:「桀紂之失天下也,失其民也。失其民者,失其心也。得天下有道,得其民,斯得天下矣。得其民有道,得其心,斯得民矣。得其心有道,所欲與之聚之,所惡勿施爾也。民之歸仁也,猶水之就下、獸之走壙也。」(皆見〈離婁〉篇)。此是儒者言政之最高原則,亦是直本孔子之仁教而言之者;堯舜三代之政規

\newpage\thispagestyle{empty}\addtocounter{page}{-1}\vspace*{-12mm}\begin{center}\noindent
\includegraphics[clip, trim=162pt 129pt 138pt 255pt, height=162mm]{ocr-input/image-1261.png}\end{center}

\newpage\markright{第一部 \quad 第五章 \quad 對於葉水心〈總述講學大旨〉之衡定}

\noindent 無論為禪為繼,皆不能背乎此。即今之言政治亦不能背乎此。背乎此者,則為霸道,為肆於民上之劫持之道、苟偷之道、極權專制之道。如此言治,而葉水心竟以「驟」責之,然則其所謂經制事功者究何在耶?

孟子言道,其言内聖者不待言,即其「語治」者,亦皆生機暢達,語意豁順,有本有源,有始有終,從無阿世苟合之意,所謂提得起放得下者是也。無論其言滕言齊,以及答梁惠王、答齊宣王,乃至(萬章〉篇之言「唐虞禪,夏后、殷、周繼」,皆釐然有以見儒者言政之原則,既不背於孔子,亦不背於堯、舜三代之政規,葉水心所謂「本統」者。其言道之振拔完具有如此,彼其充實不可以已,誠命世之大才、弘揚聖道之龍象也。而葉水心器小不堪大就,竟謂其「開德廣,語治驟。處已過,涉世疏」(此後兩點甚無謂,吾故不論),使「學者趨新逐奇,忽亡本統,使道不完而有迹」,又曰:「學者不足以知其統,而襲其跡,則以道為新說奇論矣。」此所謂「學者」即指周、張、二程言,朱子猶不在其眼下。實則周、張、程、朱之深造自得於上世之本統以及孔、孟之弘規皆超過葉水心遠甚。縱使彼等偏重內聖,多發明「踐仁盡性知天」之義,亦不得謂之「趨新逐奇」,更不得視孟子之內聖之學為「新說奇論」。即彼等言政,雖著墨不多,然無一有背於孔、孟之禮樂與仁政,亦無一不肯定堯、舜三代之政規,何得謂其「忽亡本統,使道不完而有跡」?若葉水心者始真為「不知其統」,不知道之所以為道,而徒落於第二義第三義以下之「跡」而言之也。

中國自秦漢以來直至今日,形成其歷史文化之嚴重癥結者唯在政治一關之不透。秦之法家、南宋之衰微、明末之亡於滿清,皆

\newpage\thispagestyle{empty}\addtocounter{page}{-1}\vspace*{-12mm}\begin{center}\noindent
\includegraphics[clip, trim=180pt 149pt 134pt 249pt, height=162mm]{ocr-input/image-1265.png}\end{center}

\newpage

\noindent 足以刺激人心而接觸此問題。而南宋之衰微與明之亡尤是此問題特顯之時。陳同甫、葉水心以及顧亭林、黃梨洲、王船山皆於此有實感而慨乎言之。惟此五人亦有別。能本諸第一義而言之者,黃梨洲、王船山也。至於陳同甫、葉水心、顧亭林則只落在第二義乃至第三義而言之。其關鍵即在此三人皆反對談性命天道,皆不知內聖之學之重要,皆不能貫通「堯、舜、三代之政規與孔子對於道之本統之再建以及孟子承孔子之仁教而弘揚」之線索而開擴其心志,弘大其器識,以綜觀此問題之最後症結之何所在。心思有所隔絕而不通透,遂不期而下委。雖似較切實,而寞落於第二義第三義而不自知。言經制事功而不知經制事功之關鍵,故其言經制事功只落於直接之實用主義、散文之事務主義、直覺之英雄主義(天才主義),旁及顏、李乃成為原始之體力主義。此皆不足以觸及政治問題之症結以及其要求經制事功之本旨。蓋彼等之要求經制事功初只是對家國天下而發,而此實只是政治問題,故其所言之經制事功實與政治問題之解決有必然之連結,乃是政治問題之投映。而若不能貫通史運文運,以及堯、舜、三代之政規與孔、孟對於道之本統之再建而觀之,即不能得此問題症結之所在。此本是一綜和意識之事,本是開物成務重造國體政體之事・經制事功乃是本重造之合理正常之國體政體而來之各方面之綜和構造,各方面之獨立而又相關之自本自根之生長與繁榮,此乃是結果,而不是動源。人能各安其業,自能各成其事,不待他人之日事號招也。而言經制事功實用之學者,其恰當之意義,實自高一層次上就其能開此事功者而言之,故以經制言事功,此不誤也。然以經制言事功實是開物成務重造合理正常之國體政體之事。堯、舜、三代之開物成務皆是有德有

\newpage\thispagestyle{empty}\addtocounter{page}{-1}\vspace*{-12mm}\begin{center}\noindent
\includegraphics[clip, trim=165pt 126pt 133pt 254pt, height=162mm]{ocr-input/image-1269.png}\end{center}

\newpage\markright{第一部 \quad 第五章 \quad 對於葉水心〈總述講學大旨〉之衡定}

\noindent 生命、自身挺立、合聚群力以創造之者。吾人不能只觀其業績,數他人家珍,即算是言經制事功,即算是了解道之本統。吾人須是以生命頂上去,一如堯、舜、三代之作主而作主地以觀之,內在於生命之流而存在地以觀之,以期「創造生命」之承續,重新開物成務,重開文運與史運,重見綜和構造之來臨,如是,方是真言經製事功者,方能真了解道之本統之何所是。如是,即不能不提起來以觀堯、舜、三代之政規(道德總規)以及孔子對於道之本統之再建與孟子之承孔子再建而展開之理想之弘大。此是重造之源、開物成務之本。孔、孟之言治固不能算是盡其極,然後之來者欲想以至仁大義客觀地「立千年之人極」(船山語)以解決中國政治問題之症結以開經制事功之大用,要不能不本孔、孟言治之原則以前進。自此言經制事功者是一義之言,黃梨洲、王船山庶幾近之。自今日觀之,黃、王猶不足,故仍須再本之以前進,要須推至盡頭,必解決此問題而後止。凡此吾已詳論之於《政道與治道》。決不能隔絕內聖之學,堵塞生命之機、創造之源,不知道之所以為道,不知「由道之本統至本統之再建」乃一大開合之發展,而只落於第二義第三義之跡,只是平面地、現象地、知識地言經制事功,以為如此可解決政治問題與事功問題者。若如此,則正是南轅而北轍,適得其反。此是永嘉、永康、亭林、顏、李言學之蔽固處,而蔽固之甚而徹底者為葉水心。彼標舉「道之本統」以為講學之宗旨,似彌近理,而有統貫,然而忽視孔子對於道之本統之再建,輕薄曾子、子思、孟子、《中庸》、《易傳》之承孔子仁教而展開之內聖之學為「新說奇論」,則又大亂真而過亦大。吾故詳抉其謬以明經制事功之不可如彼言。

\newpage\thispagestyle{empty}\addtocounter{page}{-1}\vspace*{-12mm}\begin{center}\noindent
\includegraphics[clip, trim=154pt 141pt 141pt 249pt, height=162mm]{ocr-input/image-1273.png}\end{center}

\newpage

《習學記言》有一條云:

\begin{quotation}\kaishu 孔子未當以辭明道。內之所安則為仁,外之所明則為學。學
即六經也。至於內外不得異稱者,於道其庶幾矣。子思之
流,始以辭明道。辭之所之,道亦之焉。非其辭也,則道不
可以明。《中庸》未必專子思作,其徒所共言也。孟子不止
於辭,而辯勝矣。苟卿本起稷下,所言皆欲挫辯士之鋒。怒
目裂眥,極口切齒。先王大道,至此散薄,無復淳完。或者
反謂其才高力強,易於有行。學者苟知辭辯之未足以盡道,
而能推見孔氏之學,以上接賢聖之統,散可復完,薄可復淳
矣。不然,斷港絕潢,爭於波靡,於道何有哉?\end{quotation}

\noindent 案:此種論調看似甚美,亦倒是徹底而乾脆,然出之於葉適之口,則完全不相應。若出之於「崇尚黃、老,返真歸樸」之道家,則有意義;出之於「言語道斷,心行路絕」,直證「不二法門」之佛家,亦有意義;出之於《易傳》所謂「默而成之,不言而信,存乎德行」,則更有意義。惟葉適之如此言,則無意義。道家反人為、反虛文、反辭辯,直指太古之渾樸,是原則上象徵一種修養之境界,並非是對於歷史事實之肯斷。佛家「言語道斷,心行路絕」明是由辭辯而至超辭辯,此亦是修行之境界。《中庸》言「博學之,審問之,慎思之,明辨之,篤行之」,最後歸於「篤行」,亦「默而成之,不言而信,存乎德行」之意。辭辯固不足以盡道,夫誰不知之?然並不因此而廢博學、審問、慎思、明辨之經歷。葉水心自非道家之境界,尤非佛家之境界。自居於言聖人之道而反辭辯,其

\newpage\thispagestyle{empty}\addtocounter{page}{-1}\vspace*{-12mm}\begin{center}\noindent
\includegraphics[clip, trim=170pt 128pt 123pt 253pt, height=162mm]{ocr-input/image-1277.png}\end{center}

\newpage\markright{第一部 \quad 第五章 \quad 對於葉水心〈總述講學大旨〉之衡定}

\noindent 意實只欲停於堯、舜、三代之原始綜和構造而不准前進耳。此非「博學、審問、慎思、明辨」而最後歸於「篤行」之義,而是對於一階段歷史事實之肯斷,將其不能原則化而原則化之以排除其他也。此尚非王船山所謂「立理限事」,而是立事限事。其為荒謬不通,斷可知矣。抑即堯、舜、三代亦非無辭辯也。假定堯舜時尚無文字,豈無口辯乎?假定言語簡單,豈無簡單之口辯乎?抑不只口辯而已矣,口辯不能決,則繼之以武力。武力定而篤行至,遂有開物成務之事業。三代不更有辭辯與征伐之事乎?〈湯誓〉、〈湯誥〉、(伊訓〉、〈泰誓〉、〈牧誓〉〈大誥〉〈康誥〉、〈酒誥〉、〈召誥〉、〈洛誥〉等等,不皆辭辯乎?若非此等辭辯,安知其行動與觀念之方向?所謂道之本統者又烏得而知之?豈是王者之辭辯與征伐便是道,孔子傳統之辭辯與德性領域之開闢便不是道乎?此得勿太势利乎?抑孔子亦非不「以辭明道」也。孔子豈無是非乎?《論語》中記弗子問禮、問仁、問智、問學、問政,孔子不皆有裁成之之答辭乎?雖其答辭簡而約、文而婉、質而直,不大聲以色,亦是慎思明辨也。出之以言辭,即是辭辯矣。思而辨之,辭而辯之,所以明其道之方向,豈真廢辭辯哉?孔子曰:「予欲無言。」子貢曰:「子如不言,則小子何述焉?」子曰:「天何言哉?四時行焉,百物生焉,天何言哉?」(〈陽貨〉第十七)此是由名言到超名言之境界,亦函名言之不可廢。其辭辯之簡約、文婉、質直,乃是辭辯之方式與內心之和平,豈真無辭辯哉?若云不止於辭辯,而要歸於德行,則可也。若云根本無辭辯,則非是。若根本反辭辯,則尤非是。「內之所安則為仁」,固不錯,然孔子斥宰予之「安」為不仁,而示應由「不安」以識仁,由不安再至於安

\newpage\thispagestyle{empty}\addtocounter{page}{-1}\vspace*{-12mm}\begin{center}\noindent
\includegraphics[clip, trim=161pt 149pt 147pt 247pt, height=162mm]{ocr-input/image-1281.png}\end{center}

\newpage

\noindent 以爲仁,此皆須明辨以通之,而後能真知仁。若冒然以「內之所安為仁」,則宰予之安何以為非仁乎?安於冥頑亦仁乎?安於私利、暴棄亦仁乎?不慎思明辨何以知仁?「外之所明則為學」,亦不錯,然云「學即六經」,六經不皆辭辯乎?通過六經之辭辯而審問、慎思以明辨之,亦辭辯也。明辨辭辯以定其是非,通其實理,則道因而明。儒者之道固不同於墨子,亦不同於道、法。此非因辭辯而明者乎?道固不止辭辯,亦不盡於辭辯,故「修辭立其誠,所以居業也。」(〈乾·文言〉)存誠以篤行之,「暢於四支,發於事業,美之至也。」(〈坤·文言〉)。若云根本廢辭辯,豈聖人之道乎?然则子思之辭而明之,孟苟之辭而辯之,有何不可乎?焉有所謂「先王大道,至此散薄,無復淳完」之說乎?若誠有之,亦只是原始的綜和構造在歷史發展中之不復永住。其不復永住亦是勢有必然,理有固然。(此「必然」是歷史的必然,非形式邏輯的必然。「固然」是歷史發展進一步實現之理上的固然,非自然事實之固然。)強其不復永住者而必永住之,以抹殺其他,此乃無比之專橫與頑固,尚何道之可言?孔子傳統之辭明辨解正是孔子對於道之本統之再建後而來之大開合,以期重開文運與史運也。其一時尚未能至客觀的(政治的,外王的)綜和構造者,亦是歷史條件之不備。然辭而明之,以定歷史之方向,亦正是期有重新綜和構造之再來臨。歷史之發展是各方面之事,焉可因辭辯所明之道一時未能實現,即從而反辭辯乎?是以葉水心反子思、孟子之辭辯,並不在明「由思辨以歸於篤行」之義,其意是欲停於原始之綜和構造而不准前進耳。如此反辭辯而講道之「淳完」乃無意義者。蓋原始之綜和構造中亦有辭辯也。嚮往原始綜和構造而欲去其辭辯,是乃自相矛

\newpage\thispagestyle{empty}\addtocounter{page}{-1}\vspace*{-12mm}\begin{center}\noindent
\includegraphics[clip, trim=167pt 128pt 131pt 258pt, height=162mm]{ocr-input/image-1285.png}\end{center}

\newpage\markright{第一部 \quad 第五章 \quad 對於葉水心〈總述講學大旨〉之衡定}

\noindent 盾者。汝欲去辭辯以「上接賢聖之統」,竟是上接一木乃伊以供人觀賞耶?此本是能篤行與否之問題,能實現與否之問題,而乃講成有無「辭辯以明道」之問題,此真「斷潢絕港」之死見,反動心理無意義之謬辯,徒見其對於道無真實感與責任感而已矣。

\section{《易傳》與周、張、二程}

\begin{quotation}\kaishu 自是而往,爭言千載絕學矣。《易》不知何人所作。雖曰伏
義畫卦,文王重之,案周太卜掌三易,經卦皆八,別皆六十
四,則畫非伏義,重非文王也。又周有司以先君所為書為
占,而文王自言王用享於岐山乎?亦非也。有《易》以来,
筮之辭義不勝多矣。

《周易》者,知道者所為,而有司所用也。孔子爲之著
〈象〉、〈象〉,蓋惜其為他異說所亂,故約之中正,以明
卦爻之旨,黜異說之妄,以示道德之歸。〔案:此亦辭辯以
明道也。然則上文言「孔子未嘗以辭明道」者妄也。

其餘〈文言〉上下〈繫〉、〈說卦〉諸篇,所著之人或在
孔子前,或在孔子後,或與孔子同時,習《易》者彙為一
書。後世不深考,以為皆孔子作。故〈象〉、〈象〉揜鬱未
振,而十翼講誦獨多。〔案:此句不通。〈彖〉〈象〉即
在十翼中。又案:〈文言〉上下〈繫〉、〈說卦〉等縱非
孔子所作,其所說之義理亦不悖于孔子之〈彖〉、〈象〉。
〈彖傳〉、〈象傳〉乃《易》之義理之綱領,〈文言〉上
下〈繫〉、〈說卦〉等即環繞此綱領而展開。總此十翼,可\end{quotation}

\newpage\thispagestyle{empty}\addtocounter{page}{-1}\vspace*{-12mm}\begin{center}\noindent
\includegraphics[clip, trim=174pt 147pt 122pt 243pt, height=162mm]{ocr-input/image-1289.png}\end{center}

\newpage

\begin{quotation}\kaishu 名曰孔門《周易》方面之義理。後之言《易》之義理者皆當
以此孔門義理為首出之範本,若云此一套不合筮辭之原義,
則自孔子已誤引,另講其原義可也,若云此一套為「新說奇
論」,則孔子之〈彖〉、〈象〉即此說奇論之綱領。總不能
謂此非孔門之義理,而一筆抹去之也。]

魏、晉而後,遂與老、莊平行,號爲孔、老。〔案:使「與
老、莊平行」者始於王弼之以道家義註《易經》。若知其不
合,則批而抉之,使復其為孔門之義理可也。周、張、程、
朱等即作此工作也。]

佛學後出,其變為禪,喜其說者,以為與孔子不異,亦援十
翼以自況,故又號為儒、釋。本朝承平時,禪說尤熾。豪傑
之士有欲修明吾說以勝之者,而周、張、二程出焉。自謂出
入於老佛甚久,已而曰吾道固有之矣。故無極太極,動靜男
女;太和參兩,形氣聚散,細感通;有直內,無方外,不
足以入堯舜之道:皆本於十翼,以為此吾所有之道,非彼
之道也。及其啟教後學,於子思、孟子之新說奇論,皆特發
明之。大抵欲抑浮屠之鋒銳,而示吾所有之道若此。然不悟
十翼非孔子作,則道之本統尚晦;不知夷狄之事,本與中國
異,而徒以新說奇論闢之,則子思、孟子之失遂彰。〔案:
此皆性與人殊、反常心理之論。]

案:佛在西南數萬里外,未嘗以其學求勝於中國。其俗無
君臣父子,安得以人倫義理責之?特中國好異者,折而後
彼,蓋禁令不立而然。聖賢在上,猶反手。惡在校是非角
勝負哉?〔案:說此空洞、愚陋、不負責任之大話、廢話\end{quotation}

\newpage\thispagestyle{empty}\addtocounter{page}{-1}\vspace*{-12mm}\begin{center}\noindent
\includegraphics[clip, trim=158pt 126pt 139pt 257pt, height=162mm]{ocr-input/image-1293.png}\end{center}

\newpage\markright{第一部 \quad 第五章 \quad 對於葉水心〈總述講學大旨〉之衡定}

\begin{quotation}\kaishu 何為哉!太冥頑無知矣!

范育序《正蒙〉,謂此書以「六經所未載聖人所不言」者
與浮屠、老子辯,豈非以病為藥,而與寇盜設郛郭,助之捍
禦乎?鳴乎!道果止於孟子而遂絕耶?其果至是而復傳耶?
孔子曰:學而時習之。然則不習而已矣。\end{quotation}

\noindent 案:葉氏〈總述講學大旨〉正文至此止。此下有一較長之案註,首言「與浮屠辯者」不解「浮屠書言識心」,「言見性」,言「滅」,言「覺」之意,「不知其所謂而強言之」。此則葉氏亦不必知,可略而不論。次言儒者不該援引《大傳〉、子思、孟子之言與之辯。此則見下文。此處不錄免重複也。此〈講學大旨〉葉氏自注謂是因范育序(正蒙〉而作。蓋欲以關周、張、二程之與佛、老角勝負之甚無謂也。並因而責及(易傳〉,責及曾子、子思、孟子,甚而責及孔子。溯而至於堯、舜、三代,逐以為「道之本統」只如此,外此皆非「道之本統」,皆當廢除,皆須視為「新說奇論」。若誠如此,則道豈但「果止於孟子而遂絕」,抑且止於孔子而遂絕,抑亦不但止于孔子而遂絕,實則當止于周公之死而遂絕也。以言經制事功之人而毫無歷史意識,竟成此「斷潢絕港」之論,不亦大可哀憐乎?茲歸於本段就其所責之《易傳〉與周、張、二程之關係而總言道之本統與孔子對於本統之再建後所開之道術之發展如下:

假定(彖傳〉〈象傳〉真是孔子所傳,(即非孔子作,亦無關),則乾、坤〈文言〉以及上、下〈繫〉所暢發之義理無一與〈彖傳〉、〈象傳〉相背者。(〈說卦〉後半部以及(序卦〉與

\newpage\thispagestyle{empty}\addtocounter{page}{-1}\vspace*{-12mm}\begin{center}\noindent
\includegraphics[clip, trim=182pt 150pt 137pt 248pt, height=162mm]{ocr-input/image-1297.png}\end{center}

\newpage

\noindent 〈雜卦〉自無甚意義,其出現當更晚,大抵就漢之象數而撰成,附于〈繫傳〉末,湊成十篇,名為十翼,此則人皆能知之。漢之象數自別是一套,與孔門義理無關。)吾人可以(彖〉、〈象〉、〈文言〉與上、下〈繫〉為代表,總名為孔門《周易〉方面之義理。因其確能代表儒家之精神。此部孔門之義理,中心思想在「窮神知化」。(下〈繫〉云:「窮神知化,德之盛也。」)而「窮神知化」之規範綱領則在(乾彖〉與〈坤彖〉。而〈乾彖〉與〈坤彖)之中心思想只在〈乾彖〉「乾道變化,各正性命,保合太和乃利貞」之一語。「知化」者知天地生化之德也。(下〈繫)云:「天地之大德曰生。」)此總名曰「天道」。「窮神」者窮生化不測之神也。上〈繫〉云:「陰陽不測之謂神。」又云:「知變化之道者,其知神之所為乎?」又云:「神無方而易無體。」(「生生之謂易」,「陰陽不測之謂神」。)又云:「易無思也,無爲也。寂然不動,感而遂通天下之故。非天下之至神,其孰能與於此?」又云:「夫易,聖人之所以極深而研幾也。唯深也,故能通天下之志。唯幾也,故能成天下之務。唯神也,故不疾而速,不行而至。」又云:「蓍之德圓而神,卦之德方以智,六爻之義易以貢。聖人以此洗心,退藏於密,吉凶與民同患。神以知來,智以藏住。其孰能與於此哉?古之聰明叡智神武而不殺者夫!是以明於天之道,而察於民之故,是興神物以前民用。聖人以此齋戒以神明其德夫。」又云:「探賾索隱,鉤深致遠,以定天下之吉凶,成天下之疊亹者,莫大乎蓍龜。」下〈繫〉又引「子曰:知幾其神乎?」又引「子曰:乾坤其易之門耶?乾陽物也,坤陰物也。陰陽合德而剛柔有體,以體天地之撰,以通神明之德。」(上〈繫〉亦曰:「乾

\newpage\thispagestyle{empty}\addtocounter{page}{-1}\vspace*{-12mm}\begin{center}\noindent
\includegraphics[clip, trim=153pt 124pt 138pt 247pt, height=162mm]{ocr-input/image-1301.png}\end{center}

\newpage\markright{第一部 \quad 第五章 \quad 對於葉水心〈總述講學大旨〉之衡定}

\noindent 坤,其易之縕耶?乾坤成列,而易立乎其中矣。乾坤毀,則無以見易。易不可見,則乾坤或幾乎息矣。」朱子注曰:「「乾坤毀』,謂卦畫不立。「乾坤息』,謂變化不行」)。上〈繫〉又云:「變而通之以盡利,鼓之舞之以盡神。」又云:「易有聖人之道四焉。以言者尚其辭,以動者尚其變,以制器者尚其象,以卜筮者尚其占。是以君子將有為也,將有行也,問焉而以言,其受命也如響。無有遠近幽深,遂知來物。非天下之至精,其孰能與於此?」「至精」即至神也。以上凡言神字,或直指天道生化之不測言,或落于蓍卦卜筮之知幾言,或自「聖人以此洗心退藏於密」,「以此齋戒以神明其德」言,要之,類而通之,其義一也。要皆極深研幾,直湊事物之裏,洞開生化之源者也。亦皆提醒人之德性之真生命而直證宇宙之真生命者也。是故生化不測即是神,神直接為化所函,故亦曰「神化」。〈說卦〉云:「神也者妙萬物而為言者也。」神以「妙用」言,不以「逕挺持體」之人格神言。而生化不測即妙用也。是故窮神即是知化,知化即是窮神。而窮不是科學之窮,(不是以器求之),科學之窮究不能至於神。知亦不是質測之知,(不是知識之知),質測之識知不能至於化。此所謂「窮神知化」者即是德性生命之證悟,亦是發之於德性生命之超越的形而上之洞見。其根據完全在「仁」。故上(繫〉云:「顯諸仁,藏諸用,鼓萬物而不與聖人同憂,盛德大業至矣哉!」此即根據仁所證悟之天道也。天道不是蹈空飄蕩之冥惑之事,而是顯之於仁,由仁以實之。「藏諸用」即是藏之於生化之大用。而生化不測之大用亦不是憑空冥惑之事,而是仁德之實功。仁即生道也。天道之生化過程以元亨利貞四字表之,故〈文言〉曰:「元者善之長也。亨者嘉之會也。

\newpage\thispagestyle{empty}\addtocounter{page}{-1}\vspace*{-12mm}\begin{center}\noindent
\includegraphics[clip, trim=167pt 146pt 126pt 238pt, height=162mm]{ocr-input/image-1305.png}\end{center}

\newpage

\noindent 利者義之和也。貞者事之乾也。君子體仁足以長人,嘉會足以合禮,利物足以和義,貞固足以幹事。」故元亨利貞之過程即是仁義禮智之過程。《中庸》以一「誠」字攝之,亦即一誠之歷程也。而周濂溪《通書》即據之以言「元亨誠之通,利貞誠之復」,即以誠字貫元亨利貞也。誠之實德之所貫,即是仁義禮智實德之所貫,綜持言之,亦可曰仁之實德之所貫。天道只是一仁字,亦只是一誠字。是則天道之生化秩序(宇宙秩序)亦即是一道德秩序也。此是發之於德性生命之必然的證悟,定然而不可移,確然而不可疑者也。

上世言帝、言天,乃至言天道、言天命,猶是發之於原始的宗教之情以言之,而且是關聯著王者之受命以言之,故隱約地有人格神之意,至少亦是冥冥中有一真宰之意。上世之言德乃是關聯著「祈天永命」而言之,此是他律之道德(德行)。自孔子出而講仁教,教人「踐仁以知天」,則「仁」之一字即成為使兩頭充實之概念:內在地使德行成為自律之德行,使人正式認識道德之本性乃惟是通過「慎獨」(守約)之工夫自覺地行其義理之當然以清澈自己之生命者,因而相應道德本性之純粹的道德意識逐正式照體挺立,于此遂擺脫上世他律道德之虛歉狀態,此即曾子之守約慎獨,孟子之「盡心知性知天」(或「踐仁盡性知天」)一系之所為;同時復超越地使原始宗教之情之天以及關聯著王者受命而言之天轉為生化不測之天,轉為天命流行之體,使高掛難諶、須通過「儀型文王,萬邦作孚」之天以及「天聽自我民聽天視自我民視」之天,正式彰著為化育之實德,彰著為大生廣生之創造之實體、寂感之真幾,於此逐擺脫上世宗教之情之天之虛歉狀態,而成為一實德彌綸之圓盈

\newpage\thispagestyle{empty}\addtocounter{page}{-1}\vspace*{-12mm}\begin{center}\noindent
\includegraphics[clip, trim=167pt 131pt 141pt 256pt, height=162mm]{ocr-input/image-1309.png}\end{center}

\newpage\markright{第一部 \quad 第五章 \quad 對於葉水心〈總述講學大旨〉之衡定}

\noindent 境界,此即《中庸》(後半部)《易傳》一系之所為。在孔子固尚未明言至此,然孔子之踐仁並非是他律者,孔子之言仁是真實德性生命之開啟,是一體之沛然莫之能禦,其踐仁是行其心之所不容已。這一體之沛然固屬渾淪整全而並未曾予以分別釐定。然而曾子由之以守約慎獨自覺地行其義理之當然,孟子由之以言性善,認為此一體之沛然即是吾人之性體,此亦並無不可者,且亦正是極順適之開展。復次,其渾淪整全之一體之沛然固無法在原則上劃定其界限者。如其有極限,則其極限必是與天地合德、與日月合明、與四時合序、與鬼神合吉凶,而此正是無限,此即示仁之體物而不遺、仁體之遍在也。而《中庸》、《易傳》即根據此仁體之遍在而言天道即仁道,仁道即生道。天道者即「乾知大始坤作成物」,生化不測之真幾、實體也。此亦無不可者,且亦正是「充實不可以已,調適而上遂」之必然的證悟。孔子固未甚顯言及此。然「天何言哉?四時行焉,百物生焉,天何言哉?」此已隱函此義。孔子之「踐仁以知天」,宗教之情之天的意味固甚重,故吾曾名之為「超越的遙契」(參看《中國哲學之特質》)。然此是孔子之存在地踐仁之敬畏意識之所至。孔子固明言「若聖與仁,則吾豈敢?」然正因其虔誠精進,而成其為聖、成其為仁。在孔子只是存在地踐履,踐履之無有窮極,而此無有窮極之踐履正函其是聖是仁,而此聖此仁之境正函以仁道實天道,天道亦不過是一體之沛然,《中庸》、《易傳》即根據孔子之證境而顯言之者。孔子之存在地踐履,與《中庸》、《易傳》之存在地顯揚,乃屬於不同之層次。孔子是作,《中庸》《易傳》是述。「作者之謂聖,述者之謂明」(〈樂記〉語)。此即為孔子之傳統,凡繼孔子而來者皆是本孔子而為

\newpage\thispagestyle{empty}\addtocounter{page}{-1}\vspace*{-12mm}\begin{center}\noindent
\includegraphics[clip, trim=183pt 162pt 124pt 236pt, height=162mm]{ocr-input/image-1313.png}\end{center}

\newpage

\noindent 言。然此言,亦是存在地言之,並非無實感之「冥惑」,亦非無本之「新說奇論」也。對孔子之存在地踐履言,或亦可如葉水心言《中庸》、《易傳》是「以辭明道」,然此「以辭明道」卻正是「述者之謂明」,述其所本所據與所宗主者之生命之風範以端正德性生命精進之方向與極致,使人有所持循與旨歸。此完全是內聖之學之明述,是照體挺立的純淨道德意識之所必函。而若無實感真契,亦不能言之如此透澈圓明也。此豈只如葉水心所謂之「以辭明道」乎?如此所言之天雖為大生廣生之實體,一體彌綸之充盈,擺脫上世言天之虚歉狀態,然敬畏之情超越之感並不因此而喪失。〈乾·文言〉言「大人者〔……】先天而天弗違,後天而奉天時」,則固不只是一天地鬼神且不能違之仁體之首出,德性生命之健行,而且亦虔誠敬畏而「奉天時」也。此即為超越與內在之圓一。是則孔子之超越的遙契與《中庸》《易傳》之內在的證悟,在仁教中常是並存而不相悖,圓融為一而不睽隔。不但此兩者圓融而不睽隔,即孟子之「盡心知性知天」與此一體彌綸充盈之天道實德亦最終圓融而為一也。心性天是一也。孟子言「萬皆備於我矣。反身而誠,樂莫大焉。」而《中庸》固亦明言「惟天下之至誠為能盡其性」乃至參天地贊化育也。所謂天道亦只是一仁一誠,一心一性而已矣。若無道德之真實感者,則固視此為空言虛意、新說奇論矣。然而此卻為孔子仁教之所必函。曾子、孟子、《中庸》《易傳》承孔子而開展,正是孔子仁教之所本有,儒者內聖之學之所固然也。

自宋儒起,始正式肯認了孔子對於道之本統之再建之道統中的地位以及仁教之殊特,始正式認識了孔門傳承之價值,始自覺地以

\newpage\thispagestyle{empty}\addtocounter{page}{-1}\vspace*{-12mm}\begin{center}\noindent
\includegraphics[clip, trim=143pt 135pt 149pt 246pt, height=162mm]{ocr-input/image-1317.png}\end{center}

\newpage\markright{第一部 \quad 第五章 \quad 對於葉水心〈總述講學大旨〉之衡定}

\noindent 曾子、孟子之守約懶獨與盡心知性知天為道德踐履之軌道,以《中庸》、《易傳》本孔子之仁教與聖證所述之德性生命精進之方向與極致為道德踐履之弘規,自覺地建立此內聖之學(心性之學)之體系,以為吾人照體挺立之道德踐履其最高目標即是成聖,人人皆可「求則得之」,「盡其在我」,以精進其德性生命者,而成聖過程之極致即是存在地證悟澈悟性命天道之為一,以使吾人之生命成為一「仁體彌綸充盈」之「先天而天弗違,後天而奉天時」之大人生命。此確是先秦孔門傳承之重認與確立,乃極其相應而並無悖謬者。須知兩漢經生並不能認識此義,魏、晉玄學是弘揚道家之玄理,南北朝、隋、唐是吸收佛教之時期。然則宋儒興起,觀此長期之沈滯與歧出,而謂孔子之道「止於孟子而遂絕,至是而復傳」,有何不可乎?此正是極得秦漢後中國學術生命發展之脈絡者。此正是行之而著、習焉而察,能承當孔子傳統之有真實生命者之言,而葉水心竟謂其「不習而已矣」。葉氏之言始真成為「不習故不知」之妄言矣。徒因孔子師弟並非一王者之集團,並無顯赫之功業以垂於世,乃使「見形而不及道」之葉水心兩眼只看王者之事功,而不知德性之學為何物,逐視孔子傳統如無物,視子思、孟子之言為「新說奇論」矣。至周、張、二程之業績更不在其眼下矣。

「無極太極,動靜男女〔周子】;太和參兩,形氣聚散,絪感通〔張子】;有直內,無方外,不足以入堯、舜之道【程子〕:皆本於十翼,以爲此吾所有之道,非彼之道也。」「本於十翼」,即知其非杜撰也。「以為此吾所有之道,非彼之道」,此誠是吾所固有之道,乃儒家內聖之學之本質地異於佛、老者。繼承《中庸》、《易傳》而闡揚之,不亦甚佳乎?然而葉水心卻責之以

\newpage\thispagestyle{empty}\addtocounter{page}{-1}\vspace*{-12mm}\begin{center}\noindent
\includegraphics[clip, trim=173pt 166pt 134pt 231pt, height=162mm]{ocr-input/image-1321.png}\end{center}

\newpage

\noindent 「不悟十翼非孔子作,則道之本統尚晦」。夫葉氏承認〈彖〉、〈象〉爲孔子所作,而又籠統以言十翼非孔子作,已矛盾而悖謬矣,復不知〈文言〉、上下〈繫〉,正本〈彖〉、〈象〉而發揮,無一與〈彖〉(象〉相刺謬者,則十翼之言正是內聖之學之極致,縱使全非孔子所作,亦是本孔子之仁教與聖證而發揚者,此是孔子之傳統,本不限於上世王者集團之「本統」,然則汝謂「道之本統尚晦」尚有意義乎?對孔子傳統言,其本統並不晦也。對上世王者集團之本統言,汝亦知夫孔子對於道之本統之再建乎?汝曾知此中開合之分際乎?然則以「十翼非孔子」,而即謂「道之本統尚晦」,是全忽視孔子仁教之地位,只視孔子為三代之檔案家者無知之妄言耳。

《習學記言》有一條責《易傳》云:「《易》以〈彖〉釋卦,皆即其畫之剛柔逆順往來之情,以明其吉凶得失之故,無所謂無思無爲,寂然不動,不疾不行之說。予嘗患浮屠氏之學至中國,而中國之人皆以其意立言。非其學能與中國相亂,而中國之人實自相亂之。今《傳》之言《易》如此,何以責夫異端?」葉水心之愚鄙自賤無過於此。如以為後之不辨儒佛隨意立言者為「自亂」可也;如以為《易傳》之言「神化」亦是自亂,無以異於異端,則太自賤,不可救藥。《易傳》之時,佛根本未至中國,《易傳》何從而自亂?《易傳》之言是自本自根,自家德性生命之發皇,是本諸其所宗主之孔子之仁教與聖證而來之高度之靈感與精英,而乃竟謂其自亂於異端,或是自亂於異端之始作俑者,是則中國人根本不應企向於高明,根本不應有超越之靈思,只應膠著於現實,局促於淺近庸俗卑陋平凡之境以自美其名曰平實,根本不應有德性生命之精進,

\newpage\thispagestyle{empty}\addtocounter{page}{-1}\vspace*{-12mm}\begin{center}\noindent
\includegraphics[clip, trim=155pt 140pt 137pt 238pt, height=162mm]{ocr-input/image-1325.png}\end{center}

\newpage\markright{第一部 \quad 第五章 \quad 對於葉水心〈總述講學大旨〉之衡定}

\noindent 不應有理想價值精神生命之嚮往,《論語》、《孟子》、《中庸》、《易傳》皆當拉雜摧燒之,凡孔子之門下皆當送集中營而澈底改造之!其為汙賤自貶,根本不知學術文化為何物有如此,無乃太過愚鄙無知乎?尚猶侈談古之人、古之人,亦太不自量而已矣!

上〈繫〉言「易無思也,無為也。寂然不動,感而遂通天下之故。非天下之至神,其孰能與於此?」此是就耆卦卜筮之知幾言,言幾之感應如此其無思無為而自然迅速如神也。「以卜筮者尚其占。是以君子將有為也,將有行也,問焉而以言,其受命也如響。無有遠近幽深,遂知來物。非天下之至精,其孰能與於此?」此一段即是言「無思無為」段之根據。卜筮之時,自是根據卦爻之剛柔逆順之規律或耆草運算之規律以斷吉凶得失。卦爻本身只是幾筆圖畫耳,蓍草本身只是一器物耳,其本身無生無情,寧能有思有為?然通過此無生無情無思無為之死物之布算卻能預知吉凶得失之幾,所謂「遂知來物」,此其感應之不可測寧非天下之至神?此一布算之感應由於問者之精誠,此只是一精誠之感應,通過一客觀之物以驗之耳,故蓍卦之器物只是一象徵,由之以洞見真實生命感應之幾耳。真實生命與感應之幾皆是至具體而至精微者。天下之事亦只是其最初之一幾動耳。此非膠著於器迹者所能知也。不然,孔子何以言「一言可以喪邦,一言可以興邦」?此非言其感應之大乎?不然,孔子又何以言「一日克己復禮,天下歸仁焉」?此非言其感應之效乎?感應之神豈是如有形之物之來來往往有行有疾乎?有行有疾是有形之物之物理運動,而非神感神應也。故《易經》之學即是由蓍卦之布算而見到生命之真幾,故云「夫易,聖人之所以極深而研幾也。唯深也,故能通天下之志。唯幾也,故能成天下之務。唯

\newpage\thispagestyle{empty}\addtocounter{page}{-1}\vspace*{-12mm}\begin{center}\noindent
\includegraphics[clip, trim=182pt 170pt 126pt 224pt, height=162mm]{ocr-input/image-1329.png}\end{center}

\newpage

\noindent 神也,故不疾而速,不行而至。」如此言《易》,正是《易》之本義,此非淺薄之頭腦膠着於事象者所能知也。此是由質礙、物結而直透其至精無礙之超越實體也。此超越實體,在《易經》,即以生化不測之神當之,或以易簡之理當之。首先,由蓍卦之布算而悟到其無思無為知幾如神之感應,由此為象徵而悟到天道無思無為之生化,再由此而歸於主體悟到「聖人以此洗心,退藏於密,吉凶與民同患」,以及「聖人以此齋戒以神明其德」。類而通之,無論在天道之生化,或在聖心之神明,皆可以「無思無為,寂然不動,感而遂通」形容之。而此總之,即曰「寂感真幾」。故超越實體者即此「寂感真幾」之謂也。神化與易簡皆其本質之屬性。此皆由精誠之德性生命、精神生命之升進之所澈悟者。「維天之命於穆不已」,所謂天命流行之體,語其實,亦不過即此寂感真幾之靜正與沛然,寂感真幾之生化不測即易道也。易道即生道也。上〈繫〉云:「夫易廣矣大矣。以言乎遠,則不禦。以言乎邇,則靜而正。以言乎天地之間,則備矣。」此遠之「不禦」,邇之「靜正」,即「寂然不動、感而遂通」之別語也。「以言乎天地之間,則備矣」,言天地之間亦不過即是此寂感真幾之一體充盈也。此若非有精誠之德性生命之昇進,焉能悟此「天行」乎?此皆有真感於古之至德、玄德(〈舜典〉稱舜語)、一德與純德,通過《易》而一起迸發之耳。若必膠着於卦畫之剛柔逆順,而不准契悟此「天行」,以為「無所謂無思無為,寂然不動,不疾不行之說」,則「大哉乾元,萬物資始,乃統天」云云(〈乾·彖〉),「至哉坤元,萬物資生,乃順承天」云云(〈坤·彖〉),亦不在卦畫之剛柔逆順中也,亦可抹而去之也。豈惟此爲然?即「觀其所感,而天地萬物之情可見矣」

\newpage\thispagestyle{empty}\addtocounter{page}{-1}\vspace*{-12mm}\begin{center}\noindent
\includegraphics[clip, trim=150pt 147pt 140pt 230pt, height=162mm]{ocr-input/image-1333.png}\end{center}

\newpage\markright{第一部 \quad 第五章 \quad 對於葉水心〈總述講學大旨〉之衡定}

\noindent (〈咸·彖〉),「觀其所恆,而天地萬物之情可見矣」(〈恆·彖〉),「正大而天地之情可見矣」((大壯·彖〉),「觀其所聚,而天地萬物之情可見矣」((萃·彖〉),等等亦皆可抹而去之也。卦畫之剛柔逆順亦「無所謂」如此等等之說也。即其他〈彖〉〈象〉如許正大洞澈之言,亦皆可抹而去之也,剋就卦畫之剛柔逆順而觀之,亦「無所謂」如此等等也。然則葉水心所謂「《易》以〈彖〉釋卦,皆即其畫之剛柔逆順往來之情,以明其吉凶得失之故,無所謂無思無為、寂然不動、不疾不行之說」,此反動心理之鄙言豈能謂為得〈彖〉、〈象〉之實乎?漢之象數膠著於卦爻,著迹於器物,已喪失〈彖〉、〈象〉之大義久矣。不意葉水心竟復欲故意曲解〈彖〉〈象〉而全抹去之也!

其所以如此厭惡「無思無為、不疾不行」者,蓋彼以為此誇大之神話、不經之妄談,與佛老許多誇奢之談相類也。

彼(講學大旨〉最後一段下案註中有云:「昔列禦寇自言忘其身而能御風,又言至誠者入火不燔,入水不濡,以是為道大,妄矣。若浮屠之妄,則又何止此?其言大地之表,六合之外,無際無極,皆其身所親歷,足所親履,目習見而耳習聞也。以為世外瓌特廣博之論,置之可矣。今儒者乃援引《大傳》:天地絪,通畫夜之道而知,不疾而速,不行而至,子思:誠之不可揜,孟子:大而化,聖而不可知,而曰吾所有之道蓋若是也。譽之者以自同,毀之者以自異。嘻!未矣!」

夫莊、列之逍遙乘化,出之以「荒唐之言,無端崖之辭」,自有其精神生命之境界,亦自有其道家之本統。佛家亦自有其本統,皆非無謂之妄言。葉水心根本不足以知之,不足與談。《大傳》、

\newpage\thispagestyle{empty}\addtocounter{page}{-1}\vspace*{-12mm}\begin{center}\noindent
\includegraphics[clip, trim=184pt 172pt 137pt 234pt, height=162mm]{ocr-input/image-1337.png}\end{center}

\newpage

\noindent 子思(《中庸〉)、孟子之言,亦自有其德性生命之本統,非來自莊、列,尤非來自佛教。皆是自本自根之發皇。儒、釋、道三教可說皆已觸發到人類精神生命之極致,皆有其自本自根之本統,而以儒者之自德性生命入為最中正。自其皆已觸發到精神生命之深處與極處言,自其外部之風光觀之,自不無相似相通處:皆是人之生命所有事,任何人焉得而絕異?惟自其內蘊之骨干言,則各有其自本自根之本統。《大傳》、子思、孟子之所言正是「吾所有之道」,非彼之所謂道也。乃葉水心於此全無所感、全無所知。夫不知為不知,「置之可也」。今乃混而一之,一概視為妄誕。子思、孟子、《大傳》,皆與其所不知之佛、老同其妄誕,其於子思、孟子、《大傳》亦總歸於無知而已矣。復責周、張、二程不該援引子思、孟子、《大傳》之語以與佛老「校是非角勝負」。夫子思、孟子、《大傳》皆先秦所固有,乃是自本自根之發皇,此爲成之史實,正是儒者之道究極本質之所在,後人不引述而明之,將何所為?責後人之不應如此作,正示子思、孟子、《大傳》其書之應當焚燒也,其人之應當送集中營也。其污賤、狠愎、無知為何如!世間焉得有如此之反動愚鄙而可以言學哉?

《習學記言》復有一條云:「孔子〈彖〉辭無所謂太極者,不知《傳》何以稱之?自老聃為虛無之祖,然猶不敢放言,曰:無名天地之始,有名萬物之母而已。莊、列始妄為名字,不勝其多。故有太始、太素、茫昧廣遠之說。傳《易》者將以本原聖人,扶立世教,而亦為太極以駭異後學。後學鼓而從之,失其會歸,而道日以離矣!」

夫太極亦極至之理而已。葉氏承認皇極。皇極者君之極至之理

\newpage\thispagestyle{empty}\addtocounter{page}{-1}\vspace*{-12mm}\begin{center}\noindent
\includegraphics[clip, trim=167pt 150pt 134pt 237pt, height=162mm]{ocr-input/image-1341.png}\end{center}

\newpage\markright{第一部 \quad 第五章 \quad 對於葉水心〈總述講學大旨〉之衡定}

\noindent 也。豈只准言皇極,不准總人生宇宙之根本而言其極至之理耶?此正是道之究極會歸之所在,道之究竟自立自見其自己之所在,而乃云因言太極而「失其會歸、而道日以離」,何其性與人殊如此之顛倒乖戾耶?上〈繫〉云:「六爻之動,三極之道也。」六爻:初二為地,三四為人,五上為天。三極之道即天地人三才各是一極至之理也。天以健為極至。地以厚為極至,人以仁為極至。天地人之道其極一也,故總曰太極,太極者亦寂感真幾生化之源之別名耳。「易有太極,是生兩儀。兩儀生四象,四象生八卦,八卦定吉凶,吉凶生大業」,(上(繫〉),此不過引附於八卦以言其生化之不測耳。此乃本德性生命之精進而證悟人生宇宙之本源,故乾卦〈象傳〉曰:「天行健,君子以自強不息。」此正是儒者之「超越智慧」之殊異於佛老處。太極者亦此「天行」之別名耳。如此言太極有何「駭異後學」處?得毋自處於幽洞已久而畏見赫日之明耶?夫(彖傳〉所無者多矣,豈必限於(彖〉辭之所有而後可言耶?葉水心以其幽暗冥惑之生命,以處於幽谷為自得,遂視(易傳》之太極與莊列之太始太素同為「茫昧廣遠之說」,猶不及老子之言無。孔子云:「上智下愚不移」,此誠可謂絕異!(太始太素是〈易緯〉中語,《淮南子》尤喜拉雜堆積。莊子不言太始太素。《列子》是偽書。葉氏不負責任混抹一氣耳。)

葉水心以為周、張、二程等「其啟教後學,於子思、孟子之新說奇論,皆特發明之。大抵欲抑浮屠之鋒銳,而示吾所有之道若此。〔……】不知夷狄之學本與中國異。而徒以新說奇論闢之,則子思、孟子之失遂彰。」案:《中庸》、《大學》本為《禮記》中之兩篇,北宋諸儒即已注意而詳講之。至朱子正式予以集註,與

\newpage\thispagestyle{empty}\addtocounter{page}{-1}\vspace*{-12mm}\begin{center}\noindent
\includegraphics[clip, trim=172pt 167pt 138pt 231pt, height=162mm]{ocr-input/image-1345.png}\end{center}

\newpage

\noindent 《論語》《孟子〉合併而為四書。宋以前是周公、孔子並稱,以五經為教本。至宋,則五經外復約之以四書,以四書為基本教本,遂進而孔、孟並稱。是以特彰四書者,以孔子為中心,以孔子傳統為本者也。四書之義理教訓固易凸顯儒家之內在精神,而亦固為「吾所有之道」本「若此」也。宋儒之能注意及此而「特發明之」,固可說由於佛教之刺激,然亦確是「吾所有之道」本「若此」,並非有所曲解比附,專欲取之以與佛教糾纏也。而其為吾所有之道本若此亦自有其永恆之價值,有其自身自足獨立之價值,乃人人由之可以決定其生命之基本方向,挺立其自己,以完成其精神生命者。其足以發生別異而抵禦佛教之作用以維持其為中國靈魂之主幹,為中華民族命脈之所在,乃其本身之足以自立之光輝。周、張、二程「於子思、孟子之新說奇論〔視為「新說奇論」根本荒謬無知】,皆特發明之」,正足以見子思、孟子之「得」,亦是四書之得,而葉水心竟謂「子思、孟子之失遂彰」!此誠何心肝哉?蓋彼以為子思(《中庸》)之「誠之不可揜」,孟子之「大而化,聖而不可知」,乃至全部性命天道之義理,與佛老之誇奢同屬茫昧冥惑之論,皆非上世本統之所有,此則徒啟毀譽同異之迷離,「譽之者以自同」固非,即「毀之者以自異」亦未見其究能異否也。此則不務其本,而徒校角勝負於其迷離徜恍之末,而迷離徜恍之茫昧虛誕正是子思、孟子之「新說奇論」之所具,故周、張、二程之援引之以「抑浮屠之鋒銳」,正足以彰顯「子思、孟子之失」也。此爲葉水心反動心理所成之謬論之中心,故必併曾子、子思、孟子、《易傳》甚至連孔子一起抹去之而後快!此其愚悍狂悖為何如!

中國自兩晉正式接觸佛教後,經過南北朝以至隋、唐,此七八

\newpage\thispagestyle{empty}\addtocounter{page}{-1}\vspace*{-12mm}\begin{center}\noindent
\includegraphics[clip, trim=165pt 148pt 134pt 239pt, height=162mm]{ocr-input/image-1349.png}\end{center}

\newpage\markright{第一部 \quad 第五章 \quad 對於葉水心〈總述講學大旨〉之衡定}

\noindent 百年之長期歷史幾將佛教全部吸收於中國。不但吸收之,且消化之而能自己開宗焉。如天臺、華嚴、禪,皆是經過消化後,中國人自所開立也。此一大教,雖發之於印度,然其對於人生問題究有極深遠之觀察,在人類之精神生命上足以決定一基本之方向與態度,故其影響如此之廣,中於人心者如此其深,其足以吸引中國聰明才智之士折而從之亦並非偶然。此種問題,乃今之所謂宗教問題。凡宗教真理皆有其廣被性與普遍性,(雖是具體的普遍)。此未可純以「神道設教以安撫愚夫愚婦」視之者。有如此長期之吸收與浸潤,有如許聰明才智之士之折從,而若弘揚聖人之道者不能予以正視,則是自己之愚陋,是此人根本無知於精神生命之基本方向之決定乃是人生之一基本問題。此非如葉水心所謂「蓋禁令不立而然」,亦非如其所謂「聖賢在上,猶反手」之易易。中國人之「折而從彼」亦非只如其所謂之「特中國好異者」之趨時,此豈是一時「好異」之新鮮事耶?葉水心說之如此輕鬆,如此無實感,正足以見其無視於精神生命之基本方向問題之重大,且亦無知於人類精神與學術文化發展之道路。周、張、程、朱等正能正視此問題,故轉而積極弘揚孔子之傳統。雖其對於佛教亦無甚深之鑽研,然而大界限則甚清楚,立場緊嚴而甚堅定,純從立以為破者也。故能有積極之建樹,以開六七百年之傳統,造成儒家聖道之復興。孔子之傳統乃是有承於三代而又有進於三代者。其所以有進於三代者,正在孔子之仁教已接觸到人生之基本處,足以予中華民族之精神生命決定一基本方向與態度。其所形成之傳統,曾子、子思、孟子、《易傳》之所發皇者,正是自本自根而將此方向與態度發展至其極深澈之極致者。大抵人類歷史之發展,其初只是就現實生活而有本能的推移,

\newpage\thispagestyle{empty}\addtocounter{page}{-1}\vspace*{-12mm}\begin{center}\noindent
\includegraphics[clip, trim=156pt 159pt 136pt 230pt, height=162mm]{ocr-input/image-1353.png}\end{center}

\newpage

\noindent 對於超越者如天、神之類,亦有一本能的神往,此即成為原始之綜和構造。在此時期,人為現實所限,人之生活態度易趨一致,故每一部落氏族易有其發自於現實與本能之共同風尚。此種膠固原始而又持久,故易造成每一民族之傳統。推移久,對於生命之感觸既深,故由本能而進至自覺,則人之精神生命之基本方向遂得有根本之決定與澈底之透出。此則雖有其歷史傳統為其背景、為之制約,然就其於精神生命之方向上有澈底之透出與根本之決定言,則亦常有其普遍性與籠罩性,蓋以其已抓到生命之本質故也。故此種方向之決定常是發自於內在生命之本質的決定,而非只是一種權現,亦非只是發自於本能之現實推移之可化而可移轉者,故雖有歷史傳統為其背景、為之制約,然常不為現實所限,而一透即透至其極者,即,澈至其「普遍性之自己」。然此種普遍性又不是邏輯、數學、科學所表現者之抽象的普遍性,而是具體的普遍,是在一聖者之生命中透顯,是在傳統背景之制約中透顯。故有耶穌之形態,有釋迦之形態,而在中國,則即是孔子之形態。有傳統背景為其制約,故必與此傳統中所遺留之禮樂制度,風俗習慣相適應、相協調、相諧一,而維護並形成此一民族之具體文化;而同時復於精神生命之方向上澈至其普遍性,故有其永恆性與真理性,而足以提攜、調整並創造其文化,使之為一常新而有生命有價值之文化。傳統背景是此文化之特殊性,而精神生命之方向則是其普遍性。就此精神生命之方向言,其普遍性永是具體的普遍,永是在傳統背景之制約並繼續此傳統中呈現,永是在此傳統所凝聚之文化生命流中層出不窮而相應此文化生命流之個人創造生命中呈現。此一普遍而又特殊之奇詭現象正是每一大教所必須具備者,亦是某一大教之所

\newpage\thispagestyle{empty}\addtocounter{page}{-1}\vspace*{-12mm}\begin{center}\noindent
\includegraphics[clip, trim=170pt 142pt 123pt 241pt, height=162mm]{ocr-input/image-1357.png}\end{center}

\newpage\markright{第一部 \quad 第五章 \quad 對於葉水心〈總述講學大旨〉之衡定}

\noindent 以能為某民族之主流主干之所必備者。不具備此現象,則只能居於副助之地位。基督教之於西方,儒教之於中國,皆是居於主流主幹之地位。佛教在印度未能取得此地位,亦如道家之在中國,而其雖爲中國所吸收,亦卻是必然要居於副助之地位。此亦正因孔子之仁教於精神生命之方向上已發皇至其極致,徹底透至其極頂之故。但亦須賴有個人創造生命之層出不窮以繼承此慧命。此則曾子、子思、孟子、《易傳》乃其直接繼承者,周、張、程朱等所造之六七百年之傳統乃是其復興而再度弘揚者。如此方能立住其主流主幹之地位。中國文化發展至自覺地決定其精神生命之方向時,則言道者即須提升至此層上而言之,不能再回於原始之綜和構造而以原始之本能推移為標準、為道之本統,停於此而忽抹一切,亦不能不正視其他決定精神生命方向之大教而相觀摩、相資益,且相制限、相善成,以期各止於至善。如是能順成其他,而又能立住自己之主流地位而籠罩之。若如葉適所說,則中國只應反至三代之原始境地,中華民族只應停於現實本能之推移。若誠如此,則中國早已僵枯,與埃及,與巴比倫無以異,而所謂上世之本統者當亦早為佛教所代替矣。于以見其輕忽孔子,妄肆譏議曾子、子思、孟子、《易傳》以及周、張、程、朱等,乃全是無知之談,不知人類精神與學術文化發展之道路,不知精神生命之方向之決定乃是一重大之問題,乃是文化創造之動力。徒以其淺躁之心靈遂視此根本問題為茫昧冥惑者。是其自己之茫昧而視真理為茫昧,自己之冥惑而視光明爲冥惑也。

不獨對於佛教須正視而有以籠罩之。對於道家亦須正視而有以籠罩之。道家固非外來者,乃是自家文化生命中之駢枝,亦於精神

\newpage\thispagestyle{empty}\addtocounter{page}{-1}\vspace*{-12mm}\begin{center}\noindent
\includegraphics[clip, trim=164pt 160pt 130pt 226pt, height=162mm]{ocr-input/image-1361.png}\end{center}

\newpage

\noindent 生命之方向上有所決定。若不能予以消化、制限,而順成之,儒家即不足以完成其為主流主幹之使命。所謂主流主干者非謂只我一家之意也。要在能己立而立人,已達而達人,不遏不禁,能不失自己之統而亦能順成他人者也。葉適之愚陋何足以知此?而猶侈談道之本統耶?魏、晉之和會孔、老,雖不足以盡孔門傳統之實蘊,要於精神生命之最高方向上,足以以道家之玄理豁醒孔門之實蘊者,焉可概視為無稽而擯之,而復猥自貶抑,自甘下縮,以斥絕《中庸》《易傳》耶?善乎李習之之言曰:「性命之書雖存,學者莫能明。是故皆入於莊、列、老、釋。不知者謂夫子之徒,不足以窮性命之道。信之者皆是也。」(〈復性書上〉)今因長期之歧出起而繼明之,豈不人間之盛事也哉?不加扶助贊嘆斯可矣。如見到有不足,則補充之而已矣。如以為非己性之所宜,非己智之所及,非己才之所能,則置而讓諸他人亦可矣。今獨不知自處,反而妄施詆毀,且毀及曾子子思、孟子、《易傳》,不亦太自污賤乎?誠不知其何所居心也。無已,亦只因無知而已矣!

葉氏之謬論本不值多辯,而吾之所以不嫌辭繁而詳辯之者,蓋欲藉此以明三代道之本統之何所是與孔子對道之本統再建之重要以及孔門傳統發展之經脈與其開合貫通之使命,並對於言事功者進一解,徹底疏通此問題之分際,以為可以相補相成,而不可形成非此即彼之相毀。此雖述古,而於今日之時代亦有其針砭之用也。非徒計較於葉氏一人而已也。葉氏自有其長,而其詆毀孔子傳統,要之則無一是處。

黃宗羲曰:「以余論之,水心『異識超曠,不假梯級,謂洙泗所講,前世帝王之典藉賴以存,開物成務之倫紀賴以著。《易》

\newpage\thispagestyle{empty}\addtocounter{page}{-1}\vspace*{-12mm}\begin{center}\noindent
\includegraphics[clip, trim=164pt 143pt 137pt 248pt, height=162mm]{ocr-input/image-1365.png}\end{center}

\newpage\markright{第一部 \quad 第五章 \quad 對於葉水心〈總述講學大旨〉之衡定}

\noindent 〈彖〉、〈象〉,夫子親筆也,十翼則訛矣。《詩》、《書》,義理所聚也,《中庸》、《大學》則後矣。曾子不在四科之目,曰參也魯。以孟子能嗣孔子,未為過也,舍孔子而宗孟,則於本統離矣。』其意欲廢後儒之浮論。所言不無過高,以言乎疵則有之。若云其概無所聞,則亦墮於浮論矣」。(《宋元學案》卷五十四,(水心學案上〉,宗羲案語)

宗羲案語中自水心「異識超曠」至「則於本統離矣」,皆水心弟子孫之宏序《習學記言》之語。(見〈水心學案下〉「水心門人」項)孫之宏盛稱其師,固其宜也。宗羲不深察,籠統引之,以作定評。殊不知孫之宏所盛稱其師之諸點,除「前世帝王」兩句外,其餘皆葉氏無知謬論之所在,此正其可恥之污點,焉得謂爲「異識超曠,不假梯級」?此正是淺躁者卑下之妄論,焉得謂為「所言不無過高,以言乎疵則有之」?此有何「過高」之可言?又豈只是「疵」而已哉?若就此而言,謂其「概無所聞」,不為「浮論」也。夫「無所聞」有何傷?不知為不知,則善矣。無知而妄肆詆毀,則惡劣矣!此正「文士」之陋習,徒憑偶有所見之一點而隨意揮洒耳。學無開承,非無故也。經制事功,通政體,此是一好題目,亦是中國文化發展中之一大癥結、一大問題,而乃言之者每以急切之心理,不能深入此問題之關鍵,以作積極之開承與建樹,徒因急切而流於反動,只成為詬詆他人、無的放矢的藉口,混打亂罵,不明分際,不識統類,結果全無經制,一無事功,於道德宗教、政道治道、科學知識,全不著邊,其所成者結果只是詞章考據而已。此亦文士之過也。平心論之,有那一點能及程、朱、陸、王乎?而葉水心者反因而上溯堯、舜、三代之本統而毀及曾子、子

\newpage\thispagestyle{empty}\addtocounter{page}{-1}\vspace*{-12mm}\begin{center}\noindent
\includegraphics[clip, trim=160pt 144pt 136pt 242pt, height=162mm]{ocr-input/image-1369.png}\end{center}

\newpage

\noindent 思、孟子與《易傳〉,甚至連孔子亦輕忽,則豈只小疵而已哉?此種「超曠」之「異識」、「過高」之言論,實只是無真感實感而只揀好聽的說而已矣。此則並其所把握之一點而亦不能使之有價值也。其所把握之一點,即在其皇極論。此與本書無關,故略而不論。

附識:以上所論之(總述講學大旨〉以及所引之(習學記言》俱見《宋元學案》卷五十四〈水心學案上〉。案:《習學記言》共五十卷,不存《永嘉叢書》中。據査光緒十年有江陰刊本,《敬鄉樓叢書》第一輯中亦收之。惜乎不流行,難尋覓,不得窺其全。此人雖不見本源,要有雜識,亦須註意也。

又附識:《朱文公文集》卷第五十六,書,問答,〈答葉正則〉四書之第四書云:

\begin{quotation}\kaishu 向來相見之日甚淺,而荷相與之意甚深。中間寓舍並坐移
晷,觀左右之意若欲有所言者,而竟囁嚅不能出口。前後書
疏往來,雖復少見鋒穎,而亦未能彼此傾倒,以求實是之
歸。但見士子傳誦所著書及答問書尺,類多籠罩包藏之語。
不唯他人所不解,意者左右亦自未能曉然於心而無所疑也。
世衰道微,以學為諱,上下相徇,識見議論日益卑下。彼既
不足言矣,而吾黨之為學者,又皆草率苟簡,未曾略識道理
規模、工夫次第,便以己見搏量湊合,撰出一般說話,高自
標置,下視古人。及考其實,則全是含胡影響之言,不敢分
明道著實處。竊料其心豈無所疑?只是已作如此聲勢,不可
復謂有所不知,遂不免一向自瞞,強作撐柱,且要如此鶻突\end{quotation}

\newpage\thispagestyle{empty}\addtocounter{page}{-1}\vspace*{-12mm}\begin{center}\noindent
\includegraphics[clip, trim=167pt 122pt 128pt 257pt, height=162mm]{ocr-input/image-1373.png}\end{center}

\newpage\markright{第一部 \quad 第五章 \quad 對於葉水心〈總述講學大旨〉之衡定}

\begin{quotation}\kaishu 將去,究竟成就得何事業?未論後世,只今日旁觀,便須有
人識破。未論他人,只自家方寸如何得安穩耶?

如來書所謂「在荊州無事,看得佛書,乃知世外 奇之說,
本不能與治道相亂,所以參雜辨爭,亦是讀者不深考爾。」
此殊可駭。不謂正則乃作如此語話也!中間得君舉書,亦深
以講究辨切為不然。此蓋無他,只是自家不曾見得親切端
的,不容有毫釐之差處,故作此見耳。

欲得會面相與劇談,庶幾彼此盡情吐露,尋一個是處。大家
講究到底,大開眼看觑,大開口說話,分明去取,直截剖
判,不須得如此遮前掩後,似說不說,做三日新婦子模樣,
不亦快哉!

孟子自許雖行霸王之事,而不動其心。究其根源,乃只在識
破詖淫邪遁四種病處。今之學者,不唯不能識此,而其所做
家計窠窟,乃反在此四種病中,便欲將此見識判斷古今,議
論聖賢,豈不誤哉?相望千里,死亡無日。因書,聊復一
言,不審明者以為如何?然勿示人,恐又起鬧,無益而有損
也。\end{quotation}

\noindent 案:葉適(字正則)對朱子為晚輩。朱子在當時為泰山北斗,又自有其威嚴,故葉氏晤見時,「若欲有所言者,而竟囁嚅不能出口。」葉氏早有其一套不同之想法。朱子覆此書時已在晚年,故有「死亡無日」之語。朱子此書所責斥者即是〈講學大旨〉、《習學記言〉中所說之見之模樣也。葉氏早已有此端緒矣・經朱子之責斥,不但未能稍有轉變,且益增其反動。故後來遂乾脆寫成〈總述

\newpage\thispagestyle{empty}\addtocounter{page}{-1}\vspace*{-12mm}\begin{center}\noindent
\includegraphics[clip, trim=173pt 158pt 152pt 247pt, height=162mm]{ocr-input/image-1377.png}\end{center}

\newpage

\noindent 講學大旨〉及《習學記言〉,悍然與孔子傳統為敵,對曾子、子思、孟子、(中庸》、《易傳〉加以詬詆,並對孔子亦不滿,此即朱子所謂「高自標置,下視古人」也。朱子此書所引葉氏之書語與〈講學大旨〉最後一段之案註中語意義相同。此朱子所認為「殊可駭,不謂正則乃作如此語話」,而葉氏後來竟絲毫未有轉動,且言之更為勇悍耳。夫人不能上企高明,必下趨反動而走邪。上智下愚不移。不移可也,反重走邪則大不可也。此不但見之于今日,自古已然矣,特今日為益甚耳。

\newpage\thispagestyle{empty}\addtocounter{page}{-1}\vspace*{-12mm}\begin{center}\noindent
\includegraphics[clip, trim=173pt 415pt 131pt 99pt, height=162mm]{ocr-input/image-1381.png}\end{center}
\newpage

\markright{}

\part{分論一\newline 濂溪與橫渠}

\newpage\thispagestyle{empty}

\newpage\markright{}

\chapter{周濂溪對於道體之體悟}

\section*{引言}\addcontentsline{toc}{section}{引言}

本書詮表宋明心性之學,從北宋起,直接斷自周濂溪。濂溪前,固有其文化生命上之先驅,如胡安定、孫泰山、石徂徠等皆為宋學興起第一階段之人物。然此書非哲學史,故略而不論。

吾今只明言,中國文化生命發展至北宋,已屆弘揚儒家內聖之學之時,此為歷史運會之自然地所迫至者。因是歷史運會之自然地所迫至,故濂溪之學,雖無師承,而心態相應,出語即合。當運會不至,面對典籍,視若無睹,即有講論,而睽隔重重。兩漢經生固無論矣,王弼何嘗無玄思?然其心態非儒家型,故雖十分著力於《周易》。而於《易傳》之窮神知化,究不相應。至唐李習之雖發憤弘揚《中庸》與《易傳》,然學非其時,孤而無應,而其本人亦不成熟,且亦不必真有相應之心態,即略有相應,而學力不足,故不能弘通,音響輒歇。運會不至故也。至乎北宋,運會成熟,心態相應,一拍即合,故濂溪之面對典籍,「默契道妙」(吳草廬語,見《宋元學案·濂溪學案下》),一若全不費力焉。

\newpage\thispagestyle{empty}\addtocounter{page}{-1}\vspace*{-12mm}\begin{center}\noindent
\includegraphics[clip, trim=161pt 143pt 131pt 251pt, height=162mm]{ocr-input/image-1394.png}\end{center}

\newpage

黄宗羲云:

\begin{quotation}\kaishu 周子之學以誠為本。從寂然不動處,握誠之本,故曰:主靜
立人極。本立而道生,千變萬化,皆從此出。化吉凶悔吝之
途,而反覆其不善之動,是主靜真得力處。靜妙於動,動即
是靜。無動無靜神也,一之至也,天之道也。千載不傳之祕
固在是矣。((宋元學案·濂溪學案下〉,宗羲案語)\end{quotation}

\noindent 宗羲此案語極為中肯。所謂「千載不傳之祕固在是矣」,並非真有若何不傳之祕密,至今始傳,實乃運會所至,心態相應,睽隔不通者至今始通,茫然不解者至今始解耳。亦非真有難解者在,斯理平常,特真有實感者,生命相應,故契接順適耳。宗羲其他諸語乃周子學之內容。吾今先從外部述其綱領,以期逐步逼近何以有此內容。所謂心態相應、生命相應者,實即道德意識之豁醒。道德意識中函有道德主體之挺立,德性動源之開發,德性人格(德性之體現者)之極致,而周子之默契此義,則自《中庸》(後半部)與(易傳〉入。《中庸》易傳〉者是先秦儒家繼承《論語》、《孟子〉而來之後期之充其極之發展。所謂「充其極」,是通過孔子踐仁以知天,孟子盡心知性以知天,而由仁與性以通澈「於穆不已」之天命,是則天道天命與仁、性打成一片,貫通而爲一,此則吾亦名曰天道性命相貫通,故道德主體頓時即須普而為絕對之大主,非只主宰吾人之生命,實亦主宰宇宙之生命,故必涵蓋乾坤,妙萬物而為言,遂亦必有對於天道天命之澈悟,此若以今語言之,即由道德的主體而透至其形而上的與宇宙論的意義。若是表面觀之,此儼

\newpage\thispagestyle{empty}\addtocounter{page}{-1}\vspace*{-12mm}\begin{center}\noindent
\includegraphics[clip, trim=164pt 126pt 129pt 254pt, height=162mm]{ocr-input/image-1398.png}\end{center}

\newpage\markright{第二部 \quad 分論一 \quad 第一章 \quad 周濂溪對於道體之體悟}

\noindent 若為空頭的外在的宇宙論之興趣,而特為某種現實感特強者所不喜,亦為囿於道德域、人文界,而未能通透澈至其極者所深厭。實則此種不喜與深厭中之割截非先秦儒家一脈相承開朗無礙之智慧之全貌,亦非北宋諸儒體悟天道天命之實義。是以若以西方哲學康德前之外在的非批判的形上學視之誤也,名之曰宇宙論中心者亦誤也,囿於人文、切感於現實,而不准涉足乎此者亦非儒家道德意識中道德主體之涵量之本義,此為道德之局限,而非儒家開朗無礙之道德智慧也。開朗無礙之道德智慧必透至此而始充其極,必先充其極始能得圓滿。圓滿者聖人踐仁知天圓教之境也。此圓教之境,《中庸》、《易傳》盛發之,北宋諸儒即契接此境而立言。故其澈悟天道天命而有形上學的意義與宇宙論的意義,是圓教義,非是空頭的外在的形上學,亦非泛宇宙論中心也。道德主體如此,則就德性動源之開發言,此道德主體作為絕對之大主者,即是道德的創造(亦即真實創造)之真幾。內聖之學,心性之學,惟是開關此道德創造之真幾以為吾人之大主,亦且為宇宙之大主。而理不空言,道不虛懸,必以德性人格以實之。德性人格者即體現此大主、體現此創造真幾之謂也。體現之極致即為聖。圓教者亦相應聖人境界而言也。故儒家道德哲學之有形上的意義與宇宙論的意義必依踐仁知天之圓教而理解始不誤,一離乎此,則迷茫而亂矣。

明乎此,則黃宗羲所簡述之周子學之內容即可得而解矣。而此簡述之內容俱見於周子之《通書》。吾以下先明《通書》,次明〈太極圖說〉。而宋、明儒六七百年之發展以及學派分立之關鍵(亦可說症結)亦於此開端而得展示。

\newpage\thispagestyle{empty}\addtocounter{page}{-1}\vspace*{-12mm}\begin{center}\noindent
\includegraphics[clip, trim=152pt 196pt 148pt 248pt, height=162mm]{ocr-input/image-1402.png}\end{center}

\newpage

\section{濂溪《通書》(《易通》)選章疏解}

\subsection{以誠體合釋乾道}

〈誠上第一〉:

\begin{quotation}\kaishu 誠者聖人之本。「大哉乾元,萬物資始。」誠之源也。「乾
道變化,各正性命。」誠斯立焉。純粹至善者也。故曰:
「一陰一陽之謂道:繼之者善也,成之者性也。」「元
亨」,誠之通。「利貞」,誠之復。大哉易也,性命之源
乎?\end{quotation}

\noindent 案:此為《通書》之第一章。此是以《中庸》之「誠」合釋〈易傳〉之〈乾彖〉。就《中庸〉言,「天地之道可一言而盡也。其爲物不貳,則其生物不測。」「不貳」即專精純一之意。此即誠也。「誠」本真實無妄意,為形容名詞,其所指目之實體即天道。天道以「生物不測」為內容,即以創生為內容。此作為實體之天道,即以誠代之亦無不可。故誠亦可轉為實體字,而曰「誠體」。誠體者即以誠為體也。誠即是體,此即是本然、自然,而當然之天道。故《中庸〉復曰:「誠者天之道也,誠之者人之道也。」「天之道」即自然而本然如此之道。誠體為創造之真幾、為真實生命,人人本有,天地之道亦只如此。惟人如不能直下體現此誠體,而須修養工夫以復之,則即屬於「人之道」。而經由修養工夫以復之,即是

\newpage\thispagestyle{empty}\addtocounter{page}{-1}\vspace*{-12mm}\begin{center}\noindent
\includegraphics[clip, trim=180pt 128pt 128pt 262pt, height=162mm]{ocr-input/image-1406.png}\end{center}

\newpage\markright{第二部 \quad 分論一 \quad 第一章 \quad 周濂溪對於道體之體悟}

\noindent 「誠之」。天之道以誠為體,人之道以誠為工夫。故又曰:「自誠明謂之性,自明誠謂之教。誠則明矣,明則誠矣。」又曰:「惟天下之至誠爲能盡其性〔……]」,又曰:「誠則形,形則著,著則明,明則動,動則變,變則化;惟天下之至誠為能化。」此皆明示誠為道德創造之真幾。形著、明動變化即誠於中形於外而起創生、改變、轉化之作用也。惟至誠為能盡性,即以誠體之內容滲透於性體,此即是性,離此別無性也。性與天道皆只是一誠體。性與天道是形式地說、客觀地說,而誠則更是內容地說主觀地說。《中庸》如此,則以《中庸》之「誠」說《易傳》之〈乾彖〉,可謂天衣無縫,自然合拍,此為儒家形上智慧之同一思路也。「千載不傳之祕」,濂溪劈頭即把握住矣。〈乾彖〉、〈繫辭傳〉之語,用一誠字點撥,實義朗現,不煩多言也。「乾道變化」不過只是一誠體之流行。此為儒者最根源之智慧,握住此義,則綱領定矣。焉有所謂由佛老而來者乎?綱領定,佛老豈無相出入之義?豈無可資為用之詞語?人間本相通,本有同屬者,豈能事事絕異?綱領不定,其他無論如何美妙,亦非儒者之智慧。心態不相應故也。試觀王弼之解〈乾彖〉即可知矣。參看《才性與玄理》第四章。

所謂「乾道變化不過只是一誠體流行者」,誠之形著明動變化即是誠體之流行,「為物不貳,生物不測」亦是誠體之流行。總起來,《中庸〉言:「誠者物之終始,不誠無物。」一切事物皆由誠成始而成終。由誠成始而成終,即是誠體貫澈於其中而成全之。在此成始成終之過程中,物得以成其為物,成其為一具體而真實之存在。設將此誠體撤銷,則物即不成其為物,不成其為存在,而歸於

\newpage\thispagestyle{empty}\addtocounter{page}{-1}\vspace*{-12mm}\begin{center}\noindent
\includegraphics[clip, trim=152pt 143pt 140pt 242pt, height=162mm]{ocr-input/image-1410.png}\end{center}

\newpage

\noindent 虛無。此即所謂「不誠無物」。無物即無終始也。自實體言,為誠體流行;自軌跡言,為終始過程;自成果言,為事事物物。吾人可規定物為一特殊的終始過程,亦可規定為在一特殊軌跡中表現的誠體流行之特殊滿足(完成)。此種規定名曰直貫型的宇宙論的規定,因「物之終始」一語即是一宇宙論的陳述。此種規定非是方法學的邏輯規定,亦非是認識論的認知規定,而乃是自實體成就上之實現的規定。

將此終始過程用於〈乾彖〉,「乾道變化,各正性命」即是一誠體流行之終始過程,就乾卦之卦辭綜言之,「元亨利貞」四字亦是一誠體流行之終始過程。(此元亨利貞四字就《易傳》所理解者言,不就原始之卜辭言。)誠體者即乾元也。創造的真幾即是元。此元字是價值觀念,不是時間觀念。有創造真幾處即是元,有真實生命處即是元。創造真幾即是體、即是主,它創造一切而不依他,故是元。真實生命是自我作主的生命,不是依待的、機械的生物生命、自然生命,故它自體獨立卽是元。《易傳》即以此創造真幾、真實生命名為乾。乾即是元,故曰乾元。乾者天之德,其義為健。故乾元者即創造性之自己也。亦得名曰創造原則。道從此說,離此無可言道。乾道即天道。天道者道德的創造真幾之道也。創造即是天,保聚即是地。此《易傳》之所以乾元坤元並建也。乾元為綱領,坤元為隸屬。乾道變化不過是一誠體之流行。自乾元之為萬物所資以為始言,濂溪即名曰「誠之源」,言誠體之發用由此為源頭也。自乾元之成始成終而創生(實現)萬物言,濂溪即由此說「誠斯立」,言誠體之所以為誠體,誠體之自建其自己,即由其成始成終而見也。設只有發源之成始,而不能成終,此為誠之偶爾一顯,

\newpage\thispagestyle{empty}\addtocounter{page}{-1}\vspace*{-12mm}\begin{center}\noindent
\includegraphics[clip, trim=177pt 128pt 127pt 257pt, height=162mm]{ocr-input/image-1414.png}\end{center}

\newpage\markright{第二部 \quad 分論一 \quad 第一章 \quad 周濂溪對於道體之體悟}

\noindent 此所謂龍頭蛇尾、有始無終,此即誠體之不能自建(自立)。誠體不能自建,誠體即為偶然,而無必然性。「誠斯立焉」一語可謂精矣。用於元亨利貞,於元亨處,濂溪說「誠之通」。於利貞處,濂溪說「誠之復」。此「復」字,濂溪下的亦妙。「復」即由「立」而言。此復字非直接是「克己復禮」之復。乃是由自建自立而自見其自己,自見其自己即復其自己也。有元(普通所謂有好的開始),即有亨。亨者內通也。即生機之不滯。故於元亨說「誠之通」。通而有定向者謂「利」,利而有終成者謂「貞」。貞者定也成也。故於利貞說「誠之復」。利貞即「各正性命」之義也。有亨而無貞,亦如有始而無終。無終即無成。無成則雖有元亨,亦虛脫,而誠體之流逐成一虛無流,而亦不成其為誠體,此為誠體之流逝。誠體之流逝其自己即是誠體之不能復,亦即不能立也。

乾道之元亨利貞,「誠之源」以至「誠斯立」,此為誠體之終始貫徹,無間朗現,故濂溪即稱此為「純粹至善」。此純粹至善之讚美,一、剋就誠體自身說,此為體性學地說;二、剋就其終始過程說,此為宇宙論地說,而此第二義尤重。綜此兩義,吾人可說,此「純粹至善」之善是體性學地規定的善,尤其是宇宙論地規定的善,乃直就誠體之流行而言也。故濂溪於此下即引〈繫辭傳〉「—陰一陽之謂道,繼之者善也,成之者性也」三語以明之。濂溪對此三語並未詳細分疏。至伊川朱子始有詳細分疏,而至伊川、朱子之分疏,亦始有問題出現。但濂溪此籠統的稱引,其意亦可得略見。吾今亦可先籠統地提示其意如此:「一陰一陽之謂道」,當然不必是說陰陽即是道,但乾道或誠體之具體流行(具體的終始過程)亦不能離乎陰陽,不能不憑藉陰陽以表現,而「一陰一陽」即

\newpage\thispagestyle{empty}\addtocounter{page}{-1}\vspace*{-12mm}\begin{center}\noindent
\includegraphics[clip, trim=167pt 141pt 122pt 239pt, height=162mm]{ocr-input/image-1418.png}\end{center}

\newpage

\noindent 是陰陽之氣之無間暢通,而道即因之而無間朗現,故「一陰一陽之謂道」,其語意當是就道之所資以顯其自己者而說道,並非說陰陽即是道也,甚至亦並非說「一陰一陽」即是道也。故此語非界定語,乃藉顯語。乾道誠體藉資陰陽之無間暢通而得有一具體之終始過程。故前文云:自實體言,為誠體流行;自軌跡言,為終始過程。落於陰陽之氣上,始得有終始過程,始得有軌跡可言。在乾道變化中,於元亨處,所謂「誠之源」處,即見有陽之申,於利貞處,所謂「誠斯立」處,即見有陰之聚(陰之屈)。總之在「乾道變化」一語中,並非分解地抽象地單說乾道誠體之自身,而是說其具體之表現。而一在具體表現中,若分解地明之,即不能不有異質之成分在,落實言之,即不能不有「氣」之觀念在,而氣並非即是道也,若渾淪圓融地言之,則道器、理氣、體用一起滾,說道不離器可,說器即是道亦可,而此「即是」非界定之「即是」,乃是圓融之「即是」。圓融的「即是」與分解地說中之是與不是並非同意也。然若謂圓融地說中不隱含有分解的異質之成分在,則亦不可通。蓋若純是同質,則亦無圓融可言。「乾道變化,各正性命」,此雖是說乾元,亦未嘗不含有坤元在其中,故至坤卦,即提出來而單講坤元矣。濂溪以後,明道喜作圓融的表示,而伊川、朱子則喜分解的表示。在分解表示中,分理氣為二,不礙其圓融地為一。明儒羅整庵、劉蕺山、黃宗羲等皆不喜此理氣之分,就此難朱子,其實皆未懂此中問題之症結。朱子系統中之問題以及由此問題而成之學派之分立,並不在此分解表示的理氣為二,而實在經過此分解後,對於此理字(道體、誠體)的內容.(涵義)之理解,進一步又實在心理為二,而心理為二與理氣為二並非同意語也,蓋心並不即

\newpage\thispagestyle{empty}\addtocounter{page}{-1}\vspace*{-12mm}\begin{center}\noindent
\includegraphics[clip, trim=163pt 127pt 133pt 253pt, height=162mm]{ocr-input/image-1422.png}\end{center}

\newpage\markright{第二部 \quad 分論一 \quad 第一章 \quad 周濂溪對於道體之體悟}

\noindent 等於氣故。此是後話,將在以後各章中陸續明之。吾今只就濂溪之籠統的稱引,先籠統地明「一陰一陽之謂道」為藉顯語,非界定語。此乃表示道為一誠體之流行,為一有軌跡之終始過程。道是一道德的創造之真幾,不能不有具體的流行,不能不有其終始的過程。此即通過其成始成終之創造生化而無間歇,而不流逝(虛脫),而了解道。此即「一陰一陽之謂道」之恰當的意義。下句云:「繼之者善也」,意即:能繼續此生化無間之道而不使它斷滅或自我而止者即是「善」。此亦是對於道德的善之宇宙論的規定、動態的規定、立體的直貫型的規定。「成之者性也」,此句是就個體說,即落在個體上而能完成(成就)此道者乃是個體之「性」也。此性即是以人人本有之誠體以為性,並非離此誠體別有其性。個體有此創造真幾之性,故能完成此道於其自己之生命中。此亦《中庸》「率性之謂道」之意(朱子解〈繫辭傳〉此三語不諦。詳見下明道章天道篇及生之謂性篇)。

濂溪於以誠體流行合釋乾卦辭「元亨利貞」及〈彖傳〉「大哉乾元,萬物資始」,「乾道變化,各正性命」,及(繫辭傳〉之三語後,即總贊之曰:「大哉易也,性命之源乎?」此言《易》之一書乃真參透性命之根源者。濂溪於性命二字亦未曾予以詳細分疏。然古人言多順經典語脈有一大體共同之理解,並有共同之默許,雖不必言,而亦可喻者。性,根據《中庸》、《孟子》,不會有異解,就吾人個體生命說,必即當下直指內在道德性之性說,直指誠體說,普遍地就宇宙萬物說,則即是誠體流行之天道。如以客觀的誠體流行之天道為準,則其具於個體(人與物)即謂之性。其具於人與物之個體,即名曰天賦或天命。性命之「命」,不是命運之

\newpage\thispagestyle{empty}\addtocounter{page}{-1}\vspace*{-12mm}\begin{center}\noindent
\includegraphics[clip, trim=171pt 146pt 143pt 253pt, height=162mm]{ocr-input/image-1426.png}\end{center}

\newpage

\noindent 命,乃是命令之命。自天道之命于(賦于)吾人言,曰命,自人之所受言,曰性。就此語脈言,命好像完全是動字、作用字,然此動字中即含有有內容之名詞義,命之作用中即含著道之實體,即以其自己命於人而為吾人之性體也。不過因著重性之超越根源,故著一命字以明其通於天道耳。其實天道、性、命是一事也。若純天道地言之(即宇宙論地言之),則天道等於天命。天命中命之動用義即等同於實體義。故曰「天命流行」也。(此「流行」一詞完全根據命令作用說。)此乃根據「維天之命於穆不已」之最根源的智慧而來。儒家的形上智慧完全本此詩句而發,無有能違之者。此於穆不已之天命永遠在起作用,此即其流行。其命於(流行於)人即為人之性,此即示性命天道相貢通,亦示性乃先天地(超越地)定然如此者。吾人之行動只應遵此性分之所定而行,乃毫無他顧可言者,是故性之,即是命之,性體亦必然起命令之作用而成道德之創造,故能完成天道于自己之生命中(成之者性也),此亦是天命流行也,在此亦是性命天道而為一也。前一義(即自天道之命於人而為人之性言)之性命天道相貫通,張橫渠言之極為精澈;後一義(即自「性之即是命之」而言)之性命天道相貫通,明道言之極為精澈,此即其所謂一本之密義。濂溪以誠體合釋《易傳》,於易見出性命之根源其實亦即此性命天道相貫通之大義也。

然理非空言,道不虛懸,必待人之體現。即就「成之者性也」而言,真能盡其性而體現此天道以至其極者曰聖人。而聖人之所以能盡其性,亦不過誠而已矣。故《中庸》曰:「惟天下之至誠為能盡其性。」濂溪即本此而曰「誠者,聖人之本」,此即此章之首句也。此言誠為聖人之本是就聖果說。言聖人之所以為聖亦並無其他

\newpage\thispagestyle{empty}\addtocounter{page}{-1}\vspace*{-12mm}\begin{center}\noindent
\includegraphics[clip, trim=168pt 129pt 136pt 254pt, height=162mm]{ocr-input/image-1430.png}\end{center}

\newpage\markright{第二部 \quad 分論一 \quad 第一章 \quad 周濂溪對於道體之體悟}

\noindent 巧妙之辦法,亦不過一誠而已。此誠雖亦是工夫字、作用字,然即在誠之工夫作用中,性之全體內容具於中,故誠亦是工夫,亦是本體,故曰誠體,而誠體亦等於性體也。聖人之生命通體是一誠字,故「自誠明謂之性」也。若就人一般言之,人人皆有此誠體,誠豈只為聖人之本耶?故知濂溪此句是就聖果說,不就聖人之為人說,乃就聖人之為聖說。亦即就體現上說,不就本有上說。

\subsection{誠體與寂感真幾}

〈誠下第二〉:

\begin{quotation}\kaishu 聖,誠而已矣。誠,五常之本,百行之源也。靜無而動有,
至正而明達也。五常百行,非誠非也,邪暗塞也,故誠則無
事矣。至易而行難,果而確,無難焉。故曰:「一日克己復
禮,天下歸仁焉」。\end{quotation}

\noindent 案:此為《通書》之第二章。此章首句「聖,誠而已矣」,仍就聖果說。此下即就聖人盡誠而見出誠體為道德的創造之源,故曰:「誠,五常之本,百行之源也。」就其為本為源,即就其為體說,則「靜無而動有,至正而明達。」此兩句是對於誠體本身之體悟。「無」與「有」是借用老子語,無礙也。靜時無聲無臭,無方所,無形跡,一塵不染,純一不雜,故曰「靜無」。靜時雖無,然非死體,故動時則虛而能應,品節不差。其隨事而應,品節不差,則即因其所應之事而有方所,有形跡,此即「動有」也。雖動有而品節不差,則仍不失其一塵不染,純一不雜之虛體。下句「至正」即呼

\newpage\thispagestyle{empty}\addtocounter{page}{-1}\vspace*{-12mm}\begin{center}\noindent
\includegraphics[clip, trim=178pt 155pt 137pt 248pt, height=162mm]{ocr-input/image-1434.png}\end{center}

\newpage

\noindent 應「靜無」,「明達」即呼應「動有」。故靜無以「至正」來了解,動有以「明達」來了解。明即「自誠明」之明,達即「利貞」之達。是故借用有無,其對於有無之思路上的理會雖同於老子,然有無所形容之實體固仍是純然儒家之義理。作用性相之形容(屬性)固可相同也。此則非決定儒道之差異者,亦非可因此即謂其屬於老學者。此章以此兩句為主,餘文字句,讀者自能通之。

〈誠幾德第三〉:

\begin{quotation}\kaishu 誠無爲,幾善惡。德:愛曰仁,宜曰義,理曰禮,通曰智,
守曰信。性焉安焉之謂聖,復焉執焉之謂賢。發微不可見,
充周不可窮之謂神。\end{quotation}

\noindent 案:此為《通書〉之第三章,乃直接繼承上章「靜無而動有,至正而明達」而來者。此言「誠無為」是指誠體本身言。「無為」即〈繫辭傳〉「易無思也、無為也」之無為。此無為之形容可單指靜無之體言,亦可駭誠體流行之全部言,即其靜無動有之流行全部皆是無思無為之自然流行。無為者,自然義、無造作義、無臆計義。此雖同於老子,然非老子所專有也。抑非同於老子,乃老子之能同於共法耳。誠體雖無為,雖靜無而動有,至正而明達,然吾人之感於物而動,其動之幾,剋就幾之為幾之本身言,則不能無差異之分化,即不能保其必純一,故有或善或惡之分歧也。其動之幾純承誠體而動者為善,以不為感性(物欲)所左右故,純是順應超越之誠體而動故。若不順應誠體而動,而為感性所左右,則即為惡。此處所言之「幾」即後來所謂「念」也。(陽明所謂隨軀殼起念,劉蕺

\newpage\thispagestyle{empty}\addtocounter{page}{-1}\vspace*{-12mm}\begin{center}\noindent
\includegraphics[clip, trim=156pt 133pt 137pt 248pt, height=162mm]{ocr-input/image-1438.png}\end{center}

\newpage\markright{第二部 \quad 分論一 \quad 第一章 \quad 周濂溪對於道體之體悟}

\noindent 山嚴分意與念之念)。

順誠體而動,則德行皆從此出,故誠體為道德的創造之真源。德分為五,即仁義禮智信是也。此皆順誠體而表現者,所謂應事而動,品節不差者是也。亦「誠,五常之本百行之源」之義也。(孔子之「仁」即相當於此處之誠體。此處言仁就五常言。推進一步言誠體是根據《中庸》、《易傳》說,不直接本孔子之仁、孟子之心說。但《中庸》、《易傳》之天道誠體即是根據孔子之仁孟子之心而轉出者,亦可謂是擴大而至其圓極者。北宋諸儒直接順先秦儒家之發展其極者而立言,而吾人不可不知其來歷以通之也。)

「性焉安焉之謂聖」三語是就體現誠體說。體現誠體,隨人之根器而有不同之方式(形態)。「性焉」即「堯舜性之也」之「性之」。「安焉」即「安而行之」之「安行」。不假工夫,本性自然如此,不待勉強而安然如此,此即謂聖。不能自然安然如此,而須待擇善而固執之以漸復其誠體,則即謂賢。性焉安焉稱體而行,其發也幾微隱幽而不可見,然而其感應迅速頓時「充周而不可窮」,揚眉瞬目,一念之動,即感應無方而無窮無盡,此即為聖而神矣。此亦與孟子所謂「大而化之之謂聖,聖而不可知之謂神」同也。「大而化之」是從廣大說。「性焉安焉」是從精微說。其極皆不可知也。「化」字亦廣大亦精微,亦不可知也。

〈聖第四〉:

\begin{quotation}\kaishu 寂然不動者誠也。感而遂通者神也。動而未形、有無之間者
幾也。誠精故明,神應故妙,幾微故幽。誠神幾曰聖人。\end{quotation}

\newpage\thispagestyle{empty}\addtocounter{page}{-1}\vspace*{-12mm}\begin{center}\noindent
\includegraphics[clip, trim=185pt 212pt 128pt 245pt, height=162mm]{ocr-input/image-1442.png}\end{center}

\newpage

\noindent 案:此為《通書》之第四章。第二、三、四連三章主旨皆是說誠體一觀念。自「靜無而動有」以至「誠無為」,落實說,主旨即在「寂然不動,感而遂通」兩句。《易·繫辭傳》云:「易無思也,無爲也,寂然不動,感而遂通天下之故,非天下之至神,其孰能與於此?」「寂然不動,感而遂通」是先秦儒家原有而亦最深之玄思(形上智慧)。濂溪即通過此兩句而了解誠體。「寂然不動者誠也」,此就誠體之體說。「感而遂通者神也」,此就誠體之用說。總之,誠體只是一個「寂感真幾」。此為對於誠體之具體的了解(內容的了解)。說天道、乾道,是籠統字(形式的、抽象的),故實之以「誠體」;誠體亦籠統,故復實之以寂感。濂溪「默契道妙」,即首先握住此最根源之智慧,而言之復如此其精微而順適,非真有默契者不能也。真能體現此道妙者曰聖人。「誠神幾曰聖人」,此為對於聖人之最具體最內在的了解。濂溪言天道、言誠體、言寂感,未有不本於聖不歸於聖者。本於聖,是表示此種理境由聖心而開發;歸於聖,是表示此種理境由聖心而證實。此為儒家之傳統精神,自孔子踐仁知天、孟子盡心知性知天而已然。天道、誠體、寂感之為實體是道德的實體。道德的實體只有通過道德意識與道德踐履而呈現而印證。聖人是道德意識道德踐履之最純然者,故其體現此實體(誠體)亦最充其極而圓滿。所謂充其極而圓滿,一在肯定並證成此實體之普遍性,即此實體是遍萬物而為實體,無一物之能外;二是聖心德量之無外,實體之絕對普遍性即在此無外之聖心德量中而為具體的呈現。不只是一外在的潛存的肯定。此圓滿之模型即是理想之聖人,現實上為孔子所代表。惟自孟子出,下屆《中庸》《易傳》之發展,不獨肯定此實體並由聖人體現此實

\newpage\thispagestyle{empty}\addtocounter{page}{-1}\vspace*{-12mm}\begin{center}\noindent
\includegraphics[clip, trim=149pt 134pt 147pt 247pt, height=162mm]{ocr-input/image-1446.png}\end{center}

\newpage\markright{第二部 \quad 分論一 \quad 第一章 \quad 周濂溪對於道體之體悟}

\noindent 體,且進一步即以此實體為人之性,以建立「人人皆可體現此實體而達至成聖之境界」之根據。此義在孔子並不顯,至孟子始建立。宋儒起無有不繼承此義而立言者。故於其明由道德踐履以達至圓滿之境時,必客觀地以天道性命相貫通為其義理之根據。此為北宋諸儒下屆朱子所首先著力者,而亦為一切理學家所共許;至於學派之分立則在主觀地體現此天道性命相貫通之實體上,對於關鍵所繫之心之了解之差異,以及由此差異而來的工夫路數之差異。

\subsection{聖道、師道(變化氣質)與聖功}

〈道第六〉:

\begin{quotation}\kaishu 聖人之道,仁義中正而已矣。守之貴,行之利,廓之配天
地。豈不易簡?豈爲難知?不守、不行、不廓耳。\end{quotation}

\noindent 案:此為《通書》之第六章。此章言聖人之道為仁義中正是就現實生活中行事之道而言。此與言天道、誠體不同。天道誠體是道德的創造之源,只是一「於穆不已」,只是「天下之臺亹」,並無特殊分際上之特定內容。而仁義中正之為道則正是就現實生活中行事之分際而表現。此是行事分際上幾個普遍的規則,由之可使吾人真有德行者,亦由之可使天道誠體真為「五常之本,百行之源」者,即真可為道德的創造之源者。仁義中正之為道使天道誠體有內容、不虛脫,而反而亦必以天道誠體為其本源(超越的根據),說到實處,實皆出自天道誠體,非由外給也。故前第二章云:「五常百行,非誠非也,邪暗塞也,故誠則無事矣。」誠則皆是,不誠皆

\newpage\thispagestyle{empty}\addtocounter{page}{-1}\vspace*{-12mm}\begin{center}\noindent
\includegraphics[clip, trim=188pt 169pt 130pt 236pt, height=162mm]{ocr-input/image-1450.png}\end{center}

\newpage

\noindent 非。誠則明通,不誠而邪暗則處處阻塞不通。通則有五常百行、有仁義中正;不通則邪謬皆非,焉有仁義中正?自本源言,一誠而已。故曰「誠者聖人之本」,又曰:「聖誠而已矣」,又曰:「誠則無事矣」,此亦至簡易而至易知者。自具體生活行事言,則仁義中正而已。真能有仁義中正,則其誠體自不邪暗。故曰:「守之貴,行之利,廓之配天地。」能守、能行、能廓,即是真能有之也。守而有此仁義中正,則人有其自身之良貴,此乃貴于己而非貴於他者,此即所謂人格之尊嚴(人格之絕對價值)。亦即〈太極圖說〉所謂「人極」也。能行此仁義中正,則無往不利,所謂「明達」或明通也。能廓此仁義中正,(擴而充之),則「充周不可窮」,德與天地配,即所謂「與天地合德」也。亦孔子踐仁知天之義也。至此,則仁義中正全部滲透於誠體之流行,而誠體之流行亦全部是仁義中正之顯現。誠體之內容即是仁義中正也。此亦至為簡易至為易知之事。其以為難者,惟在「不守、不行、不廓耳」。此亦孔子「仁遠乎哉?我欲仁斯仁至矣」之義也。

〈師第七〉:

\begin{quotation}\kaishu 或問曰:曷爲天下善?曰:師。曰:何謂也?曰:性者,剛
柔善惡中而已矣。不達。曰:剛善:為義,為直,為斷,爲
嚴毅,為乾固。惡:為猛,爲隘,爲強梁。柔善:為慈,爲
順,為巽。惡:為懈弱,爲無斷,為邪佞。惟中也者,和
也,中節也,天下之達道也,聖人之事也。故聖人立教,俾
人自易其惡,自至其中而止矣。故先覺覺後覺,暗者求於
明,而師道立矣。師道立,則善人多。善人多,則朝廷正而\end{quotation}

\newpage\thispagestyle{empty}\addtocounter{page}{-1}\vspace*{-12mm}\begin{center}\noindent
\includegraphics[clip, trim=159pt 145pt 150pt 247pt, height=162mm]{ocr-input/image-1454.png}\end{center}

\newpage\markright{第二部 \quad 分論一 \quad 第一章 \quad 周濂溪對於道體之體悟}

\begin{quotation}\kaishu 天下治矣。\end{quotation}

\noindent 案:此為《通書〉之第七章,言師道,旨在明道德踐履惟在經由師友之相感召自覺地變化氣質以完成其仁義中正之德行也。所謂「性者剛柔善惡中而已矣」,此言性實指「氣質之性」而言,不指誠體之為性,或天道性命相貫通之性而言。「氣質之性」一詞由張橫渠開始用。濂溪於此只言「性者剛柔善惡中」,此所謂性明是指「氣質」而言,因氣質始有此差異,如言誠體之為性,則只是純粹至善,無所謂剛柔善惡中之差異也。如就此種氣質而言氣質之性,則「氣質之性」意即人之氣質本身即是一種性,此即王充所謂氣性,或《人物志》所言之才性。吾意橫渠、二程言氣質或氣質之性即是此意,非如後來朱子理解為性體在氣質中濾過,因而成為在氣質限制中之性也。如照此解,則性只是一性,只是一超越之性體,只是伊川、朱子所謂「性即理也」之性,並無二性,但卻可自兩面觀,一是就其本身之本然觀,一是就其在氣質之限制或混雜中觀。如是,氣質之性便成為氣質中的性,而非氣質本身不同即是一種性也。如是「氣質之性」之「之」字便有不同的意義,一是虛係字,意即「氣質」這種性;一是表形容語句之限制字,意即氣質限制中的性(超越性體之性)。而此兩義亦常出入而不自覺。而由惟是一性(超越之性)之兩面觀,以說「氣質之性」,亦本於明道。惟明道言一性之兩面觀只是說超越之性之本然與雜在氣稟中,非以此義解說「氣質之性」一詞也。至朱子重視「義理之性」與「氣質之性」之分,而復以此一性之兩面觀之義解說此「氣質之性」之一詞。如以此解說為準,則雖肯定「氣質」爲一獨立概念,然卻並不

\newpage\thispagestyle{empty}\addtocounter{page}{-1}\vspace*{-12mm}\begin{center}\noindent
\includegraphics[clip, trim=187pt 177pt 121pt 219pt, height=162mm]{ocr-input/image-1458.png}\end{center}

\newpage

\noindent 就氣質本身之不同(種種差異相)而說為一種性。其實,就氣質本身之不同而說為一種性,因此,即名曰「氣質之性」,此恐是此詞所以立之本義。至於去認取超越之性之本然與雜在氣質中而有不同之表現之兩面觀,則是進一步的道理,而亦不妨礙「氣質之性」之建立。惟不必以一性之兩面觀之義解說此「氣質之性」一詞耳。蓋如此,勢必形成語意之混擾。即在朱子,其語脈亦有時是兩義並存者,即:有時確定地解說氣質之性為氣質限制中的超越之性,有時亦不自覺地意指其即為氣質本身之不同也。蓋此後一義乃是此詞之最順適而亦最原初之本義。

濂溪此處說剛柔善惡中之性顯然即是說此種「氣質之性」。此種氣性或才性雖須有超越之性以主宰之,亦須要本超越之性自覺地作道德實踐以變化之,然其本身之作用以及其限制性之大乃事實上不能抹殺者。孟子雖說「口之於味,耳之於聲,目之於色,四肢之於安佚,性也,有命焉,君子不謂性也」,然畢竟亦是一種性,「食色性也」(告子語),究竟亦是人性之自然。「君子不謂性」是重點義,重點雖不由此見人之所以為人之真性,然亦不能說此不是人性之自然。人本有動物性與道德性(神性)之兩面。口之於味、耳之於聲等,雖與剛柔善惡中之氣質有間,然同是氣之事,同屬「生之謂性」(性者生也)一原則,則無疑。口之於味、耳之於聲等,是人性之自然,剛柔善惡中亦是「生之謂性」之進一步的自然。於口之於味耳之於聲等,則說節制,無所謂變化,(此即荀子之著眼點,乃從最低處說),於氣質,則說變化,乃是自覺地作道德踐履之進一步者。張橫渠謂「氣質之性,君子有弗性者焉。」此明是根據孟子之語脈而來,亦是重點義,即重在以「天地之性」為

\newpage\thispagestyle{empty}\addtocounter{page}{-1}\vspace*{-12mm}\begin{center}\noindent
\includegraphics[clip, trim=158pt 158pt 152pt 231pt, height=162mm]{ocr-input/image-1462.png}\end{center}

\newpage\markright{第二部 \quad 分論一 \quad 第一章 \quad 周濂溪對於道體之體悟}

\noindent 本,不以此「氣質之性」為本也。雖不以之為本,然不能忽視其作用之大。茲以此義為「氣質之性」一詞之原義,可賠括「生之謂性」(性者生也)一原則下一切說法(告子、荀子、董子、王充、《人物志》所說之性皆在此原則下),亦可明道德踐履之實功。後來劉蕺山反對人心道心之分,義理之性與氣質之性之分,徒增繳繞,甚無謂也。詳為疏理,見下第四段。

剛性柔性皆可有善惡之表現。變化氣質者,變其惡之表現而為善之表現之謂也。善之表現即合於中正之道。既合於中正之道,則其氣質之偏者,無論為剛性之偏或柔性之偏,亦可化而為中正之氣質矣。「中也者,和也,中節也,天下之達道也,聖人之事也」,此言聖人自能合於中正之道,而亦自具中正之資(氣質或資質)。故其剛亦善,柔亦善,而無往不通也。此言「中」是就中和之資能表現中正之道言,與《中庸》之言中和不同也。《中庸》就喜怒哀樂未發已發而言中為天下之大本、和為天下之達道,至少不是說的氣性(資質、才性)之中和。而濂溪之言「中」是套於剛柔善惡一律言,只有等級之不同,故知是氣質資質之中也。蓋自兩漢以來,一般皆肯定聖人之資為中和之資也。聖人固是德性人格之最高級,然其所以至此,亦由其有中和之資使然也。一般人不具此中和之資,則須自覺地作工夫以變化其氣質之偏,以期合於中正之道。氣質之性是限制原則,亦是體現誠體以及仁義中正之道之資具。不能不有此資具,故雖聖人亦必須有中和之資。而又是一限制,故雖有中和之資,而「及其至也,雖聖人亦有所不知焉,亦有所不能焉。」此是一甚深嚴肅之義,不可輕忽也。

濂溪此章只言實功落在變化氣質上,至於如何達到此實功,則

\newpage\thispagestyle{empty}\addtocounter{page}{-1}\vspace*{-12mm}\begin{center}\noindent
\includegraphics[clip, trim=167pt 162pt 141pt 235pt, height=162mm]{ocr-input/image-1466.png}\end{center}

\newpage

\noindent 見下章。

〈思第九〉:

\begin{quotation}\kaishu 〈洪範〉曰:「思曰睿,睿作聖。」無思,本也。思通,用
也。幾動於此,誠動於彼。無思而無不通為聖人。不思,則
不能通微。不睿,則不能無不通。是則無不通生於通微。通
微生於思。故思者聖功之本,而吉凶之幾也。《易》曰:
「君子見幾而作,不俟終日。」又曰:「知幾其神乎?」\end{quotation}

\noindent 案:此為《通書》之第九章,正式言工夫。工夫者,主觀地通過心之自覺明用以體現天道誠體之謂也。天道誠體為客觀性原則,心為主觀性原則。心之自覺明用可多方以言之,而濂溪於此則根據〈洪範〉而言「思」。故知此章實濂溪之言「心」也。乃由思以明心之用。孟子亦言「心之官則思」,又言「思誠」。思乃心之通用(一般性的作用)。但此一般性之作用,就道德實踐之工夫言,亦有其特殊的意義。孟子言「心之官則思」,是對「耳目之官不思而蔽於物」而言,是則思者是表示心之解放,從感性之拘囿中而開擴其自己,是心之超越乎感性以上而明朗其自己。思乃心之明通,此為心之第一步的道德意義,即不為感性所蔽而主宰乎感性。言「思誠」則是由思之對象而規定其道德的意義,明此思是思誠體,並非成功經驗知識之一般思想也。成功經驗知識之一般思想,其所思者乃是經驗對象,而此卻是誠體。思誠者即是明朗其誠體之謂。誠體是客觀地說的道德之實體,思誠即是主觀地朗現此誠體。誠體朗現,誠體之真實性(道德創造的實體之真實性)即全部滲透於此思之明用

\newpage\thispagestyle{empty}\addtocounter{page}{-1}\vspace*{-12mm}\begin{center}\noindent
\includegraphics[clip, trim=171pt 144pt 130pt 238pt, height=162mm]{ocr-input/image-1470.png}\end{center}

\newpage\markright{第二部 \quad 分論一 \quad 第一章 \quad 周濂溪對於道體之體悟}

\noindent 中,而思亦為創闢朗潤之思,而思亦全部浸潤於誠體中,而誠體亦為瑩徹明通之誠體,此為道德意義之思,亦即心之通用之進一步地規定其道德的意義。

濂溪根據〈洪範〉之「思曰睿,睿作聖」,亦仍是以此種道德意義之思(心之通用)而言聖功。其所思者乃是「幾善惡」之幾,「動而未形,有無之間者,幾也」之幾,「幾者動之微、吉凶之先見者也」之幾。思之功即落在「幾」上用,即是要徹底通化此幾而使之歸於善,使之純然順應誠體而動,而無一毫之夾雜。此則尤顯道德踐履之功之切義。孟子言「思誠」是朗現誠體,是從正面言;濂溪言思在幾上用,是化除幾之惡,是從負面言。

就聖果言,思之最高境界是「無思之思」。故曰「無思本也」。此言「無思」即《易·繫辭傳》「易無思也,無為也,寂然不動,感而遂通天下之故」之無思無為之無思。思雖以無思為本,然不能停滯於無思。而無思亦非是槁木與死灰。言「為本」者表示以「無思」為體耳。有體必有用,故曰:「思通,用也」。「思通」即思以「通微」也。此是無思之用也。思以通微之極則是「無不通」。濂溪即以「無不通」規定「睿」。至「無不通」時,便知此思是無思之思,故曰:「無思而無不通為聖人」。既是無思,又是思之而無不通,則知此思不是有計慮、有將迎之有作有為之思,而是無作無為,惟是一誠體流行之思。此不是經驗界(感性界)之思,而是一種超越之睿思,故曰:「睿作聖」也。濂溪言思,誠體之用即注(亦是著)于其中。

以「無思而無不通」之睿恢復並證實(彰顯)誠體之流行,誠體即在「無思而無不通」中重新建立,亦即於此而全體朗現。此為

\newpage\thispagestyle{empty}\addtocounter{page}{-1}\vspace*{-12mm}\begin{center}\noindent
\includegraphics[clip, trim=158pt 157pt 148pt 239pt, height=162mm]{ocr-input/image-1474.png}\end{center}

\newpage

\noindent 誠體之具體化與真實化。故睿思過程即是誠體建立之過程。故曰:「幾動於此,誠動於彼。」思之功全在幾上用。思之「通微」即通幾之微,亦即知幾、審幾,而亦慎於幾也。在動之微處,或吉或凶,或善或惡,皆由此出,故須知、審而慎也。此亦慎獨工夫之別稱,以思之通化幾之惡而表示慎獨之工夫。在知幾中,不只是一種靜觀的知,而且是一種「化凶為吉,化惡為善」的實踐工夫。幾之動,若順寂體(誠體)而來,則純善而無惡。一念不本於寂體,則陷於邪而為惡。知幾即在動之微處而神感神應。常戒慎恐懼而保其清明之體,故能知微(通微)。知微而至神感神應,即是「無思無不通」而爲睿矣。《易·繫辭傳》曰:「顏氏之子,其殆庶幾乎?有不善未嘗不知,知之未嘗復行。」「有不善未嘗不知」,即示顏子能常保其清明之體(誠體、寂體、心體),故能知微也。「知之未嘗復行」,即示知之即化之也。故王龍溪常於此稱顏子「纔動即覺,纔覺即化。」此即示顏子之生命「誠精故明,神應故妙」(見前〈聖第四〉),庶幾近乎「無不通」之睿境矣。此確是澈底清澈自家生命之道德工夫,此是道德實踐之基本義。

幾動是現象,即如其為現象自身言之,它屬於經驗層。而知幾之知,通微之思,乃至於睿,則屬超越層,是清明心體之用。若知之即化之,則動從寂體,即經驗即超越,即用即體,則純善而無惡。斯則誠體立矣。「幾動於此,誠動於彼」,即言幾一動,而誠體之思與知即隨之而應於上。實則此兩句改言為「幾動於彼,誠動於此」,則較順適。於誠動而言「此」,從主體也。誠體之動於此而照臨乎彼幾之動,則知(思)之「通微」義顯矣。故濂溪之言「思」實即誠體註入其中之思也。故思亦可謂為誠體之用,亦即是

\newpage\thispagestyle{empty}\addtocounter{page}{-1}\vspace*{-12mm}\begin{center}\noindent
\includegraphics[clip, trim=173pt 139pt 125pt 248pt, height=162mm]{ocr-input/image-1478.png}\end{center}

\newpage\markright{第二部 \quad 分論一 \quad 第一章 \quad 周濂溪對於道體之體悟}

\noindent 相應誠體而為創關朗潤之思。及此通微之思而至乎「無不通」之睿,則幾之動即全吉而無凶、全善而無惡,乃稱體而化矣。此即「無思、思通」之全體大用,而誠體亦於焉以立,而全部朗現焉。此即聖人之境。誠體立,(立是彰顯之立,非本無今有之立),則思亦相應此誠體之「寂然不動感而遂通」而同其為「寂然不動感而遂通」,而為「無思而無不通」之思矣,此即是睿,亦即是「發微不可見(寂然不動)充周不可窮(感而遂通)之謂神」之神也。此時主觀說的「思」與客觀說的誠體全融合而為一,誠體寂感之神即是思用之神也。故《易》曰:「知幾其神乎?」《易·繫辭傳》此語,經過濂溪思之通微而至無不通之道德實踐的意義,其義比一般人初見所想之意義為深刻而積極。它不只是察事變之幾而知之於幾先。能察事變之幾而知之於幾先,固亦是神,但濂溪之此言知幾之神有一種道德的通化之意,不只是旁觀之照察,故是相應誠體寂感之神而亦為「充周不可窮」之神也。「君子見幾而作不俟終日」即是「纔動即覺,纔覺即化」之義,亦是道德的通化之義。此是《易傳》此數語之本義,故《易傳》以顏子「有不善未嘗不知,知之未嘗復行」爲例也。若只解為知幾避禍,則淺而陋矣。此正是自私,焉得能為誠體之神?故濂溪之言思、言知幾,而謂為「聖功之本」,此乃正是不失其道德踐履上通化之義者。(「思者聖功之本,而吉凶之幾也」,此中綴之以「吉凶之幾」易有誤解。實則只是思則吉,不思則凶之義。思則超化,故吉;不思則陷落,故凶。〈太極圖說〉言:「君子修之吉,小人悖之凶」,亦此意也。此是表示思為「聖功之本」,又是作聖之本質的關鍵。)

綜觀此章是濂溪言作聖之功之最深微者。言工夫,不能不言

\newpage\thispagestyle{empty}\addtocounter{page}{-1}\vspace*{-12mm}\begin{center}\noindent
\includegraphics[clip, trim=174pt 158pt 155pt 257pt, height=162mm]{ocr-input/image-1482.png}\end{center}

\newpage

\noindent 心。心是體現誠體之關鍵,故此章實即濂溪之言心。但此由思所表現之聖功猶是就心之通用(一般的作用)言,須把心提到「無思而無不通」之圓用方能至聖之睿。此圓用是以誠體之寂感之神為標準者。若問此圓用是否能由心之自己而挺立,即是否能本質地挺立起,則光只註重此心之通用而當然地如此說,似尚不能解答此問者。此即表示說,若心只是此思用,則不必然地能至此圓用之境者。即或經過一種虛靜之工夫,而可至此圓用之境,亦不必真能彰著此誠體而與誠體合一,因而其自身即是此誠體寂感之創生之神用。顯然道家經過虛靜之工夫即可至此圓用之境者,然而道家之心顯然不即是此誠體寂感之神用,當然,濂溪是儒者之心態,故其言此圓用是與誠體相湊泊者。「幾動於此,誠動於彼」,在誠動處,即是誠思之合一。然此只是當然如此說。而若心只是思用,則不必真能至此。此猶是兩者之偶然地湊泊,而不是必然地即爲一事者。吾之提出此義,旨在表示就體現誠體之工夫而註意及心而言,此時之心即不能只註意其思用,必須進一步更內在地註意其道德的實體性之體義,此即是「其圓用能本質地挺立起」之關鍵,亦是「其圓用即是此誠體寂感神用」之關鍵。此道德的實體性之體義的心即是孟子由之以說性善的心,即所謂本心,其所以為體之內容即所謂惻隱、羞惡、辭讓、是非等等者。由此開工夫更是真切於挺拔之道德踐履者,更是切近於先秦儒家所表示的道德的創造之陽剛之美者。而不是只從思用以言也。而濂溪所妙契之思用之「無思而無不通」之睿境亦正在此而充實起而挺立起,因而亦有其必然性。濂溪之妙契是用在《中庸〉與《易傳》,而于孟子之言心似不甚能真切,而亦有忽略,故于言工夫,迂曲而尋根據于〈洪範〉,而不知就教于

\newpage\thispagestyle{empty}\addtocounter{page}{-1}\vspace*{-12mm}\begin{center}\noindent
\includegraphics[clip, trim=173pt 127pt 123pt 257pt, height=162mm]{ocr-input/image-1486.png}\end{center}

\newpage\markright{第二部 \quad 分論一 \quad 第一章 \quad 周濂溪對於道體之體悟}

\noindent 孟子,可謂舍近而求遠。此固是在初創,然亦由其不能貫通先秦儒家之發展而然也。此亦是其易被人聯想為有道家意味者之故。

能就孟子之道德的實體性之體義的心而謂其即是此天道誠體之神用,因而極成其所謂「一本」者,乃是明道;能由之而開工夫而更真切于挺拔之道德踐履,更切近於先秦儒家所表示之道德創造之陽剛之美者,則為陸象山。此則乃進於濂溪者。但吾人必須於濂溪之言心處,記住此義,始能知後來之發展,以及洞澈學派分立之關鍵。

此外,〈志學第十〉:

\begin{quotation}\kaishu 聖希天,賢希聖,士希賢。伊尹、顏淵大賢也。伊尹恥其君
不為堯舜,一夫不得其所,若撻於市。顏淵不遷怒,不貳
過,三月不違仁。志伊尹之所志,學顏子之所學。過則聖,
及則賢,不及則亦不失於令名。\end{quotation}

\noindent 案:此為宋、明儒共同之意識,亦確然是典型之儒家精神。易明,亦無問題,不煩疏解。

〈聖學第二十〉:

\begin{quotation}\kaishu 聖可學乎?曰:可。曰:有要乎?曰:有。請問焉。曰:一
為要。一者,無欲也。無欲則靜虛動直。靜虛則明,明則
通。動直則公,公則溥。明通公溥,庶矣乎?\end{quotation}

案:此章言聖可學而至,此是先秦儒家本有之義,亦是宋、明

\newpage\thispagestyle{empty}\addtocounter{page}{-1}\vspace*{-12mm}\begin{center}\noindent
\includegraphics[clip, trim=164pt 148pt 124pt 236pt, height=162mm]{ocr-input/image-1490.png}\end{center}

\newpage

\noindent 儒共同之主張。至於言學聖之要(亦是學聖之工夫,亦即自己踐履以嚮往聖境之工夫),若順通以上言誠體處,則亦易明,而如此說,亦總是必要者,亦非問題之所在,故亦不煩疏解。

\subsection{}

〈順化第十一〉:

\begin{quotation}\kaishu 天以陽生萬物,以陰成萬物。生,仁也。成,義也。故聖人
在上,以仁育萬物,以義正萬民。天道行而萬物順,聖德修
而萬民化。大順大化,不見其跡,莫知其然之謂神。故天下
之眾,本在一人。道豈遠乎哉?術豈多乎哉?\end{quotation}

\noindent 案:此章言順化,順即「天道行而萬物順」之順,即依道而自然順成;化即「聖德修而萬民化」之化,即由於聖德之感召而自然潛移默化·就萬物與萬民而分別言順與化。其所以能順與化者由於道也。此言順化,就萬物與萬民說。若從體言,則說神化。《易》之爲書,主旨即在窮神知化。故「子曰:知變化之道者,其知神之所爲乎?」((繫辭傳〉)變化之道即易也。「生生之謂易」,「陰陽不測之謂神」,「故神無方而易無體」。(皆(繫辭傳〉語)此言神化從天道誠體說,即屬於能(主),而順化則屬於所(從)也。然天道誠體之神化之用即見之於「所」處順化之實也。故濂溪云:「大順大化,不見其跡,莫知其然之謂神」。子貢謂「夫子之得邦家者,所謂立之斯立,道之斯行,綏之斯來,動之斯和」(《論語·子張》第十九),即「聖德修而萬民化」一語之所由

\newpage\thispagestyle{empty}\addtocounter{page}{-1}\vspace*{-12mm}\begin{center}\noindent
\includegraphics[clip, trim=165pt 129pt 137pt 254pt, height=162mm]{ocr-input/image-1494.png}\end{center}

\newpage\markright{第二部 \quad 分論一 \quad 第一章 \quad 周濂溪對於道體之體悟}

\noindent 來。萬民之順化莫知其然,不見其跡,是所謂自然而潛移默化也。然而吾人亦知此潛移默化之所以然,此即由於「聖德修」也,亦由于夫子之在上也。故順化處之神即是誠體處之神也。

「天以陽生萬物,以陰成萬物」,陽生陰成是落在跡上說。陰陽氣也,故有跡。而所以妙用之而使之成其為生,成其為成者,則是天道誠體之神用。「神也者妙萬物而為言」(《易·說卦傳〉)。天道誠體實有能生能化之神用,然其生無生相,其化無化相,故只是一神用,而神無方所、無形跡,故亦曰寂感真幾也。而一落在氣上,則有跡矣。就其能生能化之神用言,亦曰道德的創造之真幾,一切德皆從此出。仁義有其分際表現上之定義,而大體言之,仁為生德,義為成德(裁成之成,亦可曰斷德),故以仁比陽生,以義比陰成,故曰:「生,仁也;成,義也。」此就人格生命之體現誠體而表現為德性言。故曰:「聖人在上,以仁育萬物,以義正萬民。」此是天道之在聖人生命上之表現,而仁義即是天道之內容也。故最後結之曰:「天道行而萬物順,聖德修而萬民化。大順大化,不見其跡,莫知其然之謂神。」自天道誠體而來之具體的德性固有此神用也。【不是抽象的仁義有此神用,乃是融於具體的誠體中之具體的仁義、生命表現的仁義,有此神用。又濂溪言仁義或仁義中正,其所言之仁俱是偏屬之仁。但仁亦可提起來言,如此,仁是全德,是一切德性之源,是則仁即等於誠體,故亦曰仁體。此種絕對的仁體(仁道、仁心)即是孔子所多方以指點者。明道開始精識此義,濂溪尚未能至此。濂溪對於《論語》、《孟子》體會不深,其所默契者是《中庸》、《易傳》也。故以天道誠體為第一義,而凡言德則就通常之五常言,是在第二義。故云:「誠

\newpage\thispagestyle{empty}\addtocounter{page}{-1}\vspace*{-12mm}\begin{center}\noindent
\includegraphics[clip, trim=163pt 157pt 144pt 239pt, height=162mm]{ocr-input/image-1498.png}\end{center}

\newpage

\noindent 者,五常之本,百行之源也。」其言「聖人之道仁義中正而已矣」,以及此章之言「以仁育萬物,以義正萬民」,是由第二義以通第一義,故吾解之云:仁義即是天道之內容。然仁是生德,明道即由此生德之仁提起來而言其絕對義,故仁體即是天道誠體之具體的說,即是道德的創生之誠體。仁有二特性,一曰覺,二曰健。綜此二特性而識仁體,則仁以感通為性、以潤物為用,故仁即是道德的創造性之自己也,故曰仁體。此則仁由其相對的(偏屬的)德目義解放而為形上的實體義(雖亦是道德的)。此則為濂溪所不及者。濂溪是天道誠體與仁義中正分別平說,而仁義中正是就其為德目之義言。德目之踐之至其極亦可通天道,然未能解放而即是天道誠體也。兩者猶有間而未能一。宋明儒之發展,大體是由《中庸》、《易傳》開始而逐步向《論》、《孟》轉,以孔子之仁與孟子之心為主證實天道誠體之所以為天道誠體而一之———之於仁,一之於心,重新恢復先秦儒家從孔、孟到《中庸》、《易傳》之發展,如此而知《中庸》、《易傳》是其圓境。否則,《中庸〉、《易傳》之天道誠體只是空頭的宇宙論的,亦是外在的,此則客重而主輕,濂溪、橫渠俱有此意味。本是主客觀之真實統一之圓教,然而因不能貫通先秦儒家發展之序遂顯出客重而主輕,亦可說是內輕而外重,主觀性原則(心)不足故也。逐步向《論》、《孟》轉,第一步關鍵是明道之一本論,第二步關鍵是象山之孟子學。至此而主觀性原則澈底挺起矣。由伊川而至朱子則是歧出,不合先秦儒家發展之序,亦不合先秦儒家立教之基本精神——道德的創生之陽剛之美。吾茲先言其大略,藉以明濂溪之限度以及此後發展之大體方向與夫朱子之歧出。此後逐步明之。]

\newpage\thispagestyle{empty}\addtocounter{page}{-1}\vspace*{-12mm}\begin{center}\noindent
\includegraphics[clip, trim=167pt 128pt 134pt 252pt, height=162mm]{ocr-input/image-1502.png}\end{center}

\newpage\markright{第二部 \quad 分論一 \quad 第一章 \quad 周濂溪對於道體之體悟}

〈動靜第十六〉:

\begin{quotation}\kaishu 動而無靜,靜而無動,物也。動而無動,靜而無靜,神也。
動而無動,靜而無靜,非不動不靜也。物則不通,神妙萬
物。水陰根陽,火陽根陰。五行陰陽,陰陽太極,四時運
行,萬物終始。混兮關兮,其無窮兮。\end{quotation}

\noindent 案:此章是對於神之為神之體悟,亦是上章言順化及首四章言天道誠體之總匯,亦是濂溪言道體之最有形而上的玄悟與宇宙論的旨趣者。前言天道是籠統字,故實之以誠體;誠體猶嫌籠統,故實之以寂感。是則天道誠體即是一寂感真幾,引申而為道德的創造之實體,此實體確有能生能化之神用。就此神用言,如以動靜形容之,則是「動而無動,靜而無靜」者。「動而無動」是動無動相。動言其非抽象之死體,它純然是一虛靈。然純然是虛靈之動乃無動相者。無動相言其非限定之物之動也。限定之物之動,動即是動,因而無靜矣,是即動與靜對。然而無動相之動,則動不與靜對,因而動即是靜矣。此純是即動卽靜動靜一如之虛靈寂體,故「不疾而速,不行而至」也。若是疾而速,行而至,則有動相矣,此則為一物矣。虛靈寂體,儼若靜也,然「靜而無靜」。靜言其無動相。無動相非即與動為對之靜也。故亦無靜相。無靜相,則靜即動矣。即靜即動,靜動一如,故仍是「不疾而速,不行而至」也。「不疾、不行」即靜,「而速、而至」即無靜相也。動無動相,是謂至動,至動不與靜對,故亦無所謂動也。靜無靜相,是謂至靜,至靜不與動對,故亦無所謂靜也。是即謂「即動即靜,非動非靜」之神用,

\newpage\thispagestyle{empty}\addtocounter{page}{-1}\vspace*{-12mm}\begin{center}\noindent
\includegraphics[clip, trim=159pt 158pt 141pt 230pt, height=162mm]{ocr-input/image-1506.png}\end{center}

\newpage

\noindent 而必須通過「動而無動,靜而無靜」之詭辭以妙悟之。「動而無動」非是說不動也,只言無動相而已。若真是說「不動」,則即只是靜而已。只是靜即與動為對也。「靜而無靜」亦非是說「不靜」也,只言無靜相而已。若真是說「不靜」,則只是動而已。只是動即與靜為對也。此即是「動而無動,靜而無靜,非不動不靜也」一申辨語之確義。此中之「非不動不靜」與通過「動而無動靜而無靜」而歸至一如之「即動即靜,非動非靜」中之「非動非靜」並非同一意義,故亦並不衝突也。「非不動不靜」是表示神實有動義靜義,亦實可以動靜去體會,但只是其動是「動而無動」之動,其靜是「靜而無靜」之靜,並非無所謂動靜,根本不可以動靜去體會之意也。而「即動即靜、非動非靜」中之「非動」則只是遮動相,遮動相非遮動也,亦非即只是靜也,故靜而無靜矣;而其中之「非靜」亦只是遮靜相,遮靜相非遮靜也,亦非即只是動也,故動而無動矣。

限定之物「動而無靜」,動只是動,「靜而無動」,靜只是靜。靜不通動,動不通靜,故「物則不通」也。此適於直線思考。而對於神之體悟,則須用曲線思考,故圓妙也。其本身為妙,故亦能妙萬物也。所謂「神妙萬物」者,即云限定之物雖動只是動,靜只是靜,然其動了又靜、靜了又動,而實有其生化之實事而不窮者,實是一誠體之神為之也,故生化實事之不窮實即一誠體之神之流行與充周。至此而真見「維天之命,於穆不已」也。故濂溪下文即云:「水陰根陽,火陽根陰。五行陰陽,陰陽太極。四時運行,萬物終始。混兮闢兮,其無窮兮。」「水陰根陽」,即水之陰是根於火之陽而來,亦即陰之靜根於陽動之極而來,此即動了又靜。

\newpage\thispagestyle{empty}\addtocounter{page}{-1}\vspace*{-12mm}\begin{center}\noindent
\includegraphics[clip, trim=167pt 131pt 130pt 245pt, height=162mm]{ocr-input/image-1510.png}\end{center}

\newpage\markright{第二部 \quad 分論一 \quad 第一章 \quad 周濂漢對於道體之體悟}

\noindent 「火陽根陰」,即火之陽是根於水之陰而來,亦即陽之動是根於陰靜之極而來,此即靜了又動。五行之相生相剋實即陰陽之動靜。故云「五行陰陽」,意即五行一陰陽也。而陰陽之動了又靜、靜了又動,實即一誠體之神之流行與充周,亦即其妙用。故云:「陰陽太極」,意即陰陽一太極也。言「太極」者是順陰陽言。此本於《易傳》「易有太極,是生兩儀。」因言及陰陽,故想及「太極」。是則太極即誠體之神也。如太極真意指一極至之實體,非太極外別有實體,則太極除即是天道誠體之神外,不會是別的。

因誠體之神之流行與充周,故有陰陽生化實事之不盡,因而四時得以運行而不息,萬物得以成始而成終。「混兮關兮,其無窮兮」,即指陰陽生化實事之無窮盡也。「混兮」是其幾微之始,「關兮」是其生成之著。而無論混而闢,關而混,亦可說皆是天道誠體之神用之所成也。無此誠體之神,則乾坤或幾乎息矣。(混不指太極言,混關皆是陰陽邊事。此不是以太極為渾一之氣由之開而爲二,復開而為四等等也。混闢對言,不是太極陰陽對言也。太極、誠體之神無所謂混,亦無所謂關。寂感真幾,寂亦不是混,感亦不是闢。寂感一如是神,而混與關則是氣,陰陽邊事。濂溪此處只隨《易傳》言太極,未曾言無極。又(太極圖說〉有「無極而太極。太極動而生陽」云云,此「太極動而生陽」有類於「易有太極,是生兩儀。」《易傳》太極兩儀四象八卦一套,暫置不言,依至理,此「動而生陽」當如何解?依《通書》言天道誠體之神,不曾觸及此義。然〈太極圖說〉則明言之。此義可得解乎?可會而通之乎?抑為不可通乎?〈太極圖說〉真為偽乎?或「其學未成時所作」乎?凡此等等,詳解見下節)。

\newpage\thispagestyle{empty}\addtocounter{page}{-1}\vspace*{-12mm}\begin{center}\noindent
\includegraphics[clip, trim=181pt 158pt 135pt 243pt, height=162mm]{ocr-input/image-1514.png}\end{center}

\newpage

〈理性命第二十二〉:

\begin{quotation}\kaishu 厥彰厥微,匪靈弗瑩。剛善剛惡,柔亦如之,中焉止矣。二
氣五行,化生萬物。五殊二實,二本則一。是萬為一,一實
萬分。萬一各正,小大有定。\end{quotation}

\noindent 案:此章言理、性、命,亦是極有形而上的玄悟與宇宙論的旨趣者。然全文中無理、性、命之詞,而其義卻含於其中。「厥彰厥微,匪靈弗瑩」,此言生化之事無論其彰著而成形,或其幾微而尚未形,皆是非有誠體之神以妙之不能瑩徹而通也。「靈」指誠體之神用言。誠體寂然不動感而遂通,故是虛靈而神也。此「靈」字即代表「理」。此理是形而上的實體之理,亦即誠體之理——誠體即理也。理是就誠體之妙萬物而為其「超越」而言。生化之實,無論為彰為微,是實然。實然必有其超越的所以然而妙之(而實現之)。「匪靈弗瑩」,靈即其能瑩徹而通之超越的所以然,此即「理」也。此理為實現之理或亦曰生化之理。對生化之事而為其所以實現之理也。此理不只是靜態地規則之定然之之理則,而且是動態地實現之之生理(生物不測之創生之理)。至於此理本身,則即是誠體之神、寂感真幾。對此寂感真幾,若分解之而剖示其內容,則它亦是理、亦是心亦是神。從寂感處必然函著是心。因為心才可說寂感。寂感一如卽是神,心之虛靈充周卽是神。寂感一如不只是神用,且即在其寂感神用之創生中湧發出定然之理則,此即是理。圓而神中即含著方以智。圓神是用,方智是理,此謂圓方一如,心理一如,心即是理。此理是理則之理,所謂普遍的律則也。

\newpage\thispagestyle{empty}\addtocounter{page}{-1}\vspace*{-12mm}\begin{center}\noindent
\includegraphics[clip, trim=169pt 144pt 140pt 248pt, height=162mm]{ocr-input/image-1518.png}\end{center}

\newpage\markright{第二部 \quad 分論一 \quad 第一章 \quad 周濂溪對於道體之體悟}

\noindent 此是誠體底內容之一。若此誠體,實指目之,即是心神,則此理則,即是其內容,惟此內容是實言。對誠體言,心神理俱是其內容,心神之為內容是虛言,名之為函義或較好。「此心神理是一」之誠體對生化之事言亦為理——此理是其動態的超越的所以然,即實現之理。此「理」之意義與「心神理是一」而為誠體中的內容之「理」之意義不同。此兩理字乃是兩個層次者。前一理字總指誠體而言,後一理字則是此誠體中之內容——理則之理。此兩者不可混,而朱子之分解則無此兩層意義之分別,只把實現之理視為超越的所以然之靜態的、形式意義的只是理,如是天道太極只成為只是理,而心與神遂與理分離,而被劃屬於氣,如是天道太極遂亦不可說為誠體之神、寂感真幾,而只成為一形式意義的、靜態的、超越的所以然之理,而「維天之命於穆不已」一語亦不能維持其實義。此則大非濂溪之所妙契者,亦非《中庸》、《易傳》之原義。此是學派分立之最根源的關鍵。

「剛善剛惡,柔亦如之,中焉止矣」,此三語是言「性」,顯本〈師第七〉「性者剛柔善惡中而已矣」一語而來。剛性有善有惡,柔性亦有善有惡,此皆有偏差,惟中和之資則剛柔得宜,只有其善,而無其惡。故云:「中焉止矣」,即以中和之資為標準也。此言性顯指氣性、資性或才性而言。此種性雖亦可由陰陽五行之氣化而成,但卻非通誠體之性。蓋通誠體之性即以天道誠體為性,此則不可以剛柔善惡論,此雖亦可曰「中」,但若名之曰「中」,則是指體之中,而非資性之中也。陸象山謂「中焉止矣」之中即太極(見與朱子辨(太極圖說〉書),非是。孤立看可即太極,但此言性顯指資性才性言,則此「中」字即不可說是太極。此處言性是

\newpage\thispagestyle{empty}\addtocounter{page}{-1}\vspace*{-12mm}\begin{center}\noindent
\includegraphics[clip, trim=173pt 158pt 151pt 251pt, height=162mm]{ocr-input/image-1522.png}\end{center}

\newpage

\noindent 資性才性,則與天道誠體之為理即不能通而為一者,雖亦可由陰陽氣化而來。蓋天道性命相貫通之「性」是「理」性,非「氣」性也。〈誠第一〉「大哉《易》也,性命之源乎」之語,以及所引〈乾彖〉「乾道變化,各正性命」之語中之「性」,普通皆理解為是通誠體之性,不會是氣性也。但濂溪在該處只提及此語,而並未正解性命與天道誠體之為一;而於〈師第七〉及此〈理性命〉章之言性又是指氣性言,是則天道性命相貫通之義即不顯,而此〈理性命〉章言性之三語亦與言理者不相連屬,如同孤立而橫插於其中者然。此是此章之弱點,亦由於濂溪尚未正視天道性命相貫通之實義也。(橫渠言之極精切,見下章)。

「二氣五行,化生萬物。五殊二實,二本則一。是萬爲一,一實萬分。萬一各正,小大有定。」此數語可視為理性命相貫而為一,惟其中有若干詞語實欠分明。「二氣五行,化生萬物」無問題。「五殊二實,二本則一」,此中之「實」字不甚好講。「五殊二實」可視為兩平行之駢句。「五殊」即五行之殊異。〈太極圖說〉謂:「五行之生也,各一其性」可視為此「五殊」之註語。「二實」即陰陽二氣之實。如是,殊與實俱是論謂詞(論述語中之謂詞)。此在語文上好像是如此。但以義理衡之,陰陽二氣是實,五行豈只殊而不實乎?此在措辭未見允當。「實」字對「殊」字實是隨便安置者。(意想濂溪當時似是如此。蓋設身處地以思之,前既言五殊,後言二當如何,實不好措辭也。)如果此句解為「五行之殊實即陰陽之二氣」,則「實」字為副詞。此在義理上為通順。如此解,則「五殊二實」即〈動靜第十六〉「五行陰陽,陰陽太極」中之「五行陰陽」,而「二本則一」即其中之「陰陽太極」一

\newpage\thispagestyle{empty}\addtocounter{page}{-1}\vspace*{-12mm}\begin{center}\noindent
\includegraphics[clip, trim=167pt 134pt 139pt 254pt, height=162mm]{ocr-input/image-1526.png}\end{center}

\newpage\markright{第二部 \quad 分論一 \quad 第一章 \quad 周濂溪對於道體之體悟}

\noindent 語也。而〈太極圖說〉中亦有「五行一陰陽也,陰陽一太極也」之語。此與「五行陰陽,陰陽太極」爲同一語法也。唯此處之「五殊二實,二本則一」,義理雖同於該兩處,而「五殊二實」之語法實別扭。若解為「五行之殊實即陰陽之二氣」,義理雖通,而語法語意不合。吾意當濂溪之措此語,行文上以實對殊,為兩駢語,而其心中之意指實在是說:「五行之殊實即二氣之實(實處實即二氣)」之一整語。彼下一「實」字實有兩面通之意,若定解為兩駢行之直述語,則不成義理。文欲簡古,而又以對文出之,遂自困於一不易妥順之語脈。(濂溪、橫渠俱有文欲簡古而語不通順之病。如〈誠第二〉「非誠非也」一語即不成語句。橫渠之〈西銘〉:「天地之塞,吾其體;天地之帥,吾其性」,亦別扭不通常。其他辭滯意塞者甚多。此亦有類於陶淵明之自稱為「文妙不足」也。理學家大抵皆性質朴而拙於為文。初期尤甚。至二程以口語出之,則不受文之限制。朱子、象山、陽明文皆暢達。至於說到「文妙」,統宋、明儒皆不及魏、晉之玄學家,亦不及佛教中講經教之大德。)

五行之殊實即二氣之實,而二之本則是「一」也。一即太極或天道誠體之神。「是萬為一,一實萬分」,此兩語一與萬對言,而其中之「實」字亦可有兩解。一解,此兩語當如此:承上故知二氣五行所化生之萬物因其源本於一故可匯歸而為一,而一又實可因其為萬物之體而隨物之分而亦儼若散而為萬萬之多(其自身實不可分而為多)。另一解:萬而歸一,而一理之實又分而為萬。是則「一實」為一詞,意即「一理之實」。此解,義理自通,但亦不合通常對於此種語法之誦讀。蓋此兩語皆是一萬對言,何忽轉為「一實」

\newpage\thispagestyle{empty}\addtocounter{page}{-1}\vspace*{-12mm}\begin{center}\noindent
\includegraphics[clip, trim=136pt 151pt 166pt 238pt, height=162mm]{ocr-input/image-1530.png}\end{center}

\newpage

\noindent (一理之實)而萬分耶?朱子即如此解。〈語類〉卷九十四:「問:〈理性命〉章註云:自其本而之末,則一理之實而萬物分之以為體,故萬物各有一太極。如此,則是太極有分裂乎?曰:本只是一太極,而萬物各有稟受,又自各全具一太極爾。如月在天,只一而已。及散在江湖,則隨處而見,不可謂月已分也。」此解於義理甚是,而「一實」解為「一理之實」則不必是。此亦語法問題,非義理問題。

「萬一各正,小大有定」,此「萬一」一詞亦易引生歧義。若承上文一與萬對言,則此「萬一各正」很顯然是萬與一各正。但「一」若為太極,天道誠體之神,即無所謂正不正,故「各正」若指萬與一言,顯然於義理為不可通。又,此言「各正」,很易使人想到「乾道變化,各正性命」之經文,而「各正性命」之「各」是落在個體上說,不涉指乾道也。萬物性命之得正是因乾道而得正,而乾道本身無所謂正也。若以此經文為根據,則「萬一各正」顯不能指萬與一說。是則「萬一各正」者即由於分於一而成之萬萬個體皆分別各自得正其性命也。「萬一」即個個自己意。因個個皆得各正,故「小大有定」也。「定」即隨「正」來。萬物中,小者因得一而成其為小,大者因得一而成其為大。亦〈繫辭傳〉「易簡而天下之理得矣,天下之理得,而成位乎其中矣」之義也。「成位乎其中」即「小大有定」也,亦即「各正性命」也。此皆就散殊之物說,不涉指誠體、太極之一也。「一」無所謂正不正,亦無所謂小大也。若解為「萬與一各正」,則於義理為不通。然承上文萬與一對言,則又很易使人想為「萬與一各正」也。故此亦是笨拙之滯辭。

\newpage\thispagestyle{empty}\addtocounter{page}{-1}\vspace*{-12mm}\begin{center}\noindent
\includegraphics[clip, trim=180pt 123pt 106pt 248pt, height=162mm]{ocr-input/image-1534.png}\end{center}

\newpage\markright{第二部 \quad 分論一 \quad 第一章 \quad 周濂溪對於道體之體悟}

綜此八語可視為理性命相貫通之積極表示。其中之「一」即理,此理指太極或天道誠體之神說。「性命」則藏於「萬一各正,小大有定」中。理之一妙萬物而為其體,而萬物稟受之,即其自己之「性」也。性通於理之一,故曰性體。統宇宙全體而言之,曰理曰一,此即朱子所謂「統體一太極」。分就各個體言,曰性,此即朱子所謂「物物一太極」。都是太極,故性與理之一不但相貫通,實即是一事也。命兩頭通:一、天道命之之命,天道命之,萬物受之,此言性體之根源;二、性體決定各個體之方向,命其必須如此行方是盡性,此是性之命、命令之命、決定方向之命、定然如此而不可移易之命。理與性是實體地說,命是作用地說,而作用即是性與理之作用也。此即謂天道性命之相重通。在此貫通中,性自是通於理之一(誠體)之性,不會是剛柔氣質之氣性或資性,命自是天道之命或性體之命,不會是壽夭吉凶生死富貴等命運之命或氣命之命,而是道德的命令之命,人受此命令而必然遵循之,無可移易,亦是其命也,此即是正宗儒家道德的理想主義中「性命」一詞中之命也。性命若落於氣上說,則性為氣性,命為氣命。氣命之命是命定主義之命,此即董仲舒、王充、《人物志》等所說之性命,王充所謂「性成命定」也。而其基本原則即告子所謂「生之謂性」,亦即老傳統之「性者生也」。此在宋、明儒即自橫渠開始所說之「氣質之性」。而天道性命之性則是橫渠所說之「天地之性」,後來朱子所謂「義理之性」。此種分別,在濂溪之(通書》中並不顯明。而通於誠體之性,亦未正面直說,而其所顯明地直說者卻是「性者,剛柔善惡中而已矣」,此則于言性有不盡也。然而此〈理性命〉章卻亦很顯明地隱含著此通於誠體之性,亦很顯明地

\newpage\thispagestyle{empty}\addtocounter{page}{-1}\vspace*{-12mm}\begin{center}\noindent
\includegraphics[clip, trim=143pt 132pt 160pt 255pt, height=162mm]{ocr-input/image-1538.png}\end{center}

\newpage

\noindent 可表示出天道性命之相貫通。

\subsection{結語:濂溪之造詣與不足}

綜觀《通書》之思想,如以上所解,有以下幾點當註意:

1.對於天道誠體之神、寂感真幾,有積極的體悟。所謂「默契道妙」者,即在此面有積極的意義。在濂溪之體悟中,天道誠體亦是心、亦是神、亦是理,不是如後來朱子之所分解,天道成為「只是理」,而心神屬於氣。朱子於《通書》之(理性命〉章及(太極圖說〉極有興趣,然卻不以誠體、寂感真幾解太極,此未可謂善紹。此見朱子之心態並不真能自誠體、寂感真幾理解天道,亦不真能契悟「維天之命於穆不已」也。然而寂感真幾、誠體之神卻是濂溪真有得處。

2.自體現誠體之工夫說,必須言及心,而濂溪對於孔子之踐仁以知天,孟子之盡心知性以知天,總之對於孟子之心學,並無真切的理解。彼自〈洪範〉之「思曰睿,睿作聖」以言聖功,不自孟子之「道德的實體性之體義」的心以言聖功,即示其對於心之了解並不真切。如能正視孔子之仁、孟子之心,而真能透徹之,心之「道德的實體性之體的意義」真能挺得起,則自「思」言、自「無欲」言,皆是妙諦,否則皆是隨意之偶然,並無必然。

3.對於通於誠體之性並無積極之正視,對於氣質之性與天地之性(義理之性)之分別亦無顯明之意識。天地之性或通於誠體之性或超越之性如不能挺立起,則「變化剛柔善惡之氣性以使之為合於中道之純善」之工夫便無超越之根據。既挺立矣,而不能通于孟子「道德的實體性之體義」的心而一之,則道德踐履之工夫亦不能真

\newpage\thispagestyle{empty}\addtocounter{page}{-1}\vspace*{-12mm}\begin{center}\noindent
\includegraphics[clip, trim=179pt 113pt 110pt 260pt, height=162mm]{ocr-input/image-1542.png}\end{center}

\newpage\markright{第二部 \quad 分論一 \quad 第一章 \quad 周濂溪對於道體之體悟}

\noindent 切而得其必然。

以上二、三兩點乃濂溪學中之不圓滿者。是以濂溪之造詣猶在觀賞之境界中,只對於誠體有積極之默契,其他尚有待於後來之發展。然此步默契已開一最佳之善端。

\section{關於〈太極圖說〉}

\subsection{「無極而太極,太極動而生陽」解}

〈太極圖說〉全文如下:

\begin{quotation}\kaishu 無極而太極。太極動而生陽·動極而靜,靜而生陰·靜極復
動。一動一靜,互為其根。分陰分陽,兩儀立焉。

陽變陰合,而生水火木金土。五氣順布,四時行焉。

五行一陰陽也,陰陽一太極也,太極本無極也。

五行之生也,各一其性。無極之鼻,二五之精,妙合而凝。
乾道成男,坤道成女。二氣交感,化生萬物。萬物生生,而
變化無窮焉。

惟人也,得其秀而最靈。形既生矣,神發知矣,五性感動,
而善惡分,萬事出矣。

聖人定之以中正仁義,(自注:聖人之道仁義中正而已
矣),而主靜(自注:無欲故靜),立人極焉。故聖人與天
地合其德,日月合其明,四時合其序,鬼神合其吉凶。君子
修之吉,小人悖之凶。故曰:立天之道曰陰與陽,立地之道\end{quotation}

\newpage\thispagestyle{empty}\addtocounter{page}{-1}\vspace*{-12mm}\begin{center}\noindent
\includegraphics[clip, trim=167pt 141pt 133pt 247pt, height=162mm]{ocr-input/image-1546.png}\end{center}

\newpage

\begin{quotation}\kaishu 曰柔與剛,立人之道曰仁與義。又曰:原始反終,故知死生
之說。大哉《易》也,斯其至矣。\end{quotation}

\noindent 案:此〈圖說〉全文,無論思理或語脈,皆同於《通書》,大體是根據〈動靜〉章、(理性命〉章、〈道〉章、(聖學〉章而寫成。「一動一靜,互為其根」,直至「萬物生生,而變化無窮焉」,此一整段同於〈動靜〉章「水陰根陽,火陽根陰」以下以及〈理性命〉章「二氣五行,化生萬物」等八句。「聖人定之以中正仁義而主靜,立人極焉」,依自注觀之,此句同於〈道〉章及〈聖學〉章,是此兩章之簡括。依此觀之,(圖說〉義理骨干不外此四章,不可謂非濂溪之手筆也。惟有一點不同於《通書》,此即「無極而太極,太極動而生陽」兩句是。《通書〉只言「太極」,未有「無極」;《通書〉未有「太極動而生陽」之觀念。此兩句皆生問題。然此所謂「未有」者是字面上說「未有」,義理上是否一定不為其所意許,或不為其所函蘊?此則未可遽斷。

依吾觀之,「無極而太極」一語是對於「太極」本身之體會問題,本是一事,加「無極」以形容之,本無不可。太極是正面字眼,無極是負面字眼。似可說太極是對於道體之表詮,無極是對於道體之遮詮。太極是實體詞,無極是狀詞,實只是無聲無臭、無形無狀、無方所(神無方)、無定體(易無體)、一無所有之「寂然不動感而遂通」寂感一如之誠體本身,而此即是極至之理,故曰「無極而太極」,此語意不是無極與太極。「無極」一詞雖出於《老子〉「知其白,守其黑,爲天下式。為天下式,常德不忒,復歸於無極」(王弼本,〈二十八章〉),然《老子》之使用此詞亦

\newpage\thispagestyle{empty}\addtocounter{page}{-1}\vspace*{-12mm}\begin{center}\noindent
\includegraphics[clip, trim=155pt 119pt 142pt 257pt, height=162mm]{ocr-input/image-1550.png}\end{center}

\newpage\markright{第二部 \quad 分論一 \quad 第一章 \quad 周濂溪對於道體之體悟}

\noindent 是狀詞意,故王弼注云:「不可窮也」,言無可窮極也。《老子》此章前言「復歸於嬰兒」,後言「復歸於樸」。嬰兒、無極、樸,皆示無為渾玄之境,而此即是極至之道也。〈五十八章〉「孰知其極,其無正」,王弼注云:「言誰知善治之極乎?唯無可正舉,無可形名,悶悶然而天下大化,是其極也。」(此承此章首句「其政悶悶,其民淳淳」而來,故如此注。)此亦是由「無可正舉、無可形名」之渾玄之無而顯極至之道。此是極通常之思路,不可以出於《老子》,便不可用,如此思維,亦不足以決定即是道家之思想。《老子》言「無欲」,(「我無欲而民自樸」),儒者即不可言「無欲」乎?依儒家,自然神化之道體自是「無思、無為」「無方、無體」之無可窮極也。故言道而至「不可度思,矧可射思」,無極之極乃必然不可免者。故「無極」之形容(默識)亦如王弼所謂「無稱之言、窮極之辭」之意也。以「無稱之言窮極之辭」顯其為極至之理,此即是「無極而太極」。太極是表,無極是遮。太極之所以為極至之理正因其「無可正舉、無可形名」而為至極也。此即是無極之極。以無可窮究其何所極至為極至也。若得究其何所極至,如至於某處,則即為相對有限定之物矣。相對有限定之物,焉得為太極?故無極是無有窮極之遮狀字,而太極則是如此遮狀下之表詞也。兩者正是一事。「無極而太極」意即「無極之極」,非無極與太極也。而無極亦非沒有太極之意也。「無極」中之「極」字意許為限定之極,「太極」中之極字是無限定之極。遮彼限定之極而顯其為無限定之極,此即是「太極」,此即是絕對的最後者。此種無極之極亦須曲線地(辯證地詭辭地)由默識其無方所之渾圓而展示其為極至之理。如是,「無極而太極」一語,如譯成完整的語

\newpage\thispagestyle{empty}\addtocounter{page}{-1}\vspace*{-12mm}\begin{center}\noindent
\includegraphics[clip, trim=171pt 155pt 122pt 227pt, height=162mm]{ocr-input/image-1554.png}\end{center}

\newpage

\noindent 體語句,當為:那無限定的而一無所有者但卻亦即是極至之理。如明此意,則單說無極亦可,如下文「無極之真,二五之精,妙合而凝」,即無「太極」字,「無極之真」即太極也;單說太極亦可,如《通書·動靜〉章「五行陰陽,陰陽太極」,即只說太極,而不說無極:無論單說那一面,而只是渾圓之一,故只說「一」字,亦可表示此無極之極,如《通書·理性命》章「五殊二實,二本則一」,此中亦無無極,亦無太極,但此一字即是太極,即是無極之極;如將此一字詳細展示,則太極無極俱說亦可,如此〈圖說〉下文「五行一陰陽也,陰陽一太極也,太極本無極也」,即先有太極之表,後有無極之遮。

「太極動而生陽」一語之問題,則是太極如何能動?如依濂溪《通書》之思理,此語本不生問題,但須要有一說明。但如依朱子之分解的理解,則此語有問題,如是,將太極之理完全表象為另一系統,此則不合《通書》之原意,亦不合《中庸》、《易傳〉之言誠體之神與寂感真幾,亦不能恢復「維天之命,於穆不已」之最原初的智慧。朱子之分解的理解下文明之,茲先依《通書》以明其不生問題之意。

如依《通書〉誠體之神解太極,則「無極而太極,太極動而生陽」兩語實即《通書·誠下第二》言誠體「靜無而動有」一語之引申。「靜無」即無極而太極,「動有」即太極動而生陽。「靜無」之靜與「動有」之動對言,非「動而無動,靜而無靜,神也」之靜。「動而無動、靜而無靜」中之動靜字是用來曲線地表誠體之自身,是對於誠體自身之曲線地體會,而「靜無」之靜是所謂「時也」,即言靜時以顯誠體之無聲無臭而為無極之極也,而「動有」

\newpage\thispagestyle{empty}\addtocounter{page}{-1}\vspace*{-12mm}\begin{center}\noindent
\includegraphics[clip, trim=150pt 133pt 143pt 242pt, height=162mm]{ocr-input/image-1558.png}\end{center}

\newpage\markright{第二部 \quad 分論一 \quad 第一章 \quad 周濂溪對於道體之體悟}

\noindent 之動亦是「時也」,即言動時則顯其落於有之範圍分化而為或呈現出動靜之相也,即有方所有形體之相也。靜時雖顯誠體之自身,而動時之動即定動矣。定動之動有動相,故是有也,此即「動而生陽」也。「動極而靜」,靜亦是定靜。定靜之靜有靜相,此即是「靜而生陰」也,亦是有也。(「生」是解說上之引出義,非客觀實事之出生義,動相之動即是陽,靜相之靜即是陰)。然則,此動相之動、靜相之靜,如何能從「動而無動,靜而無靜」之誠體之神而說明(而引出)?即,由「靜無」如何能說其「動有」?

誠體之「動而無動」非實是「不動」也,只是不顯動相而已。茲順其不動之動而若一露動相即是陽之有。一露動相即是限定於動。一限定於動,即是氣邊事,非神之自身也。「靜而無靜」非實是「不靜」也,只是不顯靜相而已。茲順其無靜之靜而若一露靜相即是陰之有,一露靜相即是限定於靜。一限定於靜,亦是氣邊事,非神之自身也。順「太極動而生陽,動極而靜」,這樣順著說來亦可;順「太極靜而生陰,靜極而動」,這樣順著說下來亦可。

若問此「動而無動、靜而無靜」之誠體之神用何以必露動相乃至靜相,此則不是直線思考所可能解答者。夫言「動而無動靜而無靜」之神正是自其具體感應中言之,非是隔離地言一個「動而無動,靜而無靜」之神也。在具體感應中以見其為具體的神用,然雖神也,而不能不有跡。自跡而觀之,則動是動,靜是靜,乃陰陽氣邊事;但自神之自體而觀之,則動而無動靜而無靜,不失其虛靈之純一,仍是神而不是氣也。《莊子·大宗師》云:「其一也一,其不一也一。其一也,與天為徒;其不一也,與人為徒。天與人不相勝也,是之謂真人。」此真人之境界亦可說「大而化之」之聖的境

\newpage\thispagestyle{empty}\addtocounter{page}{-1}\vspace*{-12mm}\begin{center}\noindent
\includegraphics[clip, trim=178pt 143pt 132pt 255pt, height=162mm]{ocr-input/image-1562.png}\end{center}

\newpage

\noindent 界。「其一也一」,是說神之自體。「其不一也一」,是說神之在具體感應中。在具體感應中,自跡而觀之,則不一。但神用之自體則不隨此不一而不一。如隨其不一而不一,則喪失其神而全物化而爲氣矣。此即不得為「大而化之」之聖。然「大而化之」之聖非空懸,一切現實生活之所有皆是其內容,此則不能不有「跡」。跡亦是具體之神之妙用中所必資而亦必帶有者也。善乎郭象之注《莊》曰:「今仲尼非不冥也。顧自然之理,行則影從,言則響隨。夫順物,則名跡斯立,而順物者非為名也。非為名,則至矣,而終不免乎名。則孰能解之哉?」(〈德充符〉:「天刑之,安可解」注)。不能解除,即是具體之神之妙用中所必資而亦必帶有者。所必資而亦必帶有,則其動而無動不能不隨跡而顯動相,其靜而無靜亦不能不隨跡而顯靜相,必矣。不是它要顯動相顯靜相,乃是順物隨迹所自然有者。相由迹顯,非由神顯也。妙在雖隨迹而顯相,而卻不滯執於相,此即其所以為神也,亦即「其不一也一」也。能不滯執于相,故亦能妙此相而使相亦不死。能使相亦不死,故能妙之而使其動了又靜,靜了又動而生化不窮也。若是泯相歸己,則雖相而無相;此時一切即一,全體是神,全用是體。若是物各付物而不滯執,則一即一切;此時無相而亦相,全體是跡,全體是用。此種圓融之境即是「大而化之」之聖境,宋、明儒中唯明道最能默識而雅言之。堯之不為許由,即全體是跡,而亦是全體是神。孔子之與人爲徒,亦全體是跡,而亦全體是神也。郭註已盛發之矣。彼雖用道家詞語以明之,然此種圓融之思理固是儒道之所共,非是道家所可得而專也。

然此圓融之境非即不可有分解表示上神與氣之分也。又,「太

\newpage\thispagestyle{empty}\addtocounter{page}{-1}\vspace*{-12mm}\begin{center}\noindent
\includegraphics[clip, trim=148pt 119pt 146pt 258pt, height=162mm]{ocr-input/image-1566.png}\end{center}

\newpage\markright{第二部 \quad 分論一 \quad 第一章 \quad 周濂溪對於道體之體悟}

\noindent 極動而生陽」,或「靜而生陰」,亦不可表面地徒順其字面之次序而空頭地視為外在之直線的宇宙演化而解之。若如此,則鮮能得其實義。此種動而生陽或靜而生陰,其實義毋寧是本體論的妙用義,而不是直線的宇宙論的演生義。即或有宇宙論的演生義,亦應統攝於本體論的妙用中而體會之,如此方能相應儒家形上之智慧(「維天之命於穆不已」)之智慧)而不迷失。其動而生陽實只是在其具體妙用中隨跡上之該動而顯動相,靜亦如之。非真是由其自身直線地能動而生出陽或靜而生出陰也。唯把握此本體論的妙用義,「全體是神,全體是跡」之圓融如一始能言。「五行一陰陽,陰陽一太極,太極本無極」,字面上是直線的收縮或開展,而其真實義則是「是萬為一,一實萬分」之本體論的妙用。「神也者,妙萬物而為言」,亦即遍體萬物而不遺。是則神無所不在而亦無一在,故能有此圓融之理境,否則永無得圓也。此種思理與圓境乃儒釋道之所共至,亦是其玄理之共同型態。〔王陽明《傳習錄》卷二〈答陸原靜書〉:「周子靜極而動之說,苟不善觀,亦未免有病。蓋其意從太極動而生陽、靜而生陰說來。太極生生之理,妙用無息而常體不易。太極之生生即陰陽之生生。就其生生之中,指其妙用無息者而謂之動,謂之陽之生。非謂動而後生陽也。就其生生之中指其常體不易者而謂之靜,謂之陰之生。非謂靜而後生陰也。若果靜而後生陰,動而後生陽,則是陰陽動靜截然各自爲一物矣。陰陽一氣也,一氣屈伸而為陰陽。動靜一理也,一理隱顯而為動靜。春夏可以為陽爲動,而未嘗無陰與靜也。秋冬可以為陰為靜,而未嘗無陽與動也。春夏此不息,秋冬此不息,皆可謂之陽,謂之動也。春夏此常體,秋冬此常體,皆可謂之陰、謂之靜也」。案:此解自成一義,

\newpage\thispagestyle{empty}\addtocounter{page}{-1}\vspace*{-12mm}\begin{center}\noindent
\includegraphics[clip, trim=182pt 162pt 122pt 230pt, height=162mm]{ocr-input/image-1570.png}\end{center}

\newpage

\noindent 非周子意,亦不足以遮撥「靜而後生陰,動而後而陽」之說。陽明就「妙用無息」說動說陽,就「常體不易」說陰說靜,此是陰陽動靜之借用(引喩),而不是說明陰陽動靜本身之實。例如「全體是神」是靜是陰,「全體是跡」便可是動是陽。此顯是引喩,而不足以解周子之文,亦不足明「動而生陽,靜而生陰」之非。陽明所說只是體用不二之圓融義。吾人不能就體說陰自身之實,就用說陽自身之實。故陽明就「常體不易」說陰說靜,只是譬況引喻之辭。若視此為對於周子文之善觀,則非是。視之爲諦解,尤非是。]

\subsection{「太極動而生陽」之確義}

動而無動,靜而無靜之誠體神用在其具體妙用中,即在其順物之感應中,隨跡上之該動而顯動相,隨跡上之該靜而顯靜相,此義尚可仔細解析一下。

孔子說:「張而不弛,文武弗能也。弛而不張,文武弗能也。一張一弛,文武之道也。」(《禮記·雜記下》)。誠體神用本身,無論是主觀地從聖人之誠體說,或是客觀地從天道之誠體說,皆是無所謂張弛的。張弛是現實生活上的事。張是動相,弛是靜相。人的現實生活,自然生命所呈現者,事實上自然要有這樣的一些波浪與曲折。到該張時,自然要張。你若禁止他,不讓他張,他便壓抑在那裡不得舒展,那是會出毛病的。甚至就是壓抑,也壓抑不住。到休息時(弛),自然要休息。你若不讓他休息,他便會筋疲力盡,自然要倒下了。就是拿鞭子趕打,也趕打不起。這是自然生命底限度。因這限度,現實生活自然要有一些曲折與波浪。這些曲折與波浪即是所謂跡或事,也可以說是軌跡。限度到了,到了該

\newpage\thispagestyle{empty}\addtocounter{page}{-1}\vspace*{-12mm}\begin{center}\noindent
\includegraphics[clip, trim=143pt 138pt 157pt 242pt, height=162mm]{ocr-input/image-1574.png}\end{center}

\newpage\markright{第二部 \quad 分論一 \quad 第一章 \quad 周濂溪對於道體之體悟}

\noindent 動或該靜的時候,便是所謂跡上之該動、跡上之該靜。(這個「該」是自然上的,不是道德上的。有時這兩者相順應,但卻不能同一說)。跡上之該動而動就是自然上的陽,跡上之該靜而靜就是自然上的陰。(具體地說是事,抽象地說是氣——陰陽之氣)。這本是自然生命之限度所自然有的,似乎不必由太極之動或靜才有,由誠體之神用露動相或靜相才有。然則為甚麼要說「太極動而生陽」呢?為甚麼要高一層從太極上或從誠體上說呢?

蓋若只是順自然生命之自然的張弛滾下去,那張弛常不必是合理有度的,常有一張張下去,而不知節,到發狂而死者,此卽不成其為張;亦常有一弛弛下去而不知振,直弛到腐爛而死者,此亦不成其為弛。此皆所謂縱欲敗度者是。故順其限度之該張該弛而欲成全之,不能不提升一層從精神生命之超越誠體上說。

從超越層上說下來,說動而生陽,靜而生陰,或說張而有動,弛而有靜,此所謂「生」或「有」乃是成全地生或有,不是說自然生命連同其曲折與波浪皆是存在地由誠體而生出也。此即我警告說:不可表面地徒順其字面之次序,而空頭地視為外在之直線的宇宙演化而解之。太極、誠體之神之動而生陽,靜而生陰,或張而有動,弛而有靜,是成全地生、成全地有。在誠體之神順物感應之具體妙用中,它順跡上之該動該靜其自身不能不相應而起綁縐,此即小詞家所謂「吹縐一池春水」也。一池春水本是動而無動,靜而無靜的。然而春風一起,則不能不應之而起絕絕。這活靈的春水相應風吹便成全了那些如此如此之縐綴。這些縐絕雖是應風而起,卻也是其自身之所起。因其為其自身之所起,所以也可以說為為其自身之所具——此之謂因起而具。你可以說,這春水本身本無所謂起或

\newpage\thispagestyle{empty}\addtocounter{page}{-1}\vspace*{-12mm}\begin{center}\noindent
\includegraphics[clip, trim=157pt 165pt 145pt 221pt, height=162mm]{ocr-input/image-1578.png}\end{center}

\newpage

\noindent 具,即這些綴絕本不是這春水本身所本有的。但這只是抽象地(隔離地)說這春水本身,而事實上這春水本身卻永遠是在具體的處境中,所以若是具體地圓融地說這春水本身,它必永遠是有起有具的,它必永遠帶著這些縐縐而不離的。縐縐不礙春水活靈之一,而活靈之一,也不礙其為絕綴之多,兩者相即相融而多姿多采。但是,必須這是活靈之水始可。若是一塊平平的木板或大理石,則雖有風吹,也不能應之而起縱縐。這平平的木板或大理石,只能說它是定靜之一,而不能說它是動而無動靜而無靜之活靈之一。定靜之一是不能說動而生陽的,但活靈之一卻可以說。活靈之一即象徵所謂神而不是氣,那絕縐之多即象徵所謂氣(事、跡)而不是神——此是分解地說。若圓融地說,則全神是氣,全氣是神,相即相融,永永為一。明道云:「氣外無神,神外無氣。或者謂清者神,則濁者非神乎?」此即圓融地說。(謂「清者神」者,意在說濁者是氣。在此,清者神亦是氣之清,神屬於氣,不是誠體之神。)

誠體之神順物隨跡而顯動靜相,可由春水縐縐而喻解。誠體之神,抽象地說,它自身是動而無動靜而無靜之神,但它永遠是在具體感應中顯示其活靈之一,故其動而無動靜而無靜之神永是在具體妙用中顯示,故亦是具體地永遠帶著那些動靜跡象而為動而無動靜而無靜之神——活靈之一。順其動而無動之動,在該動而動之跡上即必然地了起來而顯為動相。順其靜而無靜之靜,在該靜而靜之跡上亦必然地了起來而顯為靜相。若隔離地(抽象地)說它本身,則它只是動而無動靜而無靜之活靈之一,(此亦是隔離之一),是無所謂動靜相的;但若具體地(.圓融地)說這活靈之一,則它便必隨跡而顯動靜相。因為它是活靈之一,所以它是能顯這動

\newpage\thispagestyle{empty}\addtocounter{page}{-1}\vspace*{-12mm}\begin{center}\noindent
\includegraphics[clip, trim=170pt 147pt 135pt 233pt, height=162mm]{ocr-input/image-1582.png}\end{center}

\newpage\markright{第二部 \quad 分論一 \quad 第一章 \quad 周濂溪對於道體之體悟}

\noindent 靜之相的。它應迹而顯動靜相,即以其神而成全了跡上之動靜之事。它隨跡之動靜而顯出動靜相,剋就此動靜相本身說,這是氣。因為這動靜相不就是那活靈之一,雖然是活靈之一之所起。這動靜相與跡上之該動而動該靜而靜之事可以客觀地等同為一,一起皆是神之跡。從誠體之神活靈之一方面說,是跡;從客觀存在方面說,是事。這些——迹或事—都是神之妙用之所起、之所創生。說到最後,是成全的創生,是創生的成全。離了這成全的創生,也別無創生。一切跡與事皆統於主也。這主之成全之,也確是創生成全之。因為它是活靈之一也。試想若不是有這活靈之一,妙以成全之,那跡上該動該靜者若一味順自然生命滾下去,而至於發狂而死,或腐爛而死,則將何有存在之事之可言乎?使存在之事永遠生息下去而不至於梏亡,這便是對于存在之創造。這就是《中庸〉所謂天道之誠之「生物不測」也。這是通過本體論的妙用而顯的創生,也可以說是依誠體之神而來的形式的創生——成全事為一必然的實有而不只是一偶然的存在之創生。

譬如「張而不弛,文武弗能也」,隨跡上之該張,誠體之神即應之而成全其該張,而亦因之而顯張相,此即是動之生,動之陽。弛亦如之。就此動生動陽、陰生陰靜之跡說,全體是用,全神是迹,而一即一切,此即是物各付物,全體敞開,而神不虛懸。聖人亦不能無動靜張弛乃至喜怒哀樂以與民共之。然而其誠體之神卻不因此動靜張弛喜怒哀樂之相而徇物滯相而喪失其活靈之一。是故就其誠體之神自身說,仍是動而無動靜而無靜的,此即是全用是體,全跡是神,而一切即一,動靜張弛喜怒哀樂於密勿也。動靜張弛喜怒哀樂於密勿,則一切跡相皆為一神理之所貫,而成就其為動靜張

\newpage\thispagestyle{empty}\addtocounter{page}{-1}\vspace*{-12mm}\begin{center}\noindent
\includegraphics[clip, trim=169pt 163pt 152pt 240pt, height=162mm]{ocr-input/image-1586.png}\end{center}

\newpage

\noindent 弛喜怒哀樂之實事活事而生生不息焉。貫澈而端正此張弛等實事之理(普遍的律則)是神用應跡之所獨發。自應跡當機言,是特殊的,而所發之律則則總是異地則皆然之普遞的,此是具體的普遍,而非抽象的普遍。神用獨發律則以應之,故神用之妙之即是律以則之而不可亂。神理一也,而存在之事亦得其必然而非偶然。抽象地從律則本身說,是客觀的存有,而律則之根在神用,是以具體地說,即融於神用中而與寂感神用為一事。即神用即存有,兩者不分能所而立體地直貫於張弛等之實事。此若用孟子之話說,即是本心之「沛然莫之能禦」也;宇宙論地說,則是「維天之命於穆不已」,天道之誠之「生物不測」也。文武之道即是此誠體之神之張弛於密勿也,亦是「文王之德之純,純亦不已」也,亦是孔子之仁體之遍潤萬物而不遺也。

吾相信若根據《通書〉誠體之「靜無而動有」以解「無極而太極,太極動而生陽」,則以上之解析是一恰當之解析。如此解析則可符合於孔、孟、《中庸》、《易傳》之立體直貢型的道德創生之實義,亦更能符合於「維天之命,於穆不已」這一根源的智慧。濂溪雖根據《中庸》、《易傳》言誠體,不直接根據孟子言本心,然此誠體斷然是心神理合一的,決不會抽掉了心神寂感而只是理。縱然此心神是宇宙論的意味重,不似孟子之直從道德的心性言本心,然此誠體必然地含有「心」義則無疑,否則不能說神說寂感。心神寂感是一,皆可分析而得,而理則是此本體論的心神寂感之所獨發,亦即其自主自律性之直接顯示,故心神寂感理是一,任從一面皆可分析而得其他。心神寂感理合而為一便是誠體。而若知《中庸》、《易傳》之所言乃是繼承孔孟之所言而發展至充其極,則知

\newpage\thispagestyle{empty}\addtocounter{page}{-1}\vspace*{-12mm}\begin{center}\noindent
\includegraphics[clip, trim=173pt 139pt 135pt 245pt, height=162mm]{ocr-input/image-1590.png}\end{center}

\newpage\markright{第二部 \quad 分論一 \quad 第一章 \quad 周濂溪對於道體之體悟}

\noindent 《中庸》、《易傳》之誠體即是孔子之仁、孟子之心性之擴大,其內容完全是一。如此,則可無疑於誠體之為心神寂感理之合一。只因濂溪,橫渠亦在內,是從《中庸》、《易傳》開始,不甚能提挈之以孔子之仁與孟子之心性,遂易使人有割截之想法。然謂其誠體,甚至橫渠之太和太虛,不含有宇宙論意義的「心」義不可得也。至明道,則孔、孟、《中庸》、《易傳》完全合而為一矣。

\subsection{朱子理會「太極之理」之偏差}

然依朱子之分解,則不能有此義。朱子分解中之問題,不在理氣之分與理先氣後,乃在其對於太極之理不依據《通書》之誠體之神與寂感真幾而理解之。朱子之理解是依據伊川對於「一陰一陽之謂道」之分解表示而進行。伊川云:「一陰一陽之謂道。道非陰陽也,所以一陰一陽道也。」又云:「離了陰陽更無道。所以陰陽者是道也,陰陽氣也。氣是形而下者,道是形而上者。形而上者則是密也。」此「陰陽氣,所以陰陽是道」之分解表象嚴格地為朱子所遵守。此思路很清楚很邏輯。但光此「所以陰陽是道」之形式陳述,尚不足以決定太極不能動。因為對此超越的所以然所顯示之道本身可有不同的體悟。依朱子對於此道本身之體悟,道只是理,而心、神俱屬於氣。氣是形而下者,理是形而上者。經過這一分別之限制,道之為理只是一光禿禿之理,是抽象地只是理——但理。道只是理,太極亦只是理。太極本身渾然含具一切理。若光是由「超越的所以然」之形式陳述以顯示形而上之理,則此形而上之理亦不必即是抽象地只是理,它亦可是心神理合一的誠體之神、寂感真幾。由「所以然」之形式陳述所顯示的理假定吾人肯認此理為誠體

\newpage\thispagestyle{empty}\addtocounter{page}{-1}\vspace*{-12mm}\begin{center}\noindent
\includegraphics[clip, trim=169pt 152pt 125pt 236pt, height=162mm]{ocr-input/image-1594.png}\end{center}

\newpage

\noindent 之神,則此誠體之神之為理與此誠體本身之內容「亦是心、亦是神、亦是理」中之理不同。朱子把這兩層意義之理等同為一,而把超越的所以然之形式陳述所顯示的形上之理只看成是作為誠體內容之一的那個理,而心神俱抽掉而視為氣,如是超越的所以然所顯示之形上之理成為抽象地「只是理」(但理),而道與太極遂不可為誠體,而只成了「只是理」,而「維天之命於穆不已」之智慧亦脫落而不可見。此一直線之分解思考之清楚割截自然形成「太極不能動」之結論。此結論之出現亦可說是很邏輯的,此直線的分解思考之清楚割截所確定的「但理」是超越的靜態的所以然,而不是超越的動態的所以然。此靜態的所以然之形上之理只擺在那裡,只擺在氣後面而規律之以為其超越的所以然,而實際在生者化者變者動者俱是氣,而超越的所以然之形上之理卻並無創生妙運之神用。此是朱子之思路也。在此思路下,太極不能動,理不能動。「太極動而生陽」一語便不可通。朱子思路之推致至此結果並非偶然,亦非是後人深文周納而至者,亦非當時門人或從之問學者偶爾一問偶然想及有此一義,乃是朱子之思理與措辭所必然函有者,故當時對此問題討論獨多,乃是大家所俱容易想及者。只因朱子為(太極圖說〉作註解,要粘附著濂溪原文而說,故其辭語多模稜而有歧義,人遂不易辨,然由於其討論將此問題挑破,其真意與實義固甚顯,而其所以如此措辭亦有其思理之必然,而凡遇有模稷而可以左右解者亦可得而確定矣。

朱子〈太極圖解〉曰:

\begin{quotation}\kaishu ,此所謂無極而太極也。所以動而陽靜而陰之本體也。\end{quotation}

\newpage\thispagestyle{empty}\addtocounter{page}{-1}\vspace*{-12mm}\begin{center}\noindent
\includegraphics[clip, trim=147pt 130pt 143pt 247pt, height=162mm]{ocr-input/image-1598.png}\end{center}

\newpage\markright{第二部 \quad 分論一 \quad 第一章 \quad 周濂溪對於道體之體悟}

\begin{quotation}\kaishu (原註:太極理也,陰陽氣也。氣之所以能動靜者,理為之
宰也。)然非有以離乎陰陽也(原註:道不離氣),即陰陽
而指其本體(原註:器中之道),不雜乎陰陽而為言耳。
(原注:道是道,器是器。已上三句要離合看之,方得分
明。)〔案:已上只兩整句,「即陰陽而指其本體」非是一
整句。】

,此之動而陽靜而陰也。中者,其本體也。(原注:
即第一層之太極也。),陽之動也,之用所以行也。
2,陰之静也,之體所以立也。\end{quotation}

\noindent 案:其解首爲「無極而太極」,實即是太極。太極者乃「所以動而陽、靜而陰之本體也」,此句即等於說:太極是所以陰陽之理,亦等於說:太極是所以動之理、所以靜之理。動而陽靜而陰,是氣;而所以然則是理。「所以然」之理是無所謂動靜的「只是理」,不是誠體,神體之心神理是一之理。然則不是「太極動而生陽」,乃是太極是氣之所以能動能靜之理,而理則是不動不靜者。末兩整句是言理氣不離不雜。「理不離氣」是就理為氣之所以然說,正在氣之然處見所以然。「氣不離理」是就氣之然必有其所以然說,正在理之定然之處見氣之然。朱子解說「理不離氣」,原則地言之,乃是認理為氣之本體,本體自不能離開其所依之以為本體者。至於喻解地言之,則說理無氣,亦無掛搭處,或無附著處。說到氣不離理,則原則地言之,一個東西自不能離開其本體,除此以外,朱子再無解說。此似無須再有解說者。但若套用來布尼茲之語說之,則更顯明,即:氣若無理,則不能知其「存在之然」何以單

\newpage\thispagestyle{empty}\addtocounter{page}{-1}\vspace*{-12mm}\begin{center}\noindent
\includegraphics[clip, trim=170pt 164pt 135pt 228pt, height=162mm]{ocr-input/image-1602.png}\end{center}

\newpage

\noindent 如此而不如彼。理氣雖不相離,然亦不相雜:理是理,氣是氣。理氣不離不雜不能決定理必為只是理。理之成為「只是理」是對於理(太極)本身的體會問題。分解地言之,即使對於理體會為心神理是一之理,亦仍可說不離不雜也。

其解陽動陰靜圈說:「此〔太極〕之動而陽靜而陰也」。此解語中雖有「之」字,然卻不能視「動而陽靜而陰」為太極底動而陽靜而陰,即不能是太極在動而生陽、靜而生陰。依太極為氣之所以能動能靜之理,此解語之意只能是:氣之動是依動之理而動,故動逐繫屬於理(太極)而為理所領有矣。此只是「氣之動」之統屬于太極下,而太極(理)不動也。氣之靜亦然。故「太極之動而陽靜而陰」只能是統屬於太極下的「氣之動而陽,氣之靜而陰」也。然因為要依附濂溪原文而說,遂說成「太極之動而陽靜而陰」。其解「陽之動」則說此是「太極()之用所以行也」。「太極之用」亦當視為統屬於太極下的動用,即由陽之動可以見統屬於太極下的氣之發用。氣之發用,因統屬於太極下,遂為太極所領有,因而遂說為「太極之用」矣。實則只是氣依動之理而為動用也。其解「陰之靜」則說此是「太極()之體所以立也」,言由氣之陰靜可以見太極之自體也。此不是說陰靜是體,陽動是用,乃是說由氣之陰靜可以見太極之自體,由氣之陽動可以見統屬於太極下的發用。黃梨洲誤認為「陰靜是體,陽動是用」,非是。因有此誤認,故云:「夫太極爲之體,則陰陽皆是其用。如天之春夏陽也,秋冬陰也。人之呼陽也,吸陰也。寧可以春夏與呼為用,秋冬與吸爲體哉?」(《宋元學案·濂溪學案下〉,案語)。此種誤認非朱子意也。朱子在此雖說由陰之靜可以見太極之自體(太極之體所以

\newpage\thispagestyle{empty}\addtocounter{page}{-1}\vspace*{-12mm}\begin{center}\noindent
\includegraphics[clip, trim=151pt 143pt 139pt 231pt, height=162mm]{ocr-input/image-1606.png}\end{center}

\newpage\markright{第二部 \quad 分論一 \quad 第一章 \quad 周濂溪對於道體之體悟}

\noindent 立),由陽之動可以見統屬於太極下的發用(太極之用所以行),然如果要說用,朱子亦可說陽之動與陰之靜皆是統屬於太極下之用,不過用有動用(發散)靜用(收歛)之別耳。氣之收斂凝聚而為陰之靜亦有其所以為靜之理,此理亦仍是太極。故氣之靜用亦可統屬於太極下而謂為是太極所統馭的靜用。故嚴格言之,無論動用靜用皆直接是氣之用,而不能說是太極之用,只能說是統馭於太極下的用。若以一般體用義言之,只能說氣是體,動靜二相(二機)是其用,而不能說太極是體,動靜二相(二機)是其用;而如果要說是太極之用,則只因太極是其超越的,亦是靜態的所以然之理,故皆統屬於太極而為太極所主耳。是則由氣之動靜二用到其因統馭於太極而謂為是太極之用,這其間有一種曲折的轉進;正因這曲折的轉進所成的曲折的統屬關係,遂使太極為體、動靜為用不是一般意義的體用,因而在此統屬關係上說體用亦成不顯明不恰當者。朱子初亦有以太極為體、動靜為用之說,但因此說不恰當(朱子謂為「有病」),故為表示其心中所意謂之理氣之恰當關係,遂改說為「動靜是理〔太極〕所乘之機」(見下),但仍保留陽之動是「太極之用所以行」這一模稜隱晦的體用義,而於陰之靜則說成是「太極之體所以立」,並不說其是太極之用也。此即表示並不於太極與動靜上直接說體用,而於陽之動之為「太極之用所以行」亦須有特別之解析,並不可直說為是太極之用也。但是如果體會太極為誠體神體,即心神理是一之體,則體用正是恰當的說法,進而說體用圓融體用不二、全體是用、全用是體等妙義亦正是恰當而應有之妙義,而「動靜為理所乘之機」倒反不恰當而用不上。是則朱子說「動靜是理所乘之機」正是其思理所必有之措辭,而亦是其視理

\newpage\thispagestyle{empty}\addtocounter{page}{-1}\vspace*{-12mm}\begin{center}\noindent
\includegraphics[clip, trim=158pt 167pt 147pt 230pt, height=162mm]{ocr-input/image-1610.png}\end{center}

\newpage

\noindent (太極)為只是理而無所謂動靜之標識。此非濂溪之本義也,濂溪明說誠體神體「靜無而動有,至正而明達」,亦明說「動而無動,靜而無靜,非不動不靜也。」,此即明示誠體神體可以動靜去說,只是動是「動而無動」之動,靜是「靜而無靜」之靜,而非是氣上「動而無靜」之動、「靜而無動」之靜之相對之動靜,但非無所謂動靜之「不動不靜」也。

朱子注〈太極圖說〉「無極而太極」一語云:

\begin{quotation}\kaishu 上天之載無聲無臭,而實造化之樞紐品彙之根柢也,故曰
無極而太極。\end{quotation}

\noindent 案:此是根據聖教傳統以「上天之載」說太極也。但《詩》、《書》中「上天之載」即是「於穆不已」之天命,後來進一步說為即是「為物不貳生物不測」之天道,此皆是意謂「上天之載」是「心神理是一」之實體,即,「即存有即活動」之實體,而非「只是理」者,亦非「只存有而不活動」者。明道體會此「上天之載」則說「其體則謂之易,其用則謂之神,其理則謂之道」,亦是即存有即活動者,是心是神亦是理者,但卻非「只是理」,亦非只存有而不活動者。朱子此注語,從語句上說無問題,但其心中所意解之太極之實義卻是「只是理」,只存有而不活動者。此則不合聖教傳統之古義,亦不合濂溪之原義,亦不合明道之所體會。朱子對於明道之體會明將易體與神用俱視為氣,如是,「上天之載」即成只是理。此見下〈明道章〉。

其注〈太極圖說〉「太極動而生陽」一段云:

\newpage\thispagestyle{empty}\addtocounter{page}{-1}\vspace*{-12mm}\begin{center}\noindent
\includegraphics[clip, trim=181pt 144pt 125pt 243pt, height=162mm]{ocr-input/image-1614.png}\end{center}

\newpage\markright{第二部 \quad 分論一 \quad 第一章 \quad 周濂溪對於道體之體悟}

\begin{quotation}\kaishu 太極之有動靜是天命之流行也。所謂一陰一陽之謂道、誠者
聖人之本物之終始、而命之道也。其動也,誠之通也,繼
之者善,萬物之所資以始也。其靜也,誠之復也,成之者
性,萬物各正其性命也。動極而靜,靜極復動,一動一靜,
互為其根,命之所以流行而不已也。動而生陽,靜而生陰,
分陰分陽,兩儀立焉,分之所以一定而不移也。蓋太極者,
本然之妙也。動靜者,所乘之機也。太極,形而上之道也。
陰陽,形而下之器也。是以自其著者而觀之,則動靜不同
時,陰陽不同位,而太極無不在焉。自其微者而觀之,則沖
穆無朕,而動靜陰陽之理已悉具於其中矣。雖然,推之於
前,而不見其始之合,引之於後,而不見其終之離也。故程
子曰:動靜無端,陰陽無始。非知道者、孰能識之?\end{quotation}

\noindent 案:此註文極整飭。然於其中辭語,必須加以簡別,方能得其實義。首先「太極之有動靜」,此「有」是「因氣之動靜統馭於太極,故太極領有之」之「有」,並不是太極會動會靜因而說有動靜也。故嚴格言之,只是氣會動會靜因而可說氣有動靜,而太極只是氣之所以動靜之理,故其自身亦無所謂動靜也。如是,「太極之有動靜」與「氣之有動靜」,兩「有」字不可作一律解也。關於「太極有動靜」,朱子之表示有一轉進之發展。首先說「太極有動靜」,「太極函動靜」。「有動靜」是就流行而言,「函動靜」是就本體而言。然無論是「有」或「函」,皆當解為氣之動靜因統馭於太極故太極領有之或函具之。次則進一步說太極無所謂動靜,但有動之理與靜之理,因此說太極函眾理,具萬理。最後,即此「有

\newpage\thispagestyle{empty}\addtocounter{page}{-1}\vspace*{-12mm}\begin{center}\noindent
\includegraphics[clip, trim=163pt 146pt 146pt 248pt, height=162mm]{ocr-input/image-1618.png}\end{center}

\newpage

\noindent 動之理」、「有靜之理」、「函衆理」、「具萬理」,有字含字具字亦皆方便說,實則只是太極對動言即為動之理,對靜言即為靜之理,而無所謂有、函與具也。

其次,「太極之有動靜是天命之流行也」,關於「天命之流行」亦須予以簡別。依朱子,天命即是理,其自身無所謂「流行」,流行是假託氣之動靜而說。流行之實在氣之動靜,理之流行是仗託氣之實流行而虛說耳。何以能有此虛說?蓋因理不離氣也。氣之一動一靜,呈現為流行之實,而理亦寓焉而定然之,遂亦得虛說理之流行也。剋就理自身而言之,理實無所謂流不流、行不行也。此是就朱子意謂理(天命、太極)為只存有而不活動言。如就原初義「天命之體」是即存有即活動者而言,則所謂「流行」最初是就此體自身之「於穆不已」說。「於穆不已」是形容此體永遠不停止地起作用,即就此「不已地起作用」說「天命流行」,乃至說「天命流行之體」,言此天命不已地起作用即是流行,而此亦即是體也。此雖是就體說流行,然亦實是流而不流,無流相也,行而不行,無行相也。(然卻不是朱子所意謂的理之無所謂流不流行不行)。唯因其不已地起作用遂有氣之生化不息之實事呈現,就此生化不息之實事言,遂流有流相、行有行相之實流行,此是氣化之流行也。氣化之流行有流行相,而為其體的那於穆不已之天命流行之體實無流相,亦無行相也。雖無流行相,然卻亦不是朱子所意謂的只是理。在朱子,流行之實只在氣,而理之流行是虛說—仗託氣之實流行而說,理自身實無所謂流不流行不行也。在先秦古義,天命流行是實說,(剋就不已地起作用說),然卻是流而不流之流,行而不行之行,故無流相、無行相,只是一如如的不已地起作用

\newpage\thispagestyle{empty}\addtocounter{page}{-1}\vspace*{-12mm}\begin{center}\noindent
\includegraphics[clip, trim=177pt 128pt 133pt 260pt, height=162mm]{ocr-input/image-1622.png}\end{center}

\newpage\markright{第二部 \quad 分論一 \quad 第一章 \quad 周濂溪對於道體之體悟}

\noindent 也,故得為體。氣化流行自亦是實說,然此卻是有流相、有行相。(動是流相、行相,靜是不流相、不行相。一動一靜,合而觀之,是一總流行相,所謂生化不息也)。有流行相之氣化流行以無流行相之天命流行為其體,此體是即存有即活動之體也,亦是誠體、神體、妙萬物而為言之體也,故窮神即可知化。朱子所意謂者未能至此。朱子並不願說理(太極、天命)是體,氣之動靜是其用。他所認為恰當而願說者卻是:「太極者本然之妙也,動靜者所乘之機也。」此根本是「理氣不離不雜,理掛搭於氣,氣依傍於理」的說法。

復次,朱子亦以《通書〉之誠合釋太極,但其所意謂之誠亦非濂溪之意,至少不能盡濂溪之意。《通書·誠上第一》「誠者,聖人之本。」朱子注云:「誠者至實而無妄之謂,天所賦、物所受之正理也。人皆有之。聖人之所以聖者無他焉,以其獨能全此而已。此書與〈太極圖〉相表裡,誠即所謂太極也。」〈誠下第二〉「聖,誠而已矣。」朱子注云:「聖人之所以聖,不過全此實理而已,即所謂太極者也。」詳此,其所意謂之誠即是實理、正理,與「意謂太極只是理」同,故誠即太極,而同以實理定。「誠者至實而無妄之謂」,此不錯。但朱子卻只把這「至實無妄」移向客觀面,只代表或指目那作為客觀存有的實理、正理,故亦以「天所賦、物所受」言之。此非濂溪言誠體之意也。濂溪言誠體本乎《中庸》、《易傳》。誠固是理,但亦是心,亦是神,是「心神理是一」之體,是「即存有即活動」之體。「真實無妄」決不只是客觀地形容或指目理,亦主觀地形容或指目心,而心是本心、天心,而本心天心即是理。故〈誠下第二〉云:「誠,五常之本,百行之源

\newpage\thispagestyle{empty}\addtocounter{page}{-1}\vspace*{-12mm}\begin{center}\noindent
\includegraphics[clip, trim=158pt 162pt 152pt 231pt, height=162mm]{ocr-input/image-1626.png}\end{center}

\newpage

\noindent 也。靜無而動有,至正而明達也。」〈聖第四〉亦云:「寂然不動者,誠也。感而遂通者,神也。」凡此,若只以實理、正理言之,則皆成另一說統,非濂溪所說者之本義。如「靜無而動有,至正而明達」,朱子注云:「方靜而陰,誠固未嘗無也,以其未形而謂之無耳。及動而陽,誠非至此而後有也,以其可見而謂之有耳。靜無,則至正而已。動有,然後明與達可見也。」案此以「靜而陰」解「靜無」,以「動而陽」解「動有」,非是。濂溪說此語是剋就誠體自身而形容之,「靜無」是說誠體之寂然不動、無聲無臭,亦即〈誠幾德第三〉「誠無為」一語之意。無思無為、無聲無臭即是靜無,亦即是寂然,此說不到「靜而陰」也。朱子說「靜而陰」實仍是「陰之靜」乃「太極之體所以立」之義,即由陰靜之收斂凝聚而可以見太極之自體也。故云:「方靜而陰,誠固未嘗無也,以其未形而謂之無耳」,言未假氣之發用而形見,故謂之無耳。此豈濂溪說「靜無」之意乎?濂溪說「靜無」是說誠體自身之寂然不動,不是說由氣之「靜而陰」以見實理正理之自體也。只因朱子以實理不可以動靜言,故移向氣以言之耳。同理,濂溪說「動有」亦是剋就誠體自身而說其「感而遂通」耳。此「動」不是氣動,「有」亦不是氣有,亦說不到動而陽也。故〈聖第四〉終之云:「寂然不動者誠也,感而遂通者神也。」此雖言辭上誠神分說,而其實義則冥如為一也。就「寂然不動」說誠,是言其體也。就「感而遂通」說神,是言其用也。而用是神用,不是氣用。故在此,神用誠體是一。在此,「寂然不動」是「靜而無靜」之靜,「感而遂通」是「動而無動」之動,故即寂即感,寂感一如也,即體即用、體用一如也,即有即無、有無一如也。無是神體之無,有是神用之有,神

\newpage\thispagestyle{empty}\addtocounter{page}{-1}\vspace*{-12mm}\begin{center}\noindent
\includegraphics[clip, trim=177pt 149pt 135pt 240pt, height=162mm]{ocr-input/image-1630.png}\end{center}

\newpage\markright{第二部 \quad 分論一 \quad 第一章 \quad 周濂溪對於道體之體悟}

\noindent 體與神用是一也。在此,寂感、誠神、體用、有無,不能分作兩截說,皆是說的誠體、神體,乃至天命流行之體(非氣化流行)之自己也。此尚不是說的帶著事的體用不二、體用圓融。但感而遂通之神用不能不帶著事。因帶著事之跡而顯動相或靜相,此亦是動有。此動有是太極誠體動而生陽,靜而生陰之動有,此固已是氣,但亦不是朱子所意謂之只是「動而陽」之「動有」。朱子把「動有」說成「動而陽」,亦仍是「陽之動」乃「太極之用所以行」之義。意即由氣之動用假說理之流行也。故云:「及動而陽,誠非至此而後有也,以其可見而謂之有耳」,言假氣之動用之可見而可見故遂謂之有耳。實則理本自有,不因「陽動」「可見」之謂有而有,自亦不因「陰靜」「未形」之謂無而無也。如此言靜無動有決非濂溪意。此仍是理氣不離不雜、「動靜者所乘之機」下之說法,非剋就誠體神體自身之寂感一如、有無一如、體用一如而說的靜無動有,亦非剋就帶著事的太極誠體顯動相而為陽,顯靜相而為陰,而說的靜無動有也。朱子所說只是關聯著氣之動靜而說理之形未形,見未見耳。

至于其解「寂然不動者誠也,感而遂通者神也」,則云:「本然而未發者,實理之體;善應而不測者,實理之用。」所謂「本然而未發」即太極理體「本然之妙」因氣之「靜而陰」而未形耳,並非太極理體本身有發與未發也。氣有已發未發,喜怒哀樂之情有已發未發,而太極理體無所謂已發未發也。如此,以「本然而未發者,實理之體」解「寂然不動者誠也」決非濂溪之意。至於「善應而不測者,實理之用」,實理如果只是理,則亦無所謂應。應者只是由氣之動而陽,乃至一動一靜互為其根,或一陰一陽生化不息

\newpage\thispagestyle{empty}\addtocounter{page}{-1}\vspace*{-12mm}\begin{center}\noindent
\includegraphics[clip, trim=171pt 160pt 161pt 251pt, height=162mm]{ocr-input/image-1634.png}\end{center}

\newpage

\noindent (此即陰陽不測)以見「太極之用之所以行」;而「太極之用」本只是「動而陽」統馭於太極,故為太極所領有,「實理之用」亦是如此,並非太極實理自身真會起此用也。如是,以這樣的「善應而不測者實理之用」解「感而遂通者神也」亦決非濂溪之意,甚至亦非《易傳〉之意。朱子此類辭語,表面觀之,皆可無病,亦極不易辨。然若能進窺其思理之背景,就其心中意解之實而觀之,則其所以如此措辭自有其實義,皆可剔剝得出。至其表面模稜不諦之辭,詞語如此而意指在彼者,則是因依附經典或依附其所註之原文而說之故,或是因措辭成習、習而不察、順口滑過之故。如體用、未發已發、太極之用、實理之用、太極之有動靜、天命之流行等等皆是也。

其如此意解誠體,則雖以《通書〉之誠合釋(太極圖說〉之太極,亦只是將太極、誠體解為實理、正理,此仍只是理,而無當於濂溪所默契之道妙也。朱子自亦可就心言誠,就心言寂感,然在朱子,心是心,理是理,心理平行而不是一,故其注《通書〉即只以實理解誠也。以實理解誠以附合於太極,則「其〔太極】動也,誠之通也,繼之者善,萬物之所資以始也」,此中所謂「誠之通」亦不是(易傳》所說濂溪所體會的誠體神體「感而逐通」之通。乃是太極實理通過氣之「動而陽」而假託之以見其流行耳。至於太極之動(其動也)亦不是太極真會動,乃只是太極為氣之動之所以然之理,因而主宰統馭氣之動,故氣之動得以歸屬之耳。是以太極之動靜、誠之通等,皆須撐開講,方合乎朱子心中所意謂之實義。

最後,「自其著者而觀之,則動靜不同時,陰陽不同位,而太極無不在焉。自其微者而觀之,則沖穆無朕,而動靜陰陽之理已悉

\newpage\thispagestyle{empty}\addtocounter{page}{-1}\vspace*{-12mm}\begin{center}\noindent
\includegraphics[clip, trim=179pt 131pt 129pt 255pt, height=162mm]{ocr-input/image-1638.png}\end{center}

\newpage\markright{第二部 \quad 分論一 \quad 第一章 \quad 周濂溪對於道體之體悟}

\noindent 具於其中矣。」前者即理氣不離,後者即太極具眾理。前者尚不即是「顯微無間」,而正因不離不雜而有間也。後者亦不即是「體用一原」,蓋朱子認太極為體,動靜為用為「有病」也。故終於是「太極者,本然之妙也;動靜者,所乘之機也。」此朱子說統之實義也。

以上是對於朱子〈太極圖解〉中主要語及(太極圖說〉首段註文之簡別。此兩段文是朱子體悟道體之重要文獻,基本而有關鍵性的觀念俱在其中,然而其說法不合濂溪原義甚顯。大抵朱子是理氣不離不雜之撐開的說法,其基本原則是伊川「陰陽氣也,所以陰陽道也」之語。其最初之洞見即是對於此語之真切而清澈的把握,其對於誠體、神體、天命流行之體並無洞悟;或者說其洞悟勁力於此用不上,而只能用於伊川之原則。此兩種洞悟最初之分別甚簡單,即太極真體、上天之載或天命流行之體是「只存有而不活動」與「即存有卽活動」兩者之分別。然最初甚簡,而後來之委蛇卻極複雜,遂牽涉到各方面皆不同,而表面之辭語又大都相彷彿,此其所以極難辨別也。然若真能握住那最初之分別,其系統自異,而其實義亦自不可揜。

以下錄朱子語以證明上述之簡別為不謬。

Ⅰ、〈答楊子直〉:

\begin{quotation}\kaishu 承喻太極之說,足見用力之勤,深所歎仰。然鄙意多所未
安。今且略論其一二大者,而其曲折,則託季通言之。

蓋天地之間只有動靜兩端循環不已,更無餘事,此之謂易。
而其動其靜則必有所以動靜之理焉,是則所謂太極者也。\end{quotation}

\newpage\thispagestyle{empty}\addtocounter{page}{-1}\vspace*{-12mm}\begin{center}\noindent
\includegraphics[clip, trim=187pt 158pt 138pt 248pt, height=162mm]{ocr-input/image-1642.png}\end{center}

\newpage

\begin{quotation}\kaishu 〔中略】

熹向以太極為體,動靜為用,其言固有病。後已改之曰:
「太極者本然之妙也,動靜者,所乘之機也」。此則庶幾近
之。來諭疑於體用之云,甚當。但所以疑之之說,則與熹所
以改之之意又若不相似然。蓋謂太極函動靜則可(原註:以
本體而言也),謂太極有動靜則可(原註:以流行而言
也)。若謂太極便是動靜,則是形而上下者不可分,而「易
有太極」之言亦贅矣。其他則季通論之已極精詳。且當就此
虚心求之,久當自明。不可別生疑慮,徒自繳繞也。〔下
略〕(《朱文公文集〉卷第四十五,書,問答。〈答楊子直〉五書
之第一書)\end{quotation}

Ⅱ、《朱子語類》卷第九十四:

\begin{quotation}\kaishu 1.李問:無極之真與未發之中同否?

曰:無極之真是包動靜而言,未發之中只以靜言。〔下
略】

2.太極無方所,無形體,無地位可頓放。若以未發時言之,
未發卻只是靜。動靜陰陽皆只是形而下者。然動亦太極之
動,靜亦太極之靜,但動靜非太極耳。故周子只以無極言
之(原注:無形而有理)。未發固不可謂之太極,然中含
喜怒哀樂。喜樂屬陽,怒哀屬陰。四者初未著,而其理已
具。若對已發言之,容或可謂之太極。然終是難說。此皆
只說得個髮鬚形容。當自體認。\end{quotation}

\newpage\thispagestyle{empty}\addtocounter{page}{-1}\vspace*{-12mm}\begin{center}\noindent
\includegraphics[clip, trim=144pt 124pt 151pt 251pt, height=162mm]{ocr-input/image-1646.png}\end{center}

\newpage\markright{第二部 \quad 分論一 \quad 第一章 \quad 周濂溪對於道體之體悟}

\noindent 案:未發之中,若指性一面言,中體即是太極(無極之真),所差者只是太極與性兩名之異耳。依(中庸〉,「未發」是喜怒哀樂未發,不是中體(性體)未發也。性體無所謂發不發,亦不是性體含喜怒哀樂,乃是含喜怒哀樂之理耳。以此方式應用於太極,太極亦無所謂發不發。發不發是指氣之動靜而言。由氣之靜而不發見「太極之體所以立」,由氣之動而已發見「太極之用所以行」。反過來,不是太極「包動靜」,乃是包動靜之理。「未發之中只以靜言」,中若指心一面說,此則其可。若指性一面說,此語非是。關於未發已發之複雜理論詳見〈朱子部〉第二章朱子參究中和問題之發展。

\begin{quotation}\kaishu 3.〔上問答,略】

問:「動而生陽,靜而生陰」,注:「太極者本然之妙,
動靜者所乘之機」。太極只是理,理不可以動靜言。惟
「動而生陽、靜而生陰」,理寓於氣,不能無動靜。「所
乘之機」,乘如乘載之乘。其動靜者,乃乘載在氣上,不
覺動了靜,靜了又動。

曰:然。\end{quotation}

\noindent 案:此問者之解,朱子然之,可見其意。又案此解恐是首發之蔡季通。《語類》卷第五有一條記云:「直卿[……】又去,先生〈太極圖解〉云:動靜者所乘之機也。蔡季通聰明,看得這般處出。謂先生下此語最精。蓋太極是理,形而上者。陰陽是氣,形而下者。然理無形,而氣卻有跡。氣有動靜,則所載之理亦安得謂之無動

\newpage\thispagestyle{empty}\addtocounter{page}{-1}\vspace*{-12mm}\begin{center}\noindent
\includegraphics[clip, trim=176pt 161pt 120pt 224pt, height=162mm]{ocr-input/image-1650.png}\end{center}

\newpage

\noindent 靜?」此黃直卿述蔡季通之解說也。大抵蔡季通於這般處甚能得朱子意,故(答楊子直〉書中屢提及之也。蔡氏說與此處問者之解意同。

\begin{quotation}\kaishu 4.某常說太極是個藏頭底。動時屬陽,未動時又屬陰了。

5.太極只是涵動靜之理,卻不可以動靜分體用。蓋靜即太極
之體也,動即太極之用地。譬如扇子只是一個扇子,動搖
便是用,放下便是體。才放下時,便只是這一個道理。及
搖動時,亦只是這一個道理。\end{quotation}

\noindent 案:云「不可以動靜分體用」,則「靜即太極之體,動即太極之用」兩語為不妥。其意乃是說:不是以靜為體,以動為用,而是由氣之靜而陰見太極之自體,由氣之動而陽見太極之流行。流行是假託氣之動而說。此氣之動為「太極之用」亦是因統馭於太極而歸屬之,遂說為「太極之用」。此亦猶「乘載在氣上」,因氣之動而顯動相,因氣之靜而顯靜相,而理自身實無所謂動靜也。

\begin{quotation}\kaishu 6.梁文叔云:太極兼動静而言。

曰:不是兼動靜,太極有動靜。喜怒哀樂未發也有個太
極,喜怒哀樂已發也有個太極。只是一個太極,流行於已
發之際,飲藏於未發之時。\end{quotation}

\noindent 案:「兼動靜」即是「包動靜」、「函動靜」,此是(答楊子直)書中註語所說「以本體而言也」。「有動靜」是該書中註語所說

\newpage\thispagestyle{empty}\addtocounter{page}{-1}\vspace*{-12mm}\begin{center}\noindent
\includegraphics[clip, trim=159pt 146pt 159pt 248pt, height=162mm]{ocr-input/image-1654.png}\end{center}

\newpage\markright{第二部 \quad 分論一 \quad 第一章 \quad 周濂溪對於道體之體悟}

\noindent 「以流行而言也」。實則不是兼、包、函動靜,亦不是有動靜,乃是兼、包、函動靜之理,以理馭氣之動靜,故亦以理收動靜之事,此是靜態地自本體上而言也。至於「有動靜」,則亦是有動靜之理,氣依其所具之動之理而動,而太極(理)亦隨之而顯動理相,氣依其所具之靜之理而靜,而太極亦隨之而顯靜理相,因此遂說太極有動靜,實只是有動靜之理也,此是動態地自假託說的流行上而言也。

\begin{quotation}\kaishu 7.問:「太極動而生陽」,是有這動之理便能動而生陽否?
曰:有這動之理,便能動而生陽,有這靜之理,便能靜而
生陰。既動,則理又在動之中,既靜,則理又在靜之中。
曰:動靜是氣也。有此理為氣之主,氣便能如此否?

曰:是也。既有理,便有氣。既有氣,則理又在乎氣之
中。〔下略】

8.太極者如屋之有極、天之有極,到這裡更沒去處,理之極
至者也。陽動陰靜,非太極動靜,只是理有動靜。理不可
見,因陰陽而後知,理搭在陰陽上,如人跨馬相似。〔下
略】。

9.問:「動靜者所乘之機」。

曰:理搭於氣而行。

10.問:「動靜者所乘之機」。
曰:太極理也,動靜氣也。氣行則理亦行。二者常相依
而未嘗相離也。太極猶人,動靜猶馬。馬所以載人,人
所以乘馬。馬之一出一入,人亦與之一出一入。蓋一動\end{quotation}

\newpage\thispagestyle{empty}\addtocounter{page}{-1}\vspace*{-12mm}\begin{center}\noindent
\includegraphics[clip, trim=139pt 149pt 155pt 237pt, height=162mm]{ocr-input/image-1658.png}\end{center}

\newpage
一靜,而太極之妙未嘗不在焉。此所謂「所乘之機」,
無極二五所以妙合而凝也。

\begin{quotation}\kaishu 11.周貴卿問:「動靜者所乘之機」。
曰:機是關捩子。踏著動底機,便挑撥得那靜底;蹅著
靜底機,便挑撥得那動底;

12.「動靜者所乘之機」,機言氣機也。\end{quotation}

Ⅲ、〈答鄭子上〉:

\begin{quotation}\kaishu 〔來問】:〈太極圖〉曰:「無極而太極」。可學〔鄭子上
之名】竊謂無者,蓋無氣而有理。然理無形,故卓然而常
存;氣有象,故闔闢歛散而不一。〈圖〉又曰:「太極動而
生陽。動極而靜,靜而生陰」。太極理也,理如何動靜?有
形,則有動靜。太極無形,恐不可以動靜言。南軒云:「太
極不能無動靜」。未達其意。

〔答曰〕:理有動靜,故氣有動靜。若理無動靜,則氣何自
而有動靜乎?且以目前論之,仁便是動,義便是靜,又何關
於氣乎?他說已多得之,但此處更須子細耳。((朱文公文
集〉卷第五十六,書,問答,(答鄭子上〉十七書之第十四書)\end{quotation}

Ⅳ、吳澄曰:

\begin{quotation}\kaishu 太極無動靜。動靜者氣機也。氣機一動,則太極亦動。氣機
一靜,則太極亦靜·故朱子釋(太極圖〉曰:「太極之有動\end{quotation}

\newpage\thispagestyle{empty}\addtocounter{page}{-1}\vspace*{-12mm}\begin{center}\noindent
\includegraphics[clip, trim=183pt 131pt 107pt 243pt, height=162mm]{ocr-input/image-1662.png}\end{center}

\newpage\markright{第二部 \quad 分論一 \quad 第一章 \quad 周濂溪對於道體之體悟}

\begin{quotation}\kaishu 靜是天命之流行也」。此是為周子分解。太極不當言動靜。
以天命之有流行,故只得以動靜言也。又曰:「太極者,本
然之妙也。動靜者,所乘之機也」。機猶弩牙,弩弦乘此
機,機動則弦發,機靜則弦不發。氣動,則太極亦動;氣
靜,則太極亦靜。太極之乘此氣,猶弩弦之乘機也。故曰
「動靜者,所乘之機」。謂其所乘之氣機有動靜,而太極本
然之妙無動靜也。然弦與機卻是兩物,太極與此氣非有兩
物,只是主宰此氣者,非別有一物在氣中而主宰之也。機字
是借物為喻,不可以辭害意。(董榕輯(周子全書〉卷一)\end{quotation}

\noindent 案:吳澄此解大體不錯。惟言「太極之有動靜是天命之流行也」「是為周子分解」,此在語勢上似不甚恰當。此好像是說周子亦主「太極無動靜」,「太極不當言動靜」,而其所以言「有動靜」者,是「以天命之有流行」之故也。實則濂溪並不主「太極無動靜」,他直說「太極動而生陽」。朱子固是註解周子,但其註解自始即不相應,「陽動陰靜,非太極動靜,只是理有動靜」,此自是朱子義。故朱子註語「太極之有動靜是天命之流行也」,不是「為周子分解」,乃是因注解周子故,須引附周子而說耳。其所說者只是己義也。故從太極本無動靜而至有動靜,此中間有一曲折之跌宕,而濂溪無此跌宕也。正因有此跌宕,故「太極之有動靜」亦成另一特殊之意義,而「天命流行」亦成另一特殊之意義。如果濂溪之太極不是形而下之氣,而即是誠體、神體、天命流行之體,則其直說「太極動而生陽」之背景必完全不同於朱子。太極雖不是形而下之氣,然亦不必只是理。雖非只是理,然亦不必即是氣。如是,

\newpage\thispagestyle{empty}\addtocounter{page}{-1}\vspace*{-12mm}\begin{center}\noindent
\includegraphics[clip, trim=138pt 143pt 158pt 244pt, height=162mm]{ocr-input/image-1666.png}\end{center}

\newpage

\noindent 其對於太極之體會乃是即存有即活動者,是「心神理為一」者。(此中「心」義雖只是本體宇宙論的意義,尚未至如孟子之言心,然亦是心義,故此處只以「活動」一詞概括之。活動是activity義,不是運動motion義。)如是,其直說「太極動而生陽」,乃是可許者。濂溪明說「動而無動,靜而無靜,非不動不靜也」。此即示誠體、神體、太極真體、天命流行之體是可以動靜言,只是其動是「動而無動」之動,其靜是「靜而無靜」之靜。如其「動而無動」之動,順事應物而顯動相,即是「動而生陽」;如其「靜而無靜」之靜,順事應物而顯靜相,即是「靜而生陰」。詳解見前。此只是於穆不已的本體宇宙論的實體、道德創生的實體、太極真體、誠體、神體之不已地起作用,此是此真體之立體地、創生地、妙運地直貫,不是如朱子理氣撐開而有那些曲折、間接、跌宕、關聯的說法。是以在朱子,掛搭、附著、依傍、跨馬、所乘之機等喩解皆有本質的意義,而在濂溪義所函之說統中,則並無亦不必要此等喻解。此顯是兩系統之差異,吳澄不能知也。然其理解朱子義卻並不錯。

又,吳澄最後提及「弦與機卻是兩物,太極與此氣非有兩物」云云,此層最無實義。夫太極之為物自非有形跡之具體物,然謂太極與氣不是兩個不同的概念亦不得也。此層是理氣二不二的問題。朱子後,不滿朱子者,最喜從此著眼。如明之羅整菴、劉蕺山、黃梨洲等皆喜就此譏議朱子,而不知此非問題之所在也。關此,見下段。

V、明曹端〈辨戾〉:

\newpage\thispagestyle{empty}\addtocounter{page}{-1}\vspace*{-12mm}\begin{center}\noindent
\includegraphics[clip, trim=180pt 179pt 111pt 254pt, height=162mm]{ocr-input/image-1670.png}\end{center}

\newpage\markright{第二部 \quad 分論一 \quad 第一章 \quad 周濂溪對於道體之體悟}

\begin{quotation}\kaishu 先賢之解(太極圖說〉固將以發明周子之微奧,用釋後生之
疑惑矣。然而有人各一說者焉,有一人之說而自相齟語者
焉。且周子謂「太極動而生陽」,「靜而生陰」,則陰陽之
生由乎太極之動靜,而朱子之解極明備矣。其曰:「有太
極,則一動一靜而兩儀分;有陰陽,則一變一合而五行具」
〔案此係朱子註「陽變陰合」段之語〕,尤不異焉。及觀
《語錄》,卻謂太極不自會動靜,乘陰陽之動靜而動靜耳。
遂謂理之乘氣猶人之乘馬。「馬之一出一入,而人亦與之一
出一入。」以喻氣之一動一靜,而理亦與之一動一靜。若
然,則人為死人,而不足以為萬物之靈,理為死理,而不足
以為萬物之原。理何足尚,而人何足貴哉?今使活人乘馬,
則其出入行止疾速,一由乎人馭之何如耳。活理亦然。不之
察者,信此則疑彼矣,信彼則疑此矣。經年累歲,無所折
衷。故爲(辨戾〉以告夫同志君子云。(董榕輯《周子全書)
卷五)\end{quotation}

\noindent 案:曹端(號月川)見出濂溪之意實是「太極動而生陽」,「靜而生陰」,「陰陽之生由乎太極之動靜」,此是也,但以為朱子之注語亦是「明備」而「不異」乎此,則非是。彼不解朱子註語之背景。彼以為註語與《語錄》相矛盾(相戾),此則為註語表面辭語所惑,而不知朱子思理實一貫也。《語錄》自不會全誤,朱子與黃直卿(勉齋)之稱述蔡季通亦不會假。《語錄》之討論即是〈圖說〉注語背景之表白,故註語之言「太極有動靜」須另眼相看,其表面辭語雖「不異」,而其意指實有異也。此則為曹端所看不出

\newpage\thispagestyle{empty}\addtocounter{page}{-1}\vspace*{-12mm}\begin{center}\noindent
\includegraphics[clip, trim=179pt 152pt 134pt 247pt, height=162mm]{ocr-input/image-1674.png}\end{center}

\newpage

\noindent 矣。至於彼以為濂溪所言之太極是「活理」,是也,但以爲朱子注語所說之太極亦是活理,至《語錄》才成「死理」,則非是。朱子注語與《語錄〉一貫,則朱子實認「太極不自會動靜,乘陰陽之動靜而為動靜耳」。理固無所謂死活,但朱子所意謂之理是只存有而不活動者則無疑。彼知「死理」為非是,但不知朱子之意實如此也。彼以為理應當是「活理」,此不錯,但不知理如何能成為活理,亦不知濂溪所言之太極何以是活理也。只看「太極動而生陽」一語便認為是「活理」,宜其看不出朱子註語之有殊指也。此而看不出,則其對於理之死活之關鍵未有所知亦明矣。此後面關涉到一最根本之問題,即對於道體本身之體會是也。體會成只存有而不活動(只是理)便是死理,體會成即存有即活動(心神理是一)便是活理。

以上辨朱子體會「太極真體」之偏差,顯出死理活理之兩系統。一般人對活理系統無真體會,又無朱子分解思考之縝密與貫徹以及其身體力行之體驗,兩不著邊,只抓住理氣之二不二、朝三暮四胡纏夾,其不及朱子也亦遠矣。朱子亦未可輕議也。

《明儒學案》卷四十四〈諸儒學案上二〉論曹月川處,黃梨洲亦引及曹端〈辨戾〉之文,而曰:「先生之辨雖爲明晰,然詳以理馭氣,仍為二之。氣必待馭於理,則氣為死物。」此卽纏夾二不二之問題也。「氣必待馭於理,則氣為死物」,此則愈說愈不成話矣。下段只就(太極圖說〉以明彼不滿意於朱子之理氣為二之解說者,且明彼等所謂一(不二)究是何意也。

\newpage\thispagestyle{empty}\addtocounter{page}{-1}\vspace*{-12mm}\begin{center}\noindent
\includegraphics[clip, trim=151pt 236pt 139pt 248pt, height=162mm]{ocr-input/image-1678.png}\end{center}

\newpage\markright{第二部 \quad 分論一 \quad 第一章 \quad 周濂溪對於道體之體悟}

\subsection{對於評斥朱子理氣為二者之衡定}

朱子表揚〈太極圖說〉最力。字字衡量,句句為解,而〈語類》中文又反覆討論,其用力可謂深矣。人皆本之,號稱正宗。吾今加以簡別,覺其雖有偏差,然亦不能越過。欲明天命流行之體之真義(所謂活理系統),亦必須先明朱子系統之何所是以及其何由成。乃黃梨洲等編《宋元學案·濂溪學案〉對於朱子之註解絕不錄及,只以劉蕺山之解說為領導,兼及其他,抹過朱子,以爭學統,儼若朱子所說全非,蕺山梨洲等所說全是,此亦未見其可也。

〈濂溪學案下〉於錄朱、陸往復爭辨後,梨洲作案語云:

\begin{quotation}\kaishu 朱、陸往復幾近萬言,亦可謂無餘蘊矣。然所爭只在字義先
後之間,究竟無以大相異也。惟是朱子謂「無極即是無形,
太極即是有理」。「在無物之前,而未嘗不立於有物之後;
在陰陽之外,而未嘗不行於陰陽之中。」此朱子自以理先氣
後之說解周子,亦未得周子之意也。

羅整菴〈困知記〉謂:「無極之真,二五之精,妙合而凝三
語,不能無疑。凡物必兩,而後可以言合。太極與陰陽果二
物乎?其為物也果二,則方其未合之先,各安在耶?朱子終
身認理氣為二物,其原蓋出於此。」

不知此三語正明理氣不可相離,故加「妙合」以形容之,猶
中庸〉言體物而不可遺也。非「二五之精」,則亦無所謂
「無極之真」矣。朱子言「無形有理」,即是尋「無極之
真」於「二五之精」之外。雖曰「無形」,而實為「有\end{quotation}

\newpage\thispagestyle{empty}\addtocounter{page}{-1}\vspace*{-12mm}\begin{center}\noindent
\includegraphics[clip, trim=169pt 155pt 125pt 227pt, height=162mm]{ocr-input/image-1682.png}\end{center}

\newpage

\begin{quotation}\kaishu 物」,亦豈「無極」之意乎?故以為歧理氣出自周子者,非
也。\end{quotation}

\noindent 案:此反對「理先氣後」,又反對「歧理氣為二」,蓋亦不知先後之實義,復亦不知二不二之實義也。羅整菴並朱子、濂溪皆反對之,固無是處,即梨洲謂「歧理氣」不「出自周子」,亦未能知此中之蘊也。朱子分理氣固是本於伊川,自此而言不「出自周子」亦可,然濂溪豈即混誠體、神體、太極真體為氣而不分者乎?「動而無靜、靜而無動,物也。動而無動、靜而無靜,神也」。此非分別而何?「無極之真,二五之精,妙合而凝」,「妙合」二字固是「明理氣不可相離」,然「不可相離」豈即有礙於分理氣為二乎?朱子亦說「不離」也。《中庸》言「鬼神體物而不可遺」,豈即神、物為一而不可分乎?於以知此種爭辨實無實義,只「朝三暮四,朝四暮三」之類耳。然畢竟亦可說一,亦可說二。彼等究亦不知此中之一實義為何,二之實義為何,只纏夾渾淪,氣機鼓蕩,以為妙耳。試觀蕺山與梨洲所說之一究如何。

劉蕺山解(太極圖說〉云:

\begin{quotation}\kaishu 一陰一陽之謂道,即太極也。天地之間一氣而已。非有理而
後有氣,乃氣立而理因之寓也。就形下之中而指其形而上
者,不得不推高一層,以立至尊之位,故謂之太極,而實無
太極之可言,所謂「無極而太極」也。使實有是太極之理,
為此氣從出之母,則亦一物而已,又何以生生不息,妙萬物
而無窮乎?今曰理本無形,故謂之「無極」,無乃轉落注\end{quotation}

\newpage\thispagestyle{empty}\addtocounter{page}{-1}\vspace*{-12mm}\begin{center}\noindent
\includegraphics[clip, trim=157pt 140pt 144pt 241pt, height=162mm]{ocr-input/image-1686.png}\end{center}

\newpage\markright{第二部 \quad 分論一 \quad 第一章 \quad 周濂溪對於道體之體悟}

\begin{quotation}\kaishu 腳!太極之妙,生生不息而已矣。生陰生陽,而生水火木金
土,而生萬物,皆一氣自然之變化,而合之只是一個生意,
此造化之蘊也。

惟人得之以為人,則太極為靈秀之鍾,而一陰一陽分見於形
神之際。由是殽之為五性,而感應之途出,善惡之介分,人
事之所以萬有不齊也。

惟聖人深悟無極之理,而得其所謂靜者主之,乃在中正仁義
之間,循理為靜是也。天地此太極,聖人此太極,彼此不相
假,而若合符節,故曰合德。若必捐天地之所有而界之於
物,又獨鍾界之於人,則天地豈若是之勞也哉?

自無極說到萬物上,天地之始終也。自萬物反到無極上,聖
人之終而始也。始終之說,即生死之說。而開闢混沌,七尺
之去留,不與焉。知乎此者,可與語道矣。主靜要矣。致知
極焉。(《劉子全書》卷五,(聖學宗要〉)\end{quotation}

\noindent 案:此解隱括(圖說〉全文在內,而重點在首段。然其措辭則多突兀。此因有許多誤解而然,亦因有所避忌而然。亟欲將太極內在化之,納之於氣中一滾說,故不免矯枉過正,遂有此突兀不平之辭。「天地之間一氣而已,非有理而後有氣,乃氣立而理因之寓也。」「而實無太極之可言,所謂無極而太極也。」「使實有是太極之理為此氣從出之母,則亦一物而已。」此皆過正不穩之辭,實只加重一滾說而已。吾細讀《劉子全書》,覺其滯辭、不穩之辭太多,然其實意亦可窺。即就此文而觀之,窺其意實非不認有太極。「就形下之中而指其形而上者」云云,可見非不認有「形而上者」。然則

\newpage\thispagestyle{empty}\addtocounter{page}{-1}\vspace*{-12mm}\begin{center}\noindent
\includegraphics[clip, trim=176pt 140pt 131pt 251pt, height=162mm]{ocr-input/image-1690.png}\end{center}

\newpage

\noindent 「而實無太極之可言」自是滯辭、不穩不平之辭,亦是過正之辭。由「實無太極之可言」而說「所謂無極而太極也」,此豈周子之意乎?「無極」豈是「無太極之可言」之意耶?「無極而太極」豈是以沒有太極為太極耶?此豈非過甚矣乎?只是閉眼瞎說而已。「使實有是太極之理為此氣從出之母,則亦一物而已」,此是誤解「理生氣」之說。無論朱子之體會太極為「只是理」,或是濂溪之體會為心神理是一,皆不是說氣從理生出來,一如母之生子。即老子「天下萬物生於有,有生於無」,亦不是說萬物從「無」生出來,一如母之生子。凡此皆當善會生字之意,無人作如此滯礙之解也。若如此滯礙,則凡內聖之學言「本」者皆成不可能之辭矣。而《中庸〉言「生物不測」,亦豈是萬物從道生出來,一如母之生子耶?此是以誤解栽贓也。以此誤解而反對理先氣後亦只乖謬而已矣。「理先」者,且不必作深解,只以平常之意說之,只是以本為先耳。若必反對理本之先在性,則凡言性善、言本心、言良知以及汝劉蕺山之言意根獨體,皆成何事?此只是心不明澈,一旦纏夾下去,乃並其所講之學之本義而亦忘之耳。然而蕺山之實意,吾亦知之。且撥開這些煙霧而直窺其實意亦可矣。

蕺山生於明末,為宋、明儒學之殿軍。其承接以往之遺產厚,見六百多年來諸多分別解說,概念繁多,不勝其支離,故就其中重要論題悉欲統而一之。其子劉汋所編之《蕺山年譜》於六十六歲下,記云:

\begin{quotation}\kaishu 先生平日所見,一一與先儒悟。晚年信筆直書,姑存疑
案。仍不越誠意、已未發、氣質義理無極太極之說。於是\end{quotation}

\newpage\thispagestyle{empty}\addtocounter{page}{-1}\vspace*{-12mm}\begin{center}\noindent
\includegraphics[clip, trim=150pt 117pt 146pt 264pt, height=162mm]{ocr-input/image-1694.png}\end{center}

\newpage\markright{第二部 \quad 分論一 \quad 第一章 \quad 周濂溪對於道體之體悟}

\begin{quotation}\kaishu 斷言之日:「從來學問只有一個工夫。凡分內分外分動分
靜、說有說無劈成兩下,總屬支離。」又曰:「夫道一而
已矣。知行分言,自子思子始。誠明分言,亦自子思子始。
已未發分言,亦自子思子始。仁義分言,自孟子始。心性分
言,亦自孟子始。動靜、有無分言,自周子始。氣質義理分
言,自程子始。存心致知分言,自朱子始。聞見德性分言,
自陽明子始。頓漸分言,亦自陽明子始。凡此,皆吾夫子所
不道也。嗚乎!吾捨仲尼奚適乎?」〔案:此段文字不見今
之《全書》中。其所說亦不必盡合史實與義理之實。此不必
管。】\end{quotation}

\noindent 此下劉汋又附注云:

\begin{quotation}\kaishu 按先儒言道分析者,至先生悉統而一之。先儒心與性對,先
生曰:「性者心之性。」性與情對,先生曰:「情者性之
情。」心統性情,先生日:「心之性情。」分人欲為人心,
天理為道心,先生曰:「心只有人心,道心者人心之所以為
心。」分性為氣質義理,先生日:「性只有氣質,義理者氣
質之所以為性。」未發為靜,已發為動,先生日:「存發只
是一機,動靜只是一理。」推之,存心致知,聞見德性之
知,莫不歸之於一。然約言之,則曰:心之所以為心也。又
就心中指出本體工夫合並處,日誠意。「意根最微,誠體本
天」。此處著不得絲毫人力,惟有謹凜一法,乃得還其本
位,所謂戒慎乎其所不睹,恐懼乎其所不聞,此慎獨之說\end{quotation}

\newpage\thispagestyle{empty}\addtocounter{page}{-1}\vspace*{-12mm}\begin{center}\noindent
\includegraphics[clip, trim=179pt 161pt 128pt 233pt, height=162mm]{ocr-input/image-1698.png}\end{center}

\newpage

\begin{quotation}\kaishu 也。〔下略】\end{quotation}

\noindent 蕺山欲統而一之,故不欲橫地撐開說,亦不欲縱地拉開說。其統一之法大體是直下將形而下者向裡向上緊收於形而上者,而同時形而上者亦即全部內在化而緊吸於形而下者中,因而成其為一滾地說。此大體是本體論地卽體即用之一滾地說,在此,說「顯微無間,體用一原」,誠是如此。蕺山對於即存有即活動、於穆不已之天命流行之體確有體認,亦真有工夫。此無論自意根誠體說,或自無極太極說,皆可見其是如此。此證之其晚年最成熟之作品《人譜》即可知。彼即欲將形而下者如情、如人心、如氣質、如喜怒哀樂等,直下緊收於此於穆不已之體,而此於穆不已之體亦即全部內在化而緊吸于此形而下者中以主宰而妙運之,以成其「全體是用,全用是體」之一滾而化,一滾地如如呈現。故不作心性對言,而只說「性者心之性」;不作性情對言,而只說「情者性之情」;不說心統性情,而只說「心之性情」;不說人心道心,只說「心只有人心,道心者人心之所以為心」;不分氣質之性與義理之性,只說「性只有氣質,義理者氣質之所以為性」;不說未發為靜,已發為動,只說「存發只是一機,動靜只是一理」。此最後一點尤顯其上下緊收緊吸之精神,言之極為精彩。試看其言曰:

\begin{quotation}\kaishu 故自喜怒哀樂之存諸中而言,謂之中,不必其未發之前別有
氣象也,即天道之元亨利貞運於於穆者是也。自喜怒哀樂之
發於外而言,謂之和,不必其已發之時又有氣象也,即天道
之元亨利貞呈於化育者是也。惟存發總是一機,故中和渾是\end{quotation}

\newpage\thispagestyle{empty}\addtocounter{page}{-1}\vspace*{-12mm}\begin{center}\noindent
\includegraphics[clip, trim=140pt 129pt 146pt 245pt, height=162mm]{ocr-input/image-1702.png}\end{center}

\newpage\markright{第二部 \quad 分論一 \quad 第一章 \quad 周濂溪對於道體之體悟}

\begin{quotation}\kaishu 一性。(《劉子全書〉卷十一,〈學言中〉)\end{quotation}

又曰:

\begin{quotation}\kaishu 未發以所存而言者也。蓋曰:自其所存者而言,一理渾然,
雖無喜怒哀樂之相,而未始淪於無,是以謂之中。自其所發
者而言,泛應曲當,雖有喜怒哀樂之情,而未始著於有,是
以謂之和。可見中外只是一機,中和只是一理。絕不以前後
際言也。((劉子全書〉卷九,(答董生心意十問〉)\end{quotation}

\noindent 在此種本體論地上下緊收緊吸即體即用的一滾而化中,固可有許多甚深甚妙之談。此如:

\begin{quotation}\kaishu 只此動靜之理,分言之是陰陽,合言之是太極。故曰:「一
陰一陽之謂道」。即分即合是太極,非分非合是無極。故
曰:「陰陽不測之謂神」。(《劉子全書)卷十,學言
上))\end{quotation}

\noindent 又如:

\begin{quotation}\kaishu 涵養之功只在日用動靜語默衣食之間。就一動一靜、一語一
默一衣一食理會,則謂之養心。就時動時靜、時語時默、
時衣時食理會,則日養氣。就即動郎靜、即語即默、即衣郎
食理會,則曰養性。(《劉子全書〉卷六,〈證學雜解〉第二十\end{quotation}

\newpage\thispagestyle{empty}\addtocounter{page}{-1}\vspace*{-12mm}\begin{center}\noindent
\includegraphics[clip, trim=162pt 168pt 143pt 224pt, height=162mm]{ocr-input/image-1706.png}\end{center}

\newpage

\begin{quotation}\kaishu 則)\end{quotation}

\noindent 在此種情形下,實亦可說「實無太極之可言」。因為「即分即合是太極,非分非合是無極」。同樣亦可說「即動即靜、即語即默即衣即食是太極,非動非靜、非語非默、非衣非食是無極」。「無極」者「實無太極之可言」也。莊生云:「已爲一矣,且得有言乎?」即此「實無太極之可言」之意也。此是形而上下緊收緊吸下的圓融化境,不能視作主張上的陳述;即使視作一種陳述,亦不能視作主張上的陳述之對遮。即使在發展中各陳述對遮相消相融以期最後之圓融而化,亦不能滯在此圓融而化中之「無太極之可言」而反對彼言有太極者。蓋圓融而化即預設著一種分解歷程之分別言。故上錄兩段可順適無病,而(太極圖說解〉之首段則滯礙難通。劉蕺山之滯礙不通處即在常不自覺地將圓融而化視作一特定之主張(陳述)而以此遮彼,將圓融而化中之「無言」特定化,視作與彼分別言之各種陳述為同一層次上相對立之陳述。此則反降低自己,乃是以不熟不圓之心智談圓義者。

《劉子全書》卷十一(學言中〉亦有相同之滯礙,試看以下各條:

\begin{quotation}\kaishu 盈天地間一氣而已矣。有氣斯有數,有數斯有象,有象斯有
名,有名斯有物,有物斯有性,有性斯有道。故道,其後起
也。而求道者輒求之未始有氣之先,以為道生氣,則道亦何
物也,而能遂生氣乎?\end{quotation}

\newpage\thispagestyle{empty}\addtocounter{page}{-1}\vspace*{-12mm}\begin{center}\noindent
\includegraphics[clip, trim=165pt 203pt 134pt 233pt, height=162mm]{ocr-input/image-1710.png}\end{center}

\newpage\markright{第二部 \quad 分論一 \quad 第一章 \quad 周濂溪對於道體之體悟}

\noindent 案:此全是誤解不通之滯辭。淺陋不入者將視此為唯物論矣。此只是蕺山之別扭,非其實意也。就此別扭而觀之,彼似亦不自知其形而上下緊收緊吸、顯微無間、體用一原,究是何義!只因一時之誤解而亟欲反之,故有此乖戾之言,且並其自己所精悟之於穆不已之體而亦忘之矣!夫於穆不已之體固不離氣,然亦豈只是「一氣而已」耶?從氣歷降而說到道,道為後起,亦豈有「顯微無間,體用一原,即體即用」之義耶?

\begin{quotation}\kaishu 宋儒之言曰:道不離陰陽,亦不倚陰陽。則必立於不離不倚
之中,而又超於不離不倚之外,所謂離四句,絕百非也。幾
何而不墜於佛氏之見乎?\end{quotation}

\noindent 案:「即分即合是太極,非分非合是無極」,豈非離四句,絕百非乎?

\begin{quotation}\kaishu 或曰:虛生氣。夫虛即氣也,何生之有?吾溯之未始有氣之
先,亦無往而非氣也。當其屈也,自無而之有,有而未始
有。及其伸也,自有而之無,無而未始無也。非有非無之間
而即有卽無,是謂太虛,又表而尊之曰太極。\end{quotation}

\noindent 案:此是本橫渠之義說。橫渠依「虛即氣」反對「虛生氣」,此亦是誤解。不知「虛即氣」與「虛生氣」兩義同可說也。「有而未始有」,「無而未始無」,「非有非無之間而即有即無」,此等語句如不只是撥弄字眼而有實義,則不能只是「一氣而已」甚顯。如不

\newpage\thispagestyle{empty}\addtocounter{page}{-1}\vspace*{-12mm}\begin{center}\noindent
\includegraphics[clip, trim=164pt 164pt 141pt 230pt, height=162mm]{ocr-input/image-1714.png}\end{center}

\newpage

\noindent 能正視一分別說之虛與氣之差別,則此等語句未必可能也。

\begin{quotation}\kaishu 天者萬物之總名,非與物為君也。道者萬器之總名,非與器
為體也。性者萬形之總名,非與形為偶也。\end{quotation}

\noindent 案:此若視作「即體即用」可。然「即體即用」非不承認有體也。此三句不能表示出「即體即用」義,但卻表示出是斷定直述語,此則便成極端乖戾之言。劉蕺山真能貫徹此義乎?若然,則意根誠體亦不必言矣!郭象注〈齊物論〉之天籟中有曰:「天者萬物之總名」,此在發明道家之自然義,如此說未嘗不可,而在儒家言性體、道體亦如此說,則成大悖。劉蕺山焉可玩此乖巧!

\begin{quotation}\kaishu 子曰:「形而上者謂之道,形而下者謂之器。」程子曰:
「上下二字截得道器最分明。」又日:「道即器,器即
道。」畢竟器在斯,道亦在斯。離器而道不可見。故道器可
以上下言,不可以先後言。「有物先天地」,異端千差萬
錯,總從此句來。\end{quotation}

\noindent 案:道器「可以上下言」,亦「可以先後言」,而且先後正由上下而引申出。《大學〉言「物有本未,事有終始,知所先後,則近道矣」。本末先後豈可忽乎哉?而必反之何耶?「維天之命,於穆不已」,天命豈不先自本有乎?亦後起耶?「所惡於智者為其鑿也」。劉蕺山之智亦可謂鑿而死,往而不返者矣!

\newpage\thispagestyle{empty}\addtocounter{page}{-1}\vspace*{-12mm}\begin{center}\noindent
\includegraphics[clip, trim=166pt 200pt 128pt 231pt, height=162mm]{ocr-input/image-1718.png}\end{center}

\newpage\markright{第二部 \quad 分論一 \quad 第一章 \quad 周濂溪對於道體之體悟}

\begin{quotation}\kaishu 理即是氣之理,斷然不在氣先,不在氣外。知此,則知道心
卽人心之本心,義理之性即氣質之本性。千古支離之說可以
盡掃,而學者從事於入道之路,高之不墜於虛無,卑之不淪
於象數,而道術始歸於一乎?\end{quotation}

\noindent 案:理如果是指即活動即存有、於穆不已的天命流行之體說,則「理即是氣之理」意即:理是妙運乎氣而使之所以能生化不息者,理即是使氣所以為此氣者,理當然是氣之理。即使朱子體會成只存有而不活動,亦仍可說理是氣之理。但說「理即是氣之理」,無論此理是即存有即活動或只存有而不活動,理總是超越的、普遍的、絕對之一的實體,而不會是氣之謂詞(性質),或是氣之關聯的特質。如是,分解地說,重視此理之為本義,則亦可說理在氣先,正視此理之為超越的實體義,則亦可說理在氣外。儘管當形而上下緊收緊吸即體即用而一滾地說,亦無所謂先後,亦無所謂內外,但此亦並不礙分解地說在先在外之成立。劉蕺山只說一句「理即是氣之理」,此豈足以否定在先在外之義乎?汝總不至於把理視作氣之謂詞或關聯的特質也。依此推下來,「道心即人心之本心」(「道心者人心之所以為心」),此只是辭語之變換,仍承認有本心也。「人心」是無色地說(「心只有人心」),「本心」是有色地說(有價值意義地說),「人心之本心」意即人心之不喪失其本心者,此即人心而道心矣。故「道心者人心之所以為心」意即人心之不止於無色而所以能成其有道德意義之本心者即為道心。汝總不至於認此「所以為心」為「此無色之人心之所以為此無色之人心者」,即「所以」總不會是此實然的現象的「所以」也。此能反對

\newpage\thispagestyle{empty}\addtocounter{page}{-1}\vspace*{-12mm}\begin{center}\noindent
\includegraphics[clip, trim=169pt 151pt 131pt 234pt, height=162mm]{ocr-input/image-1722.png}\end{center}

\newpage

\noindent 分解地說人心道心之分別乎?推之,「義理之性即氣質之本性」(「性只有氣質,義理者氣質之所以為性」),亦只是辭語之變換,落實了說,仍不能否認義理之性與氣質之性之分別。依朱子,義理之性即是純然是義理的性本身、本然之性、性體之自己,氣質之性則是此性之墜在氣質裡面,意即氣質裡面的性,雜在氣質裡面而為氣質所拘限所染汙的性,此只是一性之兩面觀,即依此兩面觀的分別逐有此兩詞之建立,實則性只有一性,非有兩種性也。依通常之解法,順告子生之謂性下來,通過漢儒之言氣性、才性,直至張橫渠首言氣質之性,伊川仍之而言才性,則是就人之個體生命、氣之凝結所呈現之種種顏色如清濁、剛柔、緩急、才不才之類而說一種性,此即所謂氣質之性,即就氣質顏色之殊而說一種性,非意謂義理之性之陷在氣質裡面也。至義理之性或天地之性則是就人之內在道德性之性或是就於穆不已的天命實體(本體宇宙論地說的道德創生的實體)說另一種性,此是人之超越的、先天的真性,即人之所以異於禽獸者。而依劉蕺山,「性只有氣質之性」,意即氣質底性,而氣質則取其最廣泛的意義,意即氣、質,並不取或至少不重視而忽視那有種種顏色之殊的氣質如普通所謂脾性、才性之類,(取此最廣泛的意義,朱子亦有此意,因他說性墜在氣質裡面亦總是雜在氣裡面,此廣狹無關),是則氣質底性即如說氣之理然,意即氣、質底性體主宰也。「義理之性即氣質之本性,即氣質之所以為性」,前一句是說「義理之性」就是氣質底性體主宰(兩「之」字並不一樣),後一句是說「義理之性」就是氣質底性之所以為性者,「所以為性者」意即其為性全是義理也,此與前一句說法意義同。劉蕺山所說的「氣質之本性」並不是說氣質本身所呈現的種種

\newpage\thispagestyle{empty}\addtocounter{page}{-1}\vspace*{-12mm}\begin{center}\noindent
\includegraphics[clip, trim=150pt 135pt 141pt 241pt, height=162mm]{ocr-input/image-1726.png}\end{center}

\newpage\markright{第二部 \quad 分論一 \quad 第一章 \quad 周濂溪對於道體之體悟}

\noindent 實然的特質或顏色也,他還是說的那「天命流行,物與無妄之本體,亦即此是無聲無臭渾然至善之別名」(《劉子全書〉卷十九,(答王右仲州刺〉)。他所說的「性只有氣質,義理者氣質之所以為性」具言之當該是「性只有氣質之性,義理之性者氣質之性之所以為性」,(氣質底性之所以為性者),並不是說氣質本身之所以能成為性者,亦即並非說氣質本身所呈現的種種顏色之所以為此種種顏色者。是則「性只有氣質之性」意即只有一個作為氣質底性體主宰的性,此還是體用義,亦仍是「理即是氣之理」之義。但如此說,並不能否定朱子一性兩面觀的說法,亦不能否定自不同層面說兩種性之義。蓋本有此許多義,並不能相代替也。而自道德實踐言,亦不能不重視氣質之限制,此並非一個平鋪的體用義所能籠侗也。劉蕺山以為只說一句「理即是氣之理」,「道心即人心之本心」,「義理之性即氣質之本性」,便可盡掃「千古支離之說」,此亦晚明士人秀才氣之大言欺人,故作驚人之筆耳。言學不可以如此匆遽張皇也。言圓頓自有言圓頓之方式、路數與規範,非是如此置斷即可統而一之也。劉蕺山於此相差甚遠!

依以上之疏解,如撥開其滯辭、不穩之辭、乖戾悖謬之辭所成之煙霧,吾人可知蕺山所謂理氣一之實義是如何。其所謂「一」者蓋即形而上下緊收緊吸而即體即用,顯微無間,體用一原之一滾地說之一耳。然此種意義之「一」必預設一種分別說的「二」,非不承認在分別說下有太極之為理以及理氣之分也。「二」即由此分別說下理氣之分直接地引申出。分後如何再關合而為一,乃是進一步的事,此則決定於對於理之體會為如何。劉籤山實有其超越的分解,非是一往一滾地說也。例如其言意根誠體,言意為心之所存,

\newpage\thispagestyle{empty}\addtocounter{page}{-1}\vspace*{-12mm}\begin{center}\noindent
\includegraphics[clip, trim=172pt 160pt 132pt 232pt, height=162mm]{ocr-input/image-1730.png}\end{center}

\newpage

\noindent 非心之所發;言知(良知)藏于意,非意之所起;又嚴分意與念,謂意是一機而二用(好善惡惡),念是兩在而異情,不可混念為意:凡此等等皆是超越的分解事。惟至說理氣時,則喜作一滾地說。此蓋由於其體已立,而即於此說本體論地即體即用耳,此如其言喜怒哀樂之未發已發為「存發一機」便是。實則直下如此說,並不合《中庸〉原意,而亦何嘗不可于此作一超越的分解,作一分別說耶?是以如真貫澈其意根誠體處之分解地體認則亦不能反對理氣處之分解地體認。其一滾地說必預設一分別地說。而且在其所預設的分別說中,蕺山對於太極實體之體會確與朱子不同,即他並未體會為只是理,只存有而不活動者,他實處處更能相應那於穆不已之天命流行之體,而即扣緊此於穆不已之天命實體以言太極真體,言意根誠體,乃至言理言體,是則其所謂理、所謂體、所謂太極乃必然是即存有即活動者。此方是差別點所在。其所以能至即體即用,顯微無間,體用一原之「一」者正因其所體會之體是即存有卽活動者,而朱子之所以不能至此而亦不欲說此,而只說理氣不離不雜,只說「太極者本然之妙,動靜者所乘之機」,乃至說掛搭、附著、依傍、跨馬等義者,正因其所體會之太極為只存有而不活動者,是則終於為二,而其「一」亦是關聯的一,非即體即用無間之一,卻正是不以體用言之有間之一,非即體即用之一原之一,卻正是理氣關聯上的兩原之一。此蓋即羅整菴、劉蕺山、黃梨洲等所不滿意之歧理氣為二也。然彼等不知其故,不知就體之體會不同而加以簡別與糾正,卻只就理氣之分本身而直接去爭二不二,是以終於成朝三暮四之胡纏夾,而終不足以難朱子也。夫承認有理氣之分矣,焉可不承認理氣為二物?夫理氣為一,如非只是氣,亦非不承認有

\newpage\thispagestyle{empty}\addtocounter{page}{-1}\vspace*{-12mm}\begin{center}\noindent
\includegraphics[clip, trim=151pt 134pt 140pt 242pt, height=162mm]{ocr-input/image-1734.png}\end{center}

\newpage\markright{第二部 \quad 分論一 \quad 第一章 \quad 周濂溪對於道體之體悟}

\noindent 理,又焉可不承認分別說的理氣為二?是以欲反之,而不知焦點之何在,故終於不能反,又有許多突兀乖戾之辭也。而又連及先後,並理先氣後而反之,以成其所謂一,是皆所謂不中肯之亂反也。夫一滾地說,自無所謂先後,然亦豈能因此而否認分別說的於穆不已之天命實體、太極真體之為本有先在耶?至於由一滾地說之之中再轉出許多所謂「統而一之」之陳述,如「性者心之性」,「情者性之情」,「心之性情」,「道心即人心之所以為心」,「義理之性即氣質之本性」等等,此種「統而一之」又豈足以難朱子之分別說耶?是故朱子之差唯在其體會太極為只是理、為只存有而不活動者之一點,而劉蕺山之體會則不同於此,何不於此自覺而加以簡別與糾正,而卻只落於分別說中去作朝三暮四之糾纏耶?是其不能透澈亦明矣。

劉蕺山尚能精切其意根誠體之說,尚能扣緊於穆不已之天命實體以言太極真體與性體,故撥其雲霧,而實義自不謬。至梨洲則於此並無真工夫,真知見,其辭語尤乖謬。亟欲說理氣是一,而竟落於視理為氣之謂詞,為關聯的特質之層次,是則儒家內聖之學之言道體性體全部倒塌矣!是並其師之學亦不能守也。亦由於為其師之過甚之辭所吸,不知其底子,便執認以為實,遂順之而下滾耳。

其(太極圖講義〉云:

\begin{quotation}\kaishu 通天地,亘古今,無非一氣而已。氣本一也,而有往來闔關
升降之殊,則分之為動靜;有動靜,則不得不分之為陰陽。
然此陰陽之動靜也,千條萬緒,紛紜膠轎,而卒不克亂。萬
古此寒署也,萬古此生長收藏也,莫知其所以然而然,是即\end{quotation}

\newpage\thispagestyle{empty}\addtocounter{page}{-1}\vspace*{-12mm}\begin{center}\noindent
\includegraphics[clip, trim=150pt 159pt 150pt 232pt, height=162mm]{ocr-input/image-1738.png}\end{center}

\newpage

\begin{quotation}\kaishu 所謂理也,所謂太極也。以其不紊而言,則謂之理;以其極
至而言,則謂之太極。識得此理,則知一陰一陽即是為物不
贰也。

其曰「無極」者,初非別有一物依於氣而立,附於氣而行。
或曰:因「易有太極」一言,遂疑陰陽之變易類有一物主宰
乎其間者。是不然矣。故不得不加「無極」二字。〔下
略〕。(此文(濂溪學案〉下附於蕺山(太極圖說解)下)\end{quotation}

\noindent 如此講太極、無極,而謂可以勝過朱子,其孰能信之?如此講法,理氣誠為一物矣,然理卻只成氣之自然變化之不紊,此只成自然主義,猶非其師之形而上下緊收緊吸卽體卽用之義也。由此而有「天地之間只有氣,更無理」,以及理氣「蓋一物而兩名,非兩物而一體」等看似漂亮而實沈淪之言。《明儒學案》是其一手精作之書,遇有涉及此問題處,輒有此類之案語,吾將詳論之於〈明道章·—本篇〉之附識,以明其對於「天命流行之體」全誤解。

「維天之命,於穆不已」是先秦儒家發展其道德形上學所依據之最根源的智慧,亦是了解其言道體、性體之法眼。朱子於此雖一間未達,只講成只存有而不活動者,然其系統卻全盡而一貫。陸象山於孟子學為不謬,然其興趣卻申展不至此,其紹述孟子之本心甚警策而精透,然而一涉及此方面,則似根本未入者,此其所以不全盡,亦是其粗處也。下段即稍論朱陸之辯以終此篇焉。

\subsection{象山疑〈太極圖說〉之非是}

朱子極力表揚〈太極圖說〉而尊信之,而當時陸氏兄弟又疑其

\newpage\thispagestyle{empty}\addtocounter{page}{-1}\vspace*{-12mm}\begin{center}\noindent
\includegraphics[clip, trim=170pt 128pt 118pt 245pt, height=162mm]{ocr-input/image-1742.png}\end{center}

\newpage\markright{第二部 \quad 分論一 \quad 第一章 \quad 周濂溪對於道體之體悟}

\noindent 非周子所為,如是展開一場劇烈爭辯,結果鬧得極不愉快。吾以為陸氏兄弟之疑是一時不成熟之疑,此場辯論,客觀地說,象山是失敗者。

象山與朱子書云:「梭山兄謂:(太極圖說〉與《通書》不類,疑非周子所為;不然,或是其學未成時所作;不然,則或是傳他人之文,後人不辨也。蓋《通書·理性命〉章言:中焉止矣。二氣五行,化生萬物。五殊二實,二本則一。曰一曰中,即太極也。未嘗於其上加無極字。〈動靜〉章言五行陰陽太極,亦無無極之文。假令(太極圖說〉是其所傳,或其少時所作,則作《通書》時,不言無極,蓋已知其說之非矣。此言殆未可忽也。」

此書後面復謂:「〈太極圖說〉以無極二字冠首,而《通書》終篇未嘗一及無極字。二程言論文字至多,亦未嘗一及無極字。假令其初實有是圖,觀其後來未嘗一及無極字,可見其道之進,而不自以為是也。兄今考訂註釋,表顯尊信,如此其至,恐未得為善祖述者也。」

此辯先由梭山發難,朱子復之。梭山原書已失傳。此書是象山述其兄梭山之意而代其兄與朱子辯也。由於象山之接力,遂成往復之劇辯。謂「〈太極圖說〉與《通書》不類」,表面上亦稍有之,然並非大不類,所差者只在《通書〉「無無極之文」耳,而象山亦只就這一點而謂〈圖說〉非周子所為,或其少時學未成時所作。此疑實亦一時不成熟之皮相之疑。如吾上文所解,〈圖說〉大體是根據《通書》之〈動靜〉章、〈理性命〉章、〈道〉章、〈聖學〉章而寫成,其義理骨干不外此四章。〈圖說〉全文,無論思理或語脈皆同於《通書〉,不可謂非濂溪手筆也。「無極而太極,太極動而

\newpage\thispagestyle{empty}\addtocounter{page}{-1}\vspace*{-12mm}\begin{center}\noindent
\includegraphics[clip, trim=157pt 149pt 146pt 242pt, height=162mm]{ocr-input/image-1746.png}\end{center}

\newpage

\noindent 生陽」兩語實即《通書·誠下〉「靜無而動有」一語之引申。而濂溪亦實可有「無極之極」之思路。《通書》多言無、無思、無為,雖可有通於老子,然濂溪並不以此為諱,而此本亦為《易傳〉中固有之辭。是故在〈圖說〉之機緣說「無極而太極」,稍加一「無極」字,不應構成嚴重之問題,此不足以決定〈圖說〉之非周子所為也。「無極而太極」實只是一太極,太極是主,無極只是遮狀詞,並非一獨立之實概念。二程不言無極字,不足以決定濂溪在〈圖說〉機緣上言之之非是。即朱子大講太極,亦未曾專說無極也。蓋無極只是一遮狀詞,並非一獨立之實概念,則實處只在太極,曉其意,只說太極即可耳,本不必處處皆須帶著無極字也。故〈圖說〉與《通書》之此點差別實不足以構成〈圖說〉非周子所為之關鍵,於以知陸氏兄弟之疑實是一時不成熟之皮相之疑。若看穿周子之思理,此疑即可冰釋。象山答朱子謂「梭山氣稟寬緩,觀書未嘗草草」,言其疑並非率爾而發。實則此種問題,一時之仔細並不算數。程度到了,生命相應,一眼可以看明,程度不到,生命不相應,一時之仔細,縱往復熟讀,亦不必能穿透。梭山之疑只為表面所吸住耳。至於象山之接力,則以學問不同為背景。朱子與象山隔閔太甚。朱子自始即斥象山為禪,象山從未一辯。象山斥朱子不見道,根本不贊成他那一套。象山辯〈太極圖說〉時,年在五十,正其學問發皇頂盛之時,其接力與朱子辯,乃是借題發揮耳。故〈與陶贊仲〉書云:「此數文皆明道之文,非止一時辯論之文也。」(《象山全集》卷十五)然獨立明道,表現其獨特之學風與精神,則可,而剋就(太極圖說〉以與朱子辯圖說之為偽,則陷於一不利之境。蓋借題發揮,亦不可無泛應曲當之本領。象山之學是

\newpage\thispagestyle{empty}\addtocounter{page}{-1}\vspace*{-12mm}\begin{center}\noindent
\includegraphics[clip, trim=168pt 131pt 122pt 247pt, height=162mm]{ocr-input/image-1750.png}\end{center}

\newpage\markright{第二部 \quad 分論一 \quad 第一章 \quad 周濂溪對於道體之體悟}

\noindent 孟子學,其思理與精神不惟與朱子相睽隔,且亦根本不必契濂溪所欣趣之一套,而平素對於《通書》與(圖說〉所呈現以及所牽連之辭語與思理亦乏深研之工夫,實亦不感興趣,故其辯論自難順通而曲當。只抓住「無極」一詞,便斷定其非周子所為,為老氏之學,濂溪有知必莞爾笑其為淺而躁矣。「無」之思理可有通於老子,然不因此便是老氏之學。此中曲折,象山未能深入而順通之。而此欠缺,則固不足以服朱子,即濂溪亦不必首肯也。原象山之所以不能深入而順通此中之曲折亦正因其不能會通從孔、孟至《中庸》、《易傳》之圓滿發展之故也。自此而言,象山不及明道。當然,象山亦自有其警策於明道處。

朱子〈答陸子美〉(梭山)第一書云:「只如太極篇首一句,最是長者所深排。然殊不知不言無極,則太極同於一物,而不足為萬化之根,〔「之根」象山引之作「根本」,下同】;不言太極,則無極淪於空寂,而不能為萬化之根。只此一句,便見其下語精密,微妙無窮。」

〈答陸子美〉第二書云:「且如太極之說,熹謂周先生之意,恐學者錯認太極別為一物,故著無極二字以明之。此是推原前賢立言之本意,所以不厭重複,蓋有深指。而來諭便謂熹以太極下同一物,是則非惟不盡周先生之妙旨,而於熹之淺陋妄說,亦未察其情矣。又謂著無極字,便有虛無好高之弊,則未知尊兄所謂太極,是有形器之物耶?無形器之物耶?若果無形而但有理,則無極即是無形,太極即是有理明矣,又安得虛無而好高乎?」

案:此兩書之辨解大體不誤。象山接過來〈與朱元晦〉書引之而駁之云:「夫太極者實有是理,聖人從而發明之耳。非以空言立

\newpage\thispagestyle{empty}\addtocounter{page}{-1}\vspace*{-12mm}\begin{center}\noindent
\includegraphics[clip, trim=164pt 166pt 153pt 236pt, height=162mm]{ocr-input/image-1754.png}\end{center}

\newpage

\noindent 論,使後人簸弄於頰舌紙筆之間也。其為萬化根本,固自素定。其足不足、能不能,豈以人言不言之故耶?」此答固甚美,此見象山之精神。自太極之存有言,固不在人之言不言,然自體悟詮顯上言,說說又何妨?本來無極二字言亦可,不言亦可。「不言無極」,太極固不必即「同於一物」,但為形容其為無稱之言、窮極之辭,則言之又豈定礙?(易傳〉固只言太極,未言無極,太極亦未嘗同於一物而不足為萬化根本,然濂溪已言之矣,則解而通之有何不可?

是以朱子答象山云:「伏羲作易,自一畫以下,文王演易自乾元以下,皆未嘗言太極也,而孔子言之。孔子贊易自太極以下,未嘗言無極也,而周子言之。夫先聖後聖,豈不同條而共貫哉?若於此有以灼然實見太極之真體,則知不言者不為少,而言之者不為多矣。何至若此之紛紛哉?」案此答亦甚美。此即言亦可,不言亦可也。辯至此,本應相視而笑,莫逆於心。然意氣之動,賢者不免,遂額外生枝,愈辯愈睽矣。字裡行間,互相誤解或不盡對方之語意者甚多,兹不暇一一指正。

然朱子之正意只是如此:「至於(大傳〉曰形而上者謂之道矣,而又曰一陰一陽之謂道,此豈真以陰陽爲形而上者哉?正所以見一陰一陽雖屬形器,然所以一陰一陽者是乃道體之所為也。故語道體之至極,則謂之太極,語太極之流行,則謂之道。雖有二名,初無兩體。周子所以謂之無極,正以其無方所、無形狀,以為在無物之前,而未嘗不立於有物之後;以為在陰陽之外,而未嘗不行乎陰陽之中;以為貫通全體,無乎不在,則又初無聲奧影響之可言也。」(朱子答象山第一書)。「無極而太極,猶曰莫之為而為,

\newpage\thispagestyle{empty}\addtocounter{page}{-1}\vspace*{-12mm}\begin{center}\noindent
\includegraphics[clip, trim=171pt 137pt 121pt 239pt, height=162mm]{ocr-input/image-1758.png}\end{center}

\newpage\markright{第二部 \quad 分論一 \quad 第一章 \quad 周濂溪對於道體之體悟}

\noindent 莫之致而至;又如曰:無為之爲。皆語勢之自然,非謂別有一物也。其意固若曰:非如皇極、民極、屋極之有方所形象,而但有此理之至極耳。若曉此意,則於聖門有何違叛而不肯道乎?上天之載是就有中說無。無極而太極,是就無中說有。〔案此言有中說無無中說有,皆只是說太極一事,非如老子有無對言,亦不同於濂溪言誠體「靜無而動有」之有無〕。若實見得,即說有說無,或先或後,都無妨礙。今必如此拘泥,強生分別,曾謂不尚空言,專務事實,而反如是乎」?(朱子答象山第二書)。

此為朱子解「無極而太極」一語之正意,此解大體不誤。象山之借題發揮,雖可謂為明道之文,然就辯(太極圖說〉之為具偽言,則失敗。

依此,〈太極圖〉可能源自於道教,而〈圖說〉則斷然是濂溪之思想。自儒家義理言,此圖並無多大價值,即無此圖,〈圖說〉之義理仍可獨立被理解。要者在〈圖說〉之思想。濂溪之藉圖以寄意,其所寄之意固甚嚴整,而亦全本于《通書》,然自其「藉圖」而言,則是一時之興會,所謂好玩而已。濂溪並非必須先獨自構畫一圖以及必須對應此圖始能結構出一套義理。此即示此圖對於〈圖說〉義理並無抒意上之必然關係,亦無理解上之必然關係。(此圖所承之原本對於道教人士之修煉也許有令人起興會處,濂溪雖顛倒之而成為〈太極圖〉,然自儒家義理言,則毫不令人起興會,無美感,亦無靈感。自吾個人言,主觀上雅不願看此圖)。濂溪決非先有此圖及〈圖說〉,然後始推演出《通書〉之義理,故自時間前後說,〈圖說〉決不能早於《通書》。自義理系統之次序言,亦不能以〈圖說〉為本而解《通書》,只能以《通書〉為本而解〈圖

\newpage\thispagestyle{empty}\addtocounter{page}{-1}\vspace*{-12mm}\begin{center}\noindent
\includegraphics[clip, trim=164pt 161pt 164pt 245pt, height=162mm]{ocr-input/image-1762.png}\end{center}

\newpage

\noindent 說〉。〈圖說〉固大體根據〈動靜〉章、〈理性命〉章、〈道〉章、〈聖學〉章而寫成,然《通書〉之論誠體者卻不能見之於〈圖說〉,此即示〈圖說〉並不能為(通書〉之先在綱領或綜論,而吾人卻必須根據《通書》所論之天道誠體以理解〈圖說〉之太極,始能見濂溪思想之一貫,並得知其心目中所意謂之太極之真實義。太極固已見於《通書》,然即此《通書》中之太極,亦須根據誠體之神、寂感真幾而理解之,因《通書》中對於天道誠體有詳解,而於太極,則只是一詞而無解釋故也。如果太極不是天道誠體以外之另一實體,如果〈圖說〉中「無極而太極」之太極不能有異於《通書〉中之太極,則除以《通書》中之天道誠體之實義解〈圖說〉之太極外,不能有別法可以得知濂溪心目中所意謂之太極之實義。是則〈太極圖說〉固非傳他人之文,亦非其少時學未成時所作,而且以〈圖說〉為本為主,而以《通書》為賓為副亦非是。《宋元學案·濂溪學案》首列《通書》,次列《圖說》,是也。象山以〈圖說〉爲偽,固非,朱子不以《通書〉為主,不以誠體解太極,亦非也。

朱子解「無極而太極」固大體不誤,然彼不以誠體之神、寂感真幾解太極,而將心神寂感抽掉,只成為形式意義的「但理」,則「太極動而生陽」便不可解。朱子之視太極為形式意義的「但理」,固足另成一系統,然在其系統中,「太極為萬化根本」是何意義的根本?「語太極之流行,則謂之道」,此「太極之流行」是何意義?「在無物之前,而未嘗不立於有物之後」是何意義?「在陰陽之外,而未嘗不行乎陰陽之中」是何意義?總之,太極之為形上的實理是何意義的實理?凡此等等,如徒形式地觀之,似甚易

\newpage\thispagestyle{empty}\addtocounter{page}{-1}\vspace*{-12mm}\begin{center}\noindent
\includegraphics[clip, trim=176pt 132pt 123pt 245pt, height=162mm]{ocr-input/image-1766.png}\end{center}

\newpage\markright{第二部 \quad 分論一 \quad 第一章 \quad 周濂溪對於道體之體悟}

\noindent 解。但若在朱子系統中,想求得其真實義,則卻甚不容易。此皆由太極之為「但理」與陰陽動靜之為氣之關係而生起,亦與其言心性關係相平行,此皆為朱子所煞費苦心者,故吾人亦不能輕易看過也。其系統之獨特而與象山為對立,胥繫乎此。此見前第三段可知。朱子系統之全部展示見下〈朱子部〉。吾於此首章只言其解〈太極圖說〉者以顯其對於太極真體(天命流行之體)了解之偏差。

吾於前段末已言「於穆不已」之「天命流行之體」一詞實了解儒家言道體性體之法眼。然自濂溪始直至劉蕺山止,對於此詞之體會只有兩路可走,即一、體會為即存有即活動,二、體會為只存有而不活動。

濂溪開端,雖於孔子之仁、孟子之心性了解極少,然客觀地自本體宇宙論面言道體,彼卻是將此道體體會為即存有即活動者,而且很能提得住。(不言性體者,以濂溪尚未清楚地意識到即以此「即存有即活動」之道體言性體也)。

橫渠雖然多滯辭,然由其「知太虛即氣,則有無隱顯神化性命通一無二」之語觀之,知其對於此道體亦當是體會為即存有即活動者,而且天道性命相貫通,即以此道體言性體,橫渠言之最為精澈。其于孔子之仁、孟子之心,了解的(至少關注的)亦比濂溪為多。此見下章可知。此是漸由《中庸》、《易傳》向《論》、《孟》轉也。

盛言於穆不已之體,由之以言「一本」,以判儒佛,自明道始。明道是真能相應此於穆不已之體而體會即存有即活動者。孔子之仁、孟子之心,亦於此通而一之。客觀面與主觀面皆飽滿而無缺

\newpage\thispagestyle{empty}\addtocounter{page}{-1}\vspace*{-12mm}\begin{center}\noindent
\includegraphics[clip, trim=176pt 155pt 129pt 237pt, height=162mm]{ocr-input/image-1770.png}\end{center}

\newpage

\noindent 憾,直下會通而為一本。而且善言圓頓,疏通無滯礙。此為內聖之學圓教之模型。

伊川依其直線分解的思考方式將道體性體體會為只是理,於是即存有即活動之於穆不已之體遂泯失而不見。朱子承之,自覺地體會為只存有而不活動者,此雖不合原義,然卻能全盡而貫激地完成其為一完整之系統,客觀面與主觀面皆相應,工夫之施設亦相應。此一系統,吾於後將說為:主觀地說為靜涵靜攝之系統,客觀地說為本體論的存有之系統,簡言之,為橫攝系統,而與即存有即活動之縱貫系統為對立,此為徹底之漸教。

胡五峰本於穆不已之體言性體,而又本明道上蔡之言仁,以孔子之仁與孟子之心去形著而證實之,此已開重新自覺地言心體性體終歸是一之門,而重點落在以心著性上。此亦是縱貫系統者。

象山崛起,撤開客觀面,直本孟子而言本心,並言心即理。夫暫撇開客觀面而自孟子入可也,然必須能申展而貢徹至於穆不已之天命流行之體處,方算充其極,圓整而飽滿。雖其言心即理、攝理歸心(此是對朱子將太極體會為只是理而發),其心體是涵蓋乾坤而為言,原則上似乎已飽滿,然於其所言之心體落實於於穆不已之天命流行之體處而一之,藉以糾正朱子之失,此步工夫,象山終欠缺。此因其在此方面太無興趣故也。故於此方面之學力亦缺乏。不知「維天之命於穆不已」是先秦儒家發展其道德形上學所依據之最根源的智慧。孔子雖言仁,然天道天命之老傳統仍然繼承而不背,故踐仁以知天。孟子雖言心性,亦仍然繼承而不沒,故盡心知性以知天。發展至《中庸》《易傳》,直由天命流行之體(或為物不貳生物不測之天道誠體或神體)以言性體,此本由孔子之仁、孟子

\newpage\thispagestyle{empty}\addtocounter{page}{-1}\vspace*{-12mm}\begin{center}\noindent
\includegraphics[clip, trim=153pt 134pt 147pt 246pt, height=162mm]{ocr-input/image-1774.png}\end{center}

\newpage\markright{第二部 \quad 分論一 \quad 第一章 \quad 周濂溪對於道體之體悟}

\noindent 之本心所透至者,故得從上說下來而一之。然則後之自孟子入者,亦必須能申展而貢徹至於穆不已之天命流行之體處方算充其極,圓整而飽滿。然而象山於此既無興趣,亦乏功力。此由其與朱子辯〈太極圖說〉之第二書中語可見。試看以下之辯語:

\begin{quotation}\kaishu 至如直以陰陽為形器,而不得為道,此尤不敢聞命。易之為
道,一陰一陽而已。先後始終,動靜晦明,上下進退,往來
闔闢,盈虛消長,尊卑貴賤,表裡隱顯,向背順逆,存亡得
喪,出入行藏,何適而非一陰一陽哉?奇偶相尋,變化無
窮。故曰:「其為道也屢遷。變動不居,周流六虛,上下無
常,剛柔相易,不可為典要,惟變所適。」〔……〕今顧以
陰陽為非道,而直謂之形器,其孰為昧於道器之分哉?\end{quotation}

\noindent 案:此辯非是。若以所云云為指點語可,若以於穆不已之體不離此所云云亦可,若直以陰陽為道則不可。此處,象山確不及朱子之子細。朱子答之云:

\begin{quotation}\kaishu 若以陰陽為形而上者,則形而下者復是何物?更請見教!若
熹愚見與夫所聞,則曰凡有形有象者皆器也。其所以為是器
之理者皆道也。如是,則來書所謂始終晦明,奇偶之屬,皆
陰陽所為之器,獨其所以為是器之理,如目之明,耳之聰,
父之慈,子之孝,乃爲道耳。如此分別,似差明白。不知尊
意以為如何?(原註:此一條亦極分明。切望略加思棗,便
見愚言不為無理,而其餘亦可以類推矣。)\end{quotation}

\newpage\thispagestyle{empty}\addtocounter{page}{-1}\vspace*{-12mm}\begin{center}\noindent
\includegraphics[clip, trim=171pt 161pt 127pt 227pt, height=162mm]{ocr-input/image-1778.png}\end{center}

\newpage

\noindent 朱子雖將形而上之道體會為只是理(只存有而不活動),此點雖差,然形而上下之分,不直以陰陽為道,此大綱脈並不錯。即將道體體會為即存有即活動,亦不能直以陰陽為道也。於穆不已之體非即陰陽之變動也。此見象山之粗矣。但《象山語錄》中有一條云:

\begin{quotation}\kaishu 自形而上者言之,謂之道。自形而下者言之,謂之器。天地
亦是器,其生覆形载必有理。\end{quotation}

\noindent 此又似不直以陰陽為道矣。朱子勸其「略加思索」不誤也。此等處稍一子細便見。若將孟子之本心貫澈下去,而貫注於此,則陰陽與於穆不已之體自有別,而此體亦必「即存有即活動」也。惜乎象山之不能充盡耳。光自孟子之本心處斥朱子為支離,爲不見道,而不能貫注於「於穆不已」之體處通澈而一之,以糾正朱子對於此體只體會為「只存有而不活動」之偏矣,則不足以點醒之而使之豁然心服也。雖其言心原則上是涵蓋宇宙而為言,然於於穆不已處功力不至,理會不澈,籠侗顛預,出語有差,即是不能充其極,至於圓整而飽滿。此是象山之不足處也。然其自孟子處彰著縱貫系統,其功甚大。

陽明承象山之言本心而前進,雖其氣象之直方大不及象山,然義理之精細處則有過之。惟其契接「於穆不已」之體處仍嫌弱而不深透。此即其仍未充盡而至於圓整而飽滿也。此其所以為顯教,而亦其後學之所以有狂態也。

劉蕺山措辭多滯礙,義理之曲折處不精熟,自辯才無礙之境界言,雖若不及其前輩,然其義理之綱維則有其弘深正大處。本象山

\newpage\thispagestyle{empty}\addtocounter{page}{-1}\vspace*{-12mm}\begin{center}\noindent
\includegraphics[clip, trim=153pt 140pt 149pt 242pt, height=162mm]{ocr-input/image-1782.png}\end{center}

\newpage\markright{第二部 \quad 分論一 \quad 第一章 \quad 周濂溪對於道體之體悟}

\noindent 之言本心(此是客觀地言之,蕺山本人對於象山並無了解,且甚至有誤解,對於孟子功力亦不深)、承陽明之言良知,進一步言意根誠體,(此是由《大學》之誠意慎獨而悟出,非由孟子、象山而悟出),由此以體證本心之所以為本心,並由此以契接《中庸》之由性體言獨體,進而並契接於穆不已之天命流行之體,並明言性宗與心宗,重視以心彰性,心性終歸是一,而又不失性體之超越,由此而重新恢復明道所定之圓教之模型,此其義理綱維之所以弘深而正大也。在此綱維下,由心宗申展而貫澈至於穆不已之天命流行之體,已可謂充其極而至圓整飽滿之境矣。此綱維大體類乎胡五峰之路數,而五峰不及蕺山之詳盡。蕺山誠不愧為一殿軍!雖在雲霧駁雜之中,然其綱維之實不可揜也。此亦經過六百年之磨練,不期而轉出此境,此恐或非蕺山之所料與所盡能自覺也。吾之疏導最終特重胡五峰與劉蕺山之綱維(非是重其成就),亦與歷來一般所見不同,而亦非吾始料之所及也。理之必然迫使吾作如此之宣稱耳。

以上九人者乃宋明儒學之綱柱。即活動即存有之縱貫系統乃是上承先秦儒家之大宗。通過明道之圓教模型與五峰蕺山之綱維乃能進窺聖人「以仁發明斯道」之「渾無罅縫」(象山語)與「天地氣象」。伊川朱子「只存有而不活動」之橫攝系統是此大宗之歧出,或亦可說是此大動脈中之一「靜」。朱子力敵千軍,獨全盡而貢澈地完成此橫攝系統,此其所以為偉大。以縱貫系統融化橫攝系統而一之,則是今日之事也。

此九人,濂溪、橫渠、明道為一組,伊川、朱子為一組,象山陽明為一組,五峰蕺山為一組。而以《論》、《孟》《中庸》、《易傳》為標準。《大學》是另端別起,非由《論》、

\newpage\thispagestyle{empty}\addtocounter{page}{-1}\vspace*{-12mm}\begin{center}\noindent
\includegraphics[clip, trim=143pt 154pt 176pt 247pt, height=162mm]{ocr-input/image-1786.png}\end{center}

\newpage

\noindent 〈孟〉一根而發。此九人間之脈絡或可以圖象表之如下:

\begin{center}
\noindent\includegraphics[width=1.06\linewidth]{ocr-image-p436-2.png}
\end{center}

\noindent (蕺山對於朱子處之虛線箭頭表示未有融攝好,亦無積極之關係)。

以上所陳是吾辛勤疏導融會貫通後之所得,亦是以下各章之綜綱。以濂溪開端,故亦先列於此。

\newpage\thispagestyle{empty}\addtocounter{page}{-1}\vspace*{-12mm}\begin{center}\noindent
\includegraphics[clip, trim=195pt 402pt 119pt 122pt, height=162mm]{ocr-input/image-1791.png}\end{center}

\newpage\markright{}

\chapter{張橫渠對於「天道性命
相貫通」之展示}

\section*{引言}\addcontentsline{toc}{section}{引言}

天道性命相貫通乃宋明儒共同之意識,亦是由先秦儒家之發展所看出之共同意識,不獨橫渠為然。茲所以獨於橫渠如此標題者,乃因橫渠作品中有若干語句表現此觀念最為精切諦當,亦是濂溪後首次自覺地如此說出者。如上章所述,濂溪對此尚未有積極之正視。對此觀念,如不能正視,則道德實踐即不能切而澈,而所言之天道神化亦無綰束,人將以為乃漫蕩之空談,徒騁個人一時之穎悟而已。

《正蒙·誠明篇》云:「天所性者通極於道,氣之昏明不足以蔽之。天所命者通極於性,遇之吉凶不足以戕之。」此四句即是天道性命相貫通之最精切而諦當之表示者。横渠《正蒙〉篇數繁多,然就其中所論及之內聖之學言,則以此義理為中心觀念,其他一切皆可由此而展開,亦皆可綰攝於此中心。

横渠《正蒙》沈雄弘偉,思參造化。他人思理零星散見,或出

\newpage\thispagestyle{empty}\addtocounter{page}{-1}\vspace*{-12mm}\begin{center}\noindent
\includegraphics[clip, trim=137pt 147pt 161pt 250pt, height=162mm]{ocr-input/image-1795.png}\end{center}

\newpage

\noindent 語輕鬆簡約。惟橫渠持論成篇,自鑄偉辭。(參看附錄六、2)。誠關河之雄傑儒家之法匠。然思深理微,表之為難,亦不能無滯辭。明道與伊川均極推尊其〈西銘〉,然於《正蒙》則卻不甚能相契。茲先從明道方面說。

\begin{quotation}\kaishu 1.伯淳言:〈西銘〉,某得此意,只是須得佗子厚有如此筆
力,佗人無緣做得。孟子以後未有人及此。得此文字,省
多少言語,且教佗人讀書。要之,仁孝之理備於此。須臾
而不於此,則便不仁不孝也。(《二程全書·遺書第二上〉,
〈二先生語二上〉。元豐己未,呂與叔東見二先生語)

2.孟子論王道便實。〔……】孟子而後,卻只有〈原道〉一
篇。其間語固多病,然要之大意儘近理。若〈西銘〉,則
是(原道〉之宗祖也。(原道〉卻只說到道,元未到得
〈西銘〉意思。據子厚子文,醇然無出此文也。自孟子
後,蓋未見此書。(同上。此條雖未明標是誰語,然衡之上
條,當亦是明道語)

3.子厚則高才,其學更先從雜博中過來。(同上。此條是承評
論游酢、楊時、暢大隱、呂進伯、天祺、景庸等人說下來,恐是
一整條。(遺書〉特另起一行,獨自成為一條,或為醒目也。)\end{quotation}

\noindent 據以上1與2兩條,其推尊(西銘〉可謂至矣。第3條則是綜評。然于《正蒙》則有微辭:

\begin{quotation}\kaishu 1.形而上者謂之道,形而下者謂之器。若如或者以清、虛、\end{quotation}

\newpage\thispagestyle{empty}\addtocounter{page}{-1}\vspace*{-12mm}\begin{center}\noindent
\includegraphics[clip, trim=192pt 133pt 112pt 256pt, height=162mm]{ocr-input/image-1799.png}\end{center}

\newpage\markright{第二部 \quad 分論一 \quad 第二章 \quad 張橫渠對於「天道性命相貫通」之展示}

\begin{quotation}\kaishu 一、大為天道,則(原注:一作「此」)乃以器言,而非
道也。(《二程全書・遺書第十一〉,〈明道先生語一〉,〈師
訓〉,劉絢質夫錄。)

2.子厚以清虚一大名天道,是以器言,非形而上者。(二
程全書·二程粹言〉卷之一,〈論道篇〉。〈粹言〉,龜山訂
定,南軒編次,乃變口語為文者。〈遺書〉中無此條,蓋即上條
之變文也。)

3.〔橫渠】立清虛一大為萬物之原,恐未安。須兼清濁虛實
乃可言神。道體物不遺,不應有方所。(二程全書·遺書
第二上〉,〈二先生語二上〉。元豐己未,呂與叔東見二先生
語。〔未註明誰語,衡之上第1條,當係明道語無疑。〕)

4.橫渠教人本只謂世學膠固,故說一個清、虛、一、大,只
圖得人稍損得沒去就道理來。然而人又更別處走!今日且
只道敬。(同上。〔未注明誰語,當亦係明道語無疑。(朱子
語類〉卷第九十三,綜論孔、孟、周、程處,有一條涉及此條謂
係劉絢所記,非是。〕)\end{quotation}

\noindent 據以上四條,則知明道對於《正蒙》之言太虛神體未能相契也。據吾今日細看《正蒙》,橫渠誠有滯辭,然其實意卻並不是以太虛神體為器(氣)、為形而下者。直謂其「以器言」,非是。又據橫渠「兼體不累以存神」之義說,橫渠正是「兼清濁虛實」以言神者,神並非是單屬於清也,亦非是以神為清氣之質性,以氣說神也。明道於此,未能盡其實。此種誤會亦由於橫渠簡別不精而然。然其實意不可揜,誤會終是誤會也。又,清、虛、一、大乃四個單詞,用

\newpage\thispagestyle{empty}\addtocounter{page}{-1}\vspace*{-12mm}\begin{center}\noindent
\includegraphics[clip, trim=177pt 152pt 132pt 240pt, height=162mm]{ocr-input/image-1803.png}\end{center}

\newpage

\noindent 來形容道體(大虛神體)者,乃明道之集合而成者,非橫渠原有此集詞語也。《正蒙〉言清、言虛、言一、言大者隨處皆是,然卻無此集詞語。橫渠言清、虛、一、大,只是對於道體之另一表示,即以太虛神體說道體,亦如濂溪之以誠體、寂感真幾說道體,乃至以太極說道體也。此只是一「即活動即存有」之實體之諸般表示,亦只是一義之展轉引申。此乃是對於道體之體悟事,非只是為「世學膠固」,令其減損也。人若於此不解,「更別處走」,則聞言「太極」而不解者亦可「別處走」也。「今日且只道敬」,如明道之言敬即是「純亦不已」,亦直通「於穆不已」之體,此則工夫本體打成一片,自比橫渠較為圓熟,然非不言本體也。而橫渠言天道性命相貫通,「聖人盡道,兼體而不累」,則亦非空言清虛一大也。又,明道之言道體直就「於穆不已」之天命之體而言,此亦自較貼切而面熟,不似橫渠之別開生面,自太虛神體以言者之生硬也。然橫渠雖較生硬,而其指歸總不太差。是則明道非不言道體,而橫渠之由清、虛、一、大以言太虛神體亦不至太走失也。若謂明道因橫渠之言清虛一大,令人別處走,而不敢再言本體,只說「敬」,則非是。《朱子語類》卷第九十三,綜論孔、孟、周、程處有一條云:

\begin{quotation}\kaishu 二程不言太極者,因劉絢紀程言:清虛、一、大,恐人別
處走,今只說敬。意只在所由只一理也。一理者,言仁義中
正而主静。\end{quotation}

\noindent 案:朱子謂劉絢(質夫)紀程言,實則非劉絢所記,乃呂與叔所記

\newpage\thispagestyle{empty}\addtocounter{page}{-1}\vspace*{-12mm}\begin{center}\noindent
\includegraphics[clip, trim=155pt 133pt 145pt 248pt, height=162mm]{ocr-input/image-1807.png}\end{center}

\newpage\markright{第二部 \quad 分論一 \quad 第二章 \quad 張橫渠對於「天道性命相貫通」之展示}

\noindent 也。此一時之誤記,不關緊要。然謂「二程不言太極」是因「橫渠之言清虛一大令人別處走」之故,則非是。今以客觀義理衡之,若誠有因橫渠之言「清虛一大,恐人別處走」,遂轉而向工夫切實處「只說敬」(只說後天的敬),並把太極真體等只提練而為一「理」字,因而只說理氣,則亦只伊川為然耳。明道並不如此也。明道固說敬,然明道之言「敬」是澈上澈下事,是直通於穆不已之體而言敬。明道言誠體、神體、易體、敬體、忠體、於穆不已之體,非不言本體也。如果說太極,此即太極耳。豈因未言太極一詞,即無其實乎?此詞之言不言,乃出於一時之偶然,或因主觀之機緣,或因點掇經語之方便,並非客觀義理上必不可言也。惟至伊川,言學之重點始落在敬與致知,而少談本體。對於道體,則只依其形而上下之分解,理解為只是理,而理氣二分之清楚割截遂於焉以形成。此是一步收縮提練,固有清晰處,然此步收縮提鍊對於道體之實義固不能無遺漏也。關於明道與伊川思理之異,詳見下〈明道章〉與〈伊川章〉。

吾今只明明道雖言敬,亦言道體,而客觀言之,其所體悟之道體與濂溪、橫渠猶相近,而更能貼切於「於穆不已」之體之原義。至於其不契橫渠之言清虛一大,則只是誤解,未能盡橫渠言太虛神體之實義,至少據今日所留傳之(正蒙〉觀之,雖不免有滯辭,而其實義固不可揜,一切誤會可因精簡而免除也。

伊川亦推尊〈西銘〉而不契《正蒙〉。其(答楊時論西銘書〉云:

\begin{quotation}\kaishu 〔上略〕。西銘之論則未然。橫渠立言誠有過者,乃在〈正\end{quotation}

\newpage\thispagestyle{empty}\addtocounter{page}{-1}\vspace*{-12mm}\begin{center}\noindent
\includegraphics[clip, trim=173pt 153pt 129pt 231pt, height=162mm]{ocr-input/image-1811.png}\end{center}

\newpage

\begin{quotation}\kaishu 蒙》。〈西銘〉之為書,推理以存義,擴前聖所未發,與孟
子性善、養氣之論同功。(原註:二者亦前聖所未發)。豈
墨氏之比哉?〈西銘〉明理一而分殊,墨氏則二本而無分。
(原注:老幼及人,理一也。愛無差等,本二也。)分殊之
蔽,私勝而失仁;無分之罪,兼愛而無義。分立而推理一,
以止私勝之流,仁之方也。無別而述兼愛,至於無父之極,
義之賊也。子比而同之,過矣!且謂言體而不及用,彼欲使
人推而行之,本為用也。反謂不及,不亦異乎?((二程全
書·伊川文集〉卷五)\end{quotation}

\noindent 案:楊時(即龜山)此時之疑(西銘〉,其解悟程度蓋甚低,於儒墨之大義亦不甚了了。伊川此答,爲之解說,並推尊〈西銘〉,甚爲諦當。南宋時,陸子美(梭山)函朱子亦疑〈太極圖說〉與〈西銘〉。關於(太極圖說〉,已見〈濂溪章〉。梭山之疑〈西銘〉亦示其解悟程度有不及,故朱子答而誨之。關於此方面,朱子當時實較成熟。而象山必欲接力代其兄辨,實不免意氣之私。象山與朱子雖只辨〈太極圖說〉,而未及〈西銘〉,然朱子〈答陸子美〉第二書云:「熹之愚陋,竊願尊兄更於二家之言少賜反覆,寬心游意,必使於其所說,如出於吾之所為者,而無纖芥之疑,然後可以發言立論,而斷其可否,則其為辨也不煩,而理之所在無不得矣。朱子此言實極懇篤,而梭山亦實有不及處,則朱子於解說後而稍示勸誨,不可謂過。而象山與朱子書,則引此段而代答云:「彼方深疑其說之非,則又安能使之如出於其所為者而無纖芥之疑哉?若其如出於吾之所為者而無纖芥之疑,則無不可矣,尚何論之可立,否之

\newpage\thispagestyle{empty}\addtocounter{page}{-1}\vspace*{-12mm}\begin{center}\noindent
\includegraphics[clip, trim=149pt 129pt 140pt 242pt, height=162mm]{ocr-input/image-1815.png}\end{center}

\newpage\markright{第二部 \quad 分論一 \quad 第二章 \quad 張橫渠對於「天道性命相貫通」之展示}

\noindent 可斷哉?兄之此言無乃亦少傷於急迫而未精耶?」象山此辯實無可取。縱〈太極圖說〉有可疑,而〈西銘〉實無可疑。梭山之疑顯是幼稚。象山不加簡別,概為其兄辯,儼若其兄之所疑無有不是者,此顯是意氣之爭,不及朱子遠甚。象山與朱子辯〈太極圖說〉時,年已五十,不可謂不成熟。梭山程度有不及,不可謂象山程度有不及。然則象山對於濂溪與橫渠之思理根本不發生興趣亦甚明矣。抑此亦非只主觀興趣問題,實儒家義理(道德的形上學)所必函之一面。若能會通先秦儒家由《論》《孟》至《中庸》《易傳》之發展,則知此顯是客觀義理問題,非只主觀興趣問題也。於以見象山於此方面之義理有不足也。

〈西銘〉之所以無問題正因其所述之主客觀面之踐履規模乃儒者之所共許,此非思參造化之理論問題,故無可疑也。儒者以乾坤爲大父母,繼天立極,盡性以開展其主客觀面之德行,此伊川所謂「理一而分殊」也。「乾稱父,坤稱母,予茲藐焉,乃渾然中處。故天地之塞,吾其體。天地之帥,吾其性。」此開頭數語即推明「理一」也。此下即明主客觀面德行之開展,所謂「分殊」也。就「理一」言,人資天稟地而生,即所以明其生命之本源也。朱子解云:「乾陽坤陰,此天地之氣塞乎兩間,而人物之所資以為體者也。故曰:天地之塞,吾其體。乾健坤順,此天地之志為氣之帥,而人物之所得以為性者也。故曰:天地之帥,吾其性。深察乎此,則父乾母坤混然中處之實可見矣。」又曰:「人物並生於天地之間,其所資以為體者,皆天地之塞,其所得以為性者,皆天地之帥也。」又曰:「塞是說氣。孟子所謂以直養而無害,則塞乎天地之間,即用這個塞字。張子此篇,大抵皆古人說話集來。」又曰:

\newpage\thispagestyle{empty}\addtocounter{page}{-1}\vspace*{-12mm}\begin{center}\noindent
\includegraphics[clip, trim=171pt 145pt 129pt 242pt, height=162mm]{ocr-input/image-1819.png}\end{center}

\newpage

\noindent 「塞只是氣。吾之體即天地之氣。帥是主宰,乃天地之常理也。吾之性即天地之理。問:天地之塞,如何是塞?曰:塞與帥字皆張子用字妙處。塞乃孟子塞天地之間,體乃孟子氣體之充者。有一毫不滿之處,則非塞也。帥乃志氣之帥,而有主宰之意。此〈西銘〉借用孟子論浩然之氣處。」(以上朱子語,皆見《宋元學案・橫渠學案〉)案:朱子此解不誤。以天地之塞(氣)為吾之體(形體之體),以天地之志(理或道)為吾之性,此猶仍是天道性命相貫通之義。而此義之形上學的展示,則見之於《正蒙》。〈西銘〉引孟子成語說之,無問題。籠統地以「理」字說「帥」,亦無問題。但以太虛神體形上學地展示之,則人不能無眼生之感。故伊川謂「橫渠立言誠有過者,乃在《正蒙〉」,不在(西銘〉也,亦非謂「天道性命相貫通.」之義亦有過也。試看其如何論《正蒙》。

伊川〈答橫渠先生書)云:

\begin{quotation}\kaishu 累書所論,病倦不能詳說。試以鄙見道其略,幸不責其妄
易。

觀吾叔之見,至正而謹嚴。如「虛無即氣,則無無」之語,
深探遠賾,豈後世學者所嘗慮及也?(原註:然此語未能無
過)。餘所論,以大概氣象言之,則有苦心極力之象,而無
寬裕溫厚之氣。非明睿所照,而考索至此。故意屢偏而言多
室。小出入時有之。(原注:明所照者,如日所觀,纖微盡
識之矣。考索至者,如揣料於物,約見髣鬆爾,能無差
乎?)

更願完養思慮,涵泳義理,他日自當條暢。何日得拜見,當\end{quotation}

\newpage\thispagestyle{empty}\addtocounter{page}{-1}\vspace*{-12mm}\begin{center}\noindent
\includegraphics[clip, trim=152pt 128pt 143pt 251pt, height=162mm]{ocr-input/image-1823.png}\end{center}

\newpage\markright{第二部 \quad 分論一 \quad 第二章 \quad 張橫渠對於「天道性命相貫通」之展示}

\begin{quotation}\kaishu 以來書為據,句句而諭,字字而議,庶及精微。牽勉病軀,
不能周悉。((二程全書·伊川文集)卷之五)\end{quotation}

\noindent 案:伊川所說「累書所論」恐即指《正蒙》中〈太和篇〉之文而說。其品鑒大體不謬。「虛無即氣則無無」,「無無」原作「虛無」,茲依《正蒙·太和篇》改。〈太和篇〉云:「知虛空即氣,則有無、隱顯、神化、性命通一無二。」又云:「知太虛即氣,則無無。」伊川雖謂此語「深探遠賾」,然復註云:「此語不能無過」。此語,若孤離觀之,不明其來歷,似不能無過,且難索解。然若知其言「虛無」、「虛空」或「太虛」之來歷,則此語亦無甚過差,亦不難索解。其言「虛空即氣」是根據「太虛無形,氣之本體」而來。而「太虛無形」則是根據「清通而不可象為神」而來。是則太虛、虛空、虛無,即清通不可象之神也。此猶是誠體寂感之神之別名。以虛或太虛言之者,一在對治老子之言「無」,二在對治佛家之言「空」。以「清通而不可象為神」規定「太虛」,此確然是儒家之心靈。字面上雖有時亦言「虛無」、「虛空」,此自不甚好,亦有類於道家之「無」,佛家之「空」,而實則以「虛」為主,其意義自不同於老之無,更不同於佛之空也,橫渠之意實只是虛或太虛。若只言「太虛即氣」,則在表意上自較佳,亦少生誤會。然有時仍以虛無、虛空言之者,吾意正為對治佛老,一箭雙鵰,乃遮撥上之方便也。如〈太和篇第一〉云:「此道不明,正由懵者略知體虛空為性,不知本天道為用。」此即以相類之詞收遮撥之用也。實則其所謂「虛空」只是其心中所意謂之虛或太虛,乃繫屬於「天道」,而即為道體之性也。然而老子亦言無,釋氏亦言

\newpage\thispagestyle{empty}\addtocounter{page}{-1}\vspace*{-12mm}\begin{center}\noindent
\includegraphics[clip, trim=188pt 158pt 127pt 241pt, height=162mm]{ocr-input/image-1827.png}\end{center}

\newpage

\noindent 空,是即所謂「略知體虛空為性」也。但老子言無,釋氏言空,本不同於儒家「於穆不已」之道體之具有創生的大用,即依此義,橫渠得以遮撥之,而謂其「不知本天道為用」也。是以其言「虛空」顯是遮撥上一箭雙雕之方便語。而其實義固不同於佛老之空與無,尤其非今日所謂「太空」,或如西哲所謂虛的空間也。

伊川對此「太虛」之思理不甚能把握。伊川所重視者乃是「理」,而不知理與神一。此已開朱子之系統,而漸遠於濂溪橫渠,甚至其兄明道之以誠體神用視天道。伊川語錄載:

\begin{quotation}\kaishu 1.〔上略】。又語及太虛。先生曰:亦無太虚·遂指虚曰:
皆是理,安得謂之虛?天下無實於理者。(二程全書·遺
書第三〉,〈二先生語三〉。謝顯道記伊川先生語)

2.或謂許大太虛,先生謂此語便不是。這裡論甚大與小!
(同上)\end{quotation}

\noindent 此兩條顯是對橫渠而發。首條「亦無太虛,遂指虛曰」云云,夫橫渠之「虛」豈可指乎?此好像把横渠所言之「太虚」視作虛空的空間!此種誤想尚不及明道之誤會為較接近也。第二條「許大太虛」,此非橫渠語。此或由別人之聯想而及,而為伊川所不喜。太虛神體自不可以大小論。橫渠〈大心篇〉亦言「天大無外」,此「大」字是遍在遍覆義。又言「大其心則體天下之物」,此本孟子大體小體而言。自其不梏於見聞,由小體解放出來言,則是擴大義;自其「體天下之物」言,則亦是遍在義。此皆非量義也。若以遍在義之「大」字說「太虛大而無外」亦無不可。但「許大太虛」

\newpage\thispagestyle{empty}\addtocounter{page}{-1}\vspace*{-12mm}\begin{center}\noindent
\includegraphics[clip, trim=143pt 132pt 152pt 247pt, height=162mm]{ocr-input/image-1831.png}\end{center}

\newpage\markright{第二部 \quad 分論一 \quad 第二章 \quad 張橫渠對於「天道性命相貫通」之展示}

\noindent 則不可。蓋「許大」之詞是量之觀念,說虛的空間可,說太虛神體則不可。橫渠不至如此不通也。故此語決是別人之聯想,非橫渠語也。伊川若以為此是橫渠意,則隔閡太甚!至其所謂「這裡論甚大與小」自亦非就橫渠之「太虛神體」義而說,乃是承上條「皆是理」而說也。

其次謂《正蒙》「有苦心極力之象,而無寬裕溫厚之氣。非明睿所照,而考索至此。故意屢偏而言多窒,小出入時有之。」此評亦有諦當處。觀橫渠之自道,亦言其思索義理「其有是者,皆只是億則屢中」,又謂「譬之昏者,觀一物必貯目於一物,不如明者,舉目皆見。」(見下附錄五)。宋明儒中,真能至「明睿所照」之境者,惟明道、象山、陽明庶幾近之。然此中除內心瑩澈外,亦與所言義理之層面有關。「明所照者,如日所觀」,此是明從中發,自有照功。是以凡繫屬於主體之義理皆易運轉自如,如莊子所謂「得其環中以應無窮」者是也。休乎天均而照之以天,自然「纖微盡識」矣。此非「揣料於物」也。然義理亦確有「揣料於物」者。凡客觀地思參造化以明各概念之分際以及其分合,此確不易,故常不免「有苦心極力之象」,所謂強探力索者是。若非只是主體之冥契,而復欲由客觀分解以展示之,則非「苦心極力」,即不足以盡其中之奧蘊。客觀地思參造化即是著於存在也。(此著字無劣義)。明從中發而照之以天,則可只是主體之冥契或只是一主之運轉,而不必著於存在者。客觀地著於存在,即不免有分解,主觀地(從主而觀)不著於存在,則可無分解。有分解,即所謂「揣料於物」也。無分解,即所謂「明睿所照」也。此義理層面之大較也。但當有分解而揣料於物,亦不必盡是造詣之不熟。「意屢偏而言多

\newpage\thispagestyle{empty}\addtocounter{page}{-1}\vspace*{-12mm}\begin{center}\noindent
\includegraphics[clip, trim=161pt 169pt 154pt 225pt, height=162mm]{ocr-input/image-1835.png}\end{center}

\newpage

\noindent 窒」亦蓋有其不可免之必然性。分際複雜,煞費照顧,顧此失彼,自難周詳。此則表面觀之,意自不免於屢偏。然若知其分際,則可雖偏而不偏,此所以看此種文字貴乎通其意也。就難於周詳言,亦須看其分解之大端方向何在。若雖偏而有當,而卻迷失其原初之大端方向而不能回歸,則須予以指明而隨時有補充。至於言多窒,固有關於個人語言文字之不善巧,而亦有關於語言文字本身之局限,此孟子所以有「不以辭害意」之戒也。故「明睿所照」有是分解者,有是不分解者,而分解者中之「意偏言窒」亦可只是語言文字本身之局限,而不妨礙其中之所言亦確是「明睿所照」也。當然横渠本人亦確有文字不善巧處。

就橫渠言,其客觀地思參造化,著於存在而施分解,所謂宇宙論的興趣者,(此種興趣乃宋、明儒所共同有者,亦是會通孔、孟與《中庸》《易傳》所必然有者,亦是北宋諸儒下屆朱子以《中庸》、《易傳》為綱,以《論》《孟》為緯,所特顯者,陸、王以《論》《孟》為綱,亦必透視至此,惟不甚於此著力施分解耳),此所謂「著」亦不是順知性思考之興趣純客觀地積極地著於存在而施分解推證與構造,如希臘傳統之形上學之所為者。其著於存在而施分解乃是以道德的創造性為支點者,他是在此決定性的綱領下施分解,故其分解有定向,有範圍,此是屬於「道德的形上學」者。他根據儒家「維天之命,於穆不已」之根源智慧,一眼看定這整個宇宙即是一道德的創造。這道德的創造與見之於個人自己處之道德的創造為同一模型、同一義蘊。在此同一模型下施分解,故其分解有定向,亦不過就天道、天命、生化不已而施分解耳;其所分解出之概念有定數,亦不過是道、理、太極、命性、寂感、

\newpage\thispagestyle{empty}\addtocounter{page}{-1}\vspace*{-12mm}\begin{center}\noindent
\includegraphics[clip, trim=164pt 149pt 125pt 222pt, height=162mm]{ocr-input/image-1839.png}\end{center}

\newpage\markright{第二部 \quad 分論一 \quad 第二章 \quad 張橫渠對於「天道性命相貫通」之展示}

\noindent 神、氣、化(神化或氣化)諸詞耳。惟是因就整個宇宙言,因氣、化諸概念而想及太和與太虛,甚至太極,故顯似有著於存在之意味,亦因而得名曰形上學。但此著於存在,是在道德創造之定向下著,故其為形上學亦是道德的形上學,此大端乃決定不可移者。因其分解有定向,故意雖屢偏,而不能有大偏,言雖多窒,而不能有大乖。故伊川亦云「小出入時有之」,決不會有大出入也。此決非西方順希腊傳統而來之純知解之形上學之系統多端而又時常乖違也。

伊川謂「何日得拜見,當以來書為據,句句而諭,字字而議,庶及精微。」若真順《正蒙》之分解,「句句而諭,字字而議」,此自可消磨其偏窒,而漸及於「精微」。然在諭議之過程中,吾想伊川之「意偏言窒」必有更甚於橫渠者。觀其對於理氣之分解、心性情之分解、才之分解、中和之分解,就儒家道德哲學(道德的形上學)言,亦未能意不偏而言不窒也。其(與呂大臨論中書〉尤見支蔓膠著而偏窒,尚不及呂大臨(與叔)遠甚。凡此皆見〈伊川章〉。吾茲可別舉一例以明之。伊川承上錄〈答橫渠先生書〉,復有〈再答〉一書。〈再答横渠先生書〉云:

\begin{quotation}\kaishu 昨書中所示之意,於愚意未安,敢再請於左右。今承盈幅之
諭,詳味三反,鄙意益未安。此非侍坐之間從容辯析不能究
也。豈尺書所可道哉?況十八叔大哥皆在京師。相見,且
請熟議。異日,當請聞之。內一事云已與大哥議而未合者,
試以所見言之。〔案:此所謂「大哥」即明道也。十八叔不
知何許人。\end{quotation}

\newpage\thispagestyle{empty}\addtocounter{page}{-1}\vspace*{-12mm}\begin{center}\noindent
\includegraphics[clip, trim=173pt 179pt 162pt 232pt, height=162mm]{ocr-input/image-1843.png}\end{center}

\newpage

\begin{quotation}\kaishu 所云:「孟子曰:必有事焉而勿正,心勿忘,勿助长也。此
信乎入神之奧。若欲以思慮求之,是既已自累其心於不神
矣,惡得而求之哉?」頤以為有所事乃有思也。無思,則無
所事矣。孟子之是言,方言養氣之道如是,何遽及神乎?氣
完則理正,理正則不私,不私之至則神。自養氣至此猶遠,
不可骤同語也。以孟子觀之,自見其次第也。當以「必有事
焉而勿正」為句,心字屬下句。此說與大哥之言固無殊,但
恐言之未詳爾。遠地末由拜見,豈勝傾戀之切?餘意未能具
道。

所諭:「勿忘者但不舍其虛明善應之心爾」,此言恐未便。
既有存於心而不舍,則何謂「虛明」?安能「善應」耶?
「虛明善應」乃可存而不忘乎?〔案:此段是書末之附識,
原文低一格】\end{quotation}

\noindent 案:據此書首段,橫渠對於伊川關於(正蒙》之批評必有較多之答覆,所謂「盈幅之諭」是也。橫渠原書不存,不知其如何答。此可置而不論。(吾上文之解說是吾今日據《正蒙》客觀地言之)。關于孟子「勿忘勿助長」之問題,據此書首段所述,橫渠已與明道當面談過,但「議而未合」。吾人今日亦不知其如何談。道理有時非書函所能盡,須面談。但面談有時反更齟齬,反不如書文之能盡其意。要在能相契,聲入心通,又須能從容。關此問題,據吾今日依文獻觀之,橫渠之意反與明道相近,而伊川此書之遮撥固不得橫渠意,即與其老兄亦遠也。

明道云:

\newpage\thispagestyle{empty}\addtocounter{page}{-1}\vspace*{-12mm}\begin{center}\noindent
\includegraphics[clip, trim=181pt 154pt 137pt 239pt, height=162mm]{ocr-input/image-1847.png}\end{center}

\newpage\markright{第二部 \quad 分論一 \quad 第二章 \quad 張橫渠對於「天道性命相貫通」之展示}

\begin{quotation}\kaishu 「鳶飛戾天,魚躍於淵,言其上下察也」。此一段,子思喫
緊為人處,與「必有事焉而勿正心」之意,同活潑潑地。會
得時,活潑潑地。不會得時,只是弄精神。(二程全書·遺
書第三〉,〈二先生語三〉。謝顯道記憶平日語。標明為明道語)\end{quotation}

\noindent 此段明道以「活潑潑地」說「上下察」,此固非《中庸》引《詩》之原意,而以之說「必有事焉而勿正心」(此又以「正心」爲句),朱子以為此「說闊,人有難曉處」(詳見〈伊川章・引言〉)。實則「必有事焉而勿正,心勿忘勿助長也」(此語點句問題有許多講法,皆不礙大義,此不必問),即工夫即本體,直指本心而言亦並無不可。此蓋承上文「集義所生,非義襲而取」說下來。若依明道之思路,此即本心流行,「純亦不已」也。雖有「勿」字之禁詞,亦不礙其「活潑潑地」。然則橫渠以為「此信乎入神之奧」,而不可「以思慮求之」,未為過差,而亦與明道之意無不合也。本心流行,純亦不已,常惺惺即常寂寂,此豈非「精義入神」之奧體乎?孟子謂「所存者神,所過者化」,即存此本心誠體純亦不已之「神」也,此亦即「虛明善應」之心也。然則橫渠謂「勿忘者但不舍其虛明善應之心爾」,「此言」亦無甚「未便」處。據此觀之,則伊川「所見」及其批評全部差謬矣!即其老兄亦未必贊同也。據此一端,即可知伊川之偏窒。

橫渠年稍長於二程,而於親戚關係上亦為尊輩。先明道而卒,卒年五十八。明道有〈哭張子厚先生〉之詩(《明道文集》卷一)。呂與叔(大臨)爲橫渠弟子。橫渠卒後,乃東赴洛陽見二程。《遺書》中「元豐己未呂與叔東見二先生語」上下兩卷,即東

\newpage\thispagestyle{empty}\addtocounter{page}{-1}\vspace*{-12mm}\begin{center}\noindent
\includegraphics[clip, trim=177pt 162pt 122pt 225pt, height=162mm]{ocr-input/image-1851.png}\end{center}

\newpage

\noindent 見二程時之記錄語也。此在《程氏遺書〉中分量最多,亦最有價值。二程門人猶未曾有一人能記如此之多也。明道批評橫渠之清虛一大處皆照錄,然呂與叔未背其師也。伊川「先生云:呂與叔守橫渠學甚固。每橫渠無說處皆相從。纔有說了,便不肯回。」(《二程全書·遺書第十九》,〈伊川先生語五〉,楊遵道錄)此非呂與叔之頑固,乃實有其真切處,而伊川亦實有不足以服人處。其與伊川往復辨論「中」之問題,伊川顯不及與叔之明透。徒以伊川為師輩。故措辭客氣而尊之。然辨至最後,與叔不耐,不欲與辨之意已露于言外矣。如云:「大臨更不敢拜書先生左右,恐煩往答。」與叔並未承認自己所見為非,而竟有此語,豈非不欲再辨之意乎?此其不耐之情已溢于言表矣。

\noindent 、 大體明道成熟甚早,見理亦透澈而圓明。其(答橫渠先生定性書〉,朱子謂是在二十二、三時作,此雖不必,然相當年輕則無疑。此後幾近三十年,直至其五十四歲而卒,所造益圓熟。橫渠謙懷,不恥下問。其成熟自較晚,然確有其自得自悟、自鑄偉辭者。其中心課題即在本天道性命相貫通以言「知虛空【太虛】即氣,則有無隱顯、神化、性命,通一無二。」在其展示此義之過程中,雖不免有滯辭、蕪辭(帶點煙火氣),不及明道之貼切與圓明,然此義並不謬,而且言之極為沈雄剛拔,確是大手筆。其對於道體之體悟亦與濂溪、明道為近,不失實體之「即活動即存有」義。此是先秦儒家所抒發之道體之古義,亦是本「天命於穆不已」而來之根源智慧也。伊川比其兄只差一歲。明道卒時,伊川五十三,不可謂其猶未成熟。彼年十八,即能作〈顏子所好何學論〉。然則於五十餘年間,與其兄同生並長,決非無所用心者。彼亦當自有其獨特之

\newpage\thispagestyle{empty}\addtocounter{page}{-1}\vspace*{-12mm}\begin{center}\noindent
\includegraphics[clip, trim=146pt 135pt 148pt 246pt, height=162mm]{ocr-input/image-1855.png}\end{center}

\newpage\markright{第二部 \quad 分論一 \quad 第二章 \quad 張橫渠對於「天道性命相貫通」之展示}

\noindent 心態與抒發義理之思路,此當在五十三歲以前即已確定者。吾人如不能謂其此後近二十年中有如何之變化或轉向,則此後之二十年只能使其心態與思路更加確定。明道在時,雖知彼兄弟二人性格有不同,各有所長,然不必能留心彼兄弟二人對於道體之體悟與抒發義理之思路有不同,雖在學問上大體方向相同。(兄弟之間與朋友究竟不同,不因兄弟之親便能於學問義理上多有客觀之了解也。)吾為此言,旨在明:明道透澈圓明,而伊川並不透澈圓明;其思理確漸轉而為另一型態;對於道體之體悟,濂溪、橫渠、明道猶相近,猶不失先秦之古義,而伊川之思理卻湊泊不上,亦確有偏差。彼之心態為分解型的心態實在論的心態、後天漸教的心態。自此而言,距其兄固遠,即距橫渠亦遠也。橫渠之生命確有其原始性,有其浩瀚之元氣,是帶點第昂尼秀斯型的理想主義之情調,惟不甚圓熟而已。清澈圓熟了,即是明道。然明道並非阿坡羅型也。伊川與朱子俱帶點阿坡羅型,都重理智的分析,具實在論的心態(不管是經驗的,抑或是超越的),此非真正的理想主義。而先秦儒家固是真正理想主義之根源,濂溪橫渠、明道猶不喪失。(雖皆偏重自《中庸》《易傳》言,此無傷。)惟自伊川始轉成另一型態,至朱子而大顯,而真正理想主義的情調亦喪失。

伊川以其分解的思路、實在論的心態,將道體收縮提練而為只是理(實理),依後天漸教的工夫入路,重點落在涵養與致知,遂轉而重視《大學》之致知格物。以如此之思理,其不能理解橫渠之太虛神體固亦宜也。彼雖不覺與其老兄有異,然對於道體之體悟實不同于其老兄(只得其老兄言「天理」義之一半,即「存有」義),其工夫入路亦不同於其老兄也。朱子承之而前進,即已不滿

\newpage\thispagestyle{empty}\addtocounter{page}{-1}\vspace*{-12mm}\begin{center}\noindent
\includegraphics[clip, trim=175pt 156pt 128pt 231pt, height=162mm]{ocr-input/image-1859.png}\end{center}

\newpage

\noindent 於明道矣。不過常為之諱而已。其對於橫渠,雖亦同樣推尊〈西銘〉,且為之作〈解義〉,與〈太極圖說〉同視,然於《正蒙》則極不相應,誤解亦多。彼雖順(太極圖說〉大講太極與陰陽動靜,此亦是關於道體之體悟工作,不似伊川之枯萎,然其基本精神與思理卻只是伊川之綱維,對於道體之體悟實不能至濂溪、橫渠、明道之境。雖盛稱濂溪,然對於濂溪所言之誠體、神體實無相應之契會,因而對於其所言之太極理解亦有偏差而不盡。對於明道,則只視為渾淪太高而置之。對於橫渠之《正蒙》,則全部不相應。試看下兩條便知:

\begin{quotation}\kaishu 1.橫渠將這道理抬弄得來大,後來更奈何不下。(《朱子語
類》卷第九十三,綜論孔、孟、周、程)

2.橫渠闢釋氏輪迴之說,然其說聚散屈伸處,其弊卻是大輪
迴。蓋釋氏是個個各自輪迴,橫渠是一發和〔當作合】
了,依舊一大輪迴。(《朱子語類)卷第九十九,〈張子書
二〉)\end{quotation}

\noindent 不知何以隔閡如此之甚!彼似根本不能理解其「知虛空即氣,則有無、隱顯、神化、性命通一無二」之本體宇宙論的體用不二義。《語類》卷第九十八與九十九兩卷皆是討論張子之書者,於其基本精神幾乎完全不能相應。吾以下之疏解,於其重大之誤解隨文指正之。

吾以下之疏解以《正蒙》中三篇為主,即〈太和篇第一〉、〈誠明篇第六〉、〈大心篇第七〉是也。其餘諸篇,如〈天道篇第

\newpage\thispagestyle{empty}\addtocounter{page}{-1}\vspace*{-12mm}\begin{center}\noindent
\includegraphics[clip, trim=147pt 137pt 140pt 238pt, height=162mm]{ocr-input/image-1863.png}\end{center}

\newpage\markright{第二部 \quad 分論一 \quad 第二章 \quad 張橫渠對於「天道性命相貫通」之展示}

\noindent 三〉、〈神化篇第四〉、〈乾稱篇第十七〉,則會而通之,藉以取證。

范育為橫渠弟子,作(正蒙序〉,甚中肯要。附錄於此,以便省覽。葉水心〈總述講學大旨〉即因范育此序而發,詳見綜論部。附錄一、范育(正蒙序〉:

\begin{quotation}\kaishu 嗚呼!道一而已。亘萬世,窮天地,理有易乎是哉?語上,
極乎高明;語下,涉乎形器;語大,至於無間;語小,入於
無朕。一有窒而不通,則於理為妄。故《正蒙》之言,高者
抑之,卑者舉之,虛者實之,礙者通之,眾者一之,合者散
之。要之,立乎大中至正之矩。

天之所以運,地之所以載,日月之所以明,鬼神之所以幽
風雲之所以變,江河之所以流,物理以辨,人倫以正。造端
者微,成能者著,知德者崇,就業者廣,本末上下,貫乎一
道。過乎此者,淫遁之狂言也。不及乎此者,邪詖之卑說
也。推而放諸有形而準,推而放諸無形而準,推而放諸至動
而準,推而放諸至靜而準。無不包矣,無不盡矣,無大可過
矣,無細可遺矣。言若是乎其至矣!聖人復起,無有間於斯
文矣!(《宋元學案補遺》卷十七,(横渠學案補遺上〉)\end{quotation}

\noindent 附錄二、范育(正蒙又序〉:

\begin{quotation}\kaishu 惟夫子之為此書也,有六經之所未載,聖人之所不言。或者\end{quotation}

\newpage\thispagestyle{empty}\addtocounter{page}{-1}\vspace*{-12mm}\begin{center}\noindent
\includegraphics[clip, trim=152pt 152pt 144pt 233pt, height=162mm]{ocr-input/image-1868.png}\end{center}

\newpage

\begin{quotation}\kaishu 疑其蓋不必道。若清、虛、一、大之語,適將取訾於末學。
予則異焉。

自孔、孟沒,學絕道喪,千有餘年。處士橫議,異端間作。
若浮圖、老子之書,天下共傳,與六經並行。而其徒「移」
〔當作「侈」】其說,以為大道精微之理,儒家之所不能
談,必取吾書為正。世之儒者亦自許曰:吾之六經未嘗語
也,孔、孟未嘗及也。從而信其書、宗其道,天下靡然同
風,無敢置疑於其間。況能奮一朝之辨,而與之較是非曲直
乎哉?

子張子獨以命世之宏才、曠古之絕識,參之以博文強記之
學,質之以稽天窮地之思,與堯、舜、孔、孟合德乎數千載
之間。閔乎道之不明,斯人之迷且病,天下之理泯然其將滅
也,故爲此言,與浮圖、老子辨。夫豈好異乎哉?蓋不得已
也。

浮圖以心為法,以空為真,故《正蒙》關之以天理之大。又
曰:「知虛空即氣,則有無、隱顯、神化、性命通一無
二。

老子以「無為」爲道,故《正蒙》關之曰:「不有兩,則無
一。

至於談死生之際,曰:「輪轉不息。能脫是者,則無生
滅。」或曰:「久生不死。」故《正蒙》關之曰:「太虚不
能無氣,氣不能不聚而為萬物,萬物不能不散而為太虛。」
夫為是言者豈得已哉?使二氏者真得至道之要,不二之理,
則吾何為紛紛然與之辨哉?其爲辨者,正欲排邪說,歸至\end{quotation}

\newpage\thispagestyle{empty}\addtocounter{page}{-1}\vspace*{-12mm}\begin{center}\noindent
\includegraphics[clip, trim=172pt 136pt 120pt 242pt, height=162mm]{ocr-input/image-1872.png}\end{center}

\newpage\markright{第二部 \quad 分論一 \quad 第二章 \quad 張橫渠對於「天道性命相貫通」之展示}

\begin{quotation}\kaishu 理,使萬世不惑而已。使彼二氏者,天下信之,出於孔子之
前,則六經之言有不道者乎?孟子嘗勤勤闢楊朱墨瞿矣。
若浮圖老子之言聞乎孟子之耳,焉有不闢之者乎?故予
曰:《正蒙》之言不得已而云也。(《宋元學案補遗》卷三十
一,(呂范諸儒學案補遺〉)\end{quotation}

\noindent 案:以上兩序恐只是一序,當合並為一。

\noindent 附錄三蘇昞(正蒙序)曰:

\begin{quotation}\kaishu 先生著《正蒙》書數萬言。一日,從容請曰:敢以區別成誦
何如?先生曰:吾之作是書也,譬之枯株,根本枝葉莫不悉
備。充榮之者,皆在人功而已。又如醉盤示兒,百物具在,
顧取者如何爾。於是輒就其編會歸義例,略效《論語》
《孟子》,篇次章句,以類相從,為十七篇。(《張子全
書〉卷二,〈正蒙〉題下附注引。全文不得見。蘇昞即蘇季明,亦
橫渠門人。)\end{quotation}

\noindent 附錄四、呂大臨〈橫渠先生行狀〉云:

\begin{quotation}\kaishu 熙寧九年秋,先生感異夢,忽以書屬門人,乃集所立言,謂
之《正蒙〉,出示門人曰:此書於歷年致思之所得。其言殆
與前聖合與,大要發端示人而已。其觸類廣之,則吾將有待
於學者。\end{quotation}

\newpage\thispagestyle{empty}\addtocounter{page}{-1}\vspace*{-12mm}\begin{center}\noindent
\includegraphics[clip, trim=167pt 158pt 150pt 241pt, height=162mm]{ocr-input/image-1876.png}\end{center}

\newpage

\begin{quotation}\kaishu 正如老木之株,枝別固多,所少者潤澤華葉爾。(《張子全
書〉卷十五)\end{quotation}

\noindent 附錄五、橫渠〈自道〉中有一段云:

\begin{quotation}\kaishu 某學來三十年,自來作文字說義理無限。其有是者,皆只是
憶〔億】則屢中。譬之穿窬之盜,將竊取室中之物,而未知
物之所藏處。或探知於外人,或隔牆聽人之言,終不能自
到,說得皆未是實。觀古人之書,如探知於外人;聞朋友之
論,如聞隔牆之言。皆未得其門而入,不見宗廟之美家室
之好。比歲方似入至其中,知其中是美是善,不肯復出。天
下之議論莫能易也。譬如鑿一穴,已有見。又若既至其
中,卻無燭,未能盡室中之有。須索移動,方有所見。言移
動者,謂逐事要思。譬之昏者,觀一物必貯目於一〔物】
〔脫「物」字,當補】。不如明者,舉目皆見。此某不敢自
欺,亦不敢自謙。所言皆實事。學者,又譬之知有物而不肯
捨去者,有之;以為難入,不濟事,而去者,有之。(《張
子全書》卷七)\end{quotation}

\noindent 附錄六、橫渠〈語錄抄〉(《張子全書》卷十二):

\begin{quotation}\kaishu 1.某比來所得義理儘彌久而不能變,必是屢中於其間。只是
昔日所難,今日所易;昔日見得心煩,今日見得心約。到
近上,更約,必是精處,尤〔又】更約也。\end{quotation}

\newpage\thispagestyle{empty}\addtocounter{page}{-1}\vspace*{-12mm}\begin{center}\noindent
\includegraphics[clip, trim=174pt 130pt 131pt 258pt, height=162mm]{ocr-input/image-1880.png}\end{center}

\newpage\markright{第二部 \quad 分論一 \quad 第二章 \quad 張橫渠對於「天道性命相貫通」之展示}

\begin{quotation}\kaishu 2.當自立說以明性,不可以遺言附會解之。若孟子言:「不
成章不達」、「所性」、「四體不言而喻」,此非孔子曾
言,而孟子言之,此是心解也。\end{quotation}

\section{《正蒙·太和篇第一》疏解:道體義疏解}

\subsection{「太和所謂道」:氣與神}

\begin{quotation}\kaishu 1.太和所謂道。中函浮沈、升降、動靜相感之性,是生綱
相盪勝負屈伸之始。其來也,幾微易簡;其究也,廣大
堅固。起知於易者乾乎?效法於簡者坤乎?散殊而可象為
氣,清通而不可象為神。不如野馬絪縕,不足謂之太和。
語道者知此,謂之知道。學《易》者見此,謂之見易。不
如是,雖周公才美,其智不足稱也已。\end{quotation}

\noindent 案:此為(太和篇〉之首段,大體是根據《易傳》重新消化而成者。其所重新消化而成者,是以「太和」為首出,以「太和」規定道。「太和」即至和。太和而能創生宇宙之秩序即謂為「道」。此是總持地說。若再分解地說,則可以分解而為氣與神,分解而為乾坤知能之易與簡。此是〈太和篇〉之總綱領,亦是《正蒙》著於存在而思參造化之總綱領,其餘皆由此展轉引生。然以「野馬細紐」來形容太和,則言雖不窒,而意不能無偏。蓋野馬細是氣之事,若以氣之絪說太和、說道,則著於氣之意味太重,因而自然主義之意味亦太重,此所以易被人誤解為唯氣論也。然而橫渠以天道性

\newpage\thispagestyle{empty}\addtocounter{page}{-1}\vspace*{-12mm}\begin{center}\noindent
\includegraphics[clip, trim=161pt 160pt 148pt 232pt, height=162mm]{ocr-input/image-1884.png}\end{center}

\newpage

\noindent 命相貫通為其思參造化之重點,此實正宗之儒家思理,決不可視之為唯氣論者。是以橫渠以野馬絪縕之太和為首出之觀念,由之以說道,不是很好之消化。其與《易傳〉窮神知化之大義不能無距離,不如濂溪之由誠體說天道為簡潔精微而復能提得住也。

橫渠於此很重視莊生之野馬(易〉之絪縕。其實「絪紐」一詞不是(易傳〉中之重要詞語,亦不是其綱領概念。(繫辭傳〉下第五章(朱註分章)云:「天地絪,萬物化醇。男女構精,萬物化生。《易》〔〈損〉卦六三】曰:三人行,則損一人。一人行,則得其友。言致一也。」朱子注云:「絪,交密之狀;醇,謂厚而凝也,言氣化者也。化生,形化者也。此釋〈損〉六三爻義。」言以「天地絪組,萬物化醇;男女構精,萬物化生」釋〈損〉卦六三爻所表示之「致一」之義也。朱子注〈損〉六三爻云:「下卦本乾,而損上爻以益坤,三人行而損一人也。一陽上而一陰下,一人行而得其友也。兩相與則專,三則雜而亂。卦有此象,故戒占者當致一也。」朱注之「致一」即根據(繫辭傳〉之「言致一也」句而來。致一即專一。兩相與則專一而不雜,故以天地、男女之對偶為喻也。〈損〉六三爻〈象〉曰:「一人行,三則疑也。」此當云:「一人行,則得其友,三則疑也。」「疑」即朱注「雜而亂」之所本。此〈象傳〉即解析六三爻辭之義。「三人行,則損一人」為二,「一人行,則得其友」亦為二。此重在說對偶之專一也。故云:「三則疑也」。兩相與則專而密,精神貫注,生命契合無間。〈繫辭傳〉之作者即由此而說「天地細,萬物化醇,男女構精,萬物化生」也。此純從形而下之精與氣說。為表示專一,則亦可有之聯想。然橫渠即由此絪說太和說道,則嫌著而濁。陰陽之偶固

\newpage\thispagestyle{empty}\addtocounter{page}{-1}\vspace*{-12mm}\begin{center}\noindent
\includegraphics[clip, trim=172pt 139pt 121pt 234pt, height=162mm]{ocr-input/image-1888.png}\end{center}

\newpage\markright{第二部 \quad 分論一 \quad 第二章 \quad 張橫渠對於「天道性命相貫通」之展示}

\noindent 重要,橫渠固甚重視此兩之偶者。言氣化,不能不重視此對偶。故下文有云:「兩不立,則一不可見。」又云:「不有兩,則無一。」然所以立「兩」,則重在說「一」。(此「一」不是專一之一。)故又云:「一不可見,則兩之用息。」此「一」是妙一之一,是綜和之一,是根據神、虛而言者。(虛、太虛見下文)。「道」當該偏重在「一」處說。一固不離「兩」,然只兩氣之絪固不必即是道也。氣自身之絪組固表示一種「和」,然野馬絪組即是太和、即是道,則失之。《荀子·正名篇〉云:「生之和所生,精合感應,不事而自然,謂之性。」此言「生之和」正是自然生命絪紐之和,乃純屬於氣者。而荀子由此言「性」,正是自然生命之氣性,並無形上之意義,亦無道德之意義。而橫渠之言天道性命當不會是此自然生命之氣性,故知「太和所謂道」一語,此中所謂「太和」,若云不離野馬絪縕可,若云野馬絪縕即是太和、即是道,則非是。故「太和」一詞必進而由「太虛」以提之,方能立得住,而不落於唯氣論。「太和」固是總持地說,亦是現象學地描述地說。而其所以然之超越之體,由之可以說太和,由太和而可以說道者,則在太虛之神也。《莊子·逍遙游〉云:「野馬也,塵埃也,生物之以息相吹也。」「野馬」是「春月澤中游氣」。「塵埃」是言「天地間氣蓊鬱似塵埃揚也」。(參看郭慶藩《莊子集釋》)皆是言氣息之蓊鬱飛揚,而陰陽之絪亦即是「蓊鬱」之意也。橫渠由野馬絪縕說太和、說道,顯然是描寫之指點語,即由宇宙之廣生大生、充沛豐盛,而顯示道體之創生義。故核實言之,創生之實體是道。而非游氣之絪縕即是道也。如此理會方可不至使橫渠成為唯氣論者。此若由其展轉引申之解說,會通而觀之,自甚顯

\newpage\thispagestyle{empty}\addtocounter{page}{-1}\vspace*{-12mm}\begin{center}\noindent
\includegraphics[clip, trim=152pt 155pt 150pt 234pt, height=162mm]{ocr-input/image-1892.png}\end{center}

\newpage

\noindent 然也。

依此,「太和所謂道」一語,是對於道之總持地說,亦是現象學之描述地指點說,中含三義:一·能創生義;二·帶氣化之行程義;三・至動而不亂之秩序義(理則義)。由此三義皆可說道,有時偏於一面說。三義俱備,方是「道」一詞之完整義。橫渠雖有時喜就氣化之行程義說道,如下文「由氣化有道之名」,便是就行程義說道,此亦是共許之義,如朱子「語道體之至極,則謂之太極,語太極之流行,則謂之道」(答象山書),亦是就「流行」說道,流行即行程義也,明道有「浩浩大道」之語,「浩浩」亦行程義也,普通以大路喻道,大路亦有行程義,王弼亦云:「夫道也者,取乎萬物之所由也」,又云:「故涉之乎無物而不由,則稱之曰道」(《老子微旨例略》),此亦可喻如大路之義,但雖可就氣化之行程義說道,並非此實然平鋪之氣化即是道,必須提起來通至其創生義始可。「太和所謂道」,亦不是此實然平鋪之氣化。乃是能創生此氣化之至和也。依此,「由氣化,有道之名」只是太和之帶著氣化說而已。並非截斷其創生義,只執「實然平鋪之氣化」以為道也。道若大路,取萬物之所由,亦不只是那平擺之行程,亦必有根源義、宗主義,此即其創生義也。

「太和」是總宇宙全體而言之至和,是一極至之創生原理,並不是自然生命之絪縕之和。「不如野馬絪縕,不足謂之太和」,此乃是譬解語,亦是指點之描述語,乃就天地之廣生大生、充沛豐盛,而言其所以然之至和也。非真執著於游氣本身之絪而認為此即是道也。若如此,則真成為唯氣論矣。「中函浮沈、升降、動靜、相感之性,是生絪、相盪、勝負、屈伸之始。其來也,幾微

\newpage\thispagestyle{empty}\addtocounter{page}{-1}\vspace*{-12mm}\begin{center}\noindent
\includegraphics[clip, trim=174pt 140pt 127pt 242pt, height=162mm]{ocr-input/image-1896.png}\end{center}

\newpage\markright{第二部 \quad 分論一 \quad 第二章 \quad 張橫渠對於「天道性命相貫通」之展示}

\noindent 易簡:其究也,廣大堅固。」此數語即綜言太和之創生義。太和而能創生宇宙之秩序即曰道。

太和(道)何以能有此創生之性能?深入而分解之,則曰「乾以易知,坤以簡能」而已矣。〈繫辭傳〉首章:「乾知大始,坤作成物。乾以易知,神以簡能。」〈乾彖〉云:「大哉乾元,萬物資始」,故〈傳〉云:「乾知大始。」知,猶主也,主管之意。乾元主管宇宙之始,故其為始乃「大始」,亦即萬物之本源也。故此始即創始之始,言萬物由此而始生也。此始非時間之始,乃理體之始、價值之始。乾元為一創造原則,只是真實生命之常昭明而不陷溺,故能創生一切也。其主知大始是以至易之方式主,非有疑難也。(〈繫辭傳〉下第一章:「夫乾確然示人易矣。」第十二章:「夫乾天下之至健也,德行恆易以知險。」)明確至健故易也。常昭明,故明確。不陷溺,故至健。至易亦至和也。無間雜謂之和,純一謂之易。是以太和首表現而為「大始」之易知,由此而繁興大用也。有始即有終。乾元創始之,坤元即隨而終成之。故坤元為終成原則,或凝聚原則。終成而凝聚之,即「能」也。老子曰:「道生之,德畜之,物形之,勢成之。」道生德畜是創生原則。(此只借用助解,在道家,道生義自別)。物形勢成是凝聚原則,亦屬於「能」也。「能」為「材質」觀念。能者即有此資具而能體現、終成(具體化)乾元之創始也。乾元之知大始為「心靈」觀念。心靈創始之,材質終成之。材質之終成隤然至順,((繫辭傳〉下第一章:「夫坤隤然示人簡矣」),故其能是簡能,即以「簡」之方式表現其凝聚終成之「能」也。横渠云:「起知於易者乾乎?效法於簡者坤乎?」即根據「乾以易知,坤以簡能」而言也。「起知」

\newpage\thispagestyle{empty}\addtocounter{page}{-1}\vspace*{-12mm}\begin{center}\noindent
\includegraphics[clip, trim=189pt 161pt 129pt 240pt, height=162mm]{ocr-input/image-1900.png}\end{center}

\newpage

\noindent 句,言以易之方式表現其知大始者乃是乾也。「效法」句是根據〈繫辭傳〉上第五章「效法之謂坤」而言。朱注:「效,呈也。法,謂造化之詳密而可見者。」是則「效法於簡者坤乎」意即:以簡之方式而呈現致效其法相者乃是坤也。如嚴格對稱言之,此句當為「效能於簡者坤乎」(坤以簡能),言:以簡之方式而效其終成之能者乃是坤也。橫渠在此忽轉而本「效法之謂坤」而言,義雖通,而文不對稱,此即其隱晦處。

由「起知」與「效法」兩句,即知太和之道之創造過程可剖解而為乾知與坤能。專言之,太和之道之所以為道乃在「乾知」處,不在「坤能」處。籠統言之,則乾知坤能之終始過程即是天道之創生過程,亦即乾道之元亨利貞也。以乾元統坤元,坤元即含於「乾道變化、各正性命」之終始過程之中也。但分解專言之,則道之所以為道,太和之所以為和,須從「乾知」說,方能提得住。

由此乾知與坤能之分,再進一步宇宙論地分而為「氣」與「神」之兩概念,此即「散殊而可象為氣,清通而不可象為神」兩語之所示。神固不離氣,然畢竟神是神,而不是氣,氣是氣,而不是神,神與氣可分別建立。吾人可本《易傳》,於乾知之易處說神,於坤能之簡處說氣。無論是「效能於簡」,或是「效法於簡」,其所效之「能」或「法」總是有象有跡者。簡是言其「隤〔頹〕然至順」。雖隤然至順,而總是有象有跡,故屬於「氣」之事也。氣有象迹,可言散殊,故云「散殊而可象為氣」。此言散列殊異而可有象或可呈現為象者便是氣。乾知之易無象跡、無聲臭,然純一至和一片昭明,而不可以形隔,故可於此處說「神」。橫渠云:「清通不可象為神」。純一至和,一片昭明,即「清通」

\newpage\thispagestyle{empty}\addtocounter{page}{-1}\vspace*{-12mm}\begin{center}\noindent
\includegraphics[clip, trim=152pt 137pt 160pt 251pt, height=162mm]{ocr-input/image-1904.png}\end{center}

\newpage\markright{第二部 \quad 分論一 \quad 第二章 \quad 張橫渠對於「天道性命相貫通」之展示}

\noindent 也。無象迹、無聲臭,即「不可象」也。「不可象」是言其無象迹而不可以象論,不可以跡拘。此即是神也。

案:〈乾称篇第十七〉云:「凡可狀者有也,凡有皆象也,凡象皆氣也。氣之性本虛而神,則神與性乃氣所固有。此鬼神所以體物而不可遺也。」「氣之性」即氣之體性。此體性是氣之超越的體性,遍運乎氣而為之體者。此性是一是過,不是散殊可象之氣自身之曲曲折折之質性。氣自身曲曲折折之質性是氣之凝聚或結聚之性,是現象的性,而橫渠此處所說之「氣之性」是「遍運乎氣而為之體」之「超越的性」,本體的性,乃形而上者。氣以此為體即以此為性。橫渠以「虛而神」規定此體性,故此體性是遍是一,是清通而不可象者。謂此為氣所固有,此「固有」乃是超越地固有,因「運之而為其體」而為其所固有,不是現象地固有。(鬼神體物不可遺本《中庸〉,詳簡見下第八段)。

以上爲〈太和篇〉之首段,其主要觀念如下:

一、「太和所謂道」。

二、乾知坤能——易知簡能。「起知於易者乾乎?效法於簡者坤乎?」

三、神與氣。「散殊而可象為氣,清通不可象為神。」

\subsection{太虛與氣}

\begin{quotation}\kaishu 2.太虛無形,氣之本體。其聚其散,變化之客形爾。至靜無
感,性之淵源。有識有知,物交之客感爾。客感客形與無
感無形,惟盡性者一之。\end{quotation}

\newpage\thispagestyle{empty}\addtocounter{page}{-1}\vspace*{-12mm}\begin{center}\noindent
\includegraphics[clip, trim=188pt 213pt 124pt 236pt, height=162mm]{ocr-input/image-1908.png}\end{center}

\newpage

\noindent 案:此第二段提出「太虛」一詞,是由「清通而不可象為神」而說者。吾人即可以「清通無象之神」來規定「太虛」。太和是綜持說之詞,以明道之創生義為主。太虛是由分解而立者,一方與氣為對立,一方又定住太和之所以為和,道之所以為創生之真幾。「太虛無形,氣之本體」,此與〈乾稱篇〉「氣之性本虛與神」為同意語。「氣之性」是氣之超越的體性,是遍運乎氣而為之體,故此處直云「氣之本體」。說本體比較妥當,不易生誤解。說性則須簡別提醒。氣以太虛——清通之神———為體,則氣始活。活者變化之謂爾·浮沉、升降、動靜、相感、細、相盪、勝負、屈伸,皆氣之活用也。或聚或散亦氣之活用也。故云「其聚其散,變化之客形爾。」「其」字指氣言。氣之聚或散,乃至浮沉、升降等,皆不過是氣之變化活用之「客形」爾。「客形」是橫渠自鑄之美詞。客者過客之客,是暫時義。「客形」者即暫時之形態,或時動中之形態(temporal forms or modes),即皆氣之變化所呈之「相」也,所謂「效法之謂坤」者是也。(解見上)。氣變雖有客形,而清通之神與虛則遍而一,乃其常體。

落於個體生命上說,此清通之神、太虛即吾人之「性」也。(就其遍運乎氣而為之體言,亦可說性,此即是天地之性。此與就個體生命處說,其義一也。)此清通之神、太虛之體,在吾人生命處,如從其「至靜無感」說,則可認為是性體之最深之根源,即是性體之最深奧處最隱密處。故云:「至靜無感,性之淵源。」「性之淵源」不是說性體還有另一個最深之根源,乃是說此即是性體自身之最深奧處、最隱密處。「至靜無感」即是「寂然不動」。「寂然」是性體自身之寂然,「感而遂通」亦是性體自身之神用。

\newpage\thispagestyle{empty}\addtocounter{page}{-1}\vspace*{-12mm}\begin{center}\noindent
\includegraphics[clip, trim=154pt 140pt 158pt 250pt, height=162mm]{ocr-input/image-1912.png}\end{center}

\newpage\markright{第二部 \quad 分論一 \quad 第二章 \quad 張橫渠對於「天道性命相貫通」之展示}

\noindent 皆是就性體自身說也,亦是就清通之神、太虛自身說也。但落在個體生命處說,識與知亦是其感之形態也。此處之形態亦是性體自身(清通之神)接於物時所呈現之暫時之相,此即曰「客感」,或「感之暫時形態」(temporal forms of feeling)。故云:「有識有知,物交之客感爾」。如此,太虛固可以「清通之神」定,實亦可以「寂感真幾」定。寂感真幾即是寂感之神。總之,是指點一創造之真幾創造之實體(creative feeling,creative reality)。此真幾實體本身是即寂卽感寂感一如的;總言之曰「神」亦可,神以妙用之義定;曰太虛亦可,太虛以「清通無迹」定。

清通虛體之神雖是寂然不動感而逐通,然這只是剋就真體自身而作之形上的陳述,即真體自身自如此。但如果落於個體生命而為性,則當其與物接而有感,因形軀之限、私欲感性之雜,其感也不必能「遂通天下之故」,即,不必能通澈朗潤而無滯礙。如是,則性體即潛隱而不必能暢通,因而亦不必能成其道德創造、潤身踐形之大用。此須有一自覺地作道德實踐之勁力以復其真體,此即所謂「盡性」之工夫也。盡性者期於性體能使之充分實現或呈現之謂也。在盡性之工夫中,清通虛體之神與其所運之氣之變化之客形以及其自身接於物時所呈現之客感逐能貫通而為一。清通虛體之神全澈於客感客形中而妙運之以成其為生生之變化,而生生之變化中之客感客形亦全融化於清通虛體之神中而得其條理與真實,此即是道德創造之潤身踐形也。故云:「客感客形與無感無形,惟盡性者一也。」「一也」當為「一之也」,或直云「一之」。

故凡儒者之思參造化,言天道、言太極、言誠體、言太和、太虛,乃至寂感之神,皆不過是通澈宇宙之本源,清澈吾人之性體,

\newpage\thispagestyle{empty}\addtocounter{page}{-1}\vspace*{-12mm}\begin{center}\noindent
\includegraphics[clip, trim=148pt 147pt 145pt 236pt, height=162mm]{ocr-input/image-1916.png}\end{center}

\newpage

\noindent 以明道德創造潤身踐形所以可能之超越根據,而其實義皆落於「性」中見。亦由性體之主宰義、創生義而貞定之,決不是空頭擬議之詞,亦不是自然主義、唯氣論之由氣蒸發也。

〈乾稱篇第十七〉云:「大率天之為德,虛而善應。其應非思慮聰明可求,故謂之神。老氏況諸谷以此。」繼之云:「太虛者氣之體。氣有陰陽、屈伸、相感之無窮,故神之應也無窮。其散無數,故神之應也無數。雖無窮,其實湛然。雖無數,其實一而已。陰陽之氣,散則萬殊,人莫知其一也。合則混然,人不見其殊也。形聚為物,形潰反原:其游魂為變與?所謂變者對聚散存亡為文,非如螢雀之化,指前後身而為說也。」

案:此〈乾稱篇〉之文與此處所言者意義相類,亦可藉以助解。〈乾稱篇〉內容與規模與〈太和篇〉相似。吾詳為比讀,覺橫渠恐是先有〈乾稱篇〉,後復經消化,重新撰成(太和篇〉。〈太和篇〉在結構與措辭上,俱比較成熟,而亦甚純正,此顯是經過鍛煉與陶鑄而成者。〈乾稱篇〉恐是原初之稿,言之亦極精微,但結構之嚴整不及〈太和篇〉,而措辭多有老子之痕跡,至寫〈太和篇〉,則淘濾較淨,故較爲純正耳。但亦無甚大窒,故兩存之,而吾人亦可藉以互相發明也。

\subsection{「聚亦吾體,散亦吾體」:並論「兼體無累」義}

\begin{quotation}\kaishu 3.天地之氣,雖聚散攻取百塗,然其為理也,順而不妄。氣
之爲物,散入無形,適得吾體;聚為有象,不失吾常。太
虛不能無氣,氣不能不聚而為萬物,萬物不能不散而為太
虛。循是出入,是皆不得已而然也。然則聖人盡道其間,\end{quotation}

\newpage\thispagestyle{empty}\addtocounter{page}{-1}\vspace*{-12mm}\begin{center}\noindent
\includegraphics[clip, trim=170pt 134pt 121pt 243pt, height=162mm]{ocr-input/image-1920.png}\end{center}

\newpage\markright{第二部 \quad 分論一 \quad 第二章 \quad 張橫渠對於「天道性命相貫通」之展示}

\begin{quotation}\kaishu 兼體而不累者,存神其至矣。彼語寂滅者,往而不返。徇
生執有者,物而不化。二者雖有間矣,以言乎失道,則均
焉。聚亦吾體,散亦吾體。知死之不亡者,可與言性矣。\end{quotation}

\noindent 案:此承上第二段而言。太虛是氣之本體,而氣之聚散只是變化之客形,則氣之「散入無形,適得吾體,聚為有象,不失吾常」乃必然應有之至理。

太虛之為氣之體不是一個抽象的靜態之體,乃是遍運乎氣而妙應之之動態的具體的神用之體,故氣之或聚或散,或攻或取,雖其塗轍繁多,然皆有清通神用之體以妙運之,故皆「順而不妄」,非偶然也。「其爲理也,順而不妄」,言其聚之所以為聚,散之所以爲散,攻之所以爲攻,取之所以為取,皆有其形上地必然之道,故皆順適而不虛妄也。「為理」是此事之所以為此事之理,即「所以是如此」之意。此「理」字是虛說,其實處是通於太虛之神。有神體以妙運之,故事皆實事,非幻妄也。因此,當氣之「散入無形」,並非即歸於虛無,乃正恰因此而證得吾之清通之虛體(神體)。氣雖散,而虛體常在,故云「適得吾體」,此即下文所謂「死之不亡」也。當其「聚爲有象」,亦並非因其聚而即固結於象迹,而與虛體脫節。氣雖聚,而常體不失,故云「不失吾常」,此即下文第六段所謂「氣聚則離明得施而有形」也。「離明」即太虛常體之明也。

以上是「本體、宇宙論地」說。「本體、宇宙論的」實理實是如此。此是大中至正之道,聖人亦不過能盡此道而已。所謂盡此道者,即不偏滯于聚,亦不偏滯於散,而能貫通爲一以存神也。故

\newpage\thispagestyle{empty}\addtocounter{page}{-1}\vspace*{-12mm}\begin{center}\noindent
\includegraphics[clip, trim=159pt 144pt 154pt 252pt, height=162mm]{ocr-input/image-1924.png}\end{center}

\newpage

\noindent 云:「聖人盡道其間,兼體而不累者,存神其至矣。」此言聖人於氣之聚散之中而能盡道,以至於「兼體而不累者」,正因其能「存神」也,「存神」是其極至之工夫也。孟子曰:「君子所過者化,所存者神。」「兼體不累」亦是「所過者化」也。能存神,則不淪於虛(氣散爲虛)。「彼語寂滅者,往而不返」,是滯於散而淪於虛也。此是指佛教說。同時能存神,則亦不執於實(氣聚為實)。彼「徇生執有者,物而不化」,是滯於聚而執於實也。此是指道家說。(道家養生以期長生是「徇生執有、物而不化也」)。此兩者自有不同,然皆非大中至正之道,故云「以言乎失道,則均焉。」能兼體不累以存神,則「知死之不亡」。死非死也,乃是大往,入于幽也。死是大往,則生即是大來,由幽而明也。如是,則生死問題乃是幽明問題。「生吾順事,歿吾寧也。」死而不亡,則吾人之真實生命豈不真實常在而巍然與天地同壽者乎?知此,則「可與言性矣」,此義詳見〈誠明篇〉,詳解見下第二節第四段。

「兼體不累」,此中「兼體」一詞頗隱晦。須藉他處之文以助解。〈乾稱篇〉云:「體不偏滯,乃可謂無方無體。偏滯於晝夜陰陽者物也。若道,則兼體而無累也。以其兼體,故曰一陰一陽,又曰陰陽不測,又曰一闔一闢,又曰通乎晝夜。語其推行,故曰道。語其不測,故曰神。語其生生,故曰易。其實一物,指事異名爾。」此文可助解「兼體」之意。詳此,則「兼體」之兼即不偏滯義,「體」則無實義,非本體之體。兼體者即能兼合各相而不偏滯于一隅之謂。〈誠明篇第六〉有云:「天本參和不偏。」此「兼體」之兼即「參和不偏」之意也。所參和之體即晝夜、陰陽、動靜、聚散等之相體或事體,故此「體」字無實義,乃虛帶之詞。此

\newpage\thispagestyle{empty}\addtocounter{page}{-1}\vspace*{-12mm}\begin{center}\noindent
\includegraphics[clip, trim=168pt 122pt 119pt 253pt, height=162mm]{ocr-input/image-1928.png}\end{center}

\newpage\markright{第二部 \quad 分論一 \quad 第二章 \quad 張橫渠對於「天道性命相貫通」之展示}

\noindent 〈太和篇〉下文有云:「兩體者,虛實也,動靜也,聚散也,清濁也。其究一而已。」此剋就虛實、動靜、聚散、清濁之偶性而言「兩體」,亦可簡言之曰「兩」,故知此「體」字無實義也。能兼合(參加)各體(各事、各相、各形)而不偏滯於於一相,則即可不為相迹所累,此即不累於相迹。不累於相迹,則清通而虛體之神存矣。偏滯于一相,則「物而不化」,其究也亦「物」而已矣。故云:「偏滯於晝夜陰陽者物也」。此亦如濂溪云:「動而無靜,靜而無動,物也。動而無動,靜而無靜,神也」。「動而無靜」即偏滯於動,「靜而無動」即偏滯於靜。此即是物而不化。神則其自身動而無動、靜而無靜,圓應無方,妙運無跡,故能參和氣之動靜、聚散、虛實、有無,而不滯也。濂溪說動而無動,靜而無靜,是就誠體之神自身說,而橫渠言「兼體不累」則是就其參和相跡而不偏滯說。兩者義實相通也。有「動而無動、靜而無靜」之神體,故能有「兼體不累」、「參和不偏」之妙用也。「體不偏滯,乃可謂無方無體」,言於對偶之事體或形相無所偏滯,乃可謂「神無方而易無體」。「無方」者無方所,不為空間相所限。「無體」者無定體,不為動靜聚散所拘。此即神也、易也,亦即道也。故下文即繼之而言有偏滯者即是「物」,(「偏滯於晝夜陰陽者物也」),無偏滯而不累者即是道也。(「若道則兼體而無累也」)。因「兼體而無累」,故〈繫辭傳〉曰:「一陰一陽之謂道」,又曰:「陰陽不測之謂神」,又曰:「一闔一闢之謂變」,又曰:「通乎晝夜之道而知」。道、神、易,「其實一物,指事異名爾」。「一陰一陽之謂道」,非是說靜態地兼合了陰陽即是道,乃是說陰了又陽,陽了又陰,這樣動態地參和了陰陽而不偏滯於陰或陽,這纔見出道之

\newpage\thispagestyle{empty}\addtocounter{page}{-1}\vspace*{-12mm}\begin{center}\noindent
\includegraphics[clip, trim=166pt 144pt 131pt 240pt, height=162mm]{ocr-input/image-1932.png}\end{center}

\newpage

\noindent 妙用,即妙運乎陰陽以成此氣變也。陰而陽、陽而陰,氣變之不可測度即是神。若偏滯於陰或陽,則物而不化,有方所、有形體,非不可測度,而神亦不可見矣。故必須動態地參和陰陽以觀氣變,始可言不測,始可見神。此亦「兼體而無累」之意也。「一闔一關之謂變」,即動態地參和了闔闢而不偏滯於闔或闢,即是變也。此語引申即為「生生之謂易」。易,變易也。滯於一生,不名曰易。生而又生而不滯於一生,則易體見矣。易體即神體也,神體即道體也。非是生而又生之事跡本身為易、為神,乃由此生而又生而不偏滯於一生之事跡而見出易體或神體也。「之謂道」、「之謂神」、「之謂變」等語法,嚴格言之,皆非指事之界定語,乃是顯體之指點語。「通乎晝夜之道而知」而不偏滯於晝或夜,其知亦是兼體不累之神知也。後來伊川、朱子皆採取以下之方式表示道:陰陽非道,所以陰陽是道。此是從「所以」處表示,而橫渠則由「兼體而無累」表示。從「所以」表示較為更是形式的陳述,其直接所推證者,偏重「理」字義;而從「兼體無累」表示,則能直證神與虛,以神體虛體為道為易也,此則更易接近道之創生義、道之寂感真幾義、道之為心(天心、本心)義,而理自在其中也。此一表示方式之不同,亦啟對於道體體悟之分歧,亦是心理為一(心即理)為二(性即理)所由分之關鍵。

〈乾稱篇〉言:「若道,則兼體而無累也」,〈誠明篇〉言:「天本參和不偏」,此是客觀地言之。此〈太和篇〉「聖人盡道其間,兼體而不累者,存神其至矣」,則是主觀地從「盡道」處說。其義一也。「兼體無累」或「兼體不累」即「參和不偏」義。橫渠言「兼體」即本「參和」之「參」字說。「參」字是來自〈說卦

\newpage\thispagestyle{empty}\addtocounter{page}{-1}\vspace*{-12mm}\begin{center}\noindent
\includegraphics[clip, trim=156pt 124pt 128pt 249pt, height=162mm]{ocr-input/image-1936.png}\end{center}

\newpage\markright{第二部 \quad 分論一 \quad 第二章 \quad 張橫渠對於「天道性命相貫通」之展示}

\noindent 傳〉「參天兩地而倚數」之「參」。

〈参兩篇第二)云:

\begin{quotation}\kaishu 地所以兩,分剛柔男女而效之法也。天所以參,一太極兩儀
而象之性也。\end{quotation}

\noindent 〈說卦傳〉之語,朱子注云:

\begin{quotation}\kaishu 天圓地方。圓者一而圍三,三各一奇,故參天而為三。方者
一而圍四,四合二耦,故兩地而爲二。數皆倚此而起。故探
蓍,三變之末,其餘三奇,則三三而九;三耦則三二而六。
兩二一三則為七,兩三一二則為八。\end{quotation}

\noindent 九六七八之數倚「參天兩地」而起,而「參天兩地」亦可以說倚二、三之數而成其為參兩也,故云「參天兩地而倚數」。此是由數之三而說「參天」,以象徵天之圓,進一步象徵天德之「圓而神」;由數之二而說「兩地」,以象徵地之方,進一步象徵地德之「方以智」(剛柔有定體)。橫渠即由此「兩地」之二而說此「兩體」(如剛柔男女,推之,虛實、動靜清濁聚散等)之方德,由「參天」之三而說「兼體無累」之圓德。圓德屬於天、屬於道。參天之三即是一也,即由「三各一奇」以象徵天德圓神之一也。天德圓神純一而不可分,此是自天德之自體說。「兼體無累」、「參和不偏」則是自其用說,即由其「兼體無累」「參和不偏」之用以見其為「圓而神」也。橫渠即由「三各一奇」即是一之「參」字

\newpage\thispagestyle{empty}\addtocounter{page}{-1}\vspace*{-12mm}\begin{center}\noindent
\includegraphics[clip, trim=169pt 152pt 137pt 239pt, height=162mm]{ocr-input/image-1940.png}\end{center}

\newpage

\noindent 直接引申而為「參和不偏」之「參」字,再引申而為「兼體無累」之「兼」字,以明天德神體之圓一也。故曰「聖人盡道其間,兼體而不累者,存神其至矣」。以「參和不偏」、「兼體無累」以明天德神體之為圓為一。此天德神體之為圓為一亦即吾人之性體也。依是,「地所以兩,分剛柔男女而效之法也」,言地之所以兩者正為要分剛柔男女而呈現之以法相之定體也。「效之法」是根據〈繫辭傳〉「效法之謂坤」而來。朱注:「效,呈也。」即致呈、呈現義。朱注:「法,謂造化之詳密而可見者」,實即「定體」義,言剛柔男女之相皆有定體也。地以氣與質言,故有剛柔、男女,乃至虛實、動靜、清濁、聚散之兩體,此即其所呈現之定體也。天以德與神言,是圓是一,故云:「天所以參,一太極兩儀而象之性也。」言天所以參而「三各一奇」亦即是「圓而神」之一者正在其足以表示太極兩儀之統而為一而為一整體而足以象示之以性也。「象之性」乃由〈繫辭傳〉「成象之謂乾」一語而轉出。〈繫辭傳〉亦言「法象莫大乎天地」,即天「成象」、「成象之謂乾」也;地「效法」「效法之謂坤」也。又言「見乃謂之象,形乃謂之器」。「見」即天成象,示現以象也;「形」即地效法,剛柔男女有定體即形器也。又曰:「天垂象,見吉凶,聖人象之。」「垂象」即示現以象也,亦由天而言也。(繫辭傳〉言成象、垂象只是具體地指點地言之,橫渠即由之而較著實以言天之成象垂象,其所象而示者即是天德神體之為圓為一,亦即是性體之真實意義也。「象之性」者即由「太極兩儀之統而為一」而象示出性體之具體而真實的意義也,亦即由天之所以參而「三各一奇」亦即是「圓而神」之一而象示出也。「地所以兩」是言法相形器之定體,「天所

\newpage\thispagestyle{empty}\addtocounter{page}{-1}\vspace*{-12mm}\begin{center}\noindent
\includegraphics[clip, trim=157pt 129pt 140pt 251pt, height=162mm]{ocr-input/image-1944.png}\end{center}

\newpage\markright{第二部 \quad 分論一 \quad 第二章 \quad 張橫渠對於「天道性命相貫通」之展示}

\noindent 以參」是言萬事萬物圓一之性體。天德神體之為圓為一性體之為圓為一,即太極也。就太極說,太極不離兩儀,即太虛神體之不離氣,亦即「天本參和不偏」、「道則兼體無累」之義也,亦即〈誠明篇〉「性其總,合兩也」之義,〈乾稱篇〉「有無虛實通為一物者性也」之義。如此貫串觀之,橫渠思理固甚一貫,而亦甚清楚也。是故〈參兩篇〉繼「地所以兩」、「天所以參」一段又云:

\begin{quotation}\kaishu 一物兩體,氣也。一故神,(兩在故不測),兩故化,(推
行於一)。此天之所以參也。\end{quotation}

\noindent 此首句若分解言之,「一物」即太極、太虛神體之為圓為一,「兩體」即畫夜、陰陽、虛實、動靜等,此是屬於氣。而言「一物兩體氣也」是渾淪地言之,即「參和不偏」地言之,是表示太極太虛之不離氣,即由太極兩儀之統而為一以「即用見體」也,即氣之通貫以見天德神體之「參和不偏」、「兼體無累」也,並非說太極、太虛、天德神體亦是氣。故云「一故神」,此「一」即天德神體之「一」,而横渠自注云:「兩在故不測」,此即表示「一」之所以神正由於有「兩體」之存在而參和不偏、兼體無累,以成其生化之不測,而由此不測以見神體之妙用也。故繼之云:「兩故化」,此言正由於有兩體,故能生化也。而復自注云:「推行於一」,此言兩非死兩,正由於有兩體而能「推行於一」、參和不偏、兼體無累,始能成其化,即成其生化之不測也。「一故神」,由一必說到兩。「兩故化」,由兩必說到一。總之,是參和不偏、兼體無累,而即用之通以見體之實也。故結語云:「此天之所以參也。」此仍

\newpage\thispagestyle{empty}\addtocounter{page}{-1}\vspace*{-12mm}\begin{center}\noindent
\includegraphics[clip, trim=170pt 146pt 135pt 246pt, height=162mm]{ocr-input/image-1948.png}\end{center}

\newpage

\noindent 歸於「三各一奇」之以數目三言的天即是「圓而神」之圓一,故能參和不偏、兼體無累,而貫通氣之兩體以成生化之大用,以見天德神體(太極、太虛、道體)與夫性體之實也。此豈以道體性體為形而下之氣者乎?不離氣以見其實,非謂其本身即是氣、即是形而下者也。故〈大易篇第十四〉又云:

\begin{quotation}\kaishu 一物而兩體,其太極之謂歎?陰陽天道,象之成也。剛柔地
道,法之效也。仁義人道,性之立也。三才兩之,莫不有乾
坤之道。\end{quotation}

\noindent 此與「地所以兩」、「天所以參」一段無異指也。而此言「一物而兩體,其太極之謂欺」,此即從太極之「參和不偏」而提綱地說,與「一物兩體氣也」之偏重在即用見體說亦無異指也。「陰陽天道,象之成也」,即由陰陽以見天道之「參和不偏」、「兼體無累」也。「象之成」即「參和不偏」、「兼體無累」之象之成也。「剛柔地道,法之效也」,即由剛柔男女以見地道所呈現之法相形器之有定體也。「仁義人道,性之立也」,即「參和不偏」、「兼體無累」之性體由仁義以著之以立其實也。(易傳》只是具體地(漫畫式地)指點地言「成象」、言「效法」、言「參天兩地」、言「兼三才而兩之」,而橫渠直由參以言天德神體與性體之圓一,由兩以言形器之兩體以及其有定體,參兩通而一之,即是道體性體之「參和不偏」、「兼體無累」。其思理亦可謂深矣,而言之不易,然豈是形上形下不分者乎?其如此重言「兼體無累」與太虛神體,而朱子謂其將形而上說成形而下,不亦誤乎?橫渠不常言太

\newpage\thispagestyle{empty}\addtocounter{page}{-1}\vspace*{-12mm}\begin{center}\noindent
\includegraphics[clip, trim=156pt 125pt 137pt 251pt, height=162mm]{ocr-input/image-1952.png}\end{center}

\newpage\markright{第二部 \quad 分論一 \quad 第二章 \quad 張橫渠對於「天道性命相貫通」之展示}

\noindent 極,然天德神體、太虛神體之圓一即太極也。此豈非形而上者乎?只因朱子將道與性視為只是理,將心神視為氣,故不能契知誠體神體之實義耳。濂溪、橫渠、明道皆言誠體、神體、寂感真幾,豈皆形而下者乎?朱子謂「神化二字雖程子說得亦不甚分明,惟是橫渠推出來。」(《朱子語類》卷第九十八,〈張子之書一〉)橫渠之「推出來」並不是只推究「神、化」兩詞之字義分明,乃是能推究出神為體,為形而上,化為用,就氣言,為形而下。又其所說之程子實只是伊川,因伊川只言理氣,很少言神化,而明道則盛言神化、而又甚分明也。是則朱子於此雖推尊橫渠,而實未知其言神化之實也。

《朱子語類》卷第九十九,〈張子之書二〉,有一條云:

\begin{quotation}\kaishu 問:橫渠有「清、虛、一、大」之說,又要兼清濁虛實。

曰:渠初云「清、虛、一、大」,為伊川詰難,〔當作「為
明道詰難」】,乃云「清兼濁,虛兼實,一兼二,大兼
小」。渠本要說形而上,反成形而下。最是於此處不分明。
如〈參兩〉云,以參爲陽,兩爲陰,陽有太極,陰無太極。
他要強索精思,必得於己,而其差如此!

又問:橫渠云「太虛即氣」,乃是指理為虛,似非形而下。
曰:縱指理為虛,亦如何夾氣作一處?\end{quotation}

\noindent 案:此則差之太遠,對於橫渠所說之「參兩」義完全未解。「陽有太極,陰無太極」之語,對於參兩義誤會太甚。又,即使橫渠言「清、虛、一、大」,衡之《正蒙》,亦是就「天德神體」或「太

\newpage\thispagestyle{empty}\addtocounter{page}{-1}\vspace*{-12mm}\begin{center}\noindent
\includegraphics[clip, trim=182pt 163pt 132pt 235pt, height=162mm]{ocr-input/image-1956.png}\end{center}

\newpage

\noindent 虛神體」說。其言清濁虛實、剛柔、動靜,乃至陰陽、晝夜,乃是就氣之兩體說。「清兼濁、虛兼實、一兼二、大兼小」乃是太虛神體之清、虛、一、大兼氣之兩體方面之濁、實、二、小,非是以同一層次之氣之清者、虛者、一者、大者兼氣之濁者、實者、二者、小者。横渠之意甚明,何得混淆?朱子時,《正蒙》原文具在,何得必以二程之誤會為法而不究《正蒙》之實義乎?

「兼體」義明,則「兼體不累」之義不但所以顯體,且必函「體用圓融」之義,此則就道或就聖人「盡道」說皆然。「太虛不能無氣」,即太虛神體不能離氣而見也。神之所以為神,正因其參和氣之聚散而不偏滯,是即體之不離用,神之即氣而見也。而氣之所以聚散以生生亦正因神之妙運而使然,故其「順而不妄」,皆爲實事,皆是由於神理以成就之,此即用之不離體,氣之即神而然也。聖人盡道,潤身踐形,無往不是神體呈現,亦無往不是德業彌綸。此即體用之圓融。「彼語寂滅者,往而不返」,此指佛教而言。一心嚮往寂滅,不能返而成就聚散動靜之實事,則是有體無用也。此評自不能盡佛家大乘之菩薩道及圓教義,然宋、明儒之辨佛,要點不在此。縱亦往而返矣,然亦不是聖人盡道之圓融。要者是在儒者不贊同「緣起性空」下之空寂或寂滅。依儒者觀之,雖「一色一香無非中道」(智者語),極其圓融,亦仍是往而不返,此中本質上有不回頭處。故象山謂「儒為大中,釋為大偏」;「原其始,要其終,則私與利而已」;「儒者雖至於無聲無臭無方無體,皆主於經世。釋氏雖盡未來際普渡之,皆主於出世。」(《象山全集》卷二,(與王順伯〉書)。此中確有本質之差異,不可徒以大乘之菩薩道及圓教義為論也。橫渠(甚至全部宋、明儒)之

\newpage\thispagestyle{empty}\addtocounter{page}{-1}\vspace*{-12mm}\begin{center}\noindent
\includegraphics[clip, trim=150pt 140pt 152pt 244pt, height=162mm]{ocr-input/image-1960.png}\end{center}

\newpage\markright{第二部 \quad 分論一 \quad 第二章 \quad 張橫渠對於「天道性命相貫通」之展示}

\noindent 言,雖有不盡,然其彰顯儒佛本質之差異,以明釋氏非聖人之道,則固不謬也。此為宋、明儒共同之意識,不在其言之盡不盡也。故云:「彼語寂滅者,往而不返。徇生執有者,物而不化(此指道家養生之陷溺言)。二者雖有間矣,以言乎失道,則均焉」。皆非能盡道德創造之大中至正之道也。徇生執有者、偏滯於氣之聚,固不能兼體無累以存神,彼語寂滅者、夢幻人世,偏滯於空寂,亦不能兼體無累以存神。此皆喪失創生之道者也。

落於個體生命處說,能兼體無累以存神者,則知氣之聚而有形,固是吾神體之妙用,其散入無形亦是吾神體之妙用。故或聚或散,神體常在。有形之生只是客形,其聚其散乃神化之必然,不能滯窒以強留。要在兼體無累以存神,斯則得其常也。故云:「聚亦吾體,散亦吾體。知死之不亡者,可與言性矣。」聚亦得吾之體,散亦得吾之體,體、神體也。此即吾之「性」也。「盡道」即盡性也。故〈西銘〉云:「存吾順事,沒吾寧也」。君子有終無死;即形潰為死,而「形潰反原」,神體常在,焉有所謂亡者乎?「亡」者流逝不在之謂也。氣有存在與不存在,而神體則無所謂存在不存在也。此是儒者極深遠廣大極中正極莊嚴之成德之宗教,非一般偏曲之宗教也。儒者不言個體靈魂之不滅,而肯認此神體之常存。此神體是遍、是常、是一,此即是吾之性。吾之藐然之身,以「天地之帥」(神體)為性,以「天地之塞」(氣)為體(形體),是即直下肯認吾之生命乃一宇宙之生命,其有形之軀之或聚或散不過是天地之帥與天地之塞之大來大往而已。此一「吾之生命即宇宙之生命」之常在乃真正成德宗教之圓教。

\newpage\thispagestyle{empty}\addtocounter{page}{-1}\vspace*{-12mm}\begin{center}\noindent
\includegraphics[clip, trim=192pt 234pt 145pt 237pt, height=162mm]{ocr-input/image-1964.png}\end{center}

\newpage

\subsection{太虛卽氣:體用不二之圓融論並辨佛老之「體用」義}

\begin{quotation}\kaishu 4.知虛空即氣,則有無、隱顯、神化、性命,通一無二。顧
聚散、出入、形不形,能推本所從來,則深於《易》者
也。若謂虛能生氣,則虛無窮,氣有限,體用殊絕,入老
氏有生於無之論,不識所謂有無混一之常。若謂萬象為太
虛中所見之物,則物與虛不相資,形自形,性自性,形性
天人不相待,而有陷於浮屠以山河大地爲見病之說。此道
不明,正由懵者略知體虛空為性,不知本天道為用;反以
人見之小,因緣天地;明有不盡,則誣世界乾坤為幻化;
幽明不能舉其要,遂躐等妄意而然;不悟一陰一陽,範圍
天地,通乎畫夜,〔乃〕三極大中之矩,遂使儒佛、
老、莊混然一途。語天道性命者,不罔於恍惚夢幻,則定
以有生於無為窮高極微之論。入德之途,不知擇術而求,
多見其蔽於而陷於淫矣。\end{quotation}

\noindent 案:此第四段辨佛老—依據「兼體無累」以存神之體用圓融(通一無二)辨佛老體用關係之非是。體用圓融,圓者圓滿無遺,融者通一不隔。體用是一般之詞語,看其所應用而有不同之內容,如本體現象等。依橫渠之思理,體用圓融即是神體氣化之不即不離。「虛空即氣」即上段「太虛不能無氣」一語之義。「不能無氣」即不能離氣。不能離氣者即就氣化之不滯而見神體虛體之妙用也。清通之神即在氣化之不滯處見,即在氣之聚散動靜之貫通處見,此即「虛空即氣」也。但凡儒者之在宇宙論處以宇宙論之辭語(或類似

\newpage\thispagestyle{empty}\addtocounter{page}{-1}\vspace*{-12mm}\begin{center}\noindent
\includegraphics[clip, trim=156pt 155pt 152pt 231pt, height=162mm]{ocr-input/image-1968.png}\end{center}

\newpage\markright{第二部 \quad 分論一 \quad 第二章 \quad 張橫渠對於「天道性命相貫通」之展示}

\noindent 宇宙論之語調)說此義,不是以氣化之不滯、氣之聚散動靜之貫通為無待之首出,視為空頭的自然事實之如此,乃是提起來視氣化過程為天道創生之過程,而天道創生之過程即是仁體創生感潤(感潤)之過程,或神體妙運之過程,而以道德創造之性體因果、心體因果,或用康德之詞語——意志因果而貞定其真實意義而使之如此者。簡言之,其言氣化之不滯是以性體因果為條件者,是預定道德創造之性體因果為其超越根據者。性體因果過程意即此因果過程乃為性體自主之所貫。氣化之不滯不是自然成之事實,而是以性體因果為根據,則在此氣化之不滯中自然有神體虛體(性體)以貫之,因而亦可說即在此氣化之不滯處見神體。是故「虛空即氣」此種神體氣化之宇宙論的圓融辭語是道德理想主義的圓融辭語,不是自然主義唯氣論之實然的陳述。必須念念提醒此義,於儒者言天道性命之宇宙情懷方可不至有誤解。

「虛空即氣」,順橫渠之詞語,當言虛體即氣,或清通之神即氣。言「虛空」者,乃是想以一詞順通佛老而辨別之也。虛體即氣,即「全體是用」之義,(整個虛體全部是用),亦即「就用言,體在用」之義。可言虛體即氣,亦可言氣即虛體。氣即虛體,即「全用是體」之義,亦即「就體言,用在體」之義。是以此「即」字是圓融之「即」、不離之「即」、「通一無二」之「即」,非等同之卽,亦非謂詞之即。顯然神體不等同於氣。就「不等同」言,亦言神不即是氣。此「不即」乃「不等」義。顯然神亦非氣之謂詞(質性)。若如朱子所解,神屬於氣,心是氣之靈處,則神成為氣之謂詞,心成為氣之質性,此即成為實然陳述,非體用圓融義。是以「即」有二義:()「不即」,此乃不等義,亦表

\newpage\thispagestyle{empty}\addtocounter{page}{-1}\vspace*{-12mm}\begin{center}\noindent
\includegraphics[clip, trim=170pt 162pt 123pt 226pt, height=162mm]{ocr-input/image-1972.png}\end{center}

\newpage

\noindent 示非謂詞之質性義;(「即」,此表示圓融義、不離義、通一無二義。(劉蕺山為表示通一無二,反對朱子歧理氣為二,而有曰:「天地之間,一氣而已。非有理而後有氣,乃氣立而理因之寓也」。此種抑揚之間,不能無病。若就橫渠之圓融義言,亦是有理(神體)而後有氣,亦是氣立而理因之寓;但「氣立」並非寡頭者。若氣真是無本而自立,天地之間真是自然成之一氣,不待妙運之者而然,則「氣立而理因之寓」,理很可能成為氣之謂詞、質性,此則大悖。劉籤山恐亦不至如此也。是以其語不能無病。詳見濂溪章第二節第四段。)

老子言無,釋氏言空,橫渠於此提出「虛」字為準以衡量佛老之空與無。「虛」字比較具體,不像空、無之專門化與概念化。虛字是中國人通常習用之字,是地道代表中國人之心靈。虛雖然不滯,然而卻是表詮字,不像空、無之純為遮詮字——空由遮緣起法之自性而顯,無由遮造作有為而顯,而虛卻是那麼坦蕩悠然而從容,純是化境之詞,故橫渠即以清通之神說虛。虛則神、虛則妙、虛則靈、虛則化、虛則純一不雜,而亦不滯於一。游於多而不滯於多即為虛。故虛體即氣,氣即虛體,「則有無、隱顯、神化、性命通一無二」。氣之聚為有為顯,氣之散為無為隱。有無隱顯兼體而無累、清通而不滯謂為神。氣之「推行有漸為化」,「合一不測為神」(〈神化篇第四〉)。神體即性體,而性體之流行(主宰之用)即為命。是故皆「通一無二」也。

「若謂虛能生氣,則虛無窮,氣有限,體用殊絕,入老氏有生於無之論,不識所謂有無混一之常。」此批評老氏也。案此評不必諦。歸結雖在說老氏「有生於無」之非,而實旨在明「虛空即氣」

\newpage\thispagestyle{empty}\addtocounter{page}{-1}\vspace*{-12mm}\begin{center}\noindent
\includegraphics[clip, trim=161pt 154pt 154pt 243pt, height=162mm]{ocr-input/image-1976.png}\end{center}

\newpage\markright{第二部 \quad 分論一 \quad 第二章 \quad 張橫渠對於「天道性命相貫通」之展示}

\noindent 虛不生氣,故云:「若謂虛能生氣」云云也。實則此種遮撥正是伊川所謂「意屢偏而言多窒」之一例也。「天地之道可一言而盡,其為物不貳,則其生物不測」,何以不可言「虛能生氣」耶?「生」者妙運、妙應之義。以清通之神、無累之虛妙運乎氣而使其生生不息,使其動靜聚散不滯,此即是生也。仁體之感潤而萬物生長不息,此即是生也。〈天道篇第三〉云:「天道四時行,百物生,無非至教。聖人之動無非至德。夫何言哉?天體物不遺,猶仁體事無不在也。禮儀三百,威儀三千,無一物而非仁也。昊天曰明,及爾出王〔往】,昊天曰旦,及爾游衍,無一物之不體也。天之「體物不遺」,仁之「體事無不在」,豈只是靜態地擺在那裡以為其體而已耶?禮儀三百,威儀三千,皆仁心生,皆仁體貫,如是始能說「無一物而非仁」。「昊天曰明」,遍照一切,遍臨一切,人而或出往,或游衍,亦皆在其照臨之中,因而得以戒慎不墮,人道不廢。「無一物之不體」,實即無一物之不因之而生也。此(大雅板〉詩雖只言昊天鑑臨在上,似無能生之意,然橫渠引之而言「無一物之不體」,實亦意許無一物之不生。生者實現義,「使然者然」義,故天道、仁體,乃至虛體、神體皆實現原理也,皆使存在者得以有存在之理也。生者引發義滋生義。因天道之誠、仁體之潤、虛體之清通、神體之妙應而滋生引發之也。天道、仁體、虛體、神體,豈不起作用耶?是故體之即是起之。孟子言生色醉面盎背,即是生也。其所起者,落於氣上,雖步步有限,散殊有定,然起而不已,通而不滯,則亦同其「無窮」矣。豈但是「虛無窮」而「氣有限」耶?是以縱貫言之,則「虛能生氣」;橫鋪言之,則體用相卽。橫渠於此只著重「虛空即氣」之相卽,此只知其靜態之橫

\newpage\thispagestyle{empty}\addtocounter{page}{-1}\vspace*{-12mm}\begin{center}\noindent
\includegraphics[clip, trim=161pt 162pt 143pt 230pt, height=162mm]{ocr-input/image-1980.png}\end{center}

\newpage

\noindent 鋪,而忘其動態縱貫之創生義也。在動態縱貫之創生中,非如母之生子,子生而離母體,而子母異體也。神體無限而遍在,永與其所生所起者冥合為一也。是以即在縱貫中,亦是全神是氣,全氣是神也。豈是以無限之神與一步之氣對言耶?若如此,則無限之神有蹈空之處,焉能遍在耶?此誠「體用殊絕」矣!神體遍在,非抽象掛空地遍在,乃具體圓融地遍在。是以「虛無窮」,而氣之生亦必同其無窮也。兩皆飽滿而亦不礙虛之有生起之用也。

至於老子「有生於無」亦不是「體用殊絕」。道家體用之意義自不同於儒家。老子言:「天下萬物生於有,有生於無。」(〈四十章〉)「無名天地之始,有名萬物之母。」(〈一章〉)「道生之,德畜之。」(〈五十一章〉)又言:「生之,畜之。生而不有,爲而不恃,長而不宰,是謂玄德。」(〈十章〉)又言「道生一,一生二,二生三,三生萬物。」(〈四十二章〉)其言「生」首先是肯認道之為「本」義,言天地萬物以道或無為本、為根據。是則「生於」或「生」首先是言詮上義理地「出自」義或「推至」義。即,以道為本、為根據,義理地「出自」道或由道義理地「推至」某某,此皆是形式語。然則老子言如許之「生」究有否形而上的實際意義?有否實際的作用?曰自然有。但此形而上的實際意義、實際的作用卻有特別的意義。道生、德畜,亦可以說天地萬物是實際存在地由無而生出。由無而生出實即由無而開出。但此實際存在地由無而生出或開出究是何種意義或形態,實有確切規定之必要,蓋此確有特殊之意義。

老子之宇宙論地言「無」為天地萬物之始、之本,道顯似有客觀性、實體性及實現性。然此三性,說穿了,只是一種姿態,實並

\newpage\thispagestyle{empty}\addtocounter{page}{-1}\vspace*{-12mm}\begin{center}\noindent
\includegraphics[clip, trim=161pt 134pt 126pt 240pt, height=162mm]{ocr-input/image-1984.png}\end{center}

\newpage\markright{第二部 \quad 分論一 \quad 第二章 \quad 張橫渠對於「天道性命相貫通」之展示}

\noindent 無一正面之實體性的東西曰「無」而可以客觀存在地(存有論地)生天地萬物,而天地萬物亦存有論地實際存在地由無而生出也。蓋「無」是一遮詮字,由否定人為的造作有為而顯,其原初之義仍是由生活上而體驗出。道家蓋對於人為造作之苦確有實感,故遮此有為,即顯無為;遮此造作,即顯自然。故「無」一遮詞所顯示之正面意義只是「自然」,而「自然」乃是一種境界,無實物可指,不可說不可說,非名之所能定,非稱之所能謂。故王弼云:「自然者,無稱之言、窮極之辭也」(〈二十五章〉,「道法自然」註。)故道、無之客觀性、實體性只是一種姿態,乃由「本」義、「根據」義而顯示,而實則可消化於主體之自在、自然、自適、自得而為一種境界。故道家之形上學乃澈底「境界形態」之形上學,非「實有形態」之形上學。

客觀性、實體性既如此,則道之實現性亦可得而定矣。道之實現性由「生」字而引出,本亦可說創生性或生化性。但此兩詞用於儒家為恰當,而用於道家則嫌太重太烈,亦即不恰當。故只用一般意義的「實現性」以說之。道當然有實現性之意義而可以為實現原理。實現原理即是「使然者然」之理。老子云:「自古及今,其名不去,以閱衆甫。吾何以知衆甫之狀哉?以此。」(〈二十一章〉)又云:「天得一以清,地得一以寧,神得一以靈,谷得一以盈,萬物得一以生,侯王得一以爲天下貞:其致之一也。」(〈三十九章〉)此皆表示道為「使然者然」之理。說道生之,不如說道「使之然」。「使然者然」即「使如此者成其為如此」。但是這「使然者然」(生之)卻是境界形態者,而非實有形態者。王弼註第十章之「生之」云:「不塞其源也」;注「畜之」云:「不禁其

\newpage\thispagestyle{empty}\addtocounter{page}{-1}\vspace*{-12mm}\begin{center}\noindent
\includegraphics[clip, trim=161pt 166pt 152pt 230pt, height=162mm]{ocr-input/image-1988.png}\end{center}

\newpage

\noindent 性也」;注「生而不有」云:「不塞其源,則物自生,何功之有?」注「為而不恃」云:「不禁其性,則物自濟,何為之恃?」然則所謂「道生之」,所謂以無為本,實非道或無能存有論地生之也,乃是通過無為無執一種無的境界,讓開一步,不塞物之自生之源,不禁物之自濟之性,物自能生自能濟也。是則仍是物之自生、物之自濟。惟須讓開一步,不塞其源、不禁其性,以讓其自生自濟。「不塞其源」是遮造作、干涉、騷擾、亂動手腳之窒塞其生命;「不禁其性」是遮矯揉、臆計、把持、桎梏之拘禁其性(戕賊其性)。絕大工夫是在此「遮撥」上作,而由此以顯道與無。你能如此無了,則開其源、暢其流,而物自生。此即是以無為本而道生之也,亦即是「使然者然」也。此是不生之生,不著之生,境界形態之生,而天地萬物亦確是實際存在地由此種無之境界、讓開之襟懷而生出(開出)也。所謂由此種「無」之境界、讓開之襟懷而生出、而開出,意即:由於此,萬物始能暢其流、遂其性而自成其生也。此完全是消極意義之生。而就道言,此卻亦正是道之形而上的實際意義實際的作用。依道家言,此正是莫大之作用、莫大之智慧,故云「無為而無不為」,又云「無為而治」也。此即是「天下萬物生於有,有生於無」之意也。「生於有」是在「有」中呈現其實際之生長,「有生於無」是在無之境界(不塞不禁)中各暢其流、各成其為有。是故道之實現性是讓開不著之境界形態下之實現
.性。「道生之」是境界形態下不塞不禁之「不生之生」而成其自生,成其自生即是開出其自生之道,是亦即生之也——不著之生。而萬物之實際存在地由此不塞不禁之無而開出亦是因由於此無始能暢其生之流而有存在也。否則桎梏而死,焉得能生能在?對死言,

\newpage\thispagestyle{empty}\addtocounter{page}{-1}\vspace*{-12mm}\begin{center}\noindent
\includegraphics[clip, trim=130pt 138pt 122pt 239pt, height=162mm]{ocr-input/image-1992.png}\end{center}

\newpage\markright{第二部 \quad 分論一 \quad 第二章 \quad 張橫渠對於「天道性命相貫通」之展示}

\noindent 而能使之有存在(使然者然),亦是「生之」也。如果此亦是一種「存有論」,則亦是境界形態之存有論,主觀作用之存有論,而非是實有形態之存有論,非是客觀實體之存有論。此是道家之體用義。

此種體用義自與儒家不同。儒家的體用義是道德的創造實體之體用,是康德所說意志因果性(是一種特別的因果性,與自然因果性不同)之體用,是性體因果性、心體因果性之決定方向之創生的體用。故此創造實體確有能生義、生起義、引發義、感潤義、妙運義。此創造實體之客觀性、實體性、實現性(創生性、生化性)不只是一種姿態,而確是一種客觀的實體、實有之所具。惟此實體實有不是柏拉圖型的,不是智及之靜態的形式,乃是意志、德性之動態的性體、心體、虛體、神體、誠體乃至天道、天命以及太極。此創造的實體亦實有亦神用(活動),亦主觀亦客觀,乃是超然之大主。此種形上學名曰道德的形上學。如果此中亦含有一種宇宙論,乃是道德創造之宇宙論。如果亦含有一種存有論,乃是創造實體之存有論,實有形態之存有論,不只是境界形態也。儒道之別只應如此看。橫渠所謂「若謂虛能生氣,則虛無窮,氣有限,體用殊絕,入老氏有生於無之論,不識所謂有無混一之常」,此皆不諦之批評。虛氣圓融,虛亦生氣。不因「虛能生氣」,即「體用殊絕」也。老氏之有無,乃至「有生於無」,則是另一系統。他亦可以說「有生於無」,他亦可以說「有無混一」,所謂「此兩者同出而異名,同謂之玄」也。(關於道家之玄,其詳請參看《才性與玄理》:〈王弼玄理之易學〉、〈王弼之老學〉,以及〈向、郭之註莊〉三章。)

\newpage\thispagestyle{empty}\addtocounter{page}{-1}\vspace*{-12mm}\begin{center}\noindent
\includegraphics[clip, trim=166pt 167pt 153pt 241pt, height=162mm]{ocr-input/image-1996.png}\end{center}

\newpage

以上是就橫渠之批評老子說。

至於「若謂萬象為太虛中所見之物,則物與虛不相資,形自形,性自性,形性天人不相待,而有陷於浮屠以山河大地為見病之說」,此是辨佛也。橫渠所說之虛與氣乃至性與形,天與人,本不同於佛家「無自性」之為空與緣起法之為有之空有,而佛家之空有本亦不可以體用論。惟若以佛家之空有說,橫渠謂其「不相資」、「不相待」,亦確有其諦處。但有時亦可說相資相待。然則其相資相待究係何意?其究相資相待否耶?此則須詳為考論,非一二言所能盡。為免太滋蔓隔斷文氣故,乃為文專論之。請別看附錄:〈佛家體用義之衡定〉。

以上橫渠根據「虛空即氣」以評佛老,雖略而不盡,亦多不諦(批評佛家則大體是諦),然正面表儒家義則不誤也。茲仍順其正面義以言。

\subsection{}

\begin{quotation}\kaishu 5.氣块然太虛,升降飛揚,未嘗止息。《易》所謂細縕,莊
生所謂生物以息相吹野馬者與?此虛實動靜之機,陰陽剛
柔之始。浮而上者陽之清,降而下者陰之濁。其感遇聚
散,為風雨,為霜雪,萬品之流形,山川之融結,糟粕煨
爐,無非教也。\end{quotation}

\noindent 案:此段大體不出首段之意。惟「無非教也」句,則表示一新意。雖講虛講神,而虛與神不離氣,故仍就氣說。「氣块然太虛,升降飛揚,未嘗止息」三句,即就氣而總說因虛與神故能有生化之大用

\newpage\thispagestyle{empty}\addtocounter{page}{-1}\vspace*{-12mm}\begin{center}\noindent
\includegraphics[clip, trim=170pt 140pt 123pt 241pt, height=162mm]{ocr-input/image-2000.png}\end{center}

\newpage\markright{第二部 \quad 分論一 \quad 第二章 \quad 張橫渠對於「天道性命相貫通」之展示}

\noindent 也。「块」,說文謂:「霧昧塵埃也」。狀氣細盛大之象,亦表示雲蒸霞蔚、充實飽滿之象。雖块然而實至虛,虛則神矣。虛而神,故能「升降飛揚,未嘗止息」也。此块然蓊鬱之盛大飽滿,向裡看一步,即是所謂「太和」。「《易》所謂絪」,即表示太和。莊生所謂「野馬也,塵埃也,生物之以息相吹也。」亦表示太和絪縕,故能以息相吹也。因以息相吹,故能雲蒸霞蔚而有蓊鬱之氣。故首段云:「不如野馬絪,不足謂之太和。」「此〔乃】虛實動靜之機,陰陽剛柔之始」,(「此」下可補一「乃」字),即首段首長句之意也。「其感遇聚散,為風雨,為霜雪」,直至「無非教也」,乃首段「其來也幾微易簡,其究也廣大堅固」之意之轉換表示,由之以引生「無非教也」之新意。《禮記·孔子閒居》篇云:「天有四時,春秋冬夏。風雨霜露,無非教也。地載神氣,神氣風霆;風霆流形,庶物露生,無非教也。」橫渠言此,顯本《禮記》此段文而說,太和細即是道之生物不測。而「風雨霜露」「風霆流形」,亦無非虛體、神體之顯現。故充實飽滿之宇宙無處不是實理實事,即無處不是教訓也。無處不是教訓,則道即在眼前。道不遠人,道不離器。「風霆流形,庶物露生」即是道。說虛說神,此即是虛,此即是神。體用不二、充實圓盈之教,乃中國超越亦內在、最具體、最深遠、最圓融最真實之智慧之所在,乃自古而已然,此儒家所本有。明道喜說此義,橫渠亦發此義。此豈是來自禪耶?禪家「挑水砍柴,無非妙道」,以及「作用是性」諸義,亦不過是此智慧之表現於佛教耳。世俗鄙陋,推之於禪,此數典忘祖也。

\newpage\thispagestyle{empty}\addtocounter{page}{-1}\vspace*{-12mm}\begin{center}\noindent
\includegraphics[clip, trim=176pt 213pt 154pt 254pt, height=162mm]{ocr-input/image-2004.png}\end{center}

\newpage

\subsection{}

\begin{quotation}\kaishu 6.氣聚,則離明得施而有形。氣不聚,則離明不得施而無
形。方其聚也,安得不謂之客?方其散也,安得遽謂之
無?故聖人仰觀俯察,但云知幽明之故,不云知有無之
故。盈天地之間者法象而已。文理之察,非離不相睹也。
方其形也,有以知幽之因;方其不形也,有以知明之故。\end{quotation}

\noindent 案:前第二段「太虛無形,氣之本體。其聚其散,變化之客形爾」。此段則承之由虛體神體而言離明。〈神化篇〉云:「虛明照鑑,神之明也」。此「離明」之詞即剋就神體之虛明照鑑而言也。「離」即坎離之離。於卦,坎為水,離為火。火即光明之象徵。「離明」為同意之複疊詞。〈說卦〉亦言「離為目」。火與目皆取象取喻之意。而此言離明不實指火言,亦不實指目言,乃直指神體之虛明照鑑而言也。〈神化篇〉復續「神之明也」而言:「無遠近幽深、利用出入,神之充塞無間也。」神之充塞無間即明之充塞無間。此言離明是「本體、宇宙論地」言之(onto-cosmologically)。此是言「心」之「本體、宇宙論的」根據,而此神體之明亦可以說即是「宇宙心」也。

虛體神體妙運一切、充塞無間,即是明之照鑑一切、充塞無間。但通過氣之聚散而有隱顯耳。前第三段云:「氣之為物:散入無形,適得無體;聚為有象,不失吾常。」此言虛體神體之隱顯也。雖有隱顯,而虛體常存。「散入無形,適得吾體」,言此時正恰好得吾虛體神體之自存也。「聚為有象,不失吾常」,言此時雖

\newpage\thispagestyle{empty}\addtocounter{page}{-1}\vspace*{-12mm}\begin{center}\noindent
\includegraphics[clip, trim=163pt 132pt 128pt 244pt, height=162mm]{ocr-input/image-2008.png}\end{center}

\newpage\markright{第二部 \quad 分論一 \quad 第二章 \quad 張橫渠對於「天道性命相貫通」之展示}

\noindent 聚而有象,而虛體神體即寓而主宰其中,而此即吾之常體也。散不淪於空無,聚不滯於形象。不滯於形象,則吾因虛體神體之主其中而得吾之常度常則以貞定吾人之生命,此即「不失吾常」也。就虛體神體說,是如此,就「神之明」說,亦是如此。

「氣聚,則離明得施而有形。」施是施布施展之施,此是「本體、宇宙論地」施,非認識論地施,是直貫地施,非橫列地施。故此句以及下句皆是「本體、宇宙論的」辭語,非認識論的辭語。「離明得施而有形」即因氣之聚而顯也。「離明不得施而無形」即因氣不聚而隱也。然無論或隱或顯,神體之明固常存而自在也。隱顯就神之明言。聚散就氣言。故「方其聚也,安得不謂之客?」客即「客形」之客。氣之聚而有形是氣化之客形,而神之明無所謂客也,乃是常存之大主,不過因氣之聚而有具體的顯現耳。「方其散也,安得遽謂之無?」氣散而無形,只是形無,非任何都無也。而神體之明仍自在耳。只不過因氣之散而無附形之著顯耳,此亦可以說是隱或幽。故《易傳》「但云知幽明之故,不云知有無之故。」幽之未來即是明,故明之故即是過去未形之幽也。明之未來即是幽,故幽之因即是過去已形之明也。故横渠云:「方其形也,有以知幽之因〔幽之因即是眼前已形之明〕。方其不形也,有以知明之故〔明之故即是不形之幽〕。」總之,幽之因為明,明之故為幽也。橫渠辭語是就眼前之幽(不形)或明(形)以說未來明或幽之故。若就眼前之幽或明以追溯前此之明或幽為其故,亦得。總是幽之因為明,明之故為幽也。幽明之故顯是「本體、宇宙論的」辭語。

至于「盈天地之間者,法象而已。文理之察,非離不相睹也」

\newpage\thispagestyle{empty}\addtocounter{page}{-1}\vspace*{-12mm}\begin{center}\noindent
\includegraphics[clip, trim=159pt 164pt 143pt 227pt, height=162mm]{ocr-input/image-2012.png}\end{center}

\newpage

\noindent 兩語亦是「本體、宇宙論的」辭語。《易傳〉云:「成象之謂乾,效法之謂坤。」橫渠言法象本之此,惟由乾坤轉為就陰陽之氣而混言之耳。有法象即有文理,法象之文理皆離明之所呈現也。離明之所呈現,亦因離明而得相睹。故云:「文理之察,非離不相睹也。」「相睹」即因離明之遍在遍照而得相契接也。此處雖有察字、睹字,然其根據卻是「本體、宇宙論的」神明之充塞無間。察、睹是認識論的詞語,而其根據卻是「本體、宇宙論的」陳述。故不得因此察字睹字,便視「離明得施、不得施」兩句為認識論的辭語。凡此整段皆是「本體、宇宙論的」陳述,非認識論的陳述。

《朱子語類〉卷第九十九:「問:氣聚則離明得施而有形,氣不聚則離明不得施而無形。離明何謂也?曰:此說似難曉。有作日光說,有作目說。看來只是氣聚,則目得而見。不聚,則不得而見。〈易〉所謂離為目是也。」案此解非是。朱子解離為目,解「離明得施」為「目得而見」,「不得施」為「不得而見」,此正是視作認識論之辭語,全非。朱子對此段完全不相應。彼云「此說似難曉」,遂亦終於未曉也。本甚顯豁之義理,朱子何以如此隔閡?此非其滯笨也,必有其故矣。《朱子語類》卷第九十八及九十九兩卷皆討論橫渠者,其言大抵皆不相干,其於橫渠之不解可知。其故見末節總解。

\subsection{太虛即氣之體用圓融義與清氣之質性之分別}

\begin{quotation}\kaishu 7.氣之聚散於太虛,猶冰凝釋於水。知太虛即氣,則無無。
故聖人語性與天道之極,盡於參伍之神變易而已。諸子淺
妄,有有無之分,非窮理之學也。\end{quotation}

\newpage\thispagestyle{empty}\addtocounter{page}{-1}\vspace*{-12mm}\begin{center}\noindent
\includegraphics[clip, trim=170pt 145pt 143pt 247pt, height=162mm]{ocr-input/image-2016.png}\end{center}

\newpage\markright{第二部 \quad 分論一 \quad 第二章 \quad 張橫渠對於「天道性命相貫通」之展示}

\begin{quotation}\kaishu 太虛爲清,清則無礙,無礙故神。反清為濁,濁則礙,礙
則形。

凡氣清則通,昏則壅。清極則神。故聚而有間,則風行而
聲聞具達,清之驗與?不行而至,通之極與?

由太虛有天之名,由氣化有道之名,合虛與氣有性之名,
合性與知覺有心之名。\end{quotation}

\noindent 案:此四小段為一整段。第四小段「由太虛有天之名」等四句,將移下分兩節專講:一節將藉之吸收〈誠明篇〉以明性,一節將藉之吸收〈大心篇〉以明心。餘三小段為一氣,似皆綜述「太虛即氣」之義,故束於一起而綜解之。惟第三小段不能無窒礙,茲分別疏之如下。

橫渠于〈太和篇〉一則云:「散殊而可象為氣,清通而不可象為神。」再則云:「太虛無形,氣之本體。」復云:「知虛空即氣,則有無、隱顯、神化、性命通一無二。」又云:「知太虛即氣,則無無。」凡此皆明虛不離氣,卽氣見神。此本是體用不二之論,超越亦內在之圓融之論。然圓融之極,常不能令人無滯窒之誤解,而橫渠之措辭亦常不能無令人生誤解之滯辭。當時有二程之誤解,稍後有朱子之起誤解,而近人誤解為唯氣論。然細會其意,並衡諸儒家天道性命之至論,橫渠決非唯氣論,亦非誤以形而下為形而上者。誤解自是誤解,故須善會以定之也。

吾於前文第四段解「虛空即氣」時,即已明在此體用不二之義下,「即」字非等義,虛與神非是氣之謂詞(predicates),非是氣之質性(properties),「虛空即氣」非是「實然之陳述語」

\newpage\thispagestyle{empty}\addtocounter{page}{-1}\vspace*{-12mm}\begin{center}\noindent
\includegraphics[clip, trim=172pt 158pt 151pt 240pt, height=162mm]{ocr-input/image-2020.png}\end{center}

\newpage

\noindent (factual statement ) 非是「指謂語」 (predicativeproposition),乃是形而上的抒意語,指點語,乃是在體用不二下辯證的相消相融語。虛與神雖不是一隔離的獨立物(independententity),但卻是一獨立的意義(an independent meaning)。指點一個獨立的意義以為體,故曰:「太虛無形,氣之本體。」體是本體之體,不是物體之體。不能當作一個獨立的物體看,但卻可以當作有獨立意義的本體看。本體之體本可以不離其用也,是以有相融相即、不離不二之論。復次,此體是一、是全、是遍。若視為氣之實然的質性,則囿於散殊限定之氣而亦為有限量,有限有定之強度量,此乃氣之強度所蒸發之光彩,有時而盡,有時而消逝矣,此則即不能是一、是全、是遍。此則仍屬於氣之觀念、材質之觀念(material),而不能說是神。儒家說神非人格神之意義,不通過獨立物獨立體之觀念去了解,乃通過「作用」的觀念去了解。但是這作用卻是無限的妙用,是全、是一、是遍。故《易傳》日:「神也者妙萬物而為言」,又曰:「陰陽不測之謂神。」前句是就其「妙萬物」而謂之為本體、為實體,後句是即就陰陽之化之不測而見其為體。又曰:「寂然不動,感而遂通天下之故,非天下之至神,其孰能與於此?」若是氣之質性,則不能「寂然不動,感而遂通天下之故」矣。即使有所感通,亦是有限定、有範圍,此即不是遍。又曰:「唯神也,故不疾而速,不行而至」。若是氣之質性,則不能「不疾而速,不行而至」矣。還是因疾而速,因行而至,有速度,有過程。神則不可以速度論,不可以過程論,此即是全、是一、是遍。又曰:「神無方而易無體」。若是氣之質性,則不能無方矣。唯無方無體之神方可說是至虛之體。但不是隔離的獨立物

\newpage\thispagestyle{empty}\addtocounter{page}{-1}\vspace*{-12mm}\begin{center}\noindent
\includegraphics[clip, trim=176pt 138pt 136pt 249pt, height=162mm]{ocr-input/image-2024.png}\end{center}

\newpage\markright{第二部 \quad 分論一 \quad 第二章 \quad 張橫渠對於「天道性命相貫通」之展示}

\noindent 體,而卻是即由其妙萬物,萬物因之而生生不息、生化不測,而見其為神、而見其為體,此即所謂虛不離氣,即氣見神,體用不二之圓融之論也。此義必須有以善會而確認之,既不可離,亦不可滯。離則為一獨立物,體用不圓矣。滯則成為氣之質性,則成唯氣論(唯物論)矣。此神義之最後貞定與極成是在超越的道德本心之挺立。先秦儒家《中庸》《易傳》之境本是由孔子之仁與孟子之心性而發展至者。宋儒自濂溪、橫渠開始,雖直接承先秦儒家發展至之最高峰,由《中庸》、《易傳》說起,然其講天道性命實無不自覺或不自覺地以《論》《孟》之道德心性為其所共許或所默認之底據也。超越的道德本心顯然不是心理學的心。道德的本心雖不是一獨立物,然卻是一獨立的意義而為吾人道德實踐之先天根據,為吾人道德生命之本體也。此作為本體之本心決非氣之質性明矣。心理學的心是氣,而此道德的本心決不可視作氣也。在「本體、宇宙論」處,虛與氣之體用不二,亦復如此。

依以上所說,則此處第一小段中「知太虛即氣,則無無」句,完全同於第四段「知虛空即氣,則有無、隱顯、神化、性命通一無二」之意。「無無」即無所謂「無」,「無」只是氣之散而無形。形無非神無也。神體遍、常、一,氣之聚散只是變化之客形。惟神也,故氣不一于聚,乃聚而散、散而聚,而成其化。惟化也,故見神。故雖虛不離氣,即氣見神,而神之超然遍、常、一而為體也則甚明。故「氣之聚散於太虛,猶冰凝釋於水」。水體遍、常、一,冰之凝固與融化只是其客形爾。由水之遍常一見虛體,由冰之凝釋見氣化。此喻乃常用,有其恰當處;然亦只是一喻耳,不可執喻失義。如此言虛與氣之體用不二乃所以表示儒家言性命天道有其「本

\newpage\thispagestyle{empty}\addtocounter{page}{-1}\vspace*{-12mm}\begin{center}\noindent
\includegraphics[clip, trim=172pt 160pt 148pt 241pt, height=162mm]{ocr-input/image-2028.png}\end{center}

\newpage

\noindent 體、宇宙論之創生」上之充實圓融之飽滿,乃圓盈之教,非同佛、老之偏枯。「故聖人語性與天道之極,盡於參伍之神,變易而.已。」「參伍之神」即「陰陽不測之謂神」。「變易」即「生生不易之謂易」。天道性命即在此「神」與「易」中極成其道德創生之實義。非佛之空,非老之無也。故天道非他,虛體之神用而已。性命非他,虛體神用之為主而已。橫渠對於佛、老雖不必能盡其義,然此大界限之辨別並不誤也。故前第四段云:「此道不明,正由懵者略知體虛空為性,不知本天道為用」。又云:「不悟一陰一陽,範圍天地,通乎晝夜,〔乃】三極大中之矩,遂使儒、佛老、莊混然一途。語天道性命者,不罔於恍惚夢幻,則定以有生於無為窮高極微之論。」此皆沉雄剛大之言。「懵者略知體虛空為性」,然儒者之言太虛神體非佛氏之所謂空,亦非老氏之所謂無,此正自空其空,自無其無,非吾之所謂虛也。蓋儒者之言太虛神體,之言天道性命,目的乃在明:宇宙之生化即是道德之創造。故言虛言神不能離氣化。氣化是實事,不可以幻妄論。實理主實事,乃立體直貫地成其道德之創造,非只主觀的偏枯之境界。故「範圍天地之化而不過」,「通乎晝夜之道而知」的那「一陰一陽之謂道」正是天地人三極的大中至正之矩:天以此成其為天,地以此成其為地,人以此成其為人,無非是一道德的創造,故云為三極之大中至正之矩也。若不知此義,徒因空、無、與虛相差不遠,「遂使儒、佛、老、莊混然一途」,則大悖矣。

〈神化篇第四〉亦云:

\begin{quotation}\kaishu 氣有陰陽。推行有漸為化,合一不測為神。\end{quotation}

\newpage\thispagestyle{empty}\addtocounter{page}{-1}\vspace*{-12mm}\begin{center}\noindent
\includegraphics[clip, trim=171pt 140pt 104pt 246pt, height=162mm]{ocr-input/image-2032.png}\end{center}

\newpage\markright{第二部 \quad 分論一 \quad 第二章 \quad 張橫渠對於「天道性命相貫通」之展示}

\begin{quotation}\kaishu 其在人也,知〔智】義用利,則神化之事備矣。德盛者,窮
神則智不足道,知化則義不足云。天之化也,運諸氣。人之
化也,順夫時。非氣非時,則化之名何有?化之實何施?
《中庸》曰:「至誠爲能化」。孟子曰:「大而化之」。皆
以其德合陰陽,與天地同流而無不通也。所謂氣也者,非待
其蒸鬱凝聚,接於目而後知之。苟健順動止,浩然湛然之得
言,皆可名之象爾。然則象若非氣,指何為象?時若非象,
指何為時?世人取釋氏銷礙入空,學者舍惡趨善以為化,此
直可為始學遣累者薄乎云爾,豈天道神化所同日語哉?\end{quotation}

\noindent 案:此〈神化篇〉文同于(太和篇〉「知虚空即氣」云云以及「知太虛即氣」云云兩段文之意。同是表示「本體、宇宙論的」道德創造之體用不二、既超越亦內在之充實圓盈之義。化之實、化之事,雖就氣說,然必於氣之虛實、動靜、聚散、有無,兼體而不累,參和而不偏,而見出神,始可成其化。故《易傳〉曰:「窮神知化」,簡言之曰「神化」。說「氣化」乃只就化之實、化之事而言耳。說神化,則即用以明體,通體以達用也。本體宇宙論地說,即就氣之虛實、動靜、聚散、有無之參和不偏,(〈誠明篇〉云:「天本參和不偏」),兼體不累,(〈乾稱篇〉云:「若道,則兼體而無累也」),而見神,因而即說神為本體。故橫渠此處云:「合一不測為神」。「合一」即參和不偏,兼體無累之簡言也。由合一不偏不累而成生化之不測,此即是神也。道德實踐地說,能呈現超越的本心,真至「兼體而不累」,則亦是本心之神用也。此是由「聖人盡道」而見神為體。故〈太和篇〉前第三段云:「聖人盡

\newpage\thispagestyle{empty}\addtocounter{page}{-1}\vspace*{-12mm}\begin{center}\noindent
\includegraphics[clip, trim=163pt 167pt 141pt 227pt, height=162mm]{ocr-input/image-2036.png}\end{center}

\newpage

\noindent 道其間,兼體而不累者,存神其至矣。」此亦孟子所謂「君子所存者神,所過者化」也。「兼體而不累」即是「所過者化」也。然不存本心之神,焉能如此?

本心之實德曰仁義禮智。人之表現仁義之心,其極曰義精仁熟,曰精義入神。故〈神化篇〉末段云:「義以反經為本,經正則精。仁以敦化為深,化行則顯。義入神,動一靜也。仁敦化,靜一動也。仁敦化,則無體。義入神,則無方。」此言之可謂美矣。此由仁義之精熟而至神化也。而上錄之文,亦曰:「智義用利,則神化之事備矣。」智與義之用無往而不利,即上下與天地同流,無往而不通,則雖智、義,而已至于神化矣。至于神,則智之是非相泯。至于化,則義之好惡相泯。故云:「德盛者,窮神則智不足道。知化,則義不足云。」「智不足道」,非無智也,智而神矣。智而神,則唯是一神之明之「虛明照鑑」也。「義不足云」,非無義也,義而化矣。義而化,則唯是一神體之動而無動(所謂「動一靜也」)而無方也。故仁義禮智而至其極,唯是一神體之周流,而體用不二矣。故云:「天之化也,運諸氣。人之化也,順夫時。」「所謂氣也者,非待其蒸鬱凝聚,接於目而後知之。苟健順動止,浩然湛然之得言,皆可名之象爾。」可名之象即是氣。然氣也而有神以妙之,故得為浩然、爲湛然,因而即就浩然湛然而見神。人之表現仁義禮智而至神化之境,則亦必「順夫時」。「順夫時」始具體而真實。神化者具體而真實,充實而圓盈之謂也。「順夫時」即是有象而不離乎氣;全體是氣即全體是虛,全體是虛即全體是氣;全體是象即全體是神,全體是神即全體是象。此之謂神化。此之謂「本體、宇宙論的」道德創生之體用不二,既超越亦內在之充實圓

\newpage\thispagestyle{empty}\addtocounter{page}{-1}\vspace*{-12mm}\begin{center}\noindent
\includegraphics[clip, trim=167pt 146pt 137pt 240pt, height=162mm]{ocr-input/image-2040.png}\end{center}

\newpage\markright{第二部 \quad 分論一 \quad 第二章 \quad 張橫渠對於「天道性命相貫通」之展示}

\noindent 盈之教。

以上皆就「知太虛即氣,則無無」一小段而言。此而確定,則其餘可得而判矣。即間有滯辭,亦須根據體用不二義以通之,不得滯窒而成誤解。

第二小段「太虛為清,清則無礙,無礙故神」云云,此是承〈太和篇〉首段「清通而不可象為神」句而說,無問題。太虛之清、通、神,不可視為氣之謂詞,氣之質性。

第三小段「凡氣清則通,昏則壅。清極則神,故聚而有間,則風行而聲聞具達,清之驗與?不行而至,通之極與?」案此小段說氣清氣濁(昏),是就氣一條鞭說。此則令人可視神為氣之質性而屬於氣,然細會之,此本是就氣之質性說,氣之質性本有清有濁。清濁亦如聚散,亦是氣方面之兩體。(〈太和篇〉下文云:「兩體者,虛實也,動靜也,聚散也,清濁也,其究一而已。」)天地間亦本有清氣者。順清氣固可說通,清通之極固亦可說神,但此是作為清氣之質性的通與神,此是順清氣之直線地說,不是參和清濁不偏,兼體不累,所謂「合一不測之謂神」。此種順清氣之質性而說的通與神,只可算作使吾人領悟太虛神體之引路,不可謂橫渠所說之太虛神體即是氣之質性,氣所蒸發之精英,因而謂其為唯氣論也。嚴格言之,順清氣之質性一條鞭地說,雖可說通,亦只是強度的有限量的通,而不是「感而遂通天下之故」之遍通。通之極,雖亦可有類於神,然亦只是強度的有限量的神,而不是「妙萬物而為言」的神;雖亦有類於「不行而至,不疾而速」,然其為神是強度的有限量的,則亦有時而盡,所謂神采之神,皆是如此,此是假無限,不是真無限,有盡即有行、疾之過程,不是遍妙萬物而為之

\newpage\thispagestyle{empty}\addtocounter{page}{-1}\vspace*{-12mm}\begin{center}\noindent
\includegraphics[clip, trim=164pt 149pt 152pt 252pt, height=162mm]{ocr-input/image-2044.png}\end{center}

\newpage

\noindent 體的神體之「不行而至、不疾而速」。此後者是遍、常、一,動而無動,靜而無靜,無過程,無窮盡,嚴格說,實無所謂「至」(不論行不行),亦無所謂速(不論疾不疾)。此不能視作氣之質性。故橫渠此處順清氣直線地說通說神,只能算作領悟太虛神體之引路。就清氣之質性,可對於太虛神體之清通得一經驗的征驗。而經驗的徵驗究不是太虛神體本身也。對此太虛神體之先天的、超越的徵驗,惟在超越的道德本心之神。至此,則太虛神體之非可視為氣之質性全部明朗。如果作為清氣之質性的通與神與太虛神體劃不開,而將氣之觀念直線地,一條鞭地直通於太虛神體之神,結果便是神屬於氣,心亦屬於氣。此種一條鞭地著跡的想法,最顯明者便是朱子。橫渠在此亦不能自覺地劃得開,此則成混擾,故令人有唯氣論之想也。因為有此不清之混擾,故有第四小段中「合虛與氣有性之名」之滯辭。然揆之橫渠對於太虛神體之體悟以及其體用不二之論,則必須劃開。為免混擾,於氣外,必須正式建立神一觀念。而以濂溪「動而無動,靜而無靜,神也」之語確定之。是則氣之觀念即不能通於此。神之意義有時屬於氣之質性,如神氣、神采之類,但此太虛神體則不可視為氣之質性,認為屬於氣。前人多不甚能劃得開,而又時與體用不二之圓融論相混擾。故吾於此詳為分解以明義理之分齊。讀者明乎此,則對前賢之語當隨文善會,不可誤解。

\subsection{鬼神之神與太虛神體之神之不同}

\begin{quotation}\kaishu 8.鬼神者二氣之良能也。聖者至誠得天之謂。神者太虛妙應
之目。凡天地法象皆神化之糟粕爾。天道不窮,寒暑也。\end{quotation}

\newpage\thispagestyle{empty}\addtocounter{page}{-1}\vspace*{-12mm}\begin{center}\noindent
\includegraphics[clip, trim=164pt 123pt 128pt 252pt, height=162mm]{ocr-input/image-2048.png}\end{center}

\newpage\markright{第二部 \quad 分論一 \quad 第二章 \quad 張橫渠對於「天道性命相貫通」之展示}

\begin{quotation}\kaishu 眾動不窮,屈伸也。鬼神之實,不越二端而已矣。兩不
立,則一不可見。一不可見,則兩之用息。兩體者,虛實
也,動靜也,聚散也,清濁也,其究一而已。\end{quotation}

\noindent 案:此整段中、神、太虛、兩、一,皆前已解,總表「神化」之事。惟此中仍有滯辭,即「鬼神者,二氣之良能也」一語是。此是關於鬼神之問題。鬼神之神是太虛神體之神乎?抑不是乎?「鬼神者二氣之良能」是實然之陳述語。在此實然之陳述中,鬼神是陰陽二氣之質性、性能,故曰:「良能」。「鬼神之實不越二端而已」。二端就是上文之寒暑、屈伸,亦即下文之兩體。「鬼神之實不越二端」即不越氣之屈伸,此是就氣化之實然之狀說,將鬼神化歸於氣化,予以宇宙論的解析。鬼者歸也,神者伸也。氣之屈(歸回)即是鬼,氣之伸即是神。氣之屈陰也,氣之伸陽也。故「不越二端」,亦即是「二氣之良能」。如此作解,則鬼神之神不能視作即是太虛神體之神。

但〈乾稱篇〉開首云:「凡可狀皆有也。凡有皆象也。凡象皆氣也。〔案:此三句即〈太和篇〉首段「散殊而可象為氣」一語。】氣之性本虛而神,則神與性乃氣所固有。此鬼神所以體物而不可遺也。(原自注:舍氣有象否?非象有意否?)。」案此文首三句無問題。「氣之性本虛而神」以下則多滯窒。「氣之性本虛而神」一語本即是(太和篇〉首段「清通而不可象為神」以及第二段「太虛無形,氣之本體」兩語之意。但〈太和篇〉語較成熟,而「氣之性本虛而神,則神與性乃氣所固有」,則生硬滯窒。「氣之性」即氣之體。〈乾稱篇〉後文亦曰:「太虛者氣之體」。故此

\newpage\thispagestyle{empty}\addtocounter{page}{-1}\vspace*{-12mm}\begin{center}\noindent
\includegraphics[clip, trim=161pt 158pt 140pt 227pt, height=162mm]{ocr-input/image-2052.png}\end{center}

\newpage

\noindent 「性」字實即體字。但在此說「體」字實較說「性」字為順適。故吾於前文第二段解「太虛無形,氣之本體」時,曾引此助解,即將此「性」字解為體性,意同於體,且明是超越的體性,並非實然之質性。但說「氣之性」則容易使人想成氣之質性,此即成誤解。此其所以為滯辭也。「神與性乃氣所固有」句尤其窒礙不順,尤易使人想成氣之質性。故此兩語當以(太和篇〉語及〈乾稱篇〉之後文(引見前解第二段)為準而解之。如此,則〈乾稱篇〉開首此段文首三句說氣,此兩句說太虛神體。而太虛神體不應視作氣之質性。但其結語又說:「此鬼神所以體物而不可遺也」。此語又增麻煩。此又將鬼神通於太虛神體而為一矣。此是松弛之聯想,不可云精審之思。如依上文「鬼神者二氣之良能」以及「鬼神之實不越二端而已」之解析,則鬼神之神不能與太虛神體之神視作一事。

橫渠於〈乾稱篇〉由太虛神體之為體想到「鬼神體物而不可遺」是本諸《中庸》。《中庸》曰:「子曰:鬼神之為德,其盛矣乎?視之而弗見,聽之而弗聞,體物而不可遺,使天下之人齊明盛服以承祭祀,洋洋乎如在其上,如在其左右。詩曰:神之格思,不可度思,矧可射思?夫微之顯,誠之不可揜如此夫!」此言祭祀之誠敬以格神。當祭祀時,主觀方面有誠敬之心,則客觀方面之神即「洋洋乎如在其上,如在其左右」,覺得周流充滿,無所不在。即由此「洋洋乎」無所不在而謂其「體物而不可遺」。朱子注云:「是其為物之體,而物所不能遺也。」由神之體物而為之體(無所不在),故物亦不能遺而離之也。故雖視之而弗見,聽之而弗聞,然又洋洋乎而無所不在也。此即是鬼神之盛德。然必有誠敬之心,始能有此感格。故最後云:「微之顯,誠之不可揜如此夫。」此雖

\newpage\thispagestyle{empty}\addtocounter{page}{-1}\vspace*{-12mm}\begin{center}\noindent
\includegraphics[clip, trim=158pt 135pt 131pt 236pt, height=162mm]{ocr-input/image-2056.png}\end{center}

\newpage\markright{第二部 \quad 分論一 \quad 第二章 \quad 張橫渠對於「天道性命相貫通」之展示}

\noindent 覺其「體物而不可遺」,然畢竟仍是就祭祀說,仍是鬼神之義,非「本體、宇宙論的」太虛神體之義。

大體鬼神的經驗從古就有。《左傳》即多記鬼神之事。主要是就祭祀說。就祭祀說,鬼神是已存在的生命之歸於幽冥。此仍可視為幽冥中之實然的存在。視為一個體生命(自然的或是德性的)之精靈不散可,視為氣之屈伸,予以宇宙論的說明,亦可。然無論如何解析,總是屬於精氣之實然。既是精氣之實然,就前一解析看,亦無永久不散之理。此在或有或無之間。故孔子對於鬼神之態度,據《論語》說,一是「敬鬼神而遠之」,一是「未能事人,焉能事鬼?」一是「子不語怪、力、亂、神」,一是「祭神如神在」。即就《中庸》此文說,如此文真是孔子之言,則亦是就祭祀言,以誠敬為主,與「祭神如神在」同。孔子不以此為主。孔子所重視、視之為天人之綱維者,主觀地說是仁,客觀地說是天、天道、天命。鬼神的地位並不高,是仁與天道、天命之間的實然存在。孔子的超越又內在的精神是在仁與天道。普通就其對於鬼神之態度而衡量其宗教精神,非是。如自高級宗教言,則衡量其宗教精神當然須就其所言之仁與天道、天命說,而不能就其所言之鬼神說。因為在中國,鬼神並未取得一真正崇高之地位,亦並不真是一超越之實體,而基督教之神(上帝)並不是鬼神之神。基督教方面就鬼神衡量孔子之宗教精神正是減低其宗教。此是庸俗之輩之鄙陋,非知者之言。以西方宗教精神與印度比,只能與其梵天比,而不能與其種種之神比。同樣,若與中國比,則只能就天、帝、天命、天道比,而不能就鬼神比。而天、帝、天命、天道乃是孔子以前之老傳統,此代表真正之超越者。孔子承之而不悖,而復提出仁以實之。此一轉

\newpage\thispagestyle{empty}\addtocounter{page}{-1}\vspace*{-12mm}\begin{center}\noindent
\includegraphics[clip, trim=170pt 158pt 138pt 230pt, height=162mm]{ocr-input/image-2060.png}\end{center}

\newpage

\noindent 化,遂使中國無普通之宗教,但不能謂其無宗教精神與宗教境界,即不能謂其無宗教性。但此宗教性正恰恰須就天、帝、天命、天道說,而不能就鬼神說。即就鬼神說,亦是以天、帝、天命、天道為綱主而成之宗教性容量之廣大,亦即宗教精神之充其極(如友人唐君毅先生所說),所帶起者,故儒家必肯定三祭:祭天、祭祖、祭聖人。此即其非普通之宗教處。天不可以鬼神論。鬼神的觀念只能應用於祖與聖人。祖宗不必皆是有極高之德性者,然所以必祭之,乃是崇始報本之意。其死後是否成神,是否精靈仍不散,並不是重要者。故重在自己之仁德與誠敬,不重在對方之存在。至於祭聖人,是重視其德性生命,是對於其德性人格之崇敬。其死後是否成神,精靈不散,亦非重要者。故就祭祀言,仍是祭神如神在。惟祭天則不同。天不可以鬼神論,天是真正的超越體,是必須積極肯定者。踐仁以契之,正示仁與天只是一道德實體之遍在,此是儒家宗教精神之最精特處。由此亦示鬼神正是夾縫中之存在,乃是由德性所帶起者,故須以誠敬貢註之。其自身存在不存在無關也。就其自身說,仍是實然之精氣上的事。故宋儒得以陰陽二氣之屈伸明之也。及夫以二氣之屈伸明之,則其在幽冥中為一個體式的存在之義即全融化而不存。此亦示其存在不存在並不重要也。此即為宇宙論之解析。

鬼神雖為實然之精氣上的事,然當以誠敬之心感格之時,則覺其洋洋乎而無所不在,周流充滿,有類乎無限。實則彼自是屬有限之事。此是主體之誠敬之心之感通而將其擴大化、無限化,遂即以為是鬼神之盛德矣。故由鬼神之盛德反而亦可證成主體誠敬之心之神用。客觀之鬼神是有限,而主體之誠敬之心之神用則無限。此是

\newpage\thispagestyle{empty}\addtocounter{page}{-1}\vspace*{-12mm}\begin{center}\noindent
\includegraphics[clip, trim=158pt 134pt 129pt 238pt, height=162mm]{ocr-input/image-2064.png}\end{center}

\newpage\markright{第二部 \quad 分論一 \quad 第二章 \quad 張橫渠對於「天道性命相貫通」之展示}

\noindent 從道德的超越的本心之誠德上說,不是從實然之精氣上說。由是遂有從誠體上說「神」之一義。不管是祭祀時之誠,還是卜筮時之誠,還是待人接物之誠,總之,不管所關聯之對象是什麼,而道德的誠敬之心之自身即呈現一不測之神用。自孟子說「大而化之之謂聖,聖而不可知之謂神」,此神完全是誠德上的事。又曰:「萬物皆備於我矣。反身而誠,樂莫大焉。」此明示本心之無外亦即誠體之無外。又曰:「君子所存者神,所過者化,上下與天地同流,豈曰小補之哉?」此亦明示誠體之無外即是誠體之神之無外。由此而至《中庸》由誠體言天道之為物不貳則生物不測以及《易傳》之窮神知化,此皆是由誠體說神,非鬼神之神。將道德的誠體之神全融於天命、天道之中而與之合而為一,由此天命天道遂有其具體的真實內容,而不只是一形式的實體,其為生化之理、存在之理遂亦得其具體之實焉。因而遂有《中庸》《易傳》所展示之「本體、宇宙論的」道德創造、宇宙生化之體用不二,既超越亦內在之充實圓盈之「神化」論。此誠體之神,雖在「本體、宇宙論」處,不離陰陽之氣,所謂體用不二,然此是圓融義,決不可因此即視之為氣之質性,亦決不可視之為鬼神之神。

宋儒興起,濂溪之《通書》完全由誠體寂感之神說天道,即以此解其所說之太極亦是當然之事而決不會有問題者。惜乎朱子不善會也。橫渠以太虛神體詮表天道亦是繼承此義而說。其著力於體用不二之神化以辨佛老,可知其對于《中庸》、《易傳》所展示之神化之體悟並不誤也。惜乎多有滯辭,不能將鬼神之神與太虛神體之神分開說;本是體用不二之圓融論,卻常有使人視神為氣之質性處。如是遂有二程之誤解,以及近人唯氣論之誤解,亦遂有朱子視

\newpage\thispagestyle{empty}\addtocounter{page}{-1}\vspace*{-12mm}\begin{center}\noindent
\includegraphics[clip, trim=170pt 146pt 144pt 251pt, height=162mm]{ocr-input/image-2068.png}\end{center}

\newpage

\noindent 心神俱屬於氣之分解表示。而實則此皆非橫渠之本意也,亦非先秦《中庸》、《易傳》之原義也。故吾于此疏通其滯,將氣與神分別建立,不能一條鞭地將氣之觀念通於神體,不能視太虛神體為氣之質性,不能將鬼神之神與太虛神體之神混而為一,清氣之通而有類乎神只能視作體悟太虛神體之引路,鬼神之「洋洋乎如在其上,如在其左右」亦只能視作體悟太虛神體之引路。如是,則橫渠之「本體、宇宙論」中之滯辭即可釐清,而其正大之義理亦朗然而貞定矣。不惟「唯氣論」之謬解不得濫施,即朱子之分解表示亦可明其何由而至:其視性與太極為只是理,而將心神俱屬於氣,即示其對於由誠體而建立之神義並不解,對於超越之本心亦並不解也。若是鬼神之神與誠體之神劃不開,體用不二之圓融論與氣之質性之實然陳述劃不開,則朱子之系統乃必然者,其分解表示乃甚一貫者。但如此,則心性不合一,真正之自律道德不能講,而其只由「所以然」以推證之「形式的理」之存有終將塌落而不能自持,此其弊將不可勝言。此朱子學之症結也。

〈太和篇〉疏解止于此。餘尚有若干小段,皆重複雜衍之辭,義已盡,亦不必煩為疏解矣。

\noindent 附錄:朱子之評論

《朱子語類》卷第九十九,〈張子之書二〉,有以下各條:

\begin{quotation}\kaishu 1.《正蒙》所論道體,覺得源頭有未是處。故伊川云:「過
處乃在《正蒙》。」答書之中云:「非明睿所照,而考索
至此。」蓋橫渠卻只是一向苦思,求將向前去。卻欠涵\end{quotation}

\newpage\thispagestyle{empty}\addtocounter{page}{-1}\vspace*{-12mm}\begin{center}\noindent
\includegraphics[clip, trim=160pt 124pt 112pt 251pt, height=162mm]{ocr-input/image-2072.png}\end{center}

\newpage\markright{第二部 \quad 分論一 \quad 第二章 \quad 張橫渠對於「天道性命相貫通」之展示}

\begin{quotation}\kaishu 泳,以待其義理自形見處。如云:「由氣化有道之名」,
說得是好,終是生受辛苦。聖賢便不如此說。試教明道
說,便不同。如以太虛、太和為道體,卻只是說得形而下
者,皆是「發而皆中節謂之和」處。

2.《正蒙》說道體處,如太和、太虚、虚空云者,止是說
氣。說聚散處,其流乃是個大輪迴。蓋其思慮考索所至,
非性分自然之知。若語道理,惟是周子說「無極而太極」
最好。如「由太虛有天之名,由氣化有道之名,合虛與氣
有性之名,合性與知覺有心之名」,亦說得有理。「由氣
化有道之名」,如所謂「率性之謂道」是也。然使明道形
容此理,必不如此說。伊川所謂「橫渠之言誠有過者,乃
在《正蒙》」,「以清虛一大為萬物之源,有未安」〔此
當係明道語〕等語,概可見矣。

3.問:橫渠太虛之說,本是說無極,卻只說得無字。

曰:無極是該貫虛實清濁而言。無極字落在中間,太虛字
落在一邊了。便是難說。聖人熟了,說出便恁地平平。而
今把意思去形容他,卻有時偏了。明道說:「氣外無神,
神外無氣。謂清者為神,則濁者非神乎?」後來亦有人與
橫渠說,横渠卻云:「清者可以眩濁,虛者可以眩實。」
卻不知形而上者還他是理,形而下者還他是器。既說是
虛,便是與實對了。既說是清,便是與濁對了。如左丞相
大得右丞相不多。

問曰:無極且得做無形無象說。

曰:雖無形,卻有理。\end{quotation}

\newpage\thispagestyle{empty}\addtocounter{page}{-1}\vspace*{-12mm}\begin{center}\noindent
\includegraphics[clip, trim=269pt 169pt 143pt 231pt, height=162mm]{ocr-input/image-2076.png}\end{center}

\newpage

\begin{quotation}\kaishu 又問:無極太極只是一物。

曰:本是一物,被他恁地說,卻似兩物。

4.橫渠說道,止於形器中揀個好底說了。謂清為道,則濁之
中果非道乎?客感客形與無感無形,未免有兩截之病。聖
人不如此說。如曰:「形而上者謂之道」。又曰:「一陰
一陽之謂道」。

5.問:横渠云:「太虛即氣」,太虛何所指?

曰:他亦指理,但說得不分曉。

曰:太和如何?

曰:亦指氣。

曰:他又云:「由昧者指虚空為性,而不本天道」。如
何?

曰:既曰道,則不是無。釋氏便直指空了。大要渠當初說
出此道理多誤。

6.問:橫渠說「天性在人,猶水性之在冰。凝釋雖異,為理
一也。」〔案:見〈誠明篇〉】。又言:「未嘗無之謂
體,體之謂性。」〔案:亦見〈誠明篇〉】先生皆以其言
為近釋氏。冰水之喻有還元反本之病,云近釋氏則可。
「未嘗無之謂體,體之謂性」,蓋謂性之為體本虛,而理
未嘗不實。若與釋氏不同。

曰:他意不是如此。亦謂死而不亡耳。

7.問:橫渠謂「所不能無感者謂性。」〔案:此〈誠明篇〉
語】性只是理,安能感?恐此言只可名心否?

曰:橫渠此言雖未親切,然亦有個模樣。蓋感固是心,然\end{quotation}

\newpage\thispagestyle{empty}\addtocounter{page}{-1}\vspace*{-12mm}\begin{center}\noindent
\includegraphics[clip, trim=156pt 140pt 135pt 236pt, height=162mm]{ocr-input/image-2080.png}\end{center}

\newpage\markright{第二部 \quad 分論一 \quad 第二章 \quad 張橫渠對於「天道性命相貫通」之展示}

\begin{quotation}\kaishu 所以感者亦是此心中有此理方能感。理便是性。但將此句
要來解性,便未端的。如伊川說:「仁者天下之正理」,
又曰:「仁者天下之公,善之本也。」將此語來贊詠仁,
則可。.要來正解仁,則未親切。如義豈不是天下之正理?
8.橫渠闢釋氏輪回之說,然其說聚散屈伸處,其弊卻是大輪
回。蓋釋氏是個個各自輪回,橫渠是一發和了,依舊一大
輪迴。呂與叔集中亦多有此義。

9.問:「虛者仁之源。」〔案:《正蒙》中無此語。《性理
拾遺·孟子說》中有云:「敦篤虛靜者仁之本。」蓋本此
而略言之。

曰:虚只是無欲故虛。虛明無欲,此仁之所由生也。

又問:此虛字與「一大、清虛」之虛如何?

曰:這虛也只是無欲。渠便將這個喚做道體。然虛對實而
言,卻不似形而上者。

10.問:橫渠有「清虛、一、大」之說,又要兼清濁虛
實。
曰:渠初云清、虛、一大,爲伊川詰難,〔據上第3
條文及下第16條,當作「為明道詰難」〕,乃云清兼
濁,虛兼實,一兼二,大兼小。渠本要說形而上,反成
形而下。最是於此處不分明。如〈參兩〉云,以參為
陽,兩為陰,陽有太極,陰無太極。他要強索精思,必
得於己,而其差如此。
又問:橫渠云「太虛即氣」,乃是指理為虛,似非形而
下。\end{quotation}

\newpage\thispagestyle{empty}\addtocounter{page}{-1}\vspace*{-12mm}\begin{center}\noindent
\includegraphics[clip, trim=285pt 161pt 140pt 251pt, height=162mm]{ocr-input/image-2084.png}\end{center}

\newpage
曰:縱然指理為虛,亦如何夾氣作一處?

\begin{quotation}\kaishu 11.或問:橫渠先生「清、虛、一、大」之說如何?
曰:他是揀那大底說話,來該攝那小底。卻不知道纔是
恁說,便偏了,便是形而下者,不是形而上者。須是兼
清濁、虛實一萬、小大來看,方見得形而上者行乎其
間。

12.橫渠清、虛、一、大卻是偏。他後來又要兼清濁虛實
言。然皆是形而下。蓋有此理,則清濁虛實皆在其中。

13.橫渠說清、虛、一、大,恰似道有有處、有無處。須是
清濁、虛實、一二、大小皆行乎其間,乃是道也。其欲
大之,乃反小之。

14.陳後之問:橫渠清、虛、一、大,恐入空去否?
曰:也不是入空。他都向一邊了。這道理本平正。清也
有是理,濁也有是理,虛也有是理,實也有是理:皆此
理之所為也。他說成這一邊有,那一邊無。要將這一邊
去管那一邊。

15.清、虛、一、大,形容道體如此。道兼虛實言,虛只說
得一邊。

16.橫渠言清虛、一、大為道體,是於形器中揀出好底來
說了。《遺書》中明道嘗辨之。\end{quotation}

\noindent 案:以上諸評解皆非是。試觀吾之疏解,橫渠之實意當可知。

\newpage\thispagestyle{empty}\addtocounter{page}{-1}\vspace*{-12mm}\begin{center}\noindent
\includegraphics[clip, trim=153pt 242pt 148pt 250pt, height=162mm]{ocr-input/image-2088.png}\end{center}

\newpage\markright{第二部 \quad 分論一 \quad 第二章 \quad 張橫渠對於「天道性命相貫通」之展示}

\section{「合虛與氣有性之名」:性體義疏解}

\subsection{言虛言道皆結穴於性}

前第一節疏解(太和篇〉第七段中有「由太虛有天之名」等四句。本節疏解首三句以明性,以第三句為中心。蓋天道性命相貫通,是以凡言天、言道、言虚、言神,乃至言太極,目的皆在建立性體,亦可云皆結穴于性也。下第三節疏解第四句,藉之以明心。

\begin{quotation}\kaishu 由太虚有天之名。\end{quotation}

〈乾称篇〉云:

\begin{quotation}\kaishu 大率天之為德,虛而善應。其應非思慮聰明可求,故謂之
神。老氏況諸谷以此。\end{quotation}

\noindent 案:虛則至寂,寂然不動。善應則神,感而遂通。此所以「由太虛有天之名」。天即天德之天。天以健行創生為德。健行創生之德,其實處只是「虛而善應」。此即天之德也,亦即天也。天有自然義,此對遮人為。復有當然而不容已、定然而不可移之義,此即天則義。前〈太和篇〉第三段中所謂「循是出入,是皆不得已而然也」,亦即此天則義。此一義是就太虛神體之健行創生之德以運氣化,因而有氣化(出入聚散即氣化)之必然性而言。此必然性是形

\newpage\thispagestyle{empty}\addtocounter{page}{-1}\vspace*{-12mm}\begin{center}\noindent
\includegraphics[clip, trim=191pt 167pt 134pt 240pt, height=162mm]{ocr-input/image-2092.png}\end{center}

\newpage

\noindent 而上的必然性,非邏輯的必然性。當然而不容已,定然而不可移,即是形而上的必然也。

\begin{quotation}\kaishu 由氣化有道之名。\end{quotation}

\noindent 〈神化篇〉云:

神天德,化天道。德其體,道其用。一於氣而已。

虛明照鑑,神之明也。無遠近幽深利用出入,神之充塞無
間也。天下之動,神鼓之也。辭不鼓舞,則不足以盡神。

氣有陰陽。推行有漸為化,合一不測為神。

神化者天之良能,非人能。

徇物喪心,人化物而滅天理者乎?存神過化,忘物累而順性
命者乎?敦厚而不化,有體而無用也。化而自失焉,徇物而
喪己也。大德敦化,然後仁智一,而聖人之事備。性性為能
存神,物物為能過化。

義以反經為本,經正則精。仁以敦化為深,化行則顯。義入
神,動一靜也。仁敦化,靜一動也。仁敦化,則無體。義入
神,則無方。

\noindent 總此,則知由神之鼓舞而有氣之化。通體而達用,帶著氣化之用言,則曰道。故曰:「由氣化有道之名。」如此說道,是動態地說,必帶著氣化之行程言。但不只是著眼於實然之氣化,蓋氣化之用必通虛德之體而始然。無虛德之體,即無氣化可言。故曰:「神

\newpage\thispagestyle{empty}\addtocounter{page}{-1}\vspace*{-12mm}\begin{center}\noindent
\includegraphics[clip, trim=150pt 141pt 154pt 242pt, height=162mm]{ocr-input/image-2096.png}\end{center}

\newpage\markright{第二部 \quad 分論一 \quad 第二章 \quad 張橫渠對於「天道性命相貫通」之展示}

\noindent 天德,化天道。德其體,道其用。一於氣而已。」「一於氣」,言德體道用皆統一於氣而不能離氣以言也。

又,「太和所謂道」,道亦可是綜和詞。既是太和,當然不離氣之絪縕。但也不只是實然之絪縕,必有虛體以妙之。如此,若靜態地分合言之,亦可說「合虛與氣有道之名」。合虛與氣而成化,則道之名立焉。道之名由此立,道之義亦由此見。

但第三句「合虛與氣有性之名」,則是滯辭。言天、言道、言虛、言神,皆結穴於「性」。是以「性體」之詞必須另述。由「合虛與氣」說,則不諦也。

\subsection{性之名之所以立}

\begin{quotation}\kaishu 合虛與氣有性之名。\end{quotation}

茲藉此語以疏解性體義。

道為綜和詞,分之為虛與氣。動態地說,則須帶著氣化言。由虛以立體,由氣化以達用。故曰:「神天德,化天道。德其體,道其用。」道雖為綜和詞,然可偏重氣化之行程言。而性則必超越分解地偏就虛體言。作為體之神德太虛對應個體,或總對天地萬物而為其體言,則曰「性」。故(乾稱篇〉曰:

\begin{quotation}\kaishu 妙萬物而謂之神,通萬物而謂之道,體萬物而謂之性。\end{quotation}

\noindent 此三語甚好。經由妙、通萬物而體之,以為其所本所據,因而即為其體,則曰性。性與神一也,皆偏就虛體言。至於道,則本虛體以

\newpage\thispagestyle{empty}\addtocounter{page}{-1}\vspace*{-12mm}\begin{center}\noindent
\includegraphics[clip, trim=182pt 184pt 133pt 217pt, height=162mm]{ocr-input/image-2100.png}\end{center}

\newpage

\noindent 通貫萬物而成化也。此則偏就氣化之通貫言。性就太虛神德言。太虛神德之為體即天地萬物之性也。故(誠明篇〉云:

\begin{quotation}\kaishu 性者,萬物之一源,非有我之得私也。

未當無之謂體,體之謂性。\end{quotation}

又直云:「天地之性」。如云:

\begin{quotation}\kaishu 形而後有氣質之性,善反之,則天地之性存焉。\end{quotation}

\noindent 此性體是涵蓋乾坤而為言,是絕對地普遍的。雖具於個體,亦是絕對地普遍的,「非有我之得私也」。此性是我之性,亦是天地萬物之性。言道言虛,其總結穴在性。言性,即為的是建立道德創造之源,非是徒然而泛然之宇宙論也。氣化之道亦必由道德創造來貞定、來證實。故性字必偏就虛體言,所以立本也。是以由「體萬物」而言性,勝於由「合虛與氣」而言性多多矣。

〈西銘〉亦曰:

\begin{quotation}\kaishu 天地之塞,吾其體;天地之帥,吾其性。\end{quotation}

\noindent 孟子言「志,氣之帥也。」故「天地之帥」即天地之志,以志帥氣也。志之實即是太虛神德,此即吾之性也。此亦偏就太虛神德言性,不由「合虛與氣」而言也。

〈誠明篇〉亦曰:

\newpage\thispagestyle{empty}\addtocounter{page}{-1}\vspace*{-12mm}\begin{center}\noindent
\includegraphics[clip, trim=155pt 161pt 152pt 224pt, height=162mm]{ocr-input/image-2104.png}\end{center}

\newpage\markright{第二部 \quad 分論一 \quad 第二章 \quad 張橫渠對於「天道性命相貫通」之展示}

\begin{quotation}\kaishu 天所性者,通極於道,氣之昏明不足以蔽之。天所命者,通
極於性,遇之吉凶不足以戕之。不免乎蔽之戕之者,未之學
也。性通乎氣之外,命行乎氣之內。氣無內外,假有形而言
爾。故思知人,不可不知天。盡其性,然後能至於命。知性
知天,則陰陽鬼神皆吾分內爾。

莫不性諸道,命諸天。〔此段全文,見下第五段引。]\end{quotation}

\noindent 案:「通極於道」,「性諸道」之道是偏就帶著氣化之虛體神德言,重本也。道雖必帶著氣化,而不就是實然之氣化。通體以達用,若大路然,故曰道。道總是道,而不能就是氣,甚至亦不能就是氣化。「性諸道」之道是重視那通體達用之「體」義,由之以說性之根。「天所性者,通極於道」之道亦然。言天地之性只能植根於道,不能植根於氣也。道是由通體達用以見,則性之通極於道,通極於「通體達用」而重其體,即通極於此達用之體,則亦見性體之必函一道德的創造也。故《中庸》曰:「率性之謂道」也。故有時性亦同於道。自性之本義言,自超越地分解以立體言,則性同於太虛神德。自性之必函道德的創造言,此猶綜和地由通體達用以成化而見道,如此,則性同於道。所謂性外無道,道外無性也。(此處對此段文只註意「天所性者通極於道」一語,至「天所命者通極於性」以及此下「盡性至命」,「陰陽鬼神皆吾分內」云云,於後隨文明之。)

〈乾稱篇〉亦曰:

\begin{quotation}\kaishu 性通極於無,氣其一物爾。命稟同於性,遇乃適然焉。人一\end{quotation}

\newpage\thispagestyle{empty}\addtocounter{page}{-1}\vspace*{-12mm}\begin{center}\noindent
\includegraphics[clip, trim=163pt 176pt 143pt 220pt, height=162mm]{ocr-input/image-2108.png}\end{center}

\newpage

\begin{quotation}\kaishu 己百,人十己千,然有不至,猶難語性,可以語氣。行同報
異,猶難語命,可以語遇。\end{quotation}

\noindent 此完全同於〈誠明篇〉所說者之語意,不過(誠明篇〉所說尤為精練耳。「通極於無」之無即虛體也。此亦就虛體言性所以立之根,不「合虛與氣」以言也。「命稟同於性」,此命即天命之命命令之命,非命運遭遇之命。此命是人之稟受于而且同一于性者。天之命於穆不已,以成天命之流行,天道之生化,人之「稟同於性」之命亦是不已地流行其命令,亦即性之命之不已,以成道德的創造,以成道德行為之無間,純亦不已。「天命之謂性」是後溯性所以立之根源。.「命稟同於性」是前看性之命之不已以言道德之創造,以定吾人之大分。前〈誠明篇〉「天所命者通極於性」亦同此解。

是以性者,言道言虛之結穴,首先其義有二:一者性能義,二者性分義。性能者,言此性能起道德創造之大用也。性分者,言道德創造中每一道德行為皆是吾人性體中之本分也,責無旁貸而不容已之本務也,所謂必然的義務也,無條件地非如此不可也。此即是吾人之大分。

\subsection{性體之具體意義與具體內容}

性體何以能具此二義?以下試進而就性體之具體意義與具體內容而明之。

性體之具體意義與具體內容仍須就太虛神德而言之。〈太和篇〉只言「至靜無感,性之淵源。」此外未多言。「至靜無感」即是寂然不動,至寂至靜,默然無有,此是性體之最深(淵)源頭處

\newpage\thispagestyle{empty}\addtocounter{page}{-1}\vspace*{-12mm}\begin{center}\noindent
\includegraphics[clip, trim=159pt 150pt 133pt 230pt, height=162mm]{ocr-input/image-2112.png}\end{center}

\newpage\markright{第二部 \quad 分論一 \quad 第二章 \quad 張橫渠對於「天道性命相貫通」之展示}

\noindent (源)。至密至奧亦自這裡說。(後來胡五峰言「性也者,天地鬼神之奧也」,亦是繼承此義而說。)然太虛神德之至寂至靜並不與其「感而遂通」為對立,乃是即寂即感,寂感一如的。否則無以見神德。此寂感一如方是性體之最深源頭處。〈太和篇〉於此只言「至靜無感」,而略其「感而遂通」,是想與下文「有識有知,物交之客感爾」之「客感」對言。如實言之,並非性體只是無感,而凡有感皆是物交之客感。客感是經驗的、現象的,與外物接觸而始然,而即寂即感之「感而遂通」之感則是超越的。是神感神應之常感,而常感即常寂。客感與客形相應,有聚散、有動靜、有出入、有生滅,而常寂常感之神感神應則無聚散、無出入、無動靜(動而無動,靜而無靜)、無生滅。客感屬氣,而與寂為一之常感則屬神,此即虛體之神德也。故性體之具體意義仍須就太虛神德之寂感言:即寂即感,寂感一如,此其所以為神而亦所以能成化也。亦即其所以能起道德之創造也。

〈誠明篇〉云:

\begin{quotation}\kaishu 天所自不能已者謂命,不能無感者謂性。\end{quotation}

\noindent 此首句言天之命之於穆不已。天之「自不能已」即是天之命,此即太虛神德之不能已地去生化萬物以成其為宇宙論的創造也。而太虛神德之所以為神,所以能如此生化,正由其即寂即感也,即由寂感一如而見也。故此太虛神德之由體萬物而為萬物之性,此性即不能不有「寂感」以為其神用。「不能無感」當云「不能無寂感」。此承天之「自不能已」而言此以說性體也。性、命、天是一也。說天

\newpage\thispagestyle{empty}\addtocounter{page}{-1}\vspace*{-12mm}\begin{center}\noindent
\includegraphics[clip, trim=161pt 153pt 146pt 239pt, height=162mm]{ocr-input/image-2116.png}\end{center}

\newpage

\noindent 說命,結穴於性也。如此,性不是乾枯的死體,亦不是抽象的死理,乃是能起宇宙論的創造或道德的創造者。故寂感一如之神即是性體之具體的意義與具體的內容。

〈乾称篇〉云:

\begin{quotation}\kaishu 至誠,天性也。不息,天命也。人能至誠,則性盡而神可窮
矣。不息,則命行而化可知矣。學未至知化,非真得也。\end{quotation}

\noindent 此以「至誠」明天性。至誠必然地函創生之不已(不息)。「不已」即是天命之於穆不已。故云:「不息,天命也。」然「至誠」不是僱侗地說一個誠,實即是寂感之神。故濂溪亦即以寂感真幾說誠體也。而橫渠於此亦言「人能至誠,則性盡而神可窮」。此明示性與神皆在「至誠」中也。由至誠盡性而窮神,由不息命行而知化。「神天德,化天道。德其體,道其用。」此〈神化篇〉所已言者,即由神德之體以立性也。

〈乾稱篇〉又云:

\begin{quotation}\kaishu 感者性之神,性者感之體。(原自注:在天在人,其究一
也)。惟屈伸、動靜終始之能一也。故所以妙萬物而謂之
神,通萬物而謂之道,體萬物而謂之性。\end{quotation}

\noindent 案:此感即「感而遂通」之感,非物交之「客感」,故曰神。此性體之神用也。所以有此神用,以其虛也。故神用即「虛而善應」之謂。(亦(乾稱篇〉語)。雖善應,而實至寂至靜。「動而無動,

\newpage\thispagestyle{empty}\addtocounter{page}{-1}\vspace*{-12mm}\begin{center}\noindent
\includegraphics[clip, trim=168pt 138pt 129pt 245pt, height=162mm]{ocr-input/image-2120.png}\end{center}

\newpage\markright{第二部 \quad 分論一 \quad 第二章 \quad 張橫渠對於「天道性命相貫通」之展示}

\noindent 靜而無靜」故也。此即寂感一如之真幾。勉強分言之,亦可曰至寂之虛即是感之體,神感善應即是寂之用。而此寂感真幾即是性,故感是性體之神用,則性體即是發此神感之體也。此種體用只是名言對說之施設,實則體即神,神即體也。

\subsection{由「兼體」與「合兩」以明性體寂感之神}

以上由寂感之神以明性體之具體意義與內容。

茲再進而明寂感之神由「兼體」、「合兩」而見。此義是說超越之神體必兼合經驗之象而為一,始成其為具體而真實之神體,因而始得以成其宇宙之創造或道德之創造也。是則性體之神用即同天道矣。

〈誠明篇〉云:

\begin{quotation}\kaishu 性,其總,合兩也。命,其受,有則也。不極總之要,則不
至受之分。盡性窮理,而不可變,乃吾則也。\end{quotation}

\noindent 性之「總」義由「合兩」而見。「總」者即總合虛實、動靜、聚散、清濁之兩體而不偏滯於一隅(一象)以成化也。即由此不偏滯以成化,以見性體寂感之神也。此由性體之必通貫形氣以成化,(成道德之創造或道德之實事),以見性之為性也。性不是抽象地掛在那裡,乃是必起用以成道德行為之實事。合兩以成化,而不偏滯于一隅,即其不容已之創造之「不已」也。偏滯則窒息而「已」矣。〈太和篇〉所謂「聖人盡道其間,兼體而不累者,存神其至矣」是也。此處之「合兩」,即彼處之「兼體」。合兩而不偏滯,

\newpage\thispagestyle{empty}\addtocounter{page}{-1}\vspace*{-12mm}\begin{center}\noindent
\includegraphics[clip, trim=168pt 155pt 151pt 249pt, height=162mm]{ocr-input/image-2124.png}\end{center}

\newpage

\noindent 即「兼體而不累」。所以能兼體而不累者,正由于虛而神也。「所存者神」,故能「所過者化」。故「性其總,合兩也」,此非是說性由合虛實、或合動靜、或合聚散、或合清濁之兩而成。如此,則成大拼湊,焉得謂為性?其意乃是說由總合貫通兩體而不偏滯以見性體寂感之神也。又,橫渠雖云「合虛與氣有性之名」,然此處之「合兩」亦不是「合虛與氣」之兩。此處之合兩唯是就氣之動靜聚散、虛實、清濁、陰陽、剛柔等之「兩」而言。故不能根據此處之「合兩」以解說「合虛與氣有性之名」一語也。由「合虛與氣」以說性之名之所以立,此根本是滯辭。此滯辭之所以成或由於不能劃分氣與神之故。橫渠說此語時,似以為「至靜無感,性之淵源」,雖無感,而並非無感之性能,而感之性能即氣也,神即氣之質性也,故云「合虛與氣有性之名」矣。若如此,則與其思理之實義相違,故不可以如此解。如前第一節末段所剖示,太虛神體根本不可以氣言;鬼神之神亦不可與太虛神體混;「太虛無形,氣之本體」,是氣之本體,並非氣之質性;「太虛即氣」是體用圓融義,並非說太虛是氣之質性;「清通而不可象為神」,清通之神亦不可視作氣之質性,神之清通與清氣之清通不可混;太虛是氣之本體即等於說神是氣之本體,太虛神體是同意語之一詞,不能將神混作氣。然則性之名只能超越分解地偏就太虛神體之體萬物而建立,不能由「合虛與氣」而建立。由「合虛與氣」而建立,則性適成一混雜體或組合體,而此正非性。朱子說:「合虛與氣有性之名,有這氣,道理便隨在裡面。無此氣,則道理無安頓處。」此只是說理氣之關係,並未說得著「性之名」之所以立。若如此說,性是理氣之合乎?此亦與朱子自說相違也。故橫渠此語決是不諦之晦辭。若云

\newpage\thispagestyle{empty}\addtocounter{page}{-1}\vspace*{-12mm}\begin{center}\noindent
\includegraphics[clip, trim=158pt 124pt 131pt 252pt, height=162mm]{ocr-input/image-2128.png}\end{center}

\newpage\markright{第二部 \quad 分論一 \quad 第二章 \quad 張橫渠對於「天道性命相貫通」之展示}

\noindent 合太虛神體與氣之聚散動靜等而一之以見性體之真實義與創生妙用義,則可。然此正是另一義,(即此處合兩而不偏,兼體而不累以見「性體之神」義),而非「性之名」之所以立也。此如前說。(橫渠或即是根據「天本參和不偏」、「道則兼體無累」、「性其總,合兩也」諸語之義而說「合虛與氣有性之名」。若如此解,則無過。但「性之名」之所以立與性之實之由參和不偏、兼體無累見,並非同義。)

至於「命其受,有則也」等語,此處亦須捎帶一講。命即天之所命或性之所命。有命即有受。自吾稟受此命而至之言,則有定則而不可移。故曰:「命,其受,有則也。」至吾所受之命之分乃由於「極總之要」。極總之要即是盡性之極。性之實以合兩之「總」為要,故盡性之極即是極總之要也。盡性之極不是抽象地單顯此性體之純普遍性之自己,乃是具體地盡之於「兼體無累」之中。此若宇宙論地說,是盡之於氣化之中以成宇宙之生化。若道德實踐地說,則是盡之於剛柔、清濁之中而不偏滯,以成道德之實事,即,成道德之創造,道德行為之純亦不已。能如此盡,則真可以「至受命之分」矣。「分」者定也,亦即本分之定也,此是性分之所定也。道德創造中一切道德行為皆是天之所命、性之所命,皆是必然的義務而責無旁貸者,吾必須承受而致至之,此即是吾人之大分也,故曰「不極總之要,則不至受命之分。」此由「盡性窮理」,以至於命之分,「而不可變者」,便是吾之道德生命之極則。故曰:「盡性窮理,而不可變,乃吾則也。」〔〈說卦傳〉曰:「窮理盡性以至於命。」照橫渠此處所說,窮理是道德實踐地窮盡性分中之理,命之分即是性分中之理之所定。能窮盡這些理而使之有具

\newpage\thispagestyle{empty}\addtocounter{page}{-1}\vspace*{-12mm}\begin{center}\noindent
\includegraphics[clip, trim=177pt 173pt 140pt 230pt, height=162mm]{ocr-input/image-2132.png}\end{center}

\newpage

\noindent 體的呈現便是盡性,故窮理即盡性,窮理盡性即至於命。程明道曰:「窮理盡性以至於命,三事一時並了。」伊川亦謂三事「只是一事」。照橫渠此處所說,亦是「三事一時並了」。若是「三事一時並了」,則窮理決不是認識地窮究外物之理。窮是「窮盡」之窮,而不是「窮究」之窮。但橫渠有時亦表示不是「三事一時並了」。當明道與伊川說此語時,他即辯駁說,此亦是太快,其間煞有事作。(見下〈明道章)、(一本篇〉及〈伊川章·格物窮理篇〉)。如是,則三事可拉成三個階段。若照此處所說,「至受命之分」即是至性分之所命,則至命與盡性內在地勾聯而為一,「窮理」即不可能單獨地岔出去而獨行。是以「窮理」即窮盡性分之理(普遍律則),「至命」即至這些理所定之分,此三辭語為同一意義,則三事即是一事,並不可拉成三個階段。若窮理、盡性、至命各有不同之指向,則很可以不是「三事一時並了」。窮理盡性之異解尤有關鍵性之影響。當橫渠辨駁二程之說時,其視窮理為知之事,其所意謂之「至命」為「至於天道」,「命」當是「天命之謂性」之命,不是此處所說性之命、性之分之「命」。但此並不要緊。如果窮理之知沒有嚴重的異解,只是次序上之先了解,則命無論為性之命或天命之命皆不影響「三事一時並了」。因為無論性之命或天命之命皆是以理言的命令之命,不過一是從性處向前看,一是從性處向後看。理、性、命三詞之內容的意義固完全相同也。窮理即是窮究的這個性命之理,盡性即是盡的「天所性者通極於道」這個性,至命即是至的「天所命者通極於性」這個命。如是,窮理、盡性、至命,雖可布散開說,亦不礙其本質上之為一事而可以「一時並了」。但至命之命亦可是以氣言的命遇、命運、命限之

\newpage\thispagestyle{empty}\addtocounter{page}{-1}\vspace*{-12mm}\begin{center}\noindent
\includegraphics[clip, trim=159pt 145pt 141pt 239pt, height=162mm]{ocr-input/image-2136.png}\end{center}

\newpage\markright{第二部 \quad 分論一 \quad 第二章 \quad 張橫渠對於「天道性命相貫通」之展示}

\noindent 命。此是就從盡性處向前看所遭遇的限制說。如是,至命雖與窮理盡性不是一事,但亦可以在工夫之函蘊上「一時並了」。蓋如此言之之「至命」即是「俟命」,亦普通所謂盡人事以聽天命之「聽命」。此至字、俟字、聽字並無工夫可言,故若真窮得理,盡得性,即自然至於命也。故至命無論是至性之命或至天命之命,或是至以氣言的命限之命,只要窮理之知無嚴重之異解,當不礙三事之「一時並了」。吾想横渠之思理即是如此。故其辨駁二程恐只是一時之不澈,其漸教之態度不必真有礙於「一時並了」之頓教。惟想到伊川亦言三事「只是一事」,則卻有問題。或只是隨其老兄如此說而已。若衡之其言格物窮理以致知之義,則彼實不能說「只是一事」。關此,將詳解於〈伊川章〉第八節。關於明道者,將詳解於〈明道章〉第四節。詳參該兩處,則三人之意可得而明。此處只就橫渠(誠明篇〉簡註於此,暫不詳說。]

此合兩不偏以見性體之神,〈乾稱篇〉亦言之,如:

\begin{quotation}\kaishu 無所不感者虛也。感即合也、咸也。以萬物本一,故一能合
異。以其能合異。故謂之感。若非有異,則無合。天性,乾
坤陰陽也。二端故有感,本一故能合。天地生萬物,所受雖
不同,皆無須臾之不感,所謂性即天道也。\end{quotation}

\noindent 案:「無所不感」,即「感而遂通」之義。即由此見虛見神。「無所不感」之通即能合散殊之異而為一。故由感見虛見神,即由感見妙一也。此即萬物之體也。體一,故「萬物本一」。以體一,故能妙合散殊之異。散殊之異是氣化之所形也,是屬於氣邊事。體一故

\newpage\thispagestyle{empty}\addtocounter{page}{-1}\vspace*{-12mm}\begin{center}\noindent
\includegraphics[clip, trim=169pt 145pt 132pt 245pt, height=162mm]{ocr-input/image-2140.png}\end{center}

\newpage

\noindent 妙合,亦由異而見一。若無異,則一只是抽象的一,而非具體的妙合之一,亦非由感通而見之一。故性體神感神應之一是在乾坤陰陽兩端中見。有氣形之兩,如動靜、聚散、升降、出入等,始能顯出性體之具體的妙感。故云:「天性,乾坤陰陽也。」此不是由氣說性,乃是由氣見性。此不是說陰陽之氣之結聚為性,乃是不離陰陽之兩而見性體妙合之一。氣之兩端相感相應而有局限,則是客感、物感、氣感,而不是神感。無所不通為神感。一感即通全體,故神也。即通全體即是一,故云:「二端故有感,本一故能合。」又云:「以其能合異,故謂之感。」此「合異」不是局限範圍內的合異,是本一的合異,是通全體的合異,故表示此通全體的合異之感是超越的神感,故其合亦是超越的妙合,非物感氣合之有封限也。天地萬物皆在一神感妙合之中呈現,此即是性體之妙通,亦即是性體之創生也。自此而言,則「性即天道」。天道本虛以成用,性體亦如此也。天道是綜說,故「合虛與氣」可以適用於道之名之所以立,而性體是偏說,故不可再說「合虛與氣有性之名。」言性所以立創造之體。承體以起用,即率性之謂道。帶著起用之創造說,自亦可說「性即天道」。離此性體之起用亦別無另一天道可言也。故「性外無道」也。此程明道所謂「只此便是天地之化。不可對此個,別有天地之化。」(自天道結穴於性言,亦可以說「道外無性。」)但此不是性體之名之所以立。故「合虛與氣有性之名」一語為不諦也。

〈乾称篇〉又云:

\begin{quotation}\kaishu 有無虛實通為一物者性也。不能為一,非盡性也。飲食男女\end{quotation}

\newpage\thispagestyle{empty}\addtocounter{page}{-1}\vspace*{-12mm}\begin{center}\noindent
\includegraphics[clip, trim=158pt 122pt 137pt 254pt, height=162mm]{ocr-input/image-2144.png}\end{center}

\newpage\markright{第二部 \quad 分論一 \quad 第二章 \quad 張橫渠對於「天道性命相貫通」之展示}

\begin{quotation}\kaishu 皆性也。是烏可滅?然則有無皆性也。是豈無對?莊、老、
浮屠爲此說久矣。果暢真理乎?\end{quotation}

\noindent 案:此亦合兩不偏、兼體無累之意。「有無、虛實,通為一物」,即表示此皆是性體之所貫。性不單指「虛」與「無」而言也。「有」與「實」亦皆是性分中所有事。飲食男女是也。此並不是說飲食男女之事是性,乃是說此是性分中所有事,此是性體所貫之實事,此是由性體而起之道德行為中之實事,在道德行為中被肯定。胡五峰曰:「夫婦之道,人醜之矣,以淫欲為事也。聖人則安之者,以保合為義也。」(《知言》中語)此即是在道德行為中被肯定之實義。是以盡性以成道德行為、以起道德創造,不能虛脫或幻滅此等事。是以性體必在合兩中見;盡性必「有無虛實通為一物」,始見道德之性體以及道德之性體之創造。必如此,其「盡」始是具體的盡,而性亦才是具體而真實的性。如此,方真是「盡性而至於命」。是以儒者之盡性非只停於「虛」與「無」而隔絕或幻滅「有」與「實」以為性或盡性也。性必合有無虛實之兩而不偏滯而見,而盡性亦必貫通有無虛實之兩而為一物方算是盡。

兩者即「對」也。缺其一,即不算是兩,亦不算是有對。缺其一而無兩無對,即不能表示妙合之通,即是性體之偏滯,即不是具體而真實的性體。「是豈無對?」此疑問句是表示:性豈是捨動靜、聚散、出入、虛實、有無之兩或對而不貫通,而只偏於虛與無一面而無對,以為性耶?光是虛一面,即不能貫通虛實而為一,光是無一面,即不能貫通有無而為一,是則此性即不是合兩不偏之性,亦不是兼體不累之性,而只是偏枯之性。偏枯之性適足以為累

\newpage\thispagestyle{empty}\addtocounter{page}{-1}\vspace*{-12mm}\begin{center}\noindent
\includegraphics[clip, trim=176pt 172pt 150pt 233pt, height=162mm]{ocr-input/image-2148.png}\end{center}

\newpage

\noindent 耳。是以「是豈無對」者,是說:豈是無兩體之對乎?若無兩體之對,則即不能由兩體之合、之總、之兼以見性體之妙合也。此不是說要有一個東西來與性體為相對也。橫渠此語,太混略不達。若不貫通其合兩、兼體之說而觀之,而單看此段,則讀至此句,必覺不通矣。

至於下句「莊、老、浮屠為此說久矣」,是說佛、老皆偏於虛與無以為性,(老以無為體,佛以空為性),而不能貫通虛實、有無而為一物以見性也。此評當否,且不管。要之,佛、老之性或體不是起道德創造之性,則無疑。不能起道德創造,則對於有與實(氣化之實,世間萬象之實,此亦即是有),即不能有真實之肯定,此亦無疑。道家只是順應、應跡。而佛家則只是性空、幻化。雖說「實際理地不受一塵,佛事門中不捨一法」,然幻妄畢竟是幻妄,畢竟不能算是立體地、真實地貫通而為一,而肯定其為實事。依儒者觀之,此即不能算是真理之暢達,故云:「果暢真理乎?」此仍是偏枯之教,而不能算是「所存者神,所過者化」、兼體而無累之圓盈之教。佛家圓教,無論如何圓,亦仍是如此。參看附錄:〈佛家體用義之衡定〉。

又,橫渠此段說「有無虛實通為一物」為性,為盡性,固意在遮撥佛、老之偏枯,以虛與無意指佛、老所說之空與無,然在橫渠自義,則虛與實對言為兩體,無與有對言為兩體,此虛與無即不是其所說之太虛神體,而是氣散為虛為無,氣聚為有為實。虛與實,有與無皆是氣化之客形。氣散為虛為無,只是形無形虛,非任何都無,而此時正恰恰得見吾之太虛神體。故〈太和篇〉云:「散入無形,適得吾體。」氣聚為有為實,而太虛神體之常即在其中而為之

\newpage\thispagestyle{empty}\addtocounter{page}{-1}\vspace*{-12mm}\begin{center}\noindent
\includegraphics[clip, trim=162pt 143pt 137pt 238pt, height=162mm]{ocr-input/image-2152.png}\end{center}

\newpage\markright{第二部 \quad 分論一 \quad 第二章 \quad 張橫渠對於「天道性命相貫通」之展示}

\noindent 主,故〈太和篇〉云:「聚為有象,不失吾常。」性不指虛與無言,而是就太虛神體言。就氣形之虛與無而得見太虛神體,而氣形之虛與無自身不是太虛神體。性是貫通氣化之虛實兩態、有無兩態而一之。盡性者如此,即為兼體而無累。此固天道性命通而為一以言「本體、宇宙論的」生化或道德創造這一型的義理,而老子之由無名無形而說無,佛家之由法無自性而說空,其反過來,無與有之關係,空與緣起法之關係,固與此不同也,故橫渠得以偏枯視之。(但謂老之無,佛之空即是其所說之「有與無」之無,「虛與實」之虚,亦非是。)

\subsection{性善命正:「義命合一存乎理」}

以上明性體之具體意義與具體內容,則性體之性能義與性分義亦可因而明矣。以下試再順此性能性分義而言性善命正,以及「義命合一存乎理」之義。

〈誠明篇〉云:

\begin{quotation}\kaishu 盡其性,能盡人物之性。至於命者,亦能至人物之命。莫不
性諸道,命諸天。我體物,未嘗遺。物體我,知其不遺也。
至於命,然後能成己成物,而不失其道。\end{quotation}

\noindent 案:此文可視為以上所說「盡性至命」義之總結。至誠以盡性,不息以至命。盡性不但盡己之性,亦同時即盡人物之性。至命不但至自己之命(性體所命之本分),亦同時即至人物之命,蓋「莫不性諸道,命諸天」也。即人物同一性體也。(「性者萬物之一源,非

\newpage\thispagestyle{empty}\addtocounter{page}{-1}\vspace*{-12mm}\begin{center}\noindent
\includegraphics[clip, trim=171pt 172pt 154pt 232pt, height=162mm]{ocr-input/image-2156.png}\end{center}

\newpage

\noindent 有我之得私。」)人物同一性體,故我之盡性,體物而未嘗遺,因而即盡人物之性、至人物之命,則自物方面說,物亦可體我而不遺也。但此須有辨。我盡人物之性,他人亦可盡人物之性。自亦可體我未嘗遺。但「物體我,知其不遺」,此只可本體論地、潛存地說是如此,蓋同一本體也。但不能實踐地、呈現地說是如此。明道云:「萬物皆備於我,不獨人耳,物皆然。都自這裡出去。只是物不能推,人能推之耳。」明道此語卻妥當。蓋能說到同體,又能照顧到推不推。我能盡性,故盡人物之性,至人物之命,體物而未嘗遺,既能本體論地說是如此,又能實踐地說是如此。「盡」之時義大矣哉!盡即能推。此是使實踐地呈現地體物不遺之所以為可能之關鍵也。然而物不能通過心覺活動以盡其性,即是不能推擴得去,而只囿封於其墮性或物質結構之性,作為同一本源之性體在它個體內根本沒有呈現,沒有起作用,是則只是本體論地、潛存地體我而不遺,實並未能實踐地、呈現地體我而不遺也。是則人物雖同體,而亦區以別矣。此所以立人極,而人極之義大矣哉!橫渠徒因「莫不性諸道,命諸天」,而即謂「我體物未嘗遺,物體我知其不遺也」,是未能洞察此中之差別也。

此段文可視為上說之義之綜結,故亦錄於此而附帶辨明之。以下言性之「善」與命之「正」。

〈誠明篇〉續上文而言曰:

\begin{quotation}\kaishu 性於人無不善,繫其善反不善反而已。過天地之化,不善反
者也。命於人無不正,繫其順與不順而已。行險以僥倖,不
順命者也。\end{quotation}

\newpage\thispagestyle{empty}\addtocounter{page}{-1}\vspace*{-12mm}\begin{center}\noindent
\includegraphics[clip, trim=167pt 147pt 136pt 238pt, height=162mm]{ocr-input/image-2160.png}\end{center}

\newpage\markright{第二部 \quad 分論一 \quad 第二章 \quad 張橫渠對於「天道性命相貫通」之展示}

\noindent 案:性體純然至善,人人所固有,只爭呈現不呈現耳。善反而復之,則呈現而起用。不能善反而復之,則潛隱而自存。所謂「呈現起用」,宇宙論地說,即成宇宙之生化(天地之化),實踐地言之,即成道德之創造,道德行為之純亦不已。「過天地之化,不善反者也。」(繫辭傳〉稱:「易與天地準,故能彌綸天地之道。」又云:「範圍天地之化而不過,曲成萬物而不遺。」彌綸即範圍曲成而不過不遺也。過則雖窮高極遠,有似於籠罩一切,而實不能範圍天地之化。不能範圍天地之化,即不能曲成萬物而不遺。「範圍」是超越地說,即是恰恰相應天地之化而模範出之耳;「曲成」是內在地說,即是具體地、分別地與物一一相應而成就之耳。雖超越而不過,雖內在而不溺。是以超越即內在,內在即超越,皆如如相應而不過不遺也。過即函遺。過而遺,則不能成天地之化。如是,性體為虛脫,萬物為幻妄,不能見性體為宇宙生化或道德創造之性體,故亦不能見其為道德地善,此則不善反者也。故曰:「過天地之化,不善反者也。」反之善不善,以能否成道德創造而決定。

至於性之所命皆是吾人之本分,故「命於人無不正」,只爭「順與不順而已」。不順本分而行,則為不正。「行險以僥倖」,即是不順性之所命者也。案此可能有異解。

順前第四段所引「性其總,合兩也。命其受,有則也」之文而說,窮理盡性至命是一事,此命即是性之所命之命。此處「命於人無不正」,是承上文「盡其性能盡人物之性,至於命者亦能至人物之命」一段文而言,故此處之命亦當是性之所命之命。前引〈誠明篇〉「天之所命通極於性,遇之吉凶不足以戕之」,亦是性之所命

\newpage\thispagestyle{empty}\addtocounter{page}{-1}\vspace*{-12mm}\begin{center}\noindent
\includegraphics[clip, trim=173pt 152pt 152pt 249pt, height=162mm]{ocr-input/image-2164.png}\end{center}

\newpage

\noindent 之命。此是命之積極的意義。〈誠明篇〉:「天所自不能已者謂命,不能無感者謂性」,此亦是積極意義的命。大抵《正蒙》各篇主旨在陳「本體、宇宙論的」立體直貫之創造,故主要以此積極意義的命為主。故吾於此處「命於人無不正」,亦以此義解之。而橫渠恐亦即是說此義。〈誠明篇〉開首一段中有數語甚精,如:

\begin{quotation}\kaishu 義命合一存乎理,仁智合一存乎聖,動靜合一存乎神,陰陽
合一存乎道,性與天道合一存乎誠。\end{quotation}

\noindent 案:此皆從正面說。「義命合一存乎理」中之命亦是從正面說之命,所謂積極意義之命也。義是性分之當然,命即是性分當然之不容已,此皆以理言也。

但此處「命於人無不正,繫其順與不順而已」,言順言正,顯本孟子而來。而孟子所說之命卻正是命運之命、遭遇之命,如死生、歼壽、吉凶、禍福之類是。此卻是橫渠所謂以氣言之命。孟子〈盡心〉篇云:「歼壽不贰,修身以俟之,所以立命也。」又繼之曰:「莫非命也,順受其正。是故知命者,不立乎巖牆之下。盡其道而死者,正命也。桎梏死者,非正命也。」孟子在此明說「順」與「正」。但此「順受其正」,卻須關聯著「盡其道」而顯,而所顯者正是「遇之吉凶」之命,而非道本身之所命之命也。此是消極意義之命。孤立地去看「命於人無不正」兩語,以此消極意義之命解之,亦可。但就〈誠明篇〉上下文貫通看,則恐不是說此義。下文復有「性命於氣」,「性命於德」,「性天德,命天理」之文,(見下第八段)。凡性命連言,皆是積極意義之命。下文復正式檢

\newpage\thispagestyle{empty}\addtocounter{page}{-1}\vspace*{-12mm}\begin{center}\noindent
\includegraphics[clip, trim=167pt 131pt 134pt 252pt, height=162mm]{ocr-input/image-2168.png}\end{center}

\newpage\markright{第二部 \quad 分論一 \quad 第二章 \quad 張橫渠對於「天道性命相貫通」之展示}

\noindent 別出「死生修夭」之命,(亦見下第八段),則此處以積極意義之命解之,當無不可。

\subsection{天地之性與氣質之性}

〈誠明篇〉續上復云:

\begin{quotation}\kaishu 形而後有氣質之性。善反之,則天地之性存焉。故氣質之
性,君子有弗性者焉。\end{quotation}

\noindent 案:性體粹然至善,人人所固有,何以有待於反(復)?曰:正因其有呈現與不呈現耳。性體妙運物而為之體,何以有不呈現之時?曰:宇宙論地言之,無不呈現之時,而自人之道德實踐而言之,則有不呈現之時。蓋人受形體之限,不能不有氣質之偏。性體之不能呈現,或時有微露而不能盡現,皆氣質之偏限之也。如是,遂有「氣質之性」之一義。

氣質之性是形體以後事。「氣質之性」與「天地之性」之分亦始于横渠。濂溪未曾言及。「天地之性」意即天地之化所以然(超越的動態的所以然)之超越而普遍的性能。言天地之性者,承「性者萬物之一源,非有我之得私」而言,是極言其超越的普遍性。後來程、朱亦名為「義理之性」,此後學者大抵沿用之,而「天地之性」之名遂不被常用。言道德實踐,不能抹殺此分別。氣質之性是在道德實踐中,由於性體之不能暢通起用,而被肯定。性體雖以易知、以簡能,然而未嘗無險阻也。是以〈繫辭傳〉云:「夫乾,天下之至健也,德行恆易以知險。夫坤,天下之至順也,德行恆簡以

\newpage\thispagestyle{empty}\addtocounter{page}{-1}\vspace*{-12mm}\begin{center}\noindent
\includegraphics[clip, trim=160pt 155pt 140pt 232pt, height=162mm]{ocr-input/image-2172.png}\end{center}

\newpage

\noindent 知阻。」宇宙論地言之之乾坤知能,即是實踐地言之之性體知能也。性體知能之險阻即氣質之偏與雜是也。〔性體之知即孟子所謂良知,性體之能即孟子所謂良能,亦即「非才之罪」,「不能盡其才」,「非天之降才爾殊」諸語中之才。此才非普通才能之才,乃性體良能之才,是道德意義的,而且是普遍的,是單指實現良知之所覺發者而言。性體之知能,本體宇宙論地說,即是虛明照鑑之神之明:神之明即是知,而神之妙、通,即是能,知能俱從神說。若是道德實踐地說,知能即是本心。心知之,即是能之。「孩提之童無不知愛其親,及其長也,無不知敬其兄。」知愛其親即是能愛其親,知敬其兄即是能敬其兄。知能俱從道德的本心說。此本心即是吾人道德創造所以可能之先天根據(先天而固有之性能)。故心體之知能即性體之知能。此即是說心體性體先天地知愛知敬、知是知非、知善知惡,知以為則,而亦先天地當然而不容已、定然而不可移之自然地能表現、呈現出此知也。因此,遂有道德之創造,道德行為之「純亦不已」。知能,在〈繫辭傳〉分自乾坤言之。若在此亦必類比地分自乾坤言之,則神之明即是健德;神之妙、通,即是順德。俱自德言,無陰陽之氣之實義也。從道德本心說,心之良知即是乾健,心之良能即是坤順,亦皆自德言,無陰陽之氣之實義也。若神體之乾健坤順必須在氣化中表現,以成生化之實,則須落在氣上說,須要有氣之觀念,而神非氣也。(體用圓融地說,則全神是氣,全氣是神。)若本心之乾健坤順必須要在四肢百體之運動中以成活動之實,所謂「踐形」,則須落在氣上說,亦須要有氣之觀念,而本心之知能非氣也。(體用圓融地說,則全心是形,全形是心,孟子所謂醉面盎背,施予四體,不言而喩也)。

\newpage\thispagestyle{empty}\addtocounter{page}{-1}\vspace*{-12mm}\begin{center}\noindent
\includegraphics[clip, trim=162pt 129pt 123pt 242pt, height=162mm]{ocr-input/image-2176.png}\end{center}

\newpage\markright{第二部 \quad 分論一 \quad 第二章 \quad 張橫渠對於「天道性命相貫通」之展示}

氣質之性,依橫渠說詞之意,是就人的氣質之偏或雜,即氣質之特殊性,而說一種性。在中國思想傳統中,自「生之謂性」一路下來而說的氣性、才性之類,都是說的這種性,宋儒即綜括之於氣質之性。西方所說的人性(人的自然)亦即是這種性。康德所說之性脾、性好、性向、人性之特殊構造、人之特殊的自然特徵等,亦是指的這種性。但這種性實在是形而下的,實只是心理、生理、生物三串現象之結聚,總之,亦只是「生之謂性」、「性者生也」兩語之所示。在此,是建立不起真正的道德行為的,是開不出道德創造之源的。正宗儒家,如孟子所說之性,《中庸》「天命之謂性」,是想由「生之謂性」、「性者生也」,推進一步,就真正的道德行為之建立,而開出道德創造之源之性。此種性是道德創造之源,同時亦是宇宙創造之源,是絕對地普遍的,是超越的,亦是形而上的。故性直通天命、天道而為一。宋儒承之,以此爲正性。濂溪開端,對於天道、太極、誠體,有積極之體悟,而對於此種性則未能正視而明之,其言天道、太極、誠體等,未能自覺地結穴於此種性而貫之,而只言「性者剛柔善惡中而已矣」,此則未免只落於氣質上說性。此是一時之不覺,非謂濂溪必不承認此通天道誠體之絕對普遍之性也。至橫渠,則十分能正視性命天道之貫通,而結穴於此種性,而謂「性者萬物之一源,非有我之得私」,直云此種性為「天地之性」,即天地之化之淵源也。後來皆承之而不能悖。此種性是萬物之一源,絕對之普遍,則自與氣性、才性、性脾、性好、性向、人性之特殊構造、人之特殊的自然徵象之性不同。而此後者又不能隨便忽視與抹殺,故不得不就之而說一種性,此即「氣質之性」一名之所以立也。蓋人不是純靈,乃一組合體之有限存

\newpage\thispagestyle{empty}\addtocounter{page}{-1}\vspace*{-12mm}\begin{center}\noindent
\includegraphics[clip, trim=161pt 146pt 136pt 238pt, height=162mm]{ocr-input/image-2180.png}\end{center}

\newpage

\noindent 在。雖就道德創造以成聖言,必須肯定超越的「天地之性」為本體,但人亦是有形體的現實存在,故環繞其自然生命,又不能不有其自然生命一面之種種特徵與姿態,此即「人的自然」之性,所謂氣性、才性、氣質之性是也。天地之性是人的當然之性,是道德創造之性,是成聖之性,簡名曰聖性,亦猶佛家之言佛性。此是兩種性必分別建立之故。

然依後來朱子之解析,則似只承認有氣質之偏雜,而卻不甚能自覺地就氣質之偏雜說一種性,自然之性,卻是十分自覺地將「氣質之性」解說為氣質裡邊的性。性只是一義理之性,氣質之性即是此義理之性之在氣質裡邊濾過,故雜染了特殊的顏色,而不是那原來之性之純然、本然與全體。此亦可如此表示,即:氣質雖然偏雜,亦總可在這上頭表現一點義理之性,如氣質之剛者雖不必好,或甚至很不好,也總可在此透露了一點「義」,但卻不是屬於義理之性的那義性之純然、本然與全體。偏於柔者,乃至偏於緩或急,清或濁者,亦皆然。是則只有一性與氣質之對言,而不是兩種性之對言。當然說氣質與就氣質說一種自然之性,在觀念上並無多大差別。但如朱子之解「氣質之性」一詞,不合通常之語意,亦輕忽了「生之謂性」一路下來的氣性、才性等之獨立意義。如氣質之性解為氣質裡邊濾過的那義理之性,則「義理之性」又將如何解?豈不應一律解為義理裡邊濾過的那某某之性?實則兩「之」字皆是虛繫字。「氣質之性」即是就氣質之殊質而說一種自然之性。「義理之性」即是就道德理性而說一種道德創造之性。「天地之性」即是就天地之化而說一種宇宙生化或道德創造之性。義理之性並無所謂在義理裡邊濾過之性,天地之性更不能說在天地裡邊濾過之性。是

\newpage\thispagestyle{empty}\addtocounter{page}{-1}\vspace*{-12mm}\begin{center}\noindent
\includegraphics[clip, trim=166pt 133pt 134pt 248pt, height=162mm]{ocr-input/image-2184.png}\end{center}

\newpage\markright{第二部 \quad 分論一 \quad 第二章 \quad 張橫渠對於「天道性命相貫通」之展示}

\noindent 以性體受氣質或氣質之性之局限是一義,而不必即以此義解說「氣質之性」一詞也。橫渠設此詞之意是就氣質之殊而說一種性,此是通常之理解,本書從之。

氣質之性雖足以拘限或隱蔽天地之性,然「善反之,則天地之性存焉。」善反不善反,義如前定。在善反中,亦函變化氣質之工夫。儒家講天地之性唯是就道德的創造言,故只能以此性為本、為體、為絕對的標準。氣質之性雖有其獨立性,有其獨立之意義,成一套獨立之機括,然就道德實踐言,並不以此為準也。「故氣質之性,君子有弗性者焉。」「弗性」是不以之為本、為體、為準之意,並非不承認有此種性也。

氣質之性雖有獨立之意義,然總可化而從本。氣質之性一方可化,一方亦是一種限制。從可化言,「君子有弗性者焉」。從限制言,氣質之性是道德實踐中一種「限制原則」。限制原則是其消極意義。但氣質亦有積極的意義。即使化而使之中,中亦是一種氣質,所謂聖人中和之資是也。性體之具體表現不能離開個體生命之資質、氣質。即使氣質之粹然,全化而從性,性體亦要在氣質中流行表現。自此而言,氣質或氣質之性是一個表現原則,此是其積極的意義。然積極與消極兩用永遠同時常在,表現之即限制之。是以「天行健,君子以自強不息」,工夫不可以已也。雖聖人亦不能自已。故孟子說:「聖人之於天道也,命也,有性焉,君子不謂命也」。「命也」即示聖人之契合天道亦有限制與限定。否則,何以有耶穌之形態、孔子之形態、釋迦之形態?即就儒家之聖言,何以又有堯、舜與孔子之不同?但聖人之盡道不因有此限定而自推,只能遵性以自強不息,故雖「命也」,而「有性焉,君子不謂命

\newpage\thispagestyle{empty}\addtocounter{page}{-1}\vspace*{-12mm}\begin{center}\noindent
\includegraphics[clip, trim=165pt 147pt 130pt 237pt, height=162mm]{ocr-input/image-2188.png}\end{center}

\newpage

\noindent 也」。這個命是命運命限之命,不是「於穆不已」之天命之命、命令之命。乃是從氣質、資質說的命。收縮於個體生命上說,此種命是從氣質、資質上說。若從具有如此氣質、資質之個體生命之在天時、地理、人和中之遭遇言,此種命是命運之命。氣質、消極地說,既是一限制原則,則從此說的命亦是一限制原則。命之為限制原則,在口之於味、耳之於聲、目之於色、四肢之於安佚處,是積極的,當正視而重視之,不得因口體耳目之欲亦是性(自然之性,生理生物之性),而即可以不受限制而妄求。故在此,「性也,有命焉,君子不謂性也。」但在仁之於父子、義之於君臣、禮之於賓主、智之於賢者、聖人之於天道處,則是消極的,雖應正視,但卻不應過分重視此限制而自推,只應遵性(天地之性、道德創造之性)而盡性,以求衝破此限制,而期道德生命之通於無限。故在此則說:「命也,有性焉,君子不謂命也。」是則孟子明已暗示有兩種性:動物性與氣質之性為一邊,道德性之性與天地之性或義理之性為一邊。孟子說的是動物性與道德性相對立,尚未由動物性說到氣質之性。

以下言善反之工夫。

\subsection{化氣繼善以成性}

〈誠明篇〉續上復云:

\begin{quotation}\kaishu 人之剛柔緩急,有才與不才,氣之偏也。天本參和不偏。養
其氣,反之本而不偏,則盡性而天矣。

性未成,則善惡混。故亹亹而繼善者,斯為善矣。惡盡去,\end{quotation}

\newpage\thispagestyle{empty}\addtocounter{page}{-1}\vspace*{-12mm}\begin{center}\noindent
\includegraphics[clip, trim=166pt 139pt 141pt 248pt, height=162mm]{ocr-input/image-2192.png}\end{center}

\newpage\markright{第二部 \quad 分論一 \quad 第二章 \quad 張橫渠對於「天道性命相貫通」之展示}

\begin{quotation}\kaishu 則善因以亡。故舍曰善,而曰成之者性。\end{quotation}

\noindent 案:此言變化氣質與成性之道。「天本參和不偏」與(乾稱篇〉「道則兼體而無累」為同意話,亦與(誠明篇〉上文「性其總,合兩也」為同意語。天或道即是天德神體或太虛神體。而神之所以為神正因其能兼貫氣之聚散、動靜等之兩體而無累,故亦「本參和而不偏」。性體之總而合兩亦是如此。性體本應是總合氣質之剛柔緩急而不偏,而不為其所累,而見其流行之實、呈用之實,即具體而真實的性體之實,而不只是一分解地言的抽象的性體。「聖人盡道其間兼體而不累者存神其至矣。」((太和篇〉)盡道。落實言之,即是盡性。聖人能盡性以存神,故能兼體不累,參和不偏。然一般人則常不能如此,常受氣質之限制而為其所累。氣質有種種特殊之顏色而足以為性體呈現之險阻。是以人之道德實踐須有變化氣質之工夫;而在此段,橫渠便從孟子之「養氣」說。「養其氣」而化通之,使其不滯于一偏,即能反之性體呈用之本然而不偏矣。如是,則即可以說是「盡性而天矣」,即一如天之「本參和不偏」也。性體呈用參和不偏之本然亦天也,無孤懸之性體也。通過養氣之工夫,即可使此本然者如實呈現,此即是盡性。養氣工夫是柔化或通化氣質凝結之偏者,此固直接落在氣上而養之,而所以要養之,目的乃在性體呈現之暢通,固非無指向之徒然養氣者。是以養氣工夫,一方使氣質柔化而通化,一方亦即「反之本而不偏」,(反之性體呈用之本然而不偏),即所以盡性也。性體順適呈現之謂「盡」。該剛則剛,該柔則柔。性體順適呈現而無阻,氣質不滯而順體,則性體之流行即是「天行」,所謂「至誠不息」也,而不

\newpage\thispagestyle{empty}\addtocounter{page}{-1}\vspace*{-12mm}\begin{center}\noindent
\includegraphics[clip, trim=169pt 161pt 144pt 236pt, height=162mm]{ocr-input/image-2196.png}\end{center}

\newpage

\noindent 才者亦可轉化而為才矣。故曰:「盡性而天矣」。

依橫渠之解〈繫辭傳〉「繼之者善也,成之者性也」兩語而觀之,此「盡性而天」即是「成性」。性,本體論地言之,本是固有,本自現成而自存,何待吾人之「成之」?然自人之實踐而言之,則可以說此義。是以此「成」是工夫的成、彰著的成,不是存有的成,不是「本無今有」的成。但此「成性」須扣緊養氣或氣質之變化而反之性體呈用之本然而言。即,在化氣質之偏中逐步彰著地成其性。化「氣之偏」中不善的表現或惡的表現逐漸使其轉為不偏中之善的表現。繼續不斷地成此善的表現,是謂「繼善」。如是,則性即為彰著地具體地全善矣。故曰「亹亹而繼善者,斯為善矣。」性本是自善,但這是本體論地言之。今由繼善而說善,是自化「氣之偏」而彰著地說。彰著之而使之成為具體的善,呈現的善,亦猶自彰著而說「成」也。自彰著而說「成性」,即自彰著而說其為具體的、呈現的純然至善也。

「亹亹」,勤勉義、自強不息義。及至無一毫之偏,則即無惡或不善的表現,而全為善的表現。從表現上說,善惡相對而施設。及至無惡而全善,則「惡盡去」,惡之名不立。惡之名不立,則亦無所謂善,而「善因以亡」。「善因以亡」者,善之名亦可不立也,非謂無善之實也。善之名可不立,故在此即「舍曰善,而曰成之者性」。「舍曰善」即「不曰善」之意。舍者舍去也。在此,即不必說「善」,而只說「成之者性」。此即是說,在化「氣之偏」中而不斷地繼其善之表現以彰著地成其性,即在此「成」處說性,故曰「成之者性」。「成」是對本體論地說的性體之本然自存而言。不經過化氣之工夫以盡性,則性只是本體論地本然自存。故

\newpage\thispagestyle{empty}\addtocounter{page}{-1}\vspace*{-12mm}\begin{center}\noindent
\includegraphics[clip, trim=160pt 143pt 143pt 241pt, height=162mm]{ocr-input/image-2200.png}\end{center}

\newpage\markright{第二部 \quad 分論一 \quad 第二章 \quad 張橫渠對於「天道性命相貫通」之展示}

\noindent 成性即是盡性,在盡之工夫中以成之。經過化氣之工夫以盡而成之,則性始有具體的呈現,而全幅彰著焉。故成是具體呈現的成、彰著的成,非「本無今有」之成也。「成之者性也」,依橫渠此解,即是從彰著的「成」上、「盡」上說具體呈現之性也。

此解自不合〈繫辭傳〉之原意。原文「一陰一陽之謂道,繼之者善也,成之者性也」,「之」字皆代表道。言能繼此道而不斷絕者即叫做是善,能完成或成就此道者則是性體之能也。能充分盡此性體,則道即具體地重見於此矣。此仍是「率性之謂道」之意。「一陰一陽之謂道」是宇宙論地說,「率性之謂道」是個體道德實踐地說。性以成道亦是個體之道德實踐地說。故下文即繼之曰:「仁者見之謂之仁,智者見之謂之智,百姓日用而不知,故君子之道鮮矣。」此皆未能充分盡其性以成道也。「百姓日用而不知」,固未能盡,即仁者、智者亦是偏於一隅,未能盡性之全體以成道也。「仁者」盡性之仁一面,即據此面以為道即是仁,「智者」盡性之智一面,即據此面以為道即是智,實則皆非「君子之道」之全體。此是〈繫辭傳〉此數語之原意。橫渠卻解成「成性」,如是則不是「成道」矣。朱子注云:「成,言其具也。性,謂物之所受。言物生則有性,而各具斯道也。」此則只言性具斯道,而不是性以成道,亦失原意。

橫渠之解雖非〈繫辭傳〉之原意,然亦自成義理,亦自可說之義理·橫渠此段文之語意實表示此義理,但其措辭確有不諦處,亦確有隱晦處。如首句「性未成,則善惡混」,即是不諦之辭。「性未成」是對下文之「成」言。(由此可見所成的是性而不是道,甚顯。)此性自指「天地之性」言。成是彰著的成,非「本無今

\newpage\thispagestyle{empty}\addtocounter{page}{-1}\vspace*{-12mm}\begin{center}\noindent
\includegraphics[clip, trim=169pt 151pt 140pt 243pt, height=162mm]{ocr-input/image-2204.png}\end{center}

\newpage

\noindent 有」之成,則彰著地「成之」之後,固是具體的全善,而未成前,亦非可云「善惡混」也。此只能說是存有論地本然自存之善,而不能說是「善惡混」。此語太糊塗,未透澈也。若說在「氣之偏」未化以前,性體之表現可善可惡,或「善惡混」,即可。(橫渠恐即是此意,惟措辭未能善達。)若說「性未成,則善惡混」,直指性體自身如此說,則大不可。蓋如此,則性之善或至善之性乃是「本無今有」者。實則並非如此,橫渠亦不能如此說,彼亦不至是此意。是故此只是「存有論地本然自存」義與「彰著地成之」義之不同,而橫渠是在說後者,而後者不能抹殺「本然自存」義,而本然自存之性非「善惡混」也。否則何以說「天本參和不偏」?又何以說「反之本而不偏」?故「性未成,則善惡混」決是一時之糊塗語,至少亦是一時不審之辭。此性既必指「天地之性」言,決不會是「氣質之性」,則只能如此說:此性在未彰著地成之之時,其在氣質之偏中的表現是「善惡混」,或可善可惡,或有時善有時惡。然而決不能說性體自身是「善惡混」也。

橫渠雖有此一時不諦之滯辭,然其「成性」義卻甚顯明,亦自是可以成立之義。彼屢言此義,散見於《正蒙》之各篇以及〈經學理窟〉中,可見其對於此義之鄭重,決非一時偶爾之說也。茲類聚如下:

1.〈神化篇第四〉云:

\begin{quotation}\kaishu 1.1無我而後大,大成性而後聖。聖位天德不可致知謂神,
故神也者聖而不可知。

1.2神不可致思,存焉可也。化不可助長,順焉可也。存虛\end{quotation}

\newpage\thispagestyle{empty}\addtocounter{page}{-1}\vspace*{-12mm}\begin{center}\noindent
\includegraphics[clip, trim=156pt 125pt 137pt 253pt, height=162mm]{ocr-input/image-2208.png}\end{center}

\newpage\markright{第二部 \quad 分論一 \quad 第二章 \quad 張橫渠對於「天道性命相貫通」之展示}
明、久至德,順變化、達時中,仁之至義之盡也。知微
知彰,不舍而繼其善,然後可以成人性矣。

2.〈誠明篇第六〉:

\begin{quotation}\kaishu 2.1纖惡必除,善斯成性矣。察惡未盡,雖善必粗矣。

2.2不識不知,順帝之則。有思慮知識,則喪其天矣。君子
所性與天地同流異行而已焉。\end{quotation}

\noindent 案:此言「不識不知,順帝之則」是取超自覺義。此段雖未有「成性」之詞,然「不識不知,順帝之則」,「君子所性與天地同流異行」,即是「成性」之極致。

\begin{quotation}\kaishu 2.3莫非天也。陽明勝,則德性用。陰濁勝,則物欲行。領
惡而全好者,其必由學乎?\end{quotation}

\noindent 案:劉蕺山於此作案語云:「若領好以用惡,手勢更捷。然在學者分上,只得倒做。」(《宋元學案·橫渠學案上》)。好善惡惡,在蕺山是發自意體之獨知(知藏於意),在橫渠,是發自性體之神用。領有「惡惡之用」或以「惡惡之用」為領導(領惡)而全其「好善之用」(全好),則性成矣(彰著地成)。「成性」之義即藏於此「學」之工夫中。

3.〈中正篇第八:

\newpage\thispagestyle{empty}\addtocounter{page}{-1}\vspace*{-12mm}\begin{center}\noindent
\includegraphics[clip, trim=183pt 217pt 119pt 228pt, height=162mm]{ocr-input/image-2212.png}\end{center}

\newpage

\begin{quotation}\kaishu 3.1可欲之謂善,志仁則無惡也。誠善於心之謂信,充內形
外之謂美,塞乎天地之謂大,大能成性之謂聖,天地同
流、陰陽不測之謂神。

3.2君子之道,成身成性以為功者也。未至於聖,皆行而未
成之地爾。\end{quotation}

4.〈至當篇第九〉:

\begin{quotation}\kaishu 4.1愛人然後能保其身(寡助則親戚畔之)。能保其身,則
不擇地而安。(不能有其身,則資安處以置之)。不擇
地而安,蓋所達者大矣。大達於天,則成性成身矣。

4.2知及之而不以禮,性之非己有也。故知禮成性而道義
出,如天地位而易行。\end{quotation}

5.〈大易篇第十四〉:

\begin{quotation}\kaishu 5.1乾三、四位,過中重剛。庸言庸行,不足以濟之。雖大
人之盛,有所不安。外趨變化,內正性命。故其危其
疑,艱於見德者,時不得舍也。九五,大人化矣,天德
位矣,成性聖矣。故既曰利見大人,又曰聖人作而萬物
覩。亢龍以位畫為言。若聖人則不失其正,何亢之有?
5.2乾之九五日:飛龍在天,利見大人。乃大人造位天德,
成性躋聖者爾。若夫受命首出,則所性不存焉。故不曰
位乎君位!而曰位於天德;不曰大人君矣,而曰大人造\end{quotation}

\newpage\thispagestyle{empty}\addtocounter{page}{-1}\vspace*{-12mm}\begin{center}\noindent
\includegraphics[clip, trim=146pt 138pt 147pt 242pt, height=162mm]{ocr-input/image-2216.png}\end{center}

\newpage\markright{第二部 \quad 分論一 \quad 第二章 \quad 張橫渠對於「天道性命相貫通」之展示}
矣。

\begin{quotation}\kaishu 5.3成性,則躋聖而位天德。乾九二,正位於內卦之中,有
君德矣,而非上治也。九五言上治者,言乎天之德,聖
人之性。故舍曰君而謂之天,見大人德與位之皆造也。\end{quotation}

\noindent 案:「舍曰君而謂之天」即「不曰君而謂之天」,與「舍曰善而曰成之者性」為同一句法。

6.(經學理窟·氣質)章:

\begin{quotation}\kaishu 6.1人之氣質美惡與貴賤天壽之理皆是所受之分。如氣質惡
者,學即能移。今人所以多為氣所使,而不得為賢者,
蓋為不知學。古之人在鄉閭之中,其師長朋友日相教
訓,則自然賢者多。但學至於成性,則氣無由勝。孟子
謂氣壹則動志,動猶言移易。若志壹,亦能動氣。必學
至於如天,則能成性。((張子全書〉卷五,〈經學理窟〉
二)

6.2所謂勉勉者,謂繼之者善也。成之者性也,繼繼不已,
乃善而能至於成性也。〔…】(同上)\end{quotation}

\noindent 案:以上共十四條,皆言成性義。雖就「繼之者善也,成之者性也」說,不必合(繫辭傳〉之原意,然此表示工夫與學之「成性」義於義理自通,決無問題。而朱子不能解也。

《朱子語類》卷第九十九,〈張子之書二〉,有一條云:

\newpage\thispagestyle{empty}\addtocounter{page}{-1}\vspace*{-12mm}\begin{center}\noindent
\includegraphics[clip, trim=181pt 224pt 125pt 224pt, height=162mm]{ocr-input/image-2220.png}\end{center}

\newpage

\begin{quotation}\kaishu 或問:《正蒙》中說得有病處,謂是他命辭不出有差,還是
見得差?曰:他是見得差。如日:繼之者善也,方是善惡
混,云云。〔案:此引述不諦,「方」字尤不諦。】成之者
性,是到得聖人處方是成得性。所以說「知禮成性而道義
出」。似這處都見得差了。\end{quotation}

\noindent 案:橫渠只說「性未成,則善惡混」,並不說「繼善」是善惡混。勉勉繼善正是所以成性者。觀上錄6.2條可知。「性未成,則善惡混」,雖是滯辭,然由工夫以言彰著地成之之「成性」義則並無滯窒處。「知禮成性而道義出」(4.2條)尤為諦當顯明,何以說「都見得差了」?此正不差也,而朱子不能解何也?就「成之者性也」一句說,即朱子之解亦不必合(繫辭傳〉之原意。解成為具,是性之存有義,可類比「天命之謂性」;原意是「能成就此道者是性也」,可類比「率性之謂道」;而橫渠則是由工夫以彰著地去成就此性。此皆各是一義耳。單就其所說之「成性」義而說之斯可耳。

此「成性」義影響後來胡五峰甚大。胡氏《知言》即言盡心「以成性」、以「立天下之大本」。《知言》未提及横渠,然亦同樣未提及濂溪、明道。但不能說五峰未曾細讀濂溪、橫渠明道之書。其言「成性」顯是根據橫渠而來。橫渠之言「成性」,據以上所錄觀之,猶只從變化氣質與養氣說,尚未從「心」上說。然變化氣質與養氣之關鍵即在本心之呈現。孟子說浩然之氣「其為氣也,配義與道,無是餒也。是集義所生者,非義襲而取之也。行有不慊於心則餒也。」是即從本心之呈現與沛然不禦說「養浩然之氣」

\newpage\thispagestyle{empty}\addtocounter{page}{-1}\vspace*{-12mm}\begin{center}\noindent
\includegraphics[clip, trim=150pt 146pt 143pt 227pt, height=162mm]{ocr-input/image-2224.png}\end{center}

\newpage\markright{第二部 \quad 分論一 \quad 第二章 \quad 張橫渠對於「天道性命相貫通」之展示}

\noindent 也。橫渠說「養其氣,反之本而不偏,則盡性而天矣」,雖就氣質之偏說養氣,即養之而柔化通化其凝結之偏滯,尚不是說「養浩然之氣」,然養之而期其柔化與通化以收變化氣質之效,其本質的關鍵仍在本心之呈現。上錄6.1條,橫渠即引孟子「志壹則動氣,氣壹則動志」之語,而說:「動猶言移易。若志壹,亦能動氣。」志即心志,是即從心上說變化氣質也。心志呈現,能提得住,移易其「氣之偏」,柔化而通化之,此即橫渠所說之「學」之工夫。「必學而至於如天,則能成性」,是即無異於說盡心易氣以成性也。成性之關鍵最後仍落在心志之盡上。故〈誠明篇〉又有云:

\begin{quotation}\kaishu 心能盡性,人能宏道也。性不知檢其心,非道宏人也。\end{quotation}

\noindent 「盡性」即「成性」。「心能盡性」即是盡心以成性。心何以能盡性、成性?心之靈覺妙用、自主自律即足以形著性之實,性之實全在心處見。是故盡心即盡性,即成性。後來胡五峰即本此而言「心也者,知天地宰萬物以成性者也。六君子〔堯、舜、禹、湯、文王、仲尼】盡心者也,故能立天下之大本。」此言「成性」同于横渠,明是彰著地成。盡心以「立天下之大本」,「立」亦明是彰著地立,非「本無今有」之立也。最後劉蕺山亦盛言此義,此是屬於同一系之義理。朱子何以不能理解?蓋其說統中本不易接納此義理,故亦不易進入其意識中也。

蓋自濂溪言誠體、神體、寂感真幾,橫渠言天德神體、太虛神體、「天本參和不偏」、「道則兼體無累」、「性其總,合兩也」,以至明道言「於穆不已」之體,進之言易體、理體、神體、

\newpage\thispagestyle{empty}\addtocounter{page}{-1}\vspace*{-12mm}\begin{center}\noindent
\includegraphics[clip, trim=186pt 167pt 133pt 237pt, height=162mm]{ocr-input/image-2228.png}\end{center}

\newpage

\noindent 誠體、仁體、心體、敬禮、忠體,皆一。凡此種體悟道體性體者(體悟為即活動即存有者)皆非朱子所能相應。彼只本伊川「性即理也」一語之清楚割截而將道體性體乃至太極體會為只是理、只存有而不活動者,心義、神義、寂感義皆脫落而屬於氣,而又本「格物窮理」之順取之路以把握此道體性體甚至太極,此固不易接納「盡心以成性」之義,而亦不易進入其意識中也。伊川曾有「性之有形者謂之心」一語,此朦朧間亦函有一種形著義。但此「形著」義,若依伊川之說統,著實講出,只是心知之明之在「格物窮理」中形著之,亦只在心氣通過涵養後發為存在之然中形著之,此非橫渠、五峰所說之「盡心成性」義,亦無此「盡心成性」義之警策性、嚴肅性與真實性,故不易引起朱子之註目,而謂:「伊川有數語說心字皆分明,此一段卻難曉。不知有形二字合如何說?」蓋在伊川朱子之說統中,此「形著」義實無重要性,亦無本質性,實可有可無,故朱子無此意識也。(對於伊川此語詳解見〈伊川章〉)但在體悟道體性體為「即活動即存有」之系統中,「盡心成性」有其本質性與真實性,性體之實全在心中,全由心之靈覺妙用自主自律以形著之,而性之所以為性亦於焉以成以立,即成其為具體而真實的性,亦立其為具體而真實的性,而心性是一。凡自「於穆不已」之體言性者,自其回歸於《論》《孟》之仁與心言,皆必須言此義,性體始實(具體而真實),心性始一,而亦最易想及此義,此蓋理之必然者,故横渠、五峰、蕺山皆言之也。明道未言及者,以其盛言圓頓之一本,已跨過此義,而此義未嘗不隱含於其中也。濂溪未言者,以在初創,尚未涉及如是之廣也。伊川、朱子不言而朱子又不解者,以其系統之異也。陸王不言者以其純從孟子入,只

\newpage\thispagestyle{empty}\addtocounter{page}{-1}\vspace*{-12mm}\begin{center}\noindent
\includegraphics[clip, trim=149pt 155pt 147pt 225pt, height=162mm]{ocr-input/image-2232.png}\end{center}

\newpage\markright{第二部 \quad 分論一 \quad 第二章 \quad 張橫渠對於「天道性命相貫通」之展示}

\noindent 是一心之申展,不必言也。然承濂溪、横渠、明道之言道體性體者則必須言此義。此「盡心成性」義在此系統中之所以有本質性真實性與警策性,而五峰、蕺山之所以為北宋三家(濂溪、橫渠、明道)之嫡系也。朱子是伊川之嫡系,故不能解也。抑不只此而已也。

凡順「於穆不已」之體言性者(誠體神體具在內)皆視性體為一超越的、無善惡相的絕對至善之奧體、密體寂感真幾創造真幾、即活動即存有之真體。橫渠說:「性未成,則善惡混」,此固是滯辭,但其實意卻只應是意謂性體自身在未通過盡心易氣以彰著之之時只是自存的純然至善、無善惡相之真體自己,故及其通過亹疊(勉勉)繼善之盡心易氣之工夫時,即成為具體而真實的性體之全善,此即是「成性」。(「惡盡去,則善因以亡。故舍曰善,而日成之者性」。此時性即已彰著地成就或完成,雖不曰善,而實是具體而真實的、彰著的全善)。明道亦說:「人生而靜以上不容說,才說性,便已不是性也。」「不是性」是說有生以後與氣稟混雜同流,已不是那「於穆不已」之體自身之本然,故「凡人說性只是說繼之者善也」。此與橫渠意同,亦函說性體自身在未通過「繼之」之善時只是無善惡相之純然至善也。明道此義,朱子尚可以順著說,雖其所理解之性與明道所體會者不同。但至胡五峰言:「性也者天地鬼神之奥也,善不足以言之,況惡乎哉?」朱子即不耐矣。朱子於此力斥其為告子之「性無善惡」說。此顯然為誤解,亦非平情之論。蓋胡氏此義實與明道同也。至最後劉蕺山不管朱子之力斥胡氏,而仍說:「夫性無性也,況可以善惡言?」是故不言「明性」,而言「存性」。(「子貢曰:夫子之言性與天道不可得

\newpage\thispagestyle{empty}\addtocounter{page}{-1}\vspace*{-12mm}\begin{center}\noindent
\includegraphics[clip, trim=160pt 158pt 146pt 235pt, height=162mm]{ocr-input/image-2236.png}\end{center}

\newpage

\noindent 而聞也。則謂性本無性焉亦可。雖然,吾固將以存性也。」參看《劉子全書〉卷七,〈原旨〉七篇,〈原性〉。詳見〈胡五峰章〉)。彼如何「存性」?實以其所說之誠意慎獨以存之也。存之即明之,實以其所說之意知獨體澈盡「於穆不已」之性體以存而明之也。此仍是以心(意知獨體)著性、盡心成性之旨也。然則其反對言「明」者,只是反對外心以明之也。故曰:「夫性何物也,而可以明之?但恐明之盡,已非性之本然矣。為此說者,皆外心言性者也。外心言性,非徒病在性,並病在心。心與性兩病,而吾道始為天下裂。」(〈原性〉)此明是對朱子而發也。此雖未言「繼之者善」,然實與橫渠、明道、五峰之所說為同一義理,而尤近於五峰。凡此皆非朱子之所能解,必當為其所責斥矣。蓋彼對於「於穆不已」之體本無相應之契悟故也。然則橫渠之言「成性」焉可忽乎哉?

以上皆爲橫渠「成性」義之所函,吾故詳疏之如上。

\subsection{「以理言」之命與「以氣言」之命}

〈誠明篇〉續上復云:

\begin{quotation}\kaishu 德不勝氣,性命於氣;德勝其氣,性命於德。窮理盡性,則
性天德、命天理。氣之不可變者,獨死生修天而已。故言死
生,則曰有命,以言其氣也。語富貴,則曰在天,以言其理
也。此大德所以必受命,易簡理得而成位乎天地之中也。

所謂天理也者,能悅諸心,能通天下之志之理也。能使天下
悅且通,則天下必歸焉。不歸焉者,所乘所遇之不同,如仲\end{quotation}

\newpage\thispagestyle{empty}\addtocounter{page}{-1}\vspace*{-12mm}\begin{center}\noindent
\includegraphics[clip, trim=160pt 135pt 131pt 245pt, height=162mm]{ocr-input/image-2240.png}\end{center}

\newpage\markright{第二部 \quad 分論一 \quad 第二章 \quad 張橫渠對於「天道性命相貫通」之展示}

\begin{quotation}\kaishu 尼與繼世之君也。舜禹有天下而不與焉者,正謂天理馴致,
非氣稟當然,非志意所與也。必曰舜、禹云者,餘非乘勢,
則求焉者也。\end{quotation}

\noindent 案:此綜結「以理言」之命與「以氣言」之命。通過善反、成性之工夫,如果「德勝其氣」,則「性命於德」;如果「德不勝氣」,則「性命於氣」。「德」即德行之德,善反化氣之工夫即德行,亹繼善之實踐亦是德行。「性命於氣」與「性命於德」兩子句,初看「性」字是主詞,「命」字是動詞。揆之通常句法是如此,朱子開始亦作如此解,後經審思,乃以為「性命」應全作名詞解。但如此,則該兩子句無動詞,不合通常造句之習慣。橫渠蓋以為可略之耳。橫渠措辭多有別扭不通暢處,此亦其一例也。茲順朱子,性命全作名詞解。如此,此兩句之意是如此:如果吾人之德行不能勝其氣,則性與命全在氣(全是氣、全在氣上轉);如果吾人之德行能勝其氣,則性與命全在德(全是德、全在德上轉),德不勝氣,則一任氣質決定,孟子所謂「氣壹則動志」,此時吾人之性體與性體之所命全不能作主,亦即全不能呈現,性乃全轉而為氣質之性之橫決,命亦全轉而為氣性之定命,此即所謂性命全在氣、全是氣,或全在氣上轉(展轉流布)。如果德勝其氣,則化氣之偏而從理,則吾人之性體以及性體之所命皆能作主而朗現,此時性命之朗現即是德行之純亦不已,此即所謂性命全在德、全是德,或全在德上轉(展現流布)。故人經過養氣而化氣質之偏以至「窮理盡性」,則此時吾人之性即是天德,吾人之命即是天理,即是性德之所命與所「馴致」,即是吾人當然而不容已、必然而不可移之本分(必然的

\newpage\thispagestyle{empty}\addtocounter{page}{-1}\vspace*{-12mm}\begin{center}\noindent
\includegraphics[clip, trim=157pt 162pt 140pt 222pt, height=162mm]{ocr-input/image-2244.png}\end{center}

\newpage

\noindent 義務)與本分之所「馴致」。此時所剩之屬於氣而不可變者,「獨死生修夭而已」。(修當作壽,作修亦通。修長也,夭短折也。)此種命是以氣言。此是氣命之所定,乃命運命定之命,非命令本分之命。此不是德行所能改變者,但可以經由「盡其道」而至「順受其正」。

生死壽夭之命以氣言,此不錯。惟橫渠解「富貴在天」,則曰以理言,此不必諦。其如此說之根據是「大德必受命」,又根據「易簡理得而成位乎天地之中」而說。實則大德不必定受命,如孔子、釋迦、耶穌皆不曾受命,或可受命而不受。順古之聖王說,如堯、舜、禹、湯、文、武,自可說大德必受命,而理上實大德不必皆受命也。王充解「富貴在天」,是上關天星,此實仍是以氣言。王充完全自氣說性命,故有云:「用氣為性,性成命定。」性是氣性、才性,命是氣命。文王在母胎時即已受大命矣。此完全決定於生物學的先天之氣,即自然生命之強度自有其光華與富貴。堯、舜、禹、湯、文、武固亦有德,此自理想言之,歷來說其為大德,若從事實言之,雖有德,而不必為大德。至少其受命不必全決定於德,不必能全以理言。總之,大德不必能受命,受命者亦不皆有大德。此不純是理之事,畢竟英雄之氣分數多,此外還要加上所乘之勢與所遇之機。而內在地其個人生命之強度與外在地所乘之勢與所遇之機,皆是氣之事也。故「語富貴,則曰在天,以言其理也」,此解不必諦。「在天」不必一定偏於理說,亦可偏於氣說。偏於理說的天命、天道之生化與性體道德創造之純亦不已(至誠不息)為同一意義,故明道得云「只此便是天地之化,不可對此個,別有天地之化。」(「此」是指心性道德創造之沛然不禦、純亦不已

\newpage\thispagestyle{empty}\addtocounter{page}{-1}\vspace*{-12mm}\begin{center}\noindent
\includegraphics[clip, trim=166pt 145pt 127pt 235pt, height=162mm]{ocr-input/image-2248.png}\end{center}

\newpage\markright{第二部 \quad 分論一 \quad 第二章 \quad 張橫渠對於「天道性命相貫通」之展示}

\noindent 言。)而大人與天地合德亦只是合其偏於理說之「德」,此是超越的「意義」相同,「大而化之」之「化境」相同,甚至其「神」亦可說相同,而其個體生命之「氣」畢竟不能與天地之氣等量齊觀也。「氣之運化以現理」之「質」同,而量不同,其「無窮複雜」之質同,而無窮複雜之量不同(氣始可說「無窮複雜」)。即因有此不同,故個體生命之氣命與天地氣化之運行或歷史氣運之運行間始有一種遭遇上之距離與參差,因而有所乘之勢與所遇之機之不同。此則非我所能控制者。它超越乎我之個體生命以外與以上。此亦是天理中事、天命中事、天道中事,亦得簡言之曰天。此是天理、天命、天道之偏於氣化說,但亦為其神理所貫,全氣是神,全神是氣。全神是氣,則無限量之無窮複雜之氣固亦天理、天命、天道中事。就此說天理、天命、天道即是偏於氣說的天理、天命、天道,而此即對於吾個體生命有一種超越的限定,而吾個體生命對此超越限定言,即有一種遭遇上之距離與參差,因而有所乘之勢與所遇之機之不同,而此即形成吾之個體生命之命運與命遇,此即是以氣言之「氣命」。此亦是「在天」也。故「富貴在天」顯不能純以理言。凡孔子所說的「知天命」、「畏天命」、「不知命無以為君子」,以及有慨嘆意味的「天也」、「命也」等辭語,以及孟子所說的「所以立命也」,「莫非命也」,「性也,有命焉」,「命也,有性焉」,「求之有道,得之有命」等等辭語之命,皆是說的這種「命」。但是此種命雖以氣言,卻亦不能割掉它的神理之體。「氣命」之氣不是塊然的純然之氣,它是「全神是氣、全氣是神」中的氣。即因此,它對吾人所成之超越的限定始有一種莊嚴的嚴肅意義,所以才值得敬畏,而每一個體生命之遭遇乎此總不免有無限

\newpage\thispagestyle{empty}\addtocounter{page}{-1}\vspace*{-12mm}\begin{center}\noindent
\includegraphics[clip, trim=153pt 156pt 137pt 227pt, height=162mm]{ocr-input/image-2252.png}\end{center}

\newpage

\noindent 的慨嘆,雖聖人臨終亦不免嘆口氣(羅近溪語),因而「知命」、「知天命」才成為人生中一大關節。正面說的孔子之踐仁以知天,孟子之盡心知性以知天,其所知之天固首先是正面同于仁、同于心性之「以理言」的天,但決不止於此,亦必通著那不離其神理之體的無窮複雜之氣。此兩面渾而為一才是那全部的天之嚴肅意義與超越意義之所在。此不能純內在化。純內在化者是以理言的天,與性體意義同、質同、化境同的天。在此,只有性分之命,只有性之所命及其所馴致(自然而必然的結果)之命。此是「求則得之,舍則失之,是求有益於得也,是求之在我者也。」而「求之有道,得之有命,是求無盆於得也,是求之在外者也」,此命即是氣命。

至于《易·繫辭傳》「易簡而天下之理得矣,天下之理得,而成位乎天地之中矣」,此更不足為「富貴、受命」之根據。「成位乎天地之中」之「成位」不必是「受命」之位,亦不必有富貴。至《易傳》此數語之上文為「乾以易知,坤以簡能。易則易知,簡則易從。易知則有親,易從則有功。有親則可久,有功則可大。可久則賢人之德,可大則賢人之業。」此中所謂親、功、賢人之德、賢人之業,固皆從理言,如橫渠所謂「天理馴致,非氣稟當然,非志意所與」,然亦皆不必是受命之富貴也。《易傳》只承此而言「易簡而天下之理得矣,天下之理得,而成位乎其中矣。」由「易簡理得」而自可「成位」乎天地之中,但不必是受命之位。天理「能悅諸心,能通天下之志」,故能體現天理,自必有「歸焉」者。但此是泛說,不必是某一方式之歸,因而亦不必是表示「受命」之歸。如孔子能體現天理,故有三千弟子歸之,此亦是歸,亦是成位乎天地之中,但卻是成其為聖之位,並未成其為王之位。是以盡性起用

\newpage\thispagestyle{empty}\addtocounter{page}{-1}\vspace*{-12mm}\begin{center}\noindent
\includegraphics[clip, trim=164pt 135pt 124pt 240pt, height=162mm]{ocr-input/image-2256.png}\end{center}

\newpage\markright{第二部 \quad 分論一 \quad 第二章 \quad 張橫渠對於「天道性命相貫通」之展示}

\noindent 自必有其馴致之歸結,但自然而必然之歸結是性分之所定,是求之在我者,此可說是分析者,而不是綜和者。至於富貴、受命之得不得,則是綜和者,而不是分析者,此中即有氣命存焉。橫渠謂:「不歸焉者,所乘所遇之不同,如仲尼與繼世之君也。」有「所乘所遇」之不同,即有一種「求無益於得」之命存在。所乘之勢與所遇之機乃氣命之事,非理命之事。有無勢可乘,有無機可遇,即有之,吾能不能乘,吾能不能遇,皆有命存焉,非可強而致也。此中即有一種超越之限定。此統於神、理而偏於氣說之氣命所成之超越限定亦是「天」也,亦是天理、天命、天道如此也。某甲有勢而能乘,某乙無勢可乘,或即有之,而不能乘,機之遇不遇亦然,此皆儼若天命之。落實說,是勢、是遇、是氣命;統於神、理說,是天命。即在此統於神、理而偏於氣說之氣命上,把個體生命與天理、天命、天道拉遠了,而顯出天之氣運之不可測與天之氣化之無窮盡。故「大人與天地合德,與日月合明,與鬼神合吉凶,先天而天弗違」,是以理說。此時天全部內在化,吾之性體即是天,天地亦不能違背此性體。此時天與人不但拉近,而且根本是同一,同一於性體。此是「開物成務,冒天下之道」者也。而橫渠亦云:「知性知天,則陰陽鬼神皆吾分內爾。」(前第二段)此時即無氣命可說,而只有承體起用本體之直貫。但「後天而奉天時」,則是以氣說。雖大人亦不能不「奉天時」。天運如此,不可違也。故君子重「知命」,而亦即在此有「所以立命」之義也。此時,天與人即拉遠。此先天後天兩義,即孟子「盡心知性知天」、「存心養性事天」、「歼壽不貳,修身以俟,所以立命」之三義。「盡心知性知天」是先天義,後兩義是後天義。依先天義,保持道德創造之無

\newpage\thispagestyle{empty}\addtocounter{page}{-1}\vspace*{-12mm}\begin{center}\noindent
\includegraphics[clip, trim=158pt 146pt 131pt 234pt, height=162mm]{ocr-input/image-2260.png}\end{center}

\newpage

\noindent 外;依後天義,保持宗教情操之敬畏。依先天義,保持道德我之無限性;依後天義,保持我之個體存在之有限性。此兩義同時完成於儒家之「道德的形上學」中,而儒家之充其極的「道德的形上學」即完全同一於「道德的神學」,外此並無其他「道德的神學」之可三

\noindent 言。

康德說:

\begin{quotation}\kaishu 上面的列星天體以及裡面的道德法則,這兩事,我們愈是頻
頻地靜靜地去反省它們,我們心中愈是充滿了以新增的崇贊
與戰兢。我不是要去追尋它們以及去猜測它們,好像它們隱
蔽於黑暗中,或隱蔽於超出我的視線之外的超越領域中;我
實是看見它們在我眼前,並以我的存在之意識直接與它們相
連結。前者,從我在外在感覺世界中所佔有的地方開始,並
自此擴大我的連繫到一同著世界上的世界,天體中的天體之
無界限的境域,並且又把我的連繫擴入到它們的週期運行之
無限制的次數中。後者,從不可見的我,我的人格,開始,
並顯示我在一有真正無限性的世界中,但這世界只因知性而
可追尋,並因這世界,我辨識我不是在一只是偶然的狀態
中,但是在一普遍而必然的連繫中,一如我也伴同著一切那
些可見的世界一樣。前者,自無窮數的世界觀之,好像是消
失了我的重要性,使我一如一不關重要的動物,在一短暫的
時間內含具著一種生命力,也沒有人知道如何是這樣,但此
後仍必須再把它所由之以組成的物質歸還給它所居處的星球
(只是宇宙中之一粟)。但是後者則反之,它因著我的人格\end{quotation}

\newpage\thispagestyle{empty}\addtocounter{page}{-1}\vspace*{-12mm}\begin{center}\noindent
\includegraphics[clip, trim=161pt 133pt 137pt 248pt, height=162mm]{ocr-input/image-2264.png}\end{center}

\newpage\markright{第二部 \quad 分論一 \quad 第二章 \quad 張橫渠對於「天道性命相貫通」之展示}

\begin{quotation}\kaishu 無限制地升舉了我的價值,使我成為一睿智體,在我的人格
中,道德律顯示給我一種生命是獨立不依於動物性,甚至是
獨立不依於那全部感覺的世界——至少從這道德律所指定給
我的存在之使命而可以作此推斷,這使命不限制於今生底範
圍與條件,且可伸展而至於無限。((實踐理性批判)第二部
〈純粹實踐理性之方法學〉,〈結論〉中語)。\end{quotation}

康德這段話即已暗示出先天後天之兩義。康德沒有把意志自由以及自由意志所自立之道德法則視為吾人之性體,且是「萬物之一源」之性體;也沒有把那列星天體統於性體之神與理而視為神體與虛體所妙運之氣化,(至少這一部工作沒有充分透澈地作得到),此即表示其沒有作成一個充其極的「道德的形上學」;他也沒有從統於神、理而偏於氣化說的「氣命」之觀念,以及由氣命而說的天、天理、天命、天道之觀念;他所極成的是一個依據基督教傳統而來的「道德的神學」,而其三大批判自身所蘊具的固有之「道德的形上學」卻沒有充分作得成,因此其「道德的神學」與其三大批判自身所固蘊具之「道德的形上學」成了兩截而隔氣;而如果其三大批判自身所固蘊具之「道德的形上學」能充分地作得成,因而其「蘊具」可稱為「圓具」,則其「道德的神學」如非不必要,即是重疊。這一切,早已為先秦儒家所透澈,而為宋、明儒者所弘揚。康德生於十八世紀,其後於宋明儒遠矣。設其得窺儒學傳統之全貌,其道德哲學必更透澈而無疑。然依其思辨力已幾近於此矣。吾述橫渠性體義而言至此,亦略示導引之意也。關於康德之不足處,其詳請參看〈綜論部〉第三章。

\newpage\thispagestyle{empty}\addtocounter{page}{-1}\vspace*{-12mm}\begin{center}\noindent
\includegraphics[clip, trim=166pt 160pt 140pt 233pt, height=162mm]{ocr-input/image-2268.png}\end{center}

\newpage

\section{「合性與知覺有心之名」:心體義疏解}

\subsection{心之名之所以立}

案:此語亦不的當。「合性與知覺」好像是說性體中本無知覺,性是性,加上知覺才有「心之名」。此句由「合」字表示心,與上句由「合」字表示性,皆是不精熟之滯辭。殊不知在性與心處,均不應如此表示也。

依上第二節貫通之疏解,性就是太虛寂感之神。名之曰性者,是對應個體或總對天地萬物而為其體言,此是性體義;又自其能起道德之創造或宇宙之生化言,則是性能義;又自其所有之道德創造乃至「陰陽鬼神」之化皆是此性體所命之本分、當然而不容已必然而不可移者言,則是性分義。宇宙論地綜言之,只是一個虛體、神體,自對應個體或天地萬物而為之體言,則有此三義,此是「性之名」之所以立。而「合虛與氣有性之名」,則不切矣。

性之名是就太虛寂感之神(此亦曰虛體,虛即是體,亦曰神體,神即是體)說,則心之名亦不能由外此而別有所合以立。心就是剋就「寂感之神」說。寂感必然地函心義,神亦必然地函心義。「知覺」即是「寂感之神」之靈知明覺,不是吾人今日所說之「感觸的知覺」(sensible perception)。「有識有知,物交之客感爾。」「客感」之識知亦可說為此「靈知明覺」之發用。周濂溪〈太極圖說〉中所謂「形生矣,神發知矣」云云是也。「神發知矣」即是誠體之神之發為識知也。識知是此靈知明覺之發用,則

\newpage\thispagestyle{empty}\addtocounter{page}{-1}\vspace*{-12mm}\begin{center}\noindent
\includegraphics[clip, trim=140pt 143pt 133pt 233pt, height=162mm]{ocr-input/image-2272.png}\end{center}

\newpage\markright{第二部 \quad 分論一 \quad 第二章 \quad 張橫渠對於「天道性命相貫通」之展示}

\noindent 其根即是此宇宙論的靈知明覺之神亦明矣。

〈神化篇〉云:「虛明照鑑,神之明也。無遠近幽深、利用出入,神之充塞無間也。」「神之明」就是神之靈知明覺,「神之充塞無間」就是靈知明覺之充塞無間,靈知明覺即是神體之朗照(虛明照鑑)。

〈太和篇〉云:「氣聚則離明得施而有形,氣不聚則離明不得施而無形。」此言「離明」亦即神體之「虛明照鑑」也。自此而言。即是心。

心不但是「形生矣,神發知矣」之形生後之「知」,亦不但是客感之識知。依濂溪,「神發知矣」下,則云:「五性感動,而善惡分、萬事出矣。」如只以形生後所發之知為心,則此心不必能貞定而純一,此可曰心理學的心、識心、經驗心、習心、成心,而不必是貞定純一、「動而無動、靜而無靜」、動靜一如之神心真心、本心、超越心也。依橫渠(大心篇〉所說(詳見下),此客感之識知,如不加以貞定,很可以是「徇象喪心」,「存象之心,亦象而已」,而不是心。(太和篇〉未段亦有一句云:「心所以萬殊者,感外物為不一也。」此明示客感之識知實乃「存象之心」而不是心也。如非「存象之心」,心何以能說萬殊?如果能提得住,保持其貞定之純一,則其萬殊之不一只是其隨機應變之形態之不一,而其自身仍不喪其一,此則非客感之識知之心自身所能必也。是以心之名決不是只就此經驗層(感觸層)上立。然則如果首先「本體宇宙論地」說,則心之本義最深義、根源義,必須就神體之「虛明照鑑」說,而靈知明覺之知覺亦必須就此神體之明說。是以不是「合性與知覺有心之名」,乃是就性體寂感之神之靈知明覺或虛明

\newpage\thispagestyle{empty}\addtocounter{page}{-1}\vspace*{-12mm}\begin{center}\noindent
\includegraphics[clip, trim=170pt 161pt 151pt 242pt, height=162mm]{ocr-input/image-2276.png}\end{center}

\newpage

\noindent 照鑑說即是心,此心之名之所以立也。

依此,性體之全幅具體內容(真實意義)即是心,性體之全體呈現謂心。心體之全幅客觀內容(形式意義)即是性,心體之全體挺立謂性。首先性具有性體、性能、性分之三義,自心言,心亦必類比相應地具有此三義:心體義,心即是體;心能義,心能創生,心能形著;心宰義,心主于身,其所自律而命于吾人者皆是本分之素定,「大行不加,窮居不損,分定故也。」依此而言,心性完全合一——不,完全是一。若以性為準而言之,則除上三義外,尚可加兩義而為五義。一是性分所據以成之性理義,性體自具普遍法則即是理。此外,則是性覺義,性體之「神之明」即是覺。如是,性體、性能、性理、性分、性覺,五義備而性之全體明,心之全體亦明矣。此為心性是一之宇宙論的模型。

但此宇宙論的模型必須經由道德實踐以證實而貞定之。心性是一之宇宙論的模型以性為主,道德實踐之證實而貞定此模型,則須以心為主。由宇宙論的模型建立客觀性原則,即建立天地萬物之自性,雖有性覺義,亦是客觀地說,亦是客觀性原則。由道德實踐之證實而貞定之,建立主觀性原則——形著原則,具體化原則。

\subsection{「心能盡性」之總綱}

〈誠明篇〉云:

\begin{quotation}\kaishu 心能盡性,人能宏道也。性不知檢其心,非道宏人也。\end{quotation}

\noindent 案:此兩語表示主觀性原則與客觀性原則最為顯明。「心能盡性」

\newpage\thispagestyle{empty}\addtocounter{page}{-1}\vspace*{-12mm}\begin{center}\noindent
\includegraphics[clip, trim=161pt 138pt 136pt 242pt, height=162mm]{ocr-input/image-2280.png}\end{center}

\newpage\markright{第二部 \quad 分論一 \quad 第二章 \quad 張橫渠對於「天道性命相貫通」之展示}

\noindent 是道德實踐地言之,是由道德實踐以證實而貞定那客觀而宇宙論地說的心性是一之模型,具體言之,即由道德心(如惻隱、羞惡、辭讓、是非等)之主觀地、存在地、真切地呈現或覺用來充分實現或形著那客觀地說的性。故道德實踐地說的道德本心即是主觀性原則、形著原則、具體化原則。「性不知檢其心」之性,是客觀地、本體宇宙論地說的性,即性體、性能、性理、性分、性覺,五義俱備之性。此性體若無主觀地、存在地說的道德本心之真切覺用或真實呈現來形著之,它只是自存、潛存,而不能起任何作用,此即所謂「性不知檢其心」,亦孔子所謂「非道宏人」也。檢者定義、察義。心能盡性,則性自能檢其心。心不能盡性,則性雖自存而毫不能為力。故言道德實踐,以心為決定因素也。人能宏道,道自宏人。人不能宏道,則道雖不為堯存,不為桀亡,亦不能彰顯也。不能彰顯,即不能起作用。故自道德實踐言,以人之宏為主。是以性是客觀性原則、自性原則。就性體自身言,性體之在其自己是性體之客觀性,性體之對其自己是性體之主觀性。性體之在其自己是性體之自持、自存,性體之逕挺持體。性體之對其自己是性體之自覺,而此自覺之覺用即心也。此即道德的本心之所以立。道德的本心非他,本就是性體之自覺(自己覺其自己)。此是客觀地言之。若主觀地、存在地言之,即是心能盡性,當下即自本心自己之真切覺用以盡此性,以充分地形著此性。及至此真切覺用調適上遂,全幅朗現,則性體之內容全部在心,而心亦全體融於性,此即為心性之合一,主客觀之真實統一,而重返其心性本是一之宇宙論地說的模型而澈底證實而貞定之。此是「心能盡性」之總綱也。

以下試就《正蒙·大心篇》而論此義。

\newpage\thispagestyle{empty}\addtocounter{page}{-1}\vspace*{-12mm}\begin{center}\noindent
\includegraphics[clip, trim=168pt 161pt 155pt 241pt, height=162mm]{ocr-input/image-2284.png}\end{center}

\newpage

\subsection{}

〈大心篇〉云:

\begin{quotation}\kaishu 大其心,則能體天下之物。物有未體,則心為有外。世人之
心止於聞見之狹。聖人盡性,不以見聞梏其心。其視天下,
無一物非我。孟子謂盡心則知性知天,以此。天大無外,故
有外之心,不足以合天心。\end{quotation}

\noindent 案:「大其心」之大並不是空口說大話,其根本關鍵乃在是否能盡心或盡性而不為「聞見之狹」所限,故曰:「世人之心止於聞見之狹」,又曰:「聖人盡性,不以見聞梏其心。」「聖人盡性」,盡性即已函盡心。為聞見之狹所限即小,不為見聞所桎梏即大。此言大小顯本孟子大體小體之分而來。而是否能不為見聞之狹所限所梏,實含有一道德實踐之工夫在,道德實踐之定義是對應道德行為之本性、充分表現無雜念之道德心靈,以體現其自身之天理(即康德所謂自律之道德法則、定然命令)之謂。如是,所謂「大其心」根本是要從「見聞之狹」中解放。解放後的道德心靈乃根本是超越的心靈,孟子所謂「本心」。囿於見聞之狹,而為見聞所桎梏、所拘繫,總之所限制者,則是所謂經驗的心、感性的心,亦即所謂心理學的心,莊生所謂「成心」,佛家所謂識心、習心是也。此種心在條件制約中、在遷流變動中,當然不能由之建立起或表現出真正之道德行為。此種心為耳目之官所限,自然是小。心小人亦小,此即孟子所謂「從其小體為小人。」此為小,則解脫出來的超越心

\newpage\thispagestyle{empty}\addtocounter{page}{-1}\vspace*{-12mm}\begin{center}\noindent
\includegraphics[clip, trim=164pt 130pt 128pt 244pt, height=162mm]{ocr-input/image-2288.png}\end{center}

\newpage\markright{第二部 \quad 分論一 \quad 第二章 \quad 張橫渠對於「天道性命相貫通」之展示}

\noindent 靈自然是大。心大人亦大,故孟子謂「從其大體為大人。」

從「聞見之狹」中解脫出來的超越的道德本心自然不能有外,這是它的普遍性、遍在性。這普遍性是由「體天下之物」或「視天下無一物非我」而規定,這就是仁心之無外。故此普遍性是絕對而具體的普遍性,非抽象的類名之普遍性。「聖人盡性」即盡的這仁性,盡仁性即盡仁心。故云:「孟子謂盡心則知性知天,以此。」「天大無外」,性大無外,心亦大而無外。此無外之心即「天心」也。天無外、性無外,是客觀地說,心無外是主觀地說。而天與性之無外正因心之無外而得其真實義與具體義,此為主客觀之統一或合一。孟子言「萬物皆備於我」正是這仁心之無外。

仁心之無外亦不只是形式地說,而實由「體天下之物」之「體」字而見。此「體」字是表示「仁必無外」是具體的、存在的,這要在實踐中純粹的超越的道德本心真實呈現,對於天下之物真感到痛癢,始有此天心之無外。

〈天道篇第三〉云:

\begin{quotation}\kaishu 天道四時行,百物生,無非至教。聖人之動無非至德。天何
言哉?天體物不遺,猶仁體事無不在也。禮儀三百,威儀三
千,無一物而非仁也。昊天曰明,及爾出王〔同「往」】昊
天曰旦,及爾游衍!無一物之不體也。\end{quotation}

\noindent 朱子對此段文甚能欣賞。他說:「此數句從赤心片片說出來。荀、揚豈能到?」(《朱子語類》卷第九十八,〈張子之書〉),「從赤心片片說出來」是說此數句真是真切。說「荀、揚豈能到」,便

\newpage\thispagestyle{empty}\addtocounter{page}{-1}\vspace*{-12mm}\begin{center}\noindent
\includegraphics[clip, trim=164pt 166pt 149pt 232pt, height=162mm]{ocr-input/image-2292.png}\end{center}

\newpage

\noindent 不類。此文與「大其心則能體天下之物」一段完全相同。此文中「天道四時行,百物生,無非至教」,以及引〈大雅·板〉詩「昊天曰明,昊天曰旦」云云,是客觀地說;而「聖人之動,無非至德」,以及「禮儀三百,威儀三千,無一物而非仁」,則是主觀地說。兩者在聖人的踐仁盡性或踐仁知天中完全合一,故曰:「天體物不遺,猶仁體事無不在也。」仁心即天心,仁德即天道。「仁體事無不在」不是仁這個概念體事無不在,乃是仁心仁性在「盡」中體事無不在。橫渠言「大其心」云云,顯是根據孔子之仁與孟子之本心而說。但朱子極不喜「大其心」之辭,因之對此段解之極無精神,而且時露微詞。這是因為他心中有忌諱,並因為中和定說後其義理形態已成定局之故。

《朱子語類》卷第九十八載云:

\begin{quotation}\kaishu 大其心則能體天下之物。世人之心止於見聞之狹,故不能體
天下之物。唯聖人盡性,故不以所見所聞梏其心,故大而無
外。其視天下無一物非我。他只是說一個大與小。孟子謂盡
心則知性知天以此。蓋盡心則只是極其大,心極其大,則知
性知天,而無有外之心矣。\end{quotation}

\noindent 案:朱子此解只是順橫渠字面說,非由衷之言,其心中極不樂意。「他只是說一個大與小」「蓋盡心則只是極其大」等語即示不滿之意。

故繼之復載云:

\newpage\thispagestyle{empty}\addtocounter{page}{-1}\vspace*{-12mm}\begin{center}\noindent
\includegraphics[clip, trim=162pt 196pt 128pt 235pt, height=162mm]{ocr-input/image-2296.png}\end{center}

\newpage\markright{第二部 \quad 分論一 \quad 第二章 \quad 張橫渠對於「天道性命相貫通」之展示}

\begin{quotation}\kaishu 道夫問:今未到聖人盡心處,則亦莫當推去否?曰:未到那
裡,也須知說聞見之外,猶有不聞不見底道理在。若不知聞
見之外,猶有道理,則亦如何推得?要之,此亦是橫渠之
意。然孟子之意,則未必然。\end{quotation}

\noindent 案:此解橫渠之意亦非是。橫渠言囿不囿於「聞見之狹」,是要顯本心、仁心之體物而不遺之遍在性,不是說「聞見之外,猶有不聞不見底道理」。朱子此解正是要指向由格物窮理以盡心,此非孟子意,亦非橫渠之意。橫渠之引孟子作證,非只橫渠意,實亦即孟子之本意也。朱子謂「孟子之意,則未必然」,蓋朱子對于孟子「盡心知性」之義別有解說故也。

故又载云:

\begin{quotation}\kaishu 道夫曰:孟子本意,當以《大學或問》所引為正。曰:然。
孟子之意只是說窮理之至,則心自然極其全體而無餘。非是
要大其心而後知性知天也。\end{quotation}

\noindent 案:此即朱子對於孟子盡心章之別解。此別解完全非是。此誠是別扭之解,而且是誤解。朱子以爲「盡其心者,知其性也」,其意是:所以能盡其心者,是由于知其性也。此是因果顛倒。而「知其性」則又解為格物窮理。窮理之至,則心之全體大用無不明,便為盡心。是則盡心之「盡」為認知地盡,而因「性即理也」,故格物窮理即是知性。知性知到家,便算盡了心了。故《集註》註此章云:

\newpage\thispagestyle{empty}\addtocounter{page}{-1}\vspace*{-12mm}\begin{center}\noindent
\includegraphics[clip, trim=161pt 158pt 143pt 235pt, height=162mm]{ocr-input/image-2300.png}\end{center}

\newpage

\begin{quotation}\kaishu 人有是心,莫非全體。然不窮理,則有所蔽,而無以盡乎此
心之量。故能極其心之全體而無不盡者,必其能窮天理而無
不知者也。既知其理,則其所從出〔案:即「天」】,亦不
外是矣。以《大學》之序言之,知性則物格之謂,盡心則知
至之謂也。\end{quotation}

\noindent 此註完全非是,決非孟子本意。橫渠明道、伊川說及此,皆非如朱子之所理會。「盡其心者,知其性也」之句法,在某種上下文之聯絡裡,此語句亦可類比地被解為「盡心由於知性」之意,如「得其民者,得其心也」之類。但此章之語脈則不是此意,而衡之以孟子之義理,亦不能是此意。孟子言「盡」是充分實現之意,亦即「擴而充之」之意。盡心即盡的惻隱羞惡等超越的道德本心,非是盡其認知的「全體大用無不明」之智用之量。能充分體現此本心,則即「擴而充之,足以保四海」,仁不可勝用、義不可勝用等語之意。亦即「學問之道無他,求其放心而已矣」之義。亦「人皆有斯心也,賢者能勿喪耳」之義。朱子之解差之遠矣,謬之甚矣。能充分盡此心而勿喪,便算覿面相當真明白了吾人之性。蓋在孟子,此超越的道德本心即是性,即是人之所以為人之超越的性能,人之所以能發展其道德人格,所以能完成其道德行為之純亦不已之先天根據。故盡心即是知性,知即在盡中知。而知性即是盡性,「知」處並無曲折的工夫。工夫全在「盡」字。所謂「知」者,只是在「盡心」中更具體地、真切地了解了此性體而已,此性體更彰著於人之面前而已。在「盡心」中了解人之真正的本源(性體)、真正的主體,則即可以「知天」矣。因為天亦不過就是這「於穆不已」之創

\newpage\thispagestyle{empty}\addtocounter{page}{-1}\vspace*{-12mm}\begin{center}\noindent
\includegraphics[clip, trim=170pt 137pt 134pt 245pt, height=162mm]{ocr-input/image-2304.png}\end{center}

\newpage\markright{第二部 \quad 分論一 \quad 第二章 \quad 張橫渠對於「天道性命相貫通」之展示}

\noindent 造,即生化之理也。故《中庸》曰:「天地之道可一言而盡也。其為物不貳,則其生物不測。」在天,說「生物不測」;在性,則說道德創造(道德行為之純亦不已)之「沛然莫之能禦」。故天之正面函義與心、性之函義為同一也。故盡心即知性,知性即知天。此皆是立體的直貫義,「本體、宇宙論的」創造義,非如朱子之轉為認知的橫列義也。此是通常無誤的理解。橫渠此處提及此,雖重在說體物不遺,然體物不遺而為之體,即函體之而妙運之,亦是立體的直貫義,「本體、宇宙論的」妙、通義,決非朱子之認知的橫列也。即伊川提及此,亦照通常讀法解,並未以格物窮理說盡心,至少不是如朱子之因果顛倒。不知朱子在此何以必作異解。得無以格物窮理之格局横梗心中而必欲曲解孟子以遷就之耶?然要者是在將擴充的盡講成認知的盡,此種驚扭使朱子終生不能了解孟子,不能正視主觀性原則就之以言道德之實踐,而必繞出去(歧出)以表示其嚴肅的道德意識(敬)與下學上達之經驗工夫,故亦終不能了解橫渠之「心能盡性」義、明道之「一本」論、胡五峰之「盡心成性」義,以及象山之孟子學。

《語類》繼上復載云:

\begin{quotation}\kaishu 道夫曰:只如橫渠所說,亦自難下手。曰:便是橫渠有時自
要恁地說。似乎只是懸空想像,而心自然大。這般處,元只
是格物多後,自然豁然有個貫通處。這便是下學而上達也。
孟子之意,只是如此。\end{quotation}

\noindent 案:此段話相差太遠,不應如此隔閡!對應真正道德行為本身說,

\newpage\thispagestyle{empty}\addtocounter{page}{-1}\vspace*{-12mm}\begin{center}\noindent
\includegraphics[clip, trim=159pt 149pt 140pt 235pt, height=162mm]{ocr-input/image-2308.png}\end{center}

\newpage

\noindent 由盡心盡性下手,即,不為見聞之狹所梏所限,而將超越的道德本心澈底呈現,純自義上以奉行本心所自發之天理、定然律令,馴致仁心、本心、天心之遍潤而不遺,這將是最真切最中肯之下手處,何言「難下手」耶?道夫此言,根本表示其未能正視道德行為之何所是。這自不如讀一本書或撲著一件事之下手處之為較易領會與易有把柄,然這些卻正是歧出,而與真正的道德實踐為不相干者,即,對應道德實踐言,此皆不是本質地相干者。朱子不就此點醒他,卻認為「橫渠有時自要恁地說,似乎只是懸空想像,而心自然大」。這根本不解橫渠所說的「大」之切義與實義,而孟子的義理亦根本未能進到其生命內。如照朱子所說,橫渠似乎只是擰著憑空隨意說。然則孟子之言大體小體、求放心、萬物皆備於我、沛然莫之能禦,皆「懸空想像」乎?朱子不解此一系之義理,硬要解成「這般處,元只是格物多後,自然豁然有個貫通處」,以為「孟子之意,只是如此」,此才真正「自要恁地說」,亦無可如何也。

以上所錄《語類〉爲一整段。分別點之,以明朱子之非。

《語類》繼上復載云:

\begin{quotation}\kaishu 大其心則能遍體天下之物。體,猶仁體事而無不在。言心理
流行,脈絡貫通,無有不到。苟一物有未體,則便有不到
處,包括不盡,是心為有外。蓋私意間隔,而物我對立,則
雖至親且未必能無外矣。故有外之心,不足以合天心。\end{quotation}

\noindent 案:此解無問題。若順此理解,何至於認橫渠說「大」只是「懸空想像」?

\newpage\thispagestyle{empty}\addtocounter{page}{-1}\vspace*{-12mm}\begin{center}\noindent
\includegraphics[clip, trim=166pt 135pt 137pt 247pt, height=162mm]{ocr-input/image-2312.png}\end{center}

\newpage\markright{第二部 \quad 分論一 \quad 第二章 \quad 張橫渠對於「天道性命相貫通」之展示}

繼之復載云:

\begin{quotation}\kaishu 問:物有未體,則心為有外,此體字是體察之體否?曰:須
認得如何喚做體察。今官司文書行移,所謂體量、體究,是
這樣體字。或曰:是將自家這身入那事物裡面去體認否?
曰:然。猶云體群臣也。伊川〔案:當為明道】曰:天理二
字卻是自家體貼出來。是這樣體字。\end{quotation}

\noindent 案:此解「體」字。體雖有體察、體認、體究、體貼、體諒、體恕諸義,都表示親切入裡,不相隔閔之意,然自「天體物不遺」、「仁體事無不在」,就仁心之感通、關切、知痛癢、不麻木言,此「體」字卻非認知意義的「體察」「體認」「體究」之意,而是道德意義的立體直實之體諒、體恕之意。首問是「體察」否,已歧出而遠離矣,或者之間「是〔……〕入那事物裡面去體認否」亦同樣非是,而朱子皆首肯之,是其著眼點重在認知義也。朱子繼云「猶云體群臣」,須知體察、體認、體究,皆非「體群臣」之體諒、體恕義。「體貼」有時是指道德之心與情之體貼言,此則同於體諒、體恕,有時是指認知之體會言。明道說「天理二字卻是自家體貼出來」,此體貼即體會義,同於體認、體察。朱子僱侗一起說,而又重在認知意義之體認、體究與體察,是即未能真明「體物不遺」之切義與實義也。此是仁心之感通,是道德意義,非認知意義。「天體物不遺,猶仁體事無不在」,就前一句說,即《易傳》「曲成萬物而不遺」,《中庸》「鬼神體物而不可遺」之意,就後一句說,即仁心之感通遍潤一切而不遺,仁道之顯現遍成一切而不

\newpage\thispagestyle{empty}\addtocounter{page}{-1}\vspace*{-12mm}\begin{center}\noindent
\includegraphics[clip, trim=166pt 151pt 140pt 240pt, height=162mm]{ocr-input/image-2316.png}\end{center}

\newpage

\noindent 遺之意。由動字之「體之」即可轉而為名詞,而見其為萬事萬物之體也。故橫渠云:「體萬物而謂之性」,又云:「未嘗無之謂體,體之謂性。」此豈是體究、體察、體認所表示之「心之全體大用無不明丁之無外耶?朱子之心態總是向認知橫列之靜涵形態(靜攝形態)轉,而不肯向立體直貫之形態轉。

繼上又載云:

\begin{quotation}\kaishu 問:物有未體,則心為有外。體之義如何?日:此是置心在
物中究見其理,如格物致知之意。與體用之體不同。\end{quotation}

\noindent 案:此正式將「體」字講成體究之體,講成格物致知之意,完全非是。「體」字固是動詞,然卻是「本體、宇宙論地」妙之、運之、潤之、通之之「體之」,而不是認知的窮究之「體之」。此「本體、宇宙論地」體之,轉為名詞而見其為萬事萬物之體,即成體用之體。當其為動詞之時,自非體用之體。然亦總非「究見其理」之意。如此顯明之義理,朱子何竟如此誤解耶?

繼上又載云:

\begin{quotation}\kaishu 或問如何是有外之心?日:只是有私意,便內外扞格,只見
得自家身己,凡物皆不與己相關,便是有外之心。橫渠此說
固好,然只管如此說,相將便無規矩,無歸者,入於邪遁之
說。且如夫子為萬世道德之宗,都說得語意平易。從得夫子
之言,便是無外之實。若便要說天大無外,則此心便醫入虛
空裡去了。\end{quotation}

\newpage\thispagestyle{empty}\addtocounter{page}{-1}\vspace*{-12mm}\begin{center}\noindent
\includegraphics[clip, trim=161pt 134pt 136pt 248pt, height=162mm]{ocr-input/image-2320.png}\end{center}

\newpage\markright{第二部 \quad 分論一 \quad 第二章 \quad 張橫渠對於「天道性命相貫通」之展示}

\noindent 案:此綜解表示朱子心中之忌諱。此段開首數語尚好。人能去私意,恢復其超越的道德本心而自律稱理以行,感通一切而不扞格,此正是最真切最正大的道德實踐,何至有「無規矩,無歸著,入於邪遁之說」之病?其所謂「邪遁之說」即禪也。心中總放不下這忌諱,雖正大之義亦不敢說。他根本不喜「大其心」以及「天下無外」之「大」字,故雖面對真理而不肯正視,只顧說那「懸空想像」,「心便瞥入虛空裡去」,這不相干之忌諱。朱子是在這忌諱所形成之封閉心境下而轉向認知橫列之靜涵形態(靜攝形態),以為如此便可堵住那些「無規矩,無歸著,入於邪遁之說」之流弊,而殊不知如此一轉卻亦正同時遠離孔子之仁與孟子之心性之宏規。是故對於橫渠「大其心則能體天下之物」一段所根據之孔子之仁與孟子之「盡心知性知天」之義完全解不著,而對於「體」字之解析亦全成誤解。吾所以不厭煩瑣,逐條指正,正欲藉此以明先秦儒家之原義以及北宋諸儒體解之不謬,且對顯朱子學形態之何所是,以為此中差異之成,其界線甚清,其義理脈絡甚明,而人為辭語之似是與大體皆相類所迷惑,鮮能浸潤清澈,釐清其眉目。(老子所謂「濁以靜之徐清,晦以理之徐明」)吾人若想以朱子為準而曲通先秦儒家之原義以及北宋諸家之體解,講來講去,總覺觸途成滯,窒礙難通。吾處此困甚久,終於物各付物,而眉目朗然矣。〔朱子繼承北宋諸儒詮表儒家義理,其進於漢、唐經生者多矣,其功自大。然其註解四書、《易傳》以及講說北宋諸家,緊要處確有問題。由其靜涵形態所造成之纏夾與葛藤,剔解釐清確極不易。吾故自講述濂溪起即隨時關照朱子體解之不同(所謂異解、別解、誤解)與不足,以明此中癥結之所在。如此剔解抉擇,至講述朱子本人時,其

\newpage\thispagestyle{empty}\addtocounter{page}{-1}\vspace*{-12mm}\begin{center}\noindent
\includegraphics[clip, trim=169pt 161pt 143pt 236pt, height=162mm]{ocr-input/image-2324.png}\end{center}

\newpage

\noindent 思理形態之不同即全部朗現矣。]

\subsection{}

〈大心篇〉繼上云:

\begin{quotation}\kaishu 見聞之知乃物交而知,非德性所知。德性所知,不萌於見
聞。\end{quotation}

\noindent 案:上段言心體物不遺是著重在超越的道德心之為「體」義,而自此段以下,則著重在心之「知用」義。首先,「見聞之知」與「德性之知」之分別亦始於橫渠。上段言囿不囿於聞見,是根據孟子大體小體之分,著重在心靈之陷溺與否之道德意義。而由於囿不囿於聞見,從知用方面說,遂引生兩種知識之分別,而著重在心靈之囿不囿之知識意義。由不囿於聞見之知識意義之心靈之知用反顯德性心靈之為體,雖說是知識意義之知用,而實在極成德性心靈之無外,故曰德性之知。德性之知實無今日所說之知識意義也。是以見聞之知與德性之知之對揚,雖說是知用,仍是指向道德心靈之呈現,而不在純認知活動之探究也。

見聞之知是屬於「知識意義」者,即所謂經驗知識。無論是粗樸材料之獲得,或進而究知外物之質、量、與關係,總是經驗知識也。從認知活動言,見聞之知所表示之心靈活動是「萌於見聞」,是在感觸知覺中呈現,是囿於經驗而受制於經驗。若依真正經驗知識之成立言,心靈之認知活動必囿於經驗而限於經驗之範圍,始有真正知識或積極知識(或實證知識)之可言。康德所說之先驗知識

\newpage\thispagestyle{empty}\addtocounter{page}{-1}\vspace*{-12mm}\begin{center}\noindent
\includegraphics[clip, trim=161pt 137pt 138pt 245pt, height=162mm]{ocr-input/image-2329.png}\end{center}

\newpage\markright{第二部 \quad 分論一 \quad 第二章 \quad 張橫渠對於「天道性命相貫通」之展示}

\noindent 實不是經驗知識之「知識」義,而是其所說之成功經驗知識之先驗原則,或是純形式之知識如數學與幾何等。此先驗知識亦可以說是不「萌於見聞」,但卻並不是橫渠所說之德性之知。在此並無德性的意義,其所表示之心靈活動亦非德性的,乃是純認知的,不過是純形式的而已。橫渠雖分出見聞之知,卻亦並未積極探究經驗知識之構成,因而亦並未在此成一積極的知識論。其言此,乃著重其「萌於見聞」,而為見聞所限,故不能至乎體物不遺而無外之境也。

德性之知依橫渠,即是發於性體之知。此如切言之,即知愛知敬、知是知非,當惻隱自惻隱、當羞惡自羞惡、當辭讓自辭讓之知,此則自無關於見聞,故亦不萌於見聞。其知用之活動實亦無特定之經驗對象為其所適應,只不過是那超越的道德本心無外之呈現,呈現而自顯其自主自決自有天則之朗潤、遍照與曲成,故亦無關於見聞,亦不萌於見聞。客觀地說的性天之無外,實即由主觀地說的本心知用之無外來證實,實亦即是本心知用之無外之自己。故德性之知惟在表示由超越的道德本心之知用來反顯德性心靈之無外,亦即心體性體之無外,性體道體之無外,而實無認知意義也。朱子所言之「心之全體大用無不明」乃就「即物而窮其理」而見。其所知者是客觀的本體論的靜態存有之理,雖萌於見聞以及見聞所接之物,而亦實超於見聞以及見聞所接之物,雖是超於見聞,卻亦是為心之虛靈明覺所知之對象,不過是超越之對象而已。以此,其所說之「心之全體大用」實即認知之全體大用,故有認知之意義,因而亦不表示立體直貫的或承體起用的體物不遺之無外之本心也。從客觀方面說,其所知之理,綜之為太極,是本體論的靜態存有之

\newpage\thispagestyle{empty}\addtocounter{page}{-1}\vspace*{-12mm}\begin{center}\noindent
\includegraphics[clip, trim=167pt 156pt 147pt 245pt, height=162mm]{ocr-input/image-2333.png}\end{center}

\newpage

\noindent 理,而不是動態的創生之理。此即是生化之理之不自覺地轉成另一形態,即本體論的存有之形態,而不是「本體宇宙論的」即存有即活動之立體直貫之形態。而從主觀方面說,則總之為一認知橫列之靜涵形態或靜攝形態。此朱子學之特色,乃承伊川而來,既非先秦儒家之原義・亦非北宋濂溪、橫渠、明道之所體解者。(濂溪雖無「德性之知」之辭語,然其《通書·思第九》言「無思本也,思通用也」,「無思而無不通為聖人」,而又合於誠體而言,是亦即「德性之知」也。又,明道略言格物亦不同於伊川。詳見〈伊川章〉第八節〈格物窮理篇·附識〉。)

〈誠明篇〉開首云:

\begin{quotation}\kaishu 誠明所知,乃天德良知,非聞見小知而已。天人異用,不足
以言誠。天人異知,不足以盡明。所謂誠明者,性與天道不
見乎小大之別也。\end{quotation}

\noindent 此文是〈誠明篇〉之開首語,亦由「天德良知」與「聞見小知」之分別而顯本體意義的誠與明。此數語可視為了解「見聞之知」與「德性之知」之確義之指導原則。《中庸》云:「誠則明矣,明則誠矣。」又云:「誠則形,形則著,著則明,明則動,動則變,變則化。唯天下之至誠為能化。」此皆承體起用、立體直貫之「本體宇宙論的」辭語。誠體起明,明即全澈於誠。故明即誠體之朗潤與遍照。誠明一體即窮盡本心性體之全蘊,亦即窮盡性與天道之全蘊。誠明起知即是天德之良知。良知之知用亦不過就是誠明自己之一天人、合內外,而「不見乎小大之別」而已。分解言之,「天德

\newpage\thispagestyle{empty}\addtocounter{page}{-1}\vspace*{-12mm}\begin{center}\noindent
\includegraphics[clip, trim=165pt 130pt 134pt 254pt, height=162mm]{ocr-input/image-2337.png}\end{center}

\newpage\markright{第二部 \quad 分論一 \quad 第二章 \quad 張橫渠對於「天道性命相貫通」之展示}

\noindent 良知」是大,聞見之知是小。然「天德良知」非是隔離的抽象體,乃必由通天人、合內外、一小大,而見其為具體而真實的誠明之知用。天德良知具體流行,則雖不囿於見聞,亦不離乎見聞。如是,聞見之知亦只是天德良知之發用,而聞見之知不為小矣。聞見之知之所以小乃由於其不通極於天德良知,而自桎梏於聞見,遂成其爲「識心」而小矣。小即是人(人為自限),大即是天。「天人異用,不足以言誠」,誠則通天人,而人亦天矣。一切人事皆是天行。「異用」者,天是天,人是人,如是則誠體即隔絕而為抽象體,而非具體而真實之真誠也。「天人異知,不足以盡明」,明則通天人,而屬於人之聞見之知亦大而同乎天德良知矣。「異知」者,天是天,人是人,如是則誠體之明即是孤明,而不能盡其具體而真實之全體大用矣。通天人,合內外,而盡其誠明之體之實義,則無小無大,小大之別亦泯,遂化而為一體流行矣。故云:「所謂誠明者,性與天道不見乎小大之別也。」性與天道不外一誠明之體。自其一體而化言(所謂體用不二),則不見有小大之別。是故此言知用,不論其為聞見,抑或為良知,如從誠明說,皆要在反顯誠明之體之通貫(立體的通貫,非橫列的通貫),而實無普通之認知意義也(此段所言當與〈伊川章〉第八節〈格物窮理篇〉合看)。

\subsection{虛明純一之心與「存象之心」}

〈大心篇〉繼上又云:

\begin{quotation}\kaishu 由象識心,徇象喪心。知象者心。存象之心,亦象而已。謂\end{quotation}

\newpage\thispagestyle{empty}\addtocounter{page}{-1}\vspace*{-12mm}\begin{center}\noindent
\includegraphics[clip, trim=180pt 168pt 134pt 234pt, height=162mm]{ocr-input/image-2341.png}\end{center}

\newpage

\begin{quotation}\kaishu 之心可乎?\end{quotation}

\noindent 案:此段是進而由心之知用以明超越心之虛明純一。滯於見聞即是「存象之心」,不滯於于見聞,即是虛明純一之心。「象」,依橫渠用語觀之,即〈太和篇〉:「散殊而可象為氣,清通而不可象為神」,「盈天地之間者法象而已」,「凡天地法象皆神化之糟粕爾」,諸語中之象或法象,即散殊之物象也。「由象識心」是經由物象以識心。如何能經由物象以識心?此當關聯著「見聞之知乃物交而知」來了解。即經由「與物交」來了解,不是單單經由「物」來了解。〈太和篇〉云:「至靜無感,性之淵源。有識有知,物交之客感爾。」是則由於與物交而成之「客感」而表現出識與知。識與知即心之作用也。吾人由於與物交而知物,亦即由於此「知物」之關係而反示心之活動也。心在感觸經驗中活動,常逐物而復留滯物之影像於心中。此時,心中完全為物象所充塞。物之影像亦物象也,以今語言之,即心中之觀念或意象。此時心中完全是一些觀念、意象之堆集。若由此堆集之觀念意象來識心,則必「徇象喪心」。隨象之變動(生滅)而變動,隨象之紛歧而紛歧,心之自主性完全喪失,心完全物化,雖云有心,實即等于喪心。故云:「由象識心,徇象喪心」。又云:「存象之心,亦象而已」。佛家唯識宗說心者集聚義,即「存象之心」也。此是識心,並非真心。但實則「知象者心」。知「象」的是心,即在此知象上即可見出心知是超越在「象」的上面,而象是在下面,心知是能、是主,象是所、是賓。能如此反顯,則心見矣。不能如此反顯,則因見聞而滯於見聞,即存象而徇於象,心同於物,實非心也。「存象之心」即心理

\newpage\thispagestyle{empty}\addtocounter{page}{-1}\vspace*{-12mm}\begin{center}\noindent
\includegraphics[clip, trim=146pt 134pt 139pt 240pt, height=162mm]{ocr-input/image-2345.png}\end{center}

\newpage\markright{第二部 \quad 分論一 \quad 第二章 \quad 張橫渠對於「天道性命相貫通」之展示}

\noindent 學的心,經驗的心,亦即識心、習心、成心也。後來黃道周即謂「意、識、情欲,是心邊物,初不是心」。(《榕壇問業》卷十二)。而横渠早已先黃道周而言「存象之心,亦象而已,謂之心可乎?」此可見先後同揆也。亦可見宋、明儒所言的心(伊川、朱子除外)總不是心理學的心,而必須是超越的道德本心也。必見到它的主動性、純一性與虛明性,方算是見到心。心之知象固由物交之聞見而顯,但滯於聞見與不滯於聞見,卻是聖凡之關鍵。在這關鍵上,即有盡心知性之工夫在。這不是知道的多少與廣狹的問題,總之,這不是知識的問題,乃是道德心靈是否能躍起之問題。橫渠雖由知象之知來說心,但目的卻不在說知識,而在說超越的道德本心之體物而不遺,故必分別「見聞之知」與「德性之知」也。

\subsection{「合內外於耳目之外」}

〈大心篇〉繼上復云:

\begin{quotation}\kaishu 人謂己有知,由耳目有受也。人之有受,由內外之合也。知
合內外於耳目之外,則其知也過人遠矣。\end{quotation}

\noindent 案:見聞之知是由耳目與外物接,有接即有受。此亦是合內外。但這合內外是主體之心經由耳目之官與外物合,是平列的、對待的,因而亦是關聯的合,故亦必為生理器官與外物所限。能超越這拘限而「合內外於耳目之外」,則即為德性之知之合內外。故云:「知合內外於耳目之外,則其知也過人遠矣。」此「知」即德性之知也。但德性之知之合內外並不是平列的、對待的、關聯的合,而是

\newpage\thispagestyle{empty}\addtocounter{page}{-1}\vspace*{-12mm}\begin{center}\noindent
\includegraphics[clip, trim=185pt 170pt 141pt 237pt, height=162mm]{ocr-input/image-2349.png}\end{center}

\newpage

\noindent 隨超越的道德本心之「遍體天下之物而不遺」而為一體(心體性體)之所貫,一心之朗照,這是攝物歸心而為絕對的、立體的、無外的合,這是「萬物皆備於我」的合,這不是在關聯方式中的合,因而嚴格講,亦無所謂合,而只是由超越形限而來之仁心感通之不隔。若依明道之口吻說,合猶是二本,而這卻只是一本之無外。在此,合是虛說,言其並無兩端之關係的合之實義,因而亦可以說這是消極意義的合。但自「道德的形上學」言,這消極意義的合,卻是真實不隔的合,此真達到「一」的境界,故又可說是積極的合,此合不是兩端底關係,而只是一體遞潤而無外之一。德性之知即隨仁體之如是潤而如是知也。故此知不是在主客關係中呈現,它無特定之物為其對象,因而其心知主體亦不為特定之物所限,它是超越了主客關係之形式而消化了主客相對之主體相與客體相,它是朗現無對的心體大主之朗照與遍潤。前第四段引〈誠明篇〉云:「天人異知不足以盡明」,故通天人,合內外,不見乎小大之別,方是誠明之知也。

\subsection{心之郛廓義與形著義及耳目不為累而為心知之發竅}

(大心篇〉繼上復云:

\begin{quotation}\kaishu 天之明莫大於日,故有目接之,不知其幾萬里之高也。天之
聲莫大於雷霆,故有耳屬之,莫知其幾萬里之遠也。天之不
禦莫大於太虛,故心知廓之,莫究其極也。人病其以耳目見
聞累其心,而不務盡其心。故思盡其心者,必知心所從來而
後能。耳目雖爲性累,然合內外之德,知其為啟之之要也。\end{quotation}

\newpage\thispagestyle{empty}\addtocounter{page}{-1}\vspace*{-12mm}\begin{center}\noindent
\includegraphics[clip, trim=150pt 149pt 149pt 233pt, height=162mm]{ocr-input/image-2353.png}\end{center}

\newpage\markright{第二部 \quad 分論一 \quad 第二章 \quad 張橫渠對於「天道性命相貫通」之展示}

\noindent 案:此段首先以耳目之接天之明與屬天之聲類比心知郛廓天道生化之無極,以明心能盡性,心為形著原則。次則進而明心能盡性,則通天人、合內外、不見乎小大之別,由此即可由「耳目之有受」反而積極地看耳目之與物接,即,耳目不為累,反而為開啟天德良知之機要,此即後來王陽明所說之「良知之發竅」也。

日光遍照,象徵天之明,此是客觀地說。「有目接之,不知其幾萬里之高也」,此是主觀地說。主觀地說,即由目之接以證實天之明之高也。雷霆之聲震寰宇,此象徵天之聲,此是客觀地說。「有耳屬之,莫知其幾萬里之遠也」,此是主觀地說,由耳之屬以證實天之聲之遠也。心知之對於天道亦然。「天之不禦,莫大於太虛」,此是客觀地說。「不禦」即《易傳》「夫易廣矣大矣,以言乎遠則不禦」之「不禦」,亦孟子「沛然莫之能禦」之意,言天道生化之無窮盡也,無有足以禦而限之者也。天道何以能如此?以清通而至虛故也。天以太虛,故其生化莫之能禦。此是客觀地說。而「心知廓之,莫究其極也」,則是以心知之誠明契此生化之無盡而印證、證實此無盡,亦即真實化此無盡,亦如目之接日明而證實其為高,耳之接雷霆而證實其為遠。此是主觀地說。

「心知廓之」,廓,亦可如明道所謂「君子之學莫若廓然而大公,物來而順應」之廓,此是開朗義;亦可如邵堯夫所謂「心者性之郛廓」之廓,此是範圍義與形著義。以是,廓有三義:開朗、範圍與形著。城圈之外為廓,故對城圈言,廓有開朗義。但說廓說郊,其本身即有範圍義。由範圍而說形著。橫渠所說「心知廓之,莫究其極」之廓可備此三義。超見聞之「心知」遍體天下之物而不遺,自然開朗無外。以其開朗無外,故能相應「天之不禦」而知其

\newpage\thispagestyle{empty}\addtocounter{page}{-1}\vspace*{-12mm}\begin{center}\noindent
\includegraphics[clip, trim=183pt 163pt 134pt 233pt, height=162mm]{ocr-input/image-2357.png}\end{center}

\newpage

\noindent 為無窮盡。相應其「不禦」之無盡即是郛廓而範圍之。「廓之」以「相應」定。此如「範圍天地之化而不過」之範圍,此範圍亦是「相應」義。故此「範圍」是比擬說,並非有形之一定範圍也。故其實義即是「形著」,言心知相應其無盡而證實之,證實之即形著之。客觀自如者須待主觀之形著而得其真實義與具體義。故「心知廓之」之廓本於超越的道德本心之無外,而落實於對於天道之形著。心之作用即在形著,故横渠言「心能盡性」也,而孟子亦言「盡心知性知天」也。虛說之郛廓範圍義即在導成此「形著」之實義。伊川亦言「性之有形者謂之心」,(此須別解)。後來胡五峰言「盡心以成性」,以及明末劉蕺山言「性無性」,「性因心而名」,「此性之所以為上,而心其形之者與」等語(參看劉氏〈原性〉、〈原學〉等文)皆明言心之形著義。而此義卻不能為朱子所把握。他只由郛廓範圍義,而以橫渠之「心統性情」一語為根據而講心之「統」義,(此「統」是關聯地統,亦非必是橫渠之原意),與「具」義,(心具眾理,此具亦是關聯地具,非「心即理也」之創發地具),而對於形著義則不能解。此將在後講述伊川、朱子、五峰、蕺山時自明。

邵堯夫云:「性者道之形體,心者性之廓郛,身者心之區宇,物者身之舟車。」朱子最欣賞此四語,然其理解實不能盡其意。此四語實只在表明道之步步形著化與內在化。光隴侗地說道,這只是客觀地、形式地說。道必落實於性,所謂結穴於性,以性為其體。此體即內容義、本質義,亦如德之於道。故曰:「性者道之形體。」形體是虛說之比擬詞。豈是性而真可有形體耶?故知此「形體」只是體性義、本質義、內容義。言性猶是客觀地、形式地表

\newpage\thispagestyle{empty}\addtocounter{page}{-1}\vspace*{-12mm}\begin{center}\noindent
\includegraphics[clip, trim=143pt 142pt 146pt 230pt, height=162mm]{ocr-input/image-2361.png}\end{center}

\newpage\markright{第二部 \quad 分論一 \quad 第二章 \quad 張橫渠對於「天道性命相貫通」之展示}

\noindent 示,而真正主觀地、具體地表示,則在心之形著。故曰:「心者性之郛廓。」即融性於心,性始得其真實義與具體義。故心為性之郛廓,雖有範圍義,但實在導成形著義。在盡心知性或至誠盡性之過程中,心之形著性體,形著之即函限定之。但到心體性體全體朗現時,即無限定之形著,而唯是一心之沛然,亦可以說唯是一性之昭然。此義是會通孟子與《中庸》《易傳》說。若單自孟子說,則當下心即是性,當下即是一心之沛然。通過「心性當下是一」來知天,天較複雜。在此亦可說心性是天之郛廓。道、性、心,步步落實,是將《中庸》《易傳》會通到孟子。北宋諸儒下屆胡五峰及劉蕺山皆是此路。惟陸、王是單自孟子說。就先秦儒家之發展說,是先有孟子,然後再澈至《中庸》、《易傳》之境。而澈至《中庸》、《易傳》之境,始有客觀地自天道建立性體之一義。北宋諸儒繼承此義,通於孟子,始有心能盡性,盡心以成性,心性對揚之心之形著義,而其究也亦是心性合一。發展至陸、王,則單自孟子之路入,無此心性對揚之心之形著義,直下即是心性是一。直下即是一心之沛然,直下即是心體之無外,即是性體之昭然。此兩路原是一圓圈。惟自發展上,始有此兩路耳。然其究也是一。蓋無孔子之踐仁知天、孟子之盡心知性知天,亦不能有《中庸》、《易傳》之客觀地由天道建立性體之一義。而無此一義,則孔子之踐仁知天、孟子之盡心知性知天,即不能算澈盡而至圓滿,而孟子之心性亦不必能澈盡而至其絕對普遍性,人或以為有封限,只限於人之道德心靈耳。然孟子已言「萬物皆備於我」,是則其心性之絕對普遍性並無虛歉,其盡心知性知天實不只是遙遙地知天,實足以證實天之所以為天,而在本質上實同於其所說之心性。此門在孟子實已開

\newpage\thispagestyle{empty}\addtocounter{page}{-1}\vspace*{-12mm}\begin{center}\noindent
\includegraphics[clip, trim=159pt 159pt 140pt 227pt, height=162mm]{ocr-input/image-2365.png}\end{center}

\newpage

\noindent 啟,惟至《中庸》《易傳》始正式客觀地說出耳。正式客觀地說出此義,即是將孟子所說之心性直下澈至天道之奧處,故有客觀地自天道建立性體之一義。然此義一成,即必然函有主觀說與客觀說之對揚,必然函有心性之對揚,必然函有心之形著義,以期最後重返於合一。心性合一,即心與性天之合一,此時性天為一面,是客觀地說者。此種合一是把孔子之踐仁知天,孟子之盡心知性知天所敞開之合一之門正式挺立起而澈至其圓滿之極者。此能澈至其圓滿之極,則陸、王即可繼起而單自孟子之路澈至其圓滿之極。故自發展言,北宋諸儒下屆胡五峰是此圓滿之前一階段,是繼承《中庸〉、《易傳〉而再返到孟子以建立心之形著義,而陸、王則是後一階段,是直承孟子而澈至《中庸》、《易傳》之境,而心性對揚、心之形著義即不必要。從這裡說出去,不需要心性對揚、心之形著義;從那裡說進來,始需要有此心性對揚、心之形著義之過渡。此發展中之兩路正好形成一圓圈,此是發展中之圓圈,從此圓圈以明心性天是一。而理上言之、直下言之之畢竟是一,明道之一本論足以盡之矣。陸、王亦是此直下言之之一本論,惟陸、王是在發展中逼出的,故吾得以自發展上而言之。無論是圓圈所成之一本,或是直下言之之一本,朱子於此皆不能無憾焉。惟因朱子於此之有憾,始顯出陸、王之一本之挺拔。明道亦未至此光暢之境。較之陸、王,明道似只是圓悟妙悟之模型耳。

「身者心之區宇,物者身之舟車」,此言道、性、心之內在化,以極成「踐形」之體用不二之圓融論。「身」即是心之所處。心雖超乎身,亦不離乎身。光超越而不內在,是抽象地說的心。踐形之極,全身是道,全道是身。全身是道,則身雖有限而實具無限

\newpage\thispagestyle{empty}\addtocounter{page}{-1}\vspace*{-12mm}\begin{center}\noindent
\includegraphics[clip, trim=161pt 141pt 129pt 236pt, height=162mm]{ocr-input/image-2369.png}\end{center}

\newpage\markright{第二部 \quad 分論一 \quad 第二章 \quad 張橫渠對於「天道性命相貫通」之展示}

\noindent 之意義,而身不為累。全道是身,則醉面盎背,施於四體,道(性心)得其具體之呈現,而性心不空掛。身與物連屬於一起,身假物以行,故曰:「物者身之舟車。」連屬家、國、天下,甚至萬事萬物,而為一身,身之舟車即是道、性、心之所妙所通而為一體者也。由是而道性心之體物而不遺之無外亦於茲而顯,而踐形之極之「體用不二」之圓融論亦於焉以極成。此義成,則耳目不為累,而為道、性、心之發竅,而為「合內外之德」之「啟之之要」,亦不煩解而可明矣。此完全是「本體、宇宙論的」立體直貫之成物與成身。在此,物與身直通其根於道、性、心,而不得視為無根無體之幻妄;而凡經由物與身之發竅而出者皆本於道性心而來,皆是道性心之所成,不得局限於物與身一己之小而視為物與身所自具有之質性也。此兩義即是以下兩段之所說。

\subsection{「以性成身」與「因身發智」}

〈大心篇〉繼上文復云:

\begin{quotation}\kaishu 成吾身者,天之神也。不知以性成身,而自謂因身發智,貪
天功為己力,吾不知其知〔智】也。民何知哉?因物同異相
形,萬變相感,耳目內外之合,貪天功而自謂己知爾。

體物、體身,道之本也。身而體道,其為人也大矣。道能物
身。故大。不能物身、而累於身,則藐乎其卑矣。能以天體
身,則能體物也不疑。

成心忘,然後可與進於道。化則無成心矣。成心者,意之謂
與?無成心者,時中而已矣。心存,無盡性之理。故聖不可\end{quotation}

\newpage\thispagestyle{empty}\addtocounter{page}{-1}\vspace*{-12mm}\begin{center}\noindent
\includegraphics[clip, trim=159pt 164pt 145pt 230pt, height=162mm]{ocr-input/image-2373.png}\end{center}

\newpage

\begin{quotation}\kaishu 知謂神。〔案:此言「心存」即「成心」存】。

以我視物,則我大。以道體物我,則道大。故君子之大也,
大於道。大於我者容不免狂而已。\end{quotation}

\noindent 案:此四小段為一整段,皆環繞「本體宇宙論的」立體直貫之成物成身之義而說。本體宇宙論地說,「成吾身者,天之神也」。天道以太虚之神而能生化,故物與己身皆神之所成也。「身」即「己身」義,不必是身體也。言天、言道、言虛、言神,皆結穴於性。故太虛神體成吾身,則吾即當盡己之性以成吾身,即盡性以完成自己也。盡性以成己,則凡吾身之所發者,皆性體之所為,不得拘於區區一己之小,而謂是吾身之所發。若不然,則是貪天功以為己力。即就所發之智言,則是一己之私智,焉得謂之天智?故云:「不知以性成身,而自謂因身發智,貪天功為己力,吾不知其智也。」智之用為「知」。凡人因何而有知乎?不過「因物同異相形,萬變相感,耳目內外之合」而有其「知」爾。此雖是物交之客感,〈太和篇〉所謂「有識有知,物交之客感爾」,而實則是太虛神體之發用,「耳目內外之合」只是其發竅耳,「同異相形,萬變相感」,亦是太虛神體之妙通而然也。故智之知用皆非區區一己之身之所有,皆是誠明性體心體之所發也。澈通於耳目與超耳目,則無不是天德之良知也。若不知此義,而自謂是己身之知,則是貪天功也。貪而忘本,則智是私智,知亦小耳。

知「成吾身者,天之神也」,又知「以性成身」,則實踐地說,即可言「體物、體身,道之本也」之義。「體物體身」,體即體物不遺之「體」。就「道之本也」說,此語意同於「孝弟也者,

\newpage\thispagestyle{empty}\addtocounter{page}{-1}\vspace*{-12mm}\begin{center}\noindent
\includegraphics[clip, trim=160pt 140pt 124pt 238pt, height=162mm]{ocr-input/image-2377.png}\end{center}

\newpage\markright{第二部 \quad 分論一 \quad 第二章 \quad 張橫渠對於「天道性命相貫通」之展示}

\noindent 其為仁之本與」之句。實踐地說,「以性成身」即是以性「體身」而成己,「體物」即是以性「體物」而成物。故「體物、體身」是行道之基本,是表現道之本質的關鍵。以性體身而成身,則反而即可說身是「體道」之身。「體道」即體而有之之謂,亦即能表現道於己身之謂。身而體道,則即為大人。不體道,則只是一軀殼之身。就道說,若其道能「物身」而為主於身以成身,則其道大而正。若不能「物身」而「累於身」,則是苟偷、卑陋、狹小偏曲之道。「能以天體身,則能體物也不疑」,此天即道、即虛、即神,亦即吾人之性。「以天體身」即是盡性以成己。能盡性成己,自能體物以成物。此猶是《中庸》盡己性、盡人性、盡物性,以至參天地贊化育之義。

盡性以成己成物,其關鍵在「盡心」。盡心即不囿於見聞之狹而充分體現其超越的道德本心之意。本心與「成心」對言。「成心」語出莊子(齊物論〉「夫隨其成心而師之,誰獨且無師乎」句中之「成心」。成心即習心、識心。橫渠解為「意必固我」之意。人到超脫見聞之累而能保持本心之虛明純一,則「成心」化矣。化則無意必固我之私,而唯是心體流行。隨時而中。故曰:「無成心者,時中而已矣。」

第四小段易解,不贅。

\subsection{辨佛}

〈大心篇〉繼上文復云:

\begin{quotation}\kaishu 釋氏不知天命,而以心法起滅天地,以小緣大,以末緣本。\end{quotation}

\newpage\thispagestyle{empty}\addtocounter{page}{-1}\vspace*{-12mm}\begin{center}\noindent
\includegraphics[clip, trim=164pt 167pt 146pt 233pt, height=162mm]{ocr-input/image-2381.png}\end{center}

\newpage

\begin{quotation}\kaishu 其不能窮,而謂之幻妄。所謂疑冰者與?(原註:夏蟲疑
冰)。

釋氏妄意天性,而不知範圍天用,反以六根之微,因緣天
地。明不能盡,則誣天地日月為幻妄。蔽其用於一身之小,
溺其志於虛空之大。此所以語大語小,流遁失中。其過於大
也,塵芥六合。其蔽於小也,夢幻人世。謂之窮理可乎?不
知窮理,而謂盡性可乎?謂之無不知可乎?塵芥六合,謂天
地為有窮也。夢幻人世,明不能究所從也。\end{quotation}

\noindent 案:〈大心篇〉全文止于此。此段辨佛,不出〈太和篇〉之所云。其詳仍請參看附錄:〈佛家體用義之衡定〉。

\section{綜論心性合一(是一)之模型}

綜〈大心篇〉所論之心而觀之,橫渠顯是本孔子之仁與孟子之本心即性而言一超越的、形而上的普遍之本心。此本心如不為見聞(耳目之官)所拘蔽,自能體天下之物而不遺而為其體。此是一絕對普遍的本體。心即是體,故曰心體。此是主觀地、存在地言之,由其體物不遺而見其為體。〈天道篇〉:「天體物不遺猶仁體事無不在」,俱是由體物體事而見其為體。天道之「體物不遺」是客觀地、本體宇宙論地說;仁之「體事無不在」是主觀地、實踐地說。主觀地、實踐地說,即所以明「心能盡性」之超越的、形上的普遍本心也。故「天大無外」,性大無外,心亦大而無外。此即函心性天之主觀地、實踐地說之合一,而亦竟直是一也。字面上,橫渠雖

\newpage\thispagestyle{empty}\addtocounter{page}{-1}\vspace*{-12mm}\begin{center}\noindent
\includegraphics[clip, trim=164pt 133pt 124pt 242pt, height=162mm]{ocr-input/image-2385.png}\end{center}

\newpage\markright{第二部 \quad 分論一 \quad 第二章 \quad 張橫渠對於「天道性命相貫通」之展示}

\noindent 未明言至此,然意實含之,而亦甚顯明。故以「孟子謂盡心則知性知天」為證。「心能盡性」,即如能盡此本心,自能盡性也。孟子謂盡心則知性知天,知亦是在盡中知。盡心與盡性、知性是一。盡心盡性與知天,在孟子其初似不一,猶有一種超越的距離在,所謂天道遠也,然自「大而化之」「萬物皆備於我」之究竟言,則畢竟亦是一。盡心盡性即知天道而盡天道,只此心性之沛然莫之能禦便是天道。(此從正面純以於穆不已之天命之體說天,即純以天德神體或太虛神體說天。若帶著氣化說,則仍有超越的意味,由此說「命」,說「後天而奉天時」)「心能盡性」之心性對言只是為明「盡」字之言辭上的方便權設,而實則盡其本心即盡性,性體之內容全在心體見;推之,天德神體、於穆不已之體亦全在心體見。故明道云;「只心便是天」也。此自是明道圓頓智慧之透闢,横渠字面上未說至此,然意實函之,已明備此意。此一義理規範全是根據孔子之仁與孟子之本心即性而成立。只因朱子不能知此底蘊,而又異解(誤解)孟子,見故「大其心」之辭語即生厭心。

橫渠雖於〈太和篇〉先客觀地、本體宇宙論地自《易傳》之路入,重在闡明「有無隱顯、神化、性命通一無二」,以抵禦佛老,好似客觀面重、主觀面輕,然及言「聖人盡道其間,兼體而不累者,存神其至矣」,由此一面展開,而至言「心能盡性」盡心易氣以成性,則又自《中庸》《易傳》而回歸於《論》、《孟》矣。是則主觀面亦並不輕,其客觀地、本體宇宙論地言「有無隱顯、神化、性命通一無二」自始即未空頭言也,自始即緊扣主觀面通而一之而言也。人只浮光掠影看(太和篇〉之辭語,為其言太和太虛、言神言氣所吸住而不究其實,又不解其言「兼體無累」、

\newpage\thispagestyle{empty}\addtocounter{page}{-1}\vspace*{-12mm}\begin{center}\noindent
\includegraphics[clip, trim=183pt 159pt 146pt 252pt, height=162mm]{ocr-input/image-2389.png}\end{center}

\newpage

\noindent 「參和不偏」之密義、實義,又不能解其〈大心篇〉之真切義以及〈誠明篇〉之「心能盡性」義與「繼善成性」義,遂謂其客觀面重、主觀面輕,或甚至謂其空頭言宇宙論,更甚至謂其為唯氣論矣。此皆讀之而未入,或甚至根本未讀,而妄意其為如此耳,未盡其實也。

橫渠言仁猶不只〈天道篇〉「天體物不遺猶仁體事無不在」之一語。〈神化篇第四〉云:

\begin{quotation}\kaishu 敦厚而不化,有體而無用也。化而自失焉,徇物而喪己也。
大德敦化,然後仁智一而聖人之事備。性性為能存神,物物
為能過化。

義以反經為本,經正則精。仁以敦化為深,化行則顯。義入
神,動一靜也。仁敦化,靜一動也。仁敦化、則無體,義入
神、則無方。

大可為也,大而化不可為也。在熟而已。《易》謂窮神知
化,乃德盛仁熟之致,非智力能強也。

神不可致思,存焉可也。化不可助長,順焉可也。存虛明,
久至德,順變化,達時中,仁之至義之盡也。知微知彰,不
舍而繼其善,然後可以成人性矣。\end{quotation}

\noindent 此(神化篇〉,謂之講神化可,謂之講仁亦可。「大德敦化」即「仁敦化」。是則神化即「德盛仁熟之致」也。

〈至當篇第九〉亦云:

\newpage\thispagestyle{empty}\addtocounter{page}{-1}\vspace*{-12mm}\begin{center}\noindent
\includegraphics[clip, trim=158pt 188pt 137pt 244pt, height=162mm]{ocr-input/image-2394.png}\end{center}

\newpage\markright{第二部 \quad 分論一 \quad 第二章 \quad 張橫渠對於「天道性命相貫通」之展示}

\begin{quotation}\kaishu 道遠人,則不仁。

性天經,然後仁義行。故曰有父子君臣上下,然後禮義有所
錯。仁通極其性,故能致養而靜以安。義致行其知,故能盡
文而動以變。\end{quotation}

\noindent 「性天經」意即性與天得其正之意,亦即正位居體之意。「性天」亦可說性即是天,故曰「性天」。「仁通極其性」與「天所性者通極於道」、「天所命者通極於性」為同一句法。天道性命通而為一,仁性亦通而為一也。「仁通極其性,故能致養而靜以安」,此根據「仁者安仁」、「仁者靜」、「仁者壽」、「仁者樂山」以及「智及仁守」等義而說,故仁以靜為體也。然及「仁敦化」,則「靜一動也」。「靜一動」猶言「靜而動」。「靜而動」其實亦是「動而無動」之神動。「義致行其知,故能盡文而動以變」,此即「義以方外」之義。「致行其知」即致行「虛明照鑑,神之明也」(〈神化篇〉)之神知。「盡文而動以變」即方外盡文而善應也。故義以動為體。然及「義入神」,則「動一靜也」。「動一靜」猶言「動而靜」。「動而靜」其實亦是「靜而無靜」之神靜。

〈三十篇第十一〉亦云:

\begin{quotation}\kaishu 仲由樂善,故車馬衣裘喜與賢者共敝。顏子樂進〔案:意謂
樂進其德〕,故願無伐善施勞。聖人樂天,故合內外而成其
仁。\end{quotation}

\noindent 伊川於此謂:「先觀子路、顏淵之言,後觀聖人之言,分明聖人是

\newpage\thispagestyle{empty}\addtocounter{page}{-1}\vspace*{-12mm}\begin{center}\noindent
\includegraphics[clip, trim=170pt 162pt 140pt 231pt, height=162mm]{ocr-input/image-2398.png}\end{center}

\newpage

\noindent 天地氣象。」此品題不誤,故橫渠亦言「聖人樂天」。然此只是品題上對於氣象之稱讚。至於說到仁之實,則伊川、朱子之理解與橫渠、明道不同。在橫渠與明道,仁實是「通極於性」而為仁心仁體之無外,仁即是「本心即性」,亦即是「於穆不已」、「純亦不已」之體,亦即是「天德神體」或「太虛神體」之主觀地、實踐地言之者。故橫渠云「聖人樂天,故合内外而成其仁」,「仁體事無不在」,本無內外也。而明道亦以「渾然與物同體」、「以天地萬物為一體」、說仁也。而伊川、朱子則只解為「愛之理」,則與此異矣。故其贊聖人氣象與其所理解之仁不能相接也。

〈有德篇第十二〉復云:

\begin{quotation}\kaishu 不穿窬義也,謂非其有而取之曰盜亦義也。惻隱仁也,如
天亦仁也。故擴而充之,不可勝用。\end{quotation}

\noindent 案:此顯與伊川朱子之言仁異。

《性理拾遺·孟子說》亦云:

\begin{quotation}\kaishu 敦篤虛靜者仁之本。不輕妄,則是敦篤也。無所繫閡昏塞,
則是虛靜也。此難以頓悟。苟知之,須久於道,實體之,方
知其味。夫仁亦在乎熟之而已。\end{quotation}

\noindent 案:「無所繫閡昏塞」,其義理根據,在橫渠是「虛靜」(天德神體,太虛神體之虛靜),在明道即是覺潤無方也。「敦篤虛靜者仁之本」,即是仁之自體性也。明道自「一體」說仁,自「覺」(不

\newpage\thispagestyle{empty}\addtocounter{page}{-1}\vspace*{-12mm}\begin{center}\noindent
\includegraphics[clip, trim=158pt 140pt 132pt 233pt, height=162mm]{ocr-input/image-2402.png}\end{center}

\newpage\markright{第二部 \quad 分論一 \quad 第二章 \quad 張橫渠對於「天道性命相貫通」之展示}

\noindent 麻木、有感覺、能感通)說仁;橫渠亦自「一體」說仁,惟未曾想到「不麻木」之覺,而卻自「虛靜」說仁,而「虛靜」亦自清通而言也。「無所繫閡昏塞」即清通,即虛靜,此非仁心感通無礙而何?此在其提醒人處,雖不及言「覺」言「不麻木」之警策,然實義未始有異也。覺、不麻木,固切於仁,「敦篤虛靜」、清通「無繫閡」、無「昏塞」,亦切於仁也;而其為「無內外」(此函渾然一體)、「體事無不在」之仁體則一也。然朱子則極不喜此「仁體」一詞,又不贊成自「一體」說仁,尤力斥以「覺」訓仁。至於橫渠說仁,則更為其所不註意矣。

以上所錄,若類聚於一起,亦可題曰橫渠之「仁化篇」。由此觀之,其言仁亦不輕矣,而且與明道為同一思路。其如此言仁,當橫渠在時,其弟子呂與叔恐亦未有所警悟。橫渠沒,呂與叔赴洛陽見明道,明道始為之言「學者須先識仁」一段,此是有名之〈識仁篇〉。故明道之〈識仁篇〉有當機之指點性,而警策性亦大。橫渠之言仁只是在《正蒙》各篇中如此說,當機性不足,故呂與叔亦無所聞也。然義理之實則固與明道無以異也。明道〈識仁篇〉已言「〈訂頑〉〔〈西銘〉】意思乃備言此體」。實則〈西銘〉意思只是顯示(烘托)出此體,真正「備言此體」者恐在上錄之《正蒙》各篇也。據蘇丙之〈正蒙序〉以及呂與叔(大臨)之(横渠先生行狀〉所記,《正蒙》全文、橫渠晚年(將卒前)始出而示人,是則橫渠在時,明道或未及窺全豹也。人不能詳讀橫渠書,而明道又因朱子之傳承伊川,未能細簡其兄弟之異,而成為隱形者,如是,人遂以為二程爲一路,而絕異於橫渠。實則伊川、朱子與橫渠絕異,而明道與橫渠在根底上固是一路,而未始有異也。勿因明道誤解橫

\newpage\thispagestyle{empty}\addtocounter{page}{-1}\vspace*{-12mm}\begin{center}\noindent
\includegraphics[clip, trim=167pt 153pt 143pt 238pt, height=162mm]{ocr-input/image-2406.png}\end{center}

\newpage

\noindent 渠之「清虛一大」而即謂彼二人根本上有異也。(〈太和篇〉初稿可能更多隱晦,亦可能因明道之議而有修改。然據今日觀之,明道決是誤解。至伊川、朱子更不待言)明道之譏議橫渠,除「清虛一大」之誤解外,大抵是小疵病,皆是關於表示上之圓頓否者,詳見明道章。至於伊川、朱子之譏議則是根本上心態有異,因而亦函義理系統之異也。人混漫浮皮觀之,何能究其實?

總之,由「兼體無累」、「參和不偏」、「性其總合兩也」,以至「心能盡性」盡心易氣以成性,以及「仁體事無不在」,「仁以敦化為深、化行則顯」,「敦篤虛靜者仁之本」,凡此一系之義理皆表示主觀面並不輕,皆表示主客觀之統一,是亦能由《中庸〉、《易傳》而回歸於《論》《孟》,比濂溪為圓滿多矣。雖不及明道之清澈圓熟,而沈雄弘偉則過之。明道雖亦客觀地本《中庸》、《易傳〉言天道、天理,然〈識仁〉、〈定性〉俱是主觀地言心體性體,又以其圓融之智慧,盛言一本之義,則主觀面與客觀面俱已飽滿而無虛歉,故終於以「一本」為究竟了義也。至此,心性天為一之模型,所謂圓頓之教,徹底朗現矣。此由濂溪開始,通過橫渠,所體悟之天道性命所必至者,而亦不失先秦儒家發展之弘規。惟至伊川、朱子,遂漸泯失而歧出。凡此俱見以後各章所述。

伊川、朱子所以迷失而歧出之關鍵何在?茲仍就「心能盡性」,心性天合一之模型而明之。

在講〈大心篇〉之前,吾已明性具五義:

一、性體義:體萬物而謂之性,性即是體。

二、性能義:性體能起宇宙之生化、道德之創造(即道德行為
之純亦不已),故曰性能。性即是能。

\newpage\thispagestyle{empty}\addtocounter{page}{-1}\vspace*{-12mm}\begin{center}\noindent
\includegraphics[clip, trim=165pt 132pt 132pt 244pt, height=162mm]{ocr-input/image-2411.png}\end{center}

\newpage\markright{第二部 \quad 分論一 \quad 第二章 \quad 張橫渠對於「天道性命相貫通」之展示}

三、性理義:性體自具普遍法則,性即是理。

四、性分義:普遍法則之所命所定皆是必然之本分。自宇宙論
方面言,凡性體之所生化,皆是天命之不容已。自道德創
造言,凡道德行為皆是吾人之本分,亦當然而不容已,必
然而不可移。宇宙分内事即是己分内事。反之亦然。性所
定之大分即日性分。

五、性覺義:太虛寂感之神之虛明照鑑即是心。依此而言性覺
義。性之全體即是靈知明覺。

\noindent 凡此五義,任一義皆盡性體之全體:性全體是體,全體是能,全體是理,全體是分,全體是覺。任一義亦皆通其他諸義:性之為體,通能、理、分覺而為體,性之為能,通體、理、分、覺而為能;性之為理,通體、能、分、覺而為理;性之為分,通體、能、理、覺而為分;性之為覺,通體、理、分而為覺。故任一義皆是具體的普遍,非抽象的普遍。

性體有此五義,是客觀地、形式地言之。自心能盡性,主觀地、實踐而亦是實際地言之,則超越的、形而上的、普遍的本心(天心)亦具此五義:

一、心體義:心體物而不遺,心即是體。

二、心能義:心以動用為性(動而無動之動),心之靈能起宇
宙之創造,或道德之創造,心即是能。

三、心理義:心之悅理義即起理義,即活動即存有,心即是
理。此是心之自律義。

四、心宰義:心之自律即主宰而貞定吾人之行為,凡道德行為
皆是心律之所命,當然而不容已、必然而不可移,此即吾

\newpage\thispagestyle{empty}\addtocounter{page}{-1}\vspace*{-12mm}\begin{center}\noindent
\includegraphics[clip, trim=162pt 163pt 146pt 233pt, height=162mm]{ocr-input/image-2415.png}\end{center}

\newpage

\begin{quotation}\kaishu 人之大分。此由心之主宰而成,非由以限之也。依成語習
慣,無心分之語,故不曰心分,而曰心宰。心宰即性分
也。
五、心存有義:心亦動亦有,即動即有。心即是存有(實
有),即是存在之存在性,存在原則:使一道德行為存在
者,即是使天地萬物存在者。心即存有,心而性矣。\end{quotation}

\noindent 凡此五義,任一義皆盡心體之全體:心全體是體,全體是能,全體是理,全體是主宰,全體是存有(實體性的存有)。任一義亦皆通其他諸義:心之為體、理、宰、有而為體;心之為能,通體、理、宰、有而為能;心之為理,通體、能、宰、有而為理;心之為宰,通體、能、理、有而為宰;心之為有,通體、能、理、宰而為有。故任一義皆是具體的普遍,而非抽象的普遍。

自性而言之,綜此五義而曰性體。綜性體之整全言之,亦得曰理。此即是太極之為理。朱子說太極是極至之不誤也,然朱子說太極是極至之理卻成「只是理」(但理),而能義、覺義則抽掉,即將太虛寂感之神義抽掉而屬之氣。依此,太極成為不動不靜、無所謂動靜之死物。太極對動之事而為動之所以然,即為動之理,因而說太極有動之理(太極為綜說故)。太極對靜之事而為靜之所以然,即為靜之理,因而說太極有靜之理。而動靜之事之實,則屬之氣。太極本身則不動不靜,亦無所謂動靜。此無所謂動靜非「動而無動、靜而無靜」之神。是則太極只是理,只是形式的所以然(雖亦是超越的),只是靜態之存有,而非即活動即存有之動態的存有。此是對於道體、性體體悟上之不足,此不合先秦儒家由「維天之命於穆不已」之最原始之智慧而來之天命天道觀,亦不合濂溪

\newpage\thispagestyle{empty}\addtocounter{page}{-1}\vspace*{-12mm}\begin{center}\noindent
\includegraphics[clip, trim=169pt 139pt 128pt 242pt, height=162mm]{ocr-input/image-2419.png}\end{center}

\newpage\markright{第二部 \quad 分論一 \quad 第二章 \quad 張橫渠對於「天道性命相貫通」之展示}

\noindent 由誠體寂感之神以說天道,以及橫渠由太虛寂感之神以說天道性體。此即所謂迷失而傍落。抽掉寂感之神以及性能義與性覺義,即為傍落。寂感之神與性能性覺義即旁落,太極成「只是理」,而性亦成「只是理」,因而太極之為極至之理完全等同性之五義中之第三義就普遍法則而言之性理義。性與太極皆只是靜態的存有,皆只是超越的、形式的、靜態的所以然,而非超越的、具體的、動態的所以然。不過說太極、說性體,是綜言之;說到理,則曰太極含萬理,性具象理,則亦綜亦分也。綜之於性與太極,而分別表現於氣化。綜之於性與太極是一理,亦可曰統體一太極。分別表現於氣化,則有多相,亦可曰物物一太極。此朱子學之綱維也。

然就先秦儒家以及濂溪、橫渠所體悟之天道性體,則不如此。首先,綜性體之整全而謂之理,此理之層次與五義中第三義就普遍法則而說之理之層次不同,前者在層次上是高於後者,前者是綜性體之全而謂之理,後者是偏就性之全中之某一義(某一面)而說是理。

綜性體之全而謂之理亦可由「超越的所以然」而表示。譬如對人言,此性體即是吾人之所以為人之「超越的所以然」(理),即是吾人發展道德人格而成聖之「超越的所以然」(理),即是道德實踐所以可能之「超越的所以然」(理,根據)。總對天地萬物言,此性體即是天地之化之「超越的所以然」(理,極至之理,生化之理)。由這「所以然」之形式的陳述,自必函一理之意義。但如此所說之理之意義只是由形式的陳述所賦予之形式的意義,其實際內容與具體意義並無任何表明。由這「超越的所以然」可知其所實指之形式意義之理是「存在之理」(存在原則,存在之存在

\newpage\thispagestyle{empty}\addtocounter{page}{-1}\vspace*{-12mm}\begin{center}\noindent
\includegraphics[clip, trim=163pt 166pt 152pt 233pt, height=162mm]{ocr-input/image-2423.png}\end{center}

\newpage

\noindent 性),不是定義中所表示的「內在的描述的所以然」所表示之形構之理。(形構之理是類名,而存在之理不是類名)但亦只能知其為「存在之理」,而此存在之理之實義仍未表示出。此則並非只是一「超越的所以然」所能盡。因此,要明其實義仍須要進一步有具體之體悟與具體之規定。依先秦儒家以及濂溪、橫渠之所體悟與規定,此「超越的所以然」是存在之理同時即是能創生能起用之生化之理——是心性合一者,是具備五義者,是即活動即存有者,是超越的動態的所以然者。而依朱子之體悟與規定,則只是理,而不是心性合一者;是只具備性體、性理、性分之三義,而不具備性能與性覺者,是只存有而不活動者;是超越的靜態的所以然,而無所謂動靜;而不動不靜者,而非是「動而無動、靜而無靜」之動態的所以然。依此,其為存在之理只是靜態的存在之理,而非同時即是生化之理者。縱就氣之生化之實亦可言其為其生化之理,亦只是靜態地為其生化之理,而不是能創生能起用(神用、妙用)之之動態的生化之理。

就五義中第三義之性理義而言理,此之為理是就普遍法則而為言。此可只是理。然此只是性體之一義或一面。縱就此一面言,此理可成為「只是理」,然卻並不因此即謂性亦只是理。因任一義皆通其他諸義,性之為理是通性體、性能、性分、性覺而為理,此並不表示性只是理,而卻表示是心性合一者,是即活動即存有而不只是存有者,是動態的所以然,而非靜態的所以然者。

就兩層次而言性是理,皆不表示性只是理(靜態的但理)。在朱子,似不曾覺到此兩層分言之理之不同,而只把「所以然」所表示之理等同於普遍法則之理,因而太極性體只成一個只是普通法則

\newpage\thispagestyle{empty}\addtocounter{page}{-1}\vspace*{-12mm}\begin{center}\noindent
\includegraphics[clip, trim=166pt 140pt 126pt 237pt, height=162mm]{ocr-input/image-2427.png}\end{center}

\newpage\markright{第二部 \quad 分論一 \quad 第二章 \quad 張橫渠對於「天道性命相貫通」之展示}

\noindent 之只是理,只是一靜態的存有之為理,只是一存在之靜態的存在性,不過有一相與多相而已。綜起來說是一,隨氣化而有分別表現,自此而言是多。

朱子是一條鞭的(直線的)分解表象。依其如此分解表象之體悟,遂有兩結果發生:

一、其言性或太極之為理,雖亦由「超越的所以然」而得保持其為「存在之理」,但卻是靜態的,不能起生化之妙用的,即只是靜態地為存在之理,而非動態地為存在之理。實際在生者、化者、動者、靜者,只是氣,而理則只是在背後隨著其生、化、動、靜之事而靜態地定然而規律之,而為其「存在之理」而已。(朱子講到太極時,順傳統習慣,亦言其為「萬化根本」,為萬化之源。但亦只是說太極是如此,說到性,即不見有此類辭語,亦根本不能表示性體之道德的創生義。說太極,雖有那類辭語,而依其分解表象之直線思考,說來說去,卻只成一靜態的存在之理。朱子不自覺到其所習用之「萬化根本」一辭語究是何義也)此顯然不是《中庸》「天地之道可一言而盡也,其為物不貳,則其生物不測」之義,亦不是至誠不息,以及誠則形著、明、動、變、化等義。此顯非先秦儒家隨「維天之命,於穆不已」而來之對於天道性體之體悟。

二、性或太極之為存在之理如此,則心神俱傍落而屬之氣。依此,自宇宙論而言,則理與氣為橫列的相對之二,(雖亦云理先氣後);自道德實踐而言,則心與性為橫列的相對之二。因此,遂由太極性體之生物不測或道德創造之「本體、宇宙論的」立體直貫之創生型或擴充型,轉而為認識論的橫列之靜涵型或靜攝型。凡《孟子》、《中庸》、《易傳》中「本體、宇宙論的」立體直貫縱

\newpage\thispagestyle{empty}\addtocounter{page}{-1}\vspace*{-12mm}\begin{center}\noindent
\includegraphics[clip, trim=166pt 155pt 143pt 237pt, height=162mm]{ocr-input/image-2431.png}\end{center}

\newpage

\noindent 貫之辭語,皆為其所不能正視,甚至所不喜,亦皆為其所不能解,或甚至是誤解、異解。彼對濂溪〈太極圖說〉雖極推崇表揚,而對於(通書〉誠體寂感之神則不能正視,不能以之解太極,因而遂將太極解為只是理,而神則屬之氣。彼對橫渠雖亦極推崇,而實不解其對於天道性體之體悟,對於〈大心篇〉之論心尤表不滿。彼對於明道雖極客氣而含蓄,而卻露其不滿之情於繼承明道而說者。出之明道之口,則謂其太高,或置之不理;出之繼承明道而說者之口,則力肆攻擊。其不滿可知。對於胡五峰之《知言》,則列舉八端之「疑義」以致疑。對於陸象山則力攻其為禪,而亦不必言矣。凡此不解、誤解、異解與不滿,其總關鍵總症結全在「本體、宇宙論的」立體直貫型與其認識論的橫列之靜涵型之不同,而其所不滿、不解而誤解、異解者,皆是直貫型也。其所不滿之人當然不能說全無疵病,然其所以不滿之總關鍵則在此兩型之不同。彼對之無間然而無異辭微辭者,惟一伊川而已。然則除伊川而外,如許之人盡皆非是乎?此中自不能無問題可知矣。然而歷來莫能明其究竟也!

橫渠「心統性情」之語與伊川「性即理也」之語為朱子靜涵靜攝系統所以成立之兩指導原則。然「心統性情」一語不見於《正蒙》,而只在(性理拾遺〉中有此孤立之一句:「心統性情者也。」(此一句為一條)如此語因朱子之重視而有代表性或原則性,則亦只能根據《正蒙》而解之,其意不必是朱子系統中之所解也。朱子只是借用此語以說己意耳。

以上就橫渠之言天道性命與心能盡性,綜括由朱子之靜攝型所起之波瀾,以明此問題關鍵之所在,兼攝後來發展之脈絡。如此,則六百年(元朝不計)發展之傳統,雖頭緒紛紜,極難董理,而握

\newpage\thispagestyle{empty}\addtocounter{page}{-1}\vspace*{-12mm}\begin{center}\noindent
\includegraphics[clip, trim=169pt 132pt 128pt 243pt, height=162mm]{ocr-input/image-2435.png}\end{center}

\newpage\markright{第二部 \quad 分論一 \quad 第二章 \quad 張橫渠對於「天道性命相貫通」之展示}

\noindent 此關鍵,則瞭如指掌、歷然在目矣。凡此只在明宋、明儒所言之天道、天命、太極、太虛,其結穴只在性體。性體具五義是客觀地說;從天道、天命、太極、太虛而結穴於性體,所謂性與天道,性天之旨,亦皆是客觀地說。至心能盡性,心具五義,則是主觀地、實踐地說。問題只是心性合一否。說性即理與說心即理是引生之問題,尚不是根源之問題。在朱子說性即理,而不說心即理,根本乃是太極性體之為只是理,心性不合一,故函心理之為二。此是朱子認識論的橫列之靜涵靜攝型之所必至者。凡以言性具五義、心具五義,只在明宇宙之生化即是道德之創造。「本體、宇宙論的」立體直貫之創生型或擴充型乃是先秦儒家言天道性命、言心性之本然。宋儒興起,濂溪、橫渠亦是剋就此義而言天道性命,言心性。明道之盛言一本乃是此義圓滿表示之模型。胡五峰、陸象山、王陽明,乃至劉蕺山皆是繼承此直貫型而立言。此為宋、明儒之大宗。惟自伊川以至朱子始歧出而成為認識論的橫列之靜涵靜攝型。此固有其偉大,獨成一型,(其在學術文化上之作用與意義亦甚大),然顯非先秦儒家所發展成之內聖成德之學(所謂道德哲學,道德的形上學)之本義與原型。此就儒家言,此固不得為正宗也。朱子之傳統固不等於孔、孟、《中庸》、《易傳》之傳統也。此中之差別亦如西方柏拉圖傳統之「本質倫理」與康德傳統之「方向倫理」之不同,亦如佛教中唯識宗與真常心宗(講如來藏者)之不同。最高智慧之大脈,出發點與內容雖不同,而其發展之形態則固常相同也。

\newpage\thispagestyle{empty}\addtocounter{page}{-1}\vspace*{-12mm}\begin{center}\noindent
\includegraphics[clip, trim=162pt 300pt 151pt 222pt, height=162mm]{ocr-input/image-2439.png}\end{center}

\newpage\thispagestyle{empty}

\newpage\markright{}

\appendix\chapter{佛家體用義之衡定}

橫渠謂:「若謂萬象為太虛中所見之物,則物與虛不相資,形自形,性自性,形性天人不相待,而有陷於浮屠以山河大地為見病之說。」

橫渠所言之虛或太虛(儒家義)是氣之超越體,虛所妙運之氣是其用,因虛之妙運始能有氣化之用,此是創生的「意志因果」之體用,創生的性體、心體、神體、誠體因果之體用,自不能謂「萬象為太虛中所見之物」,而物與虛,形與性,自是相資而相待,且不只相資而相待,且是立體之直貫。

佛家之體用義且比老子之體用義為特別,蓋其所言之空有殊義也。橫渠欲以其用之「太虛」一詞衡量之,自無甚意義。蓋就自義言,自不能「謂萬象為太虛中所見之物」,是以「若謂」之設擬乃無意義者。而就佛家言,則佛家所言之「空」與橫渠所言之「虛」完全不同,自亦不能以「若謂萬象為太虛中所見之物」之設擬難佛家。蓋不能「謂萬象為太虛中所見之物」,而就佛家言「空」之某方面某意義言,卻可謂萬象為「空」中所見之物。橫渠之所以如此設擬,蓋重在佛家體用之不相資不相待,明其體用義根本非聖人三極大中之道而已。實則佛家之「空」,固有時可謂萬象為「空」

\newpage\thispagestyle{empty}\addtocounter{page}{-1}\vspace*{-12mm}\begin{center}\noindent
\includegraphics[clip, trim=164pt 142pt 135pt 249pt, height=162mm]{ocr-input/image-2447.png}\end{center}

\newpage

\noindent 中所見之物,有時亦不能如此說,且甚至有時(其原初根本義)亦根本不能以體用論。即發展至某境,可以說體用,其體用究是否不相資不相待,即使可相資可相待,其相資相待究是何種意義之相資與相待,此則須有待於詳察者。

\section*{一、佛家言「空」之意義:空與緣生非體用義}\addcontentsline{toc}{section}{一、佛家言「空」之意義:空與緣生非體用義}

1.佛家之空,其原初之根本義亦是共同義,只是就「諸行無常,諸法無我」說。依佛家無明苦業之意識觀,諸行諸法無常無我即是空。空者空卻諸行諸法之自體或自性也。無常無我進而以「緣生」解。緣生固釋迦佛之所說。緣備則生,緣離則滅。生則存在,滅即不存在。故緣生即函無常,常則無所謂存在與不存在。無常即函無我。人無我,法無我,故諸法無我。無我等同無自體無自性,而此即是「空」義也。故緣生即空。言諸行諸法無自體無自性,而唯是以空為性也。故亦云空性或空理,言空即是萬法之通性、萬法之共理。法無自性,以空為性。亦可類比說,法無自體,以空為體。故亦可類比說空體,言空即是其體也。如此所說之空是抒意字,就無常無我緣生而抒其意,非指實字,言並非正面有一物曰空也。故空初只是遮詮,並非表詮。若謂空即是諸法之實相,也只是說諸法實理如此(「理」字是虛說之理)、實意如此,所謂本來面目只是如此,故亦謂空為諸法之「如」相,如其實相之所是(是空)而即如此說,不增不減,故為「如」相,亦名「真如」,言此如相即是真也。故此空字如字,初無玄妙之意。並非於緣生外指一實體曰空,或有一實體於此,而以空、無、妙、如形容之也。即使

\newpage\thispagestyle{empty}\addtocounter{page}{-1}\vspace*{-12mm}\begin{center}\noindent
\includegraphics[clip, trim=158pt 123pt 133pt 251pt, height=162mm]{ocr-input/image-2451.png}\end{center}

\newpage\markright{附錄 \quad 佛家體用義之衡定}

\noindent 空、無、妙、如都是抒意之形容字,亦是形容緣生,而非形容緣生外之實體。而形容緣生亦只可說空、無、如,而無所謂妙。故《中論·觀四諦品》云:「因緣所生法,我說即是空,亦為是假名,亦是中道義。」此即所謂「緣起性空」或「性空唯名」也。

緣生無性,無性緣生。無性即是無自體無自性,而此「即是空」。若反而正面說,緣生之法以空為性,以空為體,仍須通過遮詞來了解:以無自性之空為其性,以無自體之空為其體。此性字體字皆是虛的抒意詞,故其為性並非儒家之作為實體之「性理」之性,其為體亦非儒家作為性理之誠體、心體、神體、性體之體,總之,非道德創生的實體之體。吾人不能說空是緣生之體,緣生是空之用。體用之陳述在此用不上。雖然說以空為體,以空為性,然此抒意之空性空體實並不能存在地生起緣生之用也。此即表示空與緣生之關係並非體用之關係。是以以前呂秋逸曾謂體用是儒家義,佛家之真如空性並非體用之體。其言是也。

普通就「緣生無性,無性緣生」,說為:因為緣生,所以才無性;因為無性,所以才緣生。這好像有「因此所以」的體用因果關係,實則這只是言詮上的抒意之「因此所以」,並非存在上體用因果之客觀的「因此所以」,只是邏輯的「因此所以」,並非存有論的「因此所以」。因為緣生,所以才無性,在此,我們不能說緣生是存在上的體、因,無性是存在上的用、果。因為這根本不類。反之,因為無性,所以才緣生,亦不能視無性之空為存在上的體、因,視緣生為存在上的用、果。無性之空為體,緣生為用,這好像是類了,其實仍不類。「因為緣生,所以才無性」,此因此所以不可以說體用,則「因為無性,所以才緣生」,當然亦不可說體

\newpage\thispagestyle{empty}\addtocounter{page}{-1}\vspace*{-12mm}\begin{center}\noindent
\includegraphics[clip, trim=165pt 161pt 137pt 232pt, height=162mm]{ocr-input/image-2455.png}\end{center}

\newpage

\noindent 用。因為這兩個「因此所以」是同一層次上的語法。故此只是言詮上抒意之邏輯的因此所以,非存在上體用因果之存有論的(客觀的)因此所以。

普通依《中論·觀四諦品》「以有空義故,一切法得成,若無空義者,一切則不成」一頌,更可積極一點,說是佛家亦可以成就現象,而說:正因為空,所以才說緣生,才「有」緣生之現象,才能「成就」緣生之萬法,才能「建立」萬法。實則此中所謂有、成就、建立,仍只是言詮上抒意之有、成就與建立,與上之「因此所以」同,並非存在上立一實體以有、成就或建立此緣生之大用也。所謂有、成就或建立,仍只是言詮上有、成就或建立諸行諸法之無常無我如幻如化耳。此豈有存在上體用因果之成就或建立之實義耶?

以上「緣起性空」之一般陳述乃是佛家言空之基本義,亦是共許之義。

2.但佛家言空並不只是這「緣起性空」之一般陳述即算完事,其言「緣起性空」乃所以為觀空證空而得解脫。得解脫即是證涅槃(寂滅)。能如實觀空(修中觀)而不執,則表面上雖生滅變化,萬象紛紜,好像熱鬧得很,而底子上卻是至寂至靜,一無所有,此即「當體即如」之寂滅。一如一切如,一寂一切寂,一滅一切滅。寂是正面說,滅是反面說,滅執著,滅煩惱。寂滅以真如空性而定。但這亦不只是空說。一個生命要得解脫,證涅槃,談何容易。煩惱是我煩惱,我實感有此煩惱。解脫是我解脫,我實感有從煩惱中解脫出來之要求。我之所以實感有此煩惱,是因為我的自然生命之衝動與執著即是根本有一無明在。所以修觀證如決不只是泛說空

\newpage\thispagestyle{empty}\addtocounter{page}{-1}\vspace*{-12mm}\begin{center}\noindent
\includegraphics[clip, trim=157pt 140pt 131pt 234pt, height=162mm]{ocr-input/image-2459.png}\end{center}

\newpage\markright{附錄 \quad 佛家體用義之衡定}

\noindent 說,是要真正剋就自家生命之煩惱與情執而觀而證。此所以有唯識宗之由空宗「緣起性空」之一般陳述進而將諸行諸法統攝於識而言之之故。

依唯識宗,自然生命之煩惱情執以及所牽惹沾染之一切現象皆可解剖為一識之流,緣起諸法皆可統於識之流上說,而性空亦從識之流上證。由此而有三性之說。此即一、遍計所執性,二、依他起性,三、圓成實性。

《成唯識論》卷八云:「三種自性皆不遠離心心所法。謂心心所及所變現,衆緣生故,如幻事等,非有似有,誑惑愚夫,一切皆名依他起性。愚夫於此橫執我法、有無、一貫、俱不俱等,如空華等,性相都無,一切皆名徧計所執。依他起上,彼所妄執我法俱空,此空所顯識等真性,名圓成實。是故此三不離心等。」

此三性之說實即「緣起性空」一語落於識上之加詳說(加一徧計執)。「因緣所生法,我說即是空」,即函著對於「因緣所生法」之無執。若於此「所生法」上有計執,計執有我、有法、有有、有無,有一、有異,有有無俱或不俱,有一異俱或不俱等等,便是執法有自體、有自性,不能「即是空」,當體即空。今說「即是空」,即函遮計執。于依他起上,遮去計執,所顯之識等真性,即名圓成實。「識等真性」即識之流變之真如空性。是以就八識流轉而言,心即識心也。此即示無常無我之諸行諸法皆統於識心。識之流之根在阿賴耶(第八識),故此系統亦曰阿賴耶緣起。此是煩惱情執之根,於唯識上修空觀而證圓成實,這一極深遠而長期之艱苦工夫,便是轉識成智。是故阿賴耶緣起之識心即染汙心也。此一系統之其他諸義且不論,只就修空觀而證圓成實言,其所證顯之圓

\newpage\thispagestyle{empty}\addtocounter{page}{-1}\vspace*{-12mm}\begin{center}\noindent
\includegraphics[clip, trim=159pt 138pt 131pt 246pt, height=162mm]{ocr-input/image-2463.png}\end{center}

\newpage

\noindent 成實(真如空性)與依他起(緣生)之關係仍不是存在上體用因果之關係:真如空性不是使依他起者所以能起之體(能創生的體),而依他起亦不是真如空性所生起之用。就唯識系統言,體用因果只可就阿賴耶識中「種子現行」上說,而種子現行只是識之流變上的事,亦即只是依他起本身上的事。種子現行只是識之流變之潛伏與現行,此實不可以體用說。究極之體用只當就真如空性與依他起之關係說,而此關係卻正好不可以體用說。在此亦不能說萬象為真如空性中所見之物。唯識宗雖將萬法統於識心,然畢竟仍不失緣起性空之義理規範。

3.但是大乘契經中亦有講如來藏之系統,此是通過佛性之觀念,而想說明成佛之超越的根據,因此乃講一如來藏自性清淨心,即講一超越之真心,爲一切染淨諸法之所憑依。玄奘所傳之唯識宗只講阿賴耶緣起,而此系統復推進一步講如來藏緣起,有時復亦方便名曰真如緣起:生滅與不生滅皆統於一超越之真心。此是就佛果而溯佛因,肯定一超越之真心為佛性,即為成佛之超越根據。只要體現此佛性,便立地成佛。

此佛性不只是真如空理,而且是超越之真心,將緣起性空之空理空性融於真心上說,此亦可說是心理之為一。

奘傳唯識只講阿賴耶緣起,而阿賴耶初只是染汙識。(雖說阿賴耶是無覆無記,而無記並不是清淨,不清淨即染汙)轉識成智後,智托識現,此時之識即淨識—八淨識。此可以說智與淨識一:智是虛說,識(淨識、清淨心用)是實說。但此是經過漸次階位之修行而顯,其原初只是無記染汙也。是以原初並無一超越真心為佛性。唯識宗所肯定之佛性只是理佛性,即,只肯定自性涅槃,

\newpage\thispagestyle{empty}\addtocounter{page}{-1}\vspace*{-12mm}\begin{center}\noindent
\includegraphics[clip, trim=160pt 121pt 129pt 252pt, height=162mm]{ocr-input/image-2467.png}\end{center}

\newpage\markright{附錄 \quad 佛家體用義之衡定}

\noindent 不肯定自性菩提(自性覺、本覺)。菩提是修得事,後起事。此是事佛性。事佛性不是本有的。而且成佛有種性,一闡提無佛性,是即成命定論,違反一切衆生皆有佛性皆可成佛之宗旨。而今如來藏之系統不但肯定自性涅槃,而且肯定自性清淨心,不但以真如空性之空理(寂滅)為佛性,而且以超越真心,理與心一,為佛性。是則唯識宗所分別之理佛性事佛性,在此系統內則統於一而為一理事為一之佛性,一起皆本有。此本有之佛性不但是心理不二(智如不二),而且是「色心不二」。在此系統下,似可以說體用矣。佛性真心為體,由此而生起一切法為用。蓋此時真如空性不只是就緣起無性而說之空性空理,而且提升一步與真心為一,而心固有力用覺用也。如只是空理,所謂「但理」,自不能生起,但與真心為一,則似可以言生起。在此系統中,不但似乎可以說體用,而且在某契機上似亦可說萬象為虛空(真如清淨之真心)中所見之物。其真實意義究如何,見下。

4.此一系統特為中國佛教之所喜。從佛學上說,印度原只有空宗與有宗(唯識宗)兩傳統。故此兩宗在印度皆有論。而此一講如來藏自性清淨心之系統則多根據契經而說。此一系統在印度並無顯赫之宗論。但在中國,卻有一根據此系之契經而成之《大乘起信論》。此一宗論在中國佛教中有顯赫之地位。雖在考據上,今已公認其為中國人所偽造,但印度人不造,中國人可以造,豈只准印度和尚造論耶?只要義理有據即可。實亦無所謂偽。只因佛教從印度來,故偽託馬鳴以壯聲勢耳。

又在印度,佛學和尚研究唯識講阿賴耶者初亦並非與如來藏全無交涉。彌勒之《莊嚴論》及《辨中邊論》即講如來藏,堅慧之

\newpage\thispagestyle{empty}\addtocounter{page}{-1}\vspace*{-12mm}\begin{center}\noindent
\includegraphics[clip, trim=192pt 176pt 139pt 239pt, height=162mm]{ocr-input/image-2471.png}\end{center}

\newpage

\noindent 《寶性論》及《法界無差別論》亦大講如來藏。來中國之真諦所傳之唯識即不只講染汙阿賴耶,而且推進一步講阿摩羅識(第九識),講自性清淨心,此已幾近於如來藏矣。真諦學在當時名曰「攝論宗」,蓋以無着之《攝大乘論》為主。《攝大乘論》首引《阿毘達摩大乘經》(此《經》未譯)頌「無始時來界,一切法等依,由此有諸趣,及涅槃證得。」《攝論》本身雖以此頌之「界」證明阿賴耶識,故世親即釋此「界」為雜染有漏諸法之因,此自較合於《攝論》之本義,但此無始時來之「界」為諸趣(六道衆生)及證得涅槃之所依,則真諦之解為「解性」(即如來藏)在義理上不必定非。(雖然《攝論》本身並未提到如來藏)。真諦實有貫通阿賴耶識與如來藏心之意圖,而就阿賴耶說,則即說為「解性賴耶」,此即兩者之統一。即世親早期《十地經論》亦以第八阿黎耶識為第一義心、自性清淨心。前五識為識,第六識為意,第七識為心(染汙心),而第八阿賴耶則是心、意、識以上之真心。阿賴耶本亦有「聖」義,非如後來全意謂為劣義。南北朝時地論宗之南道派(慧光系)即根據世親《十地經論》之原義而以阿黎耶為真心為佛性。世親晚年以《解深密經》為主,始成染汙阿賴耶之系統。此是後來之淘濾。但在以前,或由第八上進第九,或由第七上進第八,總有染淨之異層。即染在第八,則第九為淨。如染在第七(末那),則第八為淨。以今語言之,染則為經驗心,淨則為超越心。總有此異層之肯認。如徒劃一為染汙之阿賴耶,而不言超越之真心,則成佛即無超越之根據。世親晚年之唯識學雖整齊詳密而老練,然原始之靈光,理想主義之情調,已隨其老練而喪失。老練務實固佳,然老練務實之中常不自覺即隱函沈墮而提不起之機。即不

\newpage\thispagestyle{empty}\addtocounter{page}{-1}\vspace*{-12mm}\begin{center}\noindent
\includegraphics[clip, trim=149pt 143pt 152pt 241pt, height=162mm]{ocr-input/image-2475.png}\end{center}

\newpage\markright{附錄 \quad 佛家體用義之衡定}

\noindent 說沈墮,而亦只落於經驗(後天)上磨,此即是提不起。此在佛家,即是晚年世親通過護法以至玄奘所成之唯識宗;在儒家,即是朱子之形態。然此中確有不澈不盡處,平實未易言也。是故唯識學其初未嘗不與如來藏自性清淨心相貫通。奘傳之唯識乃是後來淘濾而成者,非必自始即如此也。故唯識學之反其初,慎審思量而成為如來藏之系統,不但於契經有據,即於論亦非全無據也。惟在印度其初並不顯豁而完整,而完整者卻在世親護法之一路。而中國之《大乘起信論》卻是繼承真諦之思路以如來藏為中心而成一條理整然義理明透之另一完整系統。故此宗論在中國佛教中起如此大之影響與作用,此並非偶然也。後來奘傳之唯識雖喧赫一時,終不能奪其席,亦並非無故。此非考據家定其為偽即能貶損其價值。故圭峰宗密判大乘佛教為空宗、有宗與性宗,而今日佛教界亦有性空唯名、虛妄唯識、真常唯心之判也。性宗(真常唯心)可說是中國佛教之所創,而亦是大乘佛教發展之自然之趨勢。中國佛教即居於此顛峰而立言,故亦可說超過印度原有之佛學傳統。內學院歐陽竟無、呂秋逸等宗奘傳之唯識,力復印度傳統之舊,雖不無價值,而力貶損中國之性宗,斥之為俗學,則亦崇洋自貶識見不開之過也。

原中國佛教之所以特喜此性宗,判之為最高之圓教,固有中國民族智慧心靈之一般傾向背景,而實亦由儒道兩家之學術培養使之然也。人皆謂宋明儒受佛老之影響,是陽儒陰釋、儒釋混雜。實則宋明儒對於佛老了解實粗略,受其影響蓋甚小。彼等自有儒家義理智慧之規範。而魏晉玄學之弘揚道家,其影響於佛教之吸收卻極大。兩晉南北朝之佛教大德非不讀中國書者。如其說宋明儒受佛老之影響,因而儒釋混雜,不如說佛教大德受儒道義理智慧風範之影

\newpage\thispagestyle{empty}\addtocounter{page}{-1}\vspace*{-12mm}\begin{center}\noindent
\includegraphics[clip, trim=184pt 173pt 143pt 237pt, height=162mm]{ocr-input/image-2479.png}\end{center}

\newpage

\noindent 響,故特喜言如來藏自性清淨心者而創性宗(真常心宗)以超過印度原有之空宗與有宗。最後,實亦無所謂誰受誰之影響,只是中華民族智慧心靈之一般傾向,隨其所宗信而到處表現耳。象山、陽明固是孟子靈魂之再現,即竺道生、慧能亦是孟子靈魂之再現於佛家。故儒自是儒,道自是道,佛自是佛,唯有其共通之形態,而宗義之殊異不可泯。故動輒謂宋明儒受佛老影響者甚無謂也(謂受其刺激而覺醒則可)。

\section*{二、《起信論》之大義}\addcontentsline{toc}{section}{二、《起信論》之大義}

佛教發展至如來藏之真常心(自性清淨心),其真如空性與緣生之關係幾似乎可以體用論矣。此形態之相似也。然由於其宗義之殊異(仍是佛),其體用義仍不可以無辨也。以下試根據《起信論》而言之。

1.《起信論》顯示大乘正義,首先以一心開二門:

\begin{quotation}\kaishu 顯示正義者,依一心有二種門。云何為二?一者,心真如
門,二者,心生滅門。是二種門,皆各總攝一切法。此義云
何?以是二門不相離故。\end{quotation}

\noindent 此所謂「一心」,即唯一超越真心也。不是阿賴耶之為經驗的識心或心理學的心也。

\newpage\thispagestyle{empty}\addtocounter{page}{-1}\vspace*{-12mm}\begin{center}\noindent
\includegraphics[clip, trim=158pt 199pt 142pt 237pt, height=162mm]{ocr-input/image-2483.png}\end{center}

\newpage\markright{附錄 \quad 佛家體用義之衡定}

2.心真如者,即是一法界大總相法門體,所謂心性不生不滅。「心真如」者,此心即真如,非五蘊平視中心法之性空為如也。此是真心之與「如理」一:就如理言,即是此心之自性,故云「心性不生不滅」。(不是說心為一蘊,為緣起法,其空性不生不滅,乃是此心自體即是不生不滅)就心言,即是真如之心(真常心),心真如即是真如心。此超越之真如心是一切法之所依與所由,故云「即是一法界大總相法門體」。

2.1一切諸法唯依妄念而有差別。若離心念,則無一切境界之
相。是故一切法從本已來,離言說相,離名字相,離心緣
相,畢竟平等,無有變異,不可破壞,唯是一心,故名真
如。

\noindent 「唯是一心」即唯是一超越之真心,即真如心、真常心。一切法之差別相、境界相、名字相、言說相、心緣相等等,皆由妄念而起。凡念即妄,念不是真心自己,念是平地起風波,念是後天的、經驗的、心理學的。由妄念而生之差別相本質上即是虛妄不實的。故若離念,化念歸心,則一切法本質上即是空如平等,只是一真心常在,不生不滅。

依法,此真如心是如亦是心,不只是如「緣起性空」之一般陳述中之空理。自如言,如實空;自心言,如實不空。故云:

2.2復次,此真如者,依言說分別,有二種義。云何為二?一

\newpage\thispagestyle{empty}\addtocounter{page}{-1}\vspace*{-12mm}\begin{center}\noindent
\includegraphics[clip, trim=185pt 174pt 139pt 236pt, height=162mm]{ocr-input/image-2487.png}\end{center}

\newpage

\begin{quotation}\kaishu 者,如實空,以能究竟顯實故。二者,如實不空,以有自
體,具足無漏性功德故。\end{quotation}

\noindent 此真如即真如心。「如實空」是空妄念而顯一心之實。「如實不空」,則是就其「具足無漏性功德」言。此只有就真心言始可能,若真如只是緣起性空之空理,則不能有此義。

2.3所言空者,從本已來,一切染法不相應故,謂離一切法差
別之相,以無虛妄心念故。當知真如自性非有相,非無
相,非非有相,非非無相,非有無俱相;非一相,非異
相,非非一相,非非異相,非一異俱相。乃至總說,依一
切眾生,以有妄心,念念分別,皆不相應,故說為空。若
無妄心,實無可空故。

\noindent 此言由妄念所生之一切分別皆與此真如心不相應,即無一能用得上、沾得上。空此妄念,即是空。而真常心之自體則不可空。

2.4所言不空者,已顯法體空無妄故,即是真心。常恆不變,
淨法滿足,則名不空。亦無有相可取,以離念境界,唯證
相應故0

\noindent 真常心不但有其自體,(其真心自己即其自體),且具足無漏性無量功德。此云「淨法」即無漏性功德。此所謂「滿足」或「具足」是就因地言,即潛具意。若通過修顯,則全體朗現,即佛果。在因

\newpage\thispagestyle{empty}\addtocounter{page}{-1}\vspace*{-12mm}\begin{center}\noindent
\includegraphics[clip, trim=151pt 143pt 149pt 240pt, height=162mm]{ocr-input/image-2491.png}\end{center}

\newpage\markright{附錄 \quad 佛家體用義之衡定}

\noindent 不減,在佛不增,此亦函「性修不二,因果不二」。此唯識家所謂事佛性。但此事佛性,亦有真常心故,故能言其本具,非純屬後起也。因果不二,則理佛性事佛性是一。

3.真如心如上述,然則染汙生滅法如何說明?此則進入生滅門。

\begin{quotation}\kaishu 心生滅者,依如來藏,故有生滅生。所謂不生不滅與生滅和
合,非一非異,名為阿黎耶識。\end{quotation}

\noindent 真常心如何而能有染汙生滅心?前已提及,只是由于妄念。念則由于不覺,忽然心起而有其念。不覺即無明。此即落於生滅心矣。生滅心念亦憑依真心而起,但其直接根源卻是無明。此猶如春風一起,吹綴一池春水。真常心即是平靜之春水,無明風動,則起縐,此即生滅心念。縐絕不離水,即憑依水而起也。但其直接根源卻是風動。生滅心念不離真心,即是憑依真心而起,但其直接起因卻是無明。真心只是其憑依因,並非其生起因。心念憑依真心而起,即示不惟淨法統於一心,即一切染法亦統於一心。惟染法是間接地統。淨法是直接地統,所謂稱性功德也。稱性即相應心性而起之功德。間接地統只是憑依義,雖不離,而實不相應。《起信論》於此忽然不覺而有心念處,即起絕處,收攝阿賴耶識。此即阿賴耶識之統於真常心(如來藏自性清淨心)。普通所謂如來藏緣起或真如緣起,實非如來藏清淨心或真如心真能緣起生滅法,若如此,則淨的生出染的,其自身必不淨,故所謂如來藏緣起乃只是無明識念之由憑依如來藏而統於如來藏,故說如來藏緣起,其實真緣

\newpage\thispagestyle{empty}\addtocounter{page}{-1}\vspace*{-12mm}\begin{center}\noindent
\includegraphics[clip, trim=165pt 163pt 147pt 236pt, height=162mm]{ocr-input/image-2495.png}\end{center}

\newpage

\noindent 起者仍是阿賴耶識也。《勝鬘經》云:「自性清淨心而有染汙,難可了知。」其實通過無明風動,起綴絕而引進阿賴耶識,即可了知。彼經又云:「彼心為煩惱所染,亦難可了知。」實則雖為煩惱所染,而實不相應、沾不上。只是無明識念憑依彼心而起,此起現,則彼即附隨而隱伏,儼若為其所染耳。實則何曾染得上?阿賴耶識(縐綴)實無自體,其根只是無明,其所憑依只是真心。無明滅,真心顯,則阿賴耶識亦滅,即絕絕滅而歸於平靜之真常心。此凸起縐絕處,其本身是生滅,其所憑依者是不生不滅,兩者和合,非一非異,不即不離,此即為阿賴耶識。故阿賴耶識之凸起一方興風作浪,開出生死流轉,一方托帶著如來藏為其所憑依,此所謂挾天子以令諸侯也。此種貫通法顯是真諦之思路。

3.1此識有二種義,能攝一切法,生一切法,云何為二?一者
覺義,二者不覺義。

\noindent 阿賴耶識統於如來藏,故自心真如言,真如心可總攝一切法,而此識本身因有覺與不覺二義,故亦可「攝一切法,生一切法」,此即開始所說「是二種門,皆各總攝一切法」也。覺即是「心體離念」,不覺即是「忽然心起而有其念」。兩者皆可就阿賴耶識說。

因為此識是不生不滅與生滅兩者的和合,兩者和合在一起,不完全是一事,亦不完全是二事。即在此非一非異的狀態下呈現出阿賴耶識。

兩者和合,非一非異,是靜態現成的加合,不是很好的表示。其實義只是憑依如來藏(不生不滅的心真如體)忽然不覺而起心

\newpage\thispagestyle{empty}\addtocounter{page}{-1}\vspace*{-12mm}\begin{center}\noindent
\includegraphics[clip, trim=158pt 140pt 134pt 238pt, height=162mm]{ocr-input/image-2499.png}\end{center}

\newpage\markright{附錄 \quad 佛家體用義之衡定}

\noindent 念。「不覺」即是於心真如體不能如實覺知,亦即是根本的無明,無始無明住地。心一昏沈而心念生滅相續,即是阿賴耶識。雖是昏沈而生滅相續,卻必須是憑依不生不滅的心真如體而起現。由於無明的插入,心就起了縐而遠離了其自體而落於「念」中,猶如春風吹動,一池春水就起了波浪而動蕩不定。波浪畢竟不離水體。不憑依水體,焉有波浪?波浪畢究是屬於水的波浪,而不是屬於麥的麥浪。此即是所謂「非異」之憑依。但水體自身實並不含有波浪,亦如小麥自身並無所謂波浪,由於風動,才起波浪。風一止,波浪即滅。可見波浪是無體無根的假象,其起因只是由於風動,然其生起卻不能不憑依水體,此即水體與假象的波浪之「非一」。依是,阿賴耶識的呈現,它有一個憑依,猶如壞人憑藉好人以作壞事,又如貪官污吏假藉權位名器以舞文弄法,又如惡僕姦奴憑藉主人以興風作浪,結果壞事都記在好人身上,寫在權位名器上,列在主人身上,實則主人、好人、權位名器自身並無這些壞事,即並不生起這些壞事,但只是壞人憑藉它們而起現。沒有這些可憑藉處,壞人惡奴汙吏是不能興風作浪的。阿賴耶識生滅心之憑依如來藏,亦復如此。它除此憑依外,它復有其自己直接的根源,那就是無明,猶如壞人之作壞事,雖憑藉好人而作,然畢竟壞人之所以為壞而作壞事,乃由于其惡劣的根性,這是壞人之所以作壞之根源(生因),其所憑藉者乃是助長或助成其作壞事之勢,非其生因。

依其所憑依之心真如體而言覺,依其由於無明而起生滅心念,而說不覺。那就是說,其超越之體是覺,其自身之行用是不覺。這樣說覺與不覺是剋就不生不滅與生滅兩者而分屬說。分屬說的覺即是心真如體自身之本覺,此完全就心真如體之在其自己、本覺之在

\newpage\thispagestyle{empty}\addtocounter{page}{-1}\vspace*{-12mm}\begin{center}\noindent
\includegraphics[clip, trim=164pt 162pt 142pt 232pt, height=162mm]{ocr-input/image-2503.png}\end{center}

\newpage

\noindent 其自己而說。但我們也可以不必這樣分屬地說,可直就此起絕絕之識自身而說覺與不覺。此則較恰合於「此識有二種義」之語意。

憑藉心真如體而起生滅心念,同時也就是心真如體全部融入生滅心念中。生滅心念雖是不覺,而畢竟是心之生滅心念;心體全在生滅心念中,故念本是念,而曰心念。嚴格說,念不覺,而心覺。是以雖在不覺之念中,而心性不泯。即就此心性不泯而曰覺。此覺是拖帶在念中、隱伏在念中,而不彰顯,其彰顯而凸出者是念,所以心遂全部沈于念中,而吾人亦即以念目之,而不曰心,亦即以識目之,而不曰如來藏之真心。實則心性即在念中,不然何以曰心念?心性即在識中,不然何以曰識(覺識、心識)?心性即在念中,即在識中,這可以說是念中之覺性、識中之覺性。真諦所謂「解性黎耶」當即是此意。猶如風動,全水是波。此時雖是波浪凸出,而吾人亦只注意波浪,然而水體附在,並未泯絕。不過吾人此時不註意平靜之水體,而只註意波浪,以波浪為主,不以水體為主,此即降而為附隨、爲隱伏,然而水體全融於波浪中,水體雖隱伏附隨而不凸出,豈不附隨而永在耶?波浪雖主,而由於風動。風止浪息,則水體平靜,即由隱伏而朗現矣。此即心真如體之全現,本覺之全現,故一識中即可說覺與不覺。一旦無明破,心念止(離念),則心體朗現,即是本覺。無明逐步破,心念逐步止,則心體(覺性)逐步顯。及其全顯,即不說念中之覺性,而說心體呈現之本覺。雖是逐步修顯,實是本有,並非後起。故不增不減,因果不二,而一念迴機,儼同本得。此即是「本覺」一義之所以立。豈有原非本有而純屬後起之覺耶?

惟吾人平常只知經驗的、心理學的心念之起伏生滅為心,而不

\newpage\thispagestyle{empty}\addtocounter{page}{-1}\vspace*{-12mm}\begin{center}\noindent
\includegraphics[clip, trim=158pt 141pt 138pt 240pt, height=162mm]{ocr-input/image-2507.png}\end{center}

\newpage\markright{附錄 \quad 佛家體用義之衡定}

\noindent 承認有一超越之心體,以為此只是一抽象。邏輯地說,是抽象,然從依宗起教、依教起修而理想地說,則不是抽象,而是心體覺性之永在、遍在,此即是本有矣,而且是一真實,是一呈現。邏輯地說,是無色的,以經驗事實為基礎。此不能決定人生之方向。凡決定人生之方向而理想地發展其人格者,皆須有此類超越真心之肯定,而且是本有、是真實、是呈現(儒釋道皆然,耶教肯定上帝亦然)。邏輯與經驗的心理事實不是唯一的標準,尤其不是價值的標準。

在修證理想上,肯定此覺性乃至本覺,當然是就成佛而說明其超越根據之說明上的事。佛是這樣成正覺,即在其這樣成正覺中,覺性乃至本覺自然是這樣呈現,因而亦是這樣本有。或依此故而說:是這樣呈現,這樣本有,則說明不說明,肯認不肯認,並無緊要,不這樣說明肯認,只這樣修證下去亦未嘗不可,何必定要先肯認此超越之本覺,先承認此識念中之覺性?曰:這種說明、肯認,雖然並不增加什麼,然在點明成佛所以可能之超越根據,使人有明確之嚮往,有清楚之認識,亦正是所關甚大。一切義理教言俱是說明。都是說明,何不說得明確而恰當(相應)?而且對這超越真心的肯定,亦不是憑空肯定者,乃是即就生滅識念中之覺性而肯認之。若在生滅識念中不正視此覺性,而唯是註意此生滅之識念,以為此生滅之識念只是識念,並無所謂覺性,吾人只是順此生滅之識念而一步一步轉化之,轉化之而成覺,此所成之覺完全是後天的、經驗的、後得的,則吾人亦可以說:這樣順逐生滅識念而轉化下去亦可仍只是在識念中轉,而根本無由達成覺性之獲得(證得),這是無窮地追下去,亦是盲目地追下去,這樣很可使標的模

\newpage\thispagestyle{empty}\addtocounter{page}{-1}\vspace*{-12mm}\begin{center}\noindent
\includegraphics[clip, trim=160pt 162pt 132pt 227pt, height=162mm]{ocr-input/image-2511.png}\end{center}

\newpage

\noindent 糊,漸次亦可根本喪失其標的。是以唯識宗不承認此覺性乃至本覺,而唯是靠後天熏習與聖教量,乃是茫然而純在識念一層中作工夫。嘉言懿行,聖教量,若不消融於覺性中以證其為真為實,這一切很可都只是些雜念,憑念轉念,實只是以念引念,永無了期,就是一時不執著於依他起,證得了圓成實,亦只是了解了一個空理,與自家生命之清澈仍不相干。若是在空計執而證圓成實上,證得圓成實即是識轉而成智,圓成實之證得不只是只得一空理,而且真能滲融於自家生命中而由此清澈了吾人之識念而成為智,則即必須承認吾人之識念中確有覺性,而不只是識念之一層。吾人主體方面有此覺性,在證得圓成實上,圓成實空如之理方能滲融進來而與覺性水乳交融,以證成吾人之生命確是一智之生命,而不是一識之生命。但若不承認有此覺性,則證得之圓成實,如非只證得一空如之理而與自家生命不相干,便是即使相干,亦是融在念上,而不必真能證成智,是則智很可能是虛脫而永不能落實者。是以覺性乃至本覺之肯認乃是必然者,而且亦是必要者。這是修證工夫所以可能(所以有實效)之超越根據,唯識宗不承認此點,此在說明之理論上不能算是明確而恰當。

從佛經方面說,自後期講佛性講如來藏之真常經出現後,佛學方面之大論師大體在流轉還滅之所依一問題上開始鑽研,一時未能通透。無著之《攝大乘論》是唯識學之開始,而只講阿賴耶識與三性,不講如來藏,是即阿賴耶方面之識心與佛性,及如來藏方面之真心未能有一超越之貫通。此步作不到,「流轉」方面有積極之說明,「還滅」方面即無積極之說明,而為成佛之超越根據的佛性之積極作用亦不能顯。無著如此開端,世親繼之,下屆護法,以至玄

\newpage\thispagestyle{empty}\addtocounter{page}{-1}\vspace*{-12mm}\begin{center}\noindent
\includegraphics[clip, trim=168pt 140pt 132pt 245pt, height=162mm]{ocr-input/image-2515.png}\end{center}

\newpage\markright{附錄 \quad 佛家體用義之衡定}

\noindent 奘,遂成普通所周知之虛妄唯識之漸教唯識宗。此一傳統號稱印度佛學之正宗。然真諦來中國,思路比較活轉,自始即想溝通如來藏與阿賴耶。《起信論》之作者更為明透。此固與地論師(北道派)有關,而與攝論師之真諦尤接近。署為真諦譯,非偶然也。然此不知名之作者確極明透,彼直以如來藏為中心,以一心開二門,予如來藏與阿賴耶以超越的貫通,而佛性為成佛之超越根據之積極作用亦全部朗現,實比今日從文獻所知之真諦明透多矣。此種義理,自佛性觀念出現後,本亦極易見到者。然見到,則易;而如見不到,則一間未達,永在隔閡中,此機亦很難撥轉也。子不見儒家之朱子?朱子號稱宋明儒之正統派,然於本心亦總一間未達也。與唯識宗之形態極相似。然則中國和尚之造《起信論》不亦宜乎?

以上是就識自身而說覺與不覺。若就不生不滅之心真如與生滅之識念而分屬說,則顯得呆板,而亦與「此識有二種義」之語意不合。當然,就識自身而說覺,是就心真如為識所憑依而即隱伏附隨於識而為識之「覺性」言。覺性即是以覺為性,真諦所謂「以解為性」。由此當然可以推證如來藏,推證分別說之不生不滅心真如體之為本覺。但就「此識有二種義」中之「覺」義言,則是著重此隱伏附隨之「覺性」義,而不是直就「心真如」自身之為本覺言。

3.2由識念中有此覺性,由此覺性而肯認離念之本覺,即心體之自己。惟在識念串系中,人常只註意此識念而滾下去,此即是「不覺」。(識念本身亦是不覺,覺則無念矣。)此即示人雖有如來藏清淨心之光明面為其體,亦總有一陰影暗中沈墮此光明體,若能在無明識念中顧視此中之覺性,則覺性漸從隱伏附隨中呈現凸出,此即名「始覺」。猶如在波浪中常常顧諟波浪中隱伏附隨之水

\newpage\thispagestyle{empty}\addtocounter{page}{-1}\vspace*{-12mm}\begin{center}\noindent
\includegraphics[clip, trim=181pt 155pt 136pt 247pt, height=162mm]{ocr-input/image-2519.png}\end{center}

\newpage

\noindent 體,知波浪只是由於風動之假象,則即不見有波浪之實,實則只是一水體,水體即由隱伏而凸現,由附隨而為主。始覺在識念中呈現有次第、有局限,如《起信論》所言之覺滅、覺異、覺住覺生等,然其本質的意義同於本覺,只差有圓滿不圓滿,究竟不究竟而已。及其究極圓滿,完全離念,直至心源,洞悟「生」之無生,那便是本覺全體朗現,即心真如體全體朗現。故云:

\begin{quotation}\kaishu 所言覺者,謂心體離念。離念相者,等虛空界,無所不徧,
法界一相,即是如來平等法身。依此法身,說名本覺。何以
故?本覺義者,對始覺義顯。以始覺者即同本覺。始覺義
者,依本覺故,而有不覺,依不覺故,說有始覺。\end{quotation}

\noindent 此段文即言「此識有二種義」中之「覺」義。如此言覺,即是不覺、始覺、本覺關聯著說。實則先肯認識中有覺性。始覺對不覺說。因有「不覺」,所以才說有開始之覺。始覺即是覺體之呈用。覺體呈用即是覺體在隱伏附隨中呈現其自己。呈現其自己即有覺了洞澈識念生滅流轉之無自體無實性而化除遠離之之作用。此是總說。若分別說,如說覺滅覺異、覺住、覺生・詳見《起信論》,茲不詳述。始覺之覺用,及其直至心源,得見心性,心即常住,即名「究竟覺」。始覺而至究竟覺即同本覺。始覺之覺有所覺,如生、住、異、滅等,皆是所覺,在覺所覺中呈現其自己。及至究竟覺同於本覺,則即無能所之相,而唯是一本覺覺體之朗現,亦即心體之離念,亦云「如來平等法身」。是以言本覺者,不是說衆生不假修行,本來即已覺悟,乃是說眾生本有此光明之覺體。對始覺而

\newpage\thispagestyle{empty}\addtocounter{page}{-1}\vspace*{-12mm}\begin{center}\noindent
\includegraphics[clip, trim=155pt 133pt 146pt 249pt, height=162mm]{ocr-input/image-2523.png}\end{center}

\newpage\markright{附錄 \quad 佛家體用義之衡定}

\noindent 言,此覺體即為本覺。

3.3在始覺過程而證得本覺中,

\begin{quotation}\kaishu 本覺隨染分別,生二種相,與彼本覺不相捨離。云何為
二?一者智淨相,二者不思議業相。

智淨相者,謂依法力熏習,如實修行,滿足方便故,破和
合識相,滅相續心相,顯現法身,智淳正故。

不思議業相者,以依智淨,能作一切勝妙境界,所謂無量
功德之相,常無斷絕,隨眾生根,自然相應,種種而現,
得利益故。\end{quotation}

\noindent 「智淨相」是空,「不思議業相」是不空。覺體自身無相可說。此二種相是依隨或關聯著染法而分別成者。所謂依隨或關聯著染法而分別成即是依對治染法而顯示。「破和合識相,滅相續心相」即是對治。對治染法,離念無念,顯現法身,此時之智即為淳正之智、無分別智。即依此淳正之智而說覺體本覺之「智淨相」,即無分別之清淨相,亦即淳正相。但,雖清淨淳正,而隨衆生根之機感又能自然現無量功德業相以利益眾生。此即覺體本覺之「不思議業相」。此兩種相與覺體如如相應,「不相捨離」。不相應而可捨離者是妄念。空卻妄念,即是覺體。覺體呈現,自有此二種相。

智淨相是空、是體。不思議業相是不空、是本覺智體之業用。吾人在此須停一停,可仔細思量此所謂體用之意義。因為此處正是可以說體用處。

「依法力熏習」,此中法力有二:一、如來藏心之内力;二、

\newpage\thispagestyle{empty}\addtocounter{page}{-1}\vspace*{-12mm}\begin{center}\noindent
\includegraphics[clip, trim=180pt 166pt 136pt 237pt, height=162mm]{ocr-input/image-2527.png}\end{center}

\newpage

\noindent 佛菩薩為緣、說法教化所起之外力。此兩種法力皆能熏習眾生,令其「如實修行」。如來藏心之內力熏習即下文之「因熏習鏡」。佛菩薩為緣之外力熏習即下文之「緣熏習鏡」。

3.4故繼之復云:

\begin{quotation}\kaishu 復次,覺體相者有四種大義,與虛空等,猶如淨鏡。

云何為四?

一者,如實空鏡,遠離一切心境界相,無法可現,非覺照義
故。

二者,因熏習鏡,謂如實不空,一切世間鏡界悉於中現,不
出不入,不失不壞,常住一心,以一切法即真實性故;又一
切染法所不能染,智體不動,具足無漏,熏眾生故。

三者,法出離鏡,謂不空法出煩惱礙智礙,離和合相,淳淨
明故。

四者,緣熏習鏡,謂依法出離故,遍照眾生之心,令修善
根,隨念示現故。\end{quotation}

\noindent 此四種大義俱以虛空與淨鏡為喻。虛空明覺體無二無別,平等遍在;淨鏡明覺體朗照而無照功。

「如實空鏡」是言覺體之在其自己,遠離一切識念,無一法可現,亦「非覺照義故」,即覺體朗現淵渟,不在能所關係之中。無法可現,即無境界相。無境界相即無「所照」,無所照自亦無「能照」。

「因熏習鏡」是心真如體具足無漏性功德能有熏習眾生使之覺

\newpage\thispagestyle{empty}\addtocounter{page}{-1}\vspace*{-12mm}\begin{center}\noindent
\includegraphics[clip, trim=157pt 137pt 134pt 237pt, height=162mm]{ocr-input/image-2531.png}\end{center}

\newpage\markright{附錄 \quad 佛家體用義之衡定}

\noindent 悟向上(厭生死苦、樂求涅槃)的作用。因熏習鏡即從因熏習力說此鏡體。從覺體之不空說此如鏡之覺體。「不空」有二義:一、心真如體常住實有,不能空卻;二、智體不動,具足無漏功德。就其常住實有不能空卻言,它雖遠離一切識念,而一切識念(世間境界)卻亦是憑依它而顯現。猶如一切影像悉憑依明鏡而顯現。明鏡並非影像之「生因」,但影像須憑依它而現。明鏡自明鏡,亦未變為影像。明鏡與影像兩不相觸,亦不相礙。明鏡自明鏡,影像自影像,此即兩不相觸,《勝鬘經》所謂「煩惱不觸心,心不觸煩惱。」此即是不相應,一切識念不與覺體相應。雖不相觸、不相應,明鏡雖不生起影像,然不礙影像須憑依明鏡而顯現,此即不相礙。總此不相應而又不相礙之兩義,即所謂「一切世間境界悉於中現,不出〔不由覺體而生出〕,不入【亦不能進入覺體中〕,不失〔因憑依覺體而不喪失】,不壞〔因憑依覺體而不破壞】。」此處雖說「一切世間境界悉於中現」,然並非因此「世間境界」而說「不空」。這些「世間境界」並非覺體之業用。雖說「不出不入,不失不壞」,然亦正因不出不入而無自體,亦因無自體,雖不失不壞而當體即如,唯是一心。故云:「常住一心,以一切法即真實性故。」此處目的在說「常住一心」,就此說不空。並非因世間境界而說不空。然覺體並非抽象地虛懸,乃即一切法當體即如,一色一香無非真實,而為具體的呈現。此即是一即一切,一切即一,常住一心,法法無非覺體之如,故云:「以一切法即真實性故」。真實性即覺體之如性。(此就一切法統於真常心言,故說真實性即覺體之如性。如就緣起性空之一般陳述言,則一切法當體即如,如即是空性之如,不能說覺體之如。然無論只是空性之如,或是覺體之

\newpage\thispagestyle{empty}\addtocounter{page}{-1}\vspace*{-12mm}\begin{center}\noindent
\includegraphics[clip, trim=165pt 157pt 137pt 235pt, height=162mm]{ocr-input/image-2535.png}\end{center}

\newpage

\noindent 如,其為「當體即如」之形態則同)。即在此「當體即如」之圓融形態下,遂有「不失不壞」之說。(雖不出不入,而亦可不失不壞,不必失壞而即可滅度。此亦不毀世間而證菩提之意。)若是分解地說,則此等世間境界因是妄念之所現,故正因其不出不入無體無性而可失可壞可離可斷。

就「智體不動,具足無漏功德」言,即是所謂「真如熏習」。唯識宗反對此說,以為真如不能熏,亦不受熏,真如熏習乃不通者,亦如「真如緣起」之不通。但唯識宗之真如只是「但理」,自不能熏,亦不能起;而講如來藏者,則是心與理一,智與如一,其言真如即是心真如、真如心,故雖在纏自能具足無漏功德,熏習衆生,及其出纏亦自能起不思議業用,作緣熏習。如來藏心在吾人生命中豈無促覺之用?豈是只待外熏而顯發者?

「法出離鏡」,法即如來藏心,出離即出纏。意即謂:如實不空的如來藏心出煩惱礙(煩惱即礙,能障真如根本智)及智礙(無明能為智證的障礙,能障世間自然業智、後得智、方便智),離和合識相,顯現了覺體的淳淨明相,因此,遂得名曰「法出離鏡」。此即是上文本覺的「智淨相」。

「緣熏習鏡」即上文本覺的「不思議業相」。

4.以上是就「覺」義而展開。此下言阿賴耶識之「不覺」義,以及生滅心之因緣、生滅相,兼及染淨熏習等,名相繁多,略而不論。

在論「真如熏習」後,而綜結之曰:

\begin{quotation}\kaishu 復次,染法從無始已來,熏習不斷,乃至得佛後則有斷。淨\end{quotation}

\newpage\thispagestyle{empty}\addtocounter{page}{-1}\vspace*{-12mm}\begin{center}\noindent
\includegraphics[clip, trim=158pt 134pt 135pt 243pt, height=162mm]{ocr-input/image-2539.png}\end{center}

\newpage\markright{附錄 \quad 佛家體用義之衡定}

\begin{quotation}\kaishu 法熏習則無有斷,盡於未來。此義云何?以真如法常熏習
故,妄心則滅,法身顯現,起用熏習,故無有斷。\end{quotation}

\noindent 染法熏習,從根本處說,即是無明熏真如,愈熏,真如愈隱,其力愈弱,衆生遂長夜生死,交引日下。(染法互相交引,亦是熏習)。這樣熏習下去,遂從無始已來,永無斷絕。然無明畢竟無根,真如心總有覺用。及其一旦顯發,乃至成佛,則染法熏習即斷。是則所謂染法熏習不斷者,乃只是從不覺以後之交引日下說,並非本質上不可斷。若真本質上不可斷,則成佛即無可能。

淨法熏習,從根本處說,即是真如熏無明。真如心雖在重重障蔽中,亦總有其內熏無明之覺用,在不自覺中,漸漸熏習無明,令其沖淡,沖淡久之,無明力即漸趨微弱。一旦自覺作意,則真如力更顯發,無明力更微弱。再益之以緣熏習之外力,內外交發,則無明識念即可斷盡,而法身顯矣。真如心在纏之熏習力以及出纏後法身之起用熏習(起不思議業用作眾生之緣熏習)皆本質上永不斷絕。即一切衆生皆已成佛,無業相可說,則亦是法身常在,一切衆生皆是法身之衆生,不因無衆生而無佛。普通所謂「無衆生亦無佛」,此只是就法身遍一切處之圓說。佛法身以一切眾生得度為內容,故離衆生即無佛。衆生得度,則衆生即是佛而非衆生。若於此而言「無衆生即無佛」則悖。佛與衆生不是緣起法,相觀待而有者。佛永是佛,衆生成佛亦永是佛。法身平等,永恆常在。佛亦無所謂涅槃不涅槃。寂滅法身永恆常在。普通所謂有涅槃有不涅槃,乃是觀待著有限色身而說。佛法身是永恆之生命,如果這是色心不二,則亦是永恆而無限之色心不二之法身永恆之生命,此即無所謂

\newpage\thispagestyle{empty}\addtocounter{page}{-1}\vspace*{-12mm}\begin{center}\noindent
\includegraphics[clip, trim=156pt 151pt 135pt 230pt, height=162mm]{ocr-input/image-2543.png}\end{center}

\newpage

\noindent 涅槃不涅槃矣。

此所謂法身,即下文所謂「真如自體相」。

4.1復次,真如自體相者,一切凡夫、聲聞、緣覺菩薩、諸
佛,無有增減,非前際生,非後際滅,畢竟常恆,從本已
來,性自滿足一切功德。所謂自體有大智慧光明義故,遍
照法界義故,真實識知義故,自性清淨心義故,常樂我淨
義故,清涼不變自在義故,具足如是過於恆沙不離不斷不
異不思議佛法,乃至滿足無有所少義故,名為如來藏,亦
名如來法身。

問曰:上說真如,其體平等,離一切相,云何復說體有如
是種種功德?

答曰:雖有此諸功德義,而無差別之相,等同一味,唯一
真如。此義云何?以無分別,離分別相,是故無二。復以
何義得說差別?以依業識生滅相示。此云何示?以一切法
本來唯心,實無於念,而有妄心不覺起念,見諸境界,故
說無明;心性不起,即是大智慧光明義故。若心起見,則
有不見之相;心性離見,即是遍照法界義故。若心有動,
非真識知,無有自性,非常、非樂、非我、非淨,熱惱衰
變,則不自在,乃至具有過恆沙等妄染之義;對此義故,
心性無動,則有過恆沙等諸淨功德相義示現。若心有起,
更見前法可念者,則有所少。如是淨法無量功德,即是一
心,更無所念,是故滿足,名為法身如來之藏。

4.2復次,真如用者,所謂諸佛如來,本在因地,發大慈悲,

\newpage\thispagestyle{empty}\addtocounter{page}{-1}\vspace*{-12mm}\begin{center}\noindent
\includegraphics[clip, trim=164pt 131pt 131pt 243pt, height=162mm]{ocr-input/image-2547.png}\end{center}

\newpage\markright{附錄 \quad 佛家體用義之衡定}

\begin{quotation}\kaishu 修諸波羅蜜,攝化眾生,立大誓願,盡欲度脫等眾生界,
亦不限劫數,盡於未來。以取一切眾生如己身故,而亦不
取眾生相。此以何義?謂如實知一切眾生及與己身,真如
平等無差別故。

以有如是大方便智,除滅無明,見本法身,自然而有不思
議業種種之用,即與真如等,偏一切處。又亦無有用相可
得。何以故?謂諸佛如來,唯是法身智相之身。第一義
諦,無有世俗境界,離於施作,但隨眾生見聞得益,故說
為用。

此用有二種。云何為二?

一者,依分別事識,凡夫二乘心所見者,名為應身。以不
知轉識現故,見從外來,取色分齊,不能盡知故。

二者,依於業識,謂諸菩薩從初發意,乃至菩薩究竟地心
所見者,名為報身。身有無量色,色有無量相,相有無量
好。所住依報,亦有無量種種莊嚴,隨所示現,即無有
邊,不可窮盡,離分齊相。隨其所應,常能住持,不毀不
失。如是功德,皆因諸波羅蜜等無漏行熏,及不思議熏之
所成就,具足無量樂相,故說為報身。

又為凡夫所見者是其粗色。隨於六道,各見不同;種種異
類,非受樂相,故說為應身。

復次,初發意菩薩等所見者,以深信真如法故,少分而
見,知彼色相莊嚴等事,無來無去,離於分齊,唯依心
現,不離真如。然此菩薩猶自分別,以未入法身位故。若
得淨心,所見微妙,其用轉勝,乃至菩薩地盡,見之究\end{quotation}

\newpage\thispagestyle{empty}\addtocounter{page}{-1}\vspace*{-12mm}\begin{center}\noindent
\includegraphics[clip, trim=282pt 165pt 152pt 239pt, height=162mm]{ocr-input/image-2551.png}\end{center}

\newpage

\begin{quotation}\kaishu 竟。若離業識,則無見相。以諸佛法身,無有彼此色相迭
相見故。

問曰:若諸佛法身離於色相者,云何能現色相?

答曰:即此法身是色體故,能現於色。所謂從本已來,色
心不二。以色性即智故,無體無形,說名智身。以智性即
色故,說名法身偏一切處。所現之色無有分齊,隨心能示
十方世界無量菩薩,無量報身,無量莊嚴,各各差別,皆
無分齊,而不相妨。此非心識分別能知,以真如自在用義
故。\end{quotation}

5.最後,在對治人我見之邪執中有云:

\begin{quotation}\kaishu 人我見者,依諸凡夫,說有五種。云何為五?

一者,聞修多羅說:如來法身畢竟寂寞,猶如虛空,〔《大
集經〉】,以不知為破著故,即謂虛空是如來性。云何對
治?明虛空相是其妄法,無體不實,以對色故有,是可見
相,令心生滅。以一切法本來是心,實無外色。若無外色
者,則無虛空之相。所謂一切境界,唯心妄起故有。若離於
妄動,則一切境界滅:唯一真心,無所不偏。此謂如來廣大
性智究竟之義,非如虛空相故。〔案:此言虛空相如空的空
間然。空的空間待色而有,故是妄法。經說「猶如虛空」,
本是一喻,喻覺體廣大,無分齊相,此正是由破執而顯,而
不解者反執此「虛空」之喻為一「虛空相」,遂認「虛空
相」為如來性。此正是人我見之邪執。此書言人我見與普通\end{quotation}

\newpage\thispagestyle{empty}\addtocounter{page}{-1}\vspace*{-12mm}\begin{center}\noindent
\includegraphics[clip, trim=170pt 141pt 135pt 242pt, height=162mm]{ocr-input/image-2555.png}\end{center}

\newpage\markright{附錄 \quad 佛家體用義之衡定}

\begin{quotation}\kaishu 不同,非常特別。此人我見是對於虛喻之誤解。]

二者,聞修多羅說:世間諸法畢竟體空,乃至涅槃真如之法
亦畢竟空,從本已來自空,離一切相,〔《大般若經》】,
以不知為破著故,即謂真如涅槃之性唯是真空。云何對治?
明與如法身自體不空,具足無量,性功德故。〔案:此人我
見是以破相之空誤用於真如法身之自體。真如法身乃至涅槃
非是緣起之相,故不能以破相之空視之。若有人於真如涅槃
起執著之相,則可如此破之。破是破人之情執,空是空人之
情執,非空真如法身乃至涅槃之自體也。此自體真常,故不
空;具足無量稱性功德,故不空。若此亦空,則成空見。]
三者,聞修多羅說:如來之藏無有增減,體備一切功德之
性,〔《如來藏經》〕,以不解故,即謂如來之藏有色心法
自相差別。云何對治?以唯依真如義說故,因生滅染義示現
說差別故。〔案:如來藏之不空是真空妙有。今聞不空,即
謂「有色心法自相差別」,此乃落於分別事識之識念,正是
染汙法,非如來藏之不空也。此人我見是對於「不空」之誤
解。署名慧思造的《大乘止觀法門》復以如來藏心具染性染
事而說不空,正是這絕大的誤解。]

四者,聞修多羅說:一切世間生死染法皆依如來藏而有,一
切諸法不離真如,〔《勝鬘》、《楞伽》等經〕,以不解
故,謂如來藏自體具有一切世間生死等法。云何對治?以如
來藏從本已來,唯有過恆沙等諸淨功德,不離不斷不異真如
義故;以過恆沙等煩惱染法唯是妄有,性自本無,從無始世
來,未曾與如來藏相應故。若如來藏體有妄法,而使證會永\end{quotation}

\newpage\thispagestyle{empty}\addtocounter{page}{-1}\vspace*{-12mm}\begin{center}\noindent
\includegraphics[clip, trim=257pt 138pt 136pt 249pt, height=162mm]{ocr-input/image-2559.png}\end{center}

\newpage

\begin{quotation}\kaishu 息妄者,則無是處故。〔案:此可對治普通對於「如來藏緣
起」之誤解。如來藏自體實不緣起生死等法,乃是無明識念
憑依如來藏而緣起者。又天台宗,如依智者(摩訶止觀》
說,只說一念三千,並不說如來藏具生死染法。〈摩訶止
觀〉並不套於如來藏阿賴耶識之超越的貫通之系統中統一切
法。後來所謂性具或理具,仍是本智者一念三千(介爾有
念,即具三千世間)而說,而且是理具事造相對而說,此理
具或性具之理字或性字並不指如來藏真心說。署名慧思造的
《大乘止觀法門》並不可靠。如真是慧思所作,則那樣系統
整然,思理綿密(而實不澈)的論典,智者何不據以為規
範,而竟無一語稱及耶?智者之圓教乃是直接本龍樹《中
論〉收於止觀上而成者,與如來藏系統並無多大關係。華嚴
宗倒是本如來藏系統而成者。]

五者,聞修多羅說:依如來藏,故有生死,依如來藏,故得
涅槃,〔《勝鬘》、《楞伽》】,以不解故,謂眾生有始;
以見始故,復謂如來所得涅槃有其終盡,還作眾生。云何對
治?以如來藏無前際故,無明之相亦無有始。若說三界外更
有眾生始起者,即是外道經說。又如來藏無有後際,諸佛所
得涅槃與之相應,則無後際故。〔案:此即上文所說真如法
身永恆常在,佛亦所謂涅槃不涅槃。衆生無始,以無明無根
故。】\end{quotation}

\noindent 以上五種邪執誤解,名曰人我見。至於「法我見」則是就二乘鈍根說,因「見有五陰生滅之法,怖畏生死,妄取涅槃」故。

\newpage\thispagestyle{empty}\addtocounter{page}{-1}\vspace*{-12mm}\begin{center}\noindent
\includegraphics[clip, trim=160pt 123pt 128pt 251pt, height=162mm]{ocr-input/image-2563.png}\end{center}

\newpage\markright{附錄 \quad 佛家體用義之衡定}

\section*{三、體用義之檢查}\addcontentsline{toc}{section}{三、體用義之檢查}

1.以上由始覺即同本覺,而言覺體有智淨相及不思議業相,並言覺體相有四種廣大義,隨而言真如熏習不斷,真如自體相不斷,以及三身,迤邏說來,皆表示如來藏心真如體有一種體用義。

一、由智淨相顯現法身,法身即是一切功德法之所聚,法身不只是真如空性,而且具足無量無漏功德(稱性功德)。是則法身即是備一種體用,皆可曰覺體(心真如體)自身之內在的體用,體用整一而為法身。

二、「依智淨能作一切勝妙境界」之不思議業相,此可曰覺體對他(衆生)的體用、外在的體用,亦可曰關聯的體用。

三、覺體相四種大義中,「因熏習鏡」(真如熏習)是覺體自身對他(無明)之內在的熏習體用。

四、「緣熏習鏡」是覺體出纏對他(眾生)之外在的、關聯的熏習體用。

五、真如熏習(淨熏)不斷,真如體恆常起影響作用,令不覺者漸次向覺。覺至究竟,法身顯現。起用熏習,恆常不斷。

六、真如自體相無斷,即如來法身常住,永不斷絕。

七、真如用無斷,即應、報身不斷。應報身俱對他而顯,屬外在體用攝。法身是體,應、報身是用。

以上七點俱有體用義,實則只是真如熏習體用與三身體用兩種。而真如熏習中之「緣熏習」仍屬三身之體用。惟因熏習(真如在纏內熏)稍特別,有獨立之意義。然依佛家,究竟體用義仍在三

\newpage\thispagestyle{empty}\addtocounter{page}{-1}\vspace*{-12mm}\begin{center}\noindent
\includegraphics[clip, trim=170pt 162pt 142pt 236pt, height=162mm]{ocr-input/image-2567.png}\end{center}

\newpage

\noindent 身,體用之恰當的意義亦在三身。

以上許多表示體用義者,華嚴宗俱統之曰「性起」。

2.法藏賢首《華嚴經探玄記》對於〈寶王如來性起品〉第三十二,作總述云:

\begin{quotation}\kaishu 《佛性論·如來藏品》云:「從自性住來至得果!故名如
來。」不改名性,顯用稱起,即如來之性起。又真理名如名
性,顯用名起名來,即如來為性起。此等從人及法,用顯品
目。又別翻一本,名《如來秘密經》。又一本名《如來興顯
經》。又此下文具有十名,並可知。\end{quotation}

\noindent 以上為釋名。又云:

\begin{quotation}\kaishu 三、宗趣者,明性起法門,即以為宗。分別此義,略作十
門。一、分相門。二、依持門。三、融攝門。四、性德門。
五、定義門。六、染淨門。七、因果門。八、通局。九、分
齊。十、建立。

初、分相者,性有三種:謂理、行、果。起亦有三:初、謂
理待了因,顯現名起。二、行性由待聞熏資發。生果名起。
三、果性起者,謂此果性更無別體,即彼理行兼具修生,至
果位時,合為果性。應機化用,名之為起。是故三位各性各
起,故云性起。今此文中,正辨後一,兼辨前二也。

二、依持門者,一、行證理成,則以理為性,行成為起。此
約菩薩位,以凡位有性而無起故。二、證圓成果,則理行為\end{quotation}

\newpage\thispagestyle{empty}\addtocounter{page}{-1}\vspace*{-12mm}\begin{center}\noindent
\includegraphics[clip, trim=159pt 143pt 130pt 236pt, height=162mm]{ocr-input/image-2571.png}\end{center}

\newpage\markright{附錄 \quad 佛家體用義之衡定}

\begin{quotation}\kaishu 性,果成為起。此約佛自德。三、理行圓成之果為性,赴感
應機之用為起。是則理行激至果用,故起唯性起也。

三、融攝門者,既行依理起,則行虛性實,虛盡實現,起唯
性起。乃至果用唯是真性之用。如金作還等,鐶虛金實,唯
是金起。思之可知。

四、性德門者,以理性即行性,是故起唯理性起。此與前門
何別者,前約以理奪行說,今約理本具行說。問:理是無
為,行是有為。理顯為法身,行滿為報身。法報不同,為無
為異。云何理性即是行耶?答:以如來藏中具足恆沙功德
故。《起信論》中,不空真如有大智慧光明義,偏照法界義
等。《涅槃》云:「佛性者名第一義空,第一義空名為智
慧。」解云:此則無為性中,具有有為功德法故。《如來藏
經》模中像等,及《寶性論》真如為種性等,皆是此義。是
故藉修引至果位,名為果性。果性赴感,名為性起。

五、定義門者,問云:下文云:「非少因緣成等正覺」。此
乃是緣起,何故唯言性起耶?釋云:有四義:一、以果海自
體當不可說不可說性。機感具緣,約緣明起。起已違緣,而
順自性。是故廢緣,但名性起。二、性體不可說,若說即名
起。今就緣說起,起無餘起,還以性為起,故名性起,不名
緣起。三、起雖攬緣,緣必無住。無性之理,顯於緣處。是
故就顯,但名性起。如從無住本立一切法等。四、若此所
起,似彼緣相,則屬緣起。今明所起,唯據淨用,順於真
性,故屬性起。

六、染淨門者,問:一切諸法皆依性立,何故下文性起之\end{quotation}

\newpage\thispagestyle{empty}\addtocounter{page}{-1}\vspace*{-12mm}\begin{center}\noindent
\includegraphics[clip, trim=270pt 119pt 139pt 233pt, height=162mm]{ocr-input/image-2575.png}\end{center}

\newpage

\begin{quotation}\kaishu 法,唯約淨法,不取染耶?答:染淨等法,雖同依真,但違
順異故,染屬無明,淨歸性起。問:染非性起,應離於真。
答:以違真故,不得離真。以違真故,不屬真用。如人顛
倒,戴靴為帽。倒即是靴,故不離靴。首戴為帽,非靴所
用。當知此中,道理亦爾。以染不離真體,故說眾生即如等
也。以不順真用,故非此性起攝。若約留惑而有淨用,亦入
性起收。問:眾生及煩惱,皆是性起不?答:皆是。何以
故?是所救故,是所斷故,所知故。是故一切無非性起。

七、因果門者,問:菩薩善根亦順性而起,何故下文唯辨佛
果?答:以未圓,故不辨耳。若約為性起因義及眷屬義,皆
性起攝。如下文藥樹王生芽時,一切樹同生等。若從此義,
初發菩提心已去,皆性起攝。唯除凡、小,以二處不生芽
故。若據為統,令彼生善,亦性起攝。如日照生盲等。

八、通局門者,問:此性起唯據佛果,何故下文菩薩自知身
中有性起菩提,一切眾生心中亦爾?答:若三乘教,眾生心
中但有因性,無果用相。此圓教中,盧舍那果法,該眾生
界。是故眾生身中亦有果相。若不爾者,則但是性,而無起
義,非此品說。文意不爾,以明性起唯果法故。但以果中,
具三世間,是故眾生亦此所攝。問:既局佛果,何故下文通
一切法?答:若三乘教,真如之性通情非情;開覺佛性,惟
局有情。故《涅槃》云:「非佛性者,謂草木等。」若圓教
中,佛性及性起,皆通依正,如下文辨。是故成佛具三世
間,國土身等皆是佛身。堤故局唯佛果,通徧非情。

九、分齊門者,既此真性融遍一切,故彼所起亦具一切。分\end{quotation}

\newpage\thispagestyle{empty}\addtocounter{page}{-1}\vspace*{-12mm}\begin{center}\noindent
\includegraphics[clip, trim=158pt 142pt 131pt 234pt, height=162mm]{ocr-input/image-2579.png}\end{center}

\newpage\markright{附錄 \quad 佛家體用義之衡定}

\begin{quotation}\kaishu 圓無際,是故分齊皆悉圓滿,無不皆具無盡法身。是故偏一
切時,一切處,一切法等。如因陀羅網,無不具足。

十、建立門者,問:法門無涯,何故下文唯辨十種?答:顯
無盡故。何等為十?一、總辨多緣以成正覺。二、正覺身。
三、語業。四、智。五、境。六、行。七菩提。八轉法
輪。九、入涅槃。十、見聞恭敬,供養得益。此十略收佛果
業用,故不增減。此十義通前九位,皆具準之。\end{quotation}

\noindent 法藏賢首以十門分別性起義,最後復只就佛果言性起,一切皆攝於圓教性起之果法中,唯言性起,不言緣起。性起即體用義。如為性,來為起,如來即性起。就人說,名曰「如來之性起」。就法說,「即如來為性起」。凡此十門所說,不離《起信論》義理之規模。

茲仍依上列七點《起信論》中所說詳檢「性起」體用義之意義。

3.上列七點表示體用義者,嚴格說,只應報身處是正面的體用義。「不思議業相」之體用同於應報身。「因熏習鏡」即因地在纏之真如熏習無明,令不覺者漸覺,此是引起「還滅」的修行工夫上之體用。「緣熏習鏡」即不思議業相對眾生為緣助促其覺悟,此亦是還滅工夫之體用。還滅後法身顯現而有應報身之用,此方是正面的真正的體用義。法藏賢首所謂「行依理起,則行虛性實。虛盡實現,起唯性起,乃至果用唯是真性之用」者是也。

在《起信論》中,染淨法互相熏習,互有影響作用。染法的根本是無明,無明也有熏習力令真如不顯,此謂「無明熏習」。淨法

\newpage\thispagestyle{empty}\addtocounter{page}{-1}\vspace*{-12mm}\begin{center}\noindent
\includegraphics[clip, trim=162pt 140pt 137pt 247pt, height=162mm]{ocr-input/image-2583.png}\end{center}

\newpage

\noindent 的根本是真如,真如也有熏習力令無明滅,此謂「真如熏習」。

在「無明熏習」中有云:「云何熏習起染法不斷?所謂以依真如法故,有於無明;以有無明染法因故,即熏習真如;以熏習故,則有妄心。以有妄心,即熏習無明;不了真如法故,不覺起念,現妄境界。以有妄境界染法緣故,即熏習妄心,令其念著,造種種業,受於一切身心等苦。」此種由無明起,展轉熏習,即使眾生完全陷於生死流轉中。

在「真如熏習」中,則云:「云何熏習起淨法不斷?所謂以有真如法故,能熏習無明;以熏習因緣力故,則令妄心厭生死苦,樂求涅槃。以此妄心有厭、求因緣故,即熏習真如,自信己性,知心妄動,無前境界,修遠離法。以如實知無前境界故,種種方便,起隨順行,不取不念,乃至久遠熏習力故,無明則滅。以無明滅故,心無有起;以無起故,境界隨滅。以因緣俱滅故,心相皆盡,名得涅槃,成自然業。」此由真如起之展轉熏習即使衆生解脫還滅而顯法身。

「真如熏習義有二種。云何為二?一者自體相熏習,二者用熏習。」

「自體相熏習者,從無始世來,具無漏法;備有不思議業,作境界之性。依此二義,恆常熏習。以有力故,能令衆生厭生死苦,樂求涅槃;自信己身有真如法,發心修行。」此即「因熏習鏡」。

「 用熏習者,即是衆生外緣之力。如果外緣有無量義,略說二種。云何為二?一者差別緣,二者平等緣。」此即「緣熏習鏡」,亦即「不思議業用」。

是故真如熏習即是令眾生起還滅修行的體用。還滅者,還滅流

\newpage\thispagestyle{empty}\addtocounter{page}{-1}\vspace*{-12mm}\begin{center}\noindent
\includegraphics[clip, trim=158pt 122pt 132pt 251pt, height=162mm]{ocr-input/image-2587.png}\end{center}

\newpage\markright{附錄 \quad 佛家體用義之衡定}

\noindent 轉以顯法身也。法身為體,應、報身為用。真如用熏習,就佛言,即應報身之不思議業相為眾生之外緣也。真如自體相熏習是在纏的真如默默中有一種影響力令眾生「厭生死苦,樂求涅槃」;由不自覺中再進而自覺肯認此真如為性。真如用熏習是出纏的(顯現的)真如之或為佛、或為菩薩、或為善知識而對於衆生為緣,其極即是佛法身之不思議業用。故正面的真正的體用即在三身處,即法身與應報身的關係處:法身為體,應報身為用。

4.嚴格說,法身自身不能算是體用,只可說是性相合一。其所具足之無漏功德性不能算是真如體之用,只是它的相。因為法身不只是真如之空性之理,而且是清淨心。心理合一,自具足無量無邊無漏功德相,故法身即是一大功德聚,而實亦無相可聚,平等一味,無差別相。相者是「依業識生滅相示」,是通過「過恆沙等妄染之義」而示現、而反顯,其實是無相之相。業識(轉識、末那)生滅相是有相之相,真有差別,而此等無漏功德相則是由滅除業識之生滅相而示現反顯,故其本身實為無相之相。其為相之「多」義亦是由業識處之實多翻上來而成為虛說的多,實無所謂多,故云:「等同一昧,唯一真如」。因為無所謂多,故亦無所謂相;相亦是虛說,故為無相之相。

例如,真如自體有:

(1)大智慧光明義;

(2)偏照法界義;

(3)真實識知義;

(4)自性清淨心義;

(5)常樂我淨義;

\newpage\thispagestyle{empty}\addtocounter{page}{-1}\vspace*{-12mm}\begin{center}\noindent
\includegraphics[clip, trim=159pt 152pt 129pt 228pt, height=162mm]{ocr-input/image-2591.png}\end{center}

\newpage

\begin{quotation}\kaishu (6)清涼不變自在義;

(7)乃至「過於恆沙不離不斷不異不思議佛法」。\end{quotation}

\noindent 這一切其實只是一相而無相。相者:

(1)對起念言,「心性不起,即是大智慧光明義」。

(2)對起見言,「心性離見,即是徧照法界義」。

(3)就心有動言,則「非真識知」;「心性無動」,則是「真
實識知義」。

(4)心有動,則「無有自性」;心性無動,則是「自性清淨心
義」。

(5)心有動,則「非常非樂非我非淨」;心無動,則是「常樂
我淨義」。

(6)心有動,則「熱惱衰變,則不自在」;心無動,則是「清
涼不變自在義」。

(7)識念處「有過恆沙等妄染之義」,則翻上來即「有過恆沙
等諸淨功德相義示現」。

「過恆沙等妄染之義」是差別實多,而「過恆沙等諸淨功德相義示現」卻是無差別之虛說的多。無差別即是相而無相,無相之相。虛多即是多而無多,等同一味。此即是如來法身之自體相,故此體相不可說為體用。

真正體用乃在法身之示現為應身與報身。吾人試看此種體用究是何種意義之體用。

5.法身之不思議業相(或業用),就凡夫二乘所見者,說為應身,就菩薩所見者,說為報身。而這些報身應身處之不思議業相又皆是依衆生之識念而現,例如凡夫二乘所見之應身是依分別事識

\newpage\thispagestyle{empty}\addtocounter{page}{-1}\vspace*{-12mm}\begin{center}\noindent
\includegraphics[clip, trim=157pt 137pt 136pt 236pt, height=162mm]{ocr-input/image-2595.png}\end{center}

\newpage\markright{附錄 \quad 佛家體用義之衡定}

\noindent (第六識——意識)而見,菩薩所見之報身是依業識或轉識(第七識—末那)而見。

應身亦曰化身,或綜曰應化身,即是隨衆生之機感而應化者。從衆生方面說曰機感,從佛方面說曰應化。機感有緣,故佛之應化亦有緣。(此即竺道生之「應有緣論」)而在有緣之機感中,衆生之所以如此見,佛之所以如此現者,乃是依於衆生之分別事識而見而現。所見所現者,如三十二相、八十種好,此是正報,及淨土,此是依報。從佛之化現說,這是佛「以依智淨,能作一切勝妙境界」。勝妙境界即佛所現起的六根境界,即身語意三業大用。佛所現的色,衆生見為殊勝色;佛所現的聲香味觸,衆生聞觸之,亦皆殊勝。佛所依住之國土,則見為淨土。佛之現也無心,而衆生之見之也有意。有意即是依分別事識「見從外來,取色分齊」。以為這些勝妙境界都是外在的、客觀的,實是來自佛身,而且實有差別分齊之相:相即是相,好即是好,國土即是空間相之國土。「分齊」意即有限量、有邊際,亦即分際限度義。分者部分、分離;齊者整齊、齊一。有分有齊即是有限有際。而且各類眾生機感不同,所見亦異。如凡人見佛,是丈六老比丘相,有三十二相,八十種好。諸天見佛,轉更勝妙,相好亦多,身量尤大。如見佛於菩提場中,在天則見為衆寶莊嚴,在人則見為草木瓦石。又二乘所見者雖有分齊,尚是殊勝妙樂。至於「凡夫所見者,是其粗色。隨於六道,各見不同。種種異類,非受樂相。故說為應身。」此是佛隨類而現,即,隨六道衆生各如其類而示現。此已不只是依分別事識而見佛之色相,且根本是六道衆生停滯於其自己所沉淪之業果之所感見,而佛為度化之,亦如其類而示現。天見其為天,人見其為人,畜生見

\newpage\thispagestyle{empty}\addtocounter{page}{-1}\vspace*{-12mm}\begin{center}\noindent
\includegraphics[clip, trim=183pt 150pt 125pt 248pt, height=162mm]{ocr-input/image-2599.png}\end{center}

\newpage

\noindent 其為獅王、象王、龍王等,餓鬼見其為餓鬼形,地獄亦見其為地獄身。而佛之如此隨類示現,自亦「非受樂相」。即就人而言,亦有時見佛乞化空鉢而歸,有時亦見佛有脊痛之苦。此皆佛之示現,非佛本身即是如此。「示現」如維摩詰示疾,非真有疾。此是真正所謂化身。惟不論凡夫二乘所見之勝妙境界,或只凡夫所見之隨類而現之受苦之「粗色」,皆是「見從外來,取色分齊」。凡夫所見之粗色苦相是凡夫及衆生停滯於其所沉淪之業果之所感見,且根本不知是佛之示現。即凡夫二乘所見之相好以及淨土亦依分別事識而「見從外來,取色分齊」,即認為佛實有如是差別分齊之色相。蓋第六意識以隨順經驗分別事象為其本性。凡夫二乘不知阿賴耶識,更不知如來藏藏識,是以《解深密經》云:「阿陀那識〔阿賴耶別名】甚深細,一切種子如瀑流。我於凡愚不開演,恐彼分別執為我。」凡夫二乘不知其所見之相好以及淨土乃至一切緣起法相皆是其自身阿賴耶識,或如來藏藏識(順《楞伽經》說),通過第七末那識(業識、轉識)之所起現,故執為實有如此差別分齊之色相存在,其實如此差別分齊之色相境相乃至界相皆是凡夫二乘依其自身之分別事識而妄執為如此,這些勝妙境界實只是業識之所起現,實並無自體自相可言也。

報身亦曰佛之自受用身,此是菩薩依於業識(轉識)所見者。此則由分別事識進入業識,已勝於凡夫二乘。蓋大乘菩薩知一切法唯是一心,皆是阿賴那識或如來藏藏識通過第七末那之所起現,一切法唯是一識,一識亦攝一切,故已離分別事識之差別分齊而見其無窮無盡。但還是有相可見,仍在識念之中,故云「依業識」而見爲佛之報身也。故云:「二者,依於業識,謂諸菩薩從初發意乃至

\newpage\thispagestyle{empty}\addtocounter{page}{-1}\vspace*{-12mm}\begin{center}\noindent
\includegraphics[clip, trim=146pt 126pt 146pt 249pt, height=162mm]{ocr-input/image-2603.png}\end{center}

\newpage\markright{附錄 \quad 佛家體用義之衡定}

\noindent 菩薩究竟地心所見者,名為報身。身有無量色,色有無量相,相有無量好。所住依果亦有無量種種莊嚴,隨所示現,即無有邊,不可窮盡,離分齊相。隨其所應,常能住持,不毀不失。如是功德皆因諸波羅蜜等無漏行熏及不思議熏之所造成,具足無量樂相,故說為報身。」

然此無量樂相亦是菩薩依業識而見,仍不能澈至佛如來法身之如如無相,故仍在識念之中。「若離業識」,則無相可見,自己之生命與佛法身如如相應,唯是平等一味。故云:「初發意菩薩等所見者,以深信真如法故,少分而見,知彼色相莊嚴等事,無來無去,離於分齊,雖依心現,不離真如。〔此地前初發心菩薩已見至此,已與凡夫二乘不同】然此菩薩猶有分別,以未入法身位故。若得淨心,所見微妙,其用轉勝。乃至菩薩地盡,見之究竟。〔此是地上菩薩直至十地始能盡見佛之報身,澈盡報身之全蘊。然猶是報身,猶有相也】若離業識,則無見相。以諸佛法身無有彼此色相迭相見故。」此最後是融報身於法身。法身無相,唯是一清淨心。故知所見報身無論如何完整全盡,猶在識念之中。若從第十地金剛後心,斷無明盡,離妄染業識,則即無相可見,唯是法身呈現。不見有相,與佛法身相應,則佛只是一法身,我離業識,我也只是一法身。法身與法身平等一如,佛法身如,我法身亦如,一如無二如,亦無此佛見彼佛,亦無此如異彼如。

依上所說,佛之應化身及報身之用亦只是幻相,不唯應化身是幻相示現,即佛之正報依報(自受用身)亦是幻相,凡依識而見者皆是幻相。(此竺道生所以有法身無色,佛無淨土,善不受報諸義也)因是幻相,故可離可滅。離業識,則當下即寂,無相可見。分

\newpage\thispagestyle{empty}\addtocounter{page}{-1}\vspace*{-12mm}\begin{center}\noindent
\includegraphics[clip, trim=194pt 166pt 140pt 249pt, height=162mm]{ocr-input/image-2607.png}\end{center}

\newpage

\noindent 解地稱理而談,用幻,則用亦可息。消用入體,則無用可說。是則體用不離亦可離。蓋佛教以「流轉還滅」為主綱。流轉依識現,化識還心,則還滅。還滅無相,自亦無識。此是「緣起性空,流轉還滅,染淨對翻,生滅不生滅對翻」綱領下體用不離而可離之體用義。

6.然此體用不離而可離是一條鞭地一直上升說,也就是分解地稱理而談。此若依華嚴宗之判教說,猶是終教見地;若依天台宗之判教說,此猶是「緣理斷九」之別教見地。此尚不是迴轉圓融地說。但此三身之體用可離而不可離,尚可三身如一,無所謂現不現,無所謂見不見之圓融地說,即恆常如是之圓融地說。此是轉分解為圓融,轉直線為曲線之智慧,此是圓教之所以立。

\begin{quotation}\kaishu 問曰:若諸佛法身離於色相者,云何能現色相?答曰:即此
法身是色體故,能現於色,所謂從本已來色心不二。以色性
即智故,無體無形,說名智身。以智性即色故,說名法身徧
一切處。所現之身無有分齊,隨心能示十方世界無量菩薩,
無量報身,無量莊嚴,各各差別,皆無分齊,而不相妨。此
非心識分別能知,以真如自在用義故。\end{quotation}

\noindent 《起信論》此段文即開一圓融地說之義理之門,亦即開一建立圓教之門。吾人可再進而審量此「色心不二」之「真如自在用義」之體用義。

本來,佛成正覺,證涅槃,得法身,如果法身不只是真如空性之「但理」,而且是清淨心,如果涅槃不只是灰身滅智,而是佛無

\newpage\thispagestyle{empty}\addtocounter{page}{-1}\vspace*{-12mm}\begin{center}\noindent
\includegraphics[clip, trim=149pt 141pt 158pt 244pt, height=162mm]{ocr-input/image-2611.png}\end{center}

\newpage\markright{附錄 \quad 佛家體用義之衡定}

\noindent 所謂涅槃不涅槃,涅槃不涅槃只是其示現之相,其本身只是一覺體法身之常住遍在,則成無上正等正覺之法身生命自然有一番氣象可說。「氣象」是儒家名詞,如所謂聖賢氣象者是。孟子言生色醉面盎背。德性潤身(《大學》云:「德潤身」),自有一番氣象可觀。此種氣象自非心識分別所能測知。故孟子云:「充實之謂美,充實而有光輝之謂大,大而化之之謂聖,聖而不可知之謂神。」聖神化境、天地氣象、神明之容、天地之美,自非心識分別所能測知。佛之「無量報身,無量莊嚴」,亦自非心識分別所能測知,此即是「真如自在用義」。籠統說之,成聖成佛,形態一也。然此中亦有本質之差別,仍須就其教義之綱領規範而衡量之。茲仍依「緣起性空,流轉還滅,染淨對翻,生滅不生滅對翻」之義理規範衡量此圓融地說中「色心不二」之「真如自在用義」之意義。

《起信論》以「從本已來色心不二」一存有論的陳述為「法身離相而又能現相」之圓融地說奠立一客觀的基礎,此即是說,法身之色相雖是二乘菩薩依識所見,卻又不只是純主觀地依識所見,而亦是客觀地從佛方面說是佛法身之自然現、真如之自在用。有此客觀基礎,佛法身始能是客觀的真實圓滿之法身,而不只是一條鞭地抽象地說的「只是法身」之孤調。但此義確極微妙複雜,吾人須予以層次之檢別。吾人可問:此「色心不二」一存有論的陳述在什麼情形下始可為「客觀的真實圓滿法身」之客觀的基礎?

6.1首先,識念之生滅流轉並非是心真如體之用。識念起綴網雖憑依不生不滅之心真如體而起現,然心真如體只是其憑依因,並非其生因;而且仍是虛妄不實,有待斷滅,故不能如儒家所說之實事,而仍只是幻事。幻事之直接生因,嚴格說,當該是無明,而不

\newpage\thispagestyle{empty}\addtocounter{page}{-1}\vspace*{-12mm}\begin{center}\noindent
\includegraphics[clip, trim=194pt 185pt 137pt 227pt, height=162mm]{ocr-input/image-2615.png}\end{center}

\newpage

\noindent 是如來藏。無明無根,幻事亦無根。無明本無,幻事亦本無。此即幻事不能為如來藏之用。普通說如來藏緣起,此很易有誤會。詳細說,當該是無明識念憑依如來藏而緣起,並非如來藏自身真能緣起生滅法也。如來藏非生滅法之體,而生滅法亦非如來藏之用,則兩者實亦可說不相資不相待。如真可相資而相待,則妄者從真者出,其真者必不真。此層是分解地說,是順應衆生無始已來而實然地說。《勝鬘經》之不染而染,染而不染,亦是此實然地說。由此遂逼出法藏賢首言真如二義:一不變,二隨緣,而有「不變隨緣,隨緣不變」之說。此「不變隨緣,隨緣不變」亦是順應衆生無始已來而實然地如此說。如在此亦可以說「色心不二」,則此「色心不二」是實然地說的一個經驗命題,縱亦可說是一存有論的陳述,亦是現象的、實然的存有論之陳述。在此意義下,色心不二,而實亦二。因不變隨緣非體用故;又以隨緣淨法不斷,而染法須斷故。此即雖云不二,而實二也。此實然地說的「色心不二」不能為客觀的真實圓滿法身之客觀基礎。然則在什麼情形下,「色心不二」始真實成立,成為一必然命題,而可為客觀的真實圓滿法身之客觀基礎?

6.2依常途想,一綜和命題(色心不二是一綜和命題),要想為一必然的,須有一超越的根據。但在此,吾人不能以如來藏心為其超越的根據,亦不能因無明識念之生滅流轉皆憑依如來藏而即謂此實然層上之色心不二即以如來藏為其超越的根據,因而得為必然的。因為彼雖憑依如來藏而起現,而如來藏卻對之並不負責故,即兩不相應故,生滅流轉須待還滅故。依此,在佛家,此「色心不二」一綜和命題,要想成為必然的,須有另一種講法。此非直線思

\newpage\thispagestyle{empty}\addtocounter{page}{-1}\vspace*{-12mm}\begin{center}\noindent
\includegraphics[clip, trim=149pt 164pt 158pt 222pt, height=162mm]{ocr-input/image-2619.png}\end{center}

\newpage\markright{附錄 \quad 佛家體用義之衡定}

\noindent 考所能解答。幻事雖憑依如來藏,而卻不能以如來藏為其能起現之體,且須還滅,因而亦不能為如來藏之用。但是如來藏真心呈現而為法身又必須有自在用,能現於色相,否則法身只是真如空性之「但理」,而不成其為法身。(但理只能是自性身,與法身作依止,而不能即是法身)如是,法身之自在用所現之色相,從何處得來?豈於生滅流轉之色相外,別有一套不生不滅之色相乎?如別有一套,則成隔絕;如不別有,則成但理。此即是兩難。此兩難之情形或可如此說:如果生滅流轉之色相必須還滅,則法身即成但理而不成其為法身;如果法身不是但理而是法身,而又不能別有一套色相,則必須還滅之生滅流轉之色相即不能還滅。此是兩難之矛盾。吾人必須衝破此矛盾,然後「色心不二」始能得其必然,而客觀的真實圓滿之法身始有其客觀的基礎而得其極成。但吾人如何能衝破此矛盾?此在佛家,似乎是依一種辯證的詭辭、曲線的智慧來解答,而不是超越的分解所能解答者。

6.3原「色心不二」一觀念之提出而復有其必然性,其原初之來歷只是菩薩道之不舍眾生。菩薩成佛必以一切眾生得度為條件為內容。是以佛之涅槃法身決不似小乘涅槃之貧乏與可憐,乃實是充實飽滿,涉及一切,以一切得度為內容。故《起信論》云:「真如用者,所謂諸佛如來,本在因地,發大慈悲,修諸波羅蜜,攝化衆生,立大誓願,盡欲度脫等衆生界,亦不限劫數,盡於未來。以取一切衆生如己身故,而亦不取衆生相。此以何義?謂如實知一切衆生及與己身真如平等無差別故。」此以真如平等無二無別之絕對普遍性(平等性)統攝盡未來際一切眾生於一己身,已身即眾生身,衆生身即是己身,而亦不取衆生相,此即無我相,無人相,無

\newpage\thispagestyle{empty}\addtocounter{page}{-1}\vspace*{-12mm}\begin{center}\noindent
\includegraphics[clip, trim=155pt 176pt 148pt 218pt, height=162mm]{ocr-input/image-2623.png}\end{center}

\newpage

\noindent 衆生相,亦無壽者相,而一切衆生卻亦在不著相中盡函攝於真如平等之一味。如是,方是圓滿無遺。(此與儒者說:體物不遺,大人連屬家國天下而為一身,仁者渾然與物同體,宇宙內事即是己分內事等等,為同一圓滿之義,形態同,本質異。)真如平等是如來藏心,衆生是一切色相。成佛不能隔絕地孤離地成佛,即是衆生色相中成佛。如是,其法身不是抽象的「但理」,而是真實而具體的圓滿法身。即在此法身之必在衆生色相中呈現,始是實踐地而亦是超越地肯定了衆生色相之必然。所謂肯定不是肯定實然層上無明識念之流轉幻妄,(此流轉幻妄定須還滅,此是實踐地必然的),而是在法身呈現必須就眾生色相、流轉幻妄而呈現,因而始必然地把這流轉幻妄帶住定住而不能空脫(蹈空),此第一步是消極意義的肯定,即拖住帶住而定住它。進一步,由拖住帶住就之而冥寂滅度之,使之融化於真如心而不為礙,則彼生滅流轉即得一真常的意義,而不復是幻妄,不復是無明,此即是轉識成智,化念歸心,而亦無所謂識念。所謂冥寂滅度只是不著相,不計執,寂滅那虛妄之分別,不是抹掉那緣起事(依他起)。如是,雖緣起而亦不緣起,雖生滅而亦不生滅,雖流轉而亦不流轉,此即是當體即真常,並不須抹掉它別尋一隔絕之真常。此是緣起事起了本質的變化,捨無常得一常之意義,捨幻妄得一真實之意義。此第二步是積極意義的肯定,亦可以說是一實踐地超越的肯定,因而得一實踐地超越的必然性。此亦即是所謂不毀不壞,此是由於超越的必然而不毀不壞,此是超越的不壞,此亦即是普通所謂「去病不去法」之意義。此超越的肯定,超越的必然,超越的不壞,實只是「去病不去法」之一語。到此,始是真正的「色心不二」,「色心不二」始真成為一個

\newpage\thispagestyle{empty}\addtocounter{page}{-1}\vspace*{-12mm}\begin{center}\noindent
\includegraphics[clip, trim=167pt 147pt 131pt 233pt, height=162mm]{ocr-input/image-2627.png}\end{center}

\newpage\markright{附錄 \quad 佛家體用義之衡定}

\noindent 必然的命題,因而始可以為客觀的真實圓滿法身之客觀基礎,它成了一個超越的存有論的陳述,由對於緣起色相之超越的肯定,必然不壞,而呈現其自己為一超越的存有論的陳述。在此是無所謂「不變隨緣,隨緣不變」的,它已超越了這一層,而成為不變與緣起相應如如而為一客觀的真實圓滿之法身,而法身亦與依報應化不分的,而亦無所謂佛之現與衆生之依識而見的,此即是「真如之自在用」,既離色相而又能現色相。

6.4無明識念雖憑依如來藏而起,然卻原是無根的,說完就完;而「不變隨緣,隨緣不變」亦原是順那無根的無明識念而下來的,是以亦原是經驗的陳述、實然的陳述,並無必然性。此皆是順應衆生無始已來而實然地如此說。但當通過「就之而滅度之而復因而能超越地肯定之」這一曲線的智慧,辯論的奇詭時,它本身起了質的變化,遂因與如來藏心相應如如而獲得一真常的意義、必然的意義,且得一無窮無盡的意義。因為法身恆常遍在,作為其自在用之衆生色相自亦恆常遍在、無窮無盡。這些在超越的肯定、必然、不壞中而取得真常義、必然義、無窮無盡義,而作為法身自在用的色相,嚴格說,亦不是如來藏心真如體自身之所創生起現,而只是順應那衆生無始已來原有的緣起色相融化之而使之與己相應,遂成為其自己之自在用。這些緣起色相,順無明識念下來,原是與如來藏心不相應的,但雖不相應,卻也虛繫無礙,所謂「不變隨緣,隨緣不變」者是。但現在通過超越的肯定,則即成為與如來藏心如如相應,遂成為其自己之自在用,但其底子還是虛繫無礙,不過原來是不相應的虛繫無礙,現在卻是相應的虛繫無礙。不相應,雖虛繫無礙,亦不能成為如來藏心之用。相應,雖虛繫無礙,卻可以為其

\newpage\thispagestyle{empty}\addtocounter{page}{-1}\vspace*{-12mm}\begin{center}\noindent
\includegraphics[clip, trim=154pt 152pt 154pt 244pt, height=162mm]{ocr-input/image-2631.png}\end{center}

\newpage

\noindent 自在之用。然亦正因是虛繫無礙的自在用,故亦是用而非用,結果只是法身之常住。是以雖云「性起」,而實不起。自在用者只是緣起色相通過超越的肯定把它轉為與真如心自己相應耳。在此,實有點假託的意味,假託原有者使其虛繫無礙與己相應,遂成為自己之自在用。此仍是佛教滅度教義下特別形態之體用。是故雖色心不二,能現色相,而實則是相而非相,無相之相;用而非用,無用之用;所謂無量功德,而實亦無一功德之相。此與就衆生緣起色相而執為實有差別之相者不同,此乃是就衆生緣起色相而不執實,而當體寂滅之,而復因而亦不壞之,(即所謂超越的肯定),所示映、映射出之虛的意義:是意義之相,不是材質的相;是意義的用,不是材質的用,是意義的功德相,不是材質的功德相;每一相、每一用、每一功德,皆是一「意義」,而實不是相;「相」是由緣起處之材質之相所示映而虛說,因而亦無所謂「每一」之多,而只是等同一味,「多」亦是由示映而虛說。此即前文所謂「有過恆沙等妄染之義」,則翻上來即「有過恆沙等功德相義示現」。此實只是法身之豐富的意義、豐富的內容,而這些意義、內容渾融而為一意義一內容。這些意義與內容之多義與相義,由緣起處之材質之相所示映映射而虛說,則當眾生不皆成佛時,即可隨衆生之機感而反映回去,此即成凡夫二乘菩薩所依識而見的應報身,此時即有三身之分別說。而在佛自身實無如此之分別。其示映進來與反映回去皆是通過還滅修行後所不期然而然者,此即是其「自在用」。

是以「色心不二」下真如之自在用,其實義當不過如此。雖滅無明識念,而法身不是「但理」;法身雖能現色相,而亦不是別有一套材質之相。此一曲線的智慧、辯證的詭譎逐衝破那分解地說的

\newpage\thispagestyle{empty}\addtocounter{page}{-1}\vspace*{-12mm}\begin{center}\noindent
\includegraphics[clip, trim=173pt 125pt 122pt 256pt, height=162mm]{ocr-input/image-2635.png}\end{center}

\newpage\markright{附錄 \quad 佛家體用義之衡定}

\noindent 直線思考之兩難,而達一圓融之境。圓融即具體而真實,分解則抽象而偏滯,故由分解必達圓融,此即圓教之所以立。

\section*{四、圓教下究竟體用義之確定}\addcontentsline{toc}{section}{四、圓教下究竟體用義之確定}

1.在中國,發展圓教者有兩系統,先有天臺,後有華嚴。本文中心是從《起信論》說起,故此節亦先從華嚴說起以與天台對比。

華嚴宗以《起信論》之超越分解為其義理根據,由之以明「大緣起陀羅尼法」。此大緣起法界,若自究竟果證、十佛自境界方面說,則如來藏心與緣起業用如如為一,一即一切,一切即一,不思議,不可說,無相狀,無分齊,圓融自在,唯是一真法界,亦唯是一毘盧遮那圓滿法身。但若「隨緣約因」,則有相可示,以顯無盡。

就可說方面說,此緣起法界,其超越根據是如來藏心。如來藏心是不變者,而不變者卻隨緣能作染淨法之因。淨法是稱性功德,與本覺相應相順,而染法則不相應而相違,雖相違,亦以如來藏心為憑依而緣起,其關鍵是無明識念之起總綴,此皆須先預定而不能背者。

依是,染淨法緣起皆統於如來藏,此可曰超越的統。就淨法說,是統且具,就染法說,是統而不具。署名為慧思造的《大乘止觀法門》以真心本具染淨二性,並以染性染事亦為不空如來藏之說明,此皆非是,違經(《勝鬘》),亦違論(《起信》),此乃由於對「統」字之誤解。(此書屬於如來藏系統,大體以《起信論》為根據,非天臺宗之作品,顯屬偽託)。

\newpage\thispagestyle{empty}\addtocounter{page}{-1}\vspace*{-12mm}\begin{center}\noindent
\includegraphics[clip, trim=165pt 151pt 147pt 250pt, height=162mm]{ocr-input/image-2639.png}\end{center}

\newpage

天臺宗智者大師的本義及刑溪與知禮之所解,皆就「一念三千」言性具、理具或體具。此一念乃識念,非真心。智者《摩訶止觀》不以《起信論》為義理根據。他是以《中論》之「空假中」收於止觀上而展開其圓教。故以「心不思議境」為觀體。此「心不思議境」非真心,乃即「介爾心」、剎那心。如說念,亦是煩惱識念。介爾一念,即具三千世間,故不思議。由此不思議之觀體展開「即空即假即中」之圓教。此「具」是更為內在地具、心理學地具,是更為內容的、強度的,此是真正的固具。而染淨之關鍵,則在止觀中之迷悟。迷則全體是眾生,悟則全體是「實相」。而真心、真智真解脫、真法身,亦於此顯,並不先立一如來藏心為超越體。嚴格說,天台宗屬《中論》系統,不屬《起信論》系統。

1.1此兩系統之「具」不同,故理具、性具、體具中之理字、性字、體字,義亦有別。在賢首,是指如來藏真心說;在智者,是指具而未顯說。故一實一虛。虛說的理字、性字、體字,是從「介爾有心,即具三千」之「即具」中直接邏輯上分析出來的,是屬於語意字,不屬於指實字。「理具」者是說介爾有心,這一剎那心原則上即具(理上即具)三千世間,「性具」者是說這一剎那心本質上即具三千世間,「體具」者即是那理字性字之別名,亦是抒意的虛說,非指實字,非實體字。在華嚴宗,若就如來藏心對染淨法言,只能說真心具淨法,不能說具染法,在此是不圓,是別教;但通過修行,而至因圓果滿時,則性起即性具。但此性起性具唯就十佛(十身佛)自身說。因之圓是就果之滿而反溯說。故皆屬性起,不屬緣起。(參看上節賢首(性起品探玄〉文)。此唯是三身(在華嚴即說十身)之清淨法;所起者是此,所具者亦是此,起

\newpage\thispagestyle{empty}\addtocounter{page}{-1}\vspace*{-12mm}\begin{center}\noindent
\includegraphics[clip, trim=170pt 128pt 132pt 252pt, height=162mm]{ocr-input/image-2643.png}\end{center}

\newpage\markright{附錄 \quad 佛家體用義之衡定}

\noindent 即具,起具一也。在此如亦說「緣起」,即是究竟果證上圓融自在無窮無盡之「法界緣起」;此是「海印三昧」之緣起:妄盡心澄,萬象齊現;亦是起而無起,故曰性起,不曰緣起也。但天臺宗之性具、理具之圓卻自始即就「介爾心」,並不是就究竟果證上說。今舍究竟果證之圓不說,如就天台宗之性具對如來藏系統之分解地說的如來藏心統攝一切法這一層而言,天臺宗之性具是圓教,而華嚴宗之如來藏心統一切之統即不能算圓教。蓋此統是隨順眾生無始已來而分解地實然地說,中間須經過阿賴耶識之一曲,由此一曲建立一憑依關係而統於如來藏,此若說具,是「曲具」,非天台宗之「直具」。曲具者偏指清淨真如,須破九界(即六道衆生加二乘菩薩)始能顯,故天台宗方面有「緣理斷九」之譏。破九斷九方顯,在此即不能說「性具」(直具)。

1.2「性具」不同,故「不變隨緣,隨緣不變」解義亦不同。此兩語不見智者《摩訶止觀》,乃是賢首先發。《一乘教義分齊章》說真如有二義:一不變,二隨緣。故有「不變隨緣,隨緣不變」之說。但須知此兩語是實然層上的分解說,在究竟果證之圓融自在上即不能說此兩語。「不變隨緣」中之隨緣是緣起,不是性起,故在此亦不能說性具。縱使依緣而起的淨法如種種修行方便,是依真如心而起,但亦是依因托緣而起,在過程上仍屬緣起,尚不能直接說性起。(參看上節賢首〈探玄〉文融攝門與定義門)。至如隨緣起之生滅流轉之染法更非性起,乃是識起,雖憑依性,而非性起,更不能說性具。後來圭峰宗密等華嚴之後學,以賢首宗旨解荊溪「十不二門」,故有知禮之辨駁。其辨駁甚的當,足彰天臺性具圓教之諦義。依天臺,一念三千性具之理不變,而「事造」即隨

\newpage\thispagestyle{empty}\addtocounter{page}{-1}\vspace*{-12mm}\begin{center}\noindent
\includegraphics[clip, trim=159pt 162pt 133pt 224pt, height=162mm]{ocr-input/image-2647.png}\end{center}

\newpage

\noindent 緣。造即由「具」而顯。造是扣緊「具」而言者,非泛言之通義。此天臺後學在華嚴宗興起之後,借用「不變隨緣」之語,而解義不同。

2.以上總說大義,以下引文證之。

荆溪湛然《金剛錍》云:

\begin{quotation}\kaishu 言心造心變,咸出大宗。小乘有言,而無其理。然諸乘中,
其名雖同,義亦少別。

有共造依報,各造正報。有共造正報,各造依報。眾生迷
故,或謂自然、梵天等造;造已,或謂情與無情〔情是有情
衆生,無情是草木瓦石】。故造名猶通〔言通佛教大小乘及
外道〕,應云心變〔言心識變現】。心變復通,應云體具
〔此即天台之理具、性具】。以無始來,心體本遍【此心體
不指如來藏真心言】。故佛體遍,由生性遍〔言佛體遍攝一
切實由眾生之煩惱心性遍滿一切〕。

遍有二種,一、寬廣遍。二、即狹遍。〔造即寬廣遍,具則
即狹遍。】所以造通於四〔言通於藏通別圓四教】,變義唯
二〔言只是限於別圓二教〕。即具唯圓,及別後位〔言「即
具」只限於圓教及別教之後位】。故藏通造六〔言藏教通教
只言造六道衆生法界】,別圆造十〔言別圓二教則言造十法
界:六道加二乘菩薩及佛,名曰十法界。〕此六及十,括大
小乘教法罄盡。由觀解異,故十與六,各分二別。〔言造六
中有藏通之別,造十中有別圓之別。此兩種分別由觀解不同
而成。\end{quotation}

\newpage\thispagestyle{empty}\addtocounter{page}{-1}\vspace*{-12mm}\begin{center}\noindent
\includegraphics[clip, trim=164pt 148pt 137pt 236pt, height=162mm]{ocr-input/image-2651.png}\end{center}

\newpage\markright{附錄 \quad 佛家體用義之衡定}

\begin{quotation}\kaishu 藏見六實〔言藏教見到六道衆生為實有,即未去法執也〕,
通見無生〔言大乘通教見到「無生」,即生無自性】,別見
前後生滅〔言別教則見到前後生滅唯是識現〕,圓見事理一
念具足〔言自家圓教則見到一念三千,即在一念事理具
足〕。論生,兩教似等〔言論到生即無生,別圓兩教似無分
別〕。明具,別教不詮〔言說到理具,則單屬圓教,別教不
及〕。種具等義,非此可述〔言種子識之具與此理具不
同】。

故別佛性,滅九方見〔言別教佛性偏指清淨真如心而言,非
滅除九界不能顯現〕。圓人即達九界、三道,即見圓伊,三
德體遍〔言由一念三千當下即見到般若、解脫、涅槃三德遍
通一切,而且即見到三德即一而三,即三而一之圓伊。]

一家所立不思議境,於一念中,理具三千。故日一念中具有
因果、凡聖、大小依正、自他。故所變處,無非三千。而
此三千,性是中理〔即空即假即中之中理〕。不當有無,有
無自爾。何以故?俱實相故【當下即是「即空即假即中」之
實相】。實相法爾具足諸法。諸法法爾性本無生。故雖三
千,有而不有。共而不雜,離亦不分。雖一一遍,亦無所
在。\end{quotation}

\noindent 由體具而緣現(事造)皆「無非三千」,故色心不二,因果不二,性修不二,染淨不二,內外不二,是真不二。(荆溪《十不二門》,舉此五最重要者例他)。

知禮云:

\newpage\thispagestyle{empty}\addtocounter{page}{-1}\vspace*{-12mm}\begin{center}\noindent
\includegraphics[clip, trim=171pt 156pt 145pt 245pt, height=162mm]{ocr-input/image-2655.png}\end{center}

\newpage

\begin{quotation}\kaishu 他宗明一理隨緣作差別法。差別是無明之相,淳一是真如之
相。隨緣時,則有差別;不隨緣時,則無差別。故知一性與
無明合,方有差別。正是合義,非體不二。以除無明,無差
別故。〔案:此他宗指華嚴宗說】。今家明三千之體隨緣起
三千之用。不隨緣時,三千宛爾。故差別法與體不二。以除
無明,有差別故。〔三千之體,此「體」是虛意字,理具未
顯即是體,顯即是用。此體非分解而立的心真如體之體〕。
驗他宗明即,即義不成。〔案:「即」即「體用即」之
即〕。以彼佛果,唯一真如。須破九界差別,歸佛界一性
故。〔案:此即華嚴宗所謂別教一乘,即唯是十佛自境界。
華嚴宗謂天台宗之一乘圓教是同教一乘,自所立者是別教一
乘。而天台宗則謂之緣理斷九〕。今家以即離分於圓別,不
易研詳。應知不談理具,單說真如隨緣,仍是離義。故第一
記〔即荊溪湛然《法華文句記》卷第一下之第一】云:以別
教中,無性德九故,自他俱斷九也。若三千世間是性德者,
九界無所破,即佛法故,即義方成,圓理始顯。故《金錍》
云:「變義唯二,即具唯圓」。故知具變雙明,方名即
是。若隨闕一,皆非圓極。((十不二門指要鈔〉,解因果不
二門語)\end{quotation}

\noindent 是故華嚴宗之「不變隨緣」是如來藏心之不變隨緣,非此「理具隨緣」。理具隨緣體用相即(由因果不二,性修不二顯)。如來藏心雖「隨緣,仍未即者,為非理具隨緣故也。」(知禮語)。知禮又云:「他宗極圓,只云性起,不云性具,深可思量。」(皆見解因

\newpage\thispagestyle{empty}\addtocounter{page}{-1}\vspace*{-12mm}\begin{center}\noindent
\includegraphics[clip, trim=167pt 131pt 137pt 254pt, height=162mm]{ocr-input/image-2659.png}\end{center}

\newpage\markright{附錄 \quad 佛家體用義之衡定}

\noindent 果不二門)蓋一預定一超越之分解,一不預定故也。

2.1華嚴宗之「不變隨緣」只是真如心不變,而「隨緣成染淨時,恆作染淨,而不失自體。是即不異無常之常,名不思議常。」「由真中不變,依他無性,所執理無,由此三義,故三性一際,同無異也。此即不壞末而常本也。經云:衆生即涅槃,不復更滅也。又約真如隨緣,依他似有,所執情有,由此三義,亦無異也。此即不動本而常末也。經云:法身流轉五道,名曰衆生也。〔……】是故真該妄末,妄澈真源,性相通融,無障無礙。」不變隨緣二義不相違。「且如圓成,雖復隨緣成於染淨,而恆不失自性清淨;只由不失自性清淨,故能隨緣成染淨也。猶如明鏡,現於染淨。雖現染淨,而恆不失鏡之明淨;只由不失鏡明淨故,方能現染淨之相。以現染淨,知鏡明淨;以鏡明淨,知現染淨。是故二義雖是一性。雖現淨法,不增鏡明;雖現染法,不汙鏡淨,非直不汙,亦乃由此反顯鏡之明淨。當知真如,道理亦爾。非直不動性淨成於染淨,亦乃由成染淨方顯性淨,非直不壞染淨明於性淨,亦乃由性淨故方成染淨。是故二義,全體相收。一性無二,豈相違耶?」總之,乃曰:「真該妄未,無不稱真。妄澈真源,體無不寂。真妄交澈,二分雙融,無礙全攝,思之可見。」(法藏賢首:〈華嚴一乘教義分齊章》,「義理分齊」第十,〈三性同異門〉)。

「不異無常之常」,此「不異」只是不離義,仍非理具之「即」。「性相通融,無障無礙」,通融無礙亦非理具之即。此種就如來藏而言之「不變隨緣」,復亦只是《勝鬘經》不染而染、染而不染之「難可了知」之義。實則亦並非真是難可了知,只是聲聞小乘難可了知,大乘菩薩仍能聽受。通過阿賴耶識之一曲折,即可

\newpage\thispagestyle{empty}\addtocounter{page}{-1}\vspace*{-12mm}\begin{center}\noindent
\includegraphics[clip, trim=156pt 164pt 131pt 217pt, height=162mm]{ocr-input/image-2663.png}\end{center}

\newpage

\noindent 了知,此處並無詭譎。《起信》《楞伽》《勝鬘》等真常經論畢竟是華嚴宗之義理綱緯。但須知此處所謂「不變隨緣,隨緣不變」,如上所已指明,是順衆生無始已來而實然地如此說。不變與隨緣非體用義。復依(探玄》文融攝門及定義門,此處亦不能說「性起」:只說隨緣成於染淨,不能說性起染淨;縱然淨法可為性之所起,而染法決不能是性之所起,染法是通過阿賴耶識之一曲而起現,嚴格說是無明識念憑依性而起,是識起,非性起。知禮說華嚴「只云性起,不云性具」,如此性起是從「不變隨緣」處說,則即此「性起」亦不盡諦。(但若從寬依統攝說,眾生及煩惱皆性起攝。參看上節《探玄》文染淨門)依賢首,嚴格而恰當的性起惟自「佛自境界」說,自三身或十身佛自體說,自海印三昧之實德緣起說,自因圓果滿說,惟是淨法,並無染法,此是「大緣起陀羅尼法」,是「隱映互現」之因陀羅法,亦是「一時炳然」之微細法:此處是真正性起義,亦是真正體用義。在此說「因該果海,果澈因源」,亦是性起義,體用義。在此,性起即性具,惟與天臺宗一念三千之「性具」為異指耳。此與「不變隨緣」處之「真該妄末,妄澈因源」不是同一層次之語句,亦不是同一意指之語句。不變隨緣處是緣起,是變化,而「此等並是實義,非變化成。此是如理智中如量境也。其餘變化等者不入此例。何以故?此並是法性家實德,法爾如是也。非謂分別情識境界。此可去情思之。」(《一乘教義分齊章》,「義理分齊」第十、十玄緣起無礙法言「因陀羅網境界門」處)。此固是依《起信論》而成之別教一乘之圓教中之性起性具義,體用義。故言華嚴宗之性起義、體用義,當自因圓果滿處說:不能自不變隨緣處說。知禮之辨是就自家一念三千之理具而與

\newpage\thispagestyle{empty}\addtocounter{page}{-1}\vspace*{-12mm}\begin{center}\noindent
\includegraphics[clip, trim=164pt 141pt 131pt 233pt, height=162mm]{ocr-input/image-2667.png}\end{center}

\newpage\markright{附錄 \quad 佛家體用義之衡定}

\noindent 華嚴之「不變隨緣」對刊也。天臺宗無華嚴家之兩層(實然層與理想層,或不變隨緣層與因該果海層)說,故是同教一乘。

華嚴家從實然層上說:(1)「不變隨緣,隨緣不變。」(2)「不壞末而常本,不動本而常末。」(3)「真該妄未,妄澈真源,性相通融,無障無礙。」(4)「雖復隨緣成於染淨,而恆不失自性清淨。祗由不失自性清淨,故能隨緣成染淨也。」(5)「非直不動性成於染淨,亦乃由成染淨方顯性淨。非直不壞染淨明於性淨,亦乃由性淨故,方成染淨。」(6)三性各二義,皆「全體相收,一性無二。」(7)「三性一際,舉一全收。」這些美妙圓融的辭語皆有催眠性與麻醉性。實皆只是就三性關聯地抒其意之抒意辭語,非是客觀的積極平鋪之體用辭語。若見其圓融無礙,而想其為客觀之平舖、積極之肯定,則成大混雜,是真悖矣。此與果海自體不同。此中之本末非體用也,真妄非體用也,真如之不變隨緣非體用也,依他之似有無性非體用也,徧計之情有理無非體用也。此等皆只是順無明識念之一一曲而實然地說其虛繫無礙。

2.2知禮云:

\begin{quotation}\kaishu 且如《記》文釋阿若文中云:「別教亦得云從無住本立一切
法。」無明覆理,能覆所覆,俱名無住。但即不即異,而分
教殊。既許所覆無住,真如安不隨緣?隨緣仍未即者,為非
理具隨緣故也。

又云:「真如在迷,能生九界」。若不隨緣,何能生九?

又《輔行》釋別教根塵一念爲迷解本,引《楞伽》云:「如
來為善不善因。」自釋云:「即理性如來也。」《楞伽》此\end{quotation}

\newpage\thispagestyle{empty}\addtocounter{page}{-1}\vspace*{-12mm}\begin{center}\noindent
\includegraphics[clip, trim=160pt 173pt 156pt 228pt, height=162mm]{ocr-input/image-2671.png}\end{center}

\newpage

\begin{quotation}\kaishu 句乃他宗隨緣之所據也。《輔行》為釋此義,引《大論》
云:「如大池水,象入則濁,珠入則清。」當知水為清濁
本,珠象為清濁之緣。據此諸文,別理豈不隨緣耶?

故知若不談體具者,隨緣與不隨緣,皆屬別教。何者?如
云:黎耶生一切法,或云:法性生一切法。豈非別教有二義
耶?

問:《淨名疏》〔智者《維摩經玄疏》】釋無明無住云:
「說自性是別教義,依他住是圓教義。」且隨緣義,真妄和
合,方造諸法,正是依他,那判屬別?

答:《疏》中言簡意高。須憑《記》釋,方彰的皆。故釋自
住,法性煩惱更互相望,俱立自他。結云:「故二自他並非
圓義。以其惑性,定能為障。破障方乃定能顯理。」釋依他
云:「更互相依,更互相即,以體同故,依而復即。」結
云:「故別圓教,俱云自他。由體同異,而判二教。」今釋
曰:性體具九,起修九用。用還依體,名同體依。此依方
即。若不爾者,今依義。故《妙樂》云:「別教無性德九,
故自他俱須斷九。」是知但理隨緣作九,全無明功。既非無
作,定能為障。故破此九,方能顯理。若全性起修,乃事即
理。豈定為障,而定可破?若執但理隨緣作九為圓義者,何
故《妙樂》中「真如在迷,能生九界」,判為別耶?故真妄
合,即義未成,猶各自住。【……】此宗,若非荊溪精簡,
圓義永沉也。」((十不二門指要鈔〉,解因果不二門)。\end{quotation}

\noindent 知禮此段文是從法性煩惱兩無住著(無住本)分判華嚴之「不變隨

\newpage\thispagestyle{empty}\addtocounter{page}{-1}\vspace*{-12mm}\begin{center}\noindent
\includegraphics[clip, trim=166pt 141pt 119pt 230pt, height=162mm]{ocr-input/image-2675.png}\end{center}

\newpage\markright{附錄 \quad 佛家體用義之衡定}

\noindent 緣」與天台之「理具隨緣」之不同。前者雖隨緣而不即,故云「猶各自住」。後者隨緣根於理具,故「更互相依,更互相即,以同體故,依而復即」。此言「同體」非同一心真如體,乃理具未顯之體與事造已顯之用為同一「一念三千」之事體之體。此「體」字無實義。

又知禮就「不變隨緣」判華嚴為別教,非圓教,此所云「別教」是依天臺藏通別圓中之「別教」言,是不共小乘之別教,相當於華嚴宗所判之終教。故此「別教」非華嚴宗所自立之「別教一乘」之別教。別教一乘乃指圓教言。然「不變隨緣」之別教正顯其一乘圓教為別教一乘,而非同教一乘也。若依華嚴宗說,「不變隨緣,隨緣不變」固非圓教也。彼固不自此說圓教也。

2.3以上知禮辨兩家隨緣義之不同,極為諦當。是故荊溪云:

\begin{quotation}\kaishu 《涅槃經》中多云佛性者,佛是果人,言一切眾生皆有果人
之性,故偏言之。〔偏有情衆生而言】。世人迷故,而不從
果。云眾生有,故失體遍。〔實則無情亦有,不獨衆生。】
又,云遍者,以由煩惱心性遍,云佛性遍。故知不識佛性遍
者,良由不知煩惱性遍故。唯心之言,豈唯真心?子尚不知
煩惱心遍,安能了知生死色遍?色何以遍?色即心故。何
者?依報共造,正報別造,豈信共遍,不信別遍耶?能造所
造,既是唯心,心體不可局方所故,所以十方佛土皆有眾生
理性心種。(金剛錍〉)。\end{quotation}

\noindent 「唯心之言,豈唯真心」?則知天台性具不偏指清淨真如心也。乃

\newpage\thispagestyle{empty}\addtocounter{page}{-1}\vspace*{-12mm}\begin{center}\noindent
\includegraphics[clip, trim=171pt 157pt 159pt 254pt, height=162mm]{ocr-input/image-2679.png}\end{center}

\newpage

\noindent 是「介爾有心,即具三千」。介爾心、剎那心即煩惱心。煩惱心性遍通一切,故佛性亦遍通一切。此言佛性不是分解地單指清淨真如心而言,乃是通過煩惱心遍而辯證地(曲線地)由煩惱言佛性,此即「煩惱即菩提」一語之確義。迷則全體是煩惱,悟則全體是佛性。心是煩惱心,則「心體不可局方所」中之「心體」不是心真如體或真如心體,乃即此煩惱心之「當體自己」之體。此煩惱心即是「生死色」,即刑溪「色心不二門」所謂「心之色心」也。此不二是真不二。「十方佛土皆有衆生理性心種」,即言皆有煩惱之佛性在。此言「理性心種」,意即煩惱心種即理性,此「理性」亦不是分解地單言真如之理也。

智者《摩訶止觀〉云:「夫一心具十法界,一法界又具十法界:百法界。一界具三十種世間〔十種衆生世間,十種國土世間,十種五陰世間〕,百法界即具三千種世間。此三千在一念心。若無心而已,介爾有心,即具三千。亦不言一心在前,一切法在後。亦不言一切法在前,一心在後。例如八相遷物,物在相前,物不被遷;相在物前,亦不被遷。前亦不可,後亦不可。祗物,論相遷;只相遷,論物。今心亦如是。若從一心生一切法者,此則是縱。若心一時含一切法者,此則是橫。縱亦不可,橫亦不可。只心是一切法,一切法是心故。非縱非橫,非一非異。玄妙深絕,非識所識,非言所言。所以稱為不可思議境,意在於斯。」((摩訶止觀》第七章(正觀〉,觀十境中第一觀陰界入境)。

此即荊溪與知禮言「理具」,辨華嚴之所本也。所謂觀不可思議境者首先觀此不思議之剎那心也。

2.4智者大師首由一念心開三千種世間。復開「一、陰界入,

\newpage\thispagestyle{empty}\addtocounter{page}{-1}\vspace*{-12mm}\begin{center}\noindent
\includegraphics[clip, trim=176pt 131pt 134pt 258pt, height=162mm]{ocr-input/image-2683.png}\end{center}

\newpage\markright{附錄 \quad 佛家體用義之衡定}

\section*{二、煩惱,三、病患,四、業相,五、魔事,六、禪定、七、諸}\addcontentsline{toc}{section}{二、煩惱,三、病患,四、業相,五、魔事,六、禪定、七、諸}

\noindent 見,八、增上慢,九、二乘,十、菩薩」之十境為觀體(爲止觀之所依據)。復以「一、觀不可思議境,二、起慈悲心,三、巧安止觀,四、破法遍,五、識通塞,六、修道品,七、對治助開,八、知次位,九、能安忍,十、無法愛」之十法門觀十境。此〈摩訶止觀》之綱緯也。

賢首後起立華嚴一乘不共之圓教,以「一、教義,二、理事,三、解行,四、因果,五、人法,六、分齊境位,七、師弟法智,八、主伴依正,九、隨其根欲示現,十、逆順體用自在」為十義。復以下列十玄門釋十義:

一、同時具足相應門。

二、一多相容不同門。

三、諸法相即自在門。

四、因陀羅網境界門。

五、微細相容安立門。

六、秘密隱顯俱成門。

七、諸藏純雜具德門。

八、十世隔法異成門。

九、唯心迴轉善成門。

十、託事顯法生解門。.

\noindent 此十玄門及十義「皆悉同時會融,成一法界緣起具德門普眼境界」,亦即「別教一乘緣起義」。

天臺宗一念三千,以十門觀十境,止觀工夫尤為切要。而「別教一乘緣起義」徒是對十身佛自境界之玄思,其實切警策不及天臺

\newpage\thispagestyle{empty}\addtocounter{page}{-1}\vspace*{-12mm}\begin{center}\noindent
\includegraphics[clip, trim=164pt 158pt 143pt 239pt, height=162mm]{ocr-input/image-2687.png}\end{center}

\newpage

\noindent 多矣。

3.由以上之比論,吾人可名華嚴宗之「不變隨緣」以及因圓果滿之「性起」為如來藏真常心之系統,名天台宗之「理具隨緣」為《中論》系統。

華嚴宗原由地論師慧光系傳來,原與地論攝論宗有關,原是繼承初期真諦唯識學而展開。真諦後玄奘宗世親晚年及護法之唯識學,是為後期唯識學。賢首先曾參與玄奘之譯場,後因理不相契,遂出而弘華嚴,以《起信論〉為宗論。至近代歐陽竟無呂秋逸等宗玄奘之唯識學,力闢《起信論》為妄。是今之爭論猶古之異同也。故華嚴宗與(起信論〉之關係特密,而亦與唯識宗始終在角鬥。蓋其入路是一,只一為阿賴耶系統,一為如來藏系統而已耳。〔呂秋逸謂《起信論〉根據魏譯《楞伽》之誤譯而成誤解。《楞伽〉原意,如來藏與藏識(阿賴耶)只是一體二名,只是一藏識名為如來藏。此論據並不可靠。]

但天臺宗則與如來藏、阿賴耶一系統並無多大關係。它是直接根據《中論》之空假中(因緣所生法,我說即是空,亦為是假名,亦是中道義)而收於止觀上講。章安灌頂記(摩訶止觀〉緣起云:「智者《觀心論〉云:歸命龍樹師」。可見智者對於龍樹之推崇。又云:「天臺傳南岳三種止觀:一、漸次,二、不定,三、圓頓。皆是大乘,俱緣實相,同名止觀」。智者所承於其師(南岳慧思)者是止觀傳統。〈摩訶止觀〉中凡提到慧思者,皆云其法門是「隨自意安樂行」。從未提及署名慧思之(大乘止觀法門〉中所說之系統。可見智者對於其前輩地論攝論師之爭論並不感興趣,亦見《大乘止觀法門〉一書之為偽託,並非天臺家義也。

\newpage\thispagestyle{empty}\addtocounter{page}{-1}\vspace*{-12mm}\begin{center}\noindent
\includegraphics[clip, trim=168pt 136pt 129pt 242pt, height=162mm]{ocr-input/image-2691.png}\end{center}

\newpage\markright{附錄 \quad 佛家體用義之衡定}

3.1《摩訶止觀》第七章〈正觀〉,觀不思議境中有云:

\begin{quotation}\kaishu 問:心起必託緣,為心具三千法?為緣具?為共具?為離
具?若心具者,心起不用緣。若緣具者,緣具不關心。若
共具者,未共各無,共時安有?若離具者,既離心離緣,
那忽心具?四句尚不可得,云何具三千法耶?答:地人
云:一切解惑、真妄,依持法性。法性持真妄,真妄依法
性也。《攝大乘》云:法性不為惑所染,不為真所淨,故
法性非依持。言依持者阿黎耶是也。無沒無明盛持一切種
子。若從地師,則心具一切法。若從攝師,則緣具一切
法。此兩師各具一邊。若法性生一切法者,法性非心非
緣。非心故而心生一切法者,非緣故亦應緣生一切法。何
得獨言法性是真妄依持耶?〔案:此評地論師如來藏系
統。此評並不諦,但足見智者並無興趣順他們的分解
說。】若言法性非依持,黎耶是依持,離法性外,別有黎
耶依持,則不關法性。若法性不離黎耶,黎耶依持即是法
性依持,何得獨言黎耶是依持?又違《經》。《經》言:
非內非外,亦非中間,亦不常自有。又違龍樹。龍樹云:
諸法不自生,亦不自他生,不共不無因。【……】云何偏
據法性、黎耶生一切法?當知四句求心不可得,求三千法
亦不可得。既橫從四句生三千法不可得,應從一念心滅生
三千法耶?心滅尚不能生一法,云何能生三千法耶?若從
心亦滅亦不滅生三千法者,亦滅亦不滅其性相違,猶如水
火,二俱不立,云何能生三千法耶?若謂心非滅非不滅生\end{quotation}

\newpage\thispagestyle{empty}\addtocounter{page}{-1}\vspace*{-12mm}\begin{center}\noindent
\includegraphics[clip, trim=232pt 171pt 147pt 232pt, height=162mm]{ocr-input/image-2695.png}\end{center}

\newpage

\begin{quotation}\kaishu 三千法者,非滅非不滅,非能非所,云何能生三千法耶?
亦縱亦橫求三千法不可得,非縱非橫求三千法亦不可得。
言語道斷,心行處滅,故名不可思議境。[……】當知第
一義中,一法不可得,況三千法?世諦中,一心尚具無量
法,況三千耶?如佛告德女:無明內有不?不也。外有
不?不也。内外有不?不也。非内非外有不?不也。佛
言:如是有。龍樹云:不自不他,不共,不無因。《大
經〉云:生生不可說,生不生不可說,不生生不可說,不
生不生不可說。有因緣故,亦可得說。謂四悉檀因緣也。
雖四句冥寂,慈悲憐憫,於無名相中,假名相說。\end{quotation}

\noindent 據此,則知《摩訶止觀〉實據(中論》四句求生不可得,遍破一切偏執,而只假名相說一念三千也。其思路是就一念三千作圓頓止觀,顯「即空即假即中」之實相。自非依據一超越分解講圓教也。此種「理具隨緣」圓教,心思極活,極為空靈,極為警策,亦是極為「作用的」,與華嚴宗真常心之「實體性的」不同也。

華嚴宗之如來藏系統是由唯識宗向超越方面進一步而轉出,天臺宗之理具系統是由空宗向裡收進一步而轉出。在印度,空有平行。在中國,天臺華嚴平行。至禪宗,則是天臺華嚴之簡化,亦是更為作用化。六祖慧能兩語盡之矣。「即心是佛,無心為道」是也。

3.2原智者言一念三千,後來荊溪、知禮所謂理具、性具、體具或圓具,是本以下思路而成:

一、摩訶止觀》第四章〈攝法〉云:

\newpage\thispagestyle{empty}\addtocounter{page}{-1}\vspace*{-12mm}\begin{center}\noindent
\includegraphics[clip, trim=161pt 144pt 135pt 237pt, height=162mm]{ocr-input/image-2699.png}\end{center}

\newpage\markright{附錄 \quad 佛家體用義之衡定}

\begin{quotation}\kaishu 六、攝一切教者,《毘婆沙〉云:「心能為一切法作名
字。」若無心,則無一切名字。當知世出世名字悉從心起。\end{quotation}

二、〈摩訶止觀〉第七章〈正觀〉,第一觀陰界入境開首云:

\begin{quotation}\kaishu 觀陰入界境者,謂五陰、十二入、十八界也。陰者,陰蓋善
法,此就因得名。又陰是積聚,生死重沓,此就果得名。入
者涉入,亦名輸門。界各界別,亦名性分。【……】若依
《華嚴》云:「心如工畫師,造種種五陰。界內界外「—
切世間中,莫不從心造。」世間色心,尚叵窮盡,況復出
世,寧可凡心知?[……]然界內外一切陰入皆由心起。佛
告比丘:「一法攝一切法,所謂心是。」《論》云:「一切
世間中,但有名與色。若欲如實觀,但當觀名色。」心是惑
本,其義如是,若欲觀察,須伐其根。如炙病得穴。\end{quotation}

三、又第四章〈攝法〉云:

\begin{quotation}\kaishu 復次,心攝諸教略有兩意:一者一切眾生心中具足一切法
門。如來明審,照其心法,按彼心說。無量教法,從心而
出。二者、如來往昔曾作漸頓觀心,偏圓具足。依此心觀,
為眾生說。教化弟子,令學如來。破塵出卷,仰寫空經,故
有一切經卷。悉為三止三觀所攝也。\end{quotation}

四、又第一章〈大意〉,論六即中云:

\newpage\thispagestyle{empty}\addtocounter{page}{-1}\vspace*{-12mm}\begin{center}\noindent
\includegraphics[clip, trim=227pt 157pt 136pt 227pt, height=162mm]{ocr-input/image-2703.png}\end{center}

\newpage

\begin{quotation}\kaishu 理即者,一念心即如來藏理。如故即空,藏故即假,理故即
中。三智一心中,具不可思議。如上說,三諦一諦,非三非
一。一色一香,一切法,一切心,亦復如是。是名理即是菩
提心,亦是理即止觀。即寂名止,即照名觀。\end{quotation}

\noindent 根據以上四點,智者所謂「一念三千」,此中「一念」是指剎那心、陰入心,亦即「無明一念心」言。〈摩訶止觀》第七章〈正觀〉第四、破法遍中第三橫豎一心明止觀云:「若無生門千萬重疊,只是無明一念、因緣所生法即空即假即中不思議三諦、一心三觀、一切種智、佛眼等法耳。無生門爾,諸餘橫門亦復如是。雖種種說,只一心三觀,無橫無豎。」「介爾有心,即具三千世間」。此心乃無明一念心,非偏指清淨真如心也。通過圓頓止觀工夫,此無明一念心即是清淨真如心;但不是分解地顯示,而是即在一念三千中作用地顯示。此圓頓止觀、不思議三諦、三智即是般若、解脫與法身。般若之用在此顯,而清淨真如心之體亦在此證。即體(真如心)即用(般若),即用即體,總在「無明一念心,此心具三諦;體達一觀,此觀具三觀」(破法遍中第三橫豎一心明止觀中語)中顯示。此為作用地顯示,非分解地預定一如來藏真如心也。

達成此三諦三觀之方法,大體是根據:

一、〈中論》「諸法不自生,亦不自他生,不共不無因,是故總無生」;

二、《涅槃經〉「生生不可說,生不生不可說,不生生不可說,不生不生不可說。」

\newpage\thispagestyle{empty}\addtocounter{page}{-1}\vspace*{-12mm}\begin{center}\noindent
\includegraphics[clip, trim=156pt 137pt 133pt 237pt, height=162mm]{ocr-input/image-2707.png}\end{center}

\newpage\markright{附錄 \quad 佛家體用義之衡定}

\noindent 兩方式而觀達一切——亦即遍破一切,遍立一切。

第七章〈正觀〉破法遍中開頭無生門破法遍云:「《佛藏》云:『劫火起時,菩薩一唾火即滅,一吹世界即成。非是先滅後成,只一唾中即滅即成』。彼經明外用內,合無生門,即破遍,即立遍,破立不須二念。若內無是德,則外無大用。寄外顯內,其相如是。須識觀心者,眾生一期將訖,即是劫盡。三毒三災火為語端。以止止之,如唾滅;以觀觀之,如吹成。」據此可知智者(摩訶止觀》是將一切分解說的經論教義由圓頓止觀作用地、詭譎地而消融之,復是作用地、詭譎地、遮詮地以明圓教,非如華嚴宗之順如來藏系統分解地明圓教也。分解地明圓教,是別教一乘;作用地明圓教,是同教一乘。別教一乘,緣理斷九,圓唯在佛。同教一乘,一念三千,當下即達九界,不待斷九始圓也。是故天臺是《中論》般若學系統,華嚴是《起信論》真常心系統。

3.3智者《法華玄義》卷第二上正解法字中云:

\begin{quotation}\kaishu 南岳師舉三種,謂眾生法、佛法、心法。

心法妙者,如《安樂行》中:「修攝其心,觀一切法不動
不退。」〔此略引《法華安樂行品》〕。又一念隨喜等。
〔荆溪湛然《釋籤》云:「又一念心隨喜等者,即觀行位
初,祗於貪瞋一念心起,體即權實,諸皆例然。隨順三
諦,故云隨喜。是故隨喜名心法妙。」】普賢觀云:「我
心自空,罪福無主」。〔《釋籤》云:「普賢觀意者,心
體即理,故云自空。誰執罪福?故云無主。應遍十界以明
罪福在一念心,方成妙觀。」】觀心無心,法不住法。又\end{quotation}

\newpage\thispagestyle{empty}\addtocounter{page}{-1}\vspace*{-12mm}\begin{center}\noindent
\includegraphics[clip, trim=161pt 158pt 139pt 231pt, height=162mm]{ocr-input/image-2711.png}\end{center}

\newpage

\begin{quotation}\kaishu 心純是法。〔《釋籤》云:「觀心無心等者,能緣之心
無,所緣之法安在?能所不二,故云純是。」〕破心微
塵,出大千經卷。是名心法妙也。\end{quotation}

又云:

\begin{quotation}\kaishu 若廣眾生法,一往通論諸因果及一切法。若廣佛法,此則據
果。若廣心法,此則據因。\end{quotation}

又云:

\begin{quotation}\kaishu 三、廣釋心法者,前所明法,豈得異心?但眾生法太廣,佛
法太高,於初學為難。然「心佛及眾生,是三無差別」者,
但自觀己心,則為易。

《涅槃》云:「一切眾生,具足三定」。上定者,謂佛性
也。能觀心性,名為上定。〔《釋籤》「應了此性具足佛法
及衆生法。雖復具足,心性冥妙,不一不多。以心性觀,則
似可見。若以衆生及佛而為觀者,則似如不逮。若以心性觀
彼界如,界如皆空,常具諸法。非空非具,而空而具。雙遮
雙照,非遮非照,亦只是一念性而已。如是之定豈不尚
耶?」〕上能兼下,即攝得眾生法也。

《華嚴〉云:「遊心法界如虚空,則知諸佛之境界。」法界
即中也,虛空即空也,心佛即假也。三種具,即佛境界也。
是為觀心仍具佛法。\end{quotation}

\newpage\thispagestyle{empty}\addtocounter{page}{-1}\vspace*{-12mm}\begin{center}\noindent
\includegraphics[clip, trim=163pt 140pt 141pt 245pt, height=162mm]{ocr-input/image-2715.png}\end{center}

\newpage\markright{附錄 \quad 佛家體用義之衡定}

\begin{quotation}\kaishu 又,遊心法界者,觀根塵相對一念心起,於十界中必屬一
界。若屬一界,即具百界千法〔百界千如〕。於一念中,悉
皆備足。此心幻師,於一日夜,常造種種眾生,種種五陰,
種種國土,所謂地獄假、實、國土,乃至佛界假、實、國
土。〔《釋籤》云:「假即衆生,實即五陰及以國土,即三
世間也。千法皆三,故有三千。」〕行人當自選擇何道可
從0

又,如虛空者,觀心自生心,不須藉緣。藉緣有心,心無生
力。心無生力,緣亦無生。心緣各無,合云何有?合尚叵
得,離則不生。尚無一生,況有百界千法耶?以心空故,從
心所生一切皆空。此空亦空。若空非空,點空設假,假亦非
假。無假無空,畢竟清淨。〔案:此即作用地、詭譎地、遮
顯清淨真如心也。]

又復佛境界者,上等佛法,下等眾生法。

又,心法者,心佛及眾生,是三無差別。是名心法也。\end{quotation}

此言一心,同于《摩訶止觀》,皆不指清淨真如心也。

3.4然並非不承認空不空如來藏,唯認其為別教四門耳。

《摩訶止觀》第七章〈正觀〉,第四破法遍中從假入空破法遍最後四門料簡云:

\begin{quotation}\kaishu 次別教四門者,即是觀別理,斷別惑,不與前同;次第修,
次第證,不與後同。《大經》云:「聞大涅槃有無上道,大
眾正行,發心出家,持戒修定,觀四諦慧,得二十五三\end{quotation}

\newpage\thispagestyle{empty}\addtocounter{page}{-1}\vspace*{-12mm}\begin{center}\noindent
\includegraphics[clip, trim=170pt 153pt 144pt 245pt, height=162mm]{ocr-input/image-2719.png}\end{center}

\newpage

\begin{quotation}\kaishu 昧。」事相次第,不殊三藏,但以大涅槃心導於諸法,以此
異前;漸修五行,以此異後。故稱為別。

言四門者,觀幻化見思,虛妄色盡,別有妙色,名為佛性。
《大經》云:「空空者,即是外道。解脫者,即是不空,即
是真善妙色。如來秘藏,不得不有。」又:「我者,即如來
藏,如來藏者即是佛性。」《如來藏經》云:「幣帛裹金,
土模內像。」凡有十譬等,即是有門也。〔案:此即不空如
來藏]

空門者,《大經〉云:「迦毘城空,如來藏空,大涅槃
空。」又云:「令諸眾生悉得無色大般涅槃。」涅槃非有,
因世俗故,名涅槃有。涅槃非色非聲,云何而言可得見聞?
即是空門。〔案:此即空如來藏】

亦空亦有門者,智者見空及與不空。若言空者,則無常樂我
淨。若言不空,誰復受是常樂我淨?如水酒酪瓶,不可說空
及以不空。是名亦空亦有門。

非有非無門者,絕四離百,言語道斷,不可說示。〈涅槃》
云:「非常非斷,名為中道」。即是其門也。

如此四門得意,通入實相。若不得意,伏惑方便,次第意
耳。《涅槃》名為菩薩聖行。〈大品〉名為不共般若。此皆
是別教四門意,非今所用也。

圓教四門:妙理頓說,異前二種〔藏通】。圓融無礙,異於
歷別。云何四門?

觀見思假,即是法界,具足佛法。又諸法即是法性因緣,乃
至第一義亦是因緣。《大經》云:「因滅無明,即得熾燃三\end{quotation}

\newpage\thispagestyle{empty}\addtocounter{page}{-1}\vspace*{-12mm}\begin{center}\noindent
\includegraphics[clip, trim=164pt 137pt 140pt 248pt, height=162mm]{ocr-input/image-2723.png}\end{center}

\newpage\markright{附錄 \quad 佛家體用義之衡定}

\begin{quotation}\kaishu 菩提燈。」是名有門。

空門者,觀幻化見思及一切法,不在因,不在緣。我及涅
槃,是二皆空。惟有空病。空病亦空,此即三諦皆空也。

云何亦空亦有門?幻化見思,雖無真實,分別假名,則不可
盡。如一微塵中,有大千經卷。於第一義而不動,善能分別
諸法相。亦如大地一,能生種種芽。無名相中,假名相說。
乃至佛亦但有名字,是為亦有亦無門。

云何非有非無門?觀幻化見思即是法性。法性不可思議。非
世,故非有。非出世,故非無。一色一香,無非中道。一中
一切中。毘盧遮那遍一切處,豈有見思而非實法?是名非有
非無門。\end{quotation}

\noindent 據此四門判教,則知智者並非不承認有空不空如來藏之說,惟一方既視為別教(天台藏通別圓之別,非華嚴別教一乘之別),一方匠心獨運亦不順此路明圓教·順此路就《華嚴經》明圓教者是華嚴宗,是杜順、智儼、賢首之一系。《起信論》及其所根據之真常經所說之空不空如來藏,華嚴宗判為大乘終教。順此終教思路之分解就《華嚴經》明圓教,其所明之一乘圓教,賢首自判為別教一乘圓教,而判天台之圓教為同教一乘圓教。同者,言其開權顯實,「一乘垂於三乘,三乘參於一乘」,以一同三,一三和合也。(賢首《華嚴一乘教義分齊章》建立一乘第一)。別者,言其唯就毘盧遮那佛圓滿法身說,隔別機權,唯是自得自證之一實,所謂「稱法本教」,非「逐機未教」者是也。

天台、華嚴俱認《華嚴經》為佛成道後第一時說。所謂稱法本

\newpage\thispagestyle{empty}\addtocounter{page}{-1}\vspace*{-12mm}\begin{center}\noindent
\includegraphics[clip, trim=188pt 164pt 142pt 248pt, height=162mm]{ocr-input/image-2727.png}\end{center}

\newpage

\noindent 教,所謂別教一乘,「即佛初成道,第二七日,在菩提樹下,猶如日出,先照高山,於海印定中,同時演說十十法門。主伴具足,圓通自在。該於九世十世,盡因陀羅、微細境界。即於此時,一切因果理事等,一切前後法門,乃至末代流通舍利、見聞等事,並同時顯現。何以故?卷舒自在故。舒則該於九世,卷則在於一時。此卷即舒,舒又即卷。何以故?同一緣起故,無二相故。〔……〕是故依此普法,一切佛法並於第二七日,一時前後說,前後一時說。如世間印法,讀文則句義前後,印之則同時顯現。同時前後,理不相違。」(《華嚴一乘教義分齊章》,第六「教起前後」)。而天臺則就「譬如日出,先照高山」(《華嚴經·出現品》),判為第一時初說,亦曰華嚴時;就「譬如從牛出乳」(《涅槃經》),判為「乳味」。

根據此義,則有一義可說,即:此種圓教亦可說是「形式的圓教」、「形式的一乘」。其言「別」雖可顯此「稱法本教」之獨特、殊勝與最高,然亦有抽象之隔別義。雖在法上說,一切佛法俱在其內,無隔無別,然此只是佛「稱性極談,如所如說」,佛初成正覺,稱所證法性之理而說。故此無隔無別是自證之無隔無別。自其不顧群機而言,實亦是隔別。隔別即抽象。隔別單顯佛自身之圓滿,抽象單顯圓滿真理之本義。此亦如單顯真理之標準,只此標準之自己便是抽象。不隔別,不抽象,不能顯出此標準。雖就佛自所證說,是具體,而非抽象,即,其自己真是證到,而非只是抽象地見到,然就普接群機而為客觀地證現言,則仍是隔別,仍是抽象。即依此義,而說為形式的圓教、形式的一乘。此種圓教,客觀地說,是圓教之在其自身,主觀地說,是佛圓滿法身之在其自身。在

\newpage\thispagestyle{empty}\addtocounter{page}{-1}\vspace*{-12mm}\begin{center}\noindent
\includegraphics[clip, trim=150pt 139pt 145pt 242pt, height=162mm]{ocr-input/image-2731.png}\end{center}

\newpage\markright{附錄 \quad 佛家體用義之衡定}

\noindent 其自身,即是圓教之模型、圓教之標準,即是形式的圓教。但模型、標準在其自身,必須經過「對其自身」而成為「在而對其自身」(在其自身與對其自身之真實統一),方是客觀地真實而具體之圓教。此則便不能隔別群機而不顧,便不能只「稱性極談」而顯高,亦須就機而顯普。聖人必須俯就,泛應曲當而無礙,其道方具體,其圓教方具體而真實,此方是具體而真實的「圓而神」。就此而言,華嚴宗之就第一時與乳味之《華嚴經》而說圓教實不及天台宗之就第五時(法華、涅槃時)與醍糊味而說圓教為真實而具體,為真正之圓教。

就義理之發展說,(凡判教俱就義理秩序說,非就歷史次序說),天台之判教實比較如理如實,精熟而通透。華嚴之判教以及其所說之圓教,是超越分解思路下的判教與圓教,天台之判教以及其所說之圓教,是辯證圓融思路下的判教與圓教,是通過那些分解而辯證詭譎地、作用地、遮詮地消融之圓教。天臺五時判教如下:

一、華嚴時:日照高山,乳味,稱理而談,以顯形式的圓教。

二、鹿苑時:日照幽谷,酪味,說四《阿含》小乘教。

三、方等時:食時,生酥味,說《維摩》、《思益》、《楞伽》、《金光明》、《勝鬘》等經。

四、般若時:禺中時,熟酥味,說《般若經》。

五、法華、涅槃時:日輪當午,醍醐味,從《般若》出《法華》、《涅槃》。

\noindent 經過前四時,至最後第五時而說圓教,便是真實而具體之圓教,其不順《起信論》走超越分解之路,而順《中論》走辯證消融之路以「一念三千」作用地、詭譎地、遮顯地明圓教,亦其宜也。

\newpage\thispagestyle{empty}\addtocounter{page}{-1}\vspace*{-12mm}\begin{center}\noindent
\includegraphics[clip, trim=184pt 170pt 137pt 238pt, height=162mm]{ocr-input/image-2735.png}\end{center}

\newpage

4.然無論是天臺宗之一念三千,乃至荊溪、湛然之由之展現而為「十不二門」((十不二門》乃荊溪就智者《法華玄義》正解妙字中別釋迹中十妙之綜結處所作之《釋籤》,即荊溪《釋籤》於此以「十不二門」收攝十妙。知禮覺此「十不二門」精要,復提出特為之作《指要鈔〉),或是華嚴宗之如來藏心「不變隨緣,隨緣不變」,乃至「因圓果滿」之性起所成之大緣起法界,如「三性同異,緣起因門六義,十玄緣起無礙,六相圓融」四門之所說,要皆總是「緣起性空,流轉還滅,染淨對翻,生滅不生滅對翻」教義下之圓融地說。

就天臺宗說,一念三千之不思議境不是因著有一個「體」而要去積極地肯定的,乃是只順著煩惱心遍而實然地如此說,其當然而必然之理想地說者仍是在就此不思議境而當下寂滅之。寂滅之,即是在圓頓止觀中如實知「即空即假即中」而證實相。實相不空懸,即在三千中。實相是具體,三千始得其必然性。是故知禮云:「況復觀心自具二種:即唯識觀及實相觀。〔……】實相觀者,即於識心體其空寂,三千宛然,即空假中。唯識觀者,照於起心變造十界,即空假中。」(《指要鈔》解色心不二門。案:于此亦可見即唯識,天臺對之亦無諍)惟有在「即空假中」之實相中,三千世間始得其遍滿不壞之必然性。三千不可亦不必離,不可亦不必壞,但可即之而可寂。如此,則仍是「流轉還滅」下之體用。實亦無所謂體用,體用皆虛說。吾人不能說實相是體,三千是用。三千即空即假即中,吾人亦不能於此說空或中是體,而假是用。此仍是般若、解脫、涅槃三德體備下之出世靜態之實相觀。就即空即假即中之實相,若拆開觀,吾人不能說空假是體用,則空與假之關係仍只是虛

\newpage\thispagestyle{empty}\addtocounter{page}{-1}\vspace*{-12mm}\begin{center}\noindent
\includegraphics[clip, trim=160pt 143pt 146pt 245pt, height=162mm]{ocr-input/image-2739.png}\end{center}

\newpage\markright{附錄 \quad 佛家體用義之衡定}

\noindent 繫無礙之關係。

在虛繫無礙中證即空即假即中,此即是實相。實相是抒意字,非實體字。一色一香無非中道,非必滅色滅香也。唯是當體即如(即空即假即中),則雖色而非色,雖香而非香,而色香宛然,此即所謂滅,此是圓融地滅,非分解地滅、隔離地滅。圓融地滅,滅而不滅,去病不去法,則幻假無礙,永無窮盡。此即是煩惱心遍,故佛體遍。遍即圓滿無盡。可是並非因一種積極的創生的實體而可令其不幻假;而使之為積極的無窮盡。佛家對於幻假事總是在這不澈之虛繫狀態中而掛搭著為圓融地無盡,總是不能客觀地積極落實也。

4.1就華嚴宗說:「不變隨緣,隨緣不變」是實然地說。在此實然地說下,吾人不能說如來藏心是體,而隨緣流轉是其用。即在十信終心已去,一念即得作佛,「一念即得具足一切教義、理事、因果等,及與一切眾生皆悉同時作佛」,而成為「因該果海,果澈因源」,因圓果滿之性起,十身佛之自境界,如理智中如量境之法性家實德緣起,而緣起就緣起說,亦仍是虛繫無礙之圓融。縱使唯一真心轉,性起具德,一時炳然,或隱映互現,而吾人仍不能說此真心為一創生的實體能創生此緣起事之大用。此體用仍是「緣起性空,流轉還滅,染淨對翻,生滅不生滅對翻」下之靜態的虛繫無礙之體用。而且在此十佛自境界中,海印三味之大緣起中,實亦可說起而無起,雖「一時炳然」,而亦可說即是寂然。雖「隱映互現」,而實亦無所謂「現」,更無所謂「互」。此真心迴轉之大緣起法實仍是順應實然說的「不變隨緣,隨緣不變」之所有而翻上來圓融無礙地寂滅之,而示現為實德而順成之,雖名曰大緣起法界,

\newpage\thispagestyle{empty}\addtocounter{page}{-1}\vspace*{-12mm}\begin{center}\noindent
\includegraphics[clip, trim=188pt 180pt 139pt 227pt, height=162mm]{ocr-input/image-2743.png}\end{center}

\newpage

\noindent 說的那麼豐滿熱鬧,實則亦可以說是一無所有,亦可以說是無一德可現。然而又實可一時炳然,亦實可隱映互現。在此種虛繫無礙的圓融狀態下,實無體可說。體用皆是過渡中的詞語。亦是虛說的詞語。此如來真心實非創生緣起法之實體也。緣起總是緣起,總是對於不可思議之假名說。第一義諦中,一法不可得,焉有所謂大緣起法界耶?緣起法總是似有無性,即在十身佛自境界亦復如是。不因佛果而即可變為有自性之實事也。

4.2賢首解「總別同異成壞」六相中之「壞相」云:「第六壞相者,椽等諸緣,各住自法,本不作故。問:現見椽等諸緣,作舍成就,何故乃說本不作耶?答:祗由不作,故舍法得成。若作舍去,不住自法者,舍義即不成。何以故?作去,失本法,舍不成故。今舍成,明知不作也。問:作去有何失?答:有斷常二失。若言椽作舍去,即失椽法。失椽法故。舍即無緣,不得有故,是斷也。若失椽法而有舍者,無緣有舍,是常也。」各住自法,不作而作;緣而非緣,非緣而緣;不斷不常,就是這樣一種不可思議、虛繫無礙之奇詭緣起。能如實知,不依事識,便是如來真心。此如來真心之與奇詭緣起實非體用關係也。縱說是法性家之實德緣起亦非體用關係也。其原初先肯定一超越之真心,是順應衆生無始已來而分解地實然地如此說。其所以如此說,是為的憑依超越真心好便說明流轉還滅。以超越真心為準而起修行工夫是還滅的過程。超越真心離念離相,平等一味,所謂空如來藏。而依之而起之還滅過程,無論是漸是頓,卻總有許多事相、意義、內容之分齊。這些分齊都是在還滅過程中顯。而當還滅至心源時,則這些分齊一起捲藏於超越之真心而銷溶無餘,而歸於無相,而同時復亦即因捲藏雖無相而

\newpage\thispagestyle{empty}\addtocounter{page}{-1}\vspace*{-12mm}\begin{center}\noindent
\includegraphics[clip, trim=155pt 152pt 147pt 232pt, height=162mm]{ocr-input/image-2747.png}\end{center}

\newpage\markright{附錄 \quad 佛家體用義之衡定}

\noindent 亦示映映射出無量功德,示映映射成無邊果海。此亦即「因位窮滿者,於第三生,即得彼究竟自在圓融果矣。由此因體,依果成故,但因位滿者,即沒於果海中也。為是證境界故,不可說也。」(《一乘教義分齊章》,諸法相即自在門)故真心之果海是經過還滅工夫之因位窮滿而示映映射成者。依真心起還滅行是體用,而此體用是返流,是過渡。及其全沒於果海,則真心呈現,寂滅無相,而體用義亦不存。縱使此海印三昧之果海,於不可說中方便假說為大緣起法,說的那麼豐滿熱鬧,還只是因位內容之映射,而實無真實之緣起,而真心與此虛映之大緣起法(所謂實德緣起)之關係亦非體用之實關係。蓋此大緣起法本是虛映虛說故。實處是在還滅之行修,而沒於果海則全成為「意義」,成為寂滅之「實德」,實無事可指,無相可說,焉有體用之實體與實事?就是著實了,說為大緣起法,其與真心之關係亦仍是虛繫無礙之關係,而非創生的體用關係、因果關係也。

5.兩圓教雖殊塗而實同歸,仍不失佛家寂滅教義也。

就天臺之「空假中」言,此中根本無體用義:空不是體,假不是用。在此,雖無所謂「萬象為太虛中所見之物」,然「物與虛不相資,形自形,性自性,形性天人不相待」,仍是可以說。雖是圓融地無礙,而假究不能因一能生之實體而為真,而只能說「即空即假即中」。雖亦可說是圓融無礙之相資相待,而存有論地不相資不相待仍可說。故其相資相待亦如「因為緣起,故說空;因為空,故說緣起」,而空並非是客觀地存有論地能生起緣起事之體,而緣起事亦非是空性之用。「因此所以」之資待關係只是詮表上之抒意關係,並非客觀實有之因果關係體用關係。「即空即假即中」之圓

\newpage\thispagestyle{empty}\addtocounter{page}{-1}\vspace*{-12mm}\begin{center}\noindent
\includegraphics[clip, trim=165pt 167pt 145pt 231pt, height=162mm]{ocr-input/image-2751.png}\end{center}

\newpage

\noindent 融的資待亦只是詮表上之抒意的資待、證「實相」的資待,並非是客觀實有上因果、體用之資待。是以其圓融無礙之相資相待只是客觀的、存有論的不相資不相待之抒意詮表上之虛繫無礙地說。(理具未顯為體,事造已顯為用,此是就或迷或悟之性修關係上說體用、說因果。即行修還滅上的體用、因果,非客觀的存有論的體用因果。橫渠說不相資不相待,是就客觀的、存有論的實體用、實因果說,故只能就空假說)。

5.1就華嚴之海印三昧實德緣起說,妄盡心澄,萬象齊現,則即可說「萬象為太虛中所見之物」,亦可說「虛與物不相資,形自形,性自性,形性天人不相待。」十身佛自境界,大緣起陀羅尼法,此中實無所謂體用義,只是毘盧遮那佛法身之遍、滿、圓、常而已。而其豐富之意義、內容,皆由還滅工夫之因位上映射而成。即方便假說,展示為大緣起法,其與如來真心之關係亦非體用、因果關係,此非如來真心之所創生,乃是因位窮滿之所映射,說有就有,說無即無者。在此,如果勉強可以說體用,亦仍只是虛繫無礙之體用,而非實體創生、實理所貫之體用。此仍可說物與虛不相資不相待。圓融無礙的相資相待實仍是客觀的、存有論的不相資不相待之虛繫無礙地說,不因唯一真心迴轉,便可成為實體用、實因果之實相資實相待也。至於「不變隨緣,隨緣不變」處物與虛之不相資不相待則尤顯。(在還滅工夫上因圓果滿之體用、因果,是行修上之體用、因果,非客觀的、存有論的實體用實因果,非橫渠所意指者。)

5.2是以佛家之空假關係、理事關係、真如心與緣起法之關係,其本身皆非體用關係。如果可以以體用模式論,則皆是「緣起

\newpage\thispagestyle{empty}\addtocounter{page}{-1}\vspace*{-12mm}\begin{center}\noindent
\includegraphics[clip, trim=164pt 144pt 133pt 238pt, height=162mm]{ocr-input/image-2755.png}\end{center}

\newpage\markright{附錄 \quad 佛家體用義之衡定}

\noindent 性空,流轉還滅,染淨對翻,生滅不生滅對翻」教義綱領下虛繫無礙之體用,「物與虛不相資,形性天人不相待」之體用。此是貫通空宗之中觀、唯識宗之三性、天臺宗之空假中、華嚴宗之如來藏真如心,皆是如此而不能違背者。是以就體用之模式說,橫渠謂其「物與虛不相資,形性天人不相待」,雖是籠統,而未始不中肯。而程明道即進一步復就此體用之總論而鞭辟入裡地謂其「只有敬以直內,而無義以方外,要之其直內者亦不是。」蓋其直內只是染淨對翻,生滅不生滅對翻,其所直之內只是心真如體也。而後來陸象山復進而以義利公私判儒佛,而謂「惟義惟公故經世,惟利惟私故出世。儒者雖至於無聲無臭,無方無體,皆主於經世。釋氏雖盡未來際普度之,皆主於出世。」此蓋是「緣起性空,流轉還滅,染淨對翻,生滅不生滅對翻」教義下之必然。雖極圓融,甚至說無世可出,無生死可度,無涅槃可得,說出如許圓融、弔詭的妙論,亦仍是圓融地滅、圓融地出世,不可詭飾而辯掩也。

5.3橫渠、明道、象山之評判,表面看之,雖極籠統粗略,然實按之,皆極中肯扼要。彼等之如此說,亦只是要顯露一道德創造性的實體用之實相資實相待,亦是很顯明地要呈現出一內在道德性之性理、實理,或天理,亦根本是一道德意識之凸出、道德意識之照體挺立。此是很顯明的一個本質的差異,佛教的苦業意識總不向此用心也。一般人並無真正的道德意識,不知道德意識為何物,又見儒佛體現真理之形態相似(俱重主體性,皆可成聖,皆可成佛等),許多形容相似,又人間本亦有許多共通者,遂攪混而恍惚。橫渠等,見出此本質的差異,亦未始非善事,而亦並不因此即泯滅或減殺佛教之價值。

\newpage\thispagestyle{empty}\addtocounter{page}{-1}\vspace*{-12mm}\begin{center}\noindent
\includegraphics[clip, trim=167pt 160pt 154pt 244pt, height=162mm]{ocr-input/image-2759.png}\end{center}

\newpage

\section*{五、道德意識之豁醒,內在道德性之性理、實}\addcontentsline{toc}{section}{五、道德意識之豁醒,內在道德性之性理、實}

\begin{quotation}\kaishu 理、天理之挺立\end{quotation}

1.要想於此「虛繫無礙」的非體用的體用進一步轉出實體所生、實理所貫之實理實事之性體因果(意志因果)之實體用,於圓融無礙之相資相待而實不相資不相待轉出客觀的、存有論的實體創生,貫通為一之實相資實相待,則必須正視這真實心之「自律、自給普遍法則,以指導吾人之行為,使吾人之行為成為普遍法則所貫之實事」這一內在道德性之挺立方可能。此即是儒家之著眼點。如此著眼,則真實心不以「緣起性空、流轉還滅、無分別智等等」來規定,而是以道德的自律、內在道德性之心性來規定。就此著眼,則「緣起性空,流轉還滅,染淨對翻,生滅不生滅對翻」只成外圍浮泛之話,而鞭辟入裡,真切於真實人生,直握驪珠,以完成其理想之使命者,則在此不在彼。

1.1握此驪珠,則一般意義的緣起,無論是經驗的或是超越的,總是可以說,而「緣起性空」這特殊化的緣起便不可以說。現實事實總是因緣生起的,而「諸法不自生,亦不自他生,不共不因,是故總無生」,而又生相宛然,這生想不可解的緣起論是特殊化的緣起,「似有無性」的緣起是特殊化的緣起,「橡等諸緣各住自法,本不作故」(賢首所說的壞相)的緣起是特殊化的緣起。這. 種特殊化的緣起,吾名之曰「弔詭的緣起」(pradoxical theory ofoccasion)。其所以要如此弔詭,是為的要說空、無性、幻妄、假名。遍計執固是「情有理無」,即依他起亦是「似有無性」。僧肇不真空論云:「非無幻化人,幻化人非真人耳。」緣生就是幻化,

\newpage\thispagestyle{empty}\addtocounter{page}{-1}\vspace*{-12mm}\begin{center}\noindent
\includegraphics[clip, trim=135pt 137pt 125pt 241pt, height=162mm]{ocr-input/image-2763.png}\end{center}

\newpage\markright{附錄 \quad 佛家體用義之衡定}

\noindent 而緣義不可解,生義不可解,而又緣、生宛然。這弔詭的緣起亦是理論化的緣起(theorized occasion)。因為若只局限於緣生的諸緣上,冷泠然而觀之,吾人總可根據一種詭辯的理論而謂其不可解。此與休謨拘囚於當下孤零零之感覺而謂因果關係不可能同。休謨只直線地說因果不可能,而這弔詭的緣起,雖緣與生俱不可解,而卻又是緣、生宛然,因果宛然。此種詭辯化的弔詭緣起,亦可曰「封閉的緣起」(occasion in a closed sense)。此是定向上的緣起,亦是加了顏色的緣起。雖曰如實知,而實不必如實知。若是順緣起之為緣起而不必著在定向上,則即是「敞開的緣起」(occasion in anopen sense),此即不必是弔詭的緣起。如順經驗而觀之,緣起本有幻化的緣起,如海市蜃樓,如幻如化,世間本有如此之幻假事;亦本有虛妄的緣起,如私意,私欲的偏執,顛倒迷亂的偏執,世間本有如此之虛妄事。然而實者總是實,焉可一律以「弔詭的緣起」而幻妄之?即使是海市蜃樓之幻化,私意私欲顛倒迷亂之虛妄,其幻化是由種種物理條件而成,其虛妄私意、私欲、顛倒迷亂而成,亦不是由弔詭的緣起而成。弔詭的緣起可以幻妄一切,此即破壞世間而違經驗,造成一種顯預的封閉的緣起論。如依超越的實體而觀之,則道德的性理、實理、天理之所貫,人之所應當為而理上必須去為者,則就其為事言,雖亦是緣起的,而卻是實事而不可以幻妄論。如是,「緣起性空」之弔詭的緣起論、詭辯理論化的緣起論、顛預封閉的緣起論,即轉化而為順理的緣起論、如實的緣起論、敞開的緣起論。在此種緣起論中,凡道德實理之所貫者皆是生化之實事,亦是道德創始之實事,凡由私意、私欲、顛倒迷亂而來者皆是幻妄。〔劉蕺山解周濂溪《通書·聖學第二十》「無欲則靜虛動

\newpage\thispagestyle{empty}\addtocounter{page}{-1}\vspace*{-12mm}\begin{center}\noindent
\includegraphics[clip, trim=161pt 161pt 155pt 248pt, height=162mm]{ocr-input/image-2767.png}\end{center}

\newpage

\noindent 直」句云:「欲原是人本無的物。無欲是聖,無欲便是學。其有焉奈之何?曰:學焉而已矣。其學焉,何如?曰:本無而忽有,去其有而已矣。孰為有處?有水即為冰。孰為無處?無冰即為水。欲與天理,虛直處只是一個。從疑〔凝〕處看是欲,從化處看是理。」(《宋元學案·濂溪學案上》)。〕

1.2握此驪珠,則流轉還滅在某一意義上亦未始不可說。私意、私念、私欲、偏執乖謬而無理之流轉當該滅,然而道德實理所貫而貞定之實事,則雖作過作完,過而不留,然卻是永當作而又作,而無所謂幻妄可斷者。此即示生滅、流轉、緣起與幻妄不可等同看。有是實事之生滅、流轉緣起,有是幻妄之生滅、流轉緣起。幻妄可斷可滅,而實事不可斷不可滅。

1.3握此驪珠,則染淨對翻,生滅不生滅對翻,以明心真如,明自性清淨心,明平等一味之法性體,明遍常一之本覺,尤其未始不可說。然不只是染淨對翻,生滅不生滅對翻,明一個清淨不生滅之真如體停在那裡而只與幻假(生滅)為虛繫無礙之圓融(所謂不變隨緣,隨緣不變),而卻是驪珠朗現,自主自律,自給普遍法則,以成就實事不妄。此是順佛家之心真如說。因儒家所講之道德的本心,如心體、性體、誠體、神體、寂感真幾、無極而太極等,亦是至寂至靜的,亦是空無妄念,一切識念不相應的,亦是自性清淨的,亦是無思無為,無聲無臭的,亦是遍常一、平等一味的。凡形容如來藏自性清淨心的那些形容詞都可用得上。但只有一點不同,即,不只是如此之形容。乃是所以要有具有如此形容之本心端在明其唯如此始能毫無條件地、超越感性利害地自給道德的普遍法則以指導吾人之行為,以成就道德行為之實事,此即象山所說「儒

\newpage\thispagestyle{empty}\addtocounter{page}{-1}\vspace*{-12mm}\begin{center}\noindent
\includegraphics[clip, trim=173pt 135pt 130pt 253pt, height=162mm]{ocr-input/image-2771.png}\end{center}

\newpage\markright{附錄 \quad 佛家體用義之衡定}

\noindent 者雖至於無聲無臭無方無體,皆主於經世。」此是儒佛之本質的差異,亦即道德意識與苦業意識之不同。是對於道德本心所可有之形容可完全同於如來藏自性清淨心之形容,則順如來藏心而直握驪珠以明此內在道德性之性體心體,亦並無不可。蓋對此驪珠言,那些形容俱是外圍的話。如來藏心並非與內在道德性必不相容。只決於有無此道德意識而已。有此驪珠即是儒,無此驪珠即是佛。如果此如來藏心單由緣起性空所指之定向來決定,不准有此內在道德性之意義,則如來藏心即為特殊教義所拘限之清淨心,雖清淨無相,然卻因特殊教義之限定而有相,此即成為真心之拘限。吾人以為真心就是真心,可不受任何拘限而解除此拘限。由緣起性空可翻至清淨心,但一旦翻上來而至清淨心呈現,則可不受那特殊教義之拘限。清淨心何以必不可自主自律、自給普遍法則,以決定吾人之行為,以成就道德行為之實事耶?何以必拘限於起返流還滅之行修而停於「即空即假即中」之虛繫而無礙之境界,或停於「大緣起陀羅尼法」之虛繫無礙之境界,而不准起道德創造而成道德行為之實事耶?吾以為此種拘限乃無理由者,對超越之真心言,乃是一種桎梏、一種虐待,故必須解除此桎梏,而予以解放。解放後,則清淨而又創造,此即為大敞開,大自在,大圓融,赫日當空而大用不息。

1.4有此驪珠,便可將如來藏自性清淨心帶起來、挺立起來,豎之以義理的骨幹,使之成為一立體的直貫,以反而成就道德行為之實事,此即所以為「經世」。無此驪珠,則如來藏自性清淨心只是停在那裡而與幻假為虛繫無礙的圓融,來迴地圓轉,弔詭以呈妙,而實骨子仍是「緣起性空,流轉還滅,染淨對翻,生滅不生滅

\newpage\thispagestyle{empty}\addtocounter{page}{-1}\vspace*{-12mm}\begin{center}\noindent
\includegraphics[clip, trim=171pt 162pt 153pt 244pt, height=162mm]{ocr-input/image-2775.png}\end{center}

\newpage

\noindent 對翻」之出世,此即象山所謂「釋氏雖盡未來際普度之,皆主於出世」,吾亦進而可說,雖「即空即假即中」,盡圓融之極致,雖無世可出,無生死可度,無涅槃可得,盡不著相之極致,亦是圓融地滅度,圓融地出世,而畢竟亦是著了相,留下個軟點,畢竟未能盡人生之極致。此即陽明所謂「佛氏不著相,其實著相。吾儒著相,其實不著相。佛怕父子累,卻逃了父子;怕君臣累,卻逃了君臣;怕夫婦累,卻逃了夫婦:都是著相,便須逃避。吾儒有個父子,還他以仁;有個君臣,還他以義;有個夫婦,還他以別;何曾著父子君臣夫婦的相?」雖可云不必逃,而即空即假即中,然而父子君臣夫婦卻畢竟不可以空假論。今點出此驪珠,挺立此軟點,使其所說帶歸於大中至正之常道,此即儒釋之大通。無此驪珠,則亦終歸於偏滯而已矣。故象山謂「儒為大中,釋為大偏」。所爭只在有無此驪珠而已矣。儒佛之絜和與會通,只在此驪珠之點醒而可能。如不知此關鍵,只看那些形容之相似,以及皆重主體性、皆可成聖、皆可成佛之形態之相似,便以為是會通,那是無意義者。此亦不能以「不毀世間而證菩提」來辯飾。如曰我不要點醒此驪珠,我只要如此說,則是氣質決定,亦任之而已矣。彼亦自有其價值,世間本不必說一樣話也。

2.握此驪珠,則不但成就道德行為之實事,且藉以消除一切發自私欲、私意、氣質之偏之不道德反道德之虛妄事,此在儒者即名曰變化氣質、克己復禮,此亦是某意義之還滅。

順佛家說,一切阿賴耶識、無明識念所起之縐絕之心理學的幻妄糾結皆在此內在道德性之性體(本心天理)之呈現中步步消除或轉化,而使之澈底淨盡。這些心理學的幻妄糾結即王陽明所謂「動

\newpage\thispagestyle{empty}\addtocounter{page}{-1}\vspace*{-12mm}\begin{center}\noindent
\includegraphics[clip, trim=174pt 142pt 141pt 254pt, height=162mm]{ocr-input/image-2779.png}\end{center}

\newpage\markright{附錄 \quad 佛家體用義之衡定}

\noindent 於氣」或「動於意」,亦即程明道所喜說之「萬物皆只是一個天理,己何與焉」,「與則便是私意」之私意之「與」。陽明所說之「動於氣」與「氣之動」不同。氣之動有善有惡,而動於氣則善者亦惡。意之動有善有惡,而動於意則善惡皆壞。是以私意之與,動於氣,動於意,皆是不順理之私,皆是心理之糾結。楊簡之「不起意」即是不起「意必固我」之意,是以此「意」亦是私。一起意,縱使有好有壞,亦不是本心天理之至善至公至常與至一。此種意亦即劉蕺山所謂「兩在而異情」之「念」,不是本心所存(所存之以為真主)之「一機二用」之真意。化念還心,真意斯現,於此給有真道德之可言。總之,不論是氣之動、意之動,楊氏之意、劉氏之念,以及動於氣、動於意,乃至私意之與,皆不是天理之本心,在此皆不能建立起真正之道德行為,皆可統攝於心理學之糾結中,亦皆可統攝於佛家阿賴耶識,無明識念所起之虛妄染汙之生滅流轉中,凡此皆須憑藉內在道德性之本心以及本心所自給之普遍法則(天理)以消除之或轉化之。這是在道德意識下憑藉內在道德性之定、常、遍以消除而淨化之,不是在苦業意識下生滅不生滅對翻,憑藉真如空性,無分別智,以寂滅之。這不是後返的滅度或當體即如的滅度之淨化,而是前向的道德創造之淨化。

2.1消除之,使之至於淨盡,則無動於氣、動於意,以及參與之私;轉化之,則是使其無私,而氣之動純為順理之氣,意之動、念之起,純為順理之意與念。如是,則雖氣也,而有本心天理以貫之,雖意與念也,而有本心天理以常而貞定之。有本心天理以貫之,則氣之動即為天理之流行。有本心天理以常而貞定之,則意而無意、念而無念,皆是本心天理之呈現。而道德的本心天理不能空

\newpage\thispagestyle{empty}\addtocounter{page}{-1}\vspace*{-12mm}\begin{center}\noindent
\includegraphics[clip, trim=161pt 164pt 149pt 233pt, height=162mm]{ocr-input/image-2783.png}\end{center}

\newpage

\noindent 掛,停在抽象的狀態中,亦必須在氣動中而為分殊的表現,亦必須在意而無意、念而無念中作具體的呈現。作分殊的表現,具體的呈現,始有真實的道德行為可言。否則,本心天理只是抽象的「體」,而沒有成為道德行為之「用」。

3.氣之動是有分殊的、有分際之不同的,此即是差別相(殊異相)。順本心天理而起之意而無意念而無念亦是有分殊的、有分際的,如陽明所謂意在於事親、意在於讀書等。順本心天理而起之意而無意念而無念實即是本心天理之在具體的分際上之具體的流註。是以如說是意,必須是意而無意,如說是念,必須是念而無念。是以此時即根本亦可以不說意,不說念,而只是本心天理之在具體分際上之具體流註。如流註於父母即為孝,流註於子女即為慈,流註於兄弟即為友愛,流註於夫婦即為相敬如賓之情愛,流註於國家政治即為忠義法律,流註於一切人文社會之活動即為以義和利以禮制事之事功:此皆是在具體分際上之具體的流注、具體的呈現。氣之動有分殊,有分際之差別相,本心天理即就此具體的分際而為具體的流註與呈現,因而亦有差別的表現。

3.1這些差別相,儒者名之曰分、分位、分際。這些不是虛妄,不是幻假,不是依識而現的。這些「分際相」是不可離的。親親、仁民、愛物、慈、孝、弟、忠、義等等差別相分際相是不可「離」的。這是分之不同,分際分位之不同。有不同之分位,故本心天理亦有差別之表現,在不同之分位上,有不同之表現。有不同之表現而成為不同之道德行為,始有真實之道德行為可言。這是普遍之在具體中表現,具體中之普遍,亦曰具體的普遍,非抽象的普遍,雖普遍而有具體之內容,不是抽離的光板。而同時具體之差

\newpage\thispagestyle{empty}\addtocounter{page}{-1}\vspace*{-12mm}\begin{center}\noindent
\includegraphics[clip, trim=171pt 137pt 136pt 247pt, height=162mm]{ocr-input/image-2787.png}\end{center}

\newpage\markright{附錄 \quad 佛家體用義之衡定}

\noindent 別、分際、分位亦因普遍之本心天理貫註於其中而有普遍之意義,永恆之意義,必然之意義:雖殊也而普遍,雖變也而永恆,雖實然也,而亦是必然:此謂普遍的特殊,有永恆意義的變化,必然的然(定然而不可移,當然而不容已),也就是真實、具體而必然的殊異與變,不只是抽離了本心天理之普遍性之無體的殊,無體的變。無體的殊與無體的變是無理由的、偶然的,非具體而真實的,此或可即是無明識念之所緣起,此而說虛妄幻假則可。然而有本心天理以貫注之之分位之殊與變則不是無明之所緣起,不可以虛妄幻假論。此即是儒者所謂「實事」。事實,則分位之殊即不可以假論,亦是實。這些實有之分位差別是須要肯定的,是因本心天理之貫註而有自體的,本心天理之貫注使之有自性,使之為實。否則,便沒有真實的道德行為之可言,本心天理亦無成就真實道德行為之具體的表現。凡道德行為都是具體的、獨特的、存在的、當下的,責無旁貸,雖父子兄弟亦不能相為的,(此即今日海德格存在倫理之所說),而其本則是本心天理之普遍性(遍、常、一)。是以道德的本心天理之成就真實的道德行為即必然地肯定了分殊之真實性,亦必然地肯定了分殊上真實道德行為之為實事。這不是「緣起性空、流轉還滅,染淨對翻,生滅不生滅對翻」教義下之所顯之如來藏心所能成就。這即是朱子所謂「理一分殊」之真實意義,亦即明道所喜稱引之以辨儒佛的(坤文言〉「敬以直內,義以方外」兩語之真實意義。

4.在「緣起性空,流轉還滅,染淨對翻,生滅不生滅對翻」之不相資之不相待中挺立起一個立體的骨幹,一個內在道德性之性理的敬義骨幹,(「敬義立而德不孤;直方大,不習無不利,則不疑

\newpage\thispagestyle{empty}\addtocounter{page}{-1}\vspace*{-12mm}\begin{center}\noindent
\includegraphics[clip, trim=163pt 161pt 148pt 239pt, height=162mm]{ocr-input/image-2791.png}\end{center}

\newpage

\noindent 其所行矣」),則即顯出實事實理相資相待之有機的體用,道德性的創生實體(性體、心體、神體、誠體、寂感真幾)之創生的體用。此即函說需要吾人對於緣起事要有個簡別。縱使說緣起,說依他起,亦要有簡別:有是虛妄幻假的緣起流轉,有是實事的緣起流轉。發自私意、私欲、私念,以及氣質偏雜之乖謬無理的緣起是虛妄,是幻假,而發自本心天理之實事則不是虛妄,不是幻假。凡是一道德行為,就其為一行為(action)言,是事,就此事之「實際的完成」(material accomplishment)言,需要有緣(各種條件)來助成,此自亦可說是緣起。例如事親之孝行,如純依本心天理精誠無雜地來作此事,(在孝行上說「無雜」是多餘的),則是此事(行為)之「形式的完成」 (天理的完成,formalaccomplishment,performed by categorical imperative);但事親這一具體的孝行需要在奉養之宜溫清之節以及孝子之聲音笑貌動作趨翔中完成,此即是其實際的完成(材質的完成),此即可曰緣生緣成。若無這些緣助,事親之孝行便不能表現。其形式的完成依天理,天理不可以緣起事說。孝行之事是緣起事,而本心天理不是緣起事。在其實際的完成中,種種緣助亦是緣起事。既是一緣起事,當然作過即完,有生有滅。此即明道所說「雖堯舜事業亦如太空中一點浮雲過目」。明道之說此語是在顯天理德性之尊嚴。但天理德性不能不在事中作具體而有分際的表現,此即事業之不容已。是以雖有生有滅,作過即完,過而不留,然依本心天理之不容已,即須作而又作,永恆地作,一日二十四小時,一生百年之中,以及繼起的無窮生命,總應是在不斷地作,此即「造次必於是,顛沛必於是」,「大孝終身慕父母」,「文王之德之純,純亦不已」諸語之

\newpage\thispagestyle{empty}\addtocounter{page}{-1}\vspace*{-12mm}\begin{center}\noindent
\includegraphics[clip, trim=176pt 141pt 130pt 244pt, height=162mm]{ocr-input/image-2795.png}\end{center}

\newpage\markright{附錄 \quad 佛家體用義之衡定}

\noindent 所示。總起來說,即是道德創造之不容已,亦是恆久而不已。擴大地說,天地之道亦只是創造之不容已,恆久而不已。此即「維天之命於穆不已」一詩語之所示。是以道德行為,雖有生有滅,緣起流轉,而卻不是虛妄幻假,根於無明,無明識念之所緣起,此乃是天理之實事,天命之不已之所命者。此乃是在本心天理之貫注中(在範圍曲成中)的緣起事,亦是本心天理之道德的創造之不容已之必然地要有這些事。

4.1在佛家,依其心理情意糾結(在此說煩惱礙礙與智礙)之籠罩觀點,普視一切緣生。緣生、無性、生滅、流轉、虛妄、幻化、無明、識念、差別、染汙,俱是相等的同意語,此是緣起之一律看,即,無明一層觀,此亦可謂顛預世間也。此非儒者所能許可。此一面顛預而混漫,則翻過來之還滅即是清淨無相、無分別智、不生不滅、空如平等;而一切「自在用」(不思議業用)只是衆生依其自己之識念之所見,而法身自身實無相可說,無業可用;最後而至圓教,色心不二,亦只是當體即如,而為「即空即假即中」虛繫無礙之圓融,而緣生無性,假仍是假,是即當體即滅度也。此是苦業意識染淨對翻所必至之結論,而亦只能如此說,無法再可有增減。此所以從判教方面說,無論天臺之藏通別圓,抑或華嚴之小始終頓圓,終於表示佛教為完整之一套,而天衣無縫也。然而此完整之一套卻就是不能成就道德行為之實事。此無「義以方外」之過也。(此不能以「衆善奉行,諸惡莫作」來辯飾)當然,人之心理情意之糾結網,上天下地,可以沾染一切,從此悟入人生之懊惱與苦難以及種種離奇古怪之乖謬與荒誕,而直接還滅之以達一清涼自在之境,自亦是生命之一途徑。然心理情意之糾結網雖可

\newpage\thispagestyle{empty}\addtocounter{page}{-1}\vspace*{-12mm}\begin{center}\noindent
\includegraphics[clip, trim=161pt 164pt 152pt 234pt, height=162mm]{ocr-input/image-2799.png}\end{center}

\newpage

\noindent 沾染一切,而不必能窮盡一切。邏輯數學之明智開關邏輯世界,道德意識之尊嚴開闢道德世界。此皆足以拆穿心理情意之糾結網而啟迪理性者也。前者雖只是形式的,不涉及存在之世界,自亦不接觸於生命,然其開關邏輯理性足以構成科學之知識,使科學知識有客觀獨立之意義,而有非心理情意糾結網所能盡者,則無疑。此層復亦可以使吾人突破心理主義之情意糾結網而暫時得一形式的定常領域(亦函一所謂「本質之領域」)之貞定,西方古希臘柏拉圖即由此路而得一澄清靈魂接近真實之生命途徑。此雖不必真切而究竟,要亦不失為突破情意糾結網而呈現客觀定常之一道。至於道德意識之開闢道德世界,則涉及存在之真實,且亦真切於生命,在邏輯理性以上而呈現出道德理性(所謂實踐理性),能與邏輯理性以究竟之歸宿,復反而能開出邏輯理性以與之為有機的統一,此是道德與知識之開合關係,而由道德理性之充盡所必然蘊函者。至於就其涉及存在之真實,真切於生命言,則是其自身之本分事,負面地消除淨化一切心理情意之糾結網,復正面地成就道德行為之實事,「取日虞淵,洗光咸池」(象山語),而不落於心理主義之「緣起性空,染淨對翻」之寂滅,此則真「世出世間」之真實統一,大成圓滿之教。只要人一且道德意識豁醒,直下立足於此而不歧出,則立見其為必然而不可移者。(苦業意識及罪惡意識皆統攝於道德意識中而得其真實義。)

\newpage\thispagestyle{empty}\addtocounter{page}{-1}\vspace*{-12mm}\begin{center}\noindent
\includegraphics[clip, trim=180pt 307pt 133pt 213pt, height=162mm]{ocr-input/image-2803.png}\end{center}

\end{document}

